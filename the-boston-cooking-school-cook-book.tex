\documentclass[10pt]{book}
\usepackage[utf8]{inputenc}
\usepackage[T1]{fontenc}
\usepackage{tgpagella}
% Page setup for folio printing on letter paper (2-up)
\usepackage[paperwidth=5.5in,paperheight=8.5in,
            top=0.75in,bottom=0.75in,
            inner=0.65in,outer=0.7in]{geometry}
\usepackage{microtype}
\usepackage{titlesec}
\usepackage{titletoc}
\usepackage{enumitem}
\usepackage{array}
\usepackage{tabularx}
\usepackage[table]{xcolor}
\usepackage{multicol}
\usepackage{needspace}
\usepackage{graphicx}
\definecolor{tablerowgray}{gray}{0.75}
\setlength{\extrarowheight}{3pt}

% Custom list formatting - reduce left padding
\setlist[itemize]{leftmargin=1em}

% Prevent vertical justification (stretching)
\raggedbottom

% Custom chapter formatting
\titleformat{\chapter}[display]
  {\normalfont\huge\bfseries\centering}
  {}
  {0pt}
  {\vspace*{-60pt}}
  [{\vspace{2ex}\centering\includegraphics[width=4in]{divider-chapter.png}\vspace{-2ex}}]
\titlespacing*{\chapter}
  {0pt}
  {20pt}
  {40pt}

% Custom section formatting
\setcounter{secnumdepth}{0}
\titleformat{\section}
  {\normalfont\Large\bfseries}
  {}
  {0pt}
  {}
  [\vspace{2pt}\titlerule]

% Custom subsection formatting
\titleformat{\subsection}
  {\normalfont\large\bfseries}
  {}
  {0pt}
  {\underline}
\titlespacing*{\subsection}
  {0pt}
  {1.5em}
  {0.3em}

% Customize table of contents - remove chapter numbers, add leader dots
\titlecontents{chapter}
  [0pt]
  {\bfseries}
  {}
  {}
  {\titlerule*[0.5pc]{.}\contentspage}

% Customize chapter headers to show only chapter name (not "Chapter X.")
\renewcommand{\chaptermark}[1]{\markboth{#1}{}}

% Remove paragraph indentation and add space between paragraphs
\setlength{\parindent}{0pt}
\setlength{\parskip}{0.8em}

% Custom title page
\renewcommand{\maketitle}{%
  \begin{titlepage}
    \centering
    \vspace*{\fill}
    {\Huge\bfseries The Boston Cooking-School Cook Book\par}
    \vspace{2em}
    \includegraphics[width=4in]{divider-chapter.png}\par
    \vspace{2em}
    {\Large Fannie Merritt Farmer\par}
    \vspace{1em}
    {\large 1910\par}
    \vspace*{\fill}
  \end{titlepage}
}

\begin{document}

\maketitle


\chapter*{Preface From The Author}


  “But for life the universe were nothing; and all that has life
  requires nourishment.”


With the progress of knowledge the needs of the human body have not been
forgotten. During the last decade much time has been given by scientists
to the study of foods and their dietetic value, and it is a subject
which rightfully should demand much consideration from all. I certainly
feel that the time is not far distant when a knowledge of the principles
of diet will be an essential part of one's education. Then mankind will
eat to live, will be able to do better mental and physical work, and
disease will be less frequent.

At the earnest solicitation of educators, pupils, and friends, I have
been urged to prepare this book, and I trust it may be a help to many
who need its aid. It is my wish that it may not only be looked upon as a
compilation of tried and tested recipes, but that it may awaken an
interest through its condensed scientific knowledge which will lead to
deeper thought and broader study of what to eat.

                                                                F. M. F.


\chapter*{Preface From Tristan}


You had said you wanted a recipe book for Christmas. I thought about looking up recipes online and putting something together, but opted to find something that already existed. I found this book in Project Gutenburg, which is a website dedicated to preserving digital copies of books that have entered the public domain. So since this is in the public domain, I can "legally" print it at Staples.

This book is quite old now, but people online still seem to think that it's a good book with solid recipes and methods. But after scanning through it multiple times there are definitely some parts that are a little \textit{dated}. For example, I changed the chapter title for "Helpful Hints for the Young Housekeeper" to just "Helpful Hints", or changed the spelling of protein from "proteib" with a B to "protein". I'm sure there are many other things like this that I did not change.

I fear there may be a tad bit of sexism in this book, but if you don't look too closely I hope you can still enjoy a recipe or two, and enjoy this as an interesting piece of history.

Love, Tristan.



\tableofcontents


\chapter{Food}



Food is anything which nourishes the body. From fifteen to twenty
elements enter into the composition of the body, of which the following
thirteen are considered: oxygen, 62 1/2\%; carbon, 21 1/2\%; hydrogen, 10\%;
nitrogen, 3\%; calcium, phosphorus, potassium, sulphur, chlorine, sodium,
magnesium, iron, and fluorine the remaining 3\%.

Food is necessary for growth, repair, and energy; therefore the elements
composing the body must be found in the food. The thirteen elements
named are formed into chemical compounds by the vegetable and animal
kingdoms to support the highest order of being, man. All food must
undergo chemical change after being taken into the body, before it can
be utilized by the body; this is the office of the digestive system.

Food is classified as follows:


\begin{tabular}{clp{3.5in}}
I. & ORGANIC & 1. Protein (nitrogenous or albuminous) \\
 & & 2. Carbohydrates (sugar and starch) \\
 & & 3. Fats and oils \\[0.5em]
II. & INORGANIC & 1. Mineral matter \\
 & & 2. Water \\
\end{tabular}

The chief office of proteins is to build and repair tissues. They
furnish energy, but at greater cost than carbohydrates, fats, and oils.
They contain nitrogen, carbon, oxygen, hydrogen, and sulphur or
phosphorus, and include all forms of animal foods (excepting fats and
glycogen) and some vegetable foods. Examples: milk, cheese, eggs, meat,
fish, cereals, peas, beans, and lentils. The principal constituent of
protein food is albumen. Albumen as found in food takes different names,
but has the same chemical composition; as, \textit{albumen} in eggs, \textit{fibrin}
in meat, \textit{casein} in milk and cheese, \textit{vegetable casein} or \textit{legumen} in
peas, beans, and lentils; and \textit{gluten} in wheat. To this same class
belongs gelatin.

The chief office of the carbohydrates is to furnish energy and maintain
heat. They contain carbon, hydrogen, and oxygen, and include foods
containing starch and sugar. Examples: vegetables, fruits, cereals,
sugars, and gums.

The chief office of fats and oils is to store energy and heat to be used
as needed, and constitute the adipose tissues of the body. Examples:
butter, cream, fat of meat, fish, cereals, nuts, and the berry of the
olive-tree.

The chief office of mineral matter is to furnish the necessary salts
which are found in all animal and vegetable foods. Examples: sodium
chloride (common salt); carbonates, sulphates and phosphates of sodium,
potassium, and magnesium; besides calcium phosphates and iron.

Water constitutes about two-thirds the weight of the body, and is in all
tissues and fluids; therefore its abundant use is necessary. One of the
greatest errors in diet is neglect to take enough water; while it is
found in all animal and vegetable food, the amount is insufficient.



\needspace{15\baselineskip}
\section*{Correct Proportions Of Food}

Age, sex, occupation, climate, and season must determine the diet of a
person in normal condition.

Liquid food (milk or milk in preparation with the various prepared foods
on the market) should constitute the diet of a child for the first
eighteen months. After the teeth appear, by which time ferments have
been developed for the digestion of starchy foods, entire wheat bread,
baked potatoes, cereals, meat broths, and occasionally boiled eggs may
be given. If mothers would use Dr. Johnson's Educators in place of the
various sweet crackers, children would be as well pleased and better
nourished; with a glass of milk they form a supper suited to the needs
of little ones, and experience has shown that children seldom tire of
them. The diet should be gradually increased by the addition of cooked
fruits, vegetables, and simple desserts; the third or fourth year fish
and meat may be introduced, if given sparingly. Always avoid salted
meats, coarse vegetables (beets, carrots, and turnips), cheese, fried
food, pastry, rich desserts, confections, condiments, tea, coffee, and
iced water. For school children the diet should be varied and abundant,
constantly bearing in mind that this is a period of great mental and
physical growth. Where children have broken down, supposedly from
over-work, the cause has often been traced to impoverished diet. It must
not be forgotten that digestive processes go on so rapidly that the
stomach is soon emptied. Thanks to the institutor of the school
luncheon-counter!

The daily average ration of an adult requires:


\begin{itemize}
\setlength{\itemsep}{0pt}
\item 4 1/2 oz. protein
\item 2 oz. fat
\item 18 oz. starch
\item 5 pints water
\end{itemize}


About one-third of the water is taken in our food, the remainder as a
beverage. To keep in health and do the best mental and physical work,
authorities agree that a mixed diet is suited for temperate climates,
although sound arguments appear from the vegetarian. Women, even though
they do the same amount of work as men, as a rule require less food.
Brain workers should take their protein in a form easily digested. In
consideration of this fact, fish and eggs form desirable substitutes for
meat. The working man needs quantity as well as quality, that the
stomach may have something to act upon. Corned beef, cabbage,
brown-bread, and pastry, will not overtax his digestion. In old age the
digestive organs lessen in activity, and the diet should be almost as
simple as that of a child, increasing the amount of carbohydrates and
decreasing the amount of proteins and fat. Many diseases which occur
after middle life are due to eating and drinking such foods as were
indulged in during vigorous manhood.



\needspace{15\baselineskip}
\section*{Water (H2o)}

Water is a transparent, odorless, tasteless liquid. It is derived from
five sources,--rains, rivers, surface-water or shallow wells, deep wells,
and springs. Water is never found pure in nature; it is nearly pure when
gathered in an open field, after a heavy rainfall, or from springs. For
town and city supply, surface-water is furnished by some adjacent pond
or lake. Samples of such water are carefully and frequently analyzed, to
make sure that it is not polluted with disease germs.

The hardness of water depends upon the amount of salts of lime and
magnesia which it contains. Soft water is free from objectionable salts,
and is preferable for household purposes. Hard water may be softened by
boiling, or by the addition of a small amount of bicarbonate of soda
(NaHCO\textsubscript{3}).

Water freezes at a temperature of 32deg F., boils at 212deg F.; when bubbles
appear on the surface and burst, the boiling-point is reached. In high
altitudes water boils at a lower temperature. From 32deg to 65deg F. water
is termed cold; from 65deg to 92deg F., tepid; 92deg to 100deg F., warm; over
that temperature, hot. Boiled water is freed from all organic
impurities, and salts of lime are precipitated: it does not ferment, and
is a valuable antiseptic. Hot water is more stimulating than cold, and
is of use taken on an empty stomach, while at a temperature of from 60deg
to 95deg F. it is used as an emetic; 90deg F. being the most favorable
temperature.

Distilled water is chemically pure and is always used for medicinal
purposes. It is flat and insipid to the taste, having been deprived of
its atmospheric gases.

There are many charged, carbonated, and mineral spring waters bottled
and put on the market; many of these are used as agreeable table
beverages. Examples: Soda Water, Apollinaris, Poland, Seltzer, and
Vichy. Some contain minerals of medicinal value. Examples: Lithia,
saline, and sulphur waters.



\needspace{15\baselineskip}
\section*{Salts}

Of all salts found in the body, the most abundant and valuable is sodium
chloride (NaCl), common salt; it exists in all tissues, secretions, and
fluids of the body, with the exception of enamel of the teeth. The
amount found in food is not always sufficient; therefore salt is used as
a condiment. It assists digestion, inasmuch as it furnishes chlorine for
hydrochloric acid found in gastric juice.

Common salt is obtained from evaporation of spring and sea-water, also
from mines. Our supply of salt obtained by evaporation comes chiefly
from Michigan and New York; mined salt from Louisiana and Kansas.

Salt is a great preservative; advantage is taken of this in salting meat
and fish.

Other salts--lime, phosphorus, magnesia, potash, sulphur, and iron--are
obtained in sufficient quantity from food we eat and water we drink. In
young children, perfect formation of bones and teeth depends upon
phosphorus and lime taken into the system; these are found in meat and
fish, but abound in cereals.



\needspace{15\baselineskip}
\section*{Starch (C6h10o5)}

Starch is a white, glistening powder; it is largely distributed
throughout the vegetable kingdom, being found most abundantly in cereals
and potatoes. Being a force-producer and heat-giver it forms one of the
most important foods. Alone it cannot sustain life, but must be taken in
combination with foods which build and repair tissues.


\textbf{Test for Starch.} A weak solution of iodine added to cold cooked starch
gives an intense blue color.

Starch is insoluble in cold water, and soluble to but a small extent in
boiling water. Cold water separates starch-grains, boiling water causes
them to swell and burst, thus forming a paste.

Starch subjected to dry heat is changed to \textit{dextrine}
(C\textsubscript{6}H\textsubscript{10}O\textsubscript{5}), British gum. Dextrine subjected to heat plus an acid
or a ferment is changed to \textit{dextrose} (C\textsubscript{6}H\textsubscript{12}O\textsubscript{6}). Dextrose
occurs in ripe fruit, honey, sweet wine, and as a manufactured product.
When grain is allowed to germinate for malting purposes, starch is
changed to dextrine and dextrose. In fermentation, dextrose is changed
to alcohol (C\textsubscript{2}H\textsubscript{5}HO) and carbon dioxide (CO\textsubscript{2}). Examples: bread
making, vinegar, and distilled liquors.

Glycogen, animal starch, is found in many animal tissues and in some
fungi. Examples: in liver of meat and oysters.

Raw starch is not digestible; consequently all foods containing starch
should be subjected to boiling water or dry heat, and thoroughly cooked.
Starch is manufactured from wheat, corn, and potatoes. \textbf{Corn-starch} is
manufactured from Indian corn. \textbf{Arrowroot}, the purest form of starch,
is obtained from two or three species of the Maranta plant, which grows
in the West Indies and other tropical countries. Bermuda arrowroot is
most highly esteemed. \textbf{Tapioca} is starch obtained from tuberous roots
of the bitter cassava, native of South America. \textbf{Sago} is starch
obtained from sago palms, native of India.



\needspace{15\baselineskip}
\section*{Sugar (C12h22o11)}

Sugar is a crystalline substance, differing from starch by its sweet
taste and solubility in cold water. As food, its uses are the same as
starch; all starch must be converted into sugar before it can be
assimilated.

The principal kinds of sugar are: cane sugar or \textit{sucrose}, grape sugar
or \textit{glucose} (C\textsubscript{6}H\textsubscript{12}O\textsubscript{6}), milk sugar or \textit{lactose}
(C\textsubscript{12}H\textsubscript{22}O\textsubscript{11}), and fruit sugar or \textit{levulose} (C\textsubscript{6}H\textsubscript{12}O\textsubscript{6}).

\textbf{Cane sugar} is obtained from sugar cane, beets, and the palm and
sugar-maple trees. Sugar cane is a grass supposed to be native to
Southern Asia, but now grown throughout the tropics, a large amount
coming from Cuba and Louisiana; it is the commonest of all, and in all
cases the manufacture is essentially the same. The products of
manufacture are: molasses, syrup, brown sugar, loaf, cut, granulated,
powdered, and confectioners' sugar. Brown sugar is cheapest, but is not
so pure or sweet as white grades; powdered and confectioners' sugars are
fine grades, pulverized, and, although seeming less sweet to the taste,
are equally pure. Confectioners' sugar when applied to the tongue will
dissolve at once; powdered sugar is a little granular.

Cane sugar when added to fruits, and allowed to cook for some time,
changes to grape sugar, losing one-third of its sweetness; therefore the
reason for adding it when fruit is nearly cooked. Cane sugar is of great
preservative value, hence its use in preserving fruits and milk; also,
for the preparation of syrups.

Three changes take place in the cooking of sugar: first, barley sugar;
second, caramel; third, carbon.

\textbf{Grape sugar} is found in honey and all sweet fruits. It appears on the
outside of dried fruits, such as raisins, dates, etc., and is only
two-thirds as sweet as cane sugar. As a manufactured product it is
obtained from the starch of corn.

\textbf{Milk sugar} is obtained from the milk of mammalia, but unlike cane
sugar does not ferment.

\textbf{Fruit sugar} is obtained from sweet fruits, and is sold as \textit{diabetin},
is sweeter than cane sugar, and is principally used by diabetic
patients.



\needspace{15\baselineskip}
\section*{Gum, Pectose, And Cellulose}

These compounds found in food are closely allied to the carbohydrates,
but are neither starchy, saccharine, nor oily. Gum exists in the juices
of almost all plants, coming from the stems, branches, and fruits.
Examples: gum arabic, gum tragacanth, and mucilage. Pectose exists in
the fleshy pulp of unripe fruit; during the process of ripening it
changes to pectin; by cooking, pectin is changed to pectosic acid, and
by longer cooking to pectic acid. Pectosic acid is jelly-like when cold;
pectic acid is jelly-like when hot or cold. Cellulose constitutes the
cell-walls of vegetable life; in very young vegetables it is possible
that it can be acted upon by the digestive ferments; in older vegetables
it becomes woody and completely indigestible.



\needspace{15\baselineskip}
\section*{Fats And Oils}

Fats and oils are found in both the animal and vegetable kingdom. Fats
are solid; oils are liquid; they may be converted into a liquid state by
application of heat; they contain three substances,--\textit{stearin} (solid),
\textit{olein} (liquid), \textit{palmitin} (semi-solid). Suet is an example where
stearin is found in excess; lard, where olein is in excess; and butter,
where palmitin is in excess. Margarin is a mixture of stearin and
palmitin. The fatty acids are formed of stearin, olein, and palmitin,
with glycerine as the base. Examples: stearic, palmitic, and oleic acid.
Butyric acid is acid found in butter. These are not sour to the taste,
but are called acids on account of their chemical composition.

Among animal fats cream and butter are of first importance as foods, on
account of their easy assimilation. Other examples are: the fat of
meats, bone-marrow, suet (the best found around the loin and kidneys of
the beef creature), lard, cottolene, coto suet, cocoanut butter,
butterine, and oleomargarine. The principal animal oils are cod liver
oil and oil found in the yolk of egg; principal vegetable oils are
olive, cottonseed, poppy, and cocoanut oils, and oils obtained from
various nuts.

Oils are divided into two classes, \textit{essential} and \textit{fixed}. Essential
oils are volatile and soluble in alcohol. Examples: clove, rose, nutmeg,
and violet. Fixed oils are non-volatile and soluble in ether, oil, or
turpentine. Examples: oil of nuts, corn meal, and mustard.

Fats may be heated to a high temperature, as considered in cookery they
have no boiling-point. When appearing to boil, it is evident water has
been added, and the temperature lowered to that of boiling water, 212deg
F.



\needspace{15\baselineskip}
\section*{Milk}


\begin{tabular}{|l|r|}
\hline
\multicolumn{2}{|l|}{\textbf{Composition}} \\
\hline
Protein & 3.4\% \\
\hline
Fat & 4\% \\
\hline
Mineral matter & .7\% \\
\hline
Water & 87\% \\
\hline
Lactose & 4.9\% \\
\hline
\end{tabular}

\vspace{10pt}

\noindent
The value of milk as a food is obvious from the fact that it constitutes
the natural food of all young mammalia during the period of their most
rapid growth. There is some danger, however, of overestimating its value
in the dietary of adults, as solid food is essential, and liquid taken
should act as a stimulant and a solvent rather than as a nutrient. One
obtains the greatest benefit from milk when taken alone at regular
intervals between meals, or before retiring, and sipped, rather than
drunk. Hot milk is often given to produce sleep.

When milk is allowed to stand for a few hours, the globules of fat,
which have been held in suspension throughout the liquid, rise to the
top in the form of \textit{cream}; this is due to their lower specific gravity.

The difference in quality of milk depends chiefly on the quantity of fat
therein: casein, lactose, and mineral matter being nearly constant,
water varying but little unless milk is adulterated.

\textbf{Why Milk Sours.} A germ found floating in the air attacks a portion of
the lactose in the milk, converting it into lactic acid; this, in turn,
acts upon the casein (protein) and precipitates it, producing what is
known as \textit{curd} and \textit{whey}. Whey contains water, salts, and some sugar.

Milk is preserved by sterilization, pasteurization, and evaporation.
\textit{Fresh condensed milk}, a form of evaporized milk, is sold in bulk, and
is preferred by many to serve with coffee. Various brands of condensed
milk and cream are on the market in tin cans, hermetically sealed.
Examples: Nestle's Swiss Condensed Milk, Eagle Condensed Milk, Daisy
Condensed Milk, Highland Evaporated Cream, Borden's Peerless Evaporated
Cream. \textit{Malted milk}--evaporized milk in combination with extracts of
malted barley and wheat--is used to a considerable extent; it is sold in
the form of powder.

Thin, or strawberry, and thick cream may be obtained from almost all
creameries. Devonshire, or clotted cream, is cream which has been
removed from milk allowed to heat slowly to a temperature of about 150deg
F.

In feeding infants with milk, sterilization or pasteurization is
sometimes recommended to avoid danger of infectious germs. By this
process milk can be kept for many days, and transported if necessary. To
prevent acidity of the stomach, add from one to two teaspoonfuls of lime
water to each half-pint of milk. Lime water may be bought at any
druggist's, or easily prepared at home.


\textbf{Lime Water.} Pour two quarts boiling water over an inch cube unslacked
lime; stir thoroughly and stand over night; in the morning pour off the
liquid that is clear, and bottle for use. Keep in a cool place.



\needspace{15\baselineskip}
\section*{Butter}


\begin{tabular}{|l|r|}
\hline
\multicolumn{2}{|l|}{\textbf{Composition}} \\
\hline
Fat & 93\% \\
\hline
Water & 5.34\% \\
\hline
Mineral matter & .95\% \\
\hline
Casein & .71\% \\
\hline
\end{tabular}

\vspace{10pt}

\noindent
Butter of commerce is made from cream of cow's milk. The quality depends
upon the breed of cow, manner of, and care in, feeding. Milk from Jersey
and Guernsey cows yields the largest amount of butter.

Butter should be kept in a cool place and well covered, otherwise it is
liable to become rancid; this is due to the albuminous constituents of
the milk, acting as a ferment, setting free the fatty acids.
First-quality butter should be used; this does not include pat butter or
fancy grades. Poor butter has not been as thoroughly worked during
manufacture, consequently more casein remains; therefore it is more apt
to become rancid. Fresh butter spoils quickly; salt acts as a
preservative. Butter which has become rancid by too long keeping may be
greatly improved by melting, heating, and quickly chilling with
ice-water. The butter will rise to the top, and may be easily removed.

Where butter cannot be afforded, there are several products on the
market which have the same chemical composition as butter, and are
equally wholesome. Examples: butterine and oleomargarine.

Buttermilk is liquid remaining after butter “has come.” When taken
fresh, it makes a wholesome beverage.



\needspace{15\baselineskip}
\section*{Cheese}


\begin{tabular}{|l|r|}
\hline
\multicolumn{2}{|l|}{\textbf{Composition}} \\
\hline
Protein & 31.23\% \\
\hline
Fat & 34.39\% \\
\hline
Water & 30.17\% \\
\hline
Mineral matter & 4.31\% \\
\hline
\end{tabular}

\vspace{10pt}

\noindent
Cheese is the solid part of sweet milk obtained by heating milk and
coagulating it by means of rennet or an acid. Rennet is an infusion made
from prepared inner membrane of the fourth stomach of the calf. The curd
is salted and subjected to pressure. Cheese is made from skim milk, milk
plus cream, or cream. Cheese is kept for a longer or shorter time,
according to the kind, that fermentation or decomposition may take
place. This is called ripening. Some cream cheeses are not allowed to
ripen. Milk from Jersey and Guernsey cows yields the largest amount of
cheese.

Cheese is very valuable food; being rich in protein, it may be used as a
substitute for meat. A pound of cheese is equal in protein to two pounds
of beef. Cheese in the raw state is difficult of digestion. This is
somewhat overcome by cooking and adding a small amount of bicarbonate of
soda. A small piece of rich cheese is often eaten to assist digestion.

The various brands of cheese take their names from the places where
made. Many foreign ones are now well imitated in this country. The
favorite kinds of skim-milk cheese are: Edam, Gruyère, and Parmesan.
Parmesan is very hard and used principally for grating. The holes in
Gruyère are due to aeration.

The favorite kinds of milk cheese are: Gloucester, Cheshire, Cheddar,
and Gorgonzola; Milk and Cream cheese: Stilton and Double Gloucester;
Cream cheese: Brie, Neufchâtel, and Camembert.



\needspace{15\baselineskip}
\section*{Fruits}

The varieties of fruits consumed are numerous, and their uses important.
They are chiefly valuable for their sugar, acids, and salts, and are
cooling, refreshing, and stimulating. They act as a tonic, and assist in
purifying the blood. Many contain a jelly-like substance, called pectin,
and several contain starch, which during the ripening process is
converted into glucose. Bananas, dates, figs, prunes, and grapes, owing
to their large amount of sugar, are the most nutritious. Melons,
oranges, lemons, and grapes contain the largest amount of water. Apples,
lemons, and oranges are valuable for their potash salts, and oranges and
lemons especially valuable for their citric acid. It is of importance to
those who are obliged to exclude much sugar from their dietary, to know
that plums, peaches, apricots, and raspberries have less sugar than
other fruits; apples, sweet cherries, grapes, and pears contain the
largest amount. Apples are obtainable nearly all the year, and on
account of their variety, cheapness, and abundance, are termed queen of
fruits.

Thoroughly ripe fruits should be freely indulged in, and to many are
more acceptable than desserts prepared in the kitchen. If possible,
fruits should always appear on the breakfast-table. In cases where
uncooked fruit cannot be freely eaten, many kinds may be cooked and
prove valuable. Never eat unripe fruit, or that which is beginning to
decay. Fruits should be wiped or rinsed before serving.



\needspace{15\baselineskip}
\section*{Vegetable Acids, And Where Found}

The principal vegetable acids are:

\textbf{I.} Acetic (HC\textsubscript{2}H\textsubscript{3}O\textsubscript{2}), found in wine and vinegar.

\textbf{II.} Tartaric (H\textsubscript{2}C\textsubscript{4}H\textsubscript{4}O\textsubscript{6}), found in grapes, pineapples, and
tamarinds.

\textbf{III.} Malic, much like tartaric, found in apples, pears, peaches,
apricots, gooseberries, and currants.

\textbf{IV.} Citric (H\textsubscript{3}C\textsubscript{6}H\textsubscript{5}O\textsubscript{7}), found in lemons, oranges, limes, and
citron.

\textbf{V.} Oxalic (H\textsubscript{2}C\textsubscript{2}O\textsubscript{4}), found in rhubarb and sorrel.

To these may be added tannic acid, obtained from gall nuts. Some fruits
contain two or more acids. Malic and citric are found in strawberries,
raspberries, gooseberries, and cherries; malic, citric, and oxalic in
cranberries.



\needspace{15\baselineskip}
\section*{Condiments}

Condiments are not classed among foods, but are known as food adjuncts.
They are used to stimulate the appetite by adding flavor to food. Among
the most important are salt, spices, and various flavorings. Salt,
according to some authorities, is called a food, being necessary to
life.

\textbf{Black pepper} is ground peppercorns. Peppercorns are the dried berries
of \textit{Piper nigrum}, grown in the West Indies, Sumatra, and other eastern
countries.

\textbf{White pepper} is made from the same berry, the outer husk being removed
before grinding. It is less irritating than black pepper to the coating
of the stomach.

\textbf{Cayenne pepper} is the powdered pod of \textit{Capsicum} grown on the eastern
coast of Africa and in Zanzibar.

\textbf{Mustard} is the ground seed of two species of the Brassica. \_Brassica
alba\textit{ yields white mustard seeds; }Brassica nigra\_, black mustard seeds.
Both species are grown in Europe and America.

\textbf{Ginger} is the pulverized dried root of \textit{Zanzibar officinale}, grown in
Jamaica, China, and India. Commercially speaking, there are three
grades,--Jamaica, best and strongest; Cochin, and African.

\textbf{Cinnamon} is the ground inner bark of \textit{Cinnamomum zeylanicum},
principally grown in Ceylon. The cinnamon of commerce (cassia) is the
powdered bark of different species of the same shrub, which is
principally grown in China, and called Chinese cinnamon. It is cheaper
than true cinnamon.

\textbf{Clove} is the ground flower buds of \textit{Caryophyllus aromaticus}, native
to the Moluccas or Spice Islands, but now grown principally in Zanzibar,
Pemba, and the West Indies.

\textbf{Pimento} (commonly called allspice) is the ground fruit of \_Eugenia
pimenta\_, grown in Jamaica and the West Indies.

\textbf{Nutmeg} is the kernel of the fruit of the \textit{Myristica fragans}, grown in
Banda Islands.

\textbf{Mace.} The fibrous network which envelops the nutmeg seed constitutes
the mace of commerce.

\textbf{Vinegar} is made from apple cider, malt, and wine, and is the product
of fermentation. It is a great preservative; hence its use in the making
of pickles, sauces, and other condiments. The amount of acetic acid in
vinegar varies from two to seven per cent.

\textbf{Capers} are flower buds of \textit{Capparis spinosa}, grown in countries
bordering the Mediterranean. They are preserved in vinegar, and bottled
for importation.

\textbf{Horseradish} is the root of \textit{Cochliaria armoracia},--a plant native to
Europe, but now grown in our own country. It is generally grated, mixed
with vinegar, and bottled.



\needspace{15\baselineskip}
\section*{Flavoring Extracts}

Many flavoring extracts are on the market. Examples: almond, vanilla,
lemon, orange, peach, and rose. These are made from the flower, fruit,
or seed from which they are named. Strawberry, pineapple, and banana
extracts are manufactured from chemicals.








\chapter{Cookery}



Cookery is the art of preparing food for the nourishment of the body.

Prehistoric man may have lived on uncooked foods, but there are no
savage races to-day who do not practise cookery in some way, however
crude. Progress in civilization has been accompanied by progress in
cookery.

Much time has been given in the last few years to the study of foods,
their necessary proportions, and manner of cooking them. Educators have
been shown by scientists that this knowledge should be disseminated; as
a result, “Cookery” is found in the curriculum of public schools of many
of our towns and cities.

Food is cooked to develop new flavors, to make it more palatable and
digestible, and to destroy micro-organisms. For cooking there are three
essentials (besides the material to be cooked),--heat, air, and moisture.

\textbf{Heat} is molecular motion, and is produced by combustion. Heat used for
cookery is obtained by the combustion of inflammable substances--wood,
coal, charcoal, coke, gas, gasoline, kerosene, and alcohol--called fuels.
Heat for cookery is applied by radiation, conduction, and convection.

\textbf{Air} is composed of oxygen, nitrogen, and argon, and surrounds
everything. Combustion cannot take place without it, the oxygen of the
air being the only supporter of combustion.

\textbf{Moisture}, in the form of water, either found in the food or added to
it.

The combined effect of heat and moisture swells and bursts
starch-grains; hardens albumen in eggs, fish, and meat; softens fibrous
portions of meat, and cellulose of vegetables.

Among fuels, kerosene oil is the cheapest; gas gives the greatest amount
of heat in the shortest time. \textit{Soft wood}, like pine, on account of its
coarse fibre, burns quickly; therefore makes the best kindling. \_Hard
wood\_, like oak and ash, having the fibres closely packed, burns slowly,
and is used in addition to pine wood for kindling coal. Where only wood
is used as a fuel, it is principally hard wood.

\textit{Charcoal} for fuel is produced by the smothered combustion of wood. It
gives an intense, even heat, therefore makes a good broiling fire. Its
use for kindling is not infrequent.

There are two kinds of coal: \textit{Anthracite}, or \textit{hard coal}. Examples:
Hard and free-burning White Ash, Shamokin, and Franklin. Nut is any kind
of hard coal obtained from screenings. \textit{Bituminous}, or \textit{soft coal}.
Example: cannel coal.

\textit{Coke} is the solid product of carbonized coal, and bears the same
relation to coal that charcoal bears to wood.

\textit{Alcohol} is employed as fuel when the chafing-dish is used.



\needspace{15\baselineskip}
\section*{Fire}

Fire for cookery is confined in a stove or range, so that heat may be
utilized and regulated. Flame-heat is obtained from kerosene, gas, or
alcohol, as used in oil-stoves, gas-stoves or gas-ranges, and
chafing-dishes.

\textbf{A cooking-stove} is a large iron box set on legs. It has a fire-box in
the front, the sides of which are lined with fireproof material similar
to that of which bricks are made. The bottom is furnished with a movable
iron grate. Underneath the fire-box is a space which extends from the
grate to a pan for receiving ashes. At the back of fire-box is a
compartment called the oven, accessible on each side of the stove by a
door. Between the oven and the top of the stove is a space for the
circulation of air.

Stoves are connected with chimney-flues by means of a stovepipe, and
have dampers to regulate the supply of air and heat, and as an outlet
for smoke and gases.

The damper below the fire-box is known as the \textit{front damper}, by means
of which the air supply is regulated, thus regulating the heat.

The oven is heated by a circulation of hot air. This is accomplished by
closing the \textit{oven-damper}, which is situated near the oven. When this
damper is left open, the hot air rushes up the chimney. The damper near
the chimney is known as the \textit{chimney-damper}. When open it gives a free
outlet for the escape of smoke and gas. When partially closed, as is
usually the case in most ranges, except when the fire is started, it
serves as a saver of heat. There is also a \textit{check}, which, when open,
cools the fire and saves heat, but should always be closed except when
used for this purpose.

Stoves are but seldom used, portable ranges having taken their places.

\textbf{A portable range} is a cooking-stove with one oven door; it often has
an under oven, of use for warming dishes and keeping food hot.

\textbf{A set range} is built in a fireplace. It usually has two ovens, one on
each side of the fire-box, or two above it at the back. Set ranges, as
they consume so large an amount of fuel, are being replaced by portable
ones.



\needspace{15\baselineskip}
\section*{How To Build A Fire}

Before starting to build a fire, free the grate from ashes. To do this,
put on covers, close front and back dampers, and open oven-damper; turn
grate, and ashes will fall into the ash receiver. If these rules are not
followed, ashes will fly over the room. Turn grate back into place,
remove the covers over fire-box, and cover grate with pieces of paper
(twisted in centre and left loose at the ends). Cover paper with small
sticks, or pieces of pine wood, being sure that the wood reaches the
ends of fire-box, and so arranged that it will admit air. Over pine wood
arrange hard wood; then sprinkle with two shovelfuls of coal. Put on
covers, open closed dampers, strike a match,--sufficient friction is
formed to burn the phosphorus, this in turn lights the sulphur, and the
sulphur the wood,--then apply the lighted match under the grate, and you
have a fire.

Now blacken the stove. Begin at front of range, and work towards the
back; as the iron heats, a good polish may be obtained. When the wood is
thoroughly kindled, add more coal. A blue flame will soon appear, which
is the gas (CO) in the coal burning to carbon dioxide (CO\textsubscript{2}), when the
blue flame changes to a white flame; then the oven-damper should be
closed. In a few moments the front damper may be nearly closed, leaving
space to admit sufficient oxygen to feed the fire. It is sometimes
forgotten that oxygen is necessary to keep a fire burning. As soon as
the coal is well ignited, half close the chimney-damper, unless the
draft be very poor.

Never allow the fire-box to be more than three-fourths filled. When
full, the draft is checked, a larger amount of fuel is consumed, and
much heat is lost. This is a point that should be impressed on the mind
of the cook.

Ashes must be removed and sifted daily; pick over and save good
coals,--which are known as cinders,--throwing out useless pieces, known as
clinkers.

If a fire is used constantly during the day, replenish coal frequently,
but in small quantities. If for any length of time the fire is not
needed, open check, the dampers being closed; when again wanted for use,
close check, open front damper, and with a poker rake out ashes from
under fire, and wait for fire to burn brightly before adding new coal.

Coal when red hot has parted with most of its heat. Some refuse to
believe this, and insist upon keeping dampers open until most of the
heat has escaped into the chimney.

To keep a fire over night, remove the ashes from under the fire, put on
enough coal to fill the box, close the dampers, and lift the back covers
enough to admit air. This is better than lifting the covers over the
fire-box and prevents poisonous gases entering the room.



\needspace{15\baselineskip}
\section*{Ways Of Cooking}

The principal ways of cooking are boiling, broiling, stewing, roasting,
baking, frying, sautéing, braising, and fricasseeing.

\textbf{Boiling} is cooking in boiling water. Solid food so cooked is called
boiled food, though literally this expression is incorrect. Examples:
boiled eggs, potatoes, mutton, etc.

Water boils at 212deg F. (sea level), and simmers at 185deg F. Slowly
boiling water has the same temperature as rapidly boiling water,
consequently is able to do the same work,--a fact often forgotten by the
cook, who is too apt “to wood” the fire that water may boil vigorously.

Watery vapor and steam pass off from boiling water. Steam is invisible;
watery vapor is visible, and is often miscalled steam. Cooking utensils
commonly used admit the escape of watery vapor and steam; thereby much
heat is lost if food is cooked in rapidly boiling water.

Water is boiled for two purposes: first, cooking of itself to destroy
organic impurities; second, for cooking foods. Boiling water toughens
and hardens albumen in eggs; toughens fibrin and dissolves tissues in
meat; bursts starch-grains and softens cellulose in cereals and
vegetables. Milk should never be allowed to boil. At boiling temperature
(214deg F.) the casein is slightly hardened, and the fat is rendered more
difficult of digestion. Milk heated over boiling water, as in a double
boiler, is called \textit{scalded milk}, and reaches a temperature of 196deg F.
When foods are cooked over hot water the process is called steaming.

\textbf{Stewing} is cooking in a small amount of hot water for a long time at
low temperature; it is the most economical way of cooking meats, as all
nutriment is retained, and the ordinary way of cooking cheaper cuts.
Thus fibre and connective tissues are softened, and the whole is made
tender and palatable.

\textbf{Broiling} is cooking over or in front of a clear fire. The food to be
cooked is usually placed in a greased broiler or on a gridiron held near
the coals, turned often at first to sear the outside,--thus preventing
escape of inner juices,--afterwards turned occasionally. Tender meats and
fish may be cooked in this way. The flavor obtained by broiling is
particularly fine; there is, however, a greater loss of weight in this
than in any other way of cooking, as the food thus cooked is exposed to
free circulation of air. When coal is not used, or a fire is not in
condition for broiling, a plan for \textit{pan broiling} has been adopted. This
is done by placing food to be cooked in a hissing hot frying-pan,
turning often as in broiling.

\textbf{Roasting} is cooking before a clear fire, with a reflector to
concentrate the heat. Heat is applied in the same way as for broiling,
the difference being that the meat for roasting is placed on a spit and
allowed to revolve, thicker pieces alway being employed. Tin-kitchens
are now but seldom used. Meats cooked in a range oven, though really
baked, are said to be roasted. Meats so cooked are pleasing to the sight
and agreeable to the palate, although, according to Edward Atkinson, not
so easily digested as when cooked at a lower temperature in the Aladdin
oven.

\textbf{Baking} is cooking in a range oven.

\textbf{Frying} is cooking by means of immersion in deep fat raised to a
temperature of 350deg to 400deg F. For frying purposes olive oil, lard, beef
drippings, cottolene, coto suet, and cocoanut butter are used. A
combination of two-thirds lard and one-third beef suet (tried out and
clarified) is better than lard alone. Cottolene, coto suet, and cocoanut
butter are economical, inasmuch as they may be heated to a high
temperature without discoloring, therefore may be used for a larger
number of fryings. Cod fat obtained from beef is often used by \textit{chefs}
for frying.

Great care should be taken in frying that fat is of the right
temperature; otherwise food so cooked will absorb fat.

Nearly all foods which do not contain eggs are dipped in flour or
crumbs, egg, and crumbs, before frying. The intense heat of fat hardens
the albumen, thus forming a coating which prevents food from “soaking
fat.”

When meat or fish is to be fried, it should be kept in a warm room for
some time previous to cooking, and wiped as dry as possible. If cold, it
decreases the temperature of the fat to such extent that a coating is
not formed quickly enough to prevent fat from penetrating the food. The
ebullition of fat is due to water found in food to be cooked.

Great care must be taken that too much is not put into the fat at one
time, not only because it lowers the temperature of the fat, but because
it causes it to bubble and go over the sides of the kettle. It is not
fat that boils, but water which fat has received from food.

All fried food on removal from fat should be drained on brown paper.

\textbf{Rules for Testing Fat for Frying.} 1. When the fat begins to smoke,
drop in an inch cube of bread from soft part of loaf, and if in forty
seconds it is golden brown, the fat is then of right temperature for
frying any cooked mixture.

2. Use same test for uncooked mixtures, allowing one minute for bread to
brown.

Many kinds of food may be fried in the same fat; new fat should be used
for batter and dough mixtures, potatoes, and fishballs; after these,
fish, meat, and croquettes. Fat should be frequently clarified.

\textbf{To Clarify Fat.} Melt fat, add raw potato cut in quarter-inch slices,
and allow fat to heat gradually; when fat ceases to bubble and potatoes
are well browned, strain through double cheese-cloth, placed over wire
strainer, into a pan. The potato absorbs any odors or gases, and
collects to itself some of the sediment, remainder settling to bottom of
kettle.

When small amount of fat is to be clarified, add to cold fat boiling
water, stir vigorously, and set aside to cool; the fat will form a cake
on top, which may be easily removed; on bottom of the cake will be found
sediment, which may be readily scraped off with a knife.

Remnants of fat, either cooked or uncooked, should be saved and tried
out, and when necessary clarified.

Fat from beef, poultry, chicken, and pork, may be used for shortening or
frying purposes; fat from mutton and smoked meats may be used for making
hard and soft soap; fat removed from soup stock, the water in which
corned beef has been cooked, and drippings from roast beef, may be tried
out, clarified, and used for shortening or frying purposes.

\textbf{To Try out Fat.} Cut in small pieces and melt in top of a double
boiler; in this way it will require less watching than if placed in
kettle on the back of range. Leaf lard is tried out in the same way; in
cutting the leaf, remove membrane. After straining lard, that which
remains may be salted, pressed, and eaten as a relish, and is called
scraps.

\textbf{Sautéing} is frying in a small quantity of fat. Food so cooked is much
more difficult of digestion than when fried in deep fat; it is
impossible to cook in this way without the food absorbing fat. A
frying-pan or griddle is used; the food is cooked on one side, then
turned, and cooked on the other.

\textbf{Braising} is stewing and baking (meat). Meat to be braised is
frequently first sautéd to prevent escape of much juice in the gravy.
The meat is placed in a pan with a small quantity of stock or water,
vegetables (carrot, turnip, celery, and onion) cut in pieces, salt,
pepper, and sweet herbs. The pan should have a tight-fitting cover. Meat
so prepared should be cooked in an oven at low uniform temperature for a
long time. This is an economical way of cooking, and the only way
besides stewing or boiling of making a large piece of tough meat
palatable and digestible.

\textbf{Fricasseeing} is sautéing and serving with a sauce. Tender meat is
fricasseed without previous cooking; less tender meat requires cooking
in hot water before fricasseeing. Although veal is obtained from a young
creature, it requires long cooking; it is usually sautéd, and then
cooked in a sauce at low temperature for a long time.



\needspace{15\baselineskip}
\section*{Various Ways Of Preparing Food For Cooking}

\textbf{Egging and Crumbing.} Use for crumbing dried bread crumbs which have
been rolled and sifted, or soft stale bread broken in pieces and forced
through a colander. An ingenious machine on the market, “The Bread
Crumber,” does this work. Egg used for crumbing should be broken into a
shallow plate and beaten with a silver fork to blend yolk and white;
dilute each egg with two tablespoons water. The crumbs should be taken
on a board; food to be fried should be first rolled in crumbs (care
being taken that all parts are covered with crumbs), then dipped in egg
mixture (equal care being taken to cover all parts), then rolled in
crumbs again; after the last crumbing remove food to a place on the
board where there are no crumbs, and shake off some of the outer ones
which make coating too thick. A broad-bladed knife with short handle--the
Teller knife--is the most convenient utensil for lifting food to be
crumbed from egg mixture. Small scallops, oysters, and crabs are more
easily crumbed by putting crumbs and fish in paper and shaking paper
until the fish is covered with crumbs. The object of first crumbing is
to dry the surface that egg may cling to it; and where a thin coating is
desired flour is often used in place of crumbs.

\textbf{Larding} is introducing small pieces of fat salt pork or bacon through
the surface of uncooked meat. The flavor of lean and dry meat is much
improved by larding; tenderloin of beef (fillet), grouse, partridge,
pigeon, and liver are often prepared in this way. Pig pork being firm,
is best for larding. Pork should be kept in a cold place that it may be
well chilled. Remove rind and use the part of pork which lies between
rind and vein. With sharp knife (which is sure to make a clean cut)
remove slices a little less than one-fourth inch thick; cut the slices
into strips a little less than one-fourth inch wide; these strips should
be two and one-fourth inches long, and are called \textit{lardoons}. Lardoons
for small birds--quail, for example--should be cut smaller and not quite
so long. To lard, insert one end of lardoon into larding-needle, hold
needle firmly, and with pointed end take up a stitch one-third inch deep
and three-fourths inch wide; draw needle through, care being taken that
lardoon is left in meat and its ends project to equal lengths. Arrange
lardoons in parallel rows, one inch apart, stitches in the alternate
rows being directly underneath each other. Lard the upper surface of
cuts of meat with the grain, never across it. In birds, insert lardoons
at right angles to breastbone on either side. When large lardoons are
forced through meat from surface to surface, the process is called
daubing. Example: Beef à la mode. Thin slices of fat salt pork placed
over meat may be substituted for larding, but flavor is not the same as
when pork is drawn through flesh, and the dish is far less sightly.

\textbf{Boning} is removing bones from meat or fish, leaving the flesh nearly
in its original shape. For boning, a small sharp knife with pointed
blade is essential. Legs of mutton and veal and loins of beef may be
ordered boned at market, no extra charge being made.

Whoever wishes to learn how to bone should first be taught boning of a
small bird; when this is accomplished, larger birds, chickens, and
turkeys may easily be done, the processes varying but little. In large
birds tendons are drawn from legs, and the wings are left on and boned.



\needspace{15\baselineskip}
\subsection*{How to Bone a Bird}

In buying birds for boning, select those which have been fresh killed,
dry picked, and not drawn. Singe, remove pinfeathers, head, and feet,
and cut off wings close to body. Lay bird on a board, breast down.

Begin at neck and with sharp knife cut through the skin the entire
length of body. Scrape the flesh from backbone until end of one
shoulder-blade is found; scrape flesh from shoulder-blade and continue
around wing joint, cutting through tendinous portions which are
encountered; then bone other side. Scrape skin from backbone the entire
length of body, working across the ribs. Free wishbone and collar-bones,
at same time removing crop and windpipe; continue down breastbone,
particular care being taken not to break the skin as it lies very near
bone, or to cut the delicate membranes which enclose entrails. Scrape
flesh from second joints and drumsticks, laying it back and drawing off
as a glove may be drawn from the hand. Withdraw carcass and put flesh
back in its original shape. In large birds where wings are boned, scrape
flesh to middle joint, where bone should be broken, leaving bone at tip
end to assist in preserving shape.



\needspace{15\baselineskip}
\subsection*{How to Measure}

Correct measurements are absolutely necessary to insure the best
results. Good judgment, with experience, has taught some to measure by
sight; but the majority need definite guides.

Tin, granite-ware, and glass measuring cups, divided in quarters or
thirds, holding one half-pint, and tea and table spoons of regulation
sizes,--which may be bought at any store where kitchen furnishings are
sold,--and a case knife, are essentials for correct measurement.
Mixing-spoons, which are little larger than tablespoons, should not be
confounded with the latter.

\textbf{Measuring Ingredients.} Flour, meal, powdered and confectioners' sugar,
and soda should be sifted before measuring. Mustard and baking-powder,
from standing in boxes, settle, therefore should be stirred to lighten;
salt frequently lumps, and these lumps should be broken. A \textit{cupful} is
measured level. To measure a cupful, put in the ingredient by spoonfuls
or from a scoop, round slightly, and level with a case knife, care being
taken not to shake the cup. \_A tablespoonful is measured level. A
teaspoonful is measured level.\_

To measure tea or table spoonfuls, dip the spoon in the ingredient,
fill, lift, and level with a knife, the sharp edge of knife being toward
tip of spoon. Divide with knife lengthwise of spoon, for a
half-spoonful; divide halves crosswise for quarters, and quarters
crosswise for eighths. Less than one-eighth of a teaspoonful is
considered a few grains.

\textbf{Measuring Liquids.} A cupful of liquid is all the cup will hold.

A tea or table spoonful is all the spoon will hold.

\textbf{Measuring Butter, Lard, etc.} To measure butter, lard, and other solid
fats, pack solidly into cup or spoon, and level with a knife.

When dry ingredients, liquids, and fats are called for in the same
recipe, measure in the order given, thereby using but one cup.



\needspace{15\baselineskip}
\subsection*{How to Combine Ingredients}

Next to measuring comes care in combining,--a fact not always recognized
by the inexperienced. Three ways are considered,--stirring, beating, and
cutting and folding.

\textbf{To stir}, mix by using circular motion, widening the circles until all
is blended. Stirring is the motion ordinarily employed in all cookery,
alone or in combination with beating.

\textbf{To beat}, turn ingredient or ingredients over and over, continually
bringing the under part to the surface, thus allowing the utensil used
for beating to be constantly brought in contact with bottom of the dish
and throughout the mixture.

\textbf{To cut and fold}, introduce one ingredient into another ingredient or
mixture by two motions: with a spoon, a repeated vertical downward
motion, known as cutting; and a turning over and over of mixture,
allowing bowl of spoon each time to come in contact with bottom of dish,
is called folding. These repeated motions are alternated until thorough
blending is accomplished.

\textbf{By stirring}, ingredients are mixed; \textit{by beating}, a large amount of
air is enclosed; \textit{by cutting and folding}, air already introduced is
prevented from escaping.



\needspace{15\baselineskip}
\subsection*{Ways of Preserving}


\begin{itemize}
\setlength{\itemsep}{0pt}
\item By Freezing: Foods which spoil readily are frozen for transportation, and must be kept packed in ice until used. Examples: Fish and poultry.
\item By Refrigeration: Foods so preserved are kept in cold storage. The cooling is accomplished by means of ice, or by a machine where compressed gas is cooled and then permitted to expand. Examples: meat, milk, butter, eggs, etc.
\item By Canning: Which is preserving in air-tight glass jars, or tin cans hermetically sealed. When fruit is canned, sugar is usually added.
\item By Sugar: Examples: fruit juices and condensed milk.
\item By Exclusion of Air: Foods are preserved by exclusion of air in other ways than canning. Examples: grapes in bran, eggs in lime water, etc.
\item By Drying: Drying consists in evaporation of nearly all moisture, and is generally combined with salting, except in vegetables and fruits.
\item By Evaporation: There are examples where considerable moistureremains, though much is driven off. Example: beef extract.
\item By Salting: There are two kinds of salting,--dry, and corning or salting in brine. Examples: salt codfish, beef, pork, tripe, etc.
\item By Smoking: Some foods, after being salted, are hung in a closed room for several hours, where hickory wood is allowed to smother. Examples: ham, beef, and fish.
\item By Pickling: Vinegar, to which salt is added, and sometimes sugar and spices, is scalded; and cucumbers, onions, and various kinds of fruit are allowed to remain in it.
\item By Oil: Examples: sardines, anchovies, etc.
\item By Antiseptics: The least wholesome way is by the use of antiseptics. Borax and salicylic acid, when employed, should be used sparingly.
\end{itemize}




\vfill
\begin{center}
\arrayrulecolor{tablerowgray}
\begin{tabularx}{\textwidth}{Xr}
\hline
\multicolumn{2}{c}{\textbf{Table Of Measures And Weights}} \\
\hline
2 cups butter (packed solidly) & 1 pound \\
\hline
4 cups flour (pastry) & 1 pound \\
\hline
2 cups granulated sugar & 1 pound \\
\hline
2 2/3 cups powdered sugar & 1 pound \\
\hline
3 1/2 cups confectioners' sugar & 1 pound \\
\hline
2 2/3 cups brown sugar & 1 pound \\
\hline
2 2/3 cups oatmeal & 1 pound \\
\hline
4 3/4 cups rolled oats & 1 pound \\
\hline
2 2/3 cups granulated corn meal & 1 pound \\
\hline
4 1/3 cups rye meal & 1 pound \\
\hline
1 7/8 cups rice & 1 pound \\
\hline
4 1/2 cups Graham flour & 1 pound \\
\hline
3 7/8 cups entire wheat flour & 1 pound \\
\hline
4 1/3 cups coffee & 1 pound \\
\hline
2 cups finely chopped meat & 1 pound \\
\hline
9 large eggs & 1 pound \\
\hline
1 square Baker's chocolate & 1 ounce \\
\hline
1/3 cup almonds blanched and chopped & 1 ounce \\
\hline
3 teaspoons & 1 tablespoon \\
\hline
16 tablespoons & 1 cup \\
\hline
2 tablespoons butter & 1 ounce \\
\hline
4 tablespoons flour & 1 ounce \\
\hline
\end{tabularx}
\arrayrulecolor{black}
\end{center}
\vfill
\pagebreak



\vfill
\begin{center}
\arrayrulecolor{tablerowgray}
\begin{tabularx}{\textwidth}{Xrr}
\hline
\multicolumn{3}{c}{\textbf{Time Tables For Cooking — Boiling}} \\
\hline
ARTICLES & Hours & Minutes \\
\hline
Coffee &  & 1 to 3 \\
\hline
Eggs, soft cooked &  & 6 to 8 \\
\hline
Eggs, hard cooked &  & 35 to 45 \\
\hline
Mutton, leg & 2 to 3 & \\
\hline
Ham, weight 12 to 14 lbs. & 4 to 5 & \\
\hline
Corned Beef or Tongue & 3 to 4 & \\
\hline
Turkey, weight 9 lbs. & 2 to 3 & \\
\hline
Fowl, weight 4 to 5 lbs. & 2 to 3 & \\
\hline
Chicken, weight 3 lbs. & 1 to 1¼ & \\
\hline
Lobster &  & 25 to 30 \\
\hline
Cod and Haddock, weight 3 to 5 lbs. &  & 20 to 30 \\
\hline
Halibut, thick piece, weight 2 to 3 lbs. &  & 30 \\
\hline
Bluefish and Bass, weight 4 to 5 lbs. &  & 40 to 45 \\
\hline
Salmon, weight 2 to 3 lbs. &  & 30 to 35 \\
\hline
Small Fish &  & 6 to 10 \\
\hline
Potatoes, white &  & 20 to 30 \\
\hline
Potatoes, sweet &  & 15 to 25 \\
\hline
Asparagus &  & 20 to 30 \\
\hline
Peas &  & 20 to 60 \\
\hline
String Beans & 1 to 2½ & \\
\hline
Lima and other Shell Beans & 1 to 1¼ & \\
\hline
Beets, young &  & 45 \\
\hline
Beets, old & 3 to 4 & \\
\hline
Cabbage &  & 35 to 60 \\
\hline
Oyster Plant &  & 45 to 60 \\
\hline
Turnips &  & 30 to 45 \\
\hline
Onions &  & 45 to 60 \\
\hline
Parsnips &  & 30 to 45 \\
\hline
Spinach &  & 25 to 30 \\
\hline
Green Corn &  & 12 to 20 \\
\hline
Cauliflower &  & 20 to 25 \\
\hline
Brussels Sprouts &  & 15 to 20 \\
\hline
Tomatoes, stewed &  & 15 to 20 \\
\hline
Rice &  & 20 to 25 \\
\hline
Macaroni &  & 20 to 30 \\
\hline
\arrayrulecolor{black}
\end{tabularx}
\end{center}
\vfill
\pagebreak


\vfill
\begin{center}
\arrayrulecolor{tablerowgray}
\begin{tabularx}{\textwidth}{Xrr}
\hline
\multicolumn{3}{c}{\textbf{Time Tables For Cooking — Broiling}} \\
\hline
ARTICLES & Hours & Minutes \\
\hline
Steak, one inch thick &  & 4 to  6 \\
\hline
Steak, one and one-half inches thick &  & 8 to 10 \\
\hline
Lamb or Mutton Chops &  & 6 to  8 \\
\hline
Lamb or Mutton Chops in paper cases &  & 10 \\
\hline
Quails or Squabs &  & 8 \\
\hline
Quails or Squabs in paper cases &  & 10 to 12 \\
\hline
Chickens &  & 20 \\
\hline
Shad, Bluefish, and Whitefish &  & 15 to 20 \\
\hline
Slices of Fish, Halibut, Salmon, and Swordfish &  & 12 to 15 \\
\hline
Small, thin Fish &  & 5 to 8 \\
\hline
Liver and Tripe &  & 4 to 5 \\
\hline
\arrayrulecolor{black}
\end{tabularx}
\end{center}
\vfill
\pagebreak


\vfill
\begin{center}
\arrayrulecolor{tablerowgray}
\begin{tabularx}{\textwidth}{Xrr}
\hline
\multicolumn{3}{c}{\textbf{Time Tables For Cooking — Baking}} \\
\hline
ARTICLES & Hours & Minutes \\
\hline
Bread (white loaf) &  & 45 to 60 \\
\hline
Bread (Graham loaf) &  & 35 to 45 \\
\hline
Bread (sticks) &  & 10 to 15 \\
\hline
Biscuits or Rolls (raised) &  & 12 to 20 \\
\hline
Biscuits (baking-powder) &  & 12 to 15 \\
\hline
Gems &  & 25 to 30 \\
\hline
Muffins (raised) &  & 30 \\
\hline
Muffins (baking-powder) &  & 20 to 25 \\
\hline
Corn Cake (thin) &  & 15 to 20 \\
\hline
Corn Cake (thick) &  & 30 to 35 \\
\hline
Gingerbread &  & 20 to 30 \\
\hline
Cookies &  & 6 to 10 \\
\hline
Sponge Cake &  & 45 to 60 \\
\hline
Cake (layer) &  & 20 to 30 \\
\hline
Cake (loaf) &  & 40 to 60 \\
\hline
Cake (pound) & 1¼ to 1½ & \\
\hline
Cake (fruit) & 1¼ to 2 & \\
\hline
Cake (wedding) & 3 & \\
\hline
Baked batter puddings &  & 35 to 45 \\
\hline
Bread puddings & 1 & \\
\hline
Tapioca or Rice Pudding & 1 & \\
\hline
Rice Pudding (poor man's) & 2 to 3 & \\
\hline
Indian Pudding & 2 to 3 & \\
\hline
Plum Pudding & 2 to 3 & \\
\hline
Custard Pudding &  & 30 to 45 \\
\hline
Custard (baked in cups) &  & 20 to 25 \\
\hline
Pies &  & 30 to 50 \\
\hline
Tarts &  & 15 to 20 \\
\hline
Patties &  & 20 to 25 \\
\hline
Vol-au-vent &  & 50 to 60 \\
\hline
\arrayrulecolor{black}
\end{tabularx}

\vspace{2em}

\arrayrulecolor{tablerowgray}
\begin{tabularx}{\textwidth}{Xrr}
\hline
ARTICLES & Hours & Minutes \\
\hline
Cheese Straws &  & 8 to 10 \\
\hline
Scalloped Oysters &  & 25 to 30 \\
\hline
Scalloped dishes of cooked mixtures &  & 12 to 15 \\
\hline
Baked Beans & 6 to 8 & \\
\hline
Braised Beef & 3½ to 4½ & \\
\hline
Beef, sirloin or rib, rare, weight  5 lbs. & 1 & 5 \\
\hline
Beef, sirloin or rib, rare, weight 10 lbs. & 1 & 30 \\
\hline
Beef, sirloin or rib, well done, weight  5 lbs. & 1 & 20 \\
\hline
Beef, sirloin or rib, well done, weight 10 lbs. & 1 & 50 \\
\hline
Beef, rump, rare, weight 10 lbs. & 1 & 35 \\
\hline
Beef, rump, well done, weight 10 lbs. & 1 & 55 \\
\hline
Beef, (fillet) &  & 20 to 30 \\
\hline
Mutton (saddle) & 1¼ to 1½ & \\
\hline
Lamb (leg) & 1¼ to 1¾ & \\
\hline
Lamb (fore-quarter) & 1 to 1¼ & \\
\hline
Lamb (chops) in paper cases &  & 15 to 20 \\
\hline
Veal (leg) & 3½ to 4 & \\
\hline
Veal (loin) & 2 to 3 & \\
\hline
Pork (chine or sparerib) & 3 to 3½ & \\
\hline
Chicken, weight 3 to 4 lbs. & 1 to 1½ & \\
\hline
Turkey, weight 9 lbs. & 2½ to 3 & \\
\hline
Goose, weight 9 lbs. & 2 & \\
\hline
Duck (domestic) & 1 to 1¼ & \\
\hline
Duck (wild) &  & 20 to 30 \\
\hline
Grouse &  & 25 to 30 \\
\hline
Partridge &  & 45 to 50 \\
\hline
Pigeons (potted) & 2 & \\
\hline
Fish (thick), weight 3 to 4 lbs. &  & 45 to 60 \\
\hline
Fish (small) &  & 20 to 30 \\
\hline
\arrayrulecolor{black}
\end{tabularx}
\end{center}
\vfill
\pagebreak


\vfill
\begin{center}
\arrayrulecolor{tablerowgray}
\begin{tabularx}{\textwidth}{Xrr}
\hline
\multicolumn{3}{c}{\textbf{Time Tables For Cooking — Frying}} \\
\hline
ARTICLES & Hours & Minutes \\
\hline
Muffins, Fritters, and Doughnuts &  & 3 to 5 \\
\hline
Croquettes and Fishballs &  & 1 \\
\hline
Potatoes, raw &  & 4 to 8 \\
\hline
Breaded Chops &  & 5 to 8 \\
\hline
Fillets of Fish &  & 4 to 6 \\
\hline
Smelts, Trout, and other small Fish &  & 3 to 5 \\
\hline
\arrayrulecolor{black}
\end{tabularx}
\end{center}
\vfill
\pagebreak


\chapter{Beverages}



A beverage is any drink. Water is the beverage provided for man by
Nature. Water is an essential to life. All beverages contain a large
percentage of water, therefore their uses should be considered:

   I. To quench thirst.

  II. To introduce water into the circulatory system.

 III. To regulate body temperature.

  IV. To assist in carrying off waste.

   V. To nourish.

  VI. To stimulate the nervous system and various organs.

 VII. For medicinal purposes.

Freshly boiled water should be used for making hot beverages; freshly
drawn water for making cold beverages.



\needspace{15\baselineskip}
\section*{Tea}

Tea is used by more than one-half the human race; and, although the
United States is not a tea-drinking country, one and one-half pounds are
consumed per capita per annum.

All tea is grown from one species of shrub, \textit{Thea}, the leaves of which
constitute the tea of commerce. Climate, elevation, soil, cultivation,
and care in picking and curing all go to make up the differences.
First-quality tea is made from young, whole leaves. Two kinds of tea are
considered:

\textit{Black tea}, made from leaves which have been allowed to ferment before
curing.

\textit{Green tea}, made from unfermented leaves artificially colored.

The best black tea comes from India and Ceylon. Some familiar brands are
Oolong, Formosa, English Breakfast, Orange Pekoe, and Flowery Pekoe. The
last two named, often employed at the “five o'clock tea,” command high
prices; they are made from the youngest leaves. Orange Pekoe is scented
with orange leaves. The best green tea comes from Japan. Some familiar
brands are Hyson, Japan, and Gunpowder.

From analysis, it has been found that tea is rich in protein, but taken
as an infusion acts as a stimulant rather than as a nutrient. The
nutriment is gained from sugar and milk served with it. The stimulating
property of tea is due to the alkaloid, \textit{theine}, together with an
essential oil; it contains an astringent, tannin. Black tea contains
less theine, essential oil, and tannin than green tea. The tannic acid,
developed from the tannin by infusion, injures the coating of the
stomach.

Although tea is not a substitute for food, it appears so for a
considerable period of time, as its stimulating effect is immediate. It
is certain that less food is required where much tea is taken, for by
its use there is less wear of the tissues, consequently need of repair.
When taken to excess, it so acts on the nervous system as to produce
sleeplessness or insomnia, and finally makes a complete wreck of its
victim. Taken in moderation, it acts as a mild stimulant, and ingests a
considerable amount of water into the system; it heats the body in
winter, and cools the body in summer. Children should never be allowed
to drink tea, and it had better be avoided by the young, while it may be
indulged in by the aged, as it proves a valuable stimulant as the
functional activities of the stomach become weakened.

Freshly boiled water should be used for making tea. Boiled, because
below the boiling-point the stimulating property, theine, would not be
extracted. Freshly boiled, because long cooking renders it flat and
insipid to taste on account of escape of its atmospheric gases. Tea
should always be infused, never boiled. Long steeping destroys the
delicate flavor by developing a larger amount of tannic acid.



\needspace{15\baselineskip}
\subsection*{How to Make Tea}


\begin{itemize}
\setlength{\itemsep}{0pt}
\setlength{\parsep}{0pt}
\item 3 teaspoons tea
\item 2 cups boiling water
\end{itemize}

\vspace{-0.5em}
\noindent%
Scald an earthen or china teapot.

Put in tea, and pour on boiling water. Let stand on back of range or in
a warm place five minutes. Strain and serve immediately, with or without
sugar and milk. Avoid second steeping of leaves with addition of a few
fresh ones. If this is done, so large an amount of tannin is extracted
that various ills are apt to follow.



\needspace{15\baselineskip}
\subsection*{Five o'Clock Tea}

When tea is made in dining or drawing-room, a “Five o'Clock Tea-kettle”
(Samovar), and tea-ball or teapot are used.



\needspace{15\baselineskip}
\subsection*{Russian Tea}

Follow recipe for making tea. Russian Tea may be served hot or cold, but
always without milk. A thin slice of lemon, from which seeds have been
removed, or a few drops of lemon juice, is allowed for each cup. Sugar
is added according to taste. In Russia a preserved strawberry to each
cup is considered an improvement. We imitate our Russian friends by
garnishing with a candied cherry.



\needspace{15\baselineskip}
\subsection*{De John's Tea}

Follow recipe for making tea and serve hot, allowing three whole cloves
to each cup. Sugar is added according to taste.



\needspace{15\baselineskip}
\subsection*{Iced Tea}


\begin{itemize}
\setlength{\itemsep}{0pt}
\setlength{\parsep}{0pt}
\item 4 teaspoons tea
\item 2 cups boiling water
\end{itemize}

\vspace{-0.5em}
\noindent%
Follow recipe for making tea. Strain into glasses one-third full of
cracked ice. Sweeten to taste, and allow one slice lemon to each glass
tea. The flavor is much finer by chilling the infusion quickly.



\needspace{15\baselineskip}
\subsection*{Wellesley Tea}

Make same as Iced Tea, having three crushed mint leaves in each glass
into which the hot infusion is strained.







\needspace{15\baselineskip}
\section*{Coffee}

The coffee-tree is native to Abyssinia, but is now grown in all tropical
countries. It belongs to the genus \textit{Coffea}, of which there are about
twenty-two species. The seeds of berries of coffee-trees constitute the
coffee of commerce. Each berry contains two seeds, with exception of
maleberry, which is a single round seed. In their natural state they are
almost tasteless; therefore color, shape, and size determine value.
Formerly, coffee was cured by exposure to the sun; but on account of
warm climate and sudden rainfalls, coffee was often injured. By the new
method coffee is washed, and then dried by steam heat.

In coffee plantations, trees are planted in parallel rows, from six to
eight feet apart, and are pruned so as never to exceed six feet in
height. Banana-trees are often grown in coffee plantations, advantage
being taken of their outspreading leaves, which protect coffee-trees
from direct rays of the sun. Brazil produces about two-thirds the coffee
used. Central America, Java, and Arabia are also coffee centres.

Tea comes to us ready for use; coffee needs roasting. In process of
roasting the seeds increase in size, but lose fifteen per cent in
weight. Roasting is necessary to develop the delightful aroma and
flavor. Java coffee is considered finest. Mocha commands a higher price,
owing to certain acidity and sparkle, which alone is not desirable; but
when combined with Java, in proportion of two parts Java to one part
Mocha, the coffee best suited to average taste is made. Some people
prefer Maleberry Java; so especial care is taken to have maleberries
separated, that they may be sold for higher price. Old Government Java
has deservedly gained a good reputation, as it is carefully inspected,
and its sale controlled by Dutch government. Strange as it may seem to
the consumer, all coffee sold as Java does not come from the island of
Java. Any coffee, wherever grown, having same characteristics and
flavor, is sold as Java. The same is true of other kinds of coffee.

The stimulating property of coffee is due to the alkaloid \textit{caffeine},
together with an essential oil. Like tea, it contains an astringent.
Coffee is more stimulating than tea, although, weight for weight, tea
contains about twice as much \textit{theine} as coffee contains \textit{caffeine}. The
smaller proportion of tea used accounts for the difference. A cup of
coffee with breakfast, and a cup of tea with supper, serve as a mild
stimulant for an adult, and form a valuable food adjunct, but should
never be found in the dietary of a child or dyspeptic. Coffee taken in
moderation quickens action of the heart, acts directly upon the nervous
system, and assists gastric digestion. Fatigue of body and mind are much
lessened by moderate use of coffee; severe exposure to cold can be
better endured by the coffee drinker. In times of war, coffee has proved
more valuable than alcoholic stimulants to keep up the enduring power of
soldiers. Coffee acts as an antidote for opium and alcoholic poisoning.
Tea and coffee are much more readily absorbed when taken on an empty
stomach; therefore this should be avoided except when used for medicinal
purposes. Coffee must be taken in moderation; its excessive use means
palpitation of the heart, tremor, insomnia, and nervous prostration.

Coffee is often adulterated with chiccory, beans, peas, and various
cereals, which are colored, roasted, and ground. By many, a small amount
of chiccory is considered an improvement, owing to the bitter principle
and volatile oil which it contains. Chiccory is void of caffeine. The
addition of chiccory may be detected by adding cold water to supposed
coffee; if chiccory is present, the liquid will be quickly discolored,
and chiccory will sink; pure coffee will float.


\textbf{Buying of Coffee.} Coffee should be bought for family use in small
quantities, freshly roasted and ground; or, if one has a coffee-mill, it
may be ground at home as needed. After being ground, unless kept air
tight, it quickly deteriorates. If not bought in air-tight cans, with
tight-fitting cover, or glass jar, it should be emptied into canister as
soon as brought from grocer's.

Coffee may be served as filtered coffee, infusion of coffee, or
decoction of coffee. Commonly speaking, boiled coffee is preferred, and
is more economical for the consumer. Coffee is ground fine, coarse, and
medium; and the grinding depends on the way in which it is to be made.
For filtered coffee have it finely ground; for boiled, coarse or medium.



\needspace{15\baselineskip}
\subsection*{Filtered Coffee}

                        (\textit{French or Percolated})


\begin{itemize}
\setlength{\itemsep}{0pt}
\setlength{\parsep}{0pt}
\item 1 cup coffee (finely ground)
\item 6 cups boiling water
\end{itemize}

\vspace{-0.5em}
\noindent%
Various kinds of coffee pots are on the market for making filtered
coffee. They all contain a strainer to hold coffee without allowing
grounds to mix with infusion. Some have additional vessel to hold
boiling water, upon which coffee-pot may rest.

Place coffee in strainer, strainer in coffee-pot, and pot on the range.
Add gradually boiling water, and allow it to filter. Cover between
additions of water. If desired stronger, re-filter. Serve at once with
cut sugar and cream.

Put sugar and cream in cup before hot coffee. There will be perceptible
difference if cream is added last. If cream is not obtainable, scalded
milk may be substituted, or part milk and part cream may be used, if a
diluted cup of coffee is desired.



\needspace{15\baselineskip}
\subsection*{Boiled Coffee}


\begin{itemize}
\setlength{\itemsep}{0pt}
\setlength{\parsep}{0pt}
\item 1 cup coffee
\item 1 egg
\item 1 cup cold water
\item 6 cups boiling water
\end{itemize}

\vspace{-0.5em}
\noindent%
Scald \textit{granite-ware} coffee-pot. Wash egg, break, and beat slightly.
Dilute with one-half the cold water, add crushed shell, and mix with
coffee. Turn into coffee-pot, pour on boiling water, and stir
thoroughly. Place on front of range, and boil three minutes. If not
boiled, coffee is cloudy; if boiled too long, too much tannic acid is
developed. The spout of pot should be covered or stuffed with soft paper
to prevent escape of fragrant aroma. Stir and pour some in a cup to be
sure that spout is free from grounds. Return to coffee-pot and repeat.
Add remaining cold water, which perfects clearing. Cold water being
heavier than hot water sinks to the bottom, carrying grounds with it.
Place on back of range for ten minutes, where coffee will not boil.
Serve at once. If any is left over, drain from grounds, and reserve for
making of jelly or other dessert.

Egg-shells may be saved and used for clearing coffee. Three egg-shells
are sufficient to effect clearing where one cup of ground coffee is
used. The shell performs no office in clearing except for the albumen
which clings to it. Burnett's Crystal Coffee Settler, or salt fish-skin,
washed, dried, and cut in inch pieces, is used for same purpose.

Coffee made with an egg has a rich flavor which egg alone can give.
Where strict economy is necessary, if great care is taken, egg may be
omitted. Coffee so made should be served from range, as much motion
causes it to become roiled.

Tin is an undesirable material for a coffee-pot, as tannic acid acts on
such metal and is apt to form a poisonous compound.

When coffee and scalded milk are served in equal proportions, it is
called \textit{Café au lait}. Coffee served with whipped cream is called
\textit{Vienna Coffee}.

\textbf{To Make a Small Pot of Coffee.} Mix one cup ground coffee with one egg,
slightly beaten, and crushed shell. To one-third of this amount add
one-third cup cold water. Turn into a scalded coffee-pot, add one pint
boiling water, and boil three minutes. Let stand on back of range ten
minutes; serve. Keep remaining coffee and egg closely covered, in a cool
place, to use two successive mornings.

\textbf{To Make Coffee for One.} Allow two tablespoons ground coffee to one cup
cold water. Add coffee to cold water, cover closely, and let stand over
night. In the morning bring to a boiling-point. If carefully poured, a
clear cup of coffee may be served.



\needspace{15\baselineskip}
\subsection*{After-Dinner Coffee}

                     (\textit{Black Coffee, or Café Noir})

For after-dinner coffee use twice the quantity of coffee, or half the
amount of liquid, given in previous recipes. Filtered coffee is often
preferred where milk or cream is not used, as is always the case with
black coffee. Serve in after-dinner coffee cups, with or without cut
sugar.

Coffee retards gastric digestion; but where the stomach has been
overtaxed by a hearty meal, café noir may prove beneficial, so great are
its stimulating effects.



\needspace{15\baselineskip}
\section*{Kola}

The preparations on the market made from the kola-nut have much the same
effect upon the system as coffee and chocolate, inasmuch as they contain
caffeine and theobromine; they are also valuable for their diastase and
a milk-digesting ferment.



\needspace{15\baselineskip}
\section*{Cocoa And Chocolate}

The cacao-tree (\textit{Theobroma cacao}) is native to Mexico. Although
successfully cultivated between the twentieth parallels of latitude, its
industry is chiefly confined to Mexico, South America, and the West
Indies. Cocoa and chocolate are both prepared from seeds of the cocoa
bean. The bean pod is from seven to ten inches long, and three to four
and one-half inches in diameter. Each pod contains from twenty to forty
seeds, imbedded in mucilaginous material. Cocoa beans are dried previous
to importation. Like coffee, they need roasting to develop flavor. After
roasting, outer covering of bean is removed; this covering makes what is
known as \textit{cocoa shells}, which have little nutritive value. The beans
are broken and sold as \textit{cocoa nibs}.

The various preparations of cocoa on the market are made from the ground
cocoa nibs, from which, by means of hydraulic pressure, a large amount
of fat is expressed, leaving a solid cake. This in turn is pulverized
and mixed with sugar, and frequently a small amount of corn-starch or
arrowroot. To some preparations cinnamon or vanilla is added. Broma
contains both arrowroot and cinnamon.

Chocolate is made from cocoa nibs, but contains a much larger proportion
of fat than cocoa preparations. Bitter, sweet, or flavored chocolate is
always sold in cakes.

The fat obtained from cocoa bean is \textit{cocoa butter}, which gives cocoa
its principal nutrient.

Cocoa and chocolate differ from tea and coffee inasmuch as they contain
nutriment as well as stimulant. \textit{Theobromine}, the active principle, is
almost identical with theine and caffeine in its composition and
effects.

Many people who abstain from the use of tea and coffee find cocoa
indispensable. Not only is it valuable for its own nutriment, but for
the large amount of milk added to it. Cocoa may be well placed in the
dietary of a child after his third year, while chocolate should be
avoided as a beverage, but may be given as a confection. Invalids and
those of weak digestion can take cocoa where chocolate would prove too
rich.



\needspace{15\baselineskip}
\subsection*{Cocoa Shells}


\begin{itemize}
\setlength{\itemsep}{0pt}
\setlength{\parsep}{0pt}
\item 1 cup cocoa shells
\item 6 cups boiling water
\end{itemize}

\vspace{-0.5em}
\noindent%
Boil shells and water three hours; as water boils away it will be
necessary to add more. Strain, and serve with milk and sugar. By adding
one-third cup cocoa nibs, a much more satisfactory drink is obtained.



\needspace{15\baselineskip}
\subsection*{Cracked Cocoa}


\begin{itemize}
\setlength{\itemsep}{0pt}
\setlength{\parsep}{0pt}
\item 1/2 cup cracked cocoa
\item 3 pints boiling water
\end{itemize}

\vspace{-0.5em}
\noindent%
Boil cracked cocoa and water two hours. Strain, and serve with milk and
sugar. If cocoa is pounded in a mortar and soaked over night in three
pints water, it will require but one hour's boiling.



\needspace{15\baselineskip}
\subsection*{Breakfast Cocoa}


\begin{itemize}
\setlength{\itemsep}{0pt}
\setlength{\parsep}{0pt}
\item 1 1/2 tablespoons prepared cocoa
\item 2 tablespoons sugar
\item 2 cups boiling water
\item 2 cups milk
\item Few grains salt
\end{itemize}

\vspace{-0.5em}
\noindent%
Scald milk. Mix cocoa, sugar, and salt, dilute with one-half cup boiling
water to make smooth paste, add remaining water, and boil one minute;
turn into scalded milk and beat two minutes, using Dover egg-beater,
when froth will form, preventing scum, which is so unsightly; this is
known as \textit{milling}.



\needspace{15\baselineskip}
\subsection*{Reception Cocoa}


\begin{itemize}
\setlength{\itemsep}{0pt}
\setlength{\parsep}{0pt}
\item 3 tablespoons cocoa
\item 1/4 cup sugar
\item A few grains salt
\item 4 cups milk
\item 3/4 cup boiling water
\end{itemize}

\vspace{-0.5em}
\noindent%
Scald milk. Mix cocoa, sugar, and salt, adding enough boiling water to
make a smooth paste; add remaining water and boil one minute; pour into
scalded milk. Beat two minutes, using Dover egg-beater.



\needspace{15\baselineskip}
\subsection*{Brandy Cocoa}


\begin{itemize}
\setlength{\itemsep}{0pt}
\setlength{\parsep}{0pt}
\item 3 tablespoons cocoa
\item 1/4 cup sugar
\item 1 1/2 cups boiling water
\item 4 cups milk
\item 3 teaspoons cooking brandy
\end{itemize}

\vspace{-0.5em}
\noindent%
Prepare as Reception Cocoa, and add brandy before milling.



\needspace{15\baselineskip}
\subsection*{Chocolate I}


\begin{itemize}
\setlength{\itemsep}{0pt}
\setlength{\parsep}{0pt}
\item 1 1/2 squares Baker's chocolate
\item 1/4 cup sugar
\item Few grains salt
\item 1 cup boiling water
\item 3 cups milk
\end{itemize}

\vspace{-0.5em}
\noindent%
Scald milk. Melt chocolate in small saucepan placed over hot water, add
sugar, salt, and gradually boiling water; when smooth, place on range
and boil one minute; add to scalded milk, mill, and serve in chocolate
cups with whipped cream. One and one-half ounces vanilla chocolate may
be substitute for Baker's chocolate; being sweetened, less sugar is
required.



\needspace{15\baselineskip}
\subsection*{Chocolate II}

Prepare same as Chocolate I., substituting one can evaporated cream or
condensed milk diluted with two cups boiling water in place of three
cups milk. If sweetened condensed milk is used, omit sugar.



\needspace{15\baselineskip}
\subsection*{Chocolate III}


\begin{itemize}
\setlength{\itemsep}{0pt}
\setlength{\parsep}{0pt}
\item 2 ozs. sweetened chocolate
\item 4 cups milk
\item Few grains salt
\item Whipped cream
\end{itemize}

\vspace{-0.5em}
\noindent%
Scald milk, add chocolate, and stir until chocolate is melted. Bring to
boiling-point, mill, and serve in chocolate cups with whipped cream
sweetened and flavored.



\needspace{15\baselineskip}
\section*{Fruit Beverages}


\needspace{15\baselineskip}
\subsection*{Lemonade}


\begin{itemize}
\setlength{\itemsep}{0pt}
\setlength{\parsep}{0pt}
\item 1 cup sugar
\item 1/3 cup lemon juice
\item 1 pint water
\end{itemize}

\vspace{-0.5em}
\noindent%
Make syrup by boiling sugar and water twelve minutes; add fruit juice,
cool, and dilute with ice-water to suit individual tastes. Lemon syrup
may be bottled and kept on hand to use as needed.



\needspace{15\baselineskip}
\subsection*{Pineapple Lemonade}


\begin{itemize}
\setlength{\itemsep}{0pt}
\setlength{\parsep}{0pt}
\item 1 pint water
\item 1 quart ice-water
\item 1 cup sugar
\item 1 can grated pineapple
\item Juice 3 lemons
\end{itemize}

\vspace{-0.5em}
\noindent%
Make syrup by boiling water and sugar ten minutes; add pineapple and
lemon juice, cool, strain, and add ice-water.



\needspace{15\baselineskip}
\subsection*{Orangeade}

Make syrup as for Lemonade. Sweeten orange juice with syrup, and dilute
by pouring over crushed ice.



\needspace{15\baselineskip}
\subsection*{Mint Julep}


\begin{minipage}{1.0\textwidth}
{\setlength{\multicolsep}{0pt}\setlength{\columnsep}{2em}\raggedcolumns%
\begin{multicols}{2}
\begin{itemize}
\setlength{\itemsep}{0pt}
\setlength{\parsep}{0pt}
\item 1 quart water
\item 2 cups sugar
\item 1 pint claret wine
\item 1 cup strawberry juice
\item 1 cup orange juice
\item Juice 8 lemons
\item 1 1/2 cups boiling water
\item 12 sprigs fresh mint
\end{itemize}
\end{multicols}}
\end{minipage}

\vspace{0.3em}
\noindent%
Make syrup by boiling quart of water and sugar twenty minutes. Separate
mint in pieces, add to the boiling water, cover, and let stand in warm
place five minutes, strain, and add to syrup; add fruit juices, and
cool. Pour into punch-bowl, add claret, and chill with a large piece of
ice; dilute with water. Garnish with fresh mint leaves and whole
strawberries.



\needspace{15\baselineskip}
\subsection*{Claret Punch}


\begin{minipage}{1.0\textwidth}
{\setlength{\multicolsep}{0pt}\setlength{\columnsep}{2em}\raggedcolumns%
\begin{multicols}{2}
\begin{itemize}
\setlength{\itemsep}{0pt}
\setlength{\parsep}{0pt}
\item 1 quart cold water
\item 1/2 cup raisins
\item 2 cups sugar
\item 2 inch piece stick cinnamon
\item Few shavings lemon rind
\item 1 1/3 cups orange juice
\item 1/3 cup lemon juice
\item 1 pint claret wine
\end{itemize}
\end{multicols}}
\end{minipage}

\vspace{0.3em}
\noindent%
Put raisins in cold water, bring slowly to boiling-point, and boil
twenty minutes; strain, add sugar, cinnamon, lemon rind, and boil five
minutes. Add fruit juice, cool, strain, pour in claret, and dilute with
ice-water.



\needspace{15\baselineskip}
\subsection*{Fruit Punch I}


\begin{itemize}
\setlength{\itemsep}{0pt}
\setlength{\parsep}{0pt}
\item 1 quart cold water
\item 2 cups sugar
\item 1/2 cup lemon juice
\item 2 cups chopped pineapple
\item 1 cup orange juice
\end{itemize}

\vspace{-0.5em}
\noindent%
Boil water, sugar, and pineapple twenty minutes; add fruit juice, cool,
strain, and dilute with ice-water.



\needspace{15\baselineskip}
\subsection*{Fruit Punch II}


\begin{minipage}{1.0\textwidth}
{\setlength{\multicolsep}{0pt}\setlength{\columnsep}{2em}\raggedcolumns%
\begin{multicols}{2}
\begin{itemize}
\setlength{\itemsep}{0pt}
\setlength{\parsep}{0pt}
\item 1 cup water
\item 2 cups sugar
\item 1 cup tea infusion
\item 1 quart Apollinaris
\item 2 cups strawberry syrup
\item Juice 5 lemons
\item Juice 5 oranges
\item 1 can grated pineapple
\item 1 cup Maraschino cherries
\end{itemize}
\end{multicols}}
\end{minipage}

\vspace{0.3em}
\noindent%
Make syrup by boiling water and sugar ten minutes; add tea, strawberry
syrup, lemon juice, orange juice, and pineapple; let stand thirty
minutes, strain, and add ice-water to make one and one-half gallons of
liquid. Add cherries and Apollinaris. Serve in punch-bowl, with large
piece of ice. This quantity will serve fifty.



\needspace{15\baselineskip}
\subsection*{Fruit Punch III}


\begin{minipage}{1.0\textwidth}
{\setlength{\multicolsep}{0pt}\setlength{\columnsep}{2em}\raggedcolumns%
\begin{multicols}{2}
\begin{itemize}
\setlength{\itemsep}{0pt}
\setlength{\parsep}{0pt}
\item 1 cup sugar
\item 1 cup hot tea infusion
\item 3/4 cup orange juice
\item 1/3 cup lemon juice
\item 1 pint ginger ale
\item 1 pint Apollinaris
\item Few slices orange
\end{itemize}
\end{multicols}}
\end{minipage}

\vspace{0.3em}
\noindent%
Pour tea over sugar, and as soon as sugar is dissolved add fruit juices.
Strain into punch-bowl over a large piece of ice, and just before
serving add ale, Apollinaris, and slices of orange.



\needspace{15\baselineskip}
\subsection*{Fruit Punch IV}


\begin{minipage}{1.0\textwidth}
{\setlength{\multicolsep}{0pt}\setlength{\columnsep}{2em}\raggedcolumns%
\begin{multicols}{2}
\begin{itemize}
\setlength{\itemsep}{0pt}
\setlength{\parsep}{0pt}
\item 9 oranges 
\item 6 lemons 
\item 1 cup grated pineapple 
\item 1 cup raspberry syrup 
\item 1 1/2 cups tea infusion 
\item 1 1/4 cups sugar 
\item 1 cup hot water 
\item 1 quart Apollinaris
\end{itemize}
\end{multicols}}
\end{minipage}

\vspace{0.3em}
\noindent%
Mix juice of oranges and lemons with pineapple, raspberry syrup, and
tea; then add a syrup made by boiling sugar and water fifteen minutes.
Turn in punch-bowl over a large piece of ice. Chill thoroughly, and just
before serving add Apollinaris.



\needspace{15\baselineskip}
\subsection*{Ginger Punch}


\begin{itemize}
\setlength{\itemsep}{0pt}
\setlength{\parsep}{0pt}
\item 1 quart cold water
\item 1 cup sugar
\item 1/2 lb. Canton ginger
\item 1/2 cup orange juice
\item 1/2 cup lemon juice
\end{itemize}

\vspace{-0.5em}
\noindent%
Chop ginger, add to water and sugar, boil fifteen minutes; add fruit
juice, cool, strain, and dilute with crushed ice.



\needspace{15\baselineskip}
\subsection*{Champagne Punch}


\begin{minipage}{1.0\textwidth}
{\setlength{\multicolsep}{0pt}\setlength{\columnsep}{2em}\raggedcolumns%
\begin{multicols}{2}
\begin{itemize}
\setlength{\itemsep}{0pt}
\setlength{\parsep}{0pt}
\item 1 cup water
\item 2 cups sugar
\item 1 quart California champagne
\item 4 tablespoons brandy
\item 2 tablespoons Medford rum
\item 2 tablespoons Orange Curaçoa
\item Juice 2 lemons
\item 2 cups tea infusion
\item Ice
\item 1 quart soda water
\end{itemize}
\end{multicols}}
\end{minipage}

\vspace{0.3em}
\noindent%
Make a syrup by boiling water and sugar ten minutes. Mix champagne,
brandy, rum, Curaçoa, lemon juice, and tea infusion. Sweeten to taste
with syrup and pour into punch-bowl over a large piece of ice. Just
before serving add soda water.



\needspace{15\baselineskip}
\subsection*{Club Punch}


\begin{minipage}{1.0\textwidth}
{\setlength{\multicolsep}{0pt}\setlength{\columnsep}{2em}\raggedcolumns%
\begin{multicols}{2}
\begin{itemize}
\setlength{\itemsep}{0pt}
\setlength{\parsep}{0pt}
\item 1 cup water
\item 2 cups sugar
\item 1 quart Burgundy
\item 1 cup rum
\item 1/3 cup brandy
\item 1/3 cup Benedictine
\item 1 quart Vichy
\item 3 sliced oranges
\item 1/2 can pineapple
\item Juice 2 lemons
\item 1 cup tea infusion
\item 2 cups ice
\end{itemize}
\end{multicols}}
\end{minipage}

\vspace{0.3em}
\noindent%
Make a syrup by boiling water and sugar ten minutes. Mix remaining
ingredients, \textit{except ice}, sweeten to taste with syrup, and pour into
punch-bowl over a large piece of ice.



\needspace{15\baselineskip}
\subsection*{Unfermented Grape Juice}


\begin{itemize}
\setlength{\itemsep}{0pt}
\setlength{\parsep}{0pt}
\item 10 lbs. grapes
\item 1 cup water
\item 3 lbs. sugar
\end{itemize}

\vspace{-0.5em}
\noindent%
Put grapes and water in granite stewpan. Heat until stones and pulp
separate; then strain through jelly bag, add sugar, heat to
boiling-point, and bottle. This will make one gallon. When served, it
should be diluted one-half with water.







\needspace{15\baselineskip}
\subsection*{Claret Cup}


\begin{minipage}{1.0\textwidth}
{\setlength{\multicolsep}{0pt}\setlength{\columnsep}{2em}\raggedcolumns%
\begin{multicols}{2}
\begin{itemize}
\setlength{\itemsep}{0pt}
\setlength{\parsep}{0pt}
\item 1 quart claret wine
\item 1/2 cup Curaçoa
\item 1 quart Apollinaris
\item 1/3 cup orange juice
\item 2 tablespoons brandy
\item Sugar
\item Mint leaves
\item Cucumber rind
\item 12 strawberries
\end{itemize}
\end{multicols}}
\end{minipage}

\vspace{0.3em}
\noindent%
Mix ingredients, except Apollinaris, using enough sugar to sweeten to
taste. Stand on ice to chill, and add chilled Apollinaris just before
serving.



\needspace{15\baselineskip}
\subsection*{Sauterne Cup}


\begin{minipage}{1.0\textwidth}
{\setlength{\multicolsep}{0pt}\setlength{\columnsep}{2em}\raggedcolumns%
\begin{multicols}{2}
\begin{itemize}
\setlength{\itemsep}{0pt}
\setlength{\parsep}{0pt}
\item 1 quart soda water
\item 2 cups Sauterne wine
\item Rind 1/2 orange
\item Rind 1/2 lemon
\item 2 tablespoons Orange Curaçoa
\item 1/2 cup sugar (scant)
\item Mint leaves
\item Few slices orange
\item 12 strawberries
\end{itemize}
\end{multicols}}
\end{minipage}

\vspace{0.3em}
\noindent%
Add Curaçoa to rind of fruit and sugar; cover, and let stand two hours.
Add Sauterne, strain, and stand on ice to chill. Add chilled soda water,
mint leaves, slices of orange, and strawberries. The success of cups
depends upon the addition of charged water just before serving.





\chapter{Bread And Bread Making}



Bread is the most important article of food, and history tells of its
use thousands of years before the Christian era. Many processes have
been employed in making and baking; and as a result, from the first flat
cake has come the perfect loaf. The study of bread making is of no
slight importance, and deserves more attention than it receives.

Considering its great value, it seems unnecessary and wrong to find poor
bread on the table; and would that our standard might be raised as high
as that of our friends across the water! Who does not appreciate the
loaf produced by the French baker, who has worked months to learn the
art of bread making?

Bread is made from flour of wheat, or other cereals, by addition of
water, salt, and a ferment. Wheat flour is best adapted for bread
making, as it contains gluten in the right proportion to make the spongy
loaf. But for its slight deficiency in fat, wheat bread is a perfect
food; hence arose the custom of spreading it with butter. It should be
remembered, in speaking of wheat bread as perfect food, that it must be
made of flour rich in gluten. Next to wheat flour ranks rye in
importance for bread making; but it is best used in combination with
wheat, for alone it makes heavy, sticky, moist bread. Corn also needs to
be used in combination with wheat for bread making, for if used alone
the bread will be crumbly.

The miller, in order to produce flour which will make the white loaf (so
sightly to many), in the process of grinding wheat has been forced to
remove the inner bran coats, so rich in mineral matter, and much of the
gluten intimately connected with them.

To understand better the details of bread making, wheat, from which
bread is principally made, should be considered.

A grain of wheat consists of (1) an outer covering or husk, which is
always removed before milling; (2) bran coats, which contain mineral
matter; (3) gluten, the protein matter and fat; and (4) starch, the
centre and largest part of the grain. Wheat is distinguished as \textit{white}
and \textit{soft}, or \textit{red} and \textit{hard}. The former is known as \textit{winter wheat},
having been sown in the fall, and living through the winter; the latter
is known as \textit{spring wheat}, having been sown in the spring. From winter
wheat, pastry flour, sometimes called St. Louis, is made; from spring
wheat, bread flour, also called Haxall. St. Louis flour takes its name
from the old process of grinding; Haxall, from the name of the inventor
of the new process. All flours are now milled by the same process. For
difference in composition of wheat flours, consult table in Chapter VI
on Cereals.

Wheat is milled for converting into flour by processes producing
essentially the same results, all requiring cleansing, grinding, and
bolting. Entire wheat flour has only the outer husk removed, the
remainder of the kernel being finely ground. \textit{Graham flour}, confounded
with entire wheat, is too often found to be an inferior flour, mixed
with coarse bran.

Grinding is accomplished by one of four systems: (1) low milling; (2)
Hungarian system, or high milling; (3) roller-milling; and (4) by a
machine known as disintegrator.

\textbf{In low milling process}, grooved stones are employed for grinding. The
stones are enclosed in a metal case, and provision is made within case
for passage of air to prevent wheat from becoming overheated. The lower
stone being permanently fixed, the upper stone being so balanced above
it that grooves may exactly correspond, when upper stone rotates, sharp
edges of grooves meet each other, and operate like a pair of scissors.
By this process flour is made ready for bolting by one grinding.

\textbf{In high milling process}, grooved stones are employed, but are kept so
far apart that at first the wheat is only bruised, and a series of
grindings and siftings is necessary. This process is applicable only to
the hardest wheats, and is partially supplanted by roller-milling.

\textbf{In roller-milling}, wheat is subjected to action of a pair of steel or
chilled-iron horizontal rollers, having toothed surfaces. They revolve
in opposite directions, at different rates of speed, and have a cutting
action.

Porcelain rollers, with rough surfaces, are sometimes employed. In this
system, grinding is accomplished by cutting rather than crushing.

“The \textbf{disintegrator} consists of a pair of circular metal disks, set
face to face, studded with circles of projecting bars so arranged that
circles of bars on one disk alternate with those of the other. The disks
are mounted on the same centre, and so closely set to one another that
projecting bars of one disk come quite close to plane surface of the
other. They are enclosed within an external casing. The disks are caused
to rotate in opposite directions with great rapidity, and the grain is
almost instantaneously reduced to a powder.”

After grinding comes bolting, by which process the different grades of
flour are obtained. The ground wheat is placed in octagonal cylinders
(covered with silk or linen bolting-cloth of different degrees of
fineness), which are allowed to rotate, thus forcing the wheat through.
The flour from first siftings contains the largest percentage of gluten.

Flour is branded under different names to suit manufacturer or dealer.
In consequence, the same wheat, milled by the same process, makes flour
which is sold under different names.

In buying flour, whether bread or pastry, select the best kept by your
grocer. Some of the well-known brands of bread flour are King Arthur,
Swan's Down, Bridal Veil, Columbia, Washburn's Extra, and Pillbury's
Best; of pastry, Best St. Louis. Bread flour should be used in all cases
where yeast is called for, with few exceptions; in other cases, pastry
flour. The difference between bread and pastry flour may be readily
determined. Take bread flour in the hand, close hand tightly, then open,
and flour will not keep in shape; if allowed to pass through fingers it
will feel slightly granular. Take pastry flour in the hand, close hand
tightly, open, and flour will be in shape, having impression of the
lines of the hand, and feeling soft and velvety to touch. Flour should
always be sifted before measuring.

\textbf{Entire wheat flour} differs from ordinary flour inasmuch as it contains
all the gluten found in wheat, the outer husk of kernels only being
removed, the remainder ground to different degrees of fineness and left
unbolted. Such flours are sold by the different health food companies,
who have agencies in the large cities. Franklin Mills, Old Grist Mill,
and Health Food flours are included in this class.

\textbf{Gluten}, the protein of wheat, is a gray, tough, elastic substance,
insoluble in water. On account of its great power of expansion, it holds
the gas developed in bread dough by fermentation, which otherwise would
escape.



\needspace{15\baselineskip}
\section*{Yeast}

Yeast is a microscopic plant of fungous growth, and is the lowest form
of vegetable life. It consists of spores, or germs, found floating in
air, and belongs to a family of which there are many species. These
spores grow by budding and division, and multiply very rapidly under
favorable conditions, and produce fermentation.

\textbf{Fermentation} is the process by which, under influence of air, warmth,
moisture, and some ferment, sugar (or dextrose, starch converted into
sugar) is changed into alcohol (C\textsubscript{2}H\textsubscript{5}HO) and carbon dioxide
(CO\textsubscript{2}). The product of all fermentation is the same. Three kinds are
considered,--alcoholic, acetic, and lactic. Where bread dough is allowed
to ferment by addition of yeast, the fermentation is \textit{alcoholic}; where
alcoholic fermentation continues too long, \textit{acetic} fermentation sets
in, which is a continuation of alcoholic. \textit{Lactic} fermentation is
fermentation which takes place when milk sours.

Liquid, dry, or compressed yeast may be used for raising bread. The
compressed yeast cakes done up in tinfoil have long proved satisfactory,
and are now almost universally used, having replaced the home-made
liquid yeast. Never use a yeast cake unless perfectly fresh, which may
be determined by its light color and absence of dark streaks.

The \textit{yeast plant} is killed at 212deg F.; life is suspended, but not
entirely destroyed, 32deg F. The temperature best suited for its growth is
from 65deg to 68deg F. The most favorable conditions for the growth of yeast
are a warm, moist, sweet, nitrogenous soil. These must be especially
considered in bread making.



\needspace{15\baselineskip}
\section*{Bread Making}

\textbf{Fermented bread} is made by mixing to a dough, flour, with a definite
quantity of water, milk, or water and milk, salt, and a ferment. Sugar
is usually added to hasten fermentation. Dough is then kneaded that the
ingredients may be thoroughly incorporated, covered, and allowed to rise
in a temperature of 68deg F., until dough has doubled its bulk. This
change has been caused by action of the ferment, which attacks some of
the starch in flour, and changes it to sugar, and sugar in turn to
alcohol and carbon dioxide, thus lightening the whole mass. Dough is
then kneaded a second time to break bubbles and distribute evenly the
carbon dioxide. It is shaped in loaves, put in greased bread pans (they
being half filled), covered, allowed to rise in temperature same as for
first rising, to double its bulk. If risen too long, it will be full of
large holes; if not risen long enough, it will be heavy and soggy. If
pans containing loaves are put in too hot a place while rising, a heavy
streak will be found near bottom of loaf.

\textbf{How to Shape Loaves and Biscuits.} To shape bread dough in loaves,
divide dough in parts, each part large enough for a loaf, knead until
smooth, and if possible avoid seams in under part of loaf. If baked in
brick pan, place two loaves in one pan, brushed between with a little
melted butter. If baked in long shallow pan, when well kneaded, roll
with both hands to lengthen, care being taken that it is smooth and of
uniform thickness. Where long loaves are baked on sheets, shape and roll
loosely in a towel sprinkled with corn meal for last rising.

To shape bread dough in biscuits, pull or cut off as many small pieces
(having them of uniform size) as there are to be biscuits. Flour palms
of hands slightly; take up each piece and shape separately, lifting,
with thumb and first two fingers of right hand, and placing in palm of
left hand, constantly moving dough round and round, while folding
towards the centre; when smooth, turn it over and roll between palms of
hands. Place in greased pans near together, brushed between with a
little melted butter, which will cause biscuits to separate easily after
baking. For finger rolls, shape biscuits and roll with one hand on part
of board where there is no flour, until of desired length, care being
taken to make smooth, of uniform size, and round at ends.

Biscuits may be shaped in a great variety of ways, but they should
always be small. Large biscuits, though equally good, never tempt one by
their daintiness.

Bread is often brushed over with milk before baking, to make a darker
crust.

Where bread is allowed to rise over night, a small piece of yeast cake
must be used; one-fourth yeast cake to one pint liquid is sufficient,
one-third yeast cake to one quart liquid. Bread mixed and baked during
the day requires a larger quantity of yeast; one yeast cake, or
sometimes even more, to one pint of liquid. Bread dough mixed with a
large quantity of yeast should be watched during rising, and cut down as
soon as mixture doubles its bulk. If proper care is taken, the bread
will be found most satisfactory, having neither “yeasty” nor sour taste.

Fermented bread was formerly raised by means of leaven.



\needspace{15\baselineskip}
\section*{Baking Of Bread}

Bread is baked: (1) To kill ferment, (2) to make soluble the starch, (3)
to drive off alcohol and carbon dioxide, and (4) to form brown crust of
pleasant flavor. Bread should be baked in a hot oven. If the oven be too
hot the crust will brown quickly before the heat has reached the centre,
and prevent further rising; loaf should continue rising for first
fifteen minutes of baking, when it should begin to brown, and continue
browning for the next twenty minutes. The last fifteen minutes it should
finish baking, when the heat may be reduced. When bread is done, it will
not cling to sides of pan, and may be easily removed. Biscuits require
more heat than loaf bread, should continue rising the first five
minutes, and begin to brown in eight minutes. Experience is the best
guide for testing temperature of oven. Various oven thermometers have
been made, but none have proved practical. Bread may be brushed over
with melted butter, three minutes before removal from oven, if a more
tender crust is desired.



\needspace{15\baselineskip}
\section*{Care Of Bread After Baking}

Remove loaves at once from pans, and place side down on a wire bread or
cake cooler. If a crisp crust is desired, allow bread to cool without
covering; if soft crust, cover with a towel during cooling. When cool,
put in tin box or stone jar, and cover closely.

Never keep bread wrapped in cloth, as the cloth will absorb moisture and
transmit an unpleasant taste to bread. Bread tins or jars should be
washed and scalded twice a week in winter, and every other day in
summer; otherwise bread is apt to mould. As there are so many ways of
using small and stale pieces of bread, care should be taken that none is
wasted.

\textbf{Unfermented bread} is raised without a ferment, the carbon dioxide
being produced by the use of soda (alkaline salt) and an acid. Soda,
employed in combination with cream of tartar, for raising mixtures, in
proportion of one-third soda to two-thirds cream of tartar, was formerly
used to a great extent, but has been generally superseded by baking
powder.

\textbf{Soda bicarbonate} (NaHCO\textsubscript{3}) is manufactured from sodium chloride
(NaCl), common salt or cryolite.

\textbf{Baking powder} is composed of soda and cream of tartar in definite,
correct proportions, mixed with small quantity of dry material (flour or
corn-starch) to keep action from taking place. If found to contain alum
or ammonia, it is impure. In using baking powder, allow two teaspoons
baking powder to each cup of flour, when eggs are not used; to egg
mixtures allow one and one-half teaspoons baking powder. When a recipe
calls for soda and cream of tartar, in substituting baking powder use
double amount of cream of tartar given.

These rules apply to the various soda and cream of tartar baking powders
on the market. Horsford's Baking Powder, the only mineral one, requires
one-third less than others.

Soda and cream of tartar, or baking powder mixtures, are made light by
liberation of gas in mixture; the gas in soda is set free by the acid in
cream of tartar; in order to accomplish this, moisture and heat are both
required. As soon as moisture is added to baking powder mixtures, the
gas will begin to escape; hence the necessity of baking as soon as
possible. If baking powder only is used for raising, put mixture to be
cooked in a hot oven.

\textbf{Cream of tartar} (HKC\textsubscript{4}O\textsubscript{6}H\textsubscript{4}) is obtained from argols found
adhering to bottom and sides of wine casks, which are ninety per cent
cream of tartar. The argols are ground and dissolved in boiling water,
coloring matter removed by filtering through animal charcoal, and by a
process of recrystallization the cream of tartar of commerce is
obtained.

The acid found in molasses, sour milk, and lemon juice will liberate gas
in soda, but the action is much quicker than when cream of tartar is
used.

Fermented and unfermented breads are raised to be made light and porous,
that they may be easily acted upon by the digestive ferments. Some
mixtures are made light by beating sufficiently to enclose a large
amount of air, and when baked in a hot oven air is forced to expand.

\textbf{Aerated bread} is made light by carbon dioxide forced into dough under
pressure. The carbon dioxide is generated from sulphuric acid and lime.
Aerated bread is of close texture, and has a flavor peculiar to itself.
It is a product of the baker's skill, but has found little favor except
in few localities.



\needspace{15\baselineskip}
\section*{Water Bread}


\begin{minipage}{1.0\textwidth}
{\setlength{\multicolsep}{0pt}\setlength{\columnsep}{2em}\raggedcolumns%
\begin{multicols}{2}
\begin{itemize}
\setlength{\itemsep}{0pt}
\setlength{\parsep}{0pt}
\item 2 cups boiling water
\item 1 tablespoon butter
\item 1 tablespoon lard
\item 1 tablespoon sugar
\item 1 1/2 teaspoons salt
\item 1/4  yeast cake dissolved in
\item 1/4  cup lukewarm water
\item 6 cups sifted flour
\end{itemize}
\end{multicols}}
\end{minipage}

\vspace{0.3em}
\noindent%
Put butter, lard, sugar, and salt in bread raiser, or large bowl without
a lip; pour on boiling water; when lukewarm, add dissolved yeast cake
and five cups of flour; then stir until thoroughly mixed, using a knife
or mixing-spoon. Add remaining flour, mix, and turn on a floured board,
leaving a clean bowl; knead until mixture is smooth, elastic to touch,
and bubbles may be seen under the surface. Some practice is required to
knead quickly, but the motion once acquired will never be forgotten.
Return to bowl, cover with a clean cloth kept for the purpose, and board
or tin cover; let rise over night in temperature of 65deg F. In morning
cut down: this is accomplished by cutting through and turning over dough
several times with a case knife, and checks fermentation for a short
time; dough may be again raised, and recut down if it is not convenient
to shape into loaves or biscuits after first cutting. When properly
cared for, bread need never sour. Toss on board slightly floured, knead,
shape into loaves or biscuits, place in greased pans, having pans nearly
half full. Cover, let rise again to double its bulk, and bake in hot
oven. (See Baking of Bread and Time Table for Baking.) This recipe will
make a double loaf of bread and pan of biscuit. Cottolene, coto suet, or
beef drippings may be used for shortening, one-third less being
required. Bread shortened with butter has a good flavor, but is not as
white as when lard is used.



\needspace{15\baselineskip}
\section*{Milk And Water Bread}


\begin{minipage}{1.0\textwidth}
{\setlength{\multicolsep}{0pt}\setlength{\columnsep}{2em}\raggedcolumns%
\begin{multicols}{2}
\begin{itemize}
\setlength{\itemsep}{0pt}
\setlength{\parsep}{0pt}
\item 1 cup scalded milk
\item 1 cup boiling water
\item 1 tablespoon lard
\item 1 tablespoon butter
\item 1 1/2 teaspoon salt
\item 1 yeast cake dissolved in
\item 1/4 cup lukewarm water
\item 6 cups sifted flour, or one cup white flour and enough entire wheat flour to knead
\end{itemize}
\end{multicols}}
\end{minipage}

\vspace{0.3em}
\noindent%
Prepare and bake as Water Bread. When entire wheat flour is used add
three tablespoons molasses. Bread may be mixed, raised, and baked in
five hours, by using one yeast cake. Bread made in this way has proved
most satisfactory. It is usually mixed in the morning, and the cook is
able to watch the dough while rising and keep it at uniform temperature.
It is often desirable to place bowl containing dough in pan of water,
keeping water at uniform temperature of from 95deg to 100deg F. Cooks who
have not proved themselves satisfactory bread makers are successful when
employing this method.



\needspace{15\baselineskip}
\section*{Entire Wheat Bread}


\begin{minipage}{1.0\textwidth}
{\setlength{\multicolsep}{0pt}\setlength{\columnsep}{2em}\raggedcolumns%
\begin{multicols}{2}
\begin{itemize}
\setlength{\itemsep}{0pt}
\setlength{\parsep}{0pt}
\item 2 cups scalded milk
\item 1/4 cup sugar or
\item 1/3 cup molasses
\item 1 teaspoon salt
\item 1 yeast cake dissolved in
\item 1/4 cup lukewarm water
\item 4 2/3 cups coarse entire wheat flour
\end{itemize}
\end{multicols}}
\end{minipage}

\vspace{0.3em}
\noindent%
Add sweetening and salt to milk; cool, and when lukewarm add dissolved
yeast cake and flour; beat well, cover, and let rise to double its bulk.
Again beat, and turn into greased bread pans, having pans one-half full;
let rise, and bake. Entire Wheat Bread should not quite double its bulk
during last rising. This mixture may be baked in gem pans.



\needspace{15\baselineskip}
\section*{German Caraway Bread}

Follow recipe for Milk and Water Bread (see p. 54), using rye flour in
place of entire wheat flour, and one tablespoon sugar for sweetening.
After first rising while kneading add one-third tablespoon caraway seed.
Shape, let rise again, and bake in a loaf.



\needspace{15\baselineskip}
\section*{Entire Wheat And White Flour Bread}

Use same ingredients as for Entire Wheat Bread, with exception of flour.
For flour use three and one-fourth cups entire wheat and two and
three-fourths cups white flour. The dough should be slightly kneaded,
and if handled quickly will not stick to board. Loaves and biscuits
should be shaped with hands instead of pouring into pans, as in Entire
Wheat Bread.



\needspace{15\baselineskip}
\section*{Graham Bread}


\begin{minipage}{1.0\textwidth}
{\setlength{\multicolsep}{0pt}\setlength{\columnsep}{2em}\raggedcolumns%
\begin{multicols}{2}
\begin{itemize}
\setlength{\itemsep}{0pt}
\setlength{\parsep}{0pt}
\item 2 1/2 cups hot liquid (water, or milk and water)
\item 1/3 cup molasses
\item 1 1/2 teaspoons salt
\item 1/4 yeast cake dissolved in
\item 1/4 cup lukewarm water
\item 3 cups flour
\item 3 cups Graham flour
\end{itemize}
\end{multicols}}
\end{minipage}

\vspace{0.3em}
\noindent%
Prepare and bake as Entire Wheat Bread. The bran remaining in sieve
after sifting Graham flour should be discarded.



\needspace{15\baselineskip}
\section*{Third Bread}


\begin{minipage}{1.0\textwidth}
{\setlength{\multicolsep}{0pt}\setlength{\columnsep}{2em}\raggedcolumns%
\begin{multicols}{2}
\begin{itemize}
\setlength{\itemsep}{0pt}
\setlength{\parsep}{0pt}
\item 2 cups lukewarm water
\item 1 yeast cake
\item 1/2 tablespoon salt
\item 1/2 cup molasses
\item 1 cup rye flour
\item 1 cup granulated corn meal
\item 3 cups flour
\end{itemize}
\end{multicols}}
\end{minipage}

\vspace{0.3em}
\noindent%
Dissolve yeast cake in water, add remaining ingredients, and mix
thoroughly. Let rise, shape, let rise again, and bake as Entire Wheat
Bread.



\needspace{15\baselineskip}
\section*{Rolled Oats Bread}


\begin{minipage}{1.0\textwidth}
{\setlength{\multicolsep}{0pt}\setlength{\columnsep}{2em}\raggedcolumns%
\begin{multicols}{2}
\begin{itemize}
\setlength{\itemsep}{0pt}
\setlength{\parsep}{0pt}
\item 2 cups boiling water
\item 1/2 cup molasses
\item 1/2 tablespoon salt
\item 1 tablespoon butter
\item 1/2 yeast cake dissolved in
\item 1/2 cup lukewarm water
\item 1 cup Rolled Oats
\item 4 1/2 cups flour
\end{itemize}
\end{multicols}}
\end{minipage}

\vspace{0.3em}
\noindent%
Add boiling water to oats and let stand one hour; add molasses, salt,
butter, dissolved yeast cake, and flour; let rise, beat thoroughly, turn
into buttered bread pans, let rise again, and bake. By using one-half
cup less flour, the dough is better suited for biscuits, but, being
soft, is difficult to handle. To make shaping of biscuits easy, take up
mixture by spoonfuls, drop into plate of flour, and have palms of hands
well covered with flour before attempting to shape.



\needspace{15\baselineskip}
\section*{Rye Biscuit}


\begin{minipage}{1.0\textwidth}
{\setlength{\multicolsep}{0pt}\setlength{\columnsep}{2em}\raggedcolumns%
\begin{multicols}{2}
\begin{itemize}
\setlength{\itemsep}{0pt}
\setlength{\parsep}{0pt}
\item 1 cup boiling water
\item 1 cup rye flakes
\item 2 tablespoons butter
\item 1/3 cup molasses
\item 1 1/2 teaspoons salt
\item 1 yeast cake dissolved in
\item 1 cup lukewarm water
\item Flour
\end{itemize}
\end{multicols}}
\end{minipage}

\vspace{0.3em}
\noindent%
Make same as Rolled Oats Bread.



\needspace{15\baselineskip}
\section*{Rye Bread}


\begin{minipage}{1.0\textwidth}
{\setlength{\multicolsep}{0pt}\setlength{\columnsep}{2em}\raggedcolumns%
\begin{multicols}{2}
\begin{itemize}
\setlength{\itemsep}{0pt}
\setlength{\parsep}{0pt}
\item 1 cup scalded milk
\item 1 cup boiling water
\item 1 tablespoon lard
\item 1 tablespoon butter
\item 1/3 cup brown sugar
\item 1 1/2 teaspoons salt
\item 1/4 yeast cake dissolved in
\item 1/4 cup lukewarm water
\item 3 cups flour
\item Rye meal
\end{itemize}
\end{multicols}}
\end{minipage}

\vspace{0.3em}
\noindent%
To milk and water add lard, butter, sugar, and salt; when lukewarm, add
dissolved yeast cake and flour, beat thoroughly, cover, and let rise
until light. Add rye meal until dough is stiff enough to knead; knead
thoroughly, let rise, shape in loaves, let rise again, and bake.



\needspace{15\baselineskip}
\section*{Date Bread}

Use recipe for Health Food Muffins (see p. 67). After the first rising,
while kneading, add two-thirds cup each of English walnut meats cut in
small pieces, and dates stoned and cut in pieces. Shape in a loaf, let
rise in pan, and bake fifty minutes in a moderate oven. This bread is
well adapted for sandwiches.



\needspace{15\baselineskip}
\section*{Boston Brown Bread}


\begin{minipage}{1.0\textwidth}
{\setlength{\multicolsep}{0pt}\setlength{\columnsep}{2em}\raggedcolumns%
\begin{multicols}{2}
\begin{itemize}
\setlength{\itemsep}{0pt}
\setlength{\parsep}{0pt}
\item 1 cup rye meal
\item 1 cup granulated corn meal
\item 1 cup Graham flour
\item 3/4 tablespoon soda
\item 1 teaspoon salt
\item 3/4 cup molasses
\item 2 cups sour milk, or 1 3/4 cups sweet milk or water
\end{itemize}
\end{multicols}}
\end{minipage}

\vspace{0.3em}
\noindent%
Mix and sift dry ingredients, add molasses and milk, stir until well
mixed, turn into a well-buttered mould, and steam three and one-half
hours. The cover should be buttered before being placed on mould, and
then tied down with string; otherwise the bread in rising might force
off cover. Mould should never be filled more than two-thirds full. A
melon mould or one-pound baking-powder boxes make the most
attractive-shaped loaves, but a five-pound lard pail answers the
purpose. For steaming, place mould on a trivet in kettle containing
boiling water, allowing water to come half-way up around mould, cover
closely, and steam, adding, as needed, more boiling water.



\needspace{15\baselineskip}
\section*{New England Brown Bread}


\begin{minipage}{1.0\textwidth}
{\setlength{\multicolsep}{0pt}\setlength{\columnsep}{2em}\raggedcolumns%
\begin{multicols}{2}
\begin{itemize}
\setlength{\itemsep}{0pt}
\setlength{\parsep}{0pt}
\item 1 1/2 cups stale bread
\item 3 1/4 cups cold water
\item 3/4 cup molasses
\item 1 1/2 teaspoons salt
\item 11/2 cups rye meal
\item 11/2 cups granulated corn meal
\item 11/2 cups graham flour
\item 3 teaspoons soda
\end{itemize}
\end{multicols}}
\end{minipage}

\vspace{0.3em}
\noindent%
Soak bread in two cups of the water over night. In the morning rub
through colander, add molasses, dry ingredients mixed and sifted, and
remaining water. Stir until well mixed, fill buttered one-pound
baking-powder boxes two-thirds full, cover, and steam two hours.



\needspace{15\baselineskip}
\section*{Indian Bread}


\begin{minipage}{1.0\textwidth}
{\setlength{\multicolsep}{0pt}\setlength{\columnsep}{2em}\raggedcolumns%
\begin{multicols}{2}
\begin{itemize}
\setlength{\itemsep}{0pt}
\setlength{\parsep}{0pt}
\item 1 1/2 cups Graham flour
\item 1 cup Indian meal
\item 1/2 tablespoon soda
\item 1 teaspoon salt
\item 1/2 cup molasses
\item 1 2/3 cups milk
\end{itemize}
\end{multicols}}
\end{minipage}

\vspace{0.3em}
\noindent%
Mix and steam same as Boston Brown Bread.



\needspace{15\baselineskip}
\section*{Steamed Graham Bread}


\begin{minipage}{1.0\textwidth}
{\setlength{\multicolsep}{0pt}\setlength{\columnsep}{2em}\raggedcolumns%
\begin{multicols}{2}
\begin{itemize}
\setlength{\itemsep}{0pt}
\setlength{\parsep}{0pt}
\item 3 cups Arlington meal
\item 1 cup flour
\item 3 1/2 teaspoons soda
\item 1 teaspoon salt
\item 1 cup molasses (scant)
\item 2 1/2 cups sour milk
\end{itemize}
\end{multicols}}
\end{minipage}

\vspace{0.3em}
\noindent%
Mix same as Boston Brown Bread and steam four hours. This bread may
often be eaten when bread containing corn meal could not be digested.



\needspace{15\baselineskip}
\section*{Parker House Rolls}


\begin{minipage}{1.0\textwidth}
{\setlength{\multicolsep}{0pt}\setlength{\columnsep}{2em}\raggedcolumns%
\begin{multicols}{2}
\begin{itemize}
\setlength{\itemsep}{0pt}
\setlength{\parsep}{0pt}
\item 2 cups scalded milk
\item 3 tablespoons butter
\item 2 tablespoons sugar
\item 1 teaspoon salt
\item 1 yeast cake dissolved in
\item 1/4 cup lukewarm water
\item Flour
\end{itemize}
\end{multicols}}
\end{minipage}

\vspace{0.3em}
\noindent%
Add butter, sugar, and salt to milk; when lukewarm, add dissolved yeast
cake and three cups of flour. Beat thoroughly, cover, and let rise until
light; cut down, and add enough flour to knead (it will take about two
and one-half cups). Let rise again, toss on slightly floured board,
knead, pat, and roll out to one-third inch thickness. Shape with
biscuit-cutter, first dipped in flour. Dip the handle of a case knife in
flour, and with it make a crease through the middle of each piece; brush
over one-half of each piece with melted butter, fold, and press edges
together. Place in greased pan, one inch apart, cover, let rise, and
bake in hot oven twelve to fifteen minutes. As rolls rise they will part
slightly, and if hastened in rising are apt to lose their shape.





Parker House Rolls may be shaped by cutting or tearing off small pieces
of dough, and shaping round like a biscuit; place in rows on floured
board, cover, and let rise fifteen minutes. With handle of large wooden
spoon, or toy rolling-pin, roll through centre of each biscuit, brush
edge of lower halves with melted butter, fold, press lightly, place in
buttered pan one inch apart, cover, let rise, and bake.



\needspace{15\baselineskip}
\section*{Salad Or Dinner Rolls}

Use same ingredients as for Parker House Rolls, allowing one-fourth cup
butter. Shape in small biscuits, place in rows on a floured board, cover
with cloth and pan, and let rise until light and well puffed. Flour
handle of wooden spoon and make a deep crease in middle of each biscuit,
take up, and press edges together. Place closely in buttered pan, cover,
let rise, and bake twelve to fifteen minutes in hot oven. From this same
mixture crescents, braids, twists, bow-knots, clover leaves, and other
fancy shapes may be made.



\needspace{15\baselineskip}
\section*{Sticks}


\begin{minipage}{1.0\textwidth}
{\setlength{\multicolsep}{0pt}\setlength{\columnsep}{2em}\raggedcolumns%
\begin{multicols}{2}
\begin{itemize}
\setlength{\itemsep}{0pt}
\setlength{\parsep}{0pt}
\item 1 cup scalded milk
\item 1/4 cup butter
\item 1 1/2 tablespoons sugar
\item 1/2 teaspoon salt
\item 1 yeast cake dissolved in
\item 1/4 cup lukewarm water
\item 1 egg white
\item 3 3/4 cups flour
\end{itemize}
\end{multicols}}
\end{minipage}

\vspace{0.3em}
\noindent%
Add butter, sugar, and salt to milk; when lukewarm, add dissolved yeast
cake, white of egg well beaten, and flour. Knead, let rise, shape, let
rise again, and start baking in a hot oven, reducing heat, that sticks
may be crisp and dry. To shape sticks, first shape as small biscuits,
roll on board (where there is no flour) with hands until eight inches in
length, keeping of uniform size and rounded ends, which may be done by
bringing fingers close to, but not over, ends of sticks.



\needspace{15\baselineskip}
\section*{Salad Sticks}

Follow recipe for Sticks. Let rise, and add salt to dough, allowing two
teaspoons to each cup of dough. Shape in small sticks, let rise again,
sprinkle with salt, and bake in a slow oven. If preferred glazed, brush
over with egg yolk slightly beaten and diluted with one-half tablespoon
cold water.



\needspace{15\baselineskip}
\section*{Swedish Rolls}

Use recipe for Salad Rolls. Roll to one-fourth inch thickness, spread
with butter, and sprinkle with two tablespoons sugar mixed with
one-third teaspoon cinnamon, one-third cup stoned raisins finely
chopped, and two tablespoons chopped citron; roll up like jelly roll,
and cut in three-fourths inch pieces. Place pieces in pan close
together, flat side down. Again let rise, and bake in a hot oven. When
rolls are taken from oven, brush over with white of egg slightly beaten,
diluted with one-half tablespoon water; return to oven to dry egg, and
thus glaze top.



\needspace{15\baselineskip}
\section*{Sweet French Rolls}


\begin{minipage}{1.0\textwidth}
{\setlength{\multicolsep}{0pt}\setlength{\columnsep}{2em}\raggedcolumns%
\begin{multicols}{2}
\begin{itemize}
\setlength{\itemsep}{0pt}
\setlength{\parsep}{0pt}
\item 1 cup milk
\item 1 yeast cake dissolved in
\item 1/4 cup lukewarm water
\item Flour
\item 1/4 cup sugar
\item 1 teaspoon salt
\item 1 egg
\item Yolk one egg
\item 1/8 teaspoon mace
\item 1/4 cup melted butter
\end{itemize}
\end{multicols}}
\end{minipage}

\vspace{0.3em}
\noindent%
Scald milk; when lukewarm, add dissolved yeast cake and one and one-half
cups flour; beat well, cover, and let rise until light. Add sugar, salt,
eggs well beaten, mace, and butter, and enough more flour to knead;
knead, let rise again, shape, and bake same as Salad Rolls, or roll in a
long strip to one-fourth inch in thickness, spread with butter, roll up
like jelly roll, and cut in one-inch pieces. Place pieces in pan close
together, flat side down. A few gratings from the rind of a lemon or
one-half teaspoon lemon extract may be substituted in place of mace.



\needspace{15\baselineskip}
\section*{Luncheon Rolls}


\begin{minipage}{1.0\textwidth}
{\setlength{\multicolsep}{0pt}\setlength{\columnsep}{2em}\raggedcolumns%
\begin{multicols}{2}
\begin{itemize}
\setlength{\itemsep}{0pt}
\setlength{\parsep}{0pt}
\item 1/2 cup scalded milk
\item 2 tablespoons sugar
\item 1/4 teaspoon salt
\item 1/2 yeast cake dissolved in
\item 2 tablespoons lukewarm water
\item 2 tablespoons melted butter
\item 1 egg
\item Few gratings from rind of lemon
\item Flour
\end{itemize}
\end{multicols}}
\end{minipage}

\vspace{0.3em}
\noindent%
Add sugar and salt to milk; when lukewarm, add dissolved yeast cake and
three-fourths cup flour. Cover and let rise; then add butter, egg well
beaten, grated rind of lemon, and enough flour to knead. Let rise again,
roll to one-half inch thickness, shape with small biscuit-cutter, place
in buttered pan close together, let rise again, and bake.



\needspace{15\baselineskip}
\section*{French Rusks}


\begin{minipage}{1.0\textwidth}
{\setlength{\multicolsep}{0pt}\setlength{\columnsep}{2em}\raggedcolumns%
\begin{multicols}{2}
\begin{itemize}
\setlength{\itemsep}{0pt}
\setlength{\parsep}{0pt}
\item 2 cups scalded milk
\item 1/4 cup butter
\item 1/4 cup sugar
\item 1 teaspoon salt
\item 1 yeast cake dissolved in Flour
\item 1 egg
\item 4 egg yolks
\item 2 egg whites
\item 3/4 teaspoon vanilla
\item 1/4 cup lukewarm water
\end{itemize}
\end{multicols}}
\end{minipage}

\vspace{0.3em}
\noindent%
Add butter, sugar, and salt to scalded milk; when lukewarm add dissolved
yeast cake and three cups flour. Cover and let rise; add egg and egg
yolks well beaten, and enough flour to knead. Let rise again, and shape
as Parker House Rolls. Before baking, make three parallel creases on top
of each roll. When nearly done, brush over with whites of eggs beaten
slightly, diluted with one tablespoon cold water and vanilla. Sprinkle
with sugar.



\needspace{15\baselineskip}
\section*{Rusks (Zweiback)}


\begin{minipage}{1.0\textwidth}
{\setlength{\multicolsep}{0pt}\setlength{\columnsep}{2em}\raggedcolumns%
\begin{multicols}{2}
\begin{itemize}
\setlength{\itemsep}{0pt}
\setlength{\parsep}{0pt}
\item 1/2 cup scalded milk
\item 1/2 teaspoon salt
\item 2 yeast cakes
\item 1/4 cup sugar
\item 1/4 cup melted butter
\item 3 eggs
\item Flour
\end{itemize}
\end{multicols}}
\end{minipage}

\vspace{0.3em}
\noindent%
Dissolve yeast cakes in milk; when lukewarm, add salt and one cup flour;
cover, and let rise until very light; then add sugar, butter, eggs
unbeaten, and flour enough to handle. Shape as finger rolls, and place
close together on a buttered sheet in parallel rows, two inches apart;
let rise again and bake twenty minutes. When cold, cut diagonally in
one-half inch slices, and brown evenly in oven.



\needspace{15\baselineskip}
\section*{German Coffee Bread}


\begin{minipage}{1.0\textwidth}
{\setlength{\multicolsep}{0pt}\setlength{\columnsep}{2em}\raggedcolumns%
\begin{multicols}{2}
\begin{itemize}
\setlength{\itemsep}{0pt}
\setlength{\parsep}{0pt}
\item 1 cup scalded milk
\item 1/3 cup butter, or butter and lard
\item 1/4 cup sugar
\item 1/2 teaspoon salt
\item 1 egg
\item 1/3 yeast cake dissolved in
\item 1/4 cup lukewarm milk
\item 1/2 cup raisins stoned and cut in pieces
\end{itemize}
\end{multicols}}
\end{minipage}

\vspace{0.3em}
\noindent%
Add butter, sugar, and salt to milk; when lukewarm, add dissolved yeast
cake, egg well beaten, flour to make stiff batter, and raisins; cover,
and let rise over night; in morning spread in buttered dripping-pan
one-half inch thick. Cover and let rise again. Before baking, brush over
with beaten egg, and cover with following mixture: Melt three
tablespoons butter, add one-third cup sugar and one teaspoon cinnamon.
When sugar is partially melted, add three tablespoons flour.



\needspace{15\baselineskip}
\section*{Coffee Cakes (Brioche)}


\begin{minipage}{1.0\textwidth}
{\setlength{\multicolsep}{0pt}\setlength{\columnsep}{2em}\raggedcolumns%
\begin{multicols}{2}
\begin{itemize}
\setlength{\itemsep}{0pt}
\setlength{\parsep}{0pt}
\item 1 cup scalded milk
\item 1/4 cup yolks of eggs
\item 1/2 cup whole eggs
\item 2/3 cup butter
\item 1/2 cup sugar
\item 2 yeast cakes
\item 1/2 teaspoon extract lemon or
\item 2 pounded cardamon seeds
\item 4 2/3 cups flour
\end{itemize}
\end{multicols}}
\end{minipage}

\vspace{0.3em}
\noindent%
Cool milk; when lukewarm, add yeast cakes, and when they are dissolved
add remaining ingredients, and beat thoroughly with hand ten minutes;
let rise six hours. Keep in ice-box over night; in morning turn on
floured board, roll in long rectangular piece one-fourth inch thick;
spread with softened butter, fold from sides toward centre to make three
layers. Cut off pieces three-fourths inch wide; cover and let rise. Take
each piece separately in hands and twist from ends in opposite
directions, coil and bring ends together at top of cake. Let rise in
pans and bake twenty minutes in a moderate oven; cool and brush over
with confectioners' sugar, moistened with boiling water to spread, and
flavored with vanilla.



\needspace{15\baselineskip}
\section*{Coffee Rolls}


\begin{minipage}{1.0\textwidth}
{\setlength{\multicolsep}{0pt}\setlength{\columnsep}{2em}\raggedcolumns%
\begin{multicols}{2}
\begin{itemize}
\setlength{\itemsep}{0pt}
\setlength{\parsep}{0pt}
\item 2 cups milk
\item 1 1/2 yeast cakes
\item 1/2 cup butter
\item 1/2 cup lard
\item 1/2 cup sugar
\item Flour
\item 1 egg
\item 1/2 teaspoon cinnamon
\item 1 teaspoon salt
\item Melted butter
\item Confectioners' sugar
\item Vanilla
\end{itemize}
\end{multicols}}
\end{minipage}

\vspace{0.3em}
\noindent%
Scald milk, when lukewarm add yeast cakes, and as soon as dissolved add
three and one-half cups flour. Beat thoroughly, cover, and let rise;
then add butter, lard, sugar, egg unbeaten, cinnamon, salt, and flour
enough to knead. Knead until well mixed, cover, and let rise. Turn
mixture on a floured cloth. Roll into a long, rectangular piece
one-fourth inch thick. Brush over with melted butter, fold from ends
toward centre to make three layers and cut off pieces three-fourths inch
wide. Cover and let rise. Take each piece separately in hands and twist
from ends in opposite directions, then shape in a coil. Place in
buttered pans, cover, again let rise, and bake in a moderate oven twenty
minutes. Cool slightly, and brush over with confectioners' sugar
moistened with boiling water and flavored with vanilla.



\needspace{15\baselineskip}
\section*{Swedish Bread}


\begin{minipage}{1.0\textwidth}
{\setlength{\multicolsep}{0pt}\setlength{\columnsep}{2em}\raggedcolumns%
\begin{multicols}{2}
\begin{itemize}
\setlength{\itemsep}{0pt}
\setlength{\parsep}{0pt}
\item 2 1/2 cups scalded milk
\item 1 yeast cake
\item Flour
\item 1/2 cup melted butter
\item 2/3 cup sugar
\item 1 egg, well beaten
\item 1/4 teaspoon salt
\item 1 teaspoon almond extract
\end{itemize}
\end{multicols}}
\end{minipage}

\vspace{0.3em}
\noindent%
Add yeast cake to one-half cup milk which has been allowed to cool until
lukewarm; as soon as dissolved add one-half cup flour, beat thoroughly,
cover, and let rise. When light, add remaining milk and four and
one-half cups flour. Stir until thoroughly mixed, cover, and again let
rise; then add remaining ingredients and one and one-half cups flour.
Toss on a floured cloth and knead, using one-half cup flour, cover, and
again let rise. Shape as Swedish Tea Braid or Tea Ring I or II, and
bake.

\textbf{Swedish Tea Braid.} Cut off three pieces of mixture of equal size and
roll, using the hands, in pieces of uniform size; then braid. Put on a
buttered sheet, cover, let rise, brush over with yolk of one egg,
slightly beaten, and diluted with one-half tablespoon cold water, and
sprinkle with finely chopped blanched almonds. Bake in a moderate oven.

\textbf{Swedish Tea Ring I.} Shape as tea braid, form in shape of ring, and
proceed as with tea braid, having almonds blanched and cut in slices
crosswise.

\textbf{Swedish Tea Ring II.} Take one-third Swedish Bread mixture and shape,
using the hands, in a long roll. Put on an unfloured board and roll,
using a rolling-pin, as thinly as possible. Mixture will adhere to board
but may be easily lifted with a knife. Spread with melted butter,
sprinkle with sugar and chopped blanched almonds or cinnamon. Roll like
a jelly roll, cut a piece from each end and join ends to form ring.
Place on a buttered sheet, and cut with scissors and shape (see
illustration). Let rise, and proceed as with Tea Ring I.



\needspace{15\baselineskip}
\section*{Dutch Apple Cake}


\begin{minipage}{1.0\textwidth}
{\setlength{\multicolsep}{0pt}\setlength{\columnsep}{2em}\raggedcolumns%
\begin{multicols}{2}
\begin{itemize}
\setlength{\itemsep}{0pt}
\setlength{\parsep}{0pt}
\item 1 cup scalded milk
\item 1/3 cup butter
\item 1/3 cup sugar
\item 1/3 teaspoon salt
\item 1 yeast cake
\item 2 eggs
\item Flour
\item Melted butter
\item 5 sour apples
\item 1/4 cup sugar
\item 1/2 teaspoon cinnamon
\item 2 tablespoons currants
\end{itemize}
\end{multicols}}
\end{minipage}

\vspace{0.3em}
\noindent%
Mix first four ingredients. When lukewarm add yeast cake, eggs unbeaten,
and flour to make a soft dough. Cover, let rise, beat thoroughly, and
again let rise. Spread in a buttered dripping-pan as thinly as possible
and brush over with melted butter. Pare, cut in eighths, and remove
cores from apples.

Press sharp edges of apples into the dough in parallel rows lengthwise
of pan. Sprinkle with sugar mixed with cinnamon and sprinkle with
currants. Cover, let rise, and bake in a moderate oven thirty minutes.
Cut in squares and serve hot or cold with whipped cream sweetened and
flavored.







\needspace{15\baselineskip}
\section*{Buns}


\begin{minipage}{1.0\textwidth}
{\setlength{\multicolsep}{0pt}\setlength{\columnsep}{2em}\raggedcolumns%
\begin{multicols}{2}
\begin{itemize}
\setlength{\itemsep}{0pt}
\setlength{\parsep}{0pt}
\item 1 cup scalded milk
\item 1/3 cup butter
\item 1/3 cup sugar
\item 1 yeast cake dissolved in
\item 1/4 cup lukewarm water
\item 1/2 teaspoon salt
\item 1/2 cup raisins stoned and cut in quarters
\item 1 teaspoon extract lemon
\item Flour
\end{itemize}
\end{multicols}}
\end{minipage}

\vspace{0.3em}
\noindent%
Add one-half sugar and salt to milk; when lukewarm, add dissolved yeast
cake and one and one-half cups flour; cover, and let rise until light;
add butter, remaining sugar, raisins, lemon, and flour to make a dough;
let rise, shape like biscuits, let rise again, and bake. If wanted
glazed, brush over with beaten egg before baking.



\needspace{15\baselineskip}
\section*{Hot Cross Buns}


\begin{minipage}{1.0\textwidth}
{\setlength{\multicolsep}{0pt}\setlength{\columnsep}{2em}\raggedcolumns%
\begin{multicols}{2}
\begin{itemize}
\setlength{\itemsep}{0pt}
\setlength{\parsep}{0pt}
\item 1 cup scalded milk
\item 1/4 cup sugar
\item 2 tablespoons butter
\item 1/2 teaspoon salt
\item 1/2 yeast cake dissolved in
\item 1/4 cup lukewarm water
\item 3/4 teaspoon cinnamon
\item 3 cups flour
\item 1 egg
\item 1/4 cup raisins stoned and quartered, or
\item 1/4 cup currants
\end{itemize}
\end{multicols}}
\end{minipage}

\vspace{0.3em}
\noindent%
Add butter, sugar, and salt to milk; when lukewarm, add dissolved yeast
cake, cinnamon, flour, and egg well beaten; when thoroughly mixed, add
raisins, cover, and let rise over night. In morning, shape in forms of
large biscuits, place in pan one inch apart, let rise, brush over with
beaten egg, and bake twenty minutes; cool, and with ornamental frosting
make a cross on top of each bun.



\needspace{15\baselineskip}
\section*{Raised Muffins}


\begin{minipage}{1.0\textwidth}
{\setlength{\multicolsep}{0pt}\setlength{\columnsep}{2em}\raggedcolumns%
\begin{multicols}{2}
\begin{itemize}
\setlength{\itemsep}{0pt}
\setlength{\parsep}{0pt}
\item 1 cup scalded milk
\item 1 cup boiling water
\item 2 tablespoons butter
\item 1/4 cup sugar
\item 3/4 teaspoon salt
\item 1/4 yeast cake
\item 1 egg
\item 4 cups flour
\end{itemize}
\end{multicols}}
\end{minipage}

\vspace{0.3em}
\noindent%
Add butter, sugar, and salt to milk and water; when lukewarm, add yeast
cake, and when dissolved, egg well beaten, and flour; beat thoroughly,
cover, and let rise over night. In morning, fill buttered muffin rings
two-thirds full; let rise until rings are full, and bake thirty minutes
in hot oven.



\needspace{15\baselineskip}
\section*{Grilled Muffins}

Put buttered muffin rings on a hot greased griddle. Fill one-half full
with raised muffin mixture, and cook slowly until well risen and browned
underneath; turn muffins and rings and brown the other side. This is a
convenient way of cooking muffins when oven is not in condition for
baking.



\needspace{15\baselineskip}
\section*{Raised Hominy Muffins}


\begin{minipage}{1.0\textwidth}
{\setlength{\multicolsep}{0pt}\setlength{\columnsep}{2em}\raggedcolumns%
\begin{multicols}{2}
\begin{itemize}
\setlength{\itemsep}{0pt}
\setlength{\parsep}{0pt}
\item 1 cup warm cooked hominy
\item 1/4 cup butter
\item 1 cup scalded milk
\item 3 tablespoons sugar
\item 1/2 teaspoon salt
\item 1/4 yeast cake
\item 1/4 cup lukewarm water
\item 3 1/4 cups flour
\end{itemize}
\end{multicols}}
\end{minipage}

\vspace{0.3em}
\noindent%
Mix first five ingredients; when lukewarm add yeast cake, dissolved in
lukewarm water and flour. Cover, and let rise over night. In the morning
cut down, fill hot buttered gem pans two-thirds full, let rise one hour,
and bake in a moderate oven. Unless cooked hominy is rather stiff more
flour will be needed.



\needspace{15\baselineskip}
\section*{Raised Rice Muffins}

Make same as Raised Hominy Muffins, substituting one cup hot boiled rice
in place of hominy, and adding the whites of two eggs beaten until
stiff.



\needspace{15\baselineskip}
\section*{Raised Oatmeal Muffins}


\begin{minipage}{1.0\textwidth}
{\setlength{\multicolsep}{0pt}\setlength{\columnsep}{2em}\raggedcolumns%
\begin{multicols}{2}
\begin{itemize}
\setlength{\itemsep}{0pt}
\setlength{\parsep}{0pt}
\item 3/4 cup scalded milk
\item 1/4 cup sugar
\item 1/2 teaspoon salt
\item 1/4 yeast cake dissolved in
\item 1/4 cup lukewarm milk
\item 1 cup cold cooked oatmeal
\item 2 1/2 cups flour
\end{itemize}
\end{multicols}}
\end{minipage}

\vspace{0.3em}
\noindent%
Add sugar and salt to scalded milk; when lukewarm, add dissolved yeast
cake. Work oatmeal into flour with tips of fingers, and add to first
mixture; beat thoroughly, cover, and let rise over night. In morning,
fill buttered iron gem pans two-thirds full, let rise on back of range
that pan may gradually heat and mixture rise to fill pan. Bake in
moderate oven twenty-five to thirty minutes.



\needspace{15\baselineskip}
\section*{Health Food Muffins}


\begin{minipage}{1.0\textwidth}
{\setlength{\multicolsep}{0pt}\setlength{\columnsep}{2em}\raggedcolumns%
\begin{multicols}{2}
\begin{itemize}
\setlength{\itemsep}{0pt}
\setlength{\parsep}{0pt}
\item 1 cup warm wheat mush
\item 1/4 cup brown sugar
\item 1/2 teaspoon salt
\item 1 tablespoon butter
\item 1/4 yeast cake
\item 1/4 cup lukewarm water
\item Flour
\end{itemize}
\end{multicols}}
\end{minipage}

\vspace{0.3em}
\noindent%
Mix first four ingredients, add yeast cake dissolved in lukewarm water,
and flour to knead. Cover, and let rise over night. In the morning cut
down, fill hot buttered gem pans two-thirds full and bake in a moderate
oven. This mixture, when baked in a loaf, makes a delicious bread.



\needspace{15\baselineskip}
\section*{Squash Biscuits}


\begin{minipage}{1.0\textwidth}
{\setlength{\multicolsep}{0pt}\setlength{\columnsep}{2em}\raggedcolumns%
\begin{multicols}{2}
\begin{itemize}
\setlength{\itemsep}{0pt}
\setlength{\parsep}{0pt}
\item 1/2 cup squash (steamed and sifted)
\item 1/4 cup sugar
\item 1/2 teaspoon salt
\item 1/2 cup scalded milk
\item 1/4 yeast cake dissolved in
\item 1/4 cup lukewarm water
\item 1/4 cup butter
\item 2 1/2 cups flour
\end{itemize}
\end{multicols}}
\end{minipage}

\vspace{0.3em}
\noindent%
Add squash, sugar, salt, and butter to milk; when lukewarm, add
dissolved yeast cake and flour; cover, and let rise over night. In
morning shape into biscuits, let rise, and bake.



\needspace{15\baselineskip}
\section*{Imperial Muffins}


\begin{minipage}{1.0\textwidth}
{\setlength{\multicolsep}{0pt}\setlength{\columnsep}{2em}\raggedcolumns%
\begin{multicols}{2}
\begin{itemize}
\setlength{\itemsep}{0pt}
\setlength{\parsep}{0pt}
\item 1 cup scalded milk
\item 1/4 cup sugar
\item 1/2 teaspoon salt
\item 1 3/4 cups flour
\item 1 cup corn meal
\item 1/4 cup butter
\item 1/3 yeast cake dissolved in 1/4 cup lukewarm water
\end{itemize}
\end{multicols}}
\end{minipage}

\vspace{0.3em}
\noindent%
Add sugar and salt to milk; when lukewarm add dissolved yeast cake, and
one and one-fourth cups flour. Cover, and let rise until light, then add
corn meal, remaining flour, and butter. Let rise over night; in the
morning fill buttered muffin rings two-thirds full; let rise until rings
are full and bake thirty minutes in hot oven.



\needspace{15\baselineskip}
\section*{Dry Toast}

Cut stale bread in one-fourth inch slices. Crust may or may not be
removed. Put slices on wire toaster, lock toaster and place over clear
fire to dry, holding some distance from coals; turn and dry other side.
Hold nearer to coals and color a golden brown on each side. Toast, if
piled compactly and allowed to stand, will soon become moist. Toast may
be buttered at table or before sending to table.



\needspace{15\baselineskip}
\section*{Water Toast}

Dip slices of dry toast quickly in boiling salted water, allowing
one-half teaspoon salt to one cup boiling water. Spread slices with
butter, and serve at once.



\needspace{15\baselineskip}
\section*{Milk Toast I}


\begin{minipage}{1.0\textwidth}
{\setlength{\multicolsep}{0pt}\setlength{\columnsep}{2em}\raggedcolumns%
\begin{multicols}{2}
\begin{itemize}
\setlength{\itemsep}{0pt}
\setlength{\parsep}{0pt}
\item 1 pint scalded milk
\item 2 tablespoons butter
\item 2 1/2 tablespoons bread flour
\item 1/2 teaspoon salt
\item Cold water
\item 6 slices dry toast
\end{itemize}
\end{multicols}}
\end{minipage}

\vspace{0.3em}
\noindent%
Add cold water gradually to flour to make a smooth, thin paste. Add to
milk, stirring constantly until thickened, cover, and cook twenty
minutes; then add salt and butter in small pieces. Dip slices of toast
separately in sauce; when soft, remove to serving dish. Pour remaining
sauce over all.



\needspace{15\baselineskip}
\section*{Milk Toast II}

Use ingredients given in Milk Toast I, omitting cold water, and make as
Thin White Sauce. Dip toast in sauce.



\needspace{15\baselineskip}
\section*{Brown Bread Milk Toast}

Make same as Milk Toast, using slices of toasted brown bread in place of
white bread. Brown bread is better toasted by first drying slices in
oven.



\needspace{15\baselineskip}
\section*{Cream Toast}

Substitute cream for milk, and omit butter in recipe for Milk Toast I or
II.



\needspace{15\baselineskip}
\section*{Tomato Cream Toast}


\begin{minipage}{1.0\textwidth}
{\setlength{\multicolsep}{0pt}\setlength{\columnsep}{2em}\raggedcolumns%
\begin{multicols}{2}
\begin{itemize}
\setlength{\itemsep}{0pt}
\setlength{\parsep}{0pt}
\item 1 1/2 cups stewed and strained tomato
\item 1/2 cup scalded cream
\item 1/4 teaspoon soda
\item 3 tablespoons butter
\item 3 tablespoons flour
\item 1/2 teaspoon salt
\item 6 slices toast
\end{itemize}
\end{multicols}}
\end{minipage}

\vspace{0.3em}
\noindent%
Put butter in saucepan; when melted and bubbling, add flour, mixed with
salt, and stir in gradually tomato, to which soda has been added, then
add cream. Dip slices of toast in sauce. Serve as soon as made.



\needspace{15\baselineskip}
\section*{German Toast}


\begin{itemize}
\setlength{\itemsep}{0pt}
\setlength{\parsep}{0pt}
\item 3 eggs
\item 1/2 teaspoon salt
\item 2 tablespoons sugar
\item 1 cup milk
\item 6 slices stale bread
\end{itemize}

\vspace{-0.5em}
\noindent%
Beat eggs slightly, add salt, sugar, and milk; strain into a shallow
dish. Soak bread in mixture until soft. Cook on a hot, well-greased
griddle; brown on one side, turn and brown other side. Serve for
breakfast or luncheon, or with a sauce for dessert.



\needspace{15\baselineskip}
\section*{Brewis}

Break stale bits or slices of brown and white bread in small pieces,
allowing one and one-half cups brown bread to one-half cup white bread.
Butter a hot frying-pan, put in bread, and cover with equal parts milk
and water. Cook until soft; add butter and salt to taste.



\needspace{15\baselineskip}
\section*{Bread For Garnishing}

Dry toast is often used for garnishing, cut in various shapes. Always
shape before toasting. Cubes of bread, toast points, and small oblong
pieces are most common. Cubes of stale bread, from which centres are
removed, are fried in deep fat and called croûstades; half-inch cubes,
browned in butter, or fried in deep fat, are called croûtons.



\needspace{15\baselineskip}
\section*{Uses For Stale Bread}

All pieces of bread should be saved and utilized. Large pieces are best
for toast. Soft stale bread, from which crust is removed, when crumbed,
is called stale breadcrumbs, or raspings, and is used for puddings,
griddle-cakes, omelets, scalloped dishes, and dipping food to be fried.
Remnants of bread, from which crusts have not been removed, are dried in
oven, rolled, and sifted. These are called dry bread crumbs, and are
useful for crumbing croquettes, cutlets, fish, meat, etc.





\chapter{Biscuits, Breakfast Cakes, And Shortcakes}




\needspace{15\baselineskip}
\section*{Batters, Sponges, And Doughs}

Batter is a mixture of flour and some liquid (usually combined with
other ingredients, as sugar, salt, eggs, etc.), of consistency to pour
easily, or to drop from a spoon.


\begin{itemize}
\setlength{\itemsep}{0pt}
\item Batters are termed thin or thick, according to their consistency.
\item Sponge is a batter to which yeast is added.
\item Dough differs from batter inasmuch as it is stiff enough to be handled.
\end{itemize}




\needspace{15\baselineskip}
\section*{Cream Scones}


\begin{minipage}{1.0\textwidth}
{\setlength{\multicolsep}{0pt}\setlength{\columnsep}{2em}\raggedcolumns%
\begin{multicols}{2}
\begin{itemize}
\setlength{\itemsep}{0pt}
\setlength{\parsep}{0pt}
\item 2 cups flour
\item 4 teaspoons baking power
\item 2 teaspoons sugar
\item 1/2 teaspoon salt
\item 4 tablespoons butter
\item 2 eggs
\item 1/3 cup cream
\end{itemize}
\end{multicols}}
\end{minipage}

\vspace{0.3em}
\noindent%
Mix and sift together flour, baking powder, sugar, and salt. Rub in
butter with tips of fingers; add eggs well beaten, and cream. Toss on a
floured board, pat, and roll to three-fourths inch in thickness. Cut in
squares, brush with white of egg, sprinkle with sugar, and bake in a hot
oven fifteen minutes.



\needspace{15\baselineskip}
\section*{Baking Powder Biscuit I}


\begin{minipage}{1.0\textwidth}
{\setlength{\multicolsep}{0pt}\setlength{\columnsep}{2em}\raggedcolumns%
\begin{multicols}{2}
\begin{itemize}
\setlength{\itemsep}{0pt}
\setlength{\parsep}{0pt}
\item 2 cups flour
\item 4 teaspoons baking powder
\item 1 teaspoon salt
\item 1 tablespoon lard
\item 3/4 cup milk and water in equal parts
\item 1 tablespoon butter
\end{itemize}
\end{multicols}}
\end{minipage}

\vspace{0.3em}
\noindent%
Mix dry ingredients, and sift twice.

Work in butter and lard with tips of fingers; add gradually the liquid,
mixing with knife to a soft dough. It is impossible to determine the
exact amount of liquid, owing to differences in flour. Toss on a floured
board, pat and roll lightly to one-half inch in thickness. Shape with a
biscuit-cutter. Place on buttered pan, and bake in hot oven twelve to
fifteen minutes. If baked in too slow an oven, the gas will escape
before it has done its work. Many obtain better results by using bread
flour.



\needspace{15\baselineskip}
\section*{Baking Powder Biscuit II}


\begin{itemize}
\setlength{\itemsep}{0pt}
\setlength{\parsep}{0pt}
\item 2 cups flour
\item 4 teaspoons baking powder
\item 2 tablespoons butter
\item 3/4 cup milk
\item 1/2 teaspoon salt
\end{itemize}

\vspace{-0.5em}
\noindent%
Mix and bake as Baking Powder Biscuit I.



\needspace{15\baselineskip}
\section*{Emergency Biscuit}

Use recipe for Baking Powder Biscuit I or II, with the addition of more
milk, that mixture may be dropped from spoon without spreading. Drop by
spoonfuls on a buttered pan, one-half inch apart. Brush over with milk,
and bake in hot oven eight minutes.



\needspace{15\baselineskip}
\section*{Fruit Rolls (Pin Wheel Biscuit)}


\begin{minipage}{1.0\textwidth}
{\setlength{\multicolsep}{0pt}\setlength{\columnsep}{2em}\raggedcolumns%
\begin{multicols}{2}
\begin{itemize}
\setlength{\itemsep}{0pt}
\setlength{\parsep}{0pt}
\item 2 cups flour
\item 4 teaspoons baking powder
\item 1/2 teaspoon salt
\item 2 tablespoons sugar
\item 2 tablespoons butter
\item 2/3 cup milk
\item 1/3 cup stoned raisins (finely chopped)
\item 2 tablespoons citron (finely chopped)
\item 1/3 teaspoon cinnamon
\end{itemize}
\end{multicols}}
\end{minipage}

\vspace{0.3em}
\noindent%
Mix as Baking Powder Biscuit II. Roll to one-fourth inch thickness,
brush over with melted butter, and sprinkle with fruit, sugar, and
cinnamon. Roll like a jelly roll; cut off pieces three-fourths inch in
thickness. Place on buttered tin, and bake in hot oven fifteen minutes.
Currants may be used in place of raisins and citron.



\needspace{15\baselineskip}
\section*{Twin Mountain Muffins}


\begin{minipage}{1.0\textwidth}
{\setlength{\multicolsep}{0pt}\setlength{\columnsep}{2em}\raggedcolumns%
\begin{multicols}{2}
\begin{itemize}
\setlength{\itemsep}{0pt}
\setlength{\parsep}{0pt}
\item 1/4 cup butter
\item 1/4 cup sugar
\item 1 egg
\item 3/4 cup milk
\item 2 cups flour
\item 3 teaspoons baking powder
\end{itemize}
\end{multicols}}
\end{minipage}

\vspace{0.3em}
\noindent%
Cream the butter; add sugar and egg well beaten; sift baking powder with
flour, and add to the first mixture, alternating with milk. Bake in
buttered tin gem pans twenty-five minutes.



\needspace{15\baselineskip}
\section*{One Egg Muffins I}


\begin{minipage}{1.0\textwidth}
{\setlength{\multicolsep}{0pt}\setlength{\columnsep}{2em}\raggedcolumns%
\begin{multicols}{2}
\begin{itemize}
\setlength{\itemsep}{0pt}
\setlength{\parsep}{0pt}
\item 3 1/2 cups flour
\item 6 teaspoons baking powder
\item 1 teaspoon salt
\item 1 1/3 cups milk
\item 3 tablespoons melted butter
\item 1 egg
\item 3 tablespoons sugar
\end{itemize}
\end{multicols}}
\end{minipage}

\vspace{0.3em}
\noindent%
Mix and sift dry ingredients; add gradually milk, egg well beaten, and
melted butter. Bake in buttered gem pans twenty-five minutes. If iron
pans are used they must be previously heated. This recipe makes thirty
muffins. Use half the proportions given and a small egg, if half the
number is required.



\needspace{15\baselineskip}
\section*{One Egg Muffins II}


\begin{minipage}{1.0\textwidth}
{\setlength{\multicolsep}{0pt}\setlength{\columnsep}{2em}\raggedcolumns%
\begin{multicols}{2}
\begin{itemize}
\setlength{\itemsep}{0pt}
\setlength{\parsep}{0pt}
\item 2 cups flour
\item 4 teaspoons baking powder
\item 1/2 teaspoon salt
\item 2 tablespoons sugar
\item 1 cup milk
\item 2 tablespoons melted butter
\item 1 egg
\end{itemize}
\end{multicols}}
\end{minipage}

\vspace{0.3em}
\noindent%
Mix and bake as One Egg Muffin I.



\needspace{15\baselineskip}
\section*{Berry Muffins I (Without Eggs)}


\begin{minipage}{1.0\textwidth}
{\setlength{\multicolsep}{0pt}\setlength{\columnsep}{2em}\raggedcolumns%
\begin{multicols}{2}
\begin{itemize}
\setlength{\itemsep}{0pt}
\setlength{\parsep}{0pt}
\item 2 cups flour
\item 1/4 cup sugar
\item 4 teaspoons baking powder
\item 2 tablespoon butter
\item 1 cup milk (scant)
\item 1 cup berries
\item 1/2 teaspoon salt
\end{itemize}
\end{multicols}}
\end{minipage}

\vspace{0.3em}
\noindent%
Mix and sift dry ingredients; work in butter with tips of fingers; add
milk and berries.



\needspace{15\baselineskip}
\section*{Berry Muffins II}


\begin{minipage}{1.0\textwidth}
{\setlength{\multicolsep}{0pt}\setlength{\columnsep}{2em}\raggedcolumns%
\begin{multicols}{2}
\begin{itemize}
\setlength{\itemsep}{0pt}
\setlength{\parsep}{0pt}
\item 1/4 cup butter
\item 1/3 cup sugar
\item 1 egg
\item 2 2/3 cups flour
\item 4 teaspoons baking powder
\item 1/2 teaspoon salt
\item 1 cup milk
\item 1 cup berries
\end{itemize}
\end{multicols}}
\end{minipage}

\vspace{0.3em}
\noindent%
Cream the butter; add gradually sugar and egg well beaten; mix and sift
flour, baking powder, and salt, reserving one-fourth cup flour to be
mixed with berries and added last; the remainder alternately with milk.



\needspace{15\baselineskip}
\section*{Queen Of Muffins}


\begin{minipage}{1.0\textwidth}
{\setlength{\multicolsep}{0pt}\setlength{\columnsep}{2em}\raggedcolumns%
\begin{multicols}{2}
\begin{itemize}
\setlength{\itemsep}{0pt}
\setlength{\parsep}{0pt}
\item 1/4 cup butter
\item 1/3 cup sugar
\item 1 egg
\item 1/2 cup milk (scant)
\item 1 1/2 cups flour
\item 2 1/2 teaspoons baking powder
\end{itemize}
\end{multicols}}
\end{minipage}

\vspace{0.3em}
\noindent%
Mix and bake same as Twin Mountain Muffins.



\needspace{15\baselineskip}
\section*{Rice Muffins}


\begin{minipage}{1.0\textwidth}
{\setlength{\multicolsep}{0pt}\setlength{\columnsep}{2em}\raggedcolumns%
\begin{multicols}{2}
\begin{itemize}
\setlength{\itemsep}{0pt}
\setlength{\parsep}{0pt}
\item 2 1/4 cups flour
\item 3/4 cup hot cooked rice
\item 5 teaspoons baking powder
\item 2 tablespoons sugar
\item 1 cup milk
\item 1 egg
\item 2 tablespoons melted butter
\item 1/2 teaspoon salt
\end{itemize}
\end{multicols}}
\end{minipage}

\vspace{0.3em}
\noindent%
Mix and sift flour, sugar, salt, and baking powder; add one-half milk,
egg well beaten, the remainder of the milk mixed with rice, and beat
thoroughly; then add butter. Bake in buttered muffin rings placed in
buttered pan or buttered gem pans.



\needspace{15\baselineskip}
\section*{Oatmeal Muffins}


\begin{minipage}{1.0\textwidth}
{\setlength{\multicolsep}{0pt}\setlength{\columnsep}{2em}\raggedcolumns%
\begin{multicols}{2}
\begin{itemize}
\setlength{\itemsep}{0pt}
\setlength{\parsep}{0pt}
\item 1 cup cooked oatmeal
\item 1 1/2 cups flour
\item 2 tablespoons sugar
\item 4 teaspoons baking powder
\item 1/2 teaspoon salt
\item 1/2 cup milk
\item 1 egg
\item 2 tablespoons melted butter
\end{itemize}
\end{multicols}}
\end{minipage}

\vspace{0.3em}
\noindent%
Mix and bake as Rice Muffins.



\needspace{15\baselineskip}
\section*{Graham Muffins I}


\begin{minipage}{1.0\textwidth}
{\setlength{\multicolsep}{0pt}\setlength{\columnsep}{2em}\raggedcolumns%
\begin{multicols}{2}
\begin{itemize}
\setlength{\itemsep}{0pt}
\setlength{\parsep}{0pt}
\item 1 1/4 cups Graham flour
\item 1 cup flour
\item 1 cup sour milk
\item 1/3 cup molasses
\item 3/4 teaspoon soda
\item 1 teaspoon salt
\end{itemize}
\end{multicols}}
\end{minipage}

\vspace{0.3em}
\noindent%
Mix and sift dry ingredients; add milk to molasses, and combine
mixtures.



\needspace{15\baselineskip}
\section*{Graham Muffins II}


\begin{minipage}{1.0\textwidth}
{\setlength{\multicolsep}{0pt}\setlength{\columnsep}{2em}\raggedcolumns%
\begin{multicols}{2}
\begin{itemize}
\setlength{\itemsep}{0pt}
\setlength{\parsep}{0pt}
\item 1 cup Graham or entire wheat flour
\item 1 cup flour
\item 1/4 cup sugar
\item 1 teaspoon salt
\item 1 cup milk
\item 1 egg
\item 1 tablespoon melted butter
\item 4 teaspoons baking powder
\end{itemize}
\end{multicols}}
\end{minipage}

\vspace{0.3em}
\noindent%
Mix and sift dry ingredients; add milk gradually, egg well beaten, and
melted butter; bake in hot oven in buttered gem pans twenty-five
minutes.



\needspace{15\baselineskip}
\section*{Rye Muffins I}

Make as Graham Muffins II, substituting rye meal for Graham flour.



\needspace{15\baselineskip}
\section*{Rye Muffins II}


\begin{minipage}{1.0\textwidth}
{\setlength{\multicolsep}{0pt}\setlength{\columnsep}{2em}\raggedcolumns%
\begin{multicols}{2}
\begin{itemize}
\setlength{\itemsep}{0pt}
\setlength{\parsep}{0pt}
\item 1 1/4 cups rye meal
\item 1 1/4 cups flour
\item 4 teaspoons baking powder
\item 1 teaspoon salt
\item 1/4 cup molasses
\item 1 1/4 cups milk
\item 1 egg
\item 1 tablespoon melted butter
\end{itemize}
\end{multicols}}
\end{minipage}

\vspace{0.3em}
\noindent%
Mix and bake as Graham Muffins II, adding molasses with milk.



\needspace{15\baselineskip}
\section*{Rye Gems}


\begin{minipage}{1.0\textwidth}
{\setlength{\multicolsep}{0pt}\setlength{\columnsep}{2em}\raggedcolumns%
\begin{multicols}{2}
\begin{itemize}
\setlength{\itemsep}{0pt}
\setlength{\parsep}{0pt}
\item 1 2/3 cups rye flour
\item 1 1/3 cups flour
\item 4 teaspoons baking powder
\item 1 teaspoon salt
\item 1/4 cup molasses
\item 1 1/4 cups milk
\item 2 eggs
\item 3 tablespoons melted butter
\end{itemize}
\end{multicols}}
\end{minipage}

\vspace{0.3em}
\noindent%
Mix and sift dry ingredients, add molasses, milk, eggs well beaten, and
butter. Bake in hot oven in buttered gem pans twenty-five minutes.



\needspace{15\baselineskip}
\section*{Corn Meal Gems}


\begin{minipage}{1.0\textwidth}
{\setlength{\multicolsep}{0pt}\setlength{\columnsep}{2em}\raggedcolumns%
\begin{multicols}{2}
\begin{itemize}
\setlength{\itemsep}{0pt}
\setlength{\parsep}{0pt}
\item 1/2 cup corn meal
\item 1 cup flour
\item 3 teaspoons baking powder
\item 1 tablespoon sugar
\item 1 tablespoon melted butter
\item 1/2 teaspoon salt
\item 3/4 cup milk
\item 1 egg
\end{itemize}
\end{multicols}}
\end{minipage}

\vspace{0.3em}
\noindent%
Mix and bake as Graham Muffins II.



\needspace{15\baselineskip}
\section*{Hominy Gems}


\begin{minipage}{1.0\textwidth}
{\setlength{\multicolsep}{0pt}\setlength{\columnsep}{2em}\raggedcolumns%
\begin{multicols}{2}
\begin{itemize}
\setlength{\itemsep}{0pt}
\setlength{\parsep}{0pt}
\item 1/4 cup hominy
\item 1/2 teaspoon salt
\item 1/2 cup boiling water
\item 1 cup scalded milk
\item 1 cup corn meal
\item 3 tablespoons sugar
\item 3 tablespoons butter
\item 2 eggs
\item 3 teaspoons baking powder
\end{itemize}
\end{multicols}}
\end{minipage}

\vspace{0.3em}
\noindent%
Add hominy mixed with salt to boiling water and let stand until hominy
absorbs water. Add scalded milk to corn meal, then add sugar and butter.
Combine mixtures, cool slightly, add yolks of eggs beaten until thick,
and whites of eggs beaten until stiff. Sift in baking powder and beat
thoroughly. Bake in hot buttered gem pans.



\needspace{15\baselineskip}
\section*{Berkshire Muffins}


\begin{minipage}{1.0\textwidth}
{\setlength{\multicolsep}{0pt}\setlength{\columnsep}{2em}\raggedcolumns%
\begin{multicols}{2}
\begin{itemize}
\setlength{\itemsep}{0pt}
\setlength{\parsep}{0pt}
\item 1/2 cup corn meal
\item 1/2 cup flour
\item 1/2 cup cooked rice
\item 2 tablespoons sugar
\item 1/2 teaspoon salt
\item 2/3 cup scalded milk (scant)
\item 1 egg
\item 1 tablespoon melted butter
\item 3 teaspoons baking powder
\end{itemize}
\end{multicols}}
\end{minipage}

\vspace{0.3em}
\noindent%
Turn scalded milk on meal, let stand five minutes; add rice, and flour
mixed and sifted with remaining dry ingredients. Add yolk of egg well
beaten, butter, and white of egg beaten stiff and dry.



\needspace{15\baselineskip}
\section*{Golden Corn Cake}


\begin{minipage}{1.0\textwidth}
{\setlength{\multicolsep}{0pt}\setlength{\columnsep}{2em}\raggedcolumns%
\begin{multicols}{2}
\begin{itemize}
\setlength{\itemsep}{0pt}
\setlength{\parsep}{0pt}
\item 3/4 cup corn meal
\item 1 1/4 cups flour
\item 1/4 cup sugar
\item 5 teaspoons baking powder
\item 1/2 teaspoon salt
\item 1 cup milk
\item 1 egg
\item 1 or 2 tablespoons melted butter
\end{itemize}
\end{multicols}}
\end{minipage}

\vspace{0.3em}
\noindent%
Mix and sift dry ingredients. Add milk, egg well beaten, and butter;
bake in shallowed buttered pan in hot oven twenty minutes.



\needspace{15\baselineskip}
\section*{Corn Cake (Sweetened With Molasses)}


\begin{minipage}{1.0\textwidth}
{\setlength{\multicolsep}{0pt}\setlength{\columnsep}{2em}\raggedcolumns%
\begin{multicols}{2}
\begin{itemize}
\setlength{\itemsep}{0pt}
\setlength{\parsep}{0pt}
\item 1 cup corn meal
\item 3/4 cup flour
\item 3 1/2 teaspoons baking powder
\item 1 teaspoon salt
\item 1/4 cup molasses
\item 3/4 cup milk
\item 1 egg
\item 1 tablespoon melted butter
\end{itemize}
\end{multicols}}
\end{minipage}

\vspace{0.3em}
\noindent%
Mix and bake as Golden Corn Cake, adding molasses to milk.



\needspace{15\baselineskip}
\section*{White Corn Cake}


\begin{minipage}{1.0\textwidth}
{\setlength{\multicolsep}{0pt}\setlength{\columnsep}{2em}\raggedcolumns%
\begin{multicols}{2}
\begin{itemize}
\setlength{\itemsep}{0pt}
\setlength{\parsep}{0pt}
\item 1/4 cup butter
\item 1/2 cup sugar
\item 1 1/3 cups milk
\item 3 egg whites
\item 1 1/4 cups white corn meal
\item 1 1/4  cups flour
\item 4 teaspoons baking powder
\item 1 teaspoon salt
\end{itemize}
\end{multicols}}
\end{minipage}

\vspace{0.3em}
\noindent%
Cream the butter; add sugar gradually; add milk, alternating with dry
ingredients, mixed and sifted. Beat thoroughly; add whites of eggs
beaten stiff. Bake in buttered cake pan thirty minutes.



\needspace{15\baselineskip}
\section*{Rich Corn Cake}


\begin{minipage}{1.0\textwidth}
{\setlength{\multicolsep}{0pt}\setlength{\columnsep}{2em}\raggedcolumns%
\begin{multicols}{2}
\begin{itemize}
\setlength{\itemsep}{0pt}
\setlength{\parsep}{0pt}
\item 1 cup corn meal
\item 1 cup white flour
\item 4 teaspoons baking powder
\item 1/4 cup sugar
\item 1/2 teaspoon salt
\item 7/8 cup milk
\item 2 eggs
\item 1/4 cup melted butter
\end{itemize}
\end{multicols}}
\end{minipage}

\vspace{0.3em}
\noindent%
Mix and sift dry ingredients. Add milk, gradually, eggs well beaten, and
butter. Bake in a buttered, shallow pan, in a hot oven.



\needspace{15\baselineskip}
\section*{Susie's Spider Corn Cake}


\begin{minipage}{1.0\textwidth}
{\setlength{\multicolsep}{0pt}\setlength{\columnsep}{2em}\raggedcolumns%
\begin{multicols}{2}
\begin{itemize}
\setlength{\itemsep}{0pt}
\setlength{\parsep}{0pt}
\item 1 1/4 cups corn meal
\item 2 cups sour milk
\item 1 teaspoon soda
\item 1 teaspoon salt
\item 2 eggs
\item 2 tablespoons butter
\end{itemize}
\end{multicols}}
\end{minipage}

\vspace{0.3em}
\noindent%
Mix soda, salt, and corn meal; gradually add eggs well beaten and milk.
Heat frying-pan, grease sides and bottom of pan with butter, turn in the
mixture, place on middle grate in hot oven, and cook twenty minutes.



\needspace{15\baselineskip}
\section*{White Corn Meal Cake}


\begin{itemize}
\setlength{\itemsep}{0pt}
\setlength{\parsep}{0pt}
\item 1 cup scalded milk
\item 1/2 cup white corn meal
\item 1 teaspoon salt
\end{itemize}

\vspace{-0.5em}
\noindent%
Add salt to corn meal, and pour on gradually milk. Turn into a buttered
shallow pan to the depth of one-fourth inch. Bake in a moderate oven
until crisp. Split and spread with butter.



\needspace{15\baselineskip}
\section*{Pop-Overs}


\begin{itemize}
\setlength{\itemsep}{0pt}
\setlength{\parsep}{0pt}
\item 1 cup flour
\item 1/4 teaspoon salt
\item 7/8 cup milk
\item 2 eggs
\item 1/2 teaspoon melted butter
\end{itemize}

\vspace{-0.5em}
\noindent%
Mix salt and flour; add milk gradually, in order to obtain a smooth
batter. Add egg, beaten until light, and butter; beat two minutes,--using
Dover egg-beater,--turn into hissing hot buttered iron gem pans, and bake
thirty to thirty-five minutes in a hot oven. They may be baked in
buttered earthen cups, when the bottom will have a glazed appearance.
Small round iron gem pans are best for Pop-overs.



\needspace{15\baselineskip}
\section*{Graham Pop-Overs}


\begin{minipage}{1.0\textwidth}
{\setlength{\multicolsep}{0pt}\setlength{\columnsep}{2em}\raggedcolumns%
\begin{multicols}{2}
\begin{itemize}
\setlength{\itemsep}{0pt}
\setlength{\parsep}{0pt}
\item 2/3 cup entire wheat flour
\item 1/3 cup flour
\item 1/4 teaspoon salt
\item 7/8 cup milk
\item 1 egg
\item 1/2 teaspoon melted butter
\end{itemize}
\end{multicols}}
\end{minipage}

\vspace{0.3em}
\noindent%
Prepare and bake as Pop-overs.



\needspace{15\baselineskip}
\section*{Breakfast Puffs}


\begin{itemize}
\setlength{\itemsep}{0pt}
\setlength{\parsep}{0pt}
\item 1 cup flour
\item 1/2 cup milk
\item 1/2 cup water
\end{itemize}

\vspace{-0.5em}
\noindent%
Mix milk and water; add gradually to flour, and beat with Dover
egg-beater until very light. Bake same as Pop-overs.



\needspace{15\baselineskip}
\section*{Fadges}


\begin{itemize}
\setlength{\itemsep}{0pt}
\setlength{\parsep}{0pt}
\item 1 cup entire wheat flour
\item 1 cup cold water
\end{itemize}

\vspace{-0.5em}
\noindent%
Add water gradually to flour, and beat with Dover egg-beater until very
light. Bake same as Pop-overs.



\needspace{15\baselineskip}
\section*{Zante Muffins}


\begin{minipage}{1.0\textwidth}
{\setlength{\multicolsep}{0pt}\setlength{\columnsep}{2em}\raggedcolumns%
\begin{multicols}{2}
\begin{itemize}
\setlength{\itemsep}{0pt}
\setlength{\parsep}{0pt}
\item 1/2 cup butter
\item 3/4 cup sugar
\item 3 eggs
\item 1 1/2 cups milk
\item 2 cups corn meal
\item 1 cup flour
\item 1 teaspoon salt
\item 5 teaspoons baking powder
\item 1/2 cup currants
\end{itemize}
\end{multicols}}
\end{minipage}

\vspace{0.3em}
\noindent%
Cream the butter; add sugar, gradually, eggs well beaten, and milk; then
add dry ingredients mixed and sifted, and currants. Bake in buttered
individual tins.



\needspace{15\baselineskip}
\section*{Maryland Biscuit}


\begin{itemize}
\setlength{\itemsep}{0pt}
\setlength{\parsep}{0pt}
\item 1 pint flour
\item 1/3 cup lard
\item 1 teaspoon salt
\item Milk and water in equal quantities
\end{itemize}

\vspace{-0.5em}
\noindent%
Mix and sift flour and salt; work in lard with tips of fingers, and
moisten to a stiff dough. Toss on slightly floured board, and beat with
rolling-pin thirty minutes, continually folding over the dough. Roll
one-third inch in thickness, shape with round cutter two inches in
diameter, prick with fork, and place on a buttered tin. Bake twenty
minutes in hot oven.



\needspace{15\baselineskip}
\section*{Griddle-Cakes}


\needspace{15\baselineskip}
\subsection*{Sour Milk Griddle-cakes}


\begin{itemize}
\setlength{\itemsep}{0pt}
\setlength{\parsep}{0pt}
\item 2 1/2 cups flour
\item 1/2 teaspoon salt
\item 2 cups sour milk
\item 1 1/4 teaspoons soda
\item 1 egg
\end{itemize}

\vspace{-0.5em}
\noindent%
Mix and sift flour, salt, and soda; add sour milk, and egg well beaten.
Drop by spoonfuls on a greased hot griddle; cook on one side. When
puffed, full of bubbles, and cooked on edges, turn, and cook other side.
Serve with butter and maple syrup.



\needspace{15\baselineskip}
\subsection*{Sweet Milk Griddle-cakes}


\begin{minipage}{1.0\textwidth}
{\setlength{\multicolsep}{0pt}\setlength{\columnsep}{2em}\raggedcolumns%
\begin{multicols}{2}
\begin{itemize}
\setlength{\itemsep}{0pt}
\setlength{\parsep}{0pt}
\item 3 cups flour
\item 1 1/2 tablespoons baking powder
\item 1 teaspoon salt
\item 1/4 cup sugar
\item 2 cups milk
\item 1 egg
\item 2 tablespoons melted butter
\end{itemize}
\end{multicols}}
\end{minipage}

\vspace{0.3em}
\noindent%
Mix and sift dry ingredients; beat egg, add milk, and pour slowly on
first mixture. Beat thoroughly, and add butter. Cook same as Sour Milk
Griddle-cakes. Begin cooking cakes at once or more baking powder will be
required.



\needspace{15\baselineskip}
\subsection*{Entire Wheat Griddle-cakes}


\begin{minipage}{1.0\textwidth}
{\setlength{\multicolsep}{0pt}\setlength{\columnsep}{2em}\raggedcolumns%
\begin{multicols}{2}
\begin{itemize}
\setlength{\itemsep}{0pt}
\setlength{\parsep}{0pt}
\item 1/2 cup entire wheat flour
\item 1 cup flour
\item 3 teaspoons baking powder
\item 1/2 teaspoon salt
\item 3 tablespoons sugar
\item 1 egg
\item 1 1/4 cups milk
\item 1 tablespoon melted butter
\end{itemize}
\end{multicols}}
\end{minipage}

\vspace{0.3em}
\noindent%
Prepare and cook same as Sweet Milk Griddle-cakes.



\needspace{15\baselineskip}
\subsection*{Corn Griddle-cakes}


\begin{minipage}{1.0\textwidth}
{\setlength{\multicolsep}{0pt}\setlength{\columnsep}{2em}\raggedcolumns%
\begin{multicols}{2}
\begin{itemize}
\setlength{\itemsep}{0pt}
\setlength{\parsep}{0pt}
\item 2 cups flour
\item 1/2 cup corn meal
\item 1 1/2 tablespoons baking powder
\item 1 1/2 teaspoons salt
\item 1/3 cup sugar
\item 1 1/2 cups boiling water
\item 1 1/4 cups milk
\item 1 egg
\item 2 tablespoons melted butter
\end{itemize}
\end{multicols}}
\end{minipage}

\vspace{0.3em}
\noindent%
Add meal to boiling water, and boil five minutes; turn into bowl, add
milk, and remaining dry ingredients mixed and sifted, then the egg well
beaten, and butter. Cook same as other griddle-cakes.



\needspace{15\baselineskip}
\subsection*{Rice Griddle-cakes I}


\begin{minipage}{1.0\textwidth}
{\setlength{\multicolsep}{0pt}\setlength{\columnsep}{2em}\raggedcolumns%
\begin{multicols}{2}
\begin{itemize}
\setlength{\itemsep}{0pt}
\setlength{\parsep}{0pt}
\item 2 1/2 cups flour
\item 1/2 cup cold cooked rice
\item 1 tablespoon baking powder
\item 1/2 teaspoon salt
\item 1/4 cup sugar
\item 1 1/2 cups milk
\item 1 egg
\item 2 tablespoons melted butter
\end{itemize}
\end{multicols}}
\end{minipage}

\vspace{0.3em}
\noindent%
Mix and sift dry ingredients. Work in rice with tips of fingers; add egg
well beaten, milk, and butter. Cook same as other griddle-cakes.



\needspace{15\baselineskip}
\subsection*{Rice Griddle-cakes II}


\begin{minipage}{1.0\textwidth}
{\setlength{\multicolsep}{0pt}\setlength{\columnsep}{2em}\raggedcolumns%
\begin{multicols}{2}
\begin{itemize}
\setlength{\itemsep}{0pt}
\setlength{\parsep}{0pt}
\item 1 cup milk
\item 1 cup warm boiled rice
\item 1/2 teaspoon salt
\item 4 egg yolks
\item 2 egg whites
\item 1 tablespoon melted butter
\item 7/8 cup flour
\end{itemize}
\end{multicols}}
\end{minipage}

\vspace{0.3em}
\noindent%
Pour milk over rice and salt, add yolks of eggs beaten until thick and
lemon color, butter, flour, and fold in whites of eggs beaten until
stiff and dry.



\needspace{15\baselineskip}
\subsection*{Bread Griddle-cakes}


\begin{minipage}{1.0\textwidth}
{\setlength{\multicolsep}{0pt}\setlength{\columnsep}{2em}\raggedcolumns%
\begin{multicols}{2}
\begin{itemize}
\setlength{\itemsep}{0pt}
\setlength{\parsep}{0pt}
\item 1 1/2 cups fine stale bread crumbs
\item 1 1/2 cups scalded milk
\item 2 tablespoons butter
\item 2 eggs
\item 1/2 cup flour
\item 1/2 teaspoon salt
\item 4 teaspoons baking powder
\end{itemize}
\end{multicols}}
\end{minipage}

\vspace{0.3em}
\noindent%
Add milk and butter to crumbs, and soak until crumbs are soft; add eggs
well beaten, then flour, salt, and baking powder mixed and sifted. Cook
same as other griddle-cakes.



\needspace{15\baselineskip}
\section*{Buckwheat Cakes}


\begin{minipage}{1.0\textwidth}
{\setlength{\multicolsep}{0pt}\setlength{\columnsep}{2em}\raggedcolumns%
\begin{multicols}{2}
\begin{itemize}
\setlength{\itemsep}{0pt}
\setlength{\parsep}{0pt}
\item 1/3 cup fine bread crumbs
\item 2 cups scalded milk
\item 1/2 teaspoon salt
\item 1/4 yeast cake
\item 1/2 cup lukewarm water
\item 1 3/4 cups buckwheat flour
\item 1 tablespoon molasses
\end{itemize}
\end{multicols}}
\end{minipage}

\vspace{0.3em}
\noindent%
Pour milk over crumbs, and soak thirty minutes; add salt, yeast cake
dissolved in lukewarm water, and buckwheat to make a batter thin enough
to pour. Let rise over night; in the morning, stir well, add molasses,
one-fourth teaspoon soda dissolved in one-fourth cup lukewarm water, and
cook same as griddle-cakes. Save enough batter to raise another mixing,
instead of using yeast cake; it will require one-half cup.



\needspace{15\baselineskip}
\section*{Waffles}


\begin{minipage}{1.0\textwidth}
{\setlength{\multicolsep}{0pt}\setlength{\columnsep}{2em}\raggedcolumns%
\begin{multicols}{2}
\begin{itemize}
\setlength{\itemsep}{0pt}
\setlength{\parsep}{0pt}
\item 1 3/4 cups flour
\item 3 teaspoons baking powder
\item 1/2 teaspoon salt
\item 1 cup milk
\item 4 egg yolks
\item 2 egg whites
\item 1 tablespoon melted butter
\end{itemize}
\end{multicols}}
\end{minipage}

\vspace{0.3em}
\noindent%
Mix and sift dry ingredients; add milk gradually, yolks of eggs well
beaten, butter, and whites of eggs beaten stiff; cook on a greased hot
waffle-iron. Serve with maple syrup.

A waffle-iron should fit closely on range, be well heated on one side,
turned, heated on other side, and thoroughly greased before iron is
filled. In filling, put a tablespoonful of mixture in each compartment
near centre of iron, cover, and mixture will spread to just fill iron.
If sufficiently heated, it should be turned almost as soon as filled and
covered. In using a new iron, special care must be taken in greasing, or
waffles will stick.



\needspace{15\baselineskip}
\section*{Waffles With Boiled Cider}

Follow directions for making Waffles. Serve with BOILED CIDER. Allow twice as much cider as sugar, and let boil until of
a syrup consistency.



\needspace{15\baselineskip}
\section*{Rice Waffles}


\begin{minipage}{1.0\textwidth}
{\setlength{\multicolsep}{0pt}\setlength{\columnsep}{2em}\raggedcolumns%
\begin{multicols}{2}
\begin{itemize}
\setlength{\itemsep}{0pt}
\setlength{\parsep}{0pt}
\item 1 3/4 cups flour
\item 2/3 cup cold cooked rice
\item 1 1/2  cups milk
\item 2 tablespoons sugar
\item 4 teaspoons baking powder
\item 1/4 teaspoon salt
\item 1 tablespoon melted butter
\item 1 egg
\end{itemize}
\end{multicols}}
\end{minipage}

\vspace{0.3em}
\noindent%
Mix and sift dry ingredients; work in rice with tips of fingers; add
milk, yolk of egg well beaten, butter, and white of egg beaten stiff.
Cook same as Waffles.



\needspace{15\baselineskip}
\section*{Virginia Waffles}


\begin{minipage}{1.0\textwidth}
{\setlength{\multicolsep}{0pt}\setlength{\columnsep}{2em}\raggedcolumns%
\begin{multicols}{2}
\begin{itemize}
\setlength{\itemsep}{0pt}
\setlength{\parsep}{0pt}
\item 1 1/2 cups boiling water
\item 1/2 cup white corn meal
\item 1 1/2 cups milk
\item 3 cups flour
\item 3 tablespoons sugar
\item 1 1/4 tablespoons baking powder
\item 1 1/2 teaspoons salt
\item 4 egg yolks
\item 2 egg whites
\item 2 tablespoons melted butter
\end{itemize}
\end{multicols}}
\end{minipage}

\vspace{0.3em}
\noindent%
Cook meal in boiling water twenty minutes; add milk, dry ingredients
mixed and sifted, yolks of eggs well beaten, butter, and whites of eggs
beaten stiff. Cook same as Waffles.







\needspace{15\baselineskip}
\section*{Raised Waffles}


\begin{minipage}{1.0\textwidth}
{\setlength{\multicolsep}{0pt}\setlength{\columnsep}{2em}\raggedcolumns%
\begin{multicols}{2}
\begin{itemize}
\setlength{\itemsep}{0pt}
\setlength{\parsep}{0pt}
\item 1 3/4 cups milk
\item 1 teaspoon salt
\item 1 tablespoon butter
\item 1/4 yeast cake
\item 1/4 cup lukewarm water
\item 2 cups flour
\item 4 egg yolks
\item 2 egg whites
\end{itemize}
\end{multicols}}
\end{minipage}

\vspace{0.3em}
\noindent%
Scald milk; add salt and butter, and when lukewarm, add yeast cake
dissolved in water, and flour. Beat well; let rise over night; add yolks
of eggs well beaten, and whites of eggs beaten stiff. Cook same as
Waffles. By using a whole yeast cake, the mixture will rise in one and
one-half hours.



\needspace{15\baselineskip}
\section*{Fried Drop Cakes}


\begin{minipage}{1.0\textwidth}
{\setlength{\multicolsep}{0pt}\setlength{\columnsep}{2em}\raggedcolumns%
\begin{multicols}{2}
\begin{itemize}
\setlength{\itemsep}{0pt}
\setlength{\parsep}{0pt}
\item 1 1/3 cups flour
\item 2 1/2 teaspoons baking powder
\item 1/4 teaspoon salt
\item 1/3 cup sugar
\item 1/2 cup milk
\item 1 egg
\item 1 teaspoon melted butter
\end{itemize}
\end{multicols}}
\end{minipage}

\vspace{0.3em}
\noindent%
Beat egg until light; add milk, dry ingredients mixed and sifted, and
melted butter. Drop by spoonfuls in hot, new, deep fat; fry until light
brown and cooked through, which must at first be determined by piercing
with a skewer, or breaking apart. Remove with a skimmer, and drain on
brown paper.



\needspace{15\baselineskip}
\section*{Rye Drop Cakes}


\begin{minipage}{1.0\textwidth}
{\setlength{\multicolsep}{0pt}\setlength{\columnsep}{2em}\raggedcolumns%
\begin{multicols}{2}
\begin{itemize}
\setlength{\itemsep}{0pt}
\setlength{\parsep}{0pt}
\item 2/3 cup rye meal
\item 2/3 cup flour
\item 2 1/2 teaspoons baking powder
\item 1/2 teaspoon salt
\item 2 tablespoons molasses
\item 1/2 cup milk
\item 1 egg
\end{itemize}
\end{multicols}}
\end{minipage}

\vspace{0.3em}
\noindent%
Mix and sift dry ingredients; add milk gradually, molasses, and egg well
beaten. Cook same as Fried Drop Cakes.



\needspace{15\baselineskip}
\section*{Raised Doughnuts}


\begin{minipage}{1.0\textwidth}
{\setlength{\multicolsep}{0pt}\setlength{\columnsep}{2em}\raggedcolumns%
\begin{multicols}{2}
\begin{itemize}
\setlength{\itemsep}{0pt}
\setlength{\parsep}{0pt}
\item 1 cup milk
\item 1/4 yeast cake
\item 1/4 cup lukewarm water
\item 1 teaspoon salt
\item 1/3 cup butter and lard mixed
\item 1 cup light brown sugar
\item 2 eggs
\item 1/2 grated nutmeg
\item Flour
\end{itemize}
\end{multicols}}
\end{minipage}

\vspace{0.3em}
\noindent%
Scald and cool milk; when lukewarm, add yeast cake dissolved in water,
salt, and flour enough to make a stiff batter; let rise over night. In
morning add shortening melted, sugar, eggs well beaten, nutmeg, and
enough flour to make a stiff dough; let rise again, and if too soft to
handle, add more flour. Toss on floured board, pat, and roll to
three-fourths inch thickness. Shape with cutter, and work between hands
until round. Place on floured board, let rise one hour, turn, and let
rise again; fry in deep fat, and drain on brown paper. Cool, and roll in
powdered sugar.



\needspace{15\baselineskip}
\section*{Doughnuts I}


\begin{minipage}{1.0\textwidth}
{\setlength{\multicolsep}{0pt}\setlength{\columnsep}{2em}\raggedcolumns%
\begin{multicols}{2}
\begin{itemize}
\setlength{\itemsep}{0pt}
\setlength{\parsep}{0pt}
\item 1 cup sugar
\item 2 1/2 tablespoons butter
\item 3 eggs
\item 1 cup milk
\item 4 teaspoons baking powder
\item 1/4 teaspoon cinnamon
\item 1/4 teaspoon grated nutmeg
\item 1 1/2 teaspoons salt
\item Flour to roll
\end{itemize}
\end{multicols}}
\end{minipage}

\vspace{0.3em}
\noindent%
Cream the butter, and add one-half sugar. Beat egg until light, add
remaining sugar, and combine mixtures. Add three and one-half cups
flour, mixed and sifted with baking powder, salt, and spices; then
enough more flour to make dough stiff enough to roll. Toss one-third of
mixture on floured board, knead slightly, pat, and roll out to
one-fourth inch thickness. Shape with a doughnut cutter, fry in deep
fat, take up on a skewer, and drain on brown paper. Add trimmings to
one-half remaining mixture, roll, shape, and fry as before; repeat.
Doughnuts should come quickly to top of fat, brown on one side, then be
turned to brown on the other; avoid turning more than once. The fat must
be kept at a uniform temperature. If too cold, doughnuts will absorb
fat; if too hot, doughnuts will brown before sufficiently risen. See
rule for testing fat.



\needspace{15\baselineskip}
\section*{Doughnuts II}


\begin{minipage}{1.0\textwidth}
{\setlength{\multicolsep}{0pt}\setlength{\columnsep}{2em}\raggedcolumns%
\begin{multicols}{2}
\begin{itemize}
\setlength{\itemsep}{0pt}
\setlength{\parsep}{0pt}
\item 4 cups flour
\item 1 1/2 teaspoons salt
\item 1 3/4 teaspoons soda
\item 1 3/4 teaspoons cream of tartar
\item 1/4 teaspoon grated nutmeg
\item 1/4 teaspoon cinnamon
\item 1/2 tablespoon butter
\item 1 cup sugar
\item 1 cup sour milk
\item 1 egg
\end{itemize}
\end{multicols}}
\end{minipage}

\vspace{0.3em}
\noindent%
Put flour in shallow pan; add salt, soda, cream of tartar, and spices.
Work in butter with tips of fingers; add sugar, egg well beaten, and
sour milk. Stir thoroughly, and toss on board thickly dredged with
flour; knead slightly, using more flour if necessary. Pat and roll out
to one-fourth inch thickness; shape, fry, and drain. Sour milk doughnuts
may be turned as soon as they come to top of fat, and frequently
afterwards.



\needspace{15\baselineskip}
\section*{Doughnuts III}


\begin{minipage}{1.0\textwidth}
{\setlength{\multicolsep}{0pt}\setlength{\columnsep}{2em}\raggedcolumns%
\begin{multicols}{2}
\begin{itemize}
\setlength{\itemsep}{0pt}
\setlength{\parsep}{0pt}
\item 2 cups sugar
\item 4 eggs
\item 1 1/3 cups sour milk
\item 4 tablespoons melted butter
\item 2 teaspoons soda
\item 2 teaspoons salt
\item 2 teaspoons baking powder
\item 1 teaspoon grated nutmeg
\item Flour
\end{itemize}
\end{multicols}}
\end{minipage}

\vspace{0.3em}
\noindent%
Mix ingredients in order given; shape, fry, and drain.



\needspace{15\baselineskip}
\section*{Crullers}


\begin{minipage}{1.0\textwidth}
{\setlength{\multicolsep}{0pt}\setlength{\columnsep}{2em}\raggedcolumns%
\begin{multicols}{2}
\begin{itemize}
\setlength{\itemsep}{0pt}
\setlength{\parsep}{0pt}
\item 1/4 cup butter
\item 1 cup sugar
\item 4 egg yolks
\item 2 egg whites
\item 4 cups flour
\item 1/4 teaspoon grated nutmeg
\item 3 1/2 teaspoons baking powder
\item 1 cup milk
\item Powdered sugar and cinnamon
\end{itemize}
\end{multicols}}
\end{minipage}

\vspace{0.3em}
\noindent%
Cream the butter, add sugar gradually, yolks of eggs well beaten, and
whites of eggs beaten stiff. Mix flour, nutmeg, and baking powder; add
alternately with milk to first mixture; toss on floured board, roll
thin, and cut in pieces three inches long by two inches wide; make four
one-inch parallel gashes crosswise at equal intervals. Take up by
running finger in and out of gashes, and lower into deep fat. Fry same
as Doughnuts I.



\needspace{15\baselineskip}
\section*{Strawberry Short Cake I}


\begin{minipage}{1.0\textwidth}
{\setlength{\multicolsep}{0pt}\setlength{\columnsep}{2em}\raggedcolumns%
\begin{multicols}{2}
\begin{itemize}
\setlength{\itemsep}{0pt}
\setlength{\parsep}{0pt}
\item 2 cups flour
\item 4 teaspoons baking powder
\item 1/2 teaspoon salt
\item 2 teaspoons sugar
\item 3/4 cup milk
\item 1/4 cup butter
\end{itemize}
\end{multicols}}
\end{minipage}

\vspace{0.3em}
\noindent%
Mix dry ingredients, sift twice, work in butter with tips of fingers,
and add milk gradually. Toss on floured board, divide in two parts. Pat,
roll out, and bake twelve minutes in a hot oven in buttered Washington
pie or round layer cake tins. Split, and spread with butter. Sweeten
strawberries to taste, place on back of range until warmed, crush
slightly, and put between and on top of Short Cakes; cover top with
Cream Sauce I.



\needspace{15\baselineskip}
\section*{Strawberry Short Cake II}


\begin{minipage}{1.0\textwidth}
{\setlength{\multicolsep}{0pt}\setlength{\columnsep}{2em}\raggedcolumns%
\begin{multicols}{2}
\begin{itemize}
\setlength{\itemsep}{0pt}
\setlength{\parsep}{0pt}
\item 2 cups flour
\item 4 teaspoons baking powder
\item 1/2 teaspoon salt
\item 1 tablespoon sugar
\item 1/3 cup butter
\item 3/4 cup milk
\end{itemize}
\end{multicols}}
\end{minipage}

\vspace{0.3em}
\noindent%
Mix same as Strawberry Short Cake I. Toss and roll on floured board. Put
in round buttered tin, and shape with back of hand to fit pan.



\needspace{15\baselineskip}
\section*{Rich Strawberry Short Cake}


\begin{minipage}{1.0\textwidth}
{\setlength{\multicolsep}{0pt}\setlength{\columnsep}{2em}\raggedcolumns%
\begin{multicols}{2}
\begin{itemize}
\setlength{\itemsep}{0pt}
\setlength{\parsep}{0pt}
\item 2 cups flour
\item 1/4 cup sugar
\item 4 teaspoons baking powder
\item 1/2 teaspoon salt
\item Few grains nutmeg
\item 1 egg
\item 1/3 cup butter
\item 1 1/4 tablespoons lard
\item 1/3 cup milk
\end{itemize}
\end{multicols}}
\end{minipage}

\vspace{0.3em}
\noindent%
Mix dry ingredients and sift twice, work in shortening with tips of
fingers, add egg well beaten, and milk. Bake same as Strawberry Short
Cake II. Split cake and spread under layer with Cream Sauce II. Cover
with strawberries which have been sprinkled with powdered sugar; again
spread with sauce, and cover with upper layer.



\needspace{15\baselineskip}
\section*{Fruit Short Cake}


\begin{minipage}{1.0\textwidth}
{\setlength{\multicolsep}{0pt}\setlength{\columnsep}{2em}\raggedcolumns%
\begin{multicols}{2}
\begin{itemize}
\setlength{\itemsep}{0pt}
\setlength{\parsep}{0pt}
\item 1/4 cup butter
\item 1/2 cup sugar
\item 1 egg
\item 1/4 cup milk
\item 1 cup flour
\item 2 teaspoons baking powder
\item 1/4 teaspoon salt
\end{itemize}
\end{multicols}}
\end{minipage}

\vspace{0.3em}
\noindent%
Cream the butter, add sugar gradually, and egg well beaten. Mix and sift
flour, baking powder, and salt, adding alternately with milk to first
mixture. Beat thoroughly, and bake in a buttered round tin. Cool, spread
thickly with sweetened fruit, and cover with Cream Sauce I or II. Fresh
strawberries, peaches, apricots, raspberries, or canned quince or
pineapple may be used. When canned goods are used, drain fruit from
syrup and cut in pieces. Dilute cream for Cream Sauce with fruit syrup
in place of milk.

Any shortcake mixture may be made for individual service by shaping with
a large biscuit-cutter; or mixture may be baked in a shallow cake pan,
centre removed and filled with fruit, and pieces baked separately to
introduce to represent handles.





\chapter{Cereals}



Cereals (cultivated grasses) rank first among vegetable foods; being of
hardy growth and easy cultivation, they are more widely diffused over
the globe than any of the flowering plants. They include wheat, oats,
rye, barley, maize (Indian corn), and rice; some authorities place
buckwheat among them. Wheat probably is the most largely consumed; next
to wheat, comes rice.



\needspace{15\baselineskip}
\section*{Table Showing Composition}


\begin{tabular}{p{2in}ccccc}
\hline
Item & Proteid & Fat & Starch & Mineral matter & Water \\
\hline
Oatmeal & 15.6 & 7.3 & 68.0 & 1.9 & 7.2 \\
\arrayrulecolor{tablerowgray}\hline
Corn meal & 8.9 & 2.2 & 75.1 & 0.9 & 12.9 \\
\arrayrulecolor{tablerowgray}\hline
Wheat flour (spring) & 11.8 & 1.1 & 75.0 & 0.5 & 11.6 \\
\arrayrulecolor{tablerowgray}\hline
Entire wheat flour & 14.2 & 1.9 & 70.6 & 1.2 & 12.1 \\
\arrayrulecolor{tablerowgray}\hline
Graham flour & 13.7 & 2.2 & 70.3 & 2.0 & 11.8 \\
\arrayrulecolor{tablerowgray}\hline
Pearl barley & 9.3 & 1.0 & 77.6 & 1.3 & 10.8 \\
\arrayrulecolor{tablerowgray}\hline
Rye meal & 7.1 & 0.9 & 78.5 & 0.8 & 12.7 \\
\arrayrulecolor{tablerowgray}\hline
Rice & 7.8 & 0.4 & 79.4 & 0.4 & 12.4 \\
\arrayrulecolor{tablerowgray}\hline
Buckwheat flour & 6.1 & 1.0 & 77.2 & 1.4 & 14.3 \\
\arrayrulecolor{tablerowgray}\hline
Macaroni & 11.7 & 1.6 & 72.9 & 3.0 & 10.8 \\
\arrayrulecolor{black}
\hline
\end{tabular}

\textit{Macaroni}, \textit{spaghetti}, and \textit{vermicelli} are made from wheaten flour,
rich in gluten, moistened to a stiff dough with water, and forced
through small apertures in an iron plate by means of a screw press.
Various Italian pastes are made from the same mixture. Macaroni is
manufactured to some extent in this country, but the best comes from
Italy, Lagana and Pejero, being the favorite brand. When macaroni is
colored, it is done by the use of saffron, not by eggs, as is generally
supposed. The only egg macaroni is manufactured in strips, and comes
from Minneapolis.

Macaroni is valuable food, as it is very cheap and nutritious; but being
deficient in fat, it should be combined with cream, butter, or cheese,
to make a perfect food.

From cereals many preparations are made, used alone, or in combination
with other food products. From rice is made rice flour; from oats,
oatmeal, and oats steam-cooked and rolled,--as Rolled Avena, Quaker
Rolled Oats, H-O, etc. There are many species of corn, the principal
varieties being white, yellow, and red. From corn is made corn
meal,--both white and yellow,--corn-starch, hominy, maizena, cerealine,
samp, and hulled corn; from wheat, wheaten or white flour, Wheatena,
Wheatlet, Rolled Wheat, Pettijohn's, etc. Rye is used for Rye Flakes,
meal, and flour; barley, for flour and pearl barley. Buckwheat,
throughout the United States, is used only when made into flour for
buckwheat cakes.

For family use, cereals should be bought in small quantities, and kept
in glass jars, tightly covered. Many cereal preparations are on the
market for making breakfast mushes, put up in one and two pound
packages, with directions for cooking. In nearly all cases, time allowed
for cooking is not sufficient, unless dish containing cereal is brought
in direct contact with fire, which is not the best way. Mushes should be
cooked over hot water after the first five minutes; if a double boiler
is not procurable, improvise one. Boiling water and salt should always
be added to cereals, allowing one teaspoon salt to each cup of
cereal,--boiled to soften cellulose and swell starch-grains, salted to
give flavor. Indian meal and finely ground preparations should be mixed
with cold water before adding boiling water, to prevent lumping.



\needspace{15\baselineskip}
\section*{Table For Cooking Cereals}


\begin{tabular}{p{2in}p{0.8in}p{1.2in}p{1in}}
\hline
Kind & Quantity & Water & Time \\
\hline
Steam-cooked and rolled oats, Rolled Avena, Quaker Rolled Oats, H-O, Old Grist Mill, Rolled Oats, & 1 cup & 1 3/4 cups & 30 minutes \\
\arrayrulecolor{tablerowgray}\hline
Steam-cooked and rolled wheats, Old Grist Mill, Rye Flakes, Pettijohn's, etc. & 1 cup & 1 1/4 cups & 20 minutes \\
\arrayrulecolor{tablerowgray}\hline
Rice (steamed) & 1 cup & 2 3/4--3 1/4 cups (according to age of rice) & 45--60 minutes \\
\arrayrulecolor{tablerowgray}\hline
Indian meal & 1 cup & 3 1/2 cups & 3 hours \\
\arrayrulecolor{tablerowgray}\hline
Vitos & 1 cup & 4 1/2 cups & 30 minutes \\
\arrayrulecolor{tablerowgray}\hline
Wheatlet, Wheatena, Wheat Germ, Toasted Wheat, & 1 cup & 3 3/4 cups & 30 minutes \\
\arrayrulecolor{tablerowgray}\hline
Oatmeal (coarse) & 1 cup & 4 cups & 3 hours \\
\arrayrulecolor{tablerowgray}\hline
Hominy (fine) & 1 cup & 4 cups & 1 hour \\
\arrayrulecolor{tablerowgray}\hline
Oatmeal Mush with Apples &  &  &  \\
\arrayrulecolor{black}
\hline
\end{tabular}

Core apples, leaving large cavities; pare, and cook until soft in syrup
made by boiling sugar and water together, allowing one cup sugar to one
and one-half cups water. Fill cavities with oatmeal mush; serve with
sugar and cream. The syrup should be saved and re-used. Berries, sliced
bananas, or sliced peaches, are acceptably served with any breakfast
cereal.



\needspace{15\baselineskip}
\section*{Cereal With Fruit}


\begin{itemize}
\setlength{\itemsep}{0pt}
\setlength{\parsep}{0pt}
\item 3/4 cup Wheat Germ
\item 3/4 cup cold water
\item 2 cups boiling water
\item 1 teaspoon salt
\item 1/2 lb. dates, stoned, and cut in pieces
\end{itemize}

\vspace{-0.5em}
\noindent%
Mix cereal, salt, and cold water; add to boiling water placed on front
of range. Boil five minutes, steam in double boiler thirty minutes; stir
in dates, and serve with cream. To serve for breakfast, or as a simple
dessert.



\needspace{15\baselineskip}
\section*{Fried Mushes}

Mush left over from breakfast may be packed in greased, one pound
baking-powder box, and covered, which will prevent crust from forming.
The next morning remove from box, slice thinly, dip in flour, and sauté.
Serve with maple syrup.



\needspace{15\baselineskip}
\section*{Fried Corn Meal Mush, Or Fried Hominy}

Pack corn meal or hominy mush in greased, one pound baking-powder boxes,
or small bread pan, cool, and cover. Cut in thin slices, and sauté; cook
slowly, if preferred crisp and dry. Where mushes are cooked to fry, use
less water in steaming.



\needspace{15\baselineskip}
\section*{Boiled Rice}


\begin{itemize}
\setlength{\itemsep}{0pt}
\setlength{\parsep}{0pt}
\item 1 cup rice
\item 2 quarts boiling water
\item 1 tablespoon salt
\end{itemize}

\vspace{-0.5em}
\noindent%
Pick over rice; add slowly to boiling, salted water, so as not to check
boiling of water. Boil thirty minutes, or until soft, which may be
determined by testing kernels. Old rice absorbs much more water than new
rice, and takes longer for cooking. Drain in coarse strainer, and pour
over one quart hot water; return to kettle in which it was cooked;
cover, place on back of range, and let stand to dry off, when kernels
are distinct. When stirring rice, always use a fork to avoid breaking
kernels.



\needspace{15\baselineskip}
\section*{Steamed Rice}


\begin{itemize}
\setlength{\itemsep}{0pt}
\setlength{\parsep}{0pt}
\item 1 cup rice
\item 1 teaspoon salt
\item 2 3/4 to 3 1/4 cups boiling water (according to age of rice)
\end{itemize}

\vspace{-0.5em}
\noindent%
Put salt and water in top of double boiler, place on range, and add
gradually well-washed rice, stirring with a fork to prevent adhering to
boiler. Boil five minutes, cover, place over under part double boiler,
and steam forty-five minutes, or until kernels are soft; uncover, that
steam may escape. When rice is steamed for a simple dessert, use
one-half quantity of water given in recipe, and steam until rice has
absorbed water; then add scalded milk for remaining liquid.

\textbf{To wash rice.} Put rice in strainer, place strainer over bowl nearly
full of cold water; rub rice between hands, lift strainer from bowl, and
change water. Repeat process three or four times, until water is quite
clear.



\needspace{15\baselineskip}
\section*{Rice With Cheese}

Steam one cup rice, allowing one tablespoon salt; cover bottom of
buttered pudding-dish with rice, dot over with three-fourths tablespoon
butter, sprinkle with thin shavings mild cheese and a few grains
cayenne; repeat until rice and one-fourth pound cheese are used. Add
milk to half the depth of contents of dish, cover with buttered cracker
crumbs, and bake until cheese melts.



\needspace{15\baselineskip}
\section*{Rice À La Riston}

Finely chop two thin slices bacon, add to one-half raw medium-sized
cabbage, finely chopped; cover, and cook slowly thirty minutes. Add
one-fourth cup rice, boiled, one-half teaspoon chopped parsley, and salt
and pepper to taste. Moisten with one-half cup White Stock, and cook
fifteen minutes.



\needspace{15\baselineskip}
\section*{Turkish Pilaf I}

Wash and drain one-half cup rice, cook in one tablespoon butter until
brown, add one cup boiling water, and steam until water is absorbed. Add
one and three-fourths cups hot stewed tomatoes, cook until rice is soft,
and season with salt and pepper.



\needspace{15\baselineskip}
\section*{Turkish Pilaf II}


\begin{itemize}
\setlength{\itemsep}{0pt}
\setlength{\parsep}{0pt}
\item 1/2 cup washed rice
\item 3/4 cup tomatoes, stewed and strained
\item 1 cup Brown Stock, highly seasoned
\item 3 tablespoons butter
\end{itemize}

\vspace{-0.5em}
\noindent%
Add tomato to stock, and heat to boiling-point; add rice, and steam
until rice is soft; stir in butter with a fork, and keep uncovered that
steam may escape. Serve in place of a vegetable, or as border for
curried or fricasseed meat.



\needspace{15\baselineskip}
\section*{Turkish Pilaf III}


\begin{minipage}{1.0\textwidth}
{\setlength{\multicolsep}{0pt}\setlength{\columnsep}{2em}\raggedcolumns%
\begin{multicols}{2}
\begin{itemize}
\setlength{\itemsep}{0pt}
\setlength{\parsep}{0pt}
\item 1/3 cup rice
\item 3 tablespoons butter
\item 1/2 cup canned tomatoes
\item 1/2 cup cold cooked chicken cut in dice
\item White Stock highly seasoned
\item Salt and cayenne
\end{itemize}
\end{multicols}}
\end{minipage}

\vspace{0.3em}
\noindent%
Cook rice in boiling salted water, drain, and pour over hot water to
thoroughly rinse. Heat omelet pan, add butter, and as soon as butter is
melted add rice. Cook three minutes; then add tomatoes, chicken, and
enough stock to moisten. Cook five minutes, and season highly with salt
and cayenne. If not rich enough, add more butter.



\needspace{15\baselineskip}
\section*{Russian Pilaf}

Follow recipe for Turkish Pilaf III, substituting cold cooked lamb in
place of chicken, and add a chicken's liver sautéd in butter, then
separated into small pieces.



\needspace{15\baselineskip}
\section*{Rissoto Creole}


\begin{itemize}
\setlength{\itemsep}{0pt}
\setlength{\parsep}{0pt}
\item 3 tablespoons butter
\item 1 cup rice
\item 2 3/4 cups highly seasoned
\item Brown Stock
\item Canned pimentoes
\end{itemize}

\vspace{-0.5em}
\noindent%
Melt butter in hot frying-pan, add rice, and stir constantly until rice
is well browned. Add stock heated to boiling-point, and cook in double
boiler until soft. Turn on a serving dish, garnish with pimentoes cut in
fancy shapes, and cover with

\textbf{Creole Sauce.} Cook two tablespoons chopped onion, two tablespoons
chopped green pepper, one tablespoon chopped red pepper, or canned
pimentoes, and four tablespoons chopped fresh mushrooms, with three
tablespoons butter, five minutes. Add two tablespoons flour, one cup
tomatoes, one truffle thinly sliced, one-fourth cup sherry wine, and
salt to taste.



\needspace{15\baselineskip}
\section*{Boiled Macaroni}


\begin{itemize}
\setlength{\itemsep}{0pt}
\setlength{\parsep}{0pt}
\item 3/4 cup macaroni broken in inch pieces
\item 2 quarts boiling water
\item 1 tablespoon salt
\item 1/2 cup cream
\end{itemize}

\vspace{-0.5em}
\noindent%
Cook macaroni in boiling salted water twenty minutes or until soft,
drain in strainer, pour over it cold water to prevent pieces from
adhering; add cream, reheat, and season with salt.



\needspace{15\baselineskip}
\section*{Macaroni With White Sauce}


\begin{itemize}
\setlength{\itemsep}{0pt}
\setlength{\parsep}{0pt}
\item 3/4 cup macaroni broken in inch pieces
\item 2 quarts boiling water
\item 1 tablespoon salt
\item 1 1/2 cups White Sauce
\end{itemize}

\vspace{-0.5em}
\noindent%
Cook as for Boiled Macaroni, and reheat in White Sauce.

\textbf{White Sauce.} Melt two tablespoons butter, add two tablespoons flour
with one-half teaspoon salt, and pour on slowly one and one-half cups
scalded milk.



\needspace{15\baselineskip}
\section*{Baked Macaroni}

Put Macaroni with White Sauce in buttered baking dish, cover with
buttered crumbs, and bake until crumbs are brown.



\needspace{15\baselineskip}
\section*{Baked Macaroni With Cheese}

Put a layer of boiled macaroni in buttered baking dish, sprinkle with
grated cheese; repeat, pour over White Sauce, cover with buttered
crumbs, and bake until crumbs are brown.



\needspace{15\baselineskip}
\section*{Macaroni With Tomato Sauce}

Reheat Boiled Macaroni in one and one-half cups of Tomato Sauce I,
sprinkle with grated cheese, and serve; or prepare as Baked Macaroni,
using Tomato in place of White Sauce.



\needspace{15\baselineskip}
\section*{Macaroni À L'Italienne}


\begin{minipage}{1.0\textwidth}
{\setlength{\multicolsep}{0pt}\setlength{\columnsep}{2em}\raggedcolumns%
\begin{multicols}{2}
\begin{itemize}
\setlength{\itemsep}{0pt}
\setlength{\parsep}{0pt}
\item 3/4 cup macaroni
\item 2 quarts boiling salted water
\item 1/2 onion
\item 2 cloves
\item 1 1/2 cups Tomato Sauce II
\item 1/2 cup grated cheese
\item 2 tablespoons wine
\item 1/2 tablespoon butter
\end{itemize}
\end{multicols}}
\end{minipage}

\vspace{0.3em}
\noindent%
Cook macaroni in boiling salted water, with butter and onion stuck with
cloves; drain, remove onion, reheat in Tomato Sauce, add cheese and
wine.



\needspace{15\baselineskip}
\section*{Macaroni, Italian Style}


\begin{minipage}{1.0\textwidth}
{\setlength{\multicolsep}{0pt}\setlength{\columnsep}{2em}\raggedcolumns%
\begin{multicols}{2}
\begin{itemize}
\setlength{\itemsep}{0pt}
\setlength{\parsep}{0pt}
\item 1 cup macaroni
\item 2 tablespoons butter
\item 2 tablespoons flour
\item 1 1/2 cups scalded milk
\item 2/3 cup grated cheese
\item Salt and paprika
\item 1/4 cup finely chopped cold boiled ham
\end{itemize}
\end{multicols}}
\end{minipage}

\vspace{0.3em}
\noindent%
Break macaroni in one-inch pieces and cook in boiling salted water,
drain, and reheat in sauce made of butter, flour, and milk, to which is
added cheese. As soon as cheese is melted, season with salt and paprika,
and turn on to a serving dish. Sprinkle with ham, and garnish with
parsley.



\needspace{15\baselineskip}
\section*{Macaroni À La Milanaise}

Cook macaroni as for Macaroni à l'Italienne, reheat in Tomato Sauce II,
add six sliced mushrooms, two slices cooked smoked beef tongue cut in
strips, and one-half cup grated cheese.



\needspace{15\baselineskip}
\section*{Spaghetti}

Spaghetti may be cooked in any way in which macaroni is cooked, but is
usually served with Tomato Sauce.

It is cooked in long strips rather than broken in pieces; to accomplish
this, hold quantity to be cooked in the hand, and dip ends in boiling
salted water; as spaghetti softens it will bend, and may be coiled under
water.



\needspace{15\baselineskip}
\section*{Knöfli}

Beat two eggs slightly and add one-fourth cup milk. Add gradually to one
cup flour mixed and sifted with one teaspoon salt. Place colander over a
kettle of boiling water, turn in one-third mixture, and force through
colander into water, using a potato masher. As soon as buttons come to
top of water, remove with skimmer to hot vegetable dish, and sprinkle
with salt and grated cheese; repeat until mixture is used. Let stand in
oven five minutes, then serve.



\needspace{15\baselineskip}
\section*{Ravioli}


\begin{minipage}{1.0\textwidth}
{\setlength{\multicolsep}{0pt}\setlength{\columnsep}{2em}\raggedcolumns%
\begin{multicols}{2}
\begin{itemize}
\setlength{\itemsep}{0pt}
\setlength{\parsep}{0pt}
\item 1 1/2 cups flour
\item 1/2 egg
\item Warm water
\item 1/4 cup cracker crumbs
\item 1/2 cup grated Parmesan cheese
\item 1/4 cup chopped cooked spinach
\item 1 egg
\item White stock
\item Salt
\item Pepper
\end{itemize}
\end{multicols}}
\end{minipage}

\vspace{0.3em}
\noindent%
Sift flour on a board, make depression in centre, drop in one-half egg,
and moisten with warm water to a stiff dough. Knead until smooth, cover,
and let stand ten minutes; then roll as thin as a sheet of paper, using
a rolling-pin. Cut in strips as long as paste, and two and three-fourth
inches wide, using a pastry jagger. Mix cracker crumbs, spinach, and
egg; moisten with stock and season with salt and pepper. Put mixture by
three-fourths teaspoon on lower half of strips of paste, two inches
apart. Fold upper part of paste over lower part. Press edges together
and between mixture with tips of thumbs, then cut apart, using pastry
jagger. Cook in White Stock ten minutes, take up with skimmer, arrange a
layer on hot serving dish, sprinkle generously with grated Parmesan
cheese, cover with Tomato Sauce; repeat twice and serve at once.



\needspace{15\baselineskip}
\section*{Tomato Sauce}


\begin{minipage}{1.0\textwidth}
{\setlength{\multicolsep}{0pt}\setlength{\columnsep}{2em}\raggedcolumns%
\begin{multicols}{2}
\begin{itemize}
\setlength{\itemsep}{0pt}
\setlength{\parsep}{0pt}
\item 1/3 cup butter
\item 1 onion, finely chopped
\item 3/4 teaspoon salt
\item Few grains pepper
\item 1 small can condensed tomato
\item 2/3 lb. lean beef
\end{itemize}
\end{multicols}}
\end{minipage}

\vspace{0.3em}
\noindent%
Cook first four ingredients eight minutes. Add tomato, 1 pint of water,
and beef cut in small pieces, and cook one and one-half hours. Remove
meat before serving. Ravioli is a national Italian dish, and the cheese
and condensed tomato may be best bought of an Italian grocer.





\chapter{Eggs}




\begin{center}
\begin{tabular}{|l|r|}
\hline
\multicolumn{2}{|l|}{\textbf{Composition}} \\
\hline
Protein & 14.9\% \\
\hline
Fat & 10.6\% \\
\hline
Mineral matter & 1\% \\
\hline
Water & 73.5\% \\
\hline
\end{tabular}
\end{center}

\vspace{10pt}

\noindent
Eggs, like milk, form a typical food, inasmuch as they contain all the
elements, in the right proportion, necessary for the support of the
body. Their highly concentrated, nutritive value renders it necessary to
use them in combination with other foods rich in starch (bread,
potatoes, etc.). In order that the stomach may have enough to act upon,
a certain amount of bulk must be furnished.

A pound of eggs (nine) is equivalent in nutritive value to a pound of
beef. From this it may be seen that eggs, at even twenty-five cents per
dozen, should not be freely used by the strict economist. Eggs being
rich in protein serve as a valuable substitute for meat. In most
families, their use in the making of cake, custard, puddings, etc.,
renders them almost indispensable. It is surprising how many intelligent
women, who look well to the affairs of the kitchen, are satisfied to use
what are termed “cooking eggs”; this shows poor judgment from an
economical standpoint. Strictly fresh eggs should always be used if
obtainable. An egg after the first twenty-four hours steadily
deteriorates. If exposed to air, owing to the porous structure of the
shell, there is an evaporation of water, air rushes in, and
decomposition takes place.

White of egg contains albumen in its purest form. Albumen coagulates at
a temperature of from 134deg to 160deg F. Herein lies the importance of
cooking eggs at a low temperature, thus rendering them easy of
digestion. Eggs cooked in boiling water are tough and horny, difficult
of digestion, and should never be served.

When eggs come from the market, they should be washed, and put away in a
cold place.

\textbf{Ways of Determining Freshness of Eggs.} 

I. Hold in front of candle flame in dark room, and the centre should look clear.

II. Place in basin of cold water, and they should sink.

III. Place large end to the cheek, and a warmth should be felt.

\textbf{Ways of Keeping Eggs.} 

I. Pack in sawdust, small end down.

II. Keep in lime water.

III. From July to September a large number of eggs are packed, small
ends down, in cases having compartments, one for each egg, and kept in
cold storage. Eggs are often kept in cold storage six months, and then
sold as cooking eggs.



\needspace{15\baselineskip}
\section*{Boiled Eggs}

Have ready a saucepan containing boiling water. Carefully put in with a
spoon the number of eggs desired, covering them with water. Remove
saucepan to back of range, where water will not boil. Cook from six to
eight minutes if liked “soft-boiled,” forty to forty-five if liked
“hard-boiled.” Eggs may be cooked by placing in cold water and allowing
water to heat gradually until the boiling-point is reached, when they
will be “soft-boiled.” In using hard-boiled eggs for making other
dishes, when taken from the hot water they should be plunged into cold
water to prevent, if possible, discoloration of yolks.

Eggs perfectly cooked should be placed and kept in water at a uniform
temperature of 175deg F.



\needspace{15\baselineskip}
\section*{Dropped Eggs (Poached)}

Have ready a shallow pan two-thirds full of boiling salted water,
allowing one-half tablespoon salt to one quart of water. Put two or
three buttered muffin rings in the water. Break each egg separately into
a cup, and carefully slip into a muffin ring. The water should cover the
eggs. When there is a film over the top, and the white is firm,
carefully remove with a buttered skimmer to circular pieces of buttered
toast, and let each person season his own egg with butter, salt, and
pepper. If cooked for an invalid, garnish with four toast points and a
bit of parsley. An egg-poacher may be used instead of muffin rings.



\needspace{15\baselineskip}
\section*{Eggs À La Finnoise}

Dropped Eggs, served with Tomato Sauce I.



\needspace{15\baselineskip}
\section*{Poached Eggs À La Reine}

Cover circular pieces of toasted bread with sliced fresh mushrooms
sautéd in butter and moistened with cream. Poach eggs and arrange on
mushrooms. Pour over all white sauce to which grated Parmesan cheese has
been added. Sprinkle with grated cheese and put in oven to brown.
Garnish with canned pimentoes cut in fancy shapes.



\needspace{15\baselineskip}
\section*{Eggs À La Suisse}


\begin{minipage}{1.0\textwidth}
{\setlength{\multicolsep}{0pt}\setlength{\columnsep}{2em}\raggedcolumns%
\begin{multicols}{2}
\begin{itemize}
\setlength{\itemsep}{0pt}
\setlength{\parsep}{0pt}
\item 4 eggs
\item 1/2 cup cream
\item 1 tablespoon butter
\item Salt
\item Pepper
\item Cayenne
\item 2 tablespoons grated cheese
\end{itemize}
\end{multicols}}
\end{minipage}

\vspace{0.3em}
\noindent%
Heat a small omelet pan, put in butter, and when melted, add cream. Slip
in the eggs one at a time, sprinkle with salt, pepper, and a few grains
of cayenne. When whites are nearly firm, sprinkle with cheese. Finish
cooking, and serve on buttered toast. Strain cream over the toast.



\needspace{15\baselineskip}
\section*{Eggs Susette}

Wash and bake six large potatoes, cut slice from top of each, scoop out
inside, and mash. To three cups mashed potato add six tablespoons finely
chopped ham, two tablespoons finely chopped parsley, whites of two eggs
well beaten, three tablespoons butter, four tablespoons cream, and salt
and pepper. Line potato shells with mixture, place in each cavity a
poached egg, cover with potato mixture, and bake until browned. Care
must be taken to have eggs delicately parched.



\needspace{15\baselineskip}
\section*{Baked Or Shirred Eggs}

Butter an egg-shirrer. Cover bottom and sides with fine cracker crumbs.
Break an egg into a cup, and carefully slip into shirrer. Cover with
seasoned buttered crumbs, and bake in moderate oven until white is firm
and crumbs brown. The shirrers should be placed on a tin plate, that
they may be easily removed from the oven.

Eggs may be baked in small tomatoes. Cut a slice from stem end of
tomato, scoop out the pulp, slip in an egg, sprinkle with salt and
pepper, cover with buttered crumbs, and bake.



\needspace{15\baselineskip}
\section*{Eggs À La Tripe}

Serve dropped eggs on Lobster Croquettes (see p. 558) shaped in flat
round cakes one-half inch thick. Garnish with lobster claws and parsley.



\needspace{15\baselineskip}
\section*{Eggs À La Benedict}

Split and toast English muffins. Sauté circular pieces of cold boiled
ham, place these over the halves of muffins, arrange on each a dropped
egg, and pour around Hollandaise Sauce II (see p. 274), diluted with
cream to make of such consistency to pour easily.



\needspace{15\baselineskip}
\section*{Eggs À La Lee}

Cover circular pieces of toasted bread with thin slices cold boiled ham.
Arrange on each a dropped egg, and pour around.

\textbf{Mushroom Purée.} Clean one-fourth pound mushrooms, break caps in
pieces, and sauté five minutes in one tablespoon butter. Add one cup
chicken stock and simmer five minutes. Rub through a sieve and thicken
with one tablespoon each butter and flour cooked together. Season with
salt and pepper.



\needspace{15\baselineskip}
\section*{Eggs À La Commodore}

Cut slices of bread in circular pieces and sauté in butter. Remove a
portion of centre, leaving a rim one-fourth inch wide. Spread cavity
thus made with pâté de foie gras purée, place a poached egg in each and
pour over a rich brown or Béchamel sauce to which is added a few drops
vinegar. Garnish with chopped truffles.



\needspace{15\baselineskip}
\section*{Eggs, Waldorf Style}

Arrange poached eggs on circular pieces of buttered toast, surround with
Brown Mushroom Sauce (see p. 268), and place a broiled mushroom cap on
each egg.



\needspace{15\baselineskip}
\section*{Poached Eggs With Sauce Bearnaise}

Poach six eggs, arrange in serving dish, cover eggs alternately with red
and yellow sauce, and garnish with parsley.

\textbf{Sauce Bearnaise.} Beat yolks three eggs slightly, add three tablespoons
olive oil, two tablespoons hot water, three-fourths tablespoon tarragon
vinegar, one-fourth teaspoon salt, and a few grains cayenne. Cook over
boiling water until mixture thickens. Color one-half the sauce with
tomato purée (tomatoes drained from their liquor, stewed, strained, and
cooked until reduced to a thick pulp).



\needspace{15\baselineskip}
\section*{Scrambled Eggs}


\begin{itemize}
\setlength{\itemsep}{0pt}
\setlength{\parsep}{0pt}
\item 5 eggs
\item 1/2 cup milk
\item 1/2 teaspoon salt
\item 1/8 teaspoon pepper
\item 2 tablespoons butter
\end{itemize}

\vspace{-0.5em}
\noindent%
Beat eggs slightly with silver fork; add salt, pepper, and milk. Heat
omelet pan, put in butter, and when melted, turn in the mixture. Cook
until of creamy consistency, stirring and scraping from bottom of the
pan.



\needspace{15\baselineskip}
\section*{Scrambled Eggs With Tomato Sauce}


\begin{minipage}{1.0\textwidth}
{\setlength{\multicolsep}{0pt}\setlength{\columnsep}{2em}\raggedcolumns%
\begin{multicols}{2}
\begin{itemize}
\setlength{\itemsep}{0pt}
\setlength{\parsep}{0pt}
\item 6 eggs
\item 1 3/4 cups tomatoes
\item 2 teaspoons sugar
\item 4 tablespoons butter
\item 1 slice onion
\item 1/2 teaspoon salt
\item 1/8 teaspoon pepper
\end{itemize}
\end{multicols}}
\end{minipage}

\vspace{0.3em}
\noindent%
Simmer tomatoes and sugar five minutes; fry butter and onion three
minutes; remove onion, and add tomatoes, seasonings, and eggs slightly
beaten. Cook same as Scrambled Eggs. Serve with entire wheat bread or
brown bread toast.



\needspace{15\baselineskip}
\section*{Scrambled Eggs With Anchovy Toast}

Spread thin slices of buttered toast with Anchovy Paste. Arrange on
platter, and cover with scrambled eggs.



\needspace{15\baselineskip}
\section*{Eggs À La Buckingham}

Make five slices milk toast, and arrange on platter. Use recipe for
Scrambled Eggs, having the eggs slightly underdone. Pour eggs over
toast, sprinkle with four tablespoons grated mild cheese. Put in oven to
melt cheese, and finish cooking eggs.



\needspace{15\baselineskip}
\section*{Eggs À La Turk}

Prepare Scrambled Eggs, and pour over six slices of toasted bread. Put
one tablespoon tomato purée on each piece, and in the centre of purée
one-half tablespoon chickens' livers sautéd in bacon fat.



\needspace{15\baselineskip}
\section*{Eggs À La Livingstone}


\begin{minipage}{1.0\textwidth}
{\setlength{\multicolsep}{0pt}\setlength{\columnsep}{2em}\raggedcolumns%
\begin{multicols}{2}
\begin{itemize}
\setlength{\itemsep}{0pt}
\setlength{\parsep}{0pt}
\item 4 eggs
\item 1/2 cup stewed and strained tomatoes
\item 1/2 teaspoon salt
\item 1/4 teaspoon paprika
\item 2 tablespoons butter
\item Pâté de foie gras
\item Finely chopped truffles
\end{itemize}
\end{multicols}}
\end{minipage}

\vspace{0.3em}
\noindent%
Beat eggs slightly, and add tomatoes, salt, and paprika. Melt butter in
an omelet pan, add seasoned eggs, and cook same as Scrambled Eggs.
Spread slices of toasted bread with pâté de foie gras. Pour over the
eggs, and sprinkle with truffles.



\needspace{15\baselineskip}
\section*{Scrambled Eggs, Country Style}

Heat omelet pan, put in two tablespoons butter, and when melted turn in
four unbeaten eggs. Cook until white is partially set, then stir until
cooking is completed, when whites will be thoroughly set. Season with
salt and pepper.



\needspace{15\baselineskip}
\section*{Buttered Eggs}

Heat omelet pan. Put in one tablespoon butter; when melted, slip in an
egg, and cook until the white is firm. Turn it over once while cooking.
Add more butter as needed, using just enough to keep egg from sticking.



\needspace{15\baselineskip}
\section*{Buttered Eggs With Tomatoes}

Cut tomatoes in one-third inch slices. Sprinkle with salt and pepper,
dredge with flour, and sauté in butter. Serve a buttered egg on each
slice of tomato.



\needspace{15\baselineskip}
\section*{Planked Eggs}

Finely chop cold cooked corned beef or corned tongue; there should be
two-thirds cup. Add an equal quantity of fine bread crumbs, moisten with
cream and season with salt and pepper. Spread mixture on plank, and make
nests and border of duchess potatoes, using rose tube. Put a buttered or
poached egg in each nest and put in oven to brown potato. Garnish with
tomatoes cut in halves and broiled, and parsley. Eggs may be sprinkled
with buttered cracker crumbs, just before sending to oven, if preferred.



\needspace{15\baselineskip}
\section*{Fried Eggs}

Fried eggs are cooked as Buttered Eggs, without being turned. In this
case the fat is taken by spoonfuls and poured over the eggs. Lard, pork,
ham, or bacon fat are usually employed,--a considerable amount being
used.



\needspace{15\baselineskip}
\section*{Eggs À La Goldenrod}


\begin{minipage}{1.0\textwidth}
{\setlength{\multicolsep}{0pt}\setlength{\columnsep}{2em}\raggedcolumns%
\begin{multicols}{2}
\begin{itemize}
\setlength{\itemsep}{0pt}
\setlength{\parsep}{0pt}
\item 3 “hard-boiled” eggs
\item 1 tablespoon butter
\item 1 tablespoon flour
\item 1 cup milk
\item 1/2 teaspoon salt
\item 1/8 teaspoon pepper
\item 5 slices toast
\item Parsley
\end{itemize}
\end{multicols}}
\end{minipage}

\vspace{0.3em}
\noindent%
Make a thin white sauce with butter, flour, milk, and seasonings.
Separate yolks from whites of eggs. Chop whites finely, and add them to
the sauce. Cut four slices of toast in halves lengthwise. Arrange on
platter, and pour over the sauce. Force the yolks through a potato ricer
or strainer, sprinkling over the top. Garnish with parsley and remaining
toast, cut in points.



\needspace{15\baselineskip}
\section*{Eggs Au Gratin}

Arrange Dropped Eggs on a shallow buttered dish. Sprinkle with grated
Parmesan cheese. Pour over eggs one pint Yellow Béchamel Sauce. Cover
with stale bread crumbs, and sprinkle with grated cheese. Brown in oven.
Tomato or White Sauce may be used.



\needspace{15\baselineskip}
\section*{Eggs In Batter}


\begin{itemize}
\setlength{\itemsep}{0pt}
\setlength{\parsep}{0pt}
\item 1 egg
\item 1 1/2 tablespoons thick cream
\item 2 tablespoons fine stale bread crumbs
\item 1/4 teaspoon salt
\end{itemize}

\vspace{-0.5em}
\noindent%
Mix cream, bread crumbs, and salt. Put one-half tablespoon of mixture in
egg-shirrer. Slip in egg, and cover with remaining mixture. Bake six
minutes in moderate oven.



\needspace{15\baselineskip}
\section*{Curried Eggs I}


\begin{minipage}{1.0\textwidth}
{\setlength{\multicolsep}{0pt}\setlength{\columnsep}{2em}\raggedcolumns%
\begin{multicols}{2}
\begin{itemize}
\setlength{\itemsep}{0pt}
\setlength{\parsep}{0pt}
\item 3 “hard-boiled” eggs
\item 2 tablespoons butter
\item 2 tablespoons flour
\item 1/4 teaspoon salt
\item 1/2 teaspoon curry powder
\item 1/8 teaspoon pepper
\item 1 cup hot milk
\end{itemize}
\end{multicols}}
\end{minipage}

\vspace{0.3em}
\noindent%
Melt butter, add flour and seasonings, and gradually hot milk. Cut eggs
in eighths lengthwise, and reheat in sauce.



\needspace{15\baselineskip}
\section*{Curried Eggs II}


\begin{minipage}{1.0\textwidth}
{\setlength{\multicolsep}{0pt}\setlength{\columnsep}{2em}\raggedcolumns%
\begin{multicols}{2}
\begin{itemize}
\setlength{\itemsep}{0pt}
\setlength{\parsep}{0pt}
\item 4 “hard-boiled” eggs
\item 2 tablespoons butter
\item 1/2 tablespoon finely chopped onion
\item 2 tablespoons flour
\item 1 teaspoon curry powder
\item 1/2 teaspoon salt
\item 1/8 teaspoon paprika
\item 1 1/3 cups scalded milk
\item 1/2 cup cooked rice
\end{itemize}
\end{multicols}}
\end{minipage}

\vspace{0.3em}
\noindent%
Chop whites of eggs and add to sauce made of butter, flour, seasonings,
and milk, then add rice; heat to boiling-point, fill puff paste cases
and sprinkle with yolks of eggs rubbed through a sieve.



\needspace{15\baselineskip}
\section*{Scalloped Eggs}


\begin{itemize}
\setlength{\itemsep}{0pt}
\setlength{\parsep}{0pt}
\item 3 “hard-boiled” eggs
\item 1 pint White Sauce I
\item 3/4 cup chopped cold meat
\item 3/4 cup buttered cracker crumbs
\end{itemize}

\vspace{-0.5em}
\noindent%
Chop eggs finely. Sprinkle bottom of a buttered baking dish with crumbs,
cover with one-half the eggs, eggs with sauce, and sauce with meat;
repeat. Cover with remaining crumbs. Place in oven on centre grate, and
bake until crumbs are brown. Ham is the best meat to use for this dish.
Chicken, veal, or fish may be used.



\needspace{15\baselineskip}
\section*{Stuffed Eggs}

Cut four “hard-boiled” eggs in halves crosswise; remove yolks, mash, and
add two tablespoons grated cheese, one teaspoon vinegar, one-fourth
teaspoon mustard, and salt and cayenne to taste. Add enough melted
butter to make mixture of the right consistency to shape. Make in balls
size of original yolks, and refill whites. Arrange on a serving dish,
pour around one cup White Sauce, cover, and reheat.



\needspace{15\baselineskip}
\section*{Stuffed Eggs In A Nest}

Cut “hard-boiled” eggs in halves lengthwise. Remove yolks, and put
whites aside in pairs. Mash yolks, and add half the amount of devilled
ham and enough melted butter to make of consistency to shape. Make in
balls size of original yolks, and refill whites. Form remainder of
mixture into a nest. Arrange eggs in the nest, and pour over one cup
White Sauce I. Sprinkle with buttered crumbs, and bake until crumbs are
brown.



\needspace{15\baselineskip}
\section*{Eggs À La Sidney}

Arrange “hard-boiled” eggs, cut in thirds lengthwise, on pieces of
toasted bread. Pour over eggs Soubise Sauce.



\needspace{15\baselineskip}
\section*{Eggs Huntington}


\begin{minipage}{1.0\textwidth}
{\setlength{\multicolsep}{0pt}\setlength{\columnsep}{2em}\raggedcolumns%
\begin{multicols}{2}
\begin{itemize}
\setlength{\itemsep}{0pt}
\setlength{\parsep}{0pt}
\item 4 “hard-boiled” eggs
\item 1 tablespoon butter
\item 1 1/2 tablespoons flour
\item 1/3 cup white stock
\item 1/3 cup milk
\item 1/2 teaspoon salt
\item Few grains cayenne
\item Grated cheese
\item 3/4 cup buttered cracker crumbs
\end{itemize}
\end{multicols}}
\end{minipage}

\vspace{0.3em}
\noindent%
Make a sauce of the butter, flour, stock, and milk; add eggs finely
chopped and salt and cayenne. Fill buttered ramequin dishes with
mixture, sprinkle with grated cheese, cover with cracker crumbs, and
bake in a moderate oven until crumbs are brown.



\needspace{15\baselineskip}
\section*{Egg Farci I}

Cut “hard-boiled” eggs in halves, crosswise. Remove yolks, and put
whites aside in pairs. Mash yolks, and add equal amount of cold cooked
chicken or veal, finely chopped. Moisten with melted butter or
Mayonnaise. Season to taste with salt, pepper, lemon juice, mustard, and
cayenne. Shape and refill whites.



\needspace{15\baselineskip}
\section*{Egg Farci II}

Clean and chop two chickens' livers, sprinkle with onion juice, and
sauté in butter. Add the yolks of four “hard-boiled” eggs rubbed through
a sieve, one teaspoon chopped parsley, and salt, pepper, and Tabasco
Sauce to taste. Refill whites of eggs with mixture, cover with grated
cheese, and bake until cheese melts. Serve in toast rings and pour
around Tomato Purée (see p. 98).



\needspace{15\baselineskip}
\section*{Lucanian Eggs}


\begin{minipage}{1.0\textwidth}
{\setlength{\multicolsep}{0pt}\setlength{\columnsep}{2em}\raggedcolumns%
\begin{multicols}{2}
\begin{itemize}
\setlength{\itemsep}{0pt}
\setlength{\parsep}{0pt}
\item 5 “hard-boiled” eggs
\item 1 cup cooked macaroni
\item 1/2 cup grated cheese
\item Essence Anchovy
\item 1 3/4 cups White Sauce I
\item Salt and paprika
\item Onion juice
\item 3/4 cup buttered crumbs
\end{itemize}
\end{multicols}}
\end{minipage}

\vspace{0.3em}
\noindent%
Cut eggs in eighths lengthwise, add macaroni, white sauce, and
seasonings. Arrange in buttered baking dish, cover with buttered crumbs,
and bake until crumbs are brown.



\needspace{15\baselineskip}
\section*{Egg Soufflé}


\begin{minipage}{1.0\textwidth}
{\setlength{\multicolsep}{0pt}\setlength{\columnsep}{2em}\raggedcolumns%
\begin{multicols}{2}
\begin{itemize}
\setlength{\itemsep}{0pt}
\setlength{\parsep}{0pt}
\item 2 tablespoons butter
\item 2 tablespoons flour
\item 1 cup milk
\item 1 cup cream
\item 4 eggs
\item 1 teaspoon salt
\item Few grains cayenne
\end{itemize}
\end{multicols}}
\end{minipage}

\vspace{0.3em}
\noindent%
Cream the butter, add flour, and pour on gradually scalded milk and
cream. Cook in double boiler five minutes, and add yolks of eggs, beaten
until thick and lemon-colored. Remove from fire, add seasonings, and
fold in whites of eggs beaten until stiff and dry. Turn into a buttered
dish, or buttered individual moulds, set in pan of hot water, and bake
in a slow oven until firm. Egg Soufflé may be served with White Sauce I,
highly seasoned with celery salt, paprika, and onion juice.



\needspace{15\baselineskip}
\section*{Egg Timbales}


\begin{minipage}{1.0\textwidth}
{\setlength{\multicolsep}{0pt}\setlength{\columnsep}{2em}\raggedcolumns%
\begin{multicols}{2}
\begin{itemize}
\setlength{\itemsep}{0pt}
\setlength{\parsep}{0pt}
\item 1 tablespoon butter
\item 1 tablespoon flour
\item 2/3 cup milk
\item 3 eggs
\item 1 tablespoon chopped parsley
\item 1/2 teaspoon salt
\item 1/8 teaspoon pepper
\item Few grains celery salt
\item Few grains cayenne
\end{itemize}
\end{multicols}}
\end{minipage}

\vspace{0.3em}
\noindent%
Make a sauce of the butter, flour, and milk; add yolks beaten until
thick and lemon-colored, then add seasonings. Beat whites of eggs until
stiff and dry, and cut and fold into first mixture. Turn into buttered
moulds, set in pan of hot water, and bake in a slow oven until firm.
Serve with Tomato Cream Sauce (see page 271).



\needspace{15\baselineskip}
\section*{Egg Croquettes}


\begin{minipage}{1.0\textwidth}
{\setlength{\multicolsep}{0pt}\setlength{\columnsep}{2em}\raggedcolumns%
\begin{multicols}{2}
\begin{itemize}
\setlength{\itemsep}{0pt}
\setlength{\parsep}{0pt}
\item 6 eggs
\item 2 tablespoons butter
\item 1 slice onion
\item 1/3 cup flour
\item 1 cup white stock
\item Salt
\item Pepper
\item 3 egg yolks
\item Stale bread crumbs
\item Grated cheese
\end{itemize}
\end{multicols}}
\end{minipage}

\vspace{0.3em}
\noindent%
Poach eggs and dry on a towel. Cook butter with onion three minutes. Add
flour and, gradually, stock. Season with salt and pepper; then add yolks
of eggs slightly beaten. Cook one minute, and cool. Cover eggs with
mixture, roll in bread crumbs and cheese, using equal parts, dip in egg,
again roll in crumbs, fry in deep fat, and drain on brown paper. These
may be served with a thin sauce, using equal parts of white stock and
cream, and seasoning with grated cheese, salt, and paprika.



\needspace{15\baselineskip}
\section*{Eggs À La Juliette}

Decorate egg-shaped individual moulds with truffles, and cold boiled
tongue cut in fancy shapes, and pistachio nuts blanched and split. Line
mould with aspic jelly, drop in a poached egg yolk, cover with aspic
jelly, let stand until firm, and turn on a thin oval slice of cold
boiled tongue.



\needspace{15\baselineskip}
\section*{Eggs À La Parisienne}

Butter small timbale moulds, sprinkle with finely chopped truffles,
parsley, and cooked beets. Break eggs, and slip one into each mould,
sprinkle with salt and pepper, set in pan of hot water, and cook until
egg is firm. Remove from moulds on octagon slices of toast, and pour
around Tomato Sauce II (see p. 270).






\needspace{15\baselineskip}
\section*{Eggs Mornay}

Break egg and slip into buttered egg-shirrers, allowing one or two eggs
to each shirrer, according to size. Cover with White Sauce II (see p.
266), seasoned with one-third cup grated cheese, paprika, and yolks two
eggs; cover with grated cheese and bake until firm.



\needspace{15\baselineskip}
\section*{Omelets}

For omelets select large eggs, allowing one egg for each person, and one
tablespoon liquid for each egg. Keep an omelet pan especially for
omelets, and see that it is kept clean and smooth. A frying-pan may be
used in place of omelet pan.



\needspace{15\baselineskip}
\section*{Plain Omelet}


\begin{minipage}{1.0\textwidth}
{\setlength{\multicolsep}{0pt}\setlength{\columnsep}{2em}\raggedcolumns%
\begin{multicols}{2}
\begin{itemize}
\setlength{\itemsep}{0pt}
\setlength{\parsep}{0pt}
\item 4 eggs
\item 1/2 teaspoon salt
\item Few grains pepper
\item 4 tablespoons hot water
\item 1 tablespoon butter
\item 1 1/2 cups Thin White Sauce
\end{itemize}
\end{multicols}}
\end{minipage}

\vspace{0.3em}
\noindent%
Separate yolks from whites. Beat yolks until thick and lemon-colored;
add salt, pepper, and hot water. Beat whites until stiff and dry,
cutting and folding them into first mixture until they have taken up
mixture. Heat omelet pan, and butter sides and bottom. Turn in mixture,
spread evenly, place on range where it will cook slowly, occasionally
turning the pan that omelet may brown evenly. When well “puffed” and
delicately browned underneath, place pan on centre grate of oven to
finish cooking the top. The omelet is cooked if it is firm to the touch
when pressed by the finger. If it clings to the finger like the beaten
white of egg, it needs longer cooking. Fold, and turn on hot platter,
and pour around one and one-half cups Thin White Sauce.

Milk is sometimes used in place of hot water, but hot water makes a more
tender omelet. A few grains baking powder are used by some cooks to hold
up an omelet.



\needspace{15\baselineskip}
\section*{To Fold And Turn An Omelet}

Hold an omelet pan by handle with the left hand. With a case knife make
two one-half inch incisions opposite each other at right angles to
handle. Place knife under the part of omelet nearest handle, tip pan to
nearly a vertical position; by carefully coaxing the omelet with knife,
it will fold and turn without breaking.



\needspace{15\baselineskip}
\section*{Omelet With Meat Or Vegetables}

Mix and cook Plain Omelet. Fold in remnants of finely chopped cooked
chicken, veal, or ham. Remnants of fish may be flaked and added to White
Sauce; or cooked peas, asparagus, or cauliflower may be added.



\needspace{15\baselineskip}
\section*{Oyster Omelet}

Mix and cook Plain Omelet. Fold in one pint oysters, parboiled, drained
from their liquor, and cut in halves. Turn on platter, and pour around
Thin White Sauce.



\needspace{15\baselineskip}
\section*{Orange Omelet}


\begin{minipage}{1.0\textwidth}
{\setlength{\multicolsep}{0pt}\setlength{\columnsep}{2em}\raggedcolumns%
\begin{multicols}{2}
\begin{itemize}
\setlength{\itemsep}{0pt}
\setlength{\parsep}{0pt}
\item 3 eggs
\item 2 tablespoons powdered sugar
\item Few grains salt
\item 1 teaspoon lemon juice
\item 2 oranges
\item 1/2 tablespoon butter
\item 2 1/2 tablespoons orange juice
\end{itemize}
\end{multicols}}
\end{minipage}

\vspace{0.3em}
\noindent%
Follow directions for Plain Omelet. Remove skin from oranges and cut in
slices, lengthwise. Fold in one-third of the slices of orange, well
sprinkled with powdered sugar; put remaining slices around omelet, and
sprinkle with sugar.



\needspace{15\baselineskip}
\section*{Jelly Omelet}

Mix and cook Plain Omelet, omitting pepper and one-half the salt, and
adding one tablespoon sugar. Spread before folding with jam, jelly, or
marmalade. Fold, turn, and sprinkle with sugar.



\needspace{15\baselineskip}
\section*{Bread Omelet}


\begin{minipage}{1.0\textwidth}
{\setlength{\multicolsep}{0pt}\setlength{\columnsep}{2em}\raggedcolumns%
\begin{multicols}{2}
\begin{itemize}
\setlength{\itemsep}{0pt}
\setlength{\parsep}{0pt}
\item 4 eggs
\item 1/2 cup milk
\item 1/2 cup stale bread crumbs
\item 3/4 teaspoon salt
\item 1/8 teaspoon pepper
\item 1 tablespoon butter
\end{itemize}
\end{multicols}}
\end{minipage}

\vspace{0.3em}
\noindent%
Soak bread crumbs fifteen minutes in milk, add beaten yolks and
seasonings, fold in whites. Cook and serve as Plain Omelet.



\needspace{15\baselineskip}
\section*{French Omelet}


\begin{itemize}
\setlength{\itemsep}{0pt}
\setlength{\parsep}{0pt}
\item 4 eggs
\item 4 tablespoons milk
\item 1/2 teaspoon salt
\item 1/2 teaspoon pepper
\item 2 tablespoons butter
\end{itemize}

\vspace{-0.5em}
\noindent%
Beat eggs slightly, just enough to blend yolks and whites, add the milk
and seasonings. Put butter in hot omelet pan; when melted, turn in the
mixture; as it cooks, prick and pick up with a fork until the whole is
of creamy consistency. Place on hotter part of range that it may brown
quickly underneath. Fold, and turn on hot platter.



\needspace{15\baselineskip}
\section*{Omelet With Croûtons}


\begin{minipage}{1.0\textwidth}
{\setlength{\multicolsep}{0pt}\setlength{\columnsep}{2em}\raggedcolumns%
\begin{multicols}{2}
\begin{itemize}
\setlength{\itemsep}{0pt}
\setlength{\parsep}{0pt}
\item 1 cup bread cut in 1/3 inch cubes
\item Butter
\item 5 eggs
\item 4 tablespoons cream
\item 1/2 teaspoon salt
\item 1/8 teaspoon pepper
\end{itemize}
\end{multicols}}
\end{minipage}

\vspace{0.3em}
\noindent%
Fry cubes of bread in butter until well browned and crisp. Beat eggs
slightly, add cream, salt, pepper, and croûtons. Put two tablespoons
butter in hot omelet pan, and as soon as melted and slightly browned
turn in mixture and cook same as French Omelet.



\needspace{15\baselineskip}
\section*{Eggs With Spinach À La Martin}

Cover the centre of a platter with finely chopped and seasoned cooked
spinach. Beat three eggs slightly, add three tablespoons hot water,
one-third teaspoon salt, one tablespoon, each, red and green pepper cut
in strips, and one tablespoon cooked ham cut in very small pieces. Heat
omelet pan, put in one and one-half tablespoons olive oil, and as soon
as heated pour in mixture. Cook same as French Omelet and turn on to
spinach. Garnish with parsley.



\needspace{15\baselineskip}
\section*{Spanish Omelet}

Mix and cook a French Omelet. Serve with Tomato Sauce in the centre and
around omelet.

\textbf{Tomato Sauce.} Cook two tablespoons of butter with one tablespoon of
finely chopped onion, until yellow. Add one and three-fourths cups
tomatoes, and cook until moisture has nearly evaporated. Add one
tablespoon sliced mushrooms, one tablespoon capers, one-fourth teaspoon
salt, and a few grains cayenne. This is improved by a small piece of red
or green pepper, finely chopped, cooked with butter and onion.



\needspace{15\baselineskip}
\section*{Rich Omelet}


\begin{itemize}
\setlength{\itemsep}{0pt}
\setlength{\parsep}{0pt}
\item 2 1/2 tablespoons flour
\item 3/4 teaspoon salt
\item 1 cup milk
\item 3 eggs
\item 3 tablespoons butter
\end{itemize}

\vspace{-0.5em}
\noindent%
Mix salt and flour, and add gradually milk. Beat eggs until thick and
lemon-colored, then add to first mixture. Heat iron frying-pan and put
in two-thirds of the butter; when butter is melted, pour in mixture. As
it cooks, lift with a griddle-cake turner so that uncooked part may run
underneath; add remaining butter as needed, and continue lifting the
cooked part until it is firm throughout. Place on hotter part of range
to brown; roll, and turn on hot platter.



\needspace{15\baselineskip}
\section*{Omelette Robespierre}


\begin{itemize}
\setlength{\itemsep}{0pt}
\setlength{\parsep}{0pt}
\item 3 eggs
\item 3 tablespoons hot water
\item 1 tablespoon powdered sugar
\item 1/8 teaspoon salt
\item 1/2 teaspoon vanilla
\end{itemize}

\vspace{-0.5em}
\noindent%
Beat eggs slightly, and add remaining ingredients. Put one and one-half
tablespoons butter in a hot omelet pan, turn in mixture and cook same as
French Omelet. Fold, turn on a hot platter, sprinkle with powdered
sugar, and score with a hot poker.



\needspace{15\baselineskip}
\section*{Almond Omelet, Caramel Sauce}


\begin{itemize}
\setlength{\itemsep}{0pt}
\setlength{\parsep}{0pt}
\item 3 eggs
\item 3 tablespoons caramel sauce
\item Few grains salt
\item 1/2 teaspoon vanilla
\end{itemize}

\vspace{-0.5em}
\noindent%
Beat yolks of eggs until thick and lemon-colored, add caramel, salt, and
vanilla, and cut and fold in whites of eggs beaten until stiff and dry.
Put three-fourths tablespoon butter in a hot omelet pan, cover bottom of
pan with shredded almonds, turn in mixture, and cook and fold same as
Plain Omelet. Pour around

\textbf{Caramel Sauce.} Pour one cup sugar in omelet pan, and stir constantly,
over hot part of range, until melted to a light brown syrup. Add
three-fourths cup hot water, and let simmer ten minutes.





\chapter{Soups}



It cannot be denied that the French excel all nations in the excellence
of their cuisine, and to their soups and sauces belong the greatest
praise. It would be well to follow their example, and it is the duty of
every housekeeper to learn the art of soup making. How may a hearty
dinner be better begun than with a thin soup? The hot liquid, taken into
an empty stomach, is easily assimilated, acts as a stimulant rather than
a nutrient (as is the popular opinion), and prepares the way for the
meal which is to follow. The cream soups and purées are so nutritious
that, with bread and butter, they furnish a satisfactory meal.

Soups are divided into two great classes: soups with stock; soups
without stock.

Soups with stock have, for their basis, beef, veal, mutton, fish,
poultry, or game, separately or in combination. They are classified as:

\textbf{Bouillon}, made from lean beef, delicately seasoned, and usually
cleared. Exception,--clam bouillon.

\textbf{Brown Soup Stock}, made from beef (two-thirds lean meat, and remainder
bone and fat), highly seasoned with vegetables, spices, and sweet herbs.

\textbf{White Soup Stock}, made from chicken or veal, with delicate seasonings.

\textbf{Consommé}, usually made from two or three kinds of meat (beef, veal,
and fowl being employed), highly seasoned with vegetables, spices, and
sweet herbs. Always served clear.

\textbf{Lamb Stock}, delicately seasoned, is served as mutton broth.

Soups without stock are classified as:

\textbf{Cream Soups}, made of vegetables or fish, with milk, and a small amount
of cream and seasonings. Always thickened.

\textbf{Purées}, made from vegetables or fish, forced through a strainer, and
retained in soup, milk, and seasonings. Generally thicker than cream
soup. Sometimes White Stock is added.

\textbf{Bisques}, generally made from shell-fish, milk, and seasonings, and
served with fish dice; made similarly to purées. They may be made of
meat, game, or vegetables, with small dice of the same.

Various names have been given to soups, according to their flavorings,
chief ingredients, the people who use them, etc. To the Scotch belongs
Scotch Broth; to the French, Pot-au-feu; to the Indo, Mulligatawny; and
to the Spanish, Olla Podrida.



\needspace{15\baselineskip}
\section*{Soup Making}

The art of soup making is more easily mastered than at first appears.
The young housekeeper is startled at the amazingly large number of
ingredients the recipe calls for, and often is discouraged. One may,
with but little expense, keep at hand what is essential for the making
of a good soup. Winter vegetables--turnips, carrots, celery, and
onions--may be bought in large or small quantities. The outer stalks of
celery, often not suitable for serving, should be saved for soups. At
seasons when celery is a luxury, the tips and roots should be saved and
dried. Sweet herbs, including thyme, savory, and marjoram, are dried and
put up in packages, retailing from five to ten cents. Bay leaves, which
should be used sparingly, may be obtained at first-class grocers' or
druggists'; seeming never to lose strength, they may be kept
indefinitely. Spices, including whole cloves, allspice berries,
peppercorns, and stick cinnamon, should be kept on hand. These
seasonings, with the addition of salt, pepper, and parsley, are the
essential flavorings for stock soups. Flour, corn-starch, arrowroot,
fine tapioca, sago, pearl barley, rice, bread, or eggs are added to give
consistency and nourishment.

In small families, where there are few left-overs, fresh meat must be
bought for the making of soup stock, as a good soup cannot be made from
a small amount of poor material. On the other hand, large families need
seldom buy fresh meat, provided all left-overs are properly cared for.
The soup kettle should receive small pieces of beef (roasted, broiled,
or stewed), veal, carcasses of fowl or chicken, chop bones, bones left
from lamb roast, and all trimmings and bones, which a careful housewife
should see are sent from the market with her order. Avoid the use of
smoked or corned meats, or large pieces of raw mutton or lamb surrounded
by fat, on account of the strong flavor so disagreeable to many. A small
piece of bacon or lean ham is sometimes cooked with vegetables for
flavor.

Beef ranks first as regards utility and economy in soup making. It
should be cut from the fore or hind shin (which cuts contain
marrow-bone), the middle cuts being most desirable. If the lower part of
shin is used, the soup, although rich in gelatin, lacks flavor, unless a
cheap piece of lean meat is used with it, which frequently is done. It
must be remembered that meat, bone, and fat in the right proportions are
all necessary; allow two-thirds lean meat, the remaining one-third bone
and fat. From the meat the soluble juices, salts, extractives (which
give color and flavor), and a small quantity of gelatin are extracted;
from the bone, gelatin (which gives the stock when cold a jelly-like
consistency) and mineral matter. Gelatin is also obtained from
cartilage, skin, tendons, and ligaments. Some of the fat is absorbed;
the remainder rises to the top and should be removed.

Soup stock making is rendered easier by use of proper utensils. Sharp
meat knives, hardwood board, two purée strainers having meshes of
different size, and a soup digester (a porcelain-lined iron pot, having
tight-fitting cover, with valve in the top), or covered granite kettle,
are essentials. An iron kettle, which formerly constituted one of the
furnishings of a range, may be used if perfectly smooth. A saw, cleaver,
and scales, although not necessary, are useful, and lighten labor.

When meat comes from market, remove from paper and put in cool place.
When ready to start stock, if scales are at hand, weigh meat and bone to
see if correct proportions have been sent. Wipe meat with clean
cheese-cloth wrung out of cold water. Cut lean meat in one-inch cubes;
by so doing, a large amount of surface is exposed to the water, and
juices are more easily drawn out. Heat frying-pan hissing hot; remove
marrow from marrow-bone, and use enough to brown one-third of the lean
meat, stirring constantly, that all parts of surface may be seared, thus
preventing escape of juices,--sacrificing a certain amount of goodness in
the stock to give additional color and flavor, which is obtained by
caramelization. Put fat, bone, and remaining lean meat in soup kettle;
cover with cold water, allowing one pint to each pound of meat, bone,
and fat. Let stand one hour, that cold water may draw out juices from
meat. Add browned meat, taking water from soup kettle to rinse out
frying-pan, that none of the coloring may be lost. Heat gradually to
boiling-point, and cook six or seven hours at low temperature. A scum
will rise on the top, which contains coagulated albuminous juices; these
give to soup its chief nutritive value; many, however, prefer a clear
soup, and have them removed. If allowed to remain, when straining, a
large part will pass through strainer. Vegetables, spices, and salt
should be added the last hour of cooking. Strain and cool quickly; by so
doing, stock is less apt to ferment. A knuckle of veal is often used for
making white soup stock. Fowl should be used for stock in preference to
chicken, as it is cheaper, and contains a larger amount of nutriment. A
cake of fat forms on stock when cold, which excludes air, and should not
be removed until stock is used. To remove fat, run a knife around edge
of bowl and carefully remove the same. A small quantity will remain,
which should be removed by passing a cloth wrung out of hot water around
edge and over top of stock. This fat should be clarified and used for
drippings. If time cannot be allowed for stock to cool before using,
take off as much fat as possible with a spoon, and remove the remainder
by passing tissue or any absorbent paper over the surface.



\needspace{15\baselineskip}
\subsection*{How to Clear Soup Stock}

Whites of eggs slightly beaten, or raw, lean beef finely chopped, are
employed for clearing soup stock. The albumen found in each effects the
clearing by drawing to itself some of the juices which have been
extracted from the meat, and by action of heat have been coagulated.
Some rise to the top and form a scum, others are precipitated.

Remove fat from stock, and put quantity to be cleared in stewpan,
allowing white and shell of one egg to each quart of stock. Beat egg
slightly, break shell in small pieces and add to stock. Place on front
of range, and stir constantly until boiling-point is reached; boil two
minutes. Set back where it may simmer twenty minutes; remove scum, and
strain through double thickness of cheese-cloth placed over a fine
strainer. If stock to be cleared is not sufficiently seasoned,
additional seasoning must be added as soon as stock has lost its
jelly-like consistency; not after clearing is effected. Many think the
flavor obtained from a few shavings of lemon rind an agreeable addition.



\needspace{15\baselineskip}
\subsection*{How to Bind Soups}

Cream soups and purées, if allowed to stand, separate, unless bound
together. To bind a soup, melt butter, and when bubbling add an equal
quantity of flour; when well mixed add to boiling soup, stirring
constantly. If recipe calls for more flour than butter, or soup is one
that should be made in double boiler, add gradually a portion of hot
mixture to butter and flour until of such consistency that it may be
poured into the mixture remaining in double boiler.



\needspace{15\baselineskip}
\section*{Soups With Meat Stock}


\needspace{15\baselineskip}
\subsection*{Brown Soup Stock}


\begin{minipage}{1.0\textwidth}
{\setlength{\multicolsep}{0pt}\setlength{\columnsep}{2em}\raggedcolumns%
\begin{multicols}{2}
\begin{itemize}
\setlength{\itemsep}{0pt}
\setlength{\parsep}{0pt}
\item 6 lbs. shin of beef
\item 3 quarts cold water
\item 1/2 teaspoon peppercorns
\item 6 cloves
\item 1/2 bay leaf
\item 3 sprigs thyme
\item 1 sprig marjoram
\item 2 sprigs parsley
\item 1/2 cup carrot
\item 1/2 cup turnip
\item 1/2 cup onion
\item 1/2 cup celery
\item 1 tablespoon salt
\end{itemize}
\end{multicols}}
\end{minipage}

\vspace{0.3em}
\noindent%
Wipe beef, and cut the lean meat in inch cubes. Brown one-third of meat
in hot frying-pan in marrow from a marrow-bone. Put remaining two-thirds
with bone and fat in soup kettle, add water, and let stand for thirty
minutes. Place on back of range, add browned meat, and heat gradually to
boiling-point. As scum rises it should be removed. Cover, and cook
slowly six hours, keeping below boiling-point during cooking. Add
vegetables and seasonings, cook one and one-half hours, strain, and cool
as quickly as possible.



\needspace{15\baselineskip}
\subsection*{Bouillon}


\begin{minipage}{1.0\textwidth}
{\setlength{\multicolsep}{0pt}\setlength{\columnsep}{2em}\raggedcolumns%
\begin{multicols}{2}
\begin{itemize}
\setlength{\itemsep}{0pt}
\setlength{\parsep}{0pt}
\item 5 lbs. lean beef from middle of round
\item 2 lbs. marrow-bone
\item 3 quarts cold water
\item 1 teaspoon peppercorns
\item 1 tablespoon salt
\item 1/3 cup carrot
\item 1/3 cup turnip
\item 1/3 cup onion
\item 1/3 cup celery
\end{itemize}
\end{multicols}}
\end{minipage}

\vspace{0.3em}
\noindent%
Wipe, and cut meat in inch cubes. Put two-thirds of meat in soup kettle,
and soak in water thirty minutes. Brown remainder in hot frying-pan with
marrow from marrow-bone. Put browned meat and bone in kettle. Heat to
boiling-point; skim thoroughly, and cook at temperature below
boiling-point five hours. Add seasonings and vegetables, cook one hour,
strain, and cool. Remove fat, and clear. Serve in bouillon cups.



\needspace{15\baselineskip}
\subsection*{Tomato Bouillon with Oysters}


\begin{minipage}{1.0\textwidth}
{\setlength{\multicolsep}{0pt}\setlength{\columnsep}{2em}\raggedcolumns%
\begin{multicols}{2}
\begin{itemize}
\setlength{\itemsep}{0pt}
\setlength{\parsep}{0pt}
\item 1 can tomatoes
\item 1 1/2 quarts bouillon
\item 1 tablespoon chopped onion
\item 1/2 bay leaf
\item 6 cloves
\item 1/2 teaspoon celery seed
\item 1/2 teaspoon peppercorns
\item 1 pint oysters
\end{itemize}
\end{multicols}}
\end{minipage}

\vspace{0.3em}
\noindent%
Mix all ingredients except oysters, and boil twenty minutes. Strain,
cool, and clear. Add parboiled oysters, and serve in bouillon cups with
small croûtons.



\needspace{15\baselineskip}
\subsection*{Iced Bouillon}

Flavor bouillon with sherry or Madeira wine, and serve cold.



\needspace{15\baselineskip}
\subsection*{Macaroni Soup}


\begin{itemize}
\setlength{\itemsep}{0pt}
\setlength{\parsep}{0pt}
\item 1 quart Brown Soup Stock
\item 1/4 cup macaroni, broken in half-inch pieces
\item Salt
\item Pepper
\end{itemize}

\vspace{-0.5em}
\noindent%
Cook macaroni in boiling salted water until soft. Drain, and add to
stock heated to boiling-point. Season with salt and pepper. Spaghetti or
other Italian pastas may be substituted for macaroni.



\needspace{15\baselineskip}
\subsection*{Tomato Soup with Stock}


\begin{minipage}{1.0\textwidth}
{\setlength{\multicolsep}{0pt}\setlength{\columnsep}{2em}\raggedcolumns%
\begin{multicols}{2}
\begin{itemize}
\setlength{\itemsep}{0pt}
\setlength{\parsep}{0pt}
\item 1 quart Brown Soup Stock
\item 1 can tomatoes
\item 1/2 teaspoon peppercorns
\item 1 small bay leaf
\item 3 cloves
\item 3 sprigs thyme
\item 4 tablespoons butter
\item 1/3 cup flour
\item 1/4 cup onion
\item 1/4 cup carrot
\item 1/4 cup celery
\item 1/4 cup raw ham
\item Salt
\item Pepper
\end{itemize}
\end{multicols}}
\end{minipage}

\vspace{0.3em}
\noindent%
Cook onion, carrot, celery, and ham in butter five minutes, add flour,
peppercorns, bay leaf, cloves, and thyme, and cook three minutes; then
add tomatoes, cover, and cook slowly one hour. When cooked in oven it
requires less watching. Rub through a strainer, add hot stock, and
season with salt and pepper.



\needspace{15\baselineskip}
\subsection*{Turkish Soup}


\begin{minipage}{1.0\textwidth}
{\setlength{\multicolsep}{0pt}\setlength{\columnsep}{2em}\raggedcolumns%
\begin{multicols}{2}
\begin{itemize}
\setlength{\itemsep}{0pt}
\setlength{\parsep}{0pt}
\item 5 cups Brown Soup Stock
\item 1/4 cup rice
\item 1 1/2 cups stewed and strained tomatoes
\item Bit of bay leaf
\item 2 slices onion
\item 10 peppercorns
\item 1/4 teaspoon celery salt
\item 2 tablespoons butter
\item 1 1/2 tablespoons flour
\end{itemize}
\end{multicols}}
\end{minipage}

\vspace{0.3em}
\noindent%
Cook rice in Brown Stock until soft. Cook bay leaf, onion, peppercorns,
and celery salt with tomatoes thirty minutes. Combine mixtures, rub
through sieve, and bind with butter and flour cooked together. Season
with salt and pepper if needed.



\needspace{15\baselineskip}
\subsection*{Creole Soup}


\begin{minipage}{1.0\textwidth}
{\setlength{\multicolsep}{0pt}\setlength{\columnsep}{2em}\raggedcolumns%
\begin{multicols}{2}
\begin{itemize}
\setlength{\itemsep}{0pt}
\setlength{\parsep}{0pt}
\item 1 quart Brown Soup Stock
\item 1 pint tomatoes
\item 3 tablespoons chopped green peppers
\item 2 tablespoons chopped onion
\item 1/4 cup butter
\item 1/3 cup flour
\item Salt
\item Pepper
\item Cayenne
\item 2 tablespoons grated horseradish
\item 1 teaspoon vinegar
\item 1/4 cup macaroni rings
\end{itemize}
\end{multicols}}
\end{minipage}

\vspace{0.3em}
\noindent%
Cook pepper and onion in butter five minutes. Add flour, stock, and
tomatoes, and simmer fifteen minutes. Strain, rub through sieve, and
season highly with salt, pepper, and cayenne. Just before serving add
horseradish, vinegar, and macaroni previously cooked and cut in rings.



\needspace{15\baselineskip}
\subsection*{Julienne Soup}

To one quart clear Brown Soup Stock, add one-fourth cup each carrot and
turnip, cut in thin strips one and one-half inches long, previously
cooked in boiling salted water, and two tablespoons, each, cooked peas
and string beans. Heat to boiling-point.



\needspace{15\baselineskip}
\subsection*{Dinner Soup}


\begin{minipage}{1.0\textwidth}
{\setlength{\multicolsep}{0pt}\setlength{\columnsep}{2em}\raggedcolumns%
\begin{multicols}{2}
\begin{itemize}
\setlength{\itemsep}{0pt}
\setlength{\parsep}{0pt}
\item 3 1/2 lbs. lean beef from round
\item 2 lbs. marrow-bone
\item 2 qts. cold water
\item 1 can tomatoes
\item 1 teaspoon peppercorns
\item 1 tablespoon salt
\item 1 tablespoon lean raw ham, finely chopped
\item 2 tablespoons butter
\item 1/3 cup, carrot
\item 1/3 cup, turnip
\item Onion |cut in small pieces
\item Celery|
\item Carrot|1/2 cup each, cut in fancy shapes
\item Turnip|
\item 1/2 cup onion
\item 1/2 cup celery
\item 1/2 teaspoon salt
\item Few grains cayenne
\item 1/4 cup Madeira wine
\item 1 teaspoon Worcestershire Sauce
\item 1 teaspoon lemon juice
\end{itemize}
\end{multicols}}
\end{minipage}

\vspace{0.3em}
\noindent%
Cut ox-tail in small pieces, wash, drain, sprinkle with salt and pepper,
dredge with flour, and fry in butter ten minutes. Add to Brown Stock,
and simmer one hour. Then add vegetables, which have been parboiled
twenty minutes; simmer until vegetables are soft, add salt, cayenne,
wine, Worcestershire Sauce, and lemon juice.



\needspace{15\baselineskip}
\subsection*{Scotch Soup}


\begin{minipage}{1.0\textwidth}
{\setlength{\multicolsep}{0pt}\setlength{\columnsep}{2em}\raggedcolumns%
\begin{multicols}{2}
\begin{itemize}
\setlength{\itemsep}{0pt}
\setlength{\parsep}{0pt}
\item 3 lbs. mutton from fore-quarter
\item 2 qts. cold water
\item 1/2 tablespoon salt
\item 1/4 teaspoon pepper
\item 2 slices turnip
\item 1/2 onion
\item 1/4 cup flour
\item 1/4 cup, carrot
\item 1/4 cup, turnip
\item 2 tablespoons pearl barley
\end{itemize}
\end{multicols}}
\end{minipage}

\vspace{0.3em}
\noindent%
Wipe meat, remove skin and fat, and cut meat in small pieces. Add water,
heat gradually to boiling-point, skim, and cook slowly two hours. After
cooking one hour, add salt, pepper, turnip, and onion. Strain, cool,
remove fat, reheat, and thicken with flour diluted with enough cold
water to pour easily. Cook carrot and turnip dice in boiling salted
water until soft; drain, and add to soup. Soak barley over night, in
cold water, drain, and cook in boiling salted water until soft; drain,
and add to soup. If barley should be cooked in the soup, it would absorb
the greater part of the stock. Barley may be omitted; in that case
sprinkle with finely chopped parsley and serve with croûtons.



\needspace{15\baselineskip}
\subsection*{White Soup Stock I}


\begin{minipage}{1.0\textwidth}
{\setlength{\multicolsep}{0pt}\setlength{\columnsep}{2em}\raggedcolumns%
\begin{multicols}{2}
\begin{itemize}
\setlength{\itemsep}{0pt}
\setlength{\parsep}{0pt}
\item 3 lbs. knuckle of veal
\item 1 lb. lean beef
\item 3 quarts boiling water
\item 1 onion
\item 6 slices carrot
\item 1 large stalk celery
\item 1/2 teaspoon peppercorns
\item 1/2 bay leaf
\item 2 sprigs thyme
\item 2 cloves
\end{itemize}
\end{multicols}}
\end{minipage}

\vspace{0.3em}
\noindent%
Wipe veal, remove from bone, and cut in small pieces; cut beef in
pieces, put bone and meat in soup kettle, cover with cold water, and
bring quickly to boiling-point; drain, throw away the water. Wash
thoroughly bones and meat in cold water; return to kettle, add
vegetables, seasonings, and three quarts boiling water. Boil three or
four hours; the stock should be reduced one half.



\needspace{15\baselineskip}
\subsection*{White Soup Stock II}


\begin{minipage}{1.0\textwidth}
{\setlength{\multicolsep}{0pt}\setlength{\columnsep}{2em}\raggedcolumns%
\begin{multicols}{2}
\begin{itemize}
\setlength{\itemsep}{0pt}
\setlength{\parsep}{0pt}
\item 4 lbs. knuckle of veal
\item 2 quarts cold water
\item 1 tablespoon salt
\item 1/2 teaspoon peppercorns
\item 1 onion
\item 2 stalks celery
\item Blade of mace
\end{itemize}
\end{multicols}}
\end{minipage}

\vspace{0.3em}
\noindent%
Wipe meat, remove from bone, and cut in small pieces. Put meat, bone,
water, and seasonings in kettle. Heat gradually to boiling-point,
skimming frequently. Simmer four or five hours, and strain. If scum has
been carefully removed, and soup is strained through double thickness of
cheese-cloth, stock will be quite clear.



\needspace{15\baselineskip}
\subsection*{White Soup Stock III}

The water in which a fowl or chicken is cooked makes White Stock.



\needspace{15\baselineskip}
\subsection*{Chicken Soup with Wine}


\begin{minipage}{1.0\textwidth}
{\setlength{\multicolsep}{0pt}\setlength{\columnsep}{2em}\raggedcolumns%
\begin{multicols}{2}
\begin{itemize}
\setlength{\itemsep}{0pt}
\setlength{\parsep}{0pt}
\item 3 lb. fowl
\item 2 quarts cold water
\item 2 slices carrot
\item 1 tablespoon salt
\item 1/2 teaspoon peppercorns
\item 1 onion, sliced
\item 2 stalks celery
\item Bit of bay leaf
\item 2 tablespoons Sauterne wine
\item 1 teaspoon beef extract
\item 1 cup cream
\item Salt
\item Pepper
\end{itemize}
\end{multicols}}
\end{minipage}

\vspace{0.3em}
\noindent%
Wipe and cut up fowl. Cover with water, and add carrot, salt,
peppercorns, onion, celery, and bay leaf. Bring quickly to
boiling-point, then let simmer until meat is tender. Remove meat and
strain stock. Chill, remove fat, reheat, and add wine, beef extract, and
cream. Season with salt and pepper.



\needspace{15\baselineskip}
\subsection*{French White Soup}


\begin{minipage}{1.0\textwidth}
{\setlength{\multicolsep}{0pt}\setlength{\columnsep}{2em}\raggedcolumns%
\begin{multicols}{2}
\begin{itemize}
\setlength{\itemsep}{0pt}
\setlength{\parsep}{0pt}
\item 4 lb. fowl
\item Knuckle of veal
\item 3 qts. cold water
\item 1 onion, sliced
\item 6 slices carrot
\item 1/2 bay leaf
\item 1 sprig parsley
\item 1/2 teaspoon thyme
\item 1/2 teaspoon peppercorns
\item 1/2 tablespoon salt
\item 1 tablespoon lean raw ham, finely chopped
\item 4 tablespoons butter
\item 3 tablespoons flour
\item 1 cup cream
\item 4 egg yolks
\end{itemize}
\end{multicols}}
\end{minipage}

\vspace{0.3em}
\noindent%
Wipe, clean, and disjoint fowl. Wipe veal, remove from bone, and cut in
small pieces. Put meat, bone, and water in kettle, heat slowly to
boiling-point, skim, and cook slowly four hours. Cook vegetables and ham
in one tablespoon butter five minutes, add to soup with peppercorns and
salt, and cook one hour. Strain, cool, and remove fat. Reheat three cups
stock, thicken with remaining butter and flour cooked together, and just
before serving add cream and egg yolks. Garnish with one-half cup cooked
green peas and Chicken Custard cut in dice.



\needspace{15\baselineskip}
\subsection*{White Soup}


\begin{minipage}{1.0\textwidth}
{\setlength{\multicolsep}{0pt}\setlength{\columnsep}{2em}\raggedcolumns%
\begin{multicols}{2}
\begin{itemize}
\setlength{\itemsep}{0pt}
\setlength{\parsep}{0pt}
\item 5 cups White Stock III
\item 1/2 tablespoon salt
\item 1/2 teaspoon peppercorns
\item 1 slice onion
\item 1 stalk celery
\item 2 cups scalded milk
\item 3 tablespoons butter
\item 4 tablespoons flour
\item 4 egg yolks
\item Salt and pepper
\end{itemize}
\end{multicols}}
\end{minipage}

\vspace{0.3em}
\noindent%
Add seasonings to stock, and simmer thirty minutes; strain, and thicken
with butter and flour cooked together; add scalded milk. Dilute eggs,
slightly beaten, with hot soup, and add to remaining soup; strain, and
season with salt and pepper. Serve at once or soup will have a curdled
appearance.



\needspace{15\baselineskip}
\subsection*{Chicken Soup}


\begin{minipage}{1.0\textwidth}
{\setlength{\multicolsep}{0pt}\setlength{\columnsep}{2em}\raggedcolumns%
\begin{multicols}{2}
\begin{itemize}
\setlength{\itemsep}{0pt}
\setlength{\parsep}{0pt}
\item 6 cups White Stock III
\item 1 tablespoon lean raw ham, finely chopped
\item 6 slices carrot, cut in cubes
\item 2 stalks celery
\item 1/2 bay leaf
\item 1/4 teaspoon peppercorns
\item 1 sliced onion
\item 1/3 cup hot boiled rice
\end{itemize}
\end{multicols}}
\end{minipage}

\vspace{0.3em}
\noindent%
Add seasonings to stock, heat gradually to boiling-point, and boil
thirty minutes; strain, and add rice.



\needspace{15\baselineskip}
\subsection*{Turkey Soup}

Break turkey carcass in pieces, removing all stuffing; put in kettle
with any bits of meat that may have been left over. Cover with cold
water, bring slowly to boiling-point, and simmer two hours. Strain,
remove fat, and season with salt and pepper. One or two outer stalks of
celery may be cooked with carcass to give additional flavor.



\needspace{15\baselineskip}
\subsection*{Hygienic Soup}


\begin{minipage}{1.0\textwidth}
{\setlength{\multicolsep}{0pt}\setlength{\columnsep}{2em}\raggedcolumns%
\begin{multicols}{2}
\begin{itemize}
\setlength{\itemsep}{0pt}
\setlength{\parsep}{0pt}
\item 6 cups White Stock III
\item 1/4 cup oatmeal
\item 2 cups scalded milk
\item 2 tablespoons butter
\item 2 tablespoons flour
\item Salt and pepper
\end{itemize}
\end{multicols}}
\end{minipage}

\vspace{0.3em}
\noindent%
Heat stock to boiling-point, add oatmeal, and boil one hour; rub through
sieve, add milk, and thicken with butter and flour cooked together.
Season with salt and pepper.



\needspace{15\baselineskip}
\subsection*{Farina Soup}


\begin{minipage}{1.0\textwidth}
{\setlength{\multicolsep}{0pt}\setlength{\columnsep}{2em}\raggedcolumns%
\begin{multicols}{2}
\begin{itemize}
\setlength{\itemsep}{0pt}
\setlength{\parsep}{0pt}
\item 4 cups White Stock III
\item 1/4 cup farina
\item 2 cups scalded milk
\item 1 cup cream
\item Few gratings of nutmeg
\item Salt and pepper
\end{itemize}
\end{multicols}}
\end{minipage}

\vspace{0.3em}
\noindent%
Heat stock to boiling-point, add farina, and boil fifteen minutes; then
add milk, cream, and seasonings.



\needspace{15\baselineskip}
\subsection*{Spring Soup}


\begin{minipage}{1.0\textwidth}
{\setlength{\multicolsep}{0pt}\setlength{\columnsep}{2em}\raggedcolumns%
\begin{multicols}{2}
\begin{itemize}
\setlength{\itemsep}{0pt}
\setlength{\parsep}{0pt}
\item 1 quart White Stock I or II
\item 1 large onion thinly sliced
\item 3 tablespoons butter
\item 1/2 cup stale baker's bread
\item 1 cup milk
\item 1 cup cream
\item 2 tablespoons flour
\item Salt and pepper
\end{itemize}
\end{multicols}}
\end{minipage}

\vspace{0.3em}
\noindent%
Cook onion fifteen minutes in one tablespoon butter; add to stock, with
bread broken in pieces. Simmer one hour; rub through sieve. Add milk,
and bind with remaining butter and flour cooked together; add cream, and
season.



\needspace{15\baselineskip}
\subsection*{Duchess Soup}


\begin{minipage}{1.0\textwidth}
{\setlength{\multicolsep}{0pt}\setlength{\columnsep}{2em}\raggedcolumns%
\begin{multicols}{2}
\begin{itemize}
\setlength{\itemsep}{0pt}
\setlength{\parsep}{0pt}
\item 4 cups White Stock III
\item 2 slices carrot, cut in cubes
\item 2 slices onion
\item 2 blades mace
\item 1/2 cup grated mild cheese
\item 1/3 cup butter
\item 1/4 cup flour
\item 1 teaspoon salt
\item 1/8 teaspoon pepper
\item 2 cups scalded milk
\end{itemize}
\end{multicols}}
\end{minipage}

\vspace{0.3em}
\noindent%
Cook vegetables three minutes in one and one-half tablespoons butter,
then add stock and mace; boil fifteen minutes, strain, and add milk.
Thicken with remaining butter and flour cooked together; add salt and
pepper. Stir in cheese, and serve as soon as cheese is melted.



\needspace{15\baselineskip}
\subsection*{Potage à la Reine}


\begin{minipage}{1.0\textwidth}
{\setlength{\multicolsep}{0pt}\setlength{\columnsep}{2em}\raggedcolumns%
\begin{multicols}{2}
\begin{itemize}
\setlength{\itemsep}{0pt}
\setlength{\parsep}{0pt}
\item 4 cups White Stock III
\item 1/2 teaspoon peppercorns
\item 1 stalk celery
\item 1 slice onion
\item 1/2 tablespoon salt
\item Yolks 3 “hard-boiled” eggs
\item 1/3 cup cracker crumbs
\item Breast meat from a boiled chicken
\item 2 cups scalded milk
\item 1/2 cup cold milk
\item 3 tablespoons butter
\item 3 tablespoons flour
\end{itemize}
\end{multicols}}
\end{minipage}

\vspace{0.3em}
\noindent%
Cook stock with seasonings twenty minutes. Rub yolks of eggs through
sieve. Soak cracker crumbs in cold milk until soft; add to eggs. Chop
meat and rub through sieve; add to egg and cracker mixture. Then pour
milk on slowly, and add to strained stock; boil three minutes. Bind with
butter and flour cooked together.



\needspace{15\baselineskip}
\subsection*{Royal Soup}


\begin{minipage}{1.0\textwidth}
{\setlength{\multicolsep}{0pt}\setlength{\columnsep}{2em}\raggedcolumns%
\begin{multicols}{2}
\begin{itemize}
\setlength{\itemsep}{0pt}
\setlength{\parsep}{0pt}
\item 1 cup stale bread crumbs
\item 1/2 cup milk
\item Yolks 3 “hard-boiled” eggs
\item Breast meat from a boiled chicken
\item Salt and pepper
\item 1 1/2 cups scalded milk
\item 3 1/2 cups White Stock III
\item 2 1/2 tablespoons butter
\item 2 1/2 tablespoons flour
\end{itemize}
\end{multicols}}
\end{minipage}

\vspace{0.3em}
\noindent%
Soak bread crumbs in milk, add yolks of eggs rubbed through a sieve and
chicken meat also rubbed through a sieve. Add gradually milk, and
chicken stock highly seasoned. Bind with butter and flour cooked
together, and season with salt and pepper.



\needspace{15\baselineskip}
\subsection*{St. Germain Soup}


\begin{minipage}{1.0\textwidth}
{\setlength{\multicolsep}{0pt}\setlength{\columnsep}{2em}\raggedcolumns%
\begin{multicols}{2}
\begin{itemize}
\setlength{\itemsep}{0pt}
\setlength{\parsep}{0pt}
\item 3 cups White Stock I, II, or III
\item 1 can Marrowfat peas
\item 1 cup cold water
\item 1/2 onion
\item Bit of bay leaf
\item Sprig of parsley
\item Blade of mace
\item 2 teaspoons sugar
\item 1 teaspoon salt
\item 1/8 teaspoon pepper
\item 2 tablespoons butter
\item 2 tablespoons corn-starch
\item 1 cup milk
\end{itemize}
\end{multicols}}
\end{minipage}

\vspace{0.3em}
\noindent%
Drain and rinse peas, reserving one-third cup; put remainder in cold
water with seasonings, and simmer one-half hour; rub through sieve and
add stock. Bind with butter and corn-starch cooked together; boil five
minutes. Add milk and reserved peas.



\needspace{15\baselineskip}
\subsection*{Imperial Soup}


\begin{minipage}{1.0\textwidth}
{\setlength{\multicolsep}{0pt}\setlength{\columnsep}{2em}\raggedcolumns%
\begin{multicols}{2}
\begin{itemize}
\setlength{\itemsep}{0pt}
\setlength{\parsep}{0pt}
\item 4 cups White Stock III
\item 2 cups stale bread crumbs
\item 2 stalks celery, broken in pieces
\item 2 slices carrot, cut in cubes
\item 1 small onion
\item 3 tablespoons butter
\item Sprig of parsley
\item 2 cloves
\item 1/2 teaspoon peppercorns
\item Bit of bay leaf
\item Blade of mace
\item 1 teaspoon salt
\item 1/2 breast boiled chicken
\item 1/3 cup blanched almonds
\item 1 cup cream
\item 1/2 cup milk
\item 2 tablespoons flour
\end{itemize}
\end{multicols}}
\end{minipage}

\vspace{0.3em}
\noindent%
Cook celery, carrot, and onion in one tablespoon butter five minutes;
tie in cheese-cloth with parsley, cloves, peppercorns, bay leaf, and
mace; add to stock with salt and bread crumbs, simmer one hour, remove
seasonings, and rub through a sieve. Chop chicken meat and rub through
sieve; pound almonds to a paste, add to chicken, then add cream. Combine
mixtures, add milk, reheat, and bind with remaining butter and flour
cooked together.



\needspace{15\baselineskip}
\subsection*{Veal and Sago Soup}


\begin{minipage}{1.0\textwidth}
{\setlength{\multicolsep}{0pt}\setlength{\columnsep}{2em}\raggedcolumns%
\begin{multicols}{2}
\begin{itemize}
\setlength{\itemsep}{0pt}
\setlength{\parsep}{0pt}
\item 2 1/2 lbs. lean veal
\item 3 quarts cold water
\item 1/4 lb. pearl sago
\item 2 cups scalded milk
\item 4 egg yolks
\item Salt and pepper
\end{itemize}
\end{multicols}}
\end{minipage}

\vspace{0.3em}
\noindent%
Order meat from market, very finely chopped. Pick over and remove
particles of fat. Cover meat with water, bring slowly to boiling-point,
and simmer two hours, skimming occasionally; strain and reheat. Soak
sago one-half hour in enough cold water to cover, stir into hot stock,
boil thirty minutes, and add milk; then pour mixture slowly on yolks of
eggs, slightly beaten. Season with salt and pepper.



\needspace{15\baselineskip}
\subsection*{Asparagus Soup}


\begin{minipage}{1.0\textwidth}
{\setlength{\multicolsep}{0pt}\setlength{\columnsep}{2em}\raggedcolumns%
\begin{multicols}{2}
\begin{itemize}
\setlength{\itemsep}{0pt}
\setlength{\parsep}{0pt}
\item 3 cups White Stock II or III
\item 1 can asparagus
\item 2 cups cold water
\item 1 slice onion
\item 1/4 cup butter
\item 1/4 cup flour
\item 2 cups scalded milk
\item Salt and pepper
\end{itemize}
\end{multicols}}
\end{minipage}

\vspace{0.3em}
\noindent%
Drain and rinse asparagus, reserve tips, and add stalks to cold water;
boil five minutes, drain, add stock, and onion; boil thirty minutes, rub
through sieve, and bind with butter and flour cooked together. Add salt,
pepper, milk, and tips.



\needspace{15\baselineskip}
\subsection*{Cream of Celery Soup}


\begin{minipage}{1.0\textwidth}
{\setlength{\multicolsep}{0pt}\setlength{\columnsep}{2em}\raggedcolumns%
\begin{multicols}{2}
\begin{itemize}
\setlength{\itemsep}{0pt}
\setlength{\parsep}{0pt}
\item 2 cups White Stock II or III
\item 3 cups celery, cut in inch pieces
\item 2 cups boiling water
\item 1 slice onion
\item 2 tablespoons butter
\item 3 tablespoons flour
\item 2 cups milk
\item 1 cup cream
\item Salt
\item Pepper
\end{itemize}
\end{multicols}}
\end{minipage}

\vspace{0.3em}
\noindent%
Parboil celery in water ten minutes; drain, add stock, cook until celery
is soft, and rub through sieve. Scald onion in milk, remove onion, add
milk to stock, bind, add cream, and season with salt and pepper.



\needspace{15\baselineskip}
\subsection*{Spinach Soup}


\begin{minipage}{1.0\textwidth}
{\setlength{\multicolsep}{0pt}\setlength{\columnsep}{2em}\raggedcolumns%
\begin{multicols}{2}
\begin{itemize}
\setlength{\itemsep}{0pt}
\setlength{\parsep}{0pt}
\item 4 cups White Stock II or III
\item 2 quarts spinach
\item 3 cups boiling water
\item 2 cups milk
\item 1/4 cup butter
\item 1/3 cup flour
\item Salt
\item Pepper
\end{itemize}
\end{multicols}}
\end{minipage}

\vspace{0.3em}
\noindent%
Wash, pick over, and cook spinach thirty minutes in boiling water to
which has been added one-fourth teaspoon powdered sugar and one-eighth
teaspoon of soda; drain, chop, and rub through sieve; add stock, heat to
boiling-point, bind, add milk, and season with salt and pepper.



\needspace{15\baselineskip}
\subsection*{Cream of Lettuce Soup}


\begin{minipage}{1.0\textwidth}
{\setlength{\multicolsep}{0pt}\setlength{\columnsep}{2em}\raggedcolumns%
\begin{multicols}{2}
\begin{itemize}
\setlength{\itemsep}{0pt}
\setlength{\parsep}{0pt}
\item 2 1/2 cups White Stock II or III
\item 2 heads lettuce finely cut
\item 2 tablespoons rice
\item 1/2 cup cream
\item 1/4 tablespoon onion, finely chopped
\item 1 tablespoon butter
\item Yolk 1 egg
\item Few grains nutmeg
\item Salt
\item Pepper
\end{itemize}
\end{multicols}}
\end{minipage}

\vspace{0.3em}
\noindent%
Cook onion five minutes in butter, add lettuce, rice, and stock. Cook
until rice is soft, then add cream, yolk of egg slightly beaten, nutmeg,
salt, and pepper. Remove outer leaves from lettuce, using only tender
part for soup.



\needspace{15\baselineskip}
\subsection*{Mushroom Soup}


\begin{minipage}{1.0\textwidth}
{\setlength{\multicolsep}{0pt}\setlength{\columnsep}{2em}\raggedcolumns%
\begin{multicols}{2}
\begin{itemize}
\setlength{\itemsep}{0pt}
\setlength{\parsep}{0pt}
\item 1/2 lb. mushrooms
\item 4 cups White Stock III
\item 1/4 cup pearl sago
\item 1 cup boiling water
\item 1 cup heavy cream
\item 4 egg yolks
\item Salt and pepper
\end{itemize}
\end{multicols}}
\end{minipage}

\vspace{0.3em}
\noindent%
Clean and chop mushrooms, and add to stock. Cook twenty minutes and rub
through a sieve. Cook sago in boiling water thirty minutes, add to
stock, and as soon as boiling-point is reached, season with salt and
pepper; then add cream and yolks of eggs.



\needspace{15\baselineskip}
\subsection*{Cream of Mushroom Soup}


\begin{minipage}{1.0\textwidth}
{\setlength{\multicolsep}{0pt}\setlength{\columnsep}{2em}\raggedcolumns%
\begin{multicols}{2}
\begin{itemize}
\setlength{\itemsep}{0pt}
\setlength{\parsep}{0pt}
\item 1/2 lb. mushrooms
\item 4 cups White Stock III
\item 1 slice onion
\item 1/4 cup butter
\item 1/4 cup flour
\item 1 cup cream
\item Salt
\item Pepper
\item 2 tablespoons Sauterne
\end{itemize}
\end{multicols}}
\end{minipage}

\vspace{0.3em}
\noindent%
Chop mushrooms, add to White Stock with onion, cook twenty minutes, and
rub through a sieve. Reheat, bind with butter and flour cooked together,
then add cream and salt and pepper to taste. Just before serving add
wine.



\needspace{15\baselineskip}
\subsection*{Cream of Watercress Soup}


\begin{minipage}{1.0\textwidth}
{\setlength{\multicolsep}{0pt}\setlength{\columnsep}{2em}\raggedcolumns%
\begin{multicols}{2}
\begin{itemize}
\setlength{\itemsep}{0pt}
\setlength{\parsep}{0pt}
\item 2 cups White Stock I, II or III
\item 2 bunches watercress
\item 3 tablespoons butter
\item 2 tablespoons flour
\item 1/2 cup milk
\item Yolk 1 egg
\item Salt
\item Pepper
\end{itemize}
\end{multicols}}
\end{minipage}

\vspace{0.3em}
\noindent%
Cut finely leaves of watercress; cook five minutes in two tablespoons
butter, add stock, and boil five minutes. Thicken with butter and flour
cooked together, add salt and pepper. Just before serving, add milk and
egg yolk, slightly beaten. Serve with slices of French bread, browned in
oven.



\needspace{15\baselineskip}
\subsection*{Cream of Cauliflower Soup}


\begin{minipage}{1.0\textwidth}
{\setlength{\multicolsep}{0pt}\setlength{\columnsep}{2em}\raggedcolumns%
\begin{multicols}{2}
\begin{itemize}
\setlength{\itemsep}{0pt}
\setlength{\parsep}{0pt}
\item 4 cups hot White Stock II or III
\item 1 cauliflower
\item 1/4 cup butter
\item 1 slice onion
\item 1 stalk celery, cut in inch pieces
\item 1/2 bay leaf
\item 1/4 cup flour
\item 2 cups milk
\item Salt
\item Pepper
\end{itemize}
\end{multicols}}
\end{minipage}

\vspace{0.3em}
\noindent%
Soak cauliflower, head down, one hour in cold water to cover; cook in
boiling salted water twenty minutes. Reserve one-half flowerets, and rub
remaining cauliflower through sieve. Cook onion, celery, and bay leaf in
butter five minutes. Remove bay leaf, then add flour, and stir into hot
stock; add cauliflower and milk. Season with salt and pepper; then
strain, add flowerets, and reheat.



\needspace{15\baselineskip}
\subsection*{Cucumber Soup}


\begin{minipage}{1.0\textwidth}
{\setlength{\multicolsep}{0pt}\setlength{\columnsep}{2em}\raggedcolumns%
\begin{multicols}{2}
\begin{itemize}
\setlength{\itemsep}{0pt}
\setlength{\parsep}{0pt}
\item 3 large cucumbers
\item 2 tablespoons butter
\item 3 tablespoons flour
\item 3 cups White Stock III
\item 1 cup milk
\item 1 slice onion
\item 2 blades mace
\item 1/2 cup cream
\item 4 egg yolks
\item Salt and pepper
\end{itemize}
\end{multicols}}
\end{minipage}

\vspace{0.3em}
\noindent%
Peel cucumbers, slice, and remove seeds. Cook in butter ten minutes;
then add flour and stock. Scald milk with onion and mace. Combine
mixtures and rub through a sieve. Reheat to boiling-point and add cream
and egg yolks. Season with salt and pepper.



\needspace{15\baselineskip}
\subsection*{Almond Soup}


\begin{minipage}{1.0\textwidth}
{\setlength{\multicolsep}{0pt}\setlength{\columnsep}{2em}\raggedcolumns%
\begin{multicols}{2}
\begin{itemize}
\setlength{\itemsep}{0pt}
\setlength{\parsep}{0pt}
\item 2/3 cup almonds
\item 6 bitter almonds
\item 4 tablespoons cold water
\item 1/8 teaspoon salt
\item 3 cups White Stock III
\item 1 small onion
\item 3 stalks celery
\item 3 tablespoons butter
\item 3 tablespoons flour
\item 2 cups scalded milk
\item 1 cup cream
\item Salt and pepper
\end{itemize}
\end{multicols}}
\end{minipage}

\vspace{0.3em}
\noindent%
Blanch, chop, and pound almonds in a mortar. Add gradually water and
salt; then add stock, sliced onion, and celery, let simmer one hour, and
rub through a sieve. Melt butter, add flour, and pour on gradually the
hot liquor; then add milk, cream, and salt and pepper to taste. Serve
with Mock Almonds.



\needspace{15\baselineskip}
\subsection*{String Bean Soup}


\begin{minipage}{1.0\textwidth}
{\setlength{\multicolsep}{0pt}\setlength{\columnsep}{2em}\raggedcolumns%
\begin{multicols}{2}
\begin{itemize}
\setlength{\itemsep}{0pt}
\setlength{\parsep}{0pt}
\item 4 cups White Stock I, II, or III
\item 2 quarts string beans
\item 2 cups scalded milk
\item 1/4 cup flour
\item 1/4 cup butter
\item Salt and pepper
\end{itemize}
\end{multicols}}
\end{minipage}

\vspace{0.3em}
\noindent%
Cook beans until soft in boiling salted water to cover; drain, and rub
through sieve. Add pulp to White Stock, then milk; bind, and season with
salt and pepper. Garnish with Fritter Beans.



\needspace{15\baselineskip}
\subsection*{Soup à la Soubise}

Thinly slice two Spanish onions, and cook ten minutes in one-fourth cup
butter, stirring constantly. Add one quart White Stock III, cook slowly
thirty minutes, and strain. Dilute three tablespoons flour with enough
cold water to pour easily, add to soup, and bring to boiling-point. Then
add one cup cream, and one tablespoon chopped green peppers, or
one-fourth cup grated cheese. Season with salt and pepper.



\needspace{15\baselineskip}
\section*{Chestnut Purée}


\begin{minipage}{1.0\textwidth}
{\setlength{\multicolsep}{0pt}\setlength{\columnsep}{2em}\raggedcolumns%
\begin{multicols}{2}
\begin{itemize}
\setlength{\itemsep}{0pt}
\setlength{\parsep}{0pt}
\item 4 cups White Stock II or III
\item 2 cups French chestnuts, boiled and mashed
\item 1 slice onion
\item 1/4 teaspoon celery salt
\item 2 cups scalded milk
\item 1/4 cup butter
\item 1/4 cup flour
\item Salt
\item Pepper
\end{itemize}
\end{multicols}}
\end{minipage}

\vspace{0.3em}
\noindent%
Cook stock, chestnuts, onion, and celery salt ten minutes; rub through
sieve, add milk, and bind. Season with salt and pepper.



\needspace{15\baselineskip}
\subsection*{Crab Soup}


\begin{minipage}{1.0\textwidth}
{\setlength{\multicolsep}{0pt}\setlength{\columnsep}{2em}\raggedcolumns%
\begin{multicols}{2}
\begin{itemize}
\setlength{\itemsep}{0pt}
\setlength{\parsep}{0pt}
\item 6 hard-shelled crabs
\item 3 cups White Stock III
\item 2/3 cup stale bread crumbs
\item 1 slice onion
\item 1 sprig parsley
\item 2 tablespoons butter
\item 2 tablespoons flour
\item 1 cup cream
\item Salt
\item Cayenne
\end{itemize}
\end{multicols}}
\end{minipage}

\vspace{0.3em}
\noindent%
Remove meat from crabs, and chop finely. Add stock, bread crumbs, onion,
and parsley, and simmer twenty minutes. Rub through a sieve, bind with
butter and flour cooked together, then add cream and seasonings. Serve
with Pulled Bread.



\needspace{15\baselineskip}
\subsection*{Philadelphia Pepper Pot}


\begin{minipage}{1.0\textwidth}
{\setlength{\multicolsep}{0pt}\setlength{\columnsep}{2em}\raggedcolumns%
\begin{multicols}{2}
\begin{itemize}
\setlength{\itemsep}{0pt}
\setlength{\parsep}{0pt}
\item 1/4 cup sliced onion
\item 1/4 cup chopped celery
\item 1/4 cup chopped green peppers
\item 4 tablespoons butter
\item 3 1/2 tablespoons flour
\item 5 cups hot White Stock III
\item 1/2 lb. honeycomb tripe, cut in cubes
\item 1 1/2 cups potato cubes
\item 1/2 teaspoon peppercorns, finely pounded
\item 3/4 tablespoon salt
\item 1/2 cup heavy cream
\end{itemize}
\end{multicols}}
\end{minipage}

\vspace{0.3em}
\noindent%
Cook vegetables in three tablespoons butter fifteen minutes; add flour,
and stir until well mixed; then add remaining ingredients except cream.
Cover, and let cook one hour. Just before serving, add cream and
remaining butter.



\needspace{15\baselineskip}
\subsection*{Mulligatawny Soup}


\begin{minipage}{1.0\textwidth}
{\setlength{\multicolsep}{0pt}\setlength{\columnsep}{2em}\raggedcolumns%
\begin{multicols}{2}
\begin{itemize}
\setlength{\itemsep}{0pt}
\setlength{\parsep}{0pt}
\item 5 cups White Stock II
\item 1 cup tomatoes
\item 1/4 cup onion, cut in slices
\item 1/4 cup carrot, cut in cubes
\item 1/4 cup celery, cut in cubes
\item 1 pepper, finely chopped
\item 1 apple, sliced
\item 1 cup raw chicken, cut in dice
\item 1/4 cup butter
\item 1/3 cup flour
\item 1 teaspoon curry powder
\item Blade of mace
\item 2 cloves
\item Sprig of parsley
\item Salt and pepper
\end{itemize}
\end{multicols}}
\end{minipage}

\vspace{0.3em}
\noindent%
Cook vegetables and chicken in butter until brown; add flour, curry
powder, mace, cloves, parsley, stock, and tomato, and simmer one hour.
Strain, reserve chicken, and rub vegetables through sieve. Add chicken
to strained soup, season with salt and pepper, and serve with boiled
rice.



\needspace{15\baselineskip}
\subsection*{Mock Turtle Soup}


\begin{minipage}{1.0\textwidth}
{\setlength{\multicolsep}{0pt}\setlength{\columnsep}{2em}\raggedcolumns%
\begin{multicols}{2}
\begin{itemize}
\setlength{\itemsep}{0pt}
\setlength{\parsep}{0pt}
\item 1 calf's head
\item 6 cloves
\item 1/2 teaspoon peppercorns
\item 6 allspice berries
\item 2 sprigs thyme
\item 1/3 cup sliced onion
\item 1/3 cup carrot, cut in dice
\item 2 cups brown stock
\item 1/4 cup butter
\item 1/2 cup flour
\item 1 cup stewed and strained tomatoes
\item Juice 1/2 lemon
\item Madeira wine
\end{itemize}
\end{multicols}}
\end{minipage}

\vspace{0.3em}
\noindent%
Clean and wash calf's head; soak one hour in cold water to cover. Cook
until tender in three quarts boiling salted water (to which seasoning
and vegetables have been added). Remove head; boil stock until reduced
to one quart. Strain and cool. Melt and brown butter, add flour, and
stir until well browned; then pour on slowly brown stock. Add
head-stock, tomato, one cup face-meat cut in dice, and lemon juice.
Simmer five minutes; add Royal custard cut in dice, and Egg Balls, or
Force-meat Balls. Add Madeira wine, and salt and pepper to taste.



\needspace{15\baselineskip}
\subsection*{Consommé}


\begin{minipage}{1.0\textwidth}
{\setlength{\multicolsep}{0pt}\setlength{\columnsep}{2em}\raggedcolumns%
\begin{multicols}{2}
\begin{itemize}
\setlength{\itemsep}{0pt}
\setlength{\parsep}{0pt}
\item 3 lbs. beef, poorer part of round
\item 1 lb. marrow-bone
\item 3 lbs. knuckle of veal
\item 1 quart chicken stock
\item 1/3 cup carrot
\item 1/3 cup turnip
\item 1/3 cup celery
\item 1/3 cup sliced onion
\item 2 tablespoons butter
\item 1 tablespoon salt
\item 1 teaspoon peppercorns
\item 4 cloves
\item 3 sprigs thyme
\item 1 sprig marjoram
\item 2 sprigs parsley
\item 1/2 bay leaf
\item 3 quarts cold water
\end{itemize}
\end{multicols}}
\end{minipage}

\vspace{0.3em}
\noindent%
Cut beef in one and one-half inch cubes, and brown one-half in some of
the marrow from marrow-bone; put remaining half in kettle with cold
water, add veal cut in pieces, browned meat, and bones. Let stand
one-half hour. Heat slowly to boiling-point, and let simmer three hours,
removing scum as it forms on top of kettle. Add one quart liquor in
which a fowl was cooked, and simmer two hours. Cook carrot, turnip,
onion, and celery in butter five minutes; then add to soup, with
remaining seasonings. Cook one and one-half hours, strain, cool quickly,
remove fat, and clear.



\needspace{15\baselineskip}
\subsection*{Consommé à la Royal}

Consommé, served with Royal custard.



\needspace{15\baselineskip}
\subsection*{Consommé au Parmesan}

Consommé, served with Parmesan Pâte à Chou.



\needspace{15\baselineskip}
\subsection*{Consommé Colbert}

To six cups Consommé add one-third cup each of cooked green peas,
flageolets, carrots cut in small cubes, and celery cut in small pieces.
Serve a poached egg in each plate of soup.



\needspace{15\baselineskip}
\subsection*{Consommé aux Pâtes}

Consommé, served with noodles, macaroni, spaghetti, or any Italian
pastes, first cooked in boiling salted water.



\needspace{15\baselineskip}
\subsection*{Consommé d'Orleans}

Consommé, served with red and white quenelles and French peas.



\needspace{15\baselineskip}
\subsection*{Consommé with Vegetables}

Consommé, served with French string beans, and cooked carrots cut in
fancy shapes with French vegetable cutters.



\needspace{15\baselineskip}
\subsection*{Consommé Princess}

Consommé, served with green peas and cooked chicken meat cut in small
dice.



\needspace{15\baselineskip}
\subsection*{Claret Consommé}

To one quart Consommé add one and one-half cups claret, which has been
cooked with a three-inch piece stick cinnamon ten minutes and one
tablespoon sugar. Color red.



\needspace{15\baselineskip}
\subsection*{Bortchock Consommé}

Make same as Consommé, adding one-third cup chopped beets with
vegetables; then add one cup finely chopped beets when clearing.



\needspace{15\baselineskip}
\section*{Soups With Fish Stock}


\needspace{15\baselineskip}
\subsection*{Clam Bouillon}

Wash and scrub with a brush one-half peck clams, changing the water
several times. Put in kettle with three cups cold water, cover tightly,
and steam until shells are well opened. Strain liquor, cool, and clear.



\needspace{15\baselineskip}
\subsection*{Oyster Stew}


\begin{itemize}
\setlength{\itemsep}{0pt}
\setlength{\parsep}{0pt}
\item 1 quart oysters
\item 4 cups scalded milk
\item 1/4 cup butter
\item 1/2 tablespoon salt
\item 1/8 teaspoon pepper
\end{itemize}

\vspace{-0.5em}
\noindent%
Clean oysters by placing in a colander and pouring over them
three-fourths cup cold water. Carefully pick over oysters, reserve
liquor, and heat it to boiling-point; strain through double
cheese-cloth, add oysters, and cook until oysters are plump and edges
begin to curl. Remove oysters with skimmer, and put in tureen with
butter, salt, and pepper. Add oyster liquor strained a second time, and
milk. Serve with oyster crackers.



\needspace{15\baselineskip}
\subsection*{Scallop Stew}

Make same as Oyster Stew, using one quart scallops in place of oysters.



\needspace{15\baselineskip}
\subsection*{Oyster Soup}


\begin{minipage}{1.0\textwidth}
{\setlength{\multicolsep}{0pt}\setlength{\columnsep}{2em}\raggedcolumns%
\begin{multicols}{2}
\begin{itemize}
\setlength{\itemsep}{0pt}
\setlength{\parsep}{0pt}
\item 1 quart oysters
\item 4 cups milk
\item 1 slice onion
\item 2 stalks celery
\item 2 blades mace
\item Sprig of parsley
\item Bit of bay leaf
\item 1/3 cup butter
\item 1/3 cup flour
\item Salt and pepper
\end{itemize}
\end{multicols}}
\end{minipage}

\vspace{0.3em}
\noindent%
Clean and pick over oysters as for Oyster Stew; reserve liquor, add
oysters slightly chopped, heat slowly to boiling-point, and let simmer
twenty minutes. Strain through cheese-cloth, reheat liquor, and thicken
with butter and flour cooked together. Scald milk with onion, celery,
mace, parsley, and bay leaf; remove seasonings, and add to oyster
liquor. Season with salt and pepper.



\needspace{15\baselineskip}
\subsection*{French Oyster Soup}


\begin{minipage}{1.0\textwidth}
{\setlength{\multicolsep}{0pt}\setlength{\columnsep}{2em}\raggedcolumns%
\begin{multicols}{2}
\begin{itemize}
\setlength{\itemsep}{0pt}
\setlength{\parsep}{0pt}
\item 1 quart oysters
\item 4 cups milk
\item 1 slice onion
\item 2 blades mace
\item 1/3 cup butter
\item 1/3 cup flour
\item 4 egg yolks
\item Salt and pepper
\end{itemize}
\end{multicols}}
\end{minipage}

\vspace{0.3em}
\noindent%
Make same as Oyster Soup, adding yolks of eggs, slightly beaten, just
before serving. Garnish with Fish Quenelles.



\needspace{15\baselineskip}
\subsection*{Oyster Soup, Amsterdam Style}


\begin{minipage}{1.0\textwidth}
{\setlength{\multicolsep}{0pt}\setlength{\columnsep}{2em}\raggedcolumns%
\begin{multicols}{2}
\begin{itemize}
\setlength{\itemsep}{0pt}
\setlength{\parsep}{0pt}
\item 1 quart oysters
\item Water
\item 3 tablespoons butter
\item 3 1/2 tablespoons flour
\item 1/2 teaspoon salt
\item Paprika
\item Celery salt
\item 1 cup cream
\end{itemize}
\end{multicols}}
\end{minipage}

\vspace{0.3em}
\noindent%
Clean, pick over, chop, and parboil oysters; drain and add to liquor
enough water to make one quart liquid. Brown butter, add flour, and pour
on gradually, while stirring constantly, oyster liquor. Let simmer
one-half hour. Season with salt, paprika, and celery salt, and just
before serving add cream.



\needspace{15\baselineskip}
\subsection*{Oyster Gumbo}


\begin{minipage}{1.0\textwidth}
{\setlength{\multicolsep}{0pt}\setlength{\columnsep}{2em}\raggedcolumns%
\begin{multicols}{2}
\begin{itemize}
\setlength{\itemsep}{0pt}
\setlength{\parsep}{0pt}
\item 1 pint oysters
\item 4 cups Fish Stock
\item 1/4 cup butter
\item 1 tablespoon chopped onion
\item 1/2 can okra
\item 1/3 can tomatoes
\item Salt
\item Pepper
\end{itemize}
\end{multicols}}
\end{minipage}

\vspace{0.3em}
\noindent%
Clean, pick over, and parboil oysters; drain, and add oyster liquor to
Fish Stock. Cook onion five minutes in one-half the butter; add to
stock. Then add okra, tomatoes heated and drained from some of their
liquor, oysters, and remaining butter. Season with salt and pepper.

\textbf{Fish Stock} is the liquor obtained by covering the head, tail, skin,
bones, and small quantity of flesh adhering to bones of fish, with cold
water, bringing slowly to boiling-point, simmering thirty minutes, and
straining.



\needspace{15\baselineskip}
\subsection*{Clam Soup with Poached Eggs}


\begin{minipage}{1.0\textwidth}
{\setlength{\multicolsep}{0pt}\setlength{\columnsep}{2em}\raggedcolumns%
\begin{multicols}{2}
\begin{itemize}
\setlength{\itemsep}{0pt}
\setlength{\parsep}{0pt}
\item 1 quart clams
\item 4 cups milk
\item 1 slice onion
\item 1/3 cup butter
\item 1/8 cup flour
\item 1 1/2 teaspoons salt
\item 1/8 teaspoon pepper
\item Few gratings nutmeg
\item 2 egg whites
\end{itemize}
\end{multicols}}
\end{minipage}

\vspace{0.3em}
\noindent%
Clean and pick over clams, using three-fourths cup cold water; reserve
liquor. Put aside soft part of clams; finely chop hard part, add to
liquor, bring gradually to boiling-point, strain, then thicken with
butter and flour cooked together. Scald milk with onion, remove onion,
add milk and soft part of clams to stock; cook two minutes. Add
seasonings, and pour over whites of eggs beaten stiff.



\needspace{15\baselineskip}
\subsection*{Clam and Oyster Soup}


\begin{minipage}{1.0\textwidth}
{\setlength{\multicolsep}{0pt}\setlength{\columnsep}{2em}\raggedcolumns%
\begin{multicols}{2}
\begin{itemize}
\setlength{\itemsep}{0pt}
\setlength{\parsep}{0pt}
\item 1 pint clams
\item 1 pint oysters
\item 4 cups milk
\item 1 slice onion
\item 2 blades mace
\item Sprig of parsley
\item Bit of bay leaf
\item 1/3 cup butter
\item 1/3 cup flour
\item Salt and pepper
\end{itemize}
\end{multicols}}
\end{minipage}

\vspace{0.3em}
\noindent%
Clean and pick over oysters, using one-third cup cold water; reserve
liquor, and add oysters slightly chopped. Clean and pick over clams,
reserve liquor, and add to hard part of clams, finely chopped; put aside
soft part of clams. Heat slowly to boiling-point clams and oysters with
liquor from both, let simmer twenty minutes and strain through
cheese-cloth. Scald milk with onion, mace, parsley, and bay leaf; remove
seasonings, and add milk to stock. Thicken with butter and flour cooked
together, add soft part of clams, and cook two minutes. Season with salt
and pepper.



\needspace{15\baselineskip}
\subsection*{Cream of Clam Soup}

Make same as French Oyster Soup, using clams in place of oysters.



\needspace{15\baselineskip}
\subsection*{Clam Consommé}

Wash two quarts clams in shell. Put in kettle with one-fourth cup cold
water, cover, and cook until shells open. Strain liquor through double
thickness cheese-cloth, add to four cups consommé, and clear.



\needspace{15\baselineskip}
\subsection*{Clam and Chicken Frappé}

Wash and scrub with a brush two quarts clams, changing water several
times. Put in kettle with one-half cup cold water, cover tightly, and
steam until shells are well opened. Remove clams from shells and strain
liquor through double thickness cheese-cloth. To one and two-thirds cups
clam liquor add two and one-half cups White Stock III, highly seasoned.
Cool, and freeze to a mush. Serve in place of a soup in frappé glasses,
and garnish with whipped cream.



\needspace{15\baselineskip}
\subsection*{Clam and Tomato Bisque}


\begin{minipage}{1.0\textwidth}
{\setlength{\multicolsep}{0pt}\setlength{\columnsep}{2em}\raggedcolumns%
\begin{multicols}{2}
\begin{itemize}
\setlength{\itemsep}{0pt}
\setlength{\parsep}{0pt}
\item 1 quart clams
\item 1 1/2 cups cold water
\item 1/3 cup butter
\item 1/3 cup flour
\item 1/2 onion
\item 2 cups cream
\item 1 cup stewed and strained tomatoes
\item 1/8 teaspoon soda
\item Salt
\item Cayenne
\end{itemize}
\end{multicols}}
\end{minipage}

\vspace{0.3em}
\noindent%
Pour water over clams, then drain. To water add hard part of clams
finely chopped. Heat slowly to boiling-point, cook twenty minutes, then
strain. Cook butter with onion five minutes; remove onion, add flour and
gradually clam water. Add cream, soft part of clams, and as soon as
boiling-point is reached, tomatoes to which soda has been added. Season
with salt and cayenne, and serve at once.



\needspace{15\baselineskip}
\subsection*{Oyster Bisque}


\begin{minipage}{1.0\textwidth}
{\setlength{\multicolsep}{0pt}\setlength{\columnsep}{2em}\raggedcolumns%
\begin{multicols}{2}
\begin{itemize}
\setlength{\itemsep}{0pt}
\setlength{\parsep}{0pt}
\item 1 quart oysters
\item 2 cups White Stock III
\item 1 1/2 cups stale bread crumbs
\item 1 slice onion
\item 2 stalks celery
\item Sprig of parsley
\item Bit of bay leaf
\item 2 tablespoons butter
\item 2 tablespoons flour
\item 4 cups scalded milk
\item Salt
\item Pepper
\end{itemize}
\end{multicols}}
\end{minipage}

\vspace{0.3em}
\noindent%
Clean and pick over oysters, reserving liquor, setting aside soft
portions, and chopping gills and tough muscles. Cook White Stock, bread
crumbs, reserved liquor, chopped oyster, onion, celery, parsley, and bay
leaf thirty minutes. Rub through a sieve, bring to boiling-point, and
bind with butter and flour cooked together. Add milk, soft portion of
oysters, and salt and pepper to taste.



\needspace{15\baselineskip}
\subsection*{Cream of Scallop Soup}


\begin{minipage}{1.0\textwidth}
{\setlength{\multicolsep}{0pt}\setlength{\columnsep}{2em}\raggedcolumns%
\begin{multicols}{2}
\begin{itemize}
\setlength{\itemsep}{0pt}
\setlength{\parsep}{0pt}
\item 1 quart scallops
\item 4 cups milk
\item 2 cloves
\item Bit of bay leaf
\item 1/4 teaspoon peppercorns
\item 1 tablespoon chopped onion
\item 5 tablespoons butter
\item 1/4 cup flour
\item Salt
\item Pepper
\end{itemize}
\end{multicols}}
\end{minipage}

\vspace{0.3em}
\noindent%
Clean scallops, reserve one-half cup and finely chop remainder. Add
these to milk, with seasonings and two tablespoons butter, and cook
slowly twenty minutes. Strain and thicken with remaining butter and
flour cooked together. Parboil reserved scallops, and add to soup. Serve
with small biscuits or oysterettes.



\needspace{15\baselineskip}
\subsection*{Lobster Bisque}


\begin{minipage}{1.0\textwidth}
{\setlength{\multicolsep}{0pt}\setlength{\columnsep}{2em}\raggedcolumns%
\begin{multicols}{2}
\begin{itemize}
\setlength{\itemsep}{0pt}
\setlength{\parsep}{0pt}
\item 2 lb. lobster
\item 2 cups cold water
\item 4 cups milk
\item 1/4 cup butter
\item 1/4 cup flour
\item 1 1/2 teaspoons salt
\item Few grains of cayenne
\end{itemize}
\end{multicols}}
\end{minipage}

\vspace{0.3em}
\noindent%
Remove meat from lobster shell. Add cold water to body bones and tough
end of claws, cut in pieces; bring slowly to boiling-point, and cook
twenty minutes. Drain, reserve liquor, and thicken with butter and flour
cooked together. Scald milk with tail meat of lobster, finely chopped;
strain, and add to liquor. Season with salt and cayenne; then add tender
claw meat, cut in dice, and body meat. When coral is found in lobster,
wash, wipe, force through fine strainer, put in a mortar with butter,
work until well blended, then add flour, and stir into soup. If a richer
soup is desired, White Stock may be used in place of water.









\chapter{Soups Without Stock}




\needspace{15\baselineskip}
\section*{Black Bean Soup}


\begin{minipage}{1.0\textwidth}
{\setlength{\multicolsep}{0pt}\setlength{\columnsep}{2em}\raggedcolumns%
\begin{multicols}{2}
\begin{itemize}
\setlength{\itemsep}{0pt}
\setlength{\parsep}{0pt}
\item 1 pint black beans
\item 2 quarts cold water
\item 1 small onion
\item 2 stalks celery, or
\item 1/4 teaspoon celery salt
\item 1/2 tablespoon salt
\item 1/8 teaspoon pepper
\item 1/4 teaspoon mustard
\item Few grains cayenne
\item 3 tablespoons butter
\item 1 1/2 tablespoons flour
\item 2 “hard-boiled” eggs
\item 1 lemon
\end{itemize}
\end{multicols}}
\end{minipage}

\vspace{0.3em}
\noindent%
Soak beans over night; in the morning drain and add cold water. Slice
onion, and cook five minutes with half the butter, adding to beans, with
celery stalks broken in pieces. Simmer three or four hours, or until
beans are soft; add more water as water boils away. Rub through a sieve,
reheat to the boiling-point, and add salt, pepper, mustard, and cayenne
well mixed. Bind with remaining butter and flour cooked together. Cut
eggs in thin slices, and lemon in thin slices, removing seeds. Put in
tureen, and strain the soup over them.



\needspace{15\baselineskip}
\section*{Baked Bean Soup}


\begin{minipage}{1.0\textwidth}
{\setlength{\multicolsep}{0pt}\setlength{\columnsep}{2em}\raggedcolumns%
\begin{multicols}{2}
\begin{itemize}
\setlength{\itemsep}{0pt}
\setlength{\parsep}{0pt}
\item 3 cups cold baked beans
\item 3 pints water
\item 2 slices onion
\item 2 stalks celery
\item 1 1/2 cups stewed and strained tomatoes
\item 2 tablespoons butter
\item 2 tablespoons flour
\item 1 tablespoon Chili sauce
\item Salt
\item Pepper
\end{itemize}
\end{multicols}}
\end{minipage}

\vspace{0.3em}
\noindent%
Put beans, water, onion, and celery in saucepan; bring to boiling-point
and simmer thirty minutes. Rub through a sieve, add tomato, and Chili
sauce, season to taste with salt and pepper, and bind with the butter
and flour cooked together. Serve with Crisp Crackers.



\needspace{15\baselineskip}
\section*{Cream Of Lima Bean Soup}


\begin{minipage}{1.0\textwidth}
{\setlength{\multicolsep}{0pt}\setlength{\columnsep}{2em}\raggedcolumns%
\begin{multicols}{2}
\begin{itemize}
\setlength{\itemsep}{0pt}
\setlength{\parsep}{0pt}
\item 1 cup dried lima beans
\item 3 pints cold water
\item 2 slices onion
\item 4 slices carrot
\item 1 cup cream or milk
\item 4 tablespoons butter
\item 2 tablespoons flour
\item 1 teaspoon salt
\item 1/2 teaspoon pepper
\end{itemize}
\end{multicols}}
\end{minipage}

\vspace{0.3em}
\noindent%
Soak beans over night; in the morning drain and add cold water; cook
until soft, and rub through a sieve. Cut vegetables in small cubes, and
cook five minutes in half the butter; remove vegetables, add flour,
salt, and pepper, and stir into boiling soup. Add cream, reheat, strain,
and add remaining butter in small pieces.



\needspace{15\baselineskip}
\section*{Cream Of Artichoke Soup}


\begin{minipage}{1.0\textwidth}
{\setlength{\multicolsep}{0pt}\setlength{\columnsep}{2em}\raggedcolumns%
\begin{multicols}{2}
\begin{itemize}
\setlength{\itemsep}{0pt}
\setlength{\parsep}{0pt}
\item 6 artichokes
\item 4 cups boiling water
\item 2 tablespoons butter
\item 2 tablespoons flour
\item 1 1/2 teaspoons salt
\item Few grains cayenne
\item Few gratings nutmeg
\item 2 tablespoons Sauterne wine
\item 1 cup scalded cream
\item 1 egg
\item 2 cucumbers
\end{itemize}
\end{multicols}}
\end{minipage}

\vspace{0.3em}
\noindent%
Cook artichokes in boiling water until soft, and rub through a sieve.
Melt butter, add flour and seasonings, pour on hot liquor, and cook one
minute. Add cream, wine, and egg slightly beaten. Pare cucumbers, cut in
one-third inch cubes, sauté in butter, and add to soup. Jerusalem
artichokes are used for the making of this soup.



\needspace{15\baselineskip}
\section*{Celery Soup I}


\begin{minipage}{1.0\textwidth}
{\setlength{\multicolsep}{0pt}\setlength{\columnsep}{2em}\raggedcolumns%
\begin{multicols}{2}
\begin{itemize}
\setlength{\itemsep}{0pt}
\setlength{\parsep}{0pt}
\item 3 cups celery (cut in one-half inch pieces)
\item 1 pint boiling water
\item 2 1/2 cups milk
\item 1 slice onion
\item 3 tablespoons butter
\item 1/4 cup flour
\item Salt and pepper
\end{itemize}
\end{multicols}}
\end{minipage}

\vspace{0.3em}
\noindent%
Wash and scrape celery before cutting in pieces, cook in boiling water
until soft, and rub through a sieve. Scald milk with the onion, remove
onion, and add milk to celery. Bind with butter and flour cooked
together. Season with salt and pepper. Outer and old stalks of celery
may be utilized for soups. Serve with croûtons, crisp crackers, or
pulled bread.



\needspace{15\baselineskip}
\section*{Celery Soup II}


\begin{minipage}{1.0\textwidth}
{\setlength{\multicolsep}{0pt}\setlength{\columnsep}{2em}\raggedcolumns%
\begin{multicols}{2}
\begin{itemize}
\setlength{\itemsep}{0pt}
\setlength{\parsep}{0pt}
\item 3 stalks celery
\item 3 cups milk
\item 1 slice onion
\item 3 tablespoons butter
\item 3 tablespoons flour
\item Salt and pepper
\item 1 cup cream
\end{itemize}
\end{multicols}}
\end{minipage}

\vspace{0.3em}
\noindent%
Break celery in one-inch pieces, and pound in a mortar. Cook in double
boiler with onion and milk twenty minutes. Thicken with butter and flour
cooked together. Season with salt and pepper, add cream, strain into
tureen, and serve at once.



\needspace{15\baselineskip}
\section*{Corn Soup}


\begin{minipage}{1.0\textwidth}
{\setlength{\multicolsep}{0pt}\setlength{\columnsep}{2em}\raggedcolumns%
\begin{multicols}{2}
\begin{itemize}
\setlength{\itemsep}{0pt}
\setlength{\parsep}{0pt}
\item 1 can corn
\item 1 pint boiling water
\item 1 pint milk
\item 1 slice onion
\item 2 tablespoons butter
\item 2 tablespoons flour
\item 1 teaspoon salt
\item Few grains pepper
\end{itemize}
\end{multicols}}
\end{minipage}

\vspace{0.3em}
\noindent%
Chop the corn, add water, and simmer twenty minutes; rub through a
sieve. Scald milk with onion, remove onion, and add milk to corn. Bind
with butter and flour cooked together. Add salt and pepper.



\needspace{15\baselineskip}
\section*{Halibut Soup}


\begin{minipage}{1.0\textwidth}
{\setlength{\multicolsep}{0pt}\setlength{\columnsep}{2em}\raggedcolumns%
\begin{multicols}{2}
\begin{itemize}
\setlength{\itemsep}{0pt}
\setlength{\parsep}{0pt}
\item 3/4 cup cold boiled halibut
\item 1 pint milk
\item 1 slice onion
\item Blade of mace
\item 3 tablespoons butter
\item 1 1/2 tablespoons flour
\item 1/2 teaspoon salt
\item Few grains pepper
\end{itemize}
\end{multicols}}
\end{minipage}

\vspace{0.3em}
\noindent%
Rub fish through a sieve. Scald milk with onion and mace. Remove
seasonings, and add fish. Bind with half the butter and flour cooked
together. Add salt, pepper, and the remaining butter in small pieces.



\needspace{15\baselineskip}
\section*{Pea Soup}


\begin{minipage}{1.0\textwidth}
{\setlength{\multicolsep}{0pt}\setlength{\columnsep}{2em}\raggedcolumns%
\begin{multicols}{2}
\begin{itemize}
\setlength{\itemsep}{0pt}
\setlength{\parsep}{0pt}
\item 1 can Marrowfat peas
\item 2 teaspoons sugar
\item 1 pint cold water
\item 1 pint milk
\item 1 slice onion
\item 2 tablespoons butter
\item 2 tablespoons flour
\item 1 teaspoon salt
\item 1/8 teaspoon pepper
\end{itemize}
\end{multicols}}
\end{minipage}

\vspace{0.3em}
\noindent%
Drain peas from their liquor, add sugar and cold water, and simmer
twenty minutes. Rub through a sieve, reheat, and thicken with butter and
flour cooked together. Scald milk with onion, remove onion, and add milk
to pea mixture, season with salt and pepper. Peas too old to serve as a
vegetable may be utilized for soups.



\needspace{15\baselineskip}
\section*{Split Pea Soup}


\begin{minipage}{1.0\textwidth}
{\setlength{\multicolsep}{0pt}\setlength{\columnsep}{2em}\raggedcolumns%
\begin{multicols}{2}
\begin{itemize}
\setlength{\itemsep}{0pt}
\setlength{\parsep}{0pt}
\item 1 cup dried split peas
\item 2 1/2 quarts cold water
\item 1 pint milk
\item 1/2 onion
\item 3 tablespoons butter
\item 2 tablespoons flour
\item 1 1/2 teaspoons salt
\item 1/8 teaspoon pepper
\item 2--inch cube fat salt pork
\end{itemize}
\end{multicols}}
\end{minipage}

\vspace{0.3em}
\noindent%
Pick over peas and soak several hours, drain, add cold water, pork, and
onion. Simmer three or four hours, or until soft; rub through a sieve.
Add butter and flour cooked together, salt, and pepper. Dilute with
milk, adding more if necessary. The water in which a ham has been cooked
may be used; in such case omit salt.



\needspace{15\baselineskip}
\section*{Kornlet Soup}


\begin{minipage}{1.0\textwidth}
{\setlength{\multicolsep}{0pt}\setlength{\columnsep}{2em}\raggedcolumns%
\begin{multicols}{2}
\begin{itemize}
\setlength{\itemsep}{0pt}
\setlength{\parsep}{0pt}
\item 1 can kornlet
\item 1 pint cold water
\item 1 quart milk, scalded
\item 4 tablespoons butter
\item 1 tablespoon chopped onion
\item 4 tablespoons flour
\item 1 1/2 teaspoons salt
\item Few grains pepper
\end{itemize}
\end{multicols}}
\end{minipage}

\vspace{0.3em}
\noindent%
Cook kornlet in cold water twenty minutes; rub through a sieve, and add
milk. Fry butter and onion three minutes; remove onion, add flour, salt,
and pepper, and stir into boiling soup.



\needspace{15\baselineskip}
\section*{Potato Soup}


\begin{minipage}{1.0\textwidth}
{\setlength{\multicolsep}{0pt}\setlength{\columnsep}{2em}\raggedcolumns%
\begin{multicols}{2}
\begin{itemize}
\setlength{\itemsep}{0pt}
\setlength{\parsep}{0pt}
\item 3 potatoes
\item 1 quart milk
\item 2 slices onion
\item 3 tablespoons butter
\item 2 tablespoons flour
\item 1 1/2 teaspoons salt
\item 1/4 teaspoon celery salt
\item 1/8 teaspoon pepper
\item Few grains cayenne
\item 1 teaspoon chopped parsley
\end{itemize}
\end{multicols}}
\end{minipage}

\vspace{0.3em}
\noindent%
Cook potatoes in boiling salted water; when soft, rub through a
strainer. Scald milk with onion, remove onion, and add milk slowly to
potatoes. Melt half the butter, add dry ingredients, stir until well
mixed, then stir into boiling soup; cook one minute, strain, add
remaining butter, and sprinkle with parsley.



\needspace{15\baselineskip}
\section*{Appledore Soup}

Make same as Potato Soup, and add, just before serving, three
tablespoons tomato catsup.



\needspace{15\baselineskip}
\section*{Swiss Potato Soup}


\begin{minipage}{1.0\textwidth}
{\setlength{\multicolsep}{0pt}\setlength{\columnsep}{2em}\raggedcolumns%
\begin{multicols}{2}
\begin{itemize}
\setlength{\itemsep}{0pt}
\setlength{\parsep}{0pt}
\item 4 small potatoes
\item 1 large flat white turnip
\item 3 cups boiling water
\item 1 quart scalded milk
\item 1/2 onion
\item 4 tablespoons butter
\item 1/3 cup flour
\item 1 1/2 teaspoons salt
\item 1/8 teaspoon pepper
\end{itemize}
\end{multicols}}
\end{minipage}

\vspace{0.3em}
\noindent%
Wash, pare, and cut potatoes in halves. Wash, pare, and cut turnips in
one-quarter inch slices. Parboil together ten minutes, drain, add onion
cut in slices, and three cups boiling water. Cook until vegetables are
soft; drain, reserving the water to add to vegetables after rubbing them
through a sieve. Add milk, reheat, and bind with butter and flour cooked
together. Season with salt and pepper.



\needspace{15\baselineskip}
\section*{Leek And Potato Soup}


\begin{minipage}{1.0\textwidth}
{\setlength{\multicolsep}{0pt}\setlength{\columnsep}{2em}\raggedcolumns%
\begin{multicols}{2}
\begin{itemize}
\setlength{\itemsep}{0pt}
\setlength{\parsep}{0pt}
\item 1 bunch leeks
\item 1 cup celery
\item 2 1/2 tablespoons butter
\item 1 quart milk
\item 2 1/2 cups potatoes
\item 2 tablespoons butter
\item 2 tablespoons flour
\item Salt and pepper
\item Cayenne
\end{itemize}
\end{multicols}}
\end{minipage}

\vspace{0.3em}
\noindent%
Cut leeks and celery in very thin slices crosswise and cook in two and
one-half tablespoons butter, stirring constantly, ten minutes. Add milk,
and cook in double boiler forty minutes. Cut potatoes in slices and cut
slices in small pieces; then cook in boiling salted water ten minutes.
Melt two tablespoons butter, add flour, milk with vegetables and
potatoes. Cook until potatoes are soft, and season with salt, pepper,
and cayenne.



\needspace{15\baselineskip}
\section*{Vegetable Soup}


\begin{minipage}{1.0\textwidth}
{\setlength{\multicolsep}{0pt}\setlength{\columnsep}{2em}\raggedcolumns%
\begin{multicols}{2}
\begin{itemize}
\setlength{\itemsep}{0pt}
\setlength{\parsep}{0pt}
\item 1/3 cup carrot
\item 1/3 cup turnip
\item 1/2 cup celery
\item 1 1/2 cups potato
\item 1/2 onion
\item 1 quart water
\item 5 tablespoons butter
\item 1/2 tablespoon finely chopped parsley
\item Salt and pepper
\end{itemize}
\end{multicols}}
\end{minipage}

\vspace{0.3em}
\noindent%
Wash and scrape a small carrot; cut in quarters lengthwise; cut quarters
in thirds lengthwise; cut strips thus made in thin slices crosswise.
Wash and pare half a turnip, and cut and slice same as carrot. Wash,
pare, and cut potatoes in small pieces. Wash and scrape celery and cut
in quarter-inch pieces. Prepare vegetables before measuring. Cut onion
in thin slices. Mix vegetables (except potatoes), and cook ten minutes,
in four tablespoons butter, stirring constantly. Add potatoes, cover,
and cook two minutes. Add water, and boil one hour. Beat with spoon or
fork to break vegetables. Add remaining butter and parsley. Season with
salt and pepper.



\needspace{15\baselineskip}
\section*{Salmon Soup}


\begin{minipage}{1.0\textwidth}
{\setlength{\multicolsep}{0pt}\setlength{\columnsep}{2em}\raggedcolumns%
\begin{multicols}{2}
\begin{itemize}
\setlength{\itemsep}{0pt}
\setlength{\parsep}{0pt}
\item 1/3 can salmon
\item 1 quart scalded milk
\item 2 tablespoons butter
\item 4 tablespoons flour
\item 1 1/2 teaspoons salt
\item Few grains pepper
\end{itemize}
\end{multicols}}
\end{minipage}

\vspace{0.3em}
\noindent%
Drain oil from salmon, remove skin and bones, rub through a sieve. Add
gradually the milk, season, and bind.



\needspace{15\baselineskip}
\section*{Squash Soup}


\begin{minipage}{1.0\textwidth}
{\setlength{\multicolsep}{0pt}\setlength{\columnsep}{2em}\raggedcolumns%
\begin{multicols}{2}
\begin{itemize}
\setlength{\itemsep}{0pt}
\setlength{\parsep}{0pt}
\item 3/4 cup cooked squash
\item 1 quart milk
\item 1 slice onion
\item 2 tablespoons butter
\item 3 tablespoons flour
\item 1 teaspoon salt
\item Few grains pepper
\item 1/4 teaspoon celery salt
\end{itemize}
\end{multicols}}
\end{minipage}

\vspace{0.3em}
\noindent%
Rub squash through a sieve before measuring. Scald milk with onion,
remove onion, and add milk to squash; season, and bind.



\needspace{15\baselineskip}
\section*{Tomato Soup}


\begin{minipage}{1.0\textwidth}
{\setlength{\multicolsep}{0pt}\setlength{\columnsep}{2em}\raggedcolumns%
\begin{multicols}{2}
\begin{itemize}
\setlength{\itemsep}{0pt}
\setlength{\parsep}{0pt}
\item 1 can tomatoes
\item 1 pint water
\item 12 peppercorns
\item Bit of bay leaf
\item 4 cloves
\item 2 teaspoons sugar
\item 1 teaspoon salt
\item 1/8 teaspoon soda
\item 2 tablespoons butter
\item 3 tablespoons flour
\item 1 slice onion
\end{itemize}
\end{multicols}}
\end{minipage}

\vspace{0.3em}
\noindent%
Cook tomatoes, water, peppercorns, bay leaf, cloves, and sugar twenty
minutes; strain, and add salt and soda; bind, and strain into tureen.



\needspace{15\baselineskip}
\section*{Cream Of Tomato Soup}


\begin{minipage}{1.0\textwidth}
{\setlength{\multicolsep}{0pt}\setlength{\columnsep}{2em}\raggedcolumns%
\begin{multicols}{2}
\begin{itemize}
\setlength{\itemsep}{0pt}
\setlength{\parsep}{0pt}
\item 1/2 can tomatoes
\item 2 teaspoons sugar
\item 1/4 teaspoon soda
\item 1 quart milk
\item 1 slice onion
\item 4 tablespoons flour
\item 1 teaspoon salt
\item 1/8 teaspoon pepper
\item 1/3 cup butter
\end{itemize}
\end{multicols}}
\end{minipage}

\vspace{0.3em}
\noindent%
Scald milk with onion, remove onion, and thicken milk with flour diluted
with cold water until thin enough to pour, being careful that the
mixture is free from lumps; cook twenty minutes, stirring constantly at
first. Cook tomatoes with sugar fifteen minutes, add soda, and rub
through a sieve; combine mixtures, and strain into tureen over butter,
salt, and pepper.



\needspace{15\baselineskip}
\section*{Mock Bisque Soup}


\begin{minipage}{1.0\textwidth}
{\setlength{\multicolsep}{0pt}\setlength{\columnsep}{2em}\raggedcolumns%
\begin{multicols}{2}
\begin{itemize}
\setlength{\itemsep}{0pt}
\setlength{\parsep}{0pt}
\item 1/2 can tomatoes
\item 2 teaspoons sugar
\item 1/4 teaspoon soda
\item 1/2 onion, stuck with 6 cloves
\item Sprig of parsley
\item Bit of bay leaf
\item 3/4 cup stale bread crumbs
\item 4 cups milk
\item 1/2 tablespoon salt
\item 1/8 teaspoon pepper
\item 1/3 cup butter
\end{itemize}
\end{multicols}}
\end{minipage}

\vspace{0.3em}
\noindent%
Scald milk with bread crumbs, onion, parsley, and bay leaf. Remove
seasonings and rub through a sieve. Cook tomatoes with sugar fifteen
minutes; add soda and rub through a sieve. Reheat bread and milk to
boiling-point, add tomatoes, and pour at once into tureen over butter,
salt, and pepper. Serve with croûtons, crisp crackers, or Souffléd
crackers.



\needspace{15\baselineskip}
\section*{Tapioca Wine Soup}


\begin{minipage}{1.0\textwidth}
{\setlength{\multicolsep}{0pt}\setlength{\columnsep}{2em}\raggedcolumns%
\begin{multicols}{2}
\begin{itemize}
\setlength{\itemsep}{0pt}
\setlength{\parsep}{0pt}
\item 1/3 cup pearl tapioca
\item 1 cup cold water
\item 3 cups boiling water
\item 1/2 teaspoon salt
\item 3--inch piece stick cinnamon
\item 1 pint claret wine
\item 1/2 cup powdered sugar
\end{itemize}
\end{multicols}}
\end{minipage}

\vspace{0.3em}
\noindent%
Soak tapioca in cold water two hours. Drain, add to boiling water with
salt and cinnamon; let boil three minutes, then cook in double boiler
until tapioca is transparent. Cool, add wine and sugar. Serve very cold.



\needspace{15\baselineskip}
\section*{Chowders}


\needspace{15\baselineskip}
\subsection*{Corn Chowder}


\begin{minipage}{1.0\textwidth}
{\setlength{\multicolsep}{0pt}\setlength{\columnsep}{2em}\raggedcolumns%
\begin{multicols}{2}
\begin{itemize}
\setlength{\itemsep}{0pt}
\setlength{\parsep}{0pt}
\item 1 can corn
\item 4 cups potatoes, cut in 1/4-inch slices
\item 1 1/2-inch cube fat salt pork
\item 1 sliced onion
\item 4 cups scalded milk
\item 8 common crackers
\item 3 tablespoons butter
\item Salt and pepper
\end{itemize}
\end{multicols}}
\end{minipage}

\vspace{0.3em}
\noindent%
Cut pork in small pieces and try out; add onion and cook five minutes,
stirring often that onion may not burn; strain fat into a stewpan.
Parboil potatoes five minutes in boiling water to cover; drain, and add
potatoes to fat; then add two cups boiling water; cook until potatoes
are soft, add corn and milk, then heat to boiling-point. Season with
salt and pepper; add butter, and crackers split and soaked in enough
cold milk to moisten. Remove crackers, turn chowder into a tureen, and
put crackers on top.



\needspace{15\baselineskip}
\subsection*{Fish Chowder}


\begin{minipage}{1.0\textwidth}
{\setlength{\multicolsep}{0pt}\setlength{\columnsep}{2em}\raggedcolumns%
\begin{multicols}{2}
\begin{itemize}
\setlength{\itemsep}{0pt}
\setlength{\parsep}{0pt}
\item 4 lb. cod or haddock
\item 6 cups potatoes cut in 1/4-inch slices, or
\item 4 cups potatoes cut in 3/4-inch cubes
\item 1 sliced onion
\item 1 1/2-inch cube fat salt pork
\item 1 tablespoon salt
\item 1/8 teaspoon pepper
\item 3 tablespoons butter
\item 4 cups scalded milk
\item 8 common crackers
\end{itemize}
\end{multicols}}
\end{minipage}

\vspace{0.3em}
\noindent%
Order the fish skinned, but head and tail left on. Cut off head and tail
and remove fish from backbone. Cut fish in two-inch pieces and set
aside. Put head, tail, and backbone broken in pieces, in stewpan; add
two cups cold water and bring slowly to boiling-point; cook twenty
minutes. Cut salt pork in small pieces and try out, add onion, and fry
five minutes; strain fat into stewpan. Parboil potatoes five minutes in
boiling water to cover; drain and add potatoes to fat; then add two cups
boiling water and cook five minutes. Add liquor drained from bones, then
add the fish; cover, and simmer ten minutes. Add milk, salt, pepper,
butter, and crackers split and soaked in enough cold milk to moisten,
otherwise they will be soft on the outside, but dry on the inside. Pilot
bread is sometimes used in place of common crackers.



\needspace{15\baselineskip}
\subsection*{Connecticut Chowder}


\begin{minipage}{1.0\textwidth}
{\setlength{\multicolsep}{0pt}\setlength{\columnsep}{2em}\raggedcolumns%
\begin{multicols}{2}
\begin{itemize}
\setlength{\itemsep}{0pt}
\setlength{\parsep}{0pt}
\item 4 lb. cod or haddock
\item 4 cups potatoes cut in 3/4-inch cubes
\item 1 1/2-inch cube fat salt pork
\item 1 sliced onion
\item 2 1/2 cups stewed and strained tomatoes
\item 3 tablespoons butter
\item 2/3 cup cracker crumbs
\item Salt and pepper
\end{itemize}
\end{multicols}}
\end{minipage}

\vspace{0.3em}
\noindent%
Prepare same as Fish Chowder, using liquor drained from bones for
cooking potatoes, instead of additional water. Use tomatoes in place of
milk and add cracker crumbs just before serving.



\needspace{15\baselineskip}
\subsection*{Clam Chowder}


\begin{minipage}{1.0\textwidth}
{\setlength{\multicolsep}{0pt}\setlength{\columnsep}{2em}\raggedcolumns%
\begin{multicols}{2}
\begin{itemize}
\setlength{\itemsep}{0pt}
\setlength{\parsep}{0pt}
\item 1 quart clams
\item 4 cups potatoes cut in 3/4-inch cubes
\item 1 1/2 inch cube fat salt pork
\item 1 sliced onion
\item 1 tablespoon salt
\item 1/8 teaspoon pepper
\item 4 tablespoons butter
\item 4 cups scalded milk
\item 8 common crackers
\end{itemize}
\end{multicols}}
\end{minipage}

\vspace{0.3em}
\noindent%
Clean and pick over clams, using one cup cold water, drain, reserve
liquor, heat to boiling-point, and strain. Chop finely hard part of
clams; cut pork in small pieces and try out; add onion, fry five
minutes, and strain into a stewpan. Parboil potatoes five minutes in
boiling water to cover; drain, and put a layer in bottom of stewpan, add
chopped clams, sprinkle with salt and pepper, and dredge generously with
flour; add remaining potatoes, again sprinkle with salt and pepper,
dredge with flour, and add two and one-half cups boiling water. Cook ten
minutes, add milk, soft part of clams, and butter; boil three minutes,
and add crackers split and soaked in enough cold milk to moisten. Reheat
clam water to boiling-point, and thicken with one tablespoon butter and
flour cooked together. Add to chowder just before serving.

The clam water has a tendency to cause the milk to separate, hence is
added at the last.



\needspace{15\baselineskip}
\subsection*{Rhode Island Chowder}


\begin{minipage}{1.0\textwidth}
{\setlength{\multicolsep}{0pt}\setlength{\columnsep}{2em}\raggedcolumns%
\begin{multicols}{2}
\begin{itemize}
\setlength{\itemsep}{0pt}
\setlength{\parsep}{0pt}
\item 1 quart clams
\item 3 inch cube fat salt pork
\item 1 sliced onion
\item 1/2 cup cold water
\item 4 cups potatoes cut in 3/4 inch cubes
\item 2 cups boiling water
\item 1 cup stewed and strained tomatoes
\item 1/4 teaspoon soda
\item 1 cup scalded milk
\item 1 cup scalded cream
\item 2 tablespoons butter
\item 8 common crackers
\item Salt and pepper
\end{itemize}
\end{multicols}}
\end{minipage}

\vspace{0.3em}
\noindent%
Cook pork with onion and cold water ten minutes; drain, and reserve
liquor. Wash clams and reserve liquor. Parboil potatoes five minutes,
and drain. To potatoes add reserved liquors, hard part of clams finely
chopped, and boiling water. When potatoes are nearly done, add tomatoes,
soda, soft part of clams, milk, cream, and butter. Season with salt and
pepper. Split crackers, soak in cold milk to moisten, and reheat in
chowder.



\needspace{15\baselineskip}
\subsection*{Lobster Chowder}


\begin{minipage}{1.0\textwidth}
{\setlength{\multicolsep}{0pt}\setlength{\columnsep}{2em}\raggedcolumns%
\begin{multicols}{2}
\begin{itemize}
\setlength{\itemsep}{0pt}
\setlength{\parsep}{0pt}
\item 2 lb. lobster
\item 3 tablespoons butter
\item 2 common crackers, finely pounded
\item 4 cups milk
\item 1 slice onion
\item 1 cup cold water
\item Salt
\item Paprika or cayenne
\end{itemize}
\end{multicols}}
\end{minipage}

\vspace{0.3em}
\noindent%
Remove meat from lobster shell and cut in small dice. Cream two
tablespoons butter, add liver of lobster (green part) and crackers;
scald milk with onion, remove onion, and add milk to mixture. Cook body
bones ten minutes in cold water to cover, strain, and add to mixture
with lobster dice. Season with salt and paprika.



\needspace{15\baselineskip}
\subsection*{German Chowder}


\begin{minipage}{1.0\textwidth}
{\setlength{\multicolsep}{0pt}\setlength{\columnsep}{2em}\raggedcolumns%
\begin{multicols}{2}
\begin{itemize}
\setlength{\itemsep}{0pt}
\setlength{\parsep}{0pt}
\item 3 lb. haddock
\item 1 quart cold water
\item 2 slices carrot
\item Bit of bay leaf
\item Sprig of parsley
\item 1 cracker, pounded
\item Salt, pepper, cayenne
\item 2 tablespoons melted butter
\item Few drops onion juice
\item 1 beaten egg
\item 1 quart potatoes cut in 3/4-inch cubes
\item 2--inch cube fat salt pork
\item 1 sliced onion
\item 5 tablespoons flour
\item 1 quart scalded milk
\item 1/4 cup butter
\item 8 common crackers
\end{itemize}
\end{multicols}}
\end{minipage}

\vspace{0.3em}
\noindent%
Clean, skin, and bone fish. Add to bones cold water and vegetables, and
let simmer twenty minutes. Strain stock from bones. Chop fish meat;
there should be one and one-half cups. Add cracker, seasonings, melted
butter and egg, then shape in small balls. Try out pork, add onion, and
cook five minutes. Strain, and add to fat, potatoes, balls, and fish
stock, and cook until potatoes are soft. Thicken milk with butter and
flour cooked together. Combine mixtures, and season highly with salt,
pepper, and cayenne. Add crackers, split and soaked in cold milk.





\chapter{Soup Garnishings And Force-Meats}




\needspace{15\baselineskip}
\section*{Crisp Crackers}

Split common crackers and spread thinly with butter, allowing one-fourth
teaspoon butter to each half cracker; put in pan and bake until
delicately browned.



\needspace{15\baselineskip}
\section*{Souffléd Crackers}

Split common crackers, and soak in ice water, to cover, eight minutes.
Dot over with butter, and bake in a hot oven until puffed and browned.



\needspace{15\baselineskip}
\section*{Crackers With Cheese}

Arrange zephyrettes or saltines in pan. Sprinkle with grated cheese and
bake until cheese is melted.



\needspace{15\baselineskip}
\section*{Croûtons (Duchess Crusts)}

Cut stale bread in one-third inch slices and remove crusts. Spread
thinly with butter. Cut slices in one-third inch cubes, put in pan and
bake until delicately brown, or fry in deep fat.



\needspace{15\baselineskip}
\section*{Cheese Sticks}

Cut bread sticks in halves lengthwise, spread thinly with butter,
sprinkle with grated cheese seasoned with salt and cayenne, and bake
until delicately browned.



\needspace{15\baselineskip}
\section*{Imperial Sticks In Rings}

Cut stale bread in one-third inch slices, remove crusts, spread thinly
with butter, and cut slices in one-third inch strips and rings; put in
pan and bake until delicately browned. Arrange three sticks in each
ring.



\needspace{15\baselineskip}
\section*{Mock Almonds}

Cut stale bread in one-eighth inch slices, shape with a round cutter one
and one-half inches in diameter, then shape in almond-shaped pieces.
Brush over with melted butter, put in a pan, and bake until delicately
browned.



\needspace{15\baselineskip}
\section*{Pulled Bread}

Remove crusts from a long loaf of freshly baked water bread. Pull the
bread apart until the pieces are the desired size and length, which is
best accomplished by using two three-tined forks. Cook in a slow oven
until delicately browned and thoroughly dried. A baker's French loaf may
be used for pulled bread if home-made is not at hand.



\needspace{15\baselineskip}
\section*{Egg Balls I}


\begin{itemize}
\setlength{\itemsep}{0pt}
\setlength{\parsep}{0pt}
\item 2 “hard-boiled” egg yolks 
\item 1/8 teaspoon salt
\item Few grains cayenne
\item 1/2 teaspoon melted butter
\end{itemize}

\vspace{-0.5em}
\noindent%
Rub yolks through sieve, add seasonings, and moisten with raw egg yolk
to make of consistency to handle. Shape in small balls, roll in flour,
and sauté in butter. Serve in Brown Soup Stock, Consommé, or Mock Turtle
Soup.



\needspace{15\baselineskip}
\section*{Egg Balls II}


\begin{itemize}
\setlength{\itemsep}{0pt}
\setlength{\parsep}{0pt}
\item 1 “hard-boiled” egg
\item 1/8 teaspoon salt
\item Few grains cayenne
\item 1 teaspoon heavy cream
\item 1/4 teaspoon finely chopped parsley
\end{itemize}

\vspace{-0.5em}
\noindent%
Rub yolk through a sieve, add white finely chopped, and remaining
ingredients. Add raw egg yolk to make mixture of right consistency to
handle. Shape in small balls, and poach in boiling water or stock.



\needspace{15\baselineskip}
\section*{Egg Custard}


\begin{itemize}
\setlength{\itemsep}{0pt}
\setlength{\parsep}{0pt}
\item 2 egg yolks
\item Few grains salt
\item 2 tablespoons milk
\end{itemize}

\vspace{-0.5em}
\noindent%
Beat eggs slightly, add milk and salt. Pour into small buttered cup,
place in pan of hot water, and bake until firm; cool, remove from cup,
and cut in fancy shapes with French vegetable cutters.



\needspace{15\baselineskip}
\section*{Harlequin Slices}


\begin{minipage}{1.0\textwidth}
{\setlength{\multicolsep}{0pt}\setlength{\columnsep}{2em}\raggedcolumns%
\begin{multicols}{2}
\begin{itemize}
\setlength{\itemsep}{0pt}
\setlength{\parsep}{0pt}
\item 3 egg yolks
\item 2 tablespoons milk
\item Few grains salt
\item 3 egg whites
\item Few grains salt
\item Chopped truffles
\end{itemize}
\end{multicols}}
\end{minipage}

\vspace{0.3em}
\noindent%
Beat yolks of eggs slightly, add milk and salt. Pour into small buttered
cup, place in pan of hot water and bake until firm. Beat whites of eggs
slightly, add salt, and cook same as yolks. Cool, remove from cups, cut
in slices, pack in a mould in alternate layers, and press with a weight.
A few truffles may be sprinkled between slices if desired. Remove from
mould and cut in slices. Serve in Consommé.



\needspace{15\baselineskip}
\section*{Royal Custard}


\begin{minipage}{1.0\textwidth}
{\setlength{\multicolsep}{0pt}\setlength{\columnsep}{2em}\raggedcolumns%
\begin{multicols}{2}
\begin{itemize}
\setlength{\itemsep}{0pt}
\setlength{\parsep}{0pt}
\item 3 egg yolks
\item 1 egg
\item 1/2 cup Consommé
\item 1/8 teaspoon salt
\item Slight grating nutmeg
\item Few grains cayenne
\end{itemize}
\end{multicols}}
\end{minipage}

\vspace{0.3em}
\noindent%
Beat eggs slightly, add Consommé and seasonings. Pour into a small
buttered tin mould, place in pan of hot water, and bake until firm;
cool, remove from mould, and cut in fancy shapes.



\needspace{15\baselineskip}
\section*{Chicken Custard}

Chop cooked breast meat of fowl and rub through sieve; there should be
one-fourth cup. Add one-fourth cup White Stock and one egg slightly
beaten. Season with salt, pepper, celery salt, paprika, slight grating
nutmeg, and few drops essence anchovy. Turn mixture into buttered mould,
bake in a pan of hot water until firm; cool, remove from mould, and cut
in small cubes.



\needspace{15\baselineskip}
\section*{Noodles}


\begin{itemize}
\setlength{\itemsep}{0pt}
\setlength{\parsep}{0pt}
\item 1 egg
\item 1/2 teaspoon salt
\item Flour
\end{itemize}

\vspace{-0.5em}
\noindent%
Beat egg slightly, add salt, and flour enough to make very stiff dough;
knead, toss on slightly floured board, and roll thinly as possible,
which may be as thin as paper. Cover with towel, and set aside for
twenty minutes; then cut in fancy shapes, using sharp knife or French
vegetable cutter; or the thin sheet may be rolled like jelly roll, cut
in slices as thinly as possible, and pieces unrolled. Dry, and when
needed cook twenty minutes in boiling salted water; drain, and add to
soup.

Noodles may be served as a vegetable.



\needspace{15\baselineskip}
\section*{Fritter Beans}


\begin{itemize}
\setlength{\itemsep}{0pt}
\setlength{\parsep}{0pt}
\item 1 egg
\item 2 tablespoons milk
\item 3/4 teaspoon salt
\item 1/2 cup flour
\end{itemize}

\vspace{-0.5em}
\noindent%
Beat egg until light, add milk, salt, and flour. Put through colander or
pastry tube into deep fat, and fry until brown; drain on brown paper.



\needspace{15\baselineskip}
\section*{Pâte À Choux}


\begin{minipage}{1.0\textwidth}
{\setlength{\multicolsep}{0pt}\setlength{\columnsep}{2em}\raggedcolumns%
\begin{multicols}{2}
\begin{itemize}
\setlength{\itemsep}{0pt}
\setlength{\parsep}{0pt}
\item 2 1/2 tablespoons milk
\item 1/2 teaspoon lard
\item 1/2 teaspoon butter
\item 1/8 teaspoon salt
\item 1/4 cup flour
\item 1 egg
\end{itemize}
\end{multicols}}
\end{minipage}

\vspace{0.3em}
\noindent%
Heat butter, lard, and milk to boiling-point, add flour and salt, and
stir vigorously. Remove from fire, add egg unbeaten, and stir until well
mixed. Cool, and drop small pieces from tip of teaspoon into deep fat.
Fry until brown and crisp, and drain on brown paper.



\needspace{15\baselineskip}
\section*{Parmesan Pâte À Choux}

To Pâte à Choux mixture add two tablespoons grated Parmesan cheese.



\needspace{15\baselineskip}
\section*{White Bait Garnish}

Roll trimmings of puff paste, and cut in pieces three-fourths inch long
and one-eighth inch wide; fry in deep fat until well browned, and drain
on brown paper. Serve on folded napkin, and pass with soup.



\needspace{15\baselineskip}
\section*{Fish Force-Meat I}


\begin{itemize}
\setlength{\itemsep}{0pt}
\setlength{\parsep}{0pt}
\item 1/4 cups fine stale bread crumbs
\item 1/4 cup milk
\item 1 egg
\item 2/3 cup raw fish
\item Salt
\end{itemize}

\vspace{-0.5em}
\noindent%
Cook bread and milk to a paste, add egg well beaten, and fish pounded
and forced through a purée strainer. Season with salt. A meat chopper is
of great assistance in making force-meats, as raw fish or meat may be
easily forced through it. Bass, halibut, or pickerel are the best fish
to use for force-meat. Force-meat is often shaped into small balls.



\needspace{15\baselineskip}
\section*{Fish Force-Meat II}


\begin{minipage}{1.0\textwidth}
{\setlength{\multicolsep}{0pt}\setlength{\columnsep}{2em}\raggedcolumns%
\begin{multicols}{2}
\begin{itemize}
\setlength{\itemsep}{0pt}
\setlength{\parsep}{0pt}
\item 2/3 cup raw halibut
\item 1 egg white
\item Salt
\item Pepper
\item Cayenne
\item 1/2 cup heavy cream
\end{itemize}
\end{multicols}}
\end{minipage}

\vspace{0.3em}
\noindent%
Chop fish finely, or force through a meat chopper. Pound in mortar,
adding gradually white of egg, and working until smooth. Add seasonings,
rub through a sieve, and then add cream.



\needspace{15\baselineskip}
\section*{Salmon Force-Meat}


\begin{minipage}{1.0\textwidth}
{\setlength{\multicolsep}{0pt}\setlength{\columnsep}{2em}\raggedcolumns%
\begin{multicols}{2}
\begin{itemize}
\setlength{\itemsep}{0pt}
\setlength{\parsep}{0pt}
\item 1/2 cup milk
\item 1/2 cup soft stale bread crumbs
\item 1/2 cup cold flaked salmon
\item 2 tablespoons cream
\item 1 egg
\item 2 tablespoons melted butter
\item 1/2 teaspoon salt
\item Few grains pepper
\end{itemize}
\end{multicols}}
\end{minipage}

\vspace{0.3em}
\noindent%
Cook milk and bread crumbs ten minutes, add salmon chopped and rubbed
through a sieve; then add cream, egg slightly beaten, melted butter,
salt, and pepper.



\needspace{15\baselineskip}
\section*{Oyster Force-Meat}

To Fish Force-meat add one-fourth small onion, finely chopped, and fried
five minutes in one-half tablespoon butter; then add one-third cup soft
part of oysters, parboiled and finely chopped, one-third cup mushrooms
finely chopped, and one-third cup Thick White Sauce. Season with salt,
cayenne, and one teaspoon finely chopped parsley.



\needspace{15\baselineskip}
\section*{Clam Force-Meat}

Follow recipe for Oyster Force-meat, using soft part of clams in place
of oysters.



\needspace{15\baselineskip}
\section*{Chicken Force-Meat I}


\begin{minipage}{1.0\textwidth}
{\setlength{\multicolsep}{0pt}\setlength{\columnsep}{2em}\raggedcolumns%
\begin{multicols}{2}
\begin{itemize}
\setlength{\itemsep}{0pt}
\setlength{\parsep}{0pt}
\item 1/2 cup fine stale bread crumbs
\item 1/2 cup milk
\item 2 tablespoons butter
\item 1 egg white
\item 2/3 cup breast raw chicken
\item Salt
\item Few grains cayenne
\item Slight grating nutmeg
\end{itemize}
\end{multicols}}
\end{minipage}

\vspace{0.3em}
\noindent%
Cook bread and milk to a paste, add butter, white of egg beaten stiff,
and seasonings; then add chicken pounded and forced through purée
strainer.



\needspace{15\baselineskip}
\section*{Chicken Force-Meat II}


\begin{minipage}{1.0\textwidth}
{\setlength{\multicolsep}{0pt}\setlength{\columnsep}{2em}\raggedcolumns%
\begin{multicols}{2}
\begin{itemize}
\setlength{\itemsep}{0pt}
\setlength{\parsep}{0pt}
\item 1/2 breast raw chicken
\item 1 egg white
\item Salt
\item Pepper
\item Slight grating nutmeg
\item Heavy cream
\end{itemize}
\end{multicols}}
\end{minipage}

\vspace{0.3em}
\noindent%
Chop chicken finely, or force through a meat chopper. Pound in mortar,
add gradually white of egg, and work until smooth; then add heavy cream
slowly until of right consistency, which can only be determined by
cooking a small ball in boiling salted water. Add seasonings, and rub
through sieve.



\needspace{15\baselineskip}
\section*{Quenelles}

Quenelles are made from any kind of force-meat, shaped in small balls or
between tablespoons, making an oval, or by forcing mixture through
pastry bag on buttered paper. They are cooked in boiling salted water or
stock, and are served as garnish to soups or other dishes; when served
with sauce, they are an entrée.





\chapter{Fish}



The meat of fish is the animal food next in importance to that of birds
and mammals. Fish meat, with but few exceptions, is less stimulating and
nourishing than meat of other animals, but is usually easier of
digestion. Salmon, mackerel, and eels are exceptions to these rules, and
should not be eaten by those of weak digestion. White fish, on account
of their easy digestibility, are especially desirable for those of
sedentary habits. Fish is not recommended for brain-workers on account
of the large amount of phosphorus (an element abounding largely in nerve
tissue) which it contains, but because of its easy digestibility. It is
a conceded fact that many fish contain less of this element than meat.

Fish meat is generally considered cheaper than meat of other animals.
This is true when compared with the better cuts of meat, but not so when
compared with cheaper cuts.

To obtain from fish its greatest value and flavor, it should be eaten
fresh, and in season. Turbot, which is improved by keeping, is the only
exception to this rule.

\textit{To Determine Freshness of Fish.} Examine the flesh, and it should be
firm; the eyes and gills, and they should be bright.

Broiling and baking are best methods for cooking fish. White fish may
often be fried, but oily rarely. Frozen fish are undesirable, but if
used, should be thawed in cold water just before cooking.

On account of its strong odor, fish should never be put in an ice-box
with other food, unless closely covered. A tin lard pail will be found
useful for this purpose.



\needspace{15\baselineskip}
\section*{White And Oily Fish}

White fish have fat secreted in the liver. Examples: cod, haddock,
trout, flounder, smelt, perch, etc.

Oily fish have fat distributed throughout the flesh. Examples: salmon,
eels, mackerel, bluefish, swordfish, shad, herring, etc.

\textbf{Cod} belongs to one of the most prolific fish families (Gadidoe), and
is widely distributed throughout the northern and temperate seas of both
hemispheres. On account of its abundance, cheapness, and easy
procurability, it forms, from an economical standpoint, one of the most
important fish foods. Cod have been caught weighing over a hundred
pounds, but average market cod weigh from six to ten pounds; a six-pound
cod measures about twenty-three inches in length. Large cod are cut into
steaks. The skin of cod is white, heavily mottled with gray, with a
white line running the entire length of fish on either side. Cod is
caught in shallow or deep waters. Shallow-water cod (caught off rocks)
is called rock cod; deep-water cod is called off-shore cod. Rock cod are
apt to be wormy. Cod obtained off George's Banks, Newfoundland, are
called George's cod, and are commercially known as the best fish.
Quantities of cod are preserved by drying and salting. Salted George's
cod is the best brand on the market. Cod is in season throughout the
year.

\textit{Cod Liver Oil} is obtained from cods' livers, and has great therapeutic
value. Isinglass, made from swimming bladder of cod, nearly equals in
quality that made from bladder of sturgeon.


      \textbf{Haddock} is more closely allied to cod than any other fish.

It is smaller (its average weight being about four pounds), and
differently mottled. The distinguishing mark of the haddock is a black
line running the entire length of fish on either side. Haddock is found
in the same water and in company with cod, but not so abundantly. Like
cod, haddock is cheap, and in season throughout the year. Haddock, when
dried, smoked, and salted, is known as \textit{Finnan Haddie}.

\textbf{Halibut} is the largest of the flatfish family (Pleuronectidæ),
specimens having been caught weighing from three to four hundred pounds.
Small, or chicken, halibut is the kind usually found in market, and
weighs from fifteen to twenty-five pounds. Halibut are distinctively
cold-water fish, being caught in water at from 32deg to 45deg F. They are
found in the North Atlantic and North Pacific oceans, where they are
nearly identical. The halibut has a compressed body, the skin on one
side being white, on the other light, or dark gray, and both eyes are
found on the dark side of head. Halibut is in season throughout the
year.

\textbf{Turbot} (called little halibut) is a species of the flatfish family,
being smaller than halibut, and of more delicate flavor. Turbot are in
season from January to March.

\textbf{Flounder} is a small flatfish, which closely resembles the sole which
is caught in English waters, and is often served under that name.

\textbf{Trout} are generally fresh water fish, varying much in size and
skin-coloring. Lake trout, which are the largest, reach their greatest
perfection in Lakes Huron, Michigan, and Superior, but are found in many
lakes. Salmon trout is the name applied to trout caught in New York
lakes. Brook trout, caught in brooks and small lakes, are superior
eating. Trout are in season from April to August, but a few are found
later.

\textbf{Whitefish} is the finest fish found in the Great Lakes.

\textbf{Smelts} are small salt-water fish, and are usually caught in temperate
waters at the mouths of rivers. New Brunswick and Maine send large
quantities of smelts to market. Selected smelts are the largest in size,
and command higher price. The Massachusetts Fish and Game Protective Law
forbids their sale from March 15th to June 1st. Smelts are always sold
by the pound.

\textbf{Bluefish} belongs to the Pomatomidæ family. It is widely distributed in
temperate waters, taking different names in different localities. In New
England and the Middle States it is generally called Bluefish, although
in some parts called Snappers, or Snapping Mackerel. In the Southern
States it is called Greenfish. It is in season in our markets from May
to October; as it is frozen and kept in cold storage from six to nine
months, it may be obtained throughout the year. The heavier the fish,
the better its quality. Bluefish weigh from one to eight pounds, and are
from fourteen to twenty-nine inches in length.

\textbf{Mackerel} is one of the best-known food fishes, and is caught in North
Atlantic waters. Its skin is lustrous dark blue above, with wavy
blackish lines, and silvery below. It sometimes attains a length of
eighteen inches, but is usually less. Mackerel weigh from three-fourths
of a pound to two pounds, and are sold by the piece. They are in season
from May 1st to September 1st. Mackerel, when first in market, contain
less fat than later in the season, therefore are easier of digestion.
The supply of mackerel varies greatly from year to year, and some years
is very small. \textit{Spanish mackerel} are found in waters farther south than
common mackerel, and in our markets command higher price.

\textbf{Salmon} live in both fresh and salt waters, always going, inland,
usually to the head of rivers, during the spawning season. The young
after a time seek salt water, but generally return to fresh water.
Penobscot River Salmon are the best, and come from Maine and St. John,
New Brunswick. The average weight of salmon is from fifteen to
twenty-five pounds, and the flesh is of pinkish orange color. Salmon are
in season from May to September, but frozen salmon may be obtained the
greater part of the year. In the Columbia River and its tributaries
salmon are so abundant that extensive canneries are built along the
banks.

\textbf{Shad}, like salmon, are found in both salt and fresh water, always
ascending rivers for spawning. Shad is caught on the Atlantic Coast of
the United States, and its capture constitutes one of the most important
fisheries. Shad have a silvery hue, which becomes bluish on the back;
they vary in length from eighteen to twenty-eight inches, and are always
sold by the piece, price being irrespective of size. \textit{Jack shad} are
usually cheaper than \textit{roe shad}. The roe of shad is highly esteemed.
Shad are in season from January to June. First shad in market come from
Florida, and retail from one and one-half to two dollars each. The
finest come from New Brunswick, and appear in market about the first of
May.

\textit{Caviare} is the salted roe of the sturgeon.

\textbf{Herring} are usually smoked, or smoked and salted, and, being very
cheap, are a most economical food.



\needspace{15\baselineskip}
\section*{Shellfish}


\needspace{15\baselineskip}
\subsection*{I. Bivalve Mollusks}

\textbf{Oysters} are mollusks, having two shells. The shells are on the right
and left side of the oyster, and are called right and left valves. The
one upon which the oyster rests grows faster, becomes deeper, and is
known as the left valve. The valves are fastened by a ligament, which,
on account of its elasticity, admits of opening and closing of the
shells. The oyster contains a tough muscle, by which it is attached to
the shell; the body is made up largely of the liver (which contains
\textit{glycogen}, animal starch), and is partially surrounded by fluted
layers, which are the gills. Natural oyster beds (or banks) are found in
shallow salt water having stony bottom, along the entire Atlantic Coast.
The oyster industry of the world is chiefly in the United States and
France, and on account of its increase many artificial beds have been
prepared for oyster culture. Oysters are five years old before suitable
for eating. Blue Points, which are small, plump oysters, take their name
from Blue Point, Long Island, from which place they originally came.
Their popularity grew so rapidly that the supply became inadequate for
the demand, and any small, plump oysters were soon sold for Blue Points.
During the oyster season they form the first course of a dinner, served
raw on the half shell. In our markets, selected oysters (which are
extremely large and used for broiling) Providence River, and Norfolk
oysters are familiarly known, and, taken out of the shells, are sold by
the quart. Farther south, they are sold by count.

Oysters are obtainable all the year, but are in season from September to
May. During the summer mouths they are flabby and of poor flavor,
although when fresh they are perfectly wholesome. \textit{Mussels}, eaten in
England and other parts of Europe, are similar to oysters, though of
inferior quality. Oysters are nutritious and of easy digestibility,
especially when eaten raw.

\textit{To Open Oysters.} Put a thin flat knife under the back end of the right
valve, and push forward until it cuts the strong muscle which holds the
shells together. As soon as this is done, the right valve may be raised
and separated from the left.

\textit{To Clean Oysters.} Put oysters in a strainer placed over a bowl. Pour
over oysters cold water, allowing one-half cup water to each quart
oysters. Carefully pick over oysters, taking each one separately in the
fingers, to remove any particles of shell which adhere to tough muscle.

\textbf{Clams}, among bivalve mollusks, rank in value next to oysters. They are
found just below the surface of sand and mud, above low-water mark, and
are easily dug with shovel or rake. Clams have hard or soft shells.
Soft-shell clams are dear to the New Englander. From New York to Florida
are found hard-shelled clams (quahaugs). \textit{Small quahaugs} are called
\textit{Little Neck Clams} and take the place of Blue Points at dinner, when
Blue Points are out of season.

\textbf{Scallops} are bivalve mollusks, the best being found in Long Island
Sound and Narragansett Bay. The central muscle forms the edible portion,
and is the only part sent to market. Scallops are in season from October
first to April first.



\needspace{15\baselineskip}
\subsection*{II. Crustaceans}

\textbf{Lobsters} belong to the highest order of Crustaceans, live exclusively
in sea-water, generally near rocky coasts, and are caught in pots set on
gravelly bottoms. The largest and best species are found in Atlantic
waters from Maine to New Jersey, being most abundant on Maine and
Massachusetts coasts. Lobsters have been found weighing from sixteen to
twenty-five pounds, but such have been exterminated from our coast. The
average weight is two pounds, and the length from ten to fifteen inches.
Lobsters are largest and most abundant from June to September, but are
obtainable all the year. When taken from the water, shells are of
mottled dark green color, except when found on sandy bottoms, when they
are quite red. Lobsters are generally boiled, causing the shell to turn
red.

A lobster consists of body, tail, two large claws, and four pairs of
small claws. On lower side of body, in front of large claws, are various
small organs which surround the mouth, and a long and short pair of
feelers. Under the tail are found several pairs of appendages. In the
female lobster, also called hen lobster, is found, during the breeding
season, the spawn, known as \textit{coral}. Sex is determined by the pair of
appendages in the tail which lie nearest the body; in the female they
are soft and pliable, in the male hard and stiff. At one time small
lobsters were taken in such quantities that it was feared, if the
practice was long continued, they would be exterminated. To protect the
continuance of lobster fisheries, a law has been passed in many States
prohibiting their sale unless at least ten inches long.

Lobsters shed their shells at irregular intervals, when old ones are
outgrown. The new ones begin to form and take on distinctive
characteristics before the old ones are discarded. New shells after
twenty-four hours' exposure to the water are quite hard.

Lobsters, being coarse feeders (taking almost any animal substance
attainable), are difficult of digestion, and with some create great
gastric disturbance; notwithstanding, they are seldom found diseased.

\textit{To Select a Lobster.} Take in the hand, and if heavy in proportion to
its size, the lobster is fresh. Straighten the tail, and if it springs
into place the lobster was alive (as it should have been) when put into
the pot for boiling. There is greater shrinkage in lobsters than in any
other fish.

\textit{To Open Lobsters.} Take off large claws, small claws, and separate tail
from body. Tail meat may sometimes be drawn out whole with a fork; more
often it is necessary to cut the thin shell portion (using scissors or a
can opener) in under part of the tail, then the tail meat may always be
removed whole. Separate tail meat through centre, and remove the small
intestinal vein which runs its entire length; although generally darker
than the meat, it is sometimes found of the same color. Hold body shell
firmly in left hand, and with first two fingers and thumb of right hand
draw out the body, leaving in shell the stomach (known as the \textit{lady}),
which is not edible, and also some of the green part, the \textit{liver}. The
liver may be removed by shaking the shell. The sides of the body are
covered with the \textit{lungs}; these are always discarded. Break body through
the middle and separate body bones, picking out meat that lies between
them, which is some of the sweetest and tenderest to be found. Separate
large claws at joints. If shells are thin, with a knife cut off a strip
down the sharp edge, so that shell may be broken apart and meat removed
whole. Where shell is thick, it must be broken with a mallet or hammer.
Small claws are used for garnishing. The shell of body, tail, and lower
part of large claws, if not broken, may be washed, dried, and used for
serving of lobster meat after it has been prepared. The portions of
lobsters which are not edible are \textit{lungs}, \textit{stomach} (lady), and
\textit{intestinal vein}.

\textbf{Crabs} among Crustaceans are next in importance to lobsters,
commercially speaking. They are about two and one-half inches long by
five inches wide, and are found along the Atlantic Coast from
Massachusetts to Florida, and in the Gulf of Mexico. Crabs, like
lobsters, change their shells. \textit{Soft-shell} crabs are those which have
recently shed their old shells, and the new shells have not had time to
harden; these are considered by many a great luxury. \textit{Oyster crabs}
(very small crabs found in shells with oysters) are a delicacy not often
indulged in. Crabs are in season during the spring and summer.

\textbf{Shrimps} are found largely in our Southern waters, the largest and best
coming from Lake Pontchartrain. They are about two inches long, covered
with a thin shell, and are boiled and sent to market with heads removed.
Their grayish color is changed to pink by boiling. Shrimps are in season
from May first to October first, and are generally used for salads.
Canned shrimps are much used and favorably known.

\textit{Reptiles.} Frogs and terrapin belong to a lower order of animals than
fish,--reptiles. They are both table delicacies, and are eaten by the
few.

Only the hind legs of frogs are eaten, and have much the same flavor as
chicken.

Terrapin, although sold in our large cities, specially belong to
Philadelphia, Baltimore, and Washington, where they are cooked and
served at their best. They are shipped from the South, packed in
seaweed, and may be kept for some time in a dark place. Terrapin are
found in both fresh and salt-water. The Diamond Back, salt-water
terrapin, coming from Chesapeake Bay, are considered the best, and
command a very high price. Terrapin closely resembling Diamond Back,
coming from Texas and Florida, are principally sold in our markets.
Terrapin are in season from November to April, but are best in January,
February, and March. They should always be cooked alive.



\needspace{15\baselineskip}
\section*{To Prepare Fish For Cooking}

\textbf{To Clean a Fish.} Fish are cleaned and dressed at market as ordered,
but need additional cleaning before cooking. Remove scales which have
not been taken off. This is done by drawing a knife over fish, beginning
at tail and working towards head, occasionally wiping knife and scales
from fish. Incline knife slightly towards you to prevent scales from
flying. The largest number of scales will be found on the flank. Wipe
thoroughly inside and out with cloth wrung out of cold water, removing
any clotted blood which may be found adhering to backbone.

Head and tail may or may not be removed, according to size of fish and
manner of cooking. Small fish are generally served with head and tail
left on.

\textbf{To Skin a Fish.} With sharp knife remove fins along the back and cut
off a narrow strip of skin the entire length of back. Loosen skin on one
side from bony part of gills, and being once started, if fish is fresh,
it may be readily drawn off; if flesh is soft do not work too quickly,
as it will be badly torn. By allowing knife to closely follow skin this
may be avoided. After removing skin from one side, turn fish and skin
the other side.

\textbf{To Bone a Fish.} Clean and skin before boning. Beginning at the tail,
run a sharp knife under flesh close to backbone, and with knife follow
bone (making as clean a cut as possible) its entire length, thus
accomplishing the removal of one-half the flesh; turn, and remove flesh
from other side. Pick out with fingers any small bones that may remain.
Cod, haddock, halibut, and whitefish are easily and frequently boned;
flounders and smelts occasionally.

\textbf{To Fillet Fish.} Clean, skin, and bone. A piece of fish, large or
small, freed from skin and bones, is known as a fillet. Halibut, cut in
three-fourths inch slices, is more often cut in fillets than any kind of
fish, and fillets are frequently rolled. When flounder is cut in fillets
it is served under the name of \textit{fillet of sole}. Sole found in English
waters is much esteemed, and flounder is our nearest approach to it.



\needspace{15\baselineskip}
\section*{Ways Of Cooking Fish}

\textit{To Cook Fish in Boiling Water.} Small cod, haddock, or cusk are cooked
whole in enough boiling water to cover, to which is added salt and lemon
juice or vinegar. Salt gives flavor; lemon juice or vinegar keeps the
flesh white. A long fish-kettle containing a rack on which to place fish
is useful but rather expensive. In place of fish-kettle, if the fish is
not too large to be coiled in it, a frying-basket may be used placed in
any kettle.

Large fish are cut in thick pieces for boiling, containing the number of
pounds required. Examples: salmon and halibut.

Pieces cut from large fish for boiling should be cleaned and tied in a
piece of cheese-cloth to prevent scum being deposited on the fish. If
skin is not removed before serving, scald the dark skin and scrape to
remove coloring; this may be easily accomplished by holding fish on two
forks, and lowering into boiling water the part covered with black skin;
then remove and scrape. Time required for boiling fish depends on extent
of surface exposed to water. Consult Time Table for Boiling, which will
serve as a guide. The fish is cooked when flesh leaves the bone, no
matter how long the time.





\textbf{To Broil Fish.} Cod, haddock, bluefish, and mackerel are split down the
back and broiled whole, removing head and tail or not, as desired.
Salmon, chicken halibut, and swordfish are cut in inch slices for
broiling. Smelts and other small fish are broiled whole, without
splitting. Clean and wipe fish as dry as possible, sprinkle with salt
and pepper, and place in well-greased wire broiler. Slices of fish
should be turned often while broiling; whole fish should be first
broiled on flesh side, then turned and broiled on skin side just long
enough to make skin brown and crisp.

To remove from broiler, loosen fish on one side, turn and loosen on
other side; otherwise flesh will cling to broiler. Slip from broiler to
hot platter, or place platter over fish and invert platter and broiler
together.

\textbf{To Bake Fish.} Clean, and bake on a greased fish-sheet placed in a
dripping-pan. If a fish-sheet is not at hand, place strips of cotton
cloth under fish, by which it may be lifted from pan.

\textbf{To Fry Fish.} Clean fish, and wipe as dry as possible. Sprinkle with
salt, dip in flour or crumbs, egg, and crumbs, and fry in deep fat.

\textbf{To Sauté Fish.} Prepare as for frying, and cook in frying-pan with
small amount of fat; or, if preferred, dip in granulated corn meal. Cod
steak and smelts are often cooked in this way.



\needspace{15\baselineskip}
\section*{Table Showing Composition Of The Various Fish Used For Food}


\begin{tabular}{p{1.8in}cccccc}
\hline
Item & Refuse & Proteid & Fat & Mineral matter & Carbohydrates & Water \\
\hline
Bass, black & 54.8 & 9.3 & .8 & .5 &  & 34.6 \\
\arrayrulecolor{tablerowgray}\hline
Bluefish & 55.7 & 8.3 & .5 & .5 &  & 35. \\
\arrayrulecolor{tablerowgray}\hline
Butterfish & 42.8 & 10.2 & 6.3 & .6 &  & 40.1 \\
\arrayrulecolor{tablerowgray}\hline
Cod, fresh & 52.5 & 8. & .2 & .6 &  & 38.7 \\
\arrayrulecolor{tablerowgray}\hline
Cod, salt, boneless & 22.2 & .3 & 23.1 & 54.4 &  &  \\
\arrayrulecolor{tablerowgray}\hline
Cusk & 40.3 & 10.1 & .1 & .5 &  & 49. \\
\arrayrulecolor{tablerowgray}\hline
Eels & 20.2 & 14.6 & 7.2 & .8 &  & 57.2 \\
\arrayrulecolor{tablerowgray}\hline
Flounder & 61.5 & 5.6 & .3 & .5 &  & 32.1 \\
\arrayrulecolor{tablerowgray}\hline
Haddock & 51. & 8.2 & .2 & .6 &  & 40. \\
\arrayrulecolor{tablerowgray}\hline
Halibut, sections & 17.7 & 15.1 & 4.4 & .9 &  & 61.9 \\
\arrayrulecolor{tablerowgray}\hline
Herring & 42.6 & 10.9 & 3.9 & .9 &  & 41.7 \\
\arrayrulecolor{tablerowgray}\hline
Mackerel & 44.6 & 10. & 4.3 & .7 &  & 40.4 \\
\arrayrulecolor{tablerowgray}\hline
Mackerel, Spanish & 34.6 & 13.7 & 6.2 & 1. &  & 44.5 \\
\arrayrulecolor{tablerowgray}\hline
Perch, white & 62.5 & 7.2 & 1.5 & .4 &  & 28.4 \\
\arrayrulecolor{tablerowgray}\hline
Pickerel & 47.1 & 9.8 & .2 & .7 &  & 42.2 \\
\arrayrulecolor{tablerowgray}\hline
Pompano & 45.5 & 10.2 & 4.3 & .5 &  & 39.5 \\
\arrayrulecolor{tablerowgray}\hline
Red Snapper & 46.1 & 10.6 & .6 & .7 &  & 42. \\
\arrayrulecolor{tablerowgray}\hline
Salmon & 39.2 & 12.4 & 8.1 & .9 &  & 39.4 \\
\arrayrulecolor{tablerowgray}\hline
Shad & 50.1 & 9.2 & 4.8 & .7 &  & 35.2 \\
\arrayrulecolor{tablerowgray}\hline
Shad, roe & 2.6 & 20.9 & 3.8 & 1.5 &  & 71.2 \\
\arrayrulecolor{tablerowgray}\hline
Sheepshead & 66. & 6.4 & .2 & .5 &  & 26.9 \\
\arrayrulecolor{tablerowgray}\hline
Smelts & 41.9 & 10. & 1. & 1. &  & 46.1 \\
\arrayrulecolor{tablerowgray}\hline
Trout & 48.1 & 9.8 & 1.1 & .6 &  & 40.4 \\
\arrayrulecolor{tablerowgray}\hline
Turbot & 47.7 & 6.8 & 7.5 & .7 &  & 37.3 \\
\arrayrulecolor{tablerowgray}\hline
Whitefish & 53.5 & 10.3 & 3. & .7 &  & 32.5 \\
\arrayrulecolor{tablerowgray}\hline
Lobsters & 61.7 & 5.9 & .7 & .8 & .2 & 30.7 \\
\arrayrulecolor{tablerowgray}\hline
Clams, out of shell & 10.6 & 1.1 & 2.3 & 5.2 &  & 80.8 \\
\arrayrulecolor{tablerowgray}\hline
Oysters, solid & 6.1 & 1.4 & .9 & 3.3 &  & 88.3 \\
\arrayrulecolor{tablerowgray}\hline
Crabs, soft-shell & 15.8 & 1.5 & 2. & .7 &  & 80. \\
\arrayrulecolor{black}
\hline
\end{tabular}

                                                  \textit{W. O. Atwater, Ph.D.}



\needspace{15\baselineskip}
\section*{Boiled Haddock}

Clean and boil as directed in Ways of Cooking Fish. Remove to a hot
platter, garnish with slices of “hard-boiled” eggs and parsley, and
serve with Egg Sauce. A thick piece of halibut may be boiled and served
in the same way.



\needspace{15\baselineskip}
\section*{Boiled Salmon}

Clean and boil as directed in Ways of Cooking Fish. Place on a hot
platter, remove skin, and garnish with slices of lemon and parsley.
Serve with Egg Sauce I or II, or Hollandaise Sauce.



\needspace{15\baselineskip}
\section*{Steamed Halibut, Silesian Sauce}

Steam by cooking over boiling water a piece of halibut weighing two
pounds, and serve with Silesian Sauce.


\begin{minipage}{1.0\textwidth}
{\setlength{\multicolsep}{0pt}\setlength{\columnsep}{2em}\raggedcolumns%
\begin{multicols}{2}
\begin{itemize}
\setlength{\itemsep}{0pt}
\setlength{\parsep}{0pt}
\item 1 1/2 tablespoons vinegar
\item 1/8 teaspoon powdered tarragon
\item 3 peppercorns
\item Bit of bay leaf
\item Sprig of parsley
\item 1/2 teaspoon finely chopped shallot
\item Salt and cayenne
\item 3 egg yolks
\item 2/3 cup Brown Stock
\item 1/4 cup butter
\item 1 tablespoon flour
\item 1/2 tablespoon capers
\item 1/2 tablespoon parsley
\end{itemize}
\end{multicols}}
\end{minipage}

\vspace{0.3em}
\noindent%
Cook first six ingredients until reduced one-half; strain, add yolks of
eggs well beaten, one-half, each, brown stock and butter, and cook over
hot water, stirring constantly until thickened. Then add, gradually,
remaining butter mixed with flour and stock. As soon as mixture
thickens, add capers, parsley finely chopped, and salt and cayenne.



\needspace{15\baselineskip}
\section*{Broiled Scrod}

A young cod, split down the back, and backbone removed, except a small
portion near the tail, is called a scrod. Scrod are always broiled,
spread with butter, and sprinkled with salt and pepper. Haddock is also
so dressed.



\needspace{15\baselineskip}
\section*{Broiled Chicken Halibut}

Clean and broil as directed in Ways of Cooking Fish. Spread with butter,
sprinkle with salt and pepper, and garnish with slices of lemon cut in
fancy shapes and sprinkled with paprika and parsley.



\needspace{15\baselineskip}
\section*{Broiled Swordfish}

Clean and broil fish, spread with butter, sprinkle with salt and pepper,
and serve with Cucumber Sauce I, or Horseradish Sauce I.



\needspace{15\baselineskip}
\section*{Broiled Shad Roe}

Wipe, sprinkle with salt and pepper, put on greased wire broiler, and
broil five minutes on each side. Serve with Maître d'Hôtel Butter.
Mackerel roe are delicious cooked in this way.



\needspace{15\baselineskip}
\section*{Broiled Pompano With Fricassee Of Clams}

Clean and broil fish as directed in Ways of Cooking Fish (see p. 160).
When nearly cooked, slip from broiler onto a hot platter and brush over
with melted butter. Surround with two borders of mashed potatoes,
one-inch apart, forced through a pastry bag and tube. Arrange ten halves
of clam-shells between potato borders, at equal distances; fill spaces
between shells with potato roses. Place in oven to finish cooking fish
and to brown potatoes. Just before serving, fill clam-shells with

\textbf{Fricassee of Clams.} Clean one pint clams, finely chop hard portions
and reserve soft portions. Melt two tablespoons butter, add chopped
clams, two tablespoons flour, and pour on gradually one-third cup cream.
Strain sauce, add soft part of clams, cook one minute, season with salt
and cayenne, and add yolk of one egg slightly beaten.



\needspace{15\baselineskip}
\section*{Baked Haddock With Stuffing}

Clean a four-pound haddock, sprinkle with salt inside and out, stuff,
and sew. Cut five diagonal gashes on each side of backbone and insert
narrow strips of fat salt pork, having gashes on one side come between
gashes on other side. Shape with skewers in form of letter S, and fasten
skewers with small twine. Place on greased fish-sheet in a dripping-pan,
sprinkle with salt and pepper, brush over with melted butter, dredge
with flour, and place around fish small pieces of fat salt pork. Bake
one hour in hot oven, basting as soon as fat is tried out, and continue
basting every ten minutes. Serve with Drawn Butter, Egg or Hollandaise
Sauce.



\needspace{15\baselineskip}
\section*{Fish Stuffing I}


\begin{minipage}{1.0\textwidth}
{\setlength{\multicolsep}{0pt}\setlength{\columnsep}{2em}\raggedcolumns%
\begin{multicols}{2}
\begin{itemize}
\setlength{\itemsep}{0pt}
\setlength{\parsep}{0pt}
\item 1/2 cup cracker crumbs
\item 1/2 cup stale bread crumbs
\item 1/4 cup melted butter
\item 1/4 teaspoon salt
\item 1/8 teaspoon pepper
\item Few drops onion juice
\item 1/4 cup hot water
\end{itemize}
\end{multicols}}
\end{minipage}

\vspace{0.3em}
\noindent%
Mix ingredients in order given.



\needspace{15\baselineskip}
\section*{Fish Stuffing II}


\begin{minipage}{1.0\textwidth}
{\setlength{\multicolsep}{0pt}\setlength{\columnsep}{2em}\raggedcolumns%
\begin{multicols}{2}
\begin{itemize}
\setlength{\itemsep}{0pt}
\setlength{\parsep}{0pt}
\item 1 cup cracker crumbs
\item 1/4 cup melted butter
\item 1/4 teaspoon salt
\item 1/8 teaspoon pepper
\item Few drops onion juice
\item 1 teaspoon parsley
\item 1 teaspoon capers
\item 1 teaspoon pickles
\end{itemize}
\end{multicols}}
\end{minipage}

\vspace{0.3em}
\noindent%
Mix ingredients in order given. This makes a dry, crumbly stuffing.



\needspace{15\baselineskip}
\section*{Baked Bluefish}

Clean a four-pound bluefish, stuff, sew, and bake as Baked Halibut with
Stuffing, omitting to cut gashes on sides, as the fish is rich enough
without addition of pork. Baste often with one-third cup butter melted
in two-thirds cup boiling water. Serve with Shrimp Sauce.



\needspace{15\baselineskip}
\section*{Breslin Baked Bluefish}

Split and bone a bluefish, place on a well-buttered sheet, and cook
twenty minutes in a hot oven. Cream one-fourth cup butter, add yolks two
eggs, and when well mixed add two tablespoons, each, onion, capers,
pickles, and parsley, finely chopped; two tablespoons lemon juice, one
tablespoon vinegar, one-half teaspoon salt, and one-third teaspoon
paprika. Sprinkle fish with salt, spread with mixture, and continue the
baking until fish is done. Remove to serving dish and garnish with
potato balls, cucumber ribbons, lemon cut in fancy shapes, and parsley.



\needspace{15\baselineskip}
\section*{Bluefish À L'Italienne}

Clean a four-pound bluefish, sprinkle with salt and pepper, and put on
buttered fish-sheet in a dripping-pan. Add three tablespoons white wine,
three tablespoons mushroom liquor, one-half onion finely chopped, eight
mushrooms finely chopped, and enough water to allow sufficient liquor in
pan for basting. Bake forty-five minutes in hot oven, basting five
times. Serve with Sauce à l'Italienne.



\needspace{15\baselineskip}
\section*{Baked Cod With Oyster Stuffing}

Clean a four-pound cod, sprinkle with salt and pepper, brush over with
lemon juice, stuff, and sew. Gash, skewer, and bake as Baked Halibut
with Stuffing. Serve with Oyster Sauce.



\needspace{15\baselineskip}
\section*{Oyster Stuffing}


\begin{minipage}{1.0\textwidth}
{\setlength{\multicolsep}{0pt}\setlength{\columnsep}{2em}\raggedcolumns%
\begin{multicols}{2}
\begin{itemize}
\setlength{\itemsep}{0pt}
\setlength{\parsep}{0pt}
\item 1 cup cracker crumbs
\item 1/4 cup melted butter
\item 1/2 teaspoon salt
\item 1/8 teaspoon pepper
\item 1 1/2 teaspoons lemon juice
\item 1/2 tablespoon finely chopped parsley
\item 1 cup oysters
\end{itemize}
\end{multicols}}
\end{minipage}

\vspace{0.3em}
\noindent%
Add seasonings and butter to cracker crumbs. Clean oysters, and remove
tough muscles; add soft parts to mixture, with two tablespoons oyster
liquor to moisten.



\needspace{15\baselineskip}
\section*{Baked Haddock With Oyster Stuffing}

Remove skin, head, and tail from a four-pound haddock. Bone, leaving in
large bones near head, to keep fillets in shape of the original fish.
Sprinkle with salt, and brush over with lemon juice. Lay one fillet on
greased fish-sheet in a dripping-pan, cover thickly with oysters,
cleaned and dipped in buttered cracker crumbs seasoned with salt and
pepper. Cover oysters with other fillet, brush with egg slightly beaten,
cover with buttered crumbs, and bake fifty minutes in a moderate oven.
Serve with Hollandaise Sauce I. Allow one pint oysters and one cup
cracker crumbs.



\needspace{15\baselineskip}
\section*{Baked Halibut With Tomato Sauce}


\begin{minipage}{1.0\textwidth}
{\setlength{\multicolsep}{0pt}\setlength{\columnsep}{2em}\raggedcolumns%
\begin{multicols}{2}
\begin{itemize}
\setlength{\itemsep}{0pt}
\setlength{\parsep}{0pt}
\item 2 lbs. halibut
\item 2 cups tomatoes
\item 1 cup water
\item 1 slice onion
\item 3 cloves
\item 1/2 tablespoon sugar
\item 3 tablespoons butter
\item 3 tablespoons flour
\item 3/4 teaspoon salt
\item 1/8 teaspoon pepper
\end{itemize}
\end{multicols}}
\end{minipage}

\vspace{0.3em}
\noindent%
Cook twenty minutes tomatoes, water, onion, cloves, and sugar. Melt
butter, add flour, and stir into hot mixture. Add salt and pepper, cook
ten minutes, and strain. Clean fish, put in baking-pan, pour around half
the sauce, and bake thirty-five minutes, basting often. Remove to hot
platter, pour around remaining sauce, and garnish with parsley.



\needspace{15\baselineskip}
\section*{Baked Halibut With Lobster Sauce}

Clean a piece of halibut weighing three pounds. Cut gashes in top, and
insert a narrow strip of fat salt pork in each gash. Place in dripping
pan on fish-sheet, sprinkle with salt and pepper, and dredge with flour.
Cover bottom of pan with water, add sprig of parsley, slice of onion,
two slices carrot cut in pieces, and bit of bay leaf. Bake one hour,
basting with one-fourth cup butter and the liquor in pan. Serve with
Lobster Sauce.



\needspace{15\baselineskip}
\section*{Hollenden Halibut}

Arrange six thin slices fat salt pork two and one-half inches square in
a dripping-pan. Cover with one small onion, thinly sliced, and add a bit
of bay leaf. Wipe a two-pound piece of chicken halibut and place over
pork and onion. Mask with three tablespoons butter creamed and mixed
with three tablespoons flour. Cover with three-fourths cup buttered
cracker crumbs and arrange thin strips of fat salt pork over crumbs.
Cover with buttered paper and bake fifty minutes in a moderate oven,
removing paper during the last fifteen minutes of the cooking to brown
crumbs. Remove to hot serving dish and garnish with slices of lemon cut
in fancy shapes sprinkled with finely chopped parsley and paprika.



\needspace{15\baselineskip}
\section*{Baked Mackerel}

Split fish, clean, and remove head and tail. Put in buttered
dripping-pan, sprinkle with salt and pepper, and dot over with butter
(allowing one tablespoon to a medium-sized fish), and pour over
two-thirds cup milk. Bake twenty-five minutes in hot oven.



\needspace{15\baselineskip}
\section*{Planked Shad Or Whitefish}

Clean and split a three-pound shad. Put skin side down on an oak plank
one inch thick, and a little longer and wider than the fish, sprinkle
with salt and pepper, and brush over with melted butter. Bake
twenty-five minutes in hot oven. Remove from oven, spread with butter,
and garnish with parsley and lemon. The fish should be sent to the table
on plank. Planked Shad is well cooked in a gas range having the flame
over the fish.

The Planked Whitefish of the Great Lakes has gained much favor.



\needspace{15\baselineskip}
\section*{Planked Shad With Creamed Roe}

Select a roe shad and prepare same as Planked Shad. Parboil roe in
salted, acidulated water twenty minutes. Remove outside membrane, and
mash. Melt three tablespoons butter, add one teaspoon finely chopped
shallot, and cook five minutes; add roe, sprinkle with one and one-half
tablespoons flour, and pour on gradually one-third cup cream. Cook
slowly five minutes, add two egg yolks and season highly with salt,
pepper, and lemon juice. Remove shad from oven, spread thin part with
roe mixture, cover with buttered crumbs, and return to oven to brown
crumbs. Garnish with mashed potatoes forced through a pastry bag and
tube, small tomatoes, slices of lemon and parsley.



\needspace{15\baselineskip}
\section*{Planked Haddock}

Skin and bone a haddock, leaving meat in two fillets. Sauté fillets
separately, using a generous quantity of butter and cooking until well
browned on one side. Remove to planks, sprinkle with salt and pepper.
Garnish with mashed potatoes, outlining the original shape of the fish,
making as prominent as possible head, tail, and fins. Bake until
potatoes are well browned, when fish should be thoroughly cooked. Finish
garnishing with parsley and slices of lemon sprinkled with finely
chopped parsley.



\needspace{15\baselineskip}
\section*{Baked Stuffed Smelts}

Clean and wipe as dry as possible twelve selected smelts. Stuff,
sprinkle with salt and pepper, and brush over with lemon juice. Place in
buttered shallow plate, cover with buttered paper, and bake five minutes
in hot oven. Remove from oven, sprinkle with buttered crumbs, and bake
until crumbs are brown. Serve with Sauce Bearnaise.

\textbf{Stuffing.} Cook one tablespoon finely chopped onion with one tablespoon
butter three minutes. Add one-fourth cup finely chopped mushrooms,
one-fourth cup soft part of oysters (parboiled, drained, and chopped),
one-half teaspoon chopped parsley, three tablespoons Thick White Sauce,
and one-half cup Fish Force-meat.



\needspace{15\baselineskip}
\section*{Smelts À La Langtry}

Split and bone eight selected smelts. Cut off tails, and from tail ends
of fish turn meat over one inch onto flesh side. Sprinkle with salt and
pepper, and brush over with lemon juice. Garnish with Fish Force-meat
forced through a pastry bag and tube, and fasten heads with skewers to
keep in an upright position. Arrange in a buttered pan, and pour around
white wine. Cover with buttered paper, and bake from fifteen to twenty
minutes. Just before taking from oven, sprinkle with lobster coral
forced through a strainer. Serve with Aurora Sauce.

\textbf{Aurora Sauce.} Melt three tablespoons butter, add three tablespoons
flour, and pour on gradually one and one-half cups cream and one
tablespoon meat extract. Season with salt and cayenne, and add lobster
coral and one-half cup lobster dice.



\needspace{15\baselineskip}
\section*{Baked Shad Roe With Tomato Sauce}

Cook shad roe fifteen minutes in boiling salted water to cover, with
one-half tablespoon vinegar; drain, cover with cold water, and let stand
five minutes. Remove from cold water, and place on buttered pan with
three-fourths cup Tomato Sauce I or II. Bake twenty minutes in hot oven,
basting every five minutes. Remove to a platter, and pour around
three-fourths cup Tomato Sauce.



\needspace{15\baselineskip}
\section*{Baked Fillets Of Bass Or Halibut}

Cut bass or halibut into small fillets, sprinkle with salt and pepper,
put into a shallow pan, cover with buttered paper, and bake twelve
minutes in hot oven. Arrange on a rice border, garnish with parsley, and
serve with Hollandaise Sauce II.



\needspace{15\baselineskip}
\section*{Fillets Of Halibut With Brown Sauce}

Cut a slice of halibut weighing one and one-half pounds in eight short
fillets, sprinkle with salt and pepper, put in greased pan, and bake
five minutes; drain, pour over one and one-half cups Brown Sauce I,
cover with one-half cup buttered cracker crumbs, and bake.



\needspace{15\baselineskip}
\section*{Fillets Of Haddock, White Wine Sauce}

Skin a three and one-half pound haddock, and cut in fillets. Arrange in
buttered baking-pan, pour around fish three tablespoons melted butter,
three-fourths cup white wine to which has been added one-half tablespoon
lemon juice, and two slices onion. Cover and bake. Melt two tablespoons
butter, add two tablespoons flour, and pour on liquor drained from fish;
then add one-half cup Fish Stock (made from head, tail, and bones of
fish), two tablespoons heavy cream, yolks two eggs, salt, and pepper.
Remove fillets to serving dish, pour over sauce strained through
cheese-cloth, and sprinkle with finely chopped parsley.



\needspace{15\baselineskip}
\section*{Halibut À La Poulette}


\begin{minipage}{1.0\textwidth}
{\setlength{\multicolsep}{0pt}\setlength{\columnsep}{2em}\raggedcolumns%
\begin{multicols}{2}
\begin{itemize}
\setlength{\itemsep}{0pt}
\setlength{\parsep}{0pt}
\item 1 slice of halibut, weighing 1 1/2 lbs.
\item 1/4 cup melted butter
\item 1/8 teaspoon pepper
\item 2 teaspoons lemon juice
\item Few drops onion juice
\item 1/4 teaspoon salt
\end{itemize}
\end{multicols}}
\end{minipage}

\vspace{0.3em}
\noindent%
Clean fish and cut in eight fillets. Add seasonings to melted butter,
and put dish containing butter in saucepan of hot water to keep butter
melted. Take up each fillet separately with a fork, dip in butter-roll
and fasten with a small wooden skewer. Put in a shallow pan, dredge with
flour, and bake twelve minutes in hot oven. Remove skewers, arrange on
platter for serving, pour around one and one-half cups Béchamel Sauce,
and garnish with yolks of two hard-boiled eggs rubbed through a
strainer, whites of hard-boiled eggs cut in strips, lemon cut
fan-shaped, and parsley.



\needspace{15\baselineskip}
\section*{Moulded Fish, Normandy Sauce}

Remove skin and bones from a thick piece of halibut, finely chop fish,
and force through a sieve (there should be one and one-third cups).
Pound in mortar, adding gradually whites two eggs. Add one and
one-fourth cups heavy cream, and salt, pepper, and cayenne to taste.
Turn into a buttered fish-mould, cover with buttered paper, set in pan
of hot water, and bake until fish is firm. Turn on serving dish and
surround with





\textbf{Normandy Sauce.} Cook skin and bones of fish with three slices carrot,
one slice onion, sprig of parsley, bit of bay leaf, one-fourth teaspoon
peppercorns, and two cups cold water, thirty minutes, and strain; there
should be one cup. Melt two tablespoons butter, add three tablespoons
flour, fish stock, one-third cup heavy cream, and yolks two eggs. Season
with salt, pepper, cayenne, and Sauterne.



\needspace{15\baselineskip}
\section*{Halibut À La Martin}

Clean two slices chicken halibut and cut into eight fillets. Season with
salt and brush over with lemon juice. Arrange on a tin plate covered
with cheese-cloth, fold cheese-cloth over fillets, and cook in steamer
fifteen minutes. Remove to serving dish, garnish with small shrimps, and
pour around sauce, following directions for Normandy Sauce, omitting
Sauterne, and seasoning to taste with grated cheese and Madeira.



\needspace{15\baselineskip}
\section*{Fillets Of Fish À La Bement}

Prepare and cook fish same as for Halibut à la Martin. Insert tip of
small lobster claw in each fillet, and garnish with a thin slice of
canned mushroom sprinkled with parsley and a thin circular slice of
truffle. Serve with

\textbf{Lobster Sauce III.} Remove meat from a one and one-half pound lobster
and cut claw meat in cubes. Cover remaining meat and body bones with
cold water. Add one-half small onion, sprig of parsley, bit of bay leaf,
and one-fourth teaspoon peppercorns, and cook until stock is reduced to
one cup. Melt three tablespoons butter, add three tablespoons flour, and
pour on gradually the stock; then add one-half cup heavy cream and yolks
two eggs. Season with salt, lemon juice, and paprika; then add lobster
cubes.



\needspace{15\baselineskip}
\section*{Halibut À La Rarebit}

Sprinkle two small slices halibut with salt, pepper, and lemon juice;
then brush over with melted butter, place in dripping-pan on greased
fish-sheet, and bake twelve minutes. Remove to hot platter for serving,
and pour over it a Welsh Rarebit.



\needspace{15\baselineskip}
\section*{Sandwiches Of Chicken Halibut}

Cut chicken halibut in thin fillets. Put together in pairs, with Fish or
Chicken Force-meat between, first dipping fillets in melted butter
seasoned with salt and pepper and brushing over with lemon juice. Place
in shallow pan with one-fourth cup white wine. Bake twenty minutes in
hot oven. Arrange on hot platter for serving, sprinkle with finely
chopped parsley, garnish with Tomato Jelly, and serve with Hollandaise
Sauce.



\needspace{15\baselineskip}
\section*{Sole À La Bercy}

Skin and bone two large flounders, and cut into eight fillets. Put into
a buttered pan, sprinkle with salt, pepper, and lemon juice, and add
one-fourth cup white wine. Cover and cook fifteen minutes. Remove to
serving dish, pour over Bercy Sauce, and sprinkle with finely chopped
parsley.

\textbf{Bercy Sauce.} Fry one tablespoon finely chopped shallot in one
tablespoon butter five minutes; add two tablespoons flour, and pour on
gradually the liquor left in pan with enough White Stock to make one
cup. Add two tablespoons butter, and salt and cayenne to taste.



\needspace{15\baselineskip}
\section*{Halibut Au Lit}

Wipe two slices chicken halibut, each weighing three-fourths pound. Cut
one piece in eight fillets, sprinkle with salt and lemon juice, roll and
fasten with small wooden skewers. Cook over boiling water. Cut remaining
slice in pieces about the size and shape of scallops. Dip in crumbs,
egg, and crumbs, and fry in deep fat. Arrange a steamed fillet in centre
of each fish-plate, place on top of each a cooked mushroom cap, and put
fried fish at both right and left of fillet. Serve with Mushroom Sauce,
and garnish with watercress and radishes cut in fancy shapes.

\textbf{Mushroom Sauce.} Melt three tablespoons butter, add three tablespoons
flour, and pour on gradually, while stirring constantly, one cup Fish
Stock. When boiling-point is reached, add one-half cup cream, three
mushroom caps, sliced, and one tablespoon Sauterne. Season with salt and
pepper. The Fish Stock should be made from skin and bones of halibut.
The mushroom caps on fillets should be cooked in sauce until soft.



\needspace{15\baselineskip}
\section*{Fried Cod Steaks}

Clean steaks, sprinkle with salt and pepper, and dip in granulated corn
meal. Try out slices of fat salt pork in frying-pan, remove scraps, and
sauté steaks in fat.



\needspace{15\baselineskip}
\section*{Fried Smelts}

Clean smelts, leaving on heads and tails. Sprinkle with salt and pepper,
dip in flour, egg, and crumbs, and fry three to four minutes in deep
fat. As soon as smelts are put into fat, remove fat to back of range so
that they may not become too brown before cooked through. Arrange on hot
platter, garnish with parsley, lemon, and fried gelatine. Serve with
Sauce Tartare.

Smelts are fried without being skewered, but often are skewered in
variety of shapes.

\textit{To fry gelatine.} Take up a handful and drop in hot, deep fat; it will
immediately swell and become white; it should at once be removed with a
skimmer, then drained.

Phosphated or granulated gelatine cannot be used for frying.



\needspace{15\baselineskip}
\section*{Smelts À La Menière}

Clean six selected smelts, and cut five diagonal gashes on each side.
Season with salt, pepper, and lemon juice, cover, and let stand ten
minutes. Roll in cream, dip in flour, and sauté in butter. Add to butter
in pan two tablespoons flour, one cup White Stock, one and one-third
teaspoons Anchovy Essence, and a few drops lemon juice. Just before
sauce is poured around smelts, add one and one-half tablespoons butter
and one teaspoon finely chopped parsley.



\needspace{15\baselineskip}
\section*{Fried Fillets Of Halibut Or Flounder}

Clean fish and cut in long or short fillets. If cut in long fillets,
roll, and fasten with small wooden skewers. Sprinkle fillets with salt
and pepper, dip in crumbs, egg, and crumbs, fry in deep fat, and drain
on brown paper. Serve with Sauce Tartare.



\needspace{15\baselineskip}
\section*{Fried Fish, Russian Style, Mushroom Sauce}

Cut two slices chicken halibut in fillets, sprinkle fillets with salt
and pepper, pour over one-third cup white wine, cover, and let stand
thirty minutes. Drain, dip each piece separately in heavy cream, then in
flour, and fry in deep fat. Cook skin and bones removed from fish with
five slices carrot, two slices onion, sprig parsley, bit of bay leaf,
one-fourth teaspoon peppercorns, and two cups cold water until reduced
to one cup liquid. Make sauce of two tablespoons butter, three
tablespoons flour, the fish stock, and one-third cup heavy cream. Add
yolks two eggs, salt, pepper, cayenne, and white wine to taste.

Arrange fish on serving dish, cover with one-half pound mushroom caps
cleaned, then sautéd in butter, and pour over sauce.



\needspace{15\baselineskip}
\section*{Fried Eels}

Clean eels, cut in two-inch pieces, and parboil eight minutes. Sprinkle
with salt and pepper, dip in corn meal, and sauté in pork fat.



\needspace{15\baselineskip}
\section*{Fried Stuffed Smelts}

Smelts are stuffed as for Baked Stuffed Smelts, dipped in crumbs, egg,
and crumbs, fried in deep fat, and served with Sauce Tartare.



\needspace{15\baselineskip}
\section*{Fried Shad Roe}

Parboil and cook shad roe as for Baked Shad Roe. Cut in pieces, sprinkle
with salt and pepper, and brush over with lemon juice. Dip in crumbs,
egg, and crumbs, fry in deep fat, and drain.



\needspace{15\baselineskip}
\section*{Soft-Shell Crabs}

Clean crabs, sprinkle with salt and pepper, dip in crumbs, egg, and
crumbs, fry in deep fat, and drain. Being light, they will rise to top
of fat, and should be turned while frying. Soft-shell crabs are usually
fried. Serve with Sauce Tartare.

\textbf{To Clean a Crab.} Lift and fold back the tapering points which are
found on each side of the back shell, and remove spongy substance that
lies under them. Turn crab on its back, and with a pointed knife remove
the small piece at lower part of shell, which terminates in a point;
this is called the apron.



\needspace{15\baselineskip}
\section*{Frogs' Hind Legs}

Trim and clean. Sprinkle with salt and pepper, dip in crumbs, egg, and
crumbs again, then fry three minutes in deep fat, and drain.



\needspace{15\baselineskip}
\section*{Terrapin}

To prepare terrapin for cooking, plunge into boiling water and boil five
minutes. Lift out of water with skimmer, and remove skin from feet and
tail by rubbing with a towel. Draw out head with a skewer, and rub off
skin.

\textbf{To Cook Terrapin.} Put in a kettle, cover with boiling salted water,
add two slices each of carrot and onion, and a stalk of celery. Cook
until meat is tender, which may be determined by pressing feet-meat
between thumb and finger. The time required will be from thirty-five to
forty minutes. Remove from water, cool, draw out nails from feet, cut
under shell close to upper shell and remove. Empty upper shell and
carefully remove and discard gall-bladder, sandbags, and thick, heavy
part of intestines. Any of the gall-bladder would give a bitter flavor
to the dish. The liver, small intestines, and eggs are used with the
meat.



\needspace{15\baselineskip}
\section*{Terrapin À La Baltimore}


\begin{minipage}{1.0\textwidth}
{\setlength{\multicolsep}{0pt}\setlength{\columnsep}{2em}\raggedcolumns%
\begin{multicols}{2}
\begin{itemize}
\setlength{\itemsep}{0pt}
\setlength{\parsep}{0pt}
\item 1 terrapin
\item 3/4 cup White Stock
\item 1 1/2 tablespoons wine
\item Cayenne
\item 1 1/2 tablespoons butter
\item Salt and pepper
\item 4 egg yolks
\end{itemize}
\end{multicols}}
\end{minipage}

\vspace{0.3em}
\noindent%
To stock and wine add terrapin meat, with bones cut in pieces and
entrails cut in smaller pieces; then cook slowly until liquor is reduced
one-half. Add liver separated in pieces, eggs, butter, salt, pepper, and
cayenne.



\needspace{15\baselineskip}
\section*{Terrapin À La Maryland}

Add to Terrapin à la Baltimore one tablespoon each butter and flour
creamed together, one-half cup cream, yolks two eggs slightly beaten,
and one teaspoon lemon juice; then add, just before serving, one
tablespoon Sherry wine. Pour in a deep dish and garnish with toast or
puff paste points.



\needspace{15\baselineskip}
\section*{Washington Terrapin}


\begin{minipage}{1.0\textwidth}
{\setlength{\multicolsep}{0pt}\setlength{\columnsep}{2em}\raggedcolumns%
\begin{multicols}{2}
\begin{itemize}
\setlength{\itemsep}{0pt}
\setlength{\parsep}{0pt}
\item 1 terrapin
\item 1 1/2 tablespoons butter
\item 1 1/2 tablespoons flour
\item 1 cup cream
\item 1/2 cup chopped mushrooms
\item Salt
\item Few grains cayenne
\item 2 eggs
\item 2 tablespoons Sherry wine
\end{itemize}
\end{multicols}}
\end{minipage}

\vspace{0.3em}
\noindent%
Melt the butter, add flour, and pour on slowly the cream. Add terrapin
meat with bones cut in pieces, entrails cut smaller, liver separated in
pieces, eggs of terrapin, and mushrooms. Season with salt and cayenne.
Just before serving, add eggs slightly beaten and two tablespoons Sherry
wine.



\needspace{15\baselineskip}
\section*{Ways Of Using Remnants Of Cooked Fish}


\needspace{15\baselineskip}
\subsection*{Fish à la Crême}


\begin{minipage}{1.0\textwidth}
{\setlength{\multicolsep}{0pt}\setlength{\columnsep}{2em}\raggedcolumns%
\begin{multicols}{2}
\begin{itemize}
\setlength{\itemsep}{0pt}
\setlength{\parsep}{0pt}
\item 1 3/4 cups cold flaked fish (cod, haddock, halibut, or cusk)
\item 1 cup White Sauce I
\item Bit of bay leaf
\item Sprig of parsley
\item 1/2 slice onion
\item Salt and pepper
\item 1/2 cup buttered cracker crumbs
\end{itemize}
\end{multicols}}
\end{minipage}

\vspace{0.3em}
\noindent%
Scald milk, for the making of White Sauce, with bay leaf, parsley, and
onion. Cover the bottom of small buttered platter with one-half of the
fish, sprinkle with salt and pepper, and pour over one-half the sauce;
repeat. Cover with crumbs, and bake in hot oven until crumbs are brown.
Fish à la crême, baked in scallop shells, makes an attractive luncheon
dish, or may be served for a fish course at dinner.



\needspace{15\baselineskip}
\subsection*{Turban of Fish}


\begin{minipage}{1.0\textwidth}
{\setlength{\multicolsep}{0pt}\setlength{\columnsep}{2em}\raggedcolumns%
\begin{multicols}{2}
\begin{itemize}
\setlength{\itemsep}{0pt}
\setlength{\parsep}{0pt}
\item 2 1/2 cups cold flaked fish (cod, haddock, halibut, or cusk)
\item 1 1/2 cups milk
\item 1 slice onion
\item Blade of mace
\item Sprig of parsley
\item 1/4 cup butter
\item 1/4 cup flour
\item 1/2 teaspoon salt
\item 1/8 teaspoon pepper
\item Lemon juice
\item 4 egg yolks
\item 2/3 cup buttered cracker crumbs
\end{itemize}
\end{multicols}}
\end{minipage}

\vspace{0.3em}
\noindent%
Scald milk with onion, mace, and parsley; remove seasonings. Melt
butter, add flour, salt, pepper, and gradually the milk; then add eggs,
slightly beaten. Put a layer of fish on buttered dish, sprinkle with
salt and pepper, and add a few drops of lemon juice. Cover with sauce,
continuing until fish and sauce are used, shaping in pyramid form. Cover
with crumbs, and bake in hot oven until crumbs are brown.



\needspace{15\baselineskip}
\subsection*{Fish Hash}

Take equal parts of cold flaked fish and cold boiled potatoes finely
chopped. Season with salt and pepper. Try out fat salt pork, remove
scraps, leaving enough fat in pan to moisten fish and potatoes. Put in
fish and potatoes, stir until heated, then cook until well browned
underneath; fold, and turn like an omelet.



\needspace{15\baselineskip}
\subsection*{Fish Croquettes}

To one and one-half cups cold flaked halibut or salmon add one cup thick
White Sauce. Season with salt and pepper, and spread on a plate to cool.
Shape, roll in crumbs, egg, and crumbs, and fry in deep fat; drain,
arrange on hot dish for serving, and garnish with parsley. If salmon is
used, add lemon juice and finely chopped parsley.



\needspace{15\baselineskip}
\subsection*{Fish and Egg Croquettes}

Make same as Fish Croquettes, using one cup fish and three “hard-boiled”
eggs finely chopped.



\needspace{15\baselineskip}
\subsection*{Scalloped Cod}

Line a buttered baking-dish with cold flaked cod, sprinkle with salt and
pepper, cover with a layer of oysters (first dipped in melted butter,
seasoned with onion juice, lemon juice, and a few grains of cayenne, and
then in cracker crumbs), add three tablespoons oyster liquor; repeat,
and cover with buttered cracker crumbs. Bake twenty minutes in hot oven.
Serve with Egg or Hollandaise Sauce I.



\needspace{15\baselineskip}
\subsection*{Salmon Box}

Line a bread pan, slightly buttered, with warm steamed rice. Fill the
centre with cold boiled salmon, flaked, and seasoned with salt, pepper,
and a slight grating of nutmeg. Cover with rice and steam one hour. Turn
on a hot platter for serving, and pour around Egg Sauce II.



\needspace{15\baselineskip}
\section*{Ways Of Cooking Salt Fish}


\needspace{15\baselineskip}
\subsection*{Creamed Salt Codfish}

Pick salt codfish in pieces (there should be three-fourths cup), and
soak in lukewarm water, the time depending upon hardness and saltness of
the fish. Drain, and add one cup White Sauce I. Add one beaten egg just
before sending to table. Garnish with slices of hard-boiled eggs.
Creamed Codfish is better made with cream slightly thickened in place of
White Sauce.



\needspace{15\baselineskip}
\subsection*{Fish Balls}


\begin{itemize}
\setlength{\itemsep}{0pt}
\setlength{\parsep}{0pt}
\item 1 cup salt codfish
\item 2 heaping cups potatoes
\item 1 egg
\item 1/2 tablespoon butter
\item 1/8 teaspoon pepper
\end{itemize}

\vspace{-0.5em}
\noindent%
Wash fish in cold water, and pick in very small pieces, or cut, using
scissors. Wash, pare, and soak potatoes, cutting in pieces of uniform
size before measuring. Cook fish and potatoes in boiling water to cover
until potatoes are soft. Drain through strainer, return to kettle in
which they were cooked, mash thoroughly (being sure there are no lumps
left in potato), add butter, egg well beaten, and pepper. Beat with a
fork two minutes. Add salt if necessary. Take up by spoonfuls, put in
frying-basket, and fry one minute in deep fat, allowing six fish balls
for each frying; drain on brown paper. Reheat the fat after each frying.



\needspace{15\baselineskip}
\subsection*{Salted Codfish Hash}

Prepare as for Fish Balls, omitting egg. Try out fat salt pork, remove
scraps, leaving enough fat in pan to moisten fish and potatoes. Put in
fish and potatoes, stir until heated, then cook until well browned
underneath; fold, and turn like an omelet.



\needspace{15\baselineskip}
\subsection*{Toasted Salt Fish}

Pick salt codfish in long thin strips. If very salt, it may need to be
freshened by standing for a short time in lukewarm water. Place on a
greased wire broiler, and broil until brown on one side; turn, and brown
the other. Remove to platter, and spread with butter.



\needspace{15\baselineskip}
\subsection*{Kippered Herrings}

Remove fish from can, and arrange on a platter that may be put in the
oven; sprinkle with pepper, brush over with lemon juice and melted
butter, and pour over the liquor left in can. Heat thoroughly, and
garnish with parsley and slices of lemon.



\needspace{15\baselineskip}
\subsection*{Baked Finnan Haddie}

Put fish in dripping-pan, surround with milk and water in equal
proportions, place on back of range, where it will heat slowly. Let
stand twenty-five minutes; pour off liquid, spread with butter, and bake
twenty-five minutes.



\needspace{15\baselineskip}
\subsection*{Broiled Finnan Haddie}

Broil in a greased broiler until brown on both sides. Remove to a pan,
and cover with hot water; let stand ten minutes, drain, and place on a
platter. Spread with butter, and sprinkle with pepper.



\needspace{15\baselineskip}
\subsection*{Finnan Haddie à la Delmonico}

Cut fish in strips (there should be one cup), put in baking-pan, cover
with cold water, place on back of range and allow water to heat to
boiling-point; let stand on range, keeping water below boiling-point for
twenty-five minutes, drain, and rinse thoroughly. Separate fish into
flakes, add one-half cup heavy cream and four “hard-boiled” eggs thinly
sliced. Season with cayenne, add one tablespoon butter, and sprinkle
with finely chopped parsley.



\needspace{15\baselineskip}
\section*{Ways Of Cooking Shellfish}


\needspace{15\baselineskip}
\subsection*{Oysters on the Half Shell}

Serve oysters on deep halves of the shells, allowing six to each person.
Arrange on plates of crushed ice, with one-fourth of a lemon in the
centre of each plate.



\needspace{15\baselineskip}
\subsection*{Raw Oysters}

Raw oysters are served on oyster plates, or in a block of ice. Place
block of ice on a folded napkin on platter, and garnish the base with
parsley and quarters of lemon, or ferns and lemon.

\textbf{To Block Ice for Oysters.} Use a rectangular piece of clear ice, and
with hot flatirons melt a cavity large enough to hold the oysters. Pour
water from cavity as rapidly as it forms.



\needspace{15\baselineskip}
\subsection*{Oyster Cocktail I}


\begin{minipage}{1.0\textwidth}
{\setlength{\multicolsep}{0pt}\setlength{\columnsep}{2em}\raggedcolumns%
\begin{multicols}{2}
\begin{itemize}
\setlength{\itemsep}{0pt}
\setlength{\parsep}{0pt}
\item 8 small raw oysters
\item 1 tablespoon tomato catsup
\item 1/2 tablespoon vinegar or lemon juice
\item 2 drops Tabasco
\item Salt
\item 1 teaspoon celery, finely chopped
\item 1/2 teaspoon Worcestershire Sauce
\end{itemize}
\end{multicols}}
\end{minipage}

\vspace{0.3em}
\noindent%
Mix ingredients, chill thoroughly, and serve in cocktail glasses, or
cases made from green peppers placed on a bed of crushed ice.



\needspace{15\baselineskip}
\subsection*{Oyster Cocktail II}


\begin{itemize}
\setlength{\itemsep}{0pt}
\setlength{\parsep}{0pt}
\item 6 small raw oysters
\item Tabasco Sauce
\item Lemon juice
\item Salt
\item Grape fruit
\end{itemize}

\vspace{-0.5em}
\noindent%
Cut grape fruit in halves crosswise, remove tough portions, and add
oysters seasoned with Tabasco, lemon juice, and salt.



\needspace{15\baselineskip}
\subsection*{Oyster Cocktail III}

Allow seven Blue Point oysters to each person, and season with
three-fourth tablespoon lemon juice, one-half tablespoon tomato catsup,
one-half teaspoon finely chopped shallot, three drops Tabasco sauce, few
gratings horseradish root, and salt to taste. Chill thoroughly and serve
in cocktail glasses. Sprinkle with finely chopped celery and garnish
with small pieces of red and green pepper.



\needspace{15\baselineskip}
\subsection*{Roasted Oysters}

Oysters for roasting should be bought in the shell. Wash thoroughly,
scrubbing with a brush. Put in a dripping-pan, and cook in a hot oven
until shells part. Open, sprinkle with salt and pepper, and serve in the
deep halves of the shells.



\needspace{15\baselineskip}
\subsection*{Oysters à la Ballard}

Arrange oysters on the half shell in a dripping-pan, and bake in a hot
oven until edges curl. Allow six to each serve, pouring over the
following sauce:

Mix three-fourths tablespoon melted butter, three-fourths teaspoon each
lemon juice and Sauterne, few drops Tabasco, one-fourth teaspoon finely
chopped parsley, and salt and paprika to taste. Before putting
ingredients in bowl, rub inside of bowl with a clove of garlic.



\needspace{15\baselineskip}
\subsection*{Panned Oysters}

Clean one pint large oysters. Place in dripping-pan small oblong pieces
of toast, put an oyster on each piece, sprinkle with salt and pepper,
and bake until oysters are plump. Serve with Lemon Butter.

\textbf{Lemon Butter.} Cream three tablespoons butter, add one-half teaspoon
salt, one tablespoon lemon juice, and a few grains cayenne.



\needspace{15\baselineskip}
\subsection*{Fancy Roast}

Clean one pint oysters and drain from their liquor. Put in a stewpan and
cook until oysters are plump and edges begin to curl. Shake pan to
prevent oysters from adhering to pan, or stir with a fork. Season with
salt, pepper, and two tablespoons butter, and pour over four small
slices of toast. Garnish with toast points and parsley.



\needspace{15\baselineskip}
\subsection*{Oyster Fricassee}


\begin{minipage}{1.0\textwidth}
{\setlength{\multicolsep}{0pt}\setlength{\columnsep}{2em}\raggedcolumns%
\begin{multicols}{2}
\begin{itemize}
\setlength{\itemsep}{0pt}
\setlength{\parsep}{0pt}
\item 1 pint oysters
\item Milk or cream
\item 2 tablespoons butter
\item 2 tablespoons flour
\item 1/4 teaspoon salt
\item Few grains cayenne
\item 1 teaspoon finely chopped parsley
\item 1 egg
\end{itemize}
\end{multicols}}
\end{minipage}

\vspace{0.3em}
\noindent%
Clean oysters, heat oyster liquor to boiling-point, and strain through
double thickness of cheese-cloth; add oysters to liquor and cook until
plump. Remove oysters with skimmer and add enough cream to liquor to
make a cupful. Melt butter, add flour, and pour on gradually hot liquid;
add salt, cayenne, parsley, oysters, and egg slightly beaten.



\needspace{15\baselineskip}
\subsection*{Creamed Oysters}


\begin{itemize}
\setlength{\itemsep}{0pt}
\setlength{\parsep}{0pt}
\item 1 pint oysters
\item 1 1/2 cups White Sauce II
\item 1/8 teaspoon celery salt
\end{itemize}

\vspace{-0.5em}
\noindent%
Clean, and cook oysters until plump and edges begin to curl; drain, and
add to White Sauce seasoned with celery salt. Serve on toast, in timbale
cases, patty shells, or vol-au-vents. One-fourth cup sliced mushrooms
are often added to Creamed Oysters.



\needspace{15\baselineskip}
\subsection*{Oysters in Brown Sauce}


\begin{minipage}{1.0\textwidth}
{\setlength{\multicolsep}{0pt}\setlength{\columnsep}{2em}\raggedcolumns%
\begin{multicols}{2}
\begin{itemize}
\setlength{\itemsep}{0pt}
\setlength{\parsep}{0pt}
\item 1 pint oysters
\item 1/4 cup butter
\item 1/4 cup flour
\item 1 cup oyster liquor
\item 1/2 cup milk
\item 1/2 teaspoon salt
\item 1 teaspoon Anchovy essence
\item 1/8 teaspoon pepper
\end{itemize}
\end{multicols}}
\end{minipage}

\vspace{0.3em}
\noindent%
Parboil and drain oysters, reserve liquor, heat, strain, and set aside
for sauce. Brown butter, add flour, and stir until well browned; then
add oyster liquor, milk, seasonings, and oysters. For filling patty
cases or vol-au-vents.



\needspace{15\baselineskip}
\subsection*{Savory Oysters}


\begin{minipage}{1.0\textwidth}
{\setlength{\multicolsep}{0pt}\setlength{\columnsep}{2em}\raggedcolumns%
\begin{multicols}{2}
\begin{itemize}
\setlength{\itemsep}{0pt}
\setlength{\parsep}{0pt}
\item 1 pint of oysters
\item 4 tablespoons butter
\item 4 tablespoons flour
\item 1 cup oyster liquor
\item 1/2 cup Brown Stock
\item 1 teaspoon Worcestershire Sauce
\item Few drops onion juice
\item Salt
\item Pepper
\end{itemize}
\end{multicols}}
\end{minipage}

\vspace{0.3em}
\noindent%
Clean oysters, parboil, and drain. Melt butter, add flour, and stir
until well browned. Pour on gradually, while stirring constantly, oyster
liquor and stock. Add seasonings and oysters. Serve on toast, in timbale
cases, patty shells, or vol-au-vents.



\needspace{15\baselineskip}
\subsection*{Oysters à la Astor}


\begin{minipage}{1.0\textwidth}
{\setlength{\multicolsep}{0pt}\setlength{\columnsep}{2em}\raggedcolumns%
\begin{multicols}{2}
\begin{itemize}
\setlength{\itemsep}{0pt}
\setlength{\parsep}{0pt}
\item 1 pint oysters
\item 2 tablespoons butter
\item 1 teaspoon finely chopped shallot
\item 1 tablespoon finely cut red pepper
\item 2 tablespoons flour
\item 1 1/2 teaspoons lemon juice
\item 1 1/2 teaspoons vinegar
\item 1 teaspoon Worcestershire Sauce
\item 1/2 teaspoon beef extract
\item Salt and paprika
\end{itemize}
\end{multicols}}
\end{minipage}

\vspace{0.3em}
\noindent%
Wash and pick over oysters, parboil, drain, and to liquor add enough
water to make one cup liquid; then strain through cheese-cloth. Cook
butter, shallot, and pepper three minutes, add flour, and pour on
gradually, while stirring constantly, oyster liquor. Add seasonings and
oysters. Remove oysters to small pieces of bread sautéd in butter on one
side. Pour sauce over oysters and garnish with thin slices of cucumber
pickles.



\needspace{15\baselineskip}
\subsection*{Broiled Oysters}


\begin{itemize}
\setlength{\itemsep}{0pt}
\setlength{\parsep}{0pt}
\item 1 pint selected oysters
\item 1/4 cup melted butter
\item 2/3 cup seasoned cracker crumbs
\end{itemize}

\vspace{-0.5em}
\noindent%
Clean oysters and dry between towels. Lift with plated fork by the tough
muscle and dip in butter, then in cracker crumbs which have been
seasoned with salt and pepper. Place in a buttered wire broiler and
broil over a clear fire until juices flow, turning while broiling. Serve
with or without Maître d'Hôtel Butter.



\needspace{15\baselineskip}
\subsection*{Oyster Toast}

Serve Broiled Oysters on small pieces of Milk Toast. Sprinkle with
finely chopped celery.



\needspace{15\baselineskip}
\subsection*{Oysters and Macaroni}


\begin{minipage}{1.0\textwidth}
{\setlength{\multicolsep}{0pt}\setlength{\columnsep}{2em}\raggedcolumns%
\begin{multicols}{2}
\begin{itemize}
\setlength{\itemsep}{0pt}
\setlength{\parsep}{0pt}
\item 1 pint oysters
\item 3/4 cup macaroni broken in 1 inch pieces
\item Salt and pepper
\item Flour
\item 1/2 cup buttered crumbs
\item 1/4 cup butter
\end{itemize}
\end{multicols}}
\end{minipage}

\vspace{0.3em}
\noindent%
Cook macaroni in boiling salted water until soft; drain, and rinse with
cold water. Put a layer in bottom of a buttered pudding-dish, cover with
oysters, sprinkle with salt and pepper, dredge with flour, and dot over
with one-half of the butter; repeat, and cover with buttered crumbs.
Bake twenty minutes in hot oven.



\needspace{15\baselineskip}
\subsection*{Scalloped Oysters}


\begin{minipage}{1.0\textwidth}
{\setlength{\multicolsep}{0pt}\setlength{\columnsep}{2em}\raggedcolumns%
\begin{multicols}{2}
\begin{itemize}
\setlength{\itemsep}{0pt}
\setlength{\parsep}{0pt}
\item 1 pint oysters
\item 4 tablespoons oyster liquor
\item 2 tablespoons milk or cream
\item 1/2 cup stale bread crumbs
\item 1 cup cracker crumbs
\item 1/2 cup melted butter
\item Salt
\item Pepper
\end{itemize}
\end{multicols}}
\end{minipage}

\vspace{0.3em}
\noindent%
Mix bread and cracker crumbs, and stir in butter. Put a thin layer in
bottom of a buttered shallow baking-dish, cover with oysters, and
sprinkle with salt and pepper; add one-half each oyster liquor and
cream. Repeat, and cover top with remaining crumbs. Bake thirty minutes
in hot oven. Never allow more than two layers of oysters for Scalloped
Oysters; if three layers are used, the middle layer will be underdone,
while others are properly cooked. A sprinkling of mace or grated nutmeg
to each layer is considered by many an improvement. Sherry wine may be
used in place of cream.



\needspace{15\baselineskip}
\subsection*{Sautéd Oysters}

Clean one pint oysters, sprinkle on both sides with salt and pepper.
Take up by the tough muscle with plated fork and dip in cracker crumbs.
Put two tablespoons butter in hot frying-pan, add oysters, brown on one
side, then turn and brown on the other.



\needspace{15\baselineskip}
\subsection*{Oysters with Bacon}

Clean oysters, wrap a thin slice of bacon around each, and fasten with
small wooden skewers. Put in a broiler, place broiler over dripping-pan,
and bake in a hot oven until bacon is crisp and brown, turning broiler
once during the cooking. Drain on brown paper.



\needspace{15\baselineskip}
\subsection*{Fried Oysters}

Clean, and dry between towels, selected oysters. Season with salt and
pepper, dip in flour, egg, and cracker or stale bread crumbs, and fry in
deep fat. Drain on brown paper and serve on a folded napkin. Garnish
with parsley and serve with or without Sauce Tyrolienne.



\needspace{15\baselineskip}
\subsection*{Fried Oysters in Batter}

Clean, and dry between towels, selected oysters. Dip in batter, fry in
deep fat, drain, and serve on a folded napkin; garnish with lemon and
parsley. Oysters may be parboiled, drained, and then fried.



\needspace{15\baselineskip}
\subsection*{Batter}


\begin{itemize}
\setlength{\itemsep}{0pt}
\setlength{\parsep}{0pt}
\item 2 eggs
\item 1 teaspoon salt
\item 1/8 teaspoon pepper
\item 1 cup bread flour
\item 3/4 cup milk
\end{itemize}

\vspace{-0.5em}
\noindent%
Beat eggs until light, add salt and pepper. Add milk slowly to flour,
stir until smooth and well mixed. Combine mixtures.



\needspace{15\baselineskip}
\subsection*{Fried Oysters. Philadelphia Relish}

Follow directions for Fried Oysters. Serve with \textbf{Philadelphia Relish}.


\begin{minipage}{1.0\textwidth}
{\setlength{\multicolsep}{0pt}\setlength{\columnsep}{2em}\raggedcolumns%
\begin{multicols}{2}
\begin{itemize}
\setlength{\itemsep}{0pt}
\setlength{\parsep}{0pt}
\item 2 cups cabbage, finely shredded
\item 2 green peppers, finely chopped
\item 1 teaspoon celery seed
\item 1/4 teaspoon mustard seed
\item 1/2 teaspoon salt
\item 2 tablespoons brown sugar
\item 1/4 cup vinegar
\end{itemize}
\end{multicols}}
\end{minipage}

\vspace{0.3em}
\noindent%
Mix ingredients in order given.



\needspace{15\baselineskip}
\subsection*{Little Neck Clams}

Little Neck Clams are served raw on the half shell, in same manner as
raw oysters.



\needspace{15\baselineskip}
\subsection*{Steamed Clams}

Clams for steaming should be bought in the shell and always be alive.
Wash clams thoroughly, scrubbing with a brush, changing the water
several times. Put into a large kettle, allowing one-half cup hot water
to four quarts clams; cover closely, and steam until shells partially
open, care being taken that they are not overdone. Serve with individual
dishes of melted butter. Some prefer a few drops of lemon juice or
vinegar added to the butter. If a small quantity of boiling water is put
into the dishes, the melted butter will float on top and remain hot much
longer.



\needspace{15\baselineskip}
\subsection*{Roasted Clams}

Roasted clams are served at Clam Bakes. Clams are washed in sea-water,
placed on stones which have been previously heated by burning wood on
them, ashes removed, and stones sprinkled with thin layer of seaweed.
Clams are piled on stones, covered with seaweed, and a piece of canvas
thrown over them to retain the steam.



\needspace{15\baselineskip}
\subsection*{Clams, Union League}

Fry one-half teaspoon finely chopped shallot in one and one-half
tablespoons butter five minutes; add eighteen clams and one-half cup
white wine. Cook until the shells open. Remove clams from shells and
reduce liquor to one-third cupful. Melt two tablespoons butter, add two
tablespoons flour, and pour on gradually the clam liquor; add one-fourth
cup cream and the clams, season with salt and pepper. Refill
clam-shells, sprinkle with chopped parsley, and serve on each a square
piece of fried bacon.



\needspace{15\baselineskip}
\subsection*{Clams à la Grand Union}

Clean and dry selected clams, dip in batter, fry in deep fat, and drain
on brown paper. Serve on small slices of cream toast, seasoned with
salt, celery salt, pepper, and cayenne.

\textbf{Batter.} Mix and sift one cup bread flour, one-half teaspoon salt, and
a few grains cayenne. Add gradually two-thirds cup milk, and two eggs
well beaten.



\needspace{15\baselineskip}
\subsection*{Fried Scallops}

Clean one quart scallops, turn into a saucepan, and cook until they
begin to shrivel; drain, and dry between towels. Season with salt and
pepper, roll in fine crumbs, dip in egg, again in crumbs, and fry two
minutes in deep fat; then drain on brown paper.



\needspace{15\baselineskip}
\section*{Plain Lobster}

Remove lobster meat from shell, arrange on platter, and garnish with
small claws. If two lobsters are opened, stand tail shells (put
together) in centre of platter, and arrange meat around them.



\needspace{15\baselineskip}
\section*{Lobster Cocktail}

Allow one-fourth cup lobster meat, cut in pieces, for each cocktail, and
season with two tablespoons, each, tomato catsup and Sherry wine, one
tablespoon lemon juice, six drops Tabasco Sauce, one-eighth teaspoon
finely chopped chives, and salt to taste. Chill thoroughly, and serve in
cocktail glasses.



\needspace{15\baselineskip}
\section*{Fried Lobster}

Remove lobster meat from shell. Use tail meat, divided in fourths, and
large pieces of claw meat. Sprinkle with salt, pepper, and lemon juice;
dip in crumbs, egg, and again in crumbs; fry in deep fat, drain, and
serve with Sauce Tartare.







\needspace{15\baselineskip}
\section*{Buttered Lobster}


\begin{itemize}
\setlength{\itemsep}{0pt}
\setlength{\parsep}{0pt}
\item 2 lb. lobster
\item 3 tablespoons butter
\item Salt and pepper
\item Lemon juice
\end{itemize}

\vspace{-0.5em}
\noindent%
Remove lobster meat from shell and chop slightly. Melt butter, add
lobster, and when heated, season and serve garnished with lobster claws.



\needspace{15\baselineskip}
\section*{Scalloped Lobster}


\begin{itemize}
\setlength{\itemsep}{0pt}
\setlength{\parsep}{0pt}
\item 2 lb. lobster
\item 1 1/2 cups White Sauce II
\item 1/2 teaspoon salt
\item Few grains cayenne
\item 2 teaspoons lemon juice
\end{itemize}

\vspace{-0.5em}
\noindent%
Remove lobster meat from shell and cut in cubes. Heat in White Sauce and
add seasonings. Refill lobster shells, cover with buttered crumbs, and
bake until crumbs are brown. To prevent lobster shells from curling over
lobster while baking, insert small wooden skewers of sufficient length
to keep shell in its original shape. To assist in preserving color of
shell, brush over with olive oil before putting into oven. Scalloped
lobster may be baked in buttered scallop shells, or in a buttered
baking-dish.



\needspace{15\baselineskip}
\section*{Devilled Lobster}

Scalloped lobster highly seasoned is served as Devilled Lobster. Use
larger proportions of same seasonings, with the addition of mustard.



\needspace{15\baselineskip}
\section*{Curried Lobster}

Prepare as Scalloped Lobster, adding to flour one-half teaspoon curry
powder when making White Sauce.



\needspace{15\baselineskip}
\section*{Lobster Farci}


\begin{minipage}{1.0\textwidth}
{\setlength{\multicolsep}{0pt}\setlength{\columnsep}{2em}\raggedcolumns%
\begin{multicols}{2}
\begin{itemize}
\setlength{\itemsep}{0pt}
\setlength{\parsep}{0pt}
\item 1 cup chopped lobster meat
\item Yolks 2 “hard-boiled” eggs
\item 1/2 tablespoon chopped parsley
\item 1 cup White Sauce I
\item Slight grating nutmeg
\item 1/3 cup buttered crumbs
\item Salt
\item Pepper
\end{itemize}
\end{multicols}}
\end{minipage}

\vspace{0.3em}
\noindent%
To lobster meat add yolks of eggs rubbed to a paste, parsley, sauce, and
seasonings to taste. Fill lobster shells, cover with buttered crumbs,
and bake until crumbs are brown.



\needspace{15\baselineskip}
\section*{Lobster And Oyster Filling}

                     (\textit{For Patties or Vol-au-Vent})


\begin{minipage}{1.0\textwidth}
{\setlength{\multicolsep}{0pt}\setlength{\columnsep}{2em}\raggedcolumns%
\begin{multicols}{2}
\begin{itemize}
\setlength{\itemsep}{0pt}
\setlength{\parsep}{0pt}
\item 1 pint oysters
\item 1 1/4 lb. lobster
\item 1 1/2 cups cold water
\item 1 stalk celery
\item 1 slice onion
\item Salt
\item 1/4 cup butter
\item 1/3 cup flour
\item 3/4 cup cream
\item Worcestershire Sauce
\item Lemon juice
\item Paprika
\end{itemize}
\end{multicols}}
\end{minipage}

\vspace{0.3em}
\noindent%
Clean and parboil oysters; drain, and add to liquor body bones and tough
claw meat from lobster, water, celery, and onion. Cook slowly until
stock is reduced to one cup, and strain. Make sauce of butter, flour,
strained stock, and cream. Add oysters and lobster meat cut in strips;
then add seasonings. One-half teaspoon beef extract is an improvement to
this dish.



\needspace{15\baselineskip}
\section*{Fricassee Of Lobster And Mushrooms}


\begin{minipage}{1.0\textwidth}
{\setlength{\multicolsep}{0pt}\setlength{\columnsep}{2em}\raggedcolumns%
\begin{multicols}{2}
\begin{itemize}
\setlength{\itemsep}{0pt}
\setlength{\parsep}{0pt}
\item 2 lb. lobster
\item 1/4 cup butter
\item 3/4 lb. mushrooms
\item Few drops onion juice
\item 1/4 cup flour
\item 1 1/2 cups milk
\item Salt
\item Paprika
\item 2 tablespoons Sherry wine
\end{itemize}
\end{multicols}}
\end{minipage}

\vspace{0.3em}
\noindent%
Remove lobster meat from shell and cut in strips. Cook butter with
mushrooms broken in pieces and onion juice three minutes; add flour, and
pour on gradually milk. Add lobster meat, season with salt and paprika,
and, as soon as lobster is heated, add wine. Remove to serving dish, and
garnish with puff paste or toast points and parsley.



\needspace{15\baselineskip}
\section*{Lobster And Oyster Ragout}


\begin{minipage}{1.0\textwidth}
{\setlength{\multicolsep}{0pt}\setlength{\columnsep}{2em}\raggedcolumns%
\begin{multicols}{2}
\begin{itemize}
\setlength{\itemsep}{0pt}
\setlength{\parsep}{0pt}
\item 1/4 cup butter
\item 1/4 cup flour
\item 3/4 cup oyster liquor
\item 3/4 cup cream
\item 3/4 teaspoon salt
\item 1/4 teaspoon pepper
\item Few grains cayenne
\item Few drops onion juice
\item 1 pint oysters parboiled
\item 3/4 cup lobster dice
\item 1 1/2 tablespoons Sauterne
\item 1 tablespoon finely chopped parsley
\end{itemize}
\end{multicols}}
\end{minipage}

\vspace{0.3em}
\noindent%
Make a sauce of first eight ingredients. Add oysters, lobster dice,
wine, and parsley.



\needspace{15\baselineskip}
\section*{Stuffed Lobster À La Béchamel}


\begin{minipage}{1.0\textwidth}
{\setlength{\multicolsep}{0pt}\setlength{\columnsep}{2em}\raggedcolumns%
\begin{multicols}{2}
\begin{itemize}
\setlength{\itemsep}{0pt}
\setlength{\parsep}{0pt}
\item 2 lb. lobster
\item 1 1/2 cups milk
\item Bit of bay leaf
\item 3 tablespoons butter
\item 3 tablespoons flour
\item 1/2 teaspoon salt
\item Few grains cayenne
\item Slight grating nutmeg
\item 1 teaspoon chopped parsley
\item 1 teaspoon lemon juice
\item 4 egg yolks
\item 1/2 cup buttered crumbs
\end{itemize}
\end{multicols}}
\end{minipage}

\vspace{0.3em}
\noindent%
Remove lobster meat from shell and cut in dice. Scald milk with bay
leaf, remove bay leaf and make a white sauce of butter, flour, and milk;
add salt, cayenne, nutmeg, parsley, yolks of eggs slightly beaten, and
lemon juice. Add lobster dice, refill shells, cover with buttered
crumbs, and bake until crumbs are brown. One-half chicken stock and
one-half cream may be used for sauce if a richer dish is desired.



\needspace{15\baselineskip}
\section*{Broiled Live Lobster}

Live lobsters may be dressed for broiling at market, or may be done at
home. Clean lobster and place in a buttered wire broiler. Broil eight
minutes on flesh side, turn and broil six minutes on shell side. Serve
with melted butter. Lobsters taste nearly the same when placed in
dripping-pan and baked fifteen minutes in hot oven, and are much easier
cooked.

\textbf{To Split a Live Lobster.} Cross large claws and hold firmly with left
hand. With sharp-pointed knife, held in right hand, begin at the mouth
and make a deep incision, and, with a sharp cut, draw the knife quickly
through body and entire length of tail. Open lobster, remove intestinal
vein, liver, and stomach, and crack claw shells with a mallet.



\needspace{15\baselineskip}
\section*{Baked Live Lobster. Devilled Sauce.}

Prepare lobster same as for Broiled Live Lobster and place in a
dripping-pan. Cook liver of lobster with one tablespoon butter three
minutes. Season highly with salt, cayenne, and Worcestershire Sauce.
Spread over lobster, and bake in a hot oven fifteen minutes. Remove to
platter and serve at once, allowing over one and one-half pound lobster
to each person.



\needspace{15\baselineskip}
\section*{Live Lobster En Brochette}

Split a live lobster, remove meat from tail and large claws, cut in
pieces, and arrange on skewers, alternating pieces with small slices of
bacon. Fry in deep fat and drain. Cook liver of lobster with one
tablespoon butter three minutes, season highly with mustard and cayenne,
and serve with lobster.



\needspace{15\baselineskip}
\section*{Lobster À L'Américaine}

Split a live lobster and put in a large omelet pan, sprinkle with
one-fourth onion finely chopped and a few grains of cayenne and cook
five minutes. Add one-half cup Tomato Sauce II and cook three minutes;
then add two tablespoons Sherry wine, cover, and cook in oven seven
minutes. To the liver add one tablespoon wine, two tablespoons Tomato
Sauce, and one-half tablespoon melted butter; heat in pan after lobster
has been removed. As soon as sauce is heated, strain, and pour over
lobster.



\needspace{15\baselineskip}
\section*{Lobster À La Muisset}

Cut two one and one-half pound live lobsters in pieces for serving and
crack large claws. Cook one tablespoon finely chopped shallot and three
tablespoons chopped carrot in two tablespoons butter ten minutes,
stirring constantly that carrots may not burn. Add two sprigs thyme,
one-half bay leaf, two red peppers from pepper sauce, one teaspoon salt,
one and one-third cups Brown Stock, two-thirds cup stewed and strained
tomatoes, and three tablespoons Sherry wine. Add lobster and cook
fifteen minutes. Remove lobster to serving dish, thicken sauce with
butter and flour cooked together, and add one and one-half tablespoons
brandy. Pour sauce around lobster, and sprinkle all with finely chopped
chives.





\chapter{Beef}



Meat is the name applied to the flesh of all animals used for food. Beef
is the meat of steer, ox, or cow, and is the most nutritious and largely
consumed of all animal foods. Meat is chiefly composed of the
albuminoids (fibrin, albumen, gelatin), fat, mineral matter, and water.

\textbf{Fibrin} is that substance in blood which causes it to coagulate when
shed. It consists of innumerable delicate fibrils which entangle the
blood corpuscles, and form with them a mass called blood clot. Fibrin is
insoluble in both cold and hot water.

\textbf{Albumen} is a substance found in the blood and muscle. It is soluble in
cold water, and is coagulated by hot water or heat. It begins to
coagulate at 134deg F. and becomes solid at 160deg F. Here lies the
necessity of cooking meat in hot water at a low temperature; of broiling
meat at a high temperature, to quickly sear surface.

\textbf{Gelatin} in its raw state is termed \textit{collagen}. It is a transparent,
tasteless substance, obtained by boiling with water, muscle, skin,
cartilage, bone, tendon, ligament, or membrane of animals. By this
process, collagen of connective tissues is dissolved and converted into
gelatin. Gelatin is insoluble in cold water, soluble in hot water, but
in boiling water is decomposed, and by much boiling will not solidify on
cooling. When subjected to cold water it swells, and is called hydrated
gelatin. Myosin is the albuminoid of muscle, collagen of tendons, ossein
of bones, and chondrin of cartilage and gristle.

Gelatin, although highly nitrogenous, does not act in the system as
other nitrogenous foods, as a large quantity passes out unchanged.

\textbf{Fat} is the white or yellowish oily solid substance forming the chief
part of the adipose tissue. Fat is found in thick layers directly under
the skin, in other parts of the body, in bone, and is intermingled
throughout the flesh. Fat as food is a great heat-giver and
force-producer. \textit{Suet} is the name given to fat which lies about the
loins and kidneys. Beef suet tried out and clarified is much used in
cookery for shortening and frying.

\textbf{Mineral Matter.} The largest amount of mineral matter is found in bone.
It is principally calcium phosphate (phosphate of lime). Sodium chloride
(common salt) is found in the blood and throughout the tissues.

\textbf{Water} abounds in all animals, constituting a large percentage of their
weight.

The color of meat is due to the coloring matter (hæmoglobin) which
abounds in the red corpuscles of the blood.

The distinctive flavor of meat is principally due to peptones and allied
substances, and is intensified by the presence of sodium chloride and
other salts.

The beef creature is divided by splitting through the backbone in two
parts, each part being called \textit{a side of beef}. Four hundred and fifty
pounds is good market weight for a side of beef.

The most expensive cuts come from that part of the creature where
muscles are but little used, which makes the meat finer-grained and
consequently more tender, taking less time for cooking. Many of the
cheapest cuts, though equally nutritious, need long, slow cooking to
render them tender enough to digest easily. Tough meat which has long
and coarse fibres is often found to be very juicy, on account of the
greater motion of that part of the creature, which causes the juices to
flow freely. Roasting and broiling, which develop so fine a flavor, can
only be applied to the more expensive cuts. The liver, kidneys, and
heart are of firm, close texture, and difficult of digestion. Tripe,
which is the first stomach of the ox, is easy of digestion, but on
account of the large amount of fat which it contains, it is undesirable
for those of weak digestion.

The quality of beef depends on age of the creature and manner of
feeding. The best beef is obtained from a steer of four or five years.
Good beef should be firm and of fine-grained texture, bright red in
color, and well mottled and coated with fat. The fat should be firm and
of a yellowish color. Suet should be dry, and crumble easily. Beef
should not be eaten as soon as killed, but allowed to hang and
ripen,--from two to three weeks in winter, and two weeks in summer.

Meat should be removed from paper as soon as it comes from market,
otherwise paper absorbs some of the juices.

Meat should be kept in a cool place. In winter, beef may be bought in
large quantities and cut as needed. If one chooses, a loin or rump may
be bought and kept by the butcher, who sends cuts as ordered.

Always wipe beef, before cooking, with a cheese-cloth wrung out of cold
water, but never allow it to stand in a pan of cold water, as juices
will be drawn out.



\clearpage

\needspace{15\baselineskip}
\section*{Division And Ways Of Cooking A Side Of Beef}


\begin{tabular}{|p{2.5in}|p{2.5in}|}
\hline
\multicolumn{2}{|c|}{\textbf{HIND-QUARTER}} \\
\hline
Flank (thick and boneless) & Stuffed, rolled and braised, or corned and boiled \\
\arrayrulecolor{tablerowgray}\hline
Round -- Aitchbone & Cheap roast, beef stew, or braised \\
\arrayrulecolor{tablerowgray}\hline
\hspace{0.3in}Top & Steaks, best cuts for beef tea \\
\arrayrulecolor{tablerowgray}\hline
\hspace{0.3in}Lower Part & Hamburg steaks, curry of beef, and cecils \\
\arrayrulecolor{tablerowgray}\hline
\hspace{0.3in}Vein & Steaks \\
\arrayrulecolor{tablerowgray}\hline
Rump -- Back & Choicest large roasts and cross-cut steaks \\
\arrayrulecolor{tablerowgray}\hline
\hspace{0.3in}Middle & Roasts \\
\arrayrulecolor{tablerowgray}\hline
\hspace{0.3in}Face & Inferior roasts and stews \\
\arrayrulecolor{tablerowgray}\hline
Loin -- Tip & Extra fine roasts \\
\arrayrulecolor{tablerowgray}\hline
\hspace{0.3in}Middle & Sirloin and porterhouse steaks \\
\arrayrulecolor{tablerowgray}\hline
\hspace{0.3in}First Cut & Steaks and roast \\
\arrayrulecolor{tablerowgray}\hline
The Tenderloin -- Sold as a Fillet or cut in Steaks & Larded and roasted, or broiled \\
\arrayrulecolor{tablerowgray}\hline
Hind-shin & Cheap stew or soup stock \\
\arrayrulecolor{tablerowgray}\hline
\multicolumn{2}{|c|}{\textbf{FORE-QUARTER}} \\
\hline
Five Prime Ribs & Good roast \\
\arrayrulecolor{tablerowgray}\hline
Five Chuck Rib & Small steaks and stews \\
\arrayrulecolor{tablerowgray}\hline
Neck & Hamburg steaks \\
\arrayrulecolor{tablerowgray}\hline
Sticking-piece & Mincemeat \\
\arrayrulecolor{tablerowgray}\hline
Rattle Rand -- Second Cut Thin End & Corned for boiling \\
\arrayrulecolor{tablerowgray}\hline
Brisket -- Butt End or Fancy Brisket & Finest pieces for corning \\
\arrayrulecolor{tablerowgray}\hline
Fore-shin & Soup stock and stews \\
\arrayrulecolor{tablerowgray}\hline
\multicolumn{2}{|c|}{\textbf{Other Parts of Beef Creature used for Food}} \\
\hline
Brains & Stewed, scalloped dishes, or croquettes \\
\arrayrulecolor{tablerowgray}\hline
Tongue & Boiled or braised, fresh or corned \\
\arrayrulecolor{tablerowgray}\hline
Heart & Stuffed and braised \\
\arrayrulecolor{tablerowgray}\hline
Liver & Broiled or fried \\
\arrayrulecolor{tablerowgray}\hline
Kidneys & Stewed or sautéd \\
\arrayrulecolor{tablerowgray}\hline
Tail & Soup \\
\arrayrulecolor{tablerowgray}\hline
Tripe & Lyonnaise, broiled, or fried in batter \\
\arrayrulecolor{tablerowgray}\hline
\arrayrulecolor{black}
\end{tabular}


\needspace{15\baselineskip}
\section*{The Effect Of Different Temperatures On The Cooking Of Meat}

By putting meat in cold water and allowing water to heat gradually, a
large amount of juice is extracted and meat is tasteless; and by long
cooking the connective tissues are softened and dissolved, which gives
to the stock when cold a jelly-like consistency. This principle applies
to soup making.

By putting meat in boiling water, allowing the water to boil for a few
minutes, then lowering the temperature, juices in the outer surface are
quickly coagulated, and the inner juices are prevented from escaping.
This principle applies where nutriment and flavor is desired in meat.
Examples: boiled mutton, fowl.

By putting in cold water, bringing quickly to the boiling-point, then
lowering the temperature and cooking slowly until meat is tender, some
of the goodness will be in the stock, but a large portion left in the
meat. Examples: fowl, when cooked to use for made-over dishes, Scotch
Broth.







\clearpage

\needspace{15\baselineskip}
\section*{Table Showing Composition Of Meats}


\begin{tabular}{p{2in}ccccc}
\hline
Item & Refuse & Proteid & Fat & Mineral matter & Water \\
\hline
Fore-quarter & 19.8 & 14.1 & 16.1 & .7 & 49.3 \\
\arrayrulecolor{tablerowgray}\hline
Hind-quarter & 16.3 & 15.3 & 15.6 & .8 & 52. \\
\arrayrulecolor{tablerowgray}\hline
Round & 8.5 & 18.7 & 8.8 & 1. & 63. \\
\arrayrulecolor{tablerowgray}\hline
Rump & 18.5 & 14.4 & 19. & .8 & 47.3 \\
\arrayrulecolor{tablerowgray}\hline
Loin & 12.6 & 15.9 & 17.3 & .9 & 53.3 \\
\arrayrulecolor{tablerowgray}\hline
Ribs & 20.2 & 13.6 & 20.6 & .7 & 44.9 \\
\arrayrulecolor{tablerowgray}\hline
Chuck ribs & 13.3 & 15. & 20.8 & .8 & 50.1 \\
\arrayrulecolor{tablerowgray}\hline
Tongue & 15.1 & 14.8 & 15.3 & .9 & 53.9 \\
\arrayrulecolor{tablerowgray}\hline
Heart & 16. & 20.4 & 1. & 62.6 &  \\
\arrayrulecolor{tablerowgray}\hline
Kidney & .4 & 16.9 & 4.8 & 1.2 & 76.7 \\
\arrayrulecolor{tablerowgray}\hline
Liver & 1.8 & 21.6 & 5.4 & 1.4 & 69.8 \\
\arrayrulecolor{tablerowgray}\hline
Hind-quarter & 16.7 & 13.5 & 23.5 & .7 & 45.6 \\
\arrayrulecolor{tablerowgray}\hline
Fore-quarter & 21.1 & 11.9 & 25.7 & .7 & 40.6 \\
\arrayrulecolor{tablerowgray}\hline
Leg & 17.4 & 15.1 & 14.5 & .8 & 52.2 \\
\arrayrulecolor{tablerowgray}\hline
Loin & 14.2 & 12.8 & 31.9 & .6 & 40.5 \\
\arrayrulecolor{tablerowgray}\hline
Fore-quarter & 24.5 & 14.6 & 6. & .7 & 54.2 \\
\arrayrulecolor{tablerowgray}\hline
Hind-quarter & 20.7 & 15.7 & 6.6 & .8 & 56.2 \\
\arrayrulecolor{tablerowgray}\hline
Leg & 10.5 & 18.5 & 5. & 1. & 65. \\
\arrayrulecolor{tablerowgray}\hline
Sweetbreads & 15.4 & 12.1 & 1.6 & 70.9 &  \\
\arrayrulecolor{tablerowgray}\hline
Loin of pork & 16. & 13.5 & 27.5 & .7 & 42.3 \\
\arrayrulecolor{tablerowgray}\hline
Ham, smoked & 12.7 & 14.1 & 33.2 & 4.1 & 35.9 \\
\arrayrulecolor{tablerowgray}\hline
Salt pork & 8.1 & 6.5 & 66.8 & 2.7 & 15.9 \\
\arrayrulecolor{tablerowgray}\hline
Bacon & 8.1 & 9.6 & 60.2 & 4.3 & 17.8 \\
\arrayrulecolor{tablerowgray}\hline
Chicken & 34.8 & 14.8 & 1.1 & .8 & 48.5 \\
\arrayrulecolor{tablerowgray}\hline
Fowl & 30. & 13.4 & 10.2 & .8 & 45.6 \\
\arrayrulecolor{tablerowgray}\hline
Turkey & 22.7 & 15.7 & 18.4 & .8 & 42.4 \\
\arrayrulecolor{tablerowgray}\hline
Goose & 22.2 & 10.3 & 33.8 & .6 & 33.1 \\
\arrayrulecolor{black}
\hline
\end{tabular}

                                            \textit{W. O. Atwater, Ph.D.}



\needspace{15\baselineskip}
\section*{Broiled Beefsteak}

The best cuts of beef for broiling are porterhouse, sirloin, cross-cut
of rump steaks, and second and third cuts from top of round. Porterhouse
and sirloin cuts are the most expensive, on account of the great loss in
bone and fat, although price per pound is about the same as for
cross-cut of rump. Round steak is very juicy, but, having coarser fibre,
is not as tender. Steaks should be cut at least an inch thick, and from
that to two and one-half inches. The flank end of sirloin steak should
be removed before cooking. It may be put in soup kettle, or lean part
may be chopped and utilized for meat cakes, fat tried out and clarified
for shortening.

\textbf{To Broil Steak.} Wipe with a cloth wrung out of cold water, and trim
off superfluous fat. With some of the fat grease a wire broiler, place
meat in broiler (having fat edge next to handle), and broil over a clear
fire, turning every ten seconds for the first minute, that surface may
be well seared, thus preventing escape of juices. After the first
minute, turn occasionally until well cooked on both sides. Steak cut one
inch thick will take five minutes, if liked rare; six minutes, if well
done. Remove to hot platter, spread with butter, and sprinkle with salt
and pepper.



\needspace{15\baselineskip}
\section*{Beefsteak With Maître D'Hôtel Butter}

Serve Broiled Steak with Maître d'Hôtel Butter.



\needspace{15\baselineskip}
\section*{Porterhouse Steak With Mushroom Sauce}

Serve broiled Porterhouse Steak with Mushroom Sauce.



\needspace{15\baselineskip}
\section*{Porterhouse Steak With Tomato And Mushroom Sauce}

Serve broiled Porterhouse Steak with Tomato and Mushroom Sauce.



\needspace{15\baselineskip}
\section*{Porterhouse Steak, Bordelaise Sauce}

Serve broiled porterhouse steak with

\textbf{Bordelaise Sauce.} Cook one shallot, finely chopped, with one-fourth
cup claret until claret is reduced to two tablespoons, and strain. Melt
two tablespoons butter, add one slice onion, two slices carrot, sprig of
parsley, bit of bay leaf, eight peppercorns, and one clove, and cook
until brown. Add three and one-half tablespoons flour, and when well
browned add gradually one cup Brown Stock. Strain, let simmer eight
minutes, add claret and one tablespoon butter. Season with salt and
pepper. Remove marrow from a marrow-bone and cut in one-third inch
slices; then poach in boiling water. Arrange on and around steak, and
pour around sauce.



\needspace{15\baselineskip}
\section*{Beefsteak À La Henriette}


\begin{minipage}{1.0\textwidth}
{\setlength{\multicolsep}{0pt}\setlength{\columnsep}{2em}\raggedcolumns%
\begin{multicols}{2}
\begin{itemize}
\setlength{\itemsep}{0pt}
\setlength{\parsep}{0pt}
\item 1/2 cup butter
\item 3 egg yolks
\item 1 tablespoon cold water
\item 1/2 tablespoon lemon juice
\item 1/4 teaspoon salt
\item 2 tablespoons tomato purée
\item 1 tablespoon Worcestershire Sauce
\item 1/2 tablespoon finely chopped parsley
\item Few grains cayenne
\end{itemize}
\end{multicols}}
\end{minipage}

\vspace{0.3em}
\noindent%
Wash butter, and divide in three pieces. Put one piece in saucepan with
yolks of eggs slightly beaten and mixed with water and lemon juice.
Proceed same as in making Hollandaise Sauce I (see p. 274); then add
tomato, parsley, and seasonings. Pour one-half sauce on a serving dish,
lay a broiled porterhouse steak on sauce, and cover steak with remaining
sauce. Garnish with parsley.



\needspace{15\baselineskip}
\section*{Beefsteak À La Victor Hugo}

Wipe a porterhouse steak, broil, and serve with

\textbf{Victor Hugo Sauce.} Cook one-half teaspoon finely chopped shallot in
one tablespoon tarragon vinegar five minutes. Wash one-third cup butter,
and divide in thirds. Add one piece butter to mixture, with yolks two
eggs, one teaspoon lemon juice, and one teaspoon meat extract. Cook over
hot water, stirring constantly; as soon as butter is melted, add second
piece, and then third piece. When mixture thickens, add one-half
tablespoon grated horseradish.



\needspace{15\baselineskip}
\section*{Steak À La Chiron}

Spread broiled rump steak with Hollandaise Sauce I (see p. 274) to which
is added a few drops onion juice and one-half tablespoon finely chopped
parsley.



\needspace{15\baselineskip}
\section*{Beefsteak À La Mirabeau}

Garnish a broiled porterhouse or cross-cut of rump steak with anchovies,
and stoned olives stuffed with green butter and chopped parsley. Arrange
around steak stuffed tomatoes, and fried potato balls served in shells
made from noodle mixture. Pour around the following sauce: Melt two
tablespoons butter, add two and one-half tablespoons browned flour, then
add one cup Chicken Stock. Season with one tablespoon tomato catsup and
salt and pepper.

\textbf{Noodle Shells.} Make noodle mixture (see p. 147), roll as thinly as
possible, cut in pieces, and shape over buttered inverted scallop
shells. Put in dripping-pan and bake in a slow oven. As mixture bakes it
curls from edges, when cases should be slipped from shells and pressed
firmly in insides of shells to finish cooking and leave an impression of
shells. Potato balls served in these shells make an attractive garnish
for broiled fish and meats.



\needspace{15\baselineskip}
\section*{Beefsteak With Oyster Blanket}

Wipe a sirloin steak, cut one and one-half inches thick, broil five
minutes, and remove to platter. Spread with butter and sprinkle with
salt and pepper. Clean one pint oysters, cover steak with same, sprinkle
oysters with salt and pepper and dot over with butter. Place on grate in
hot oven, and cook until oysters are plump.



\needspace{15\baselineskip}
\section*{Planked Beefsteak}

Wipe, remove superfluous fat, and pan broil seven minutes a porterhouse
or cross-cut of the rump steak cut one and three-fourths inches thick.
Butter a plank and arrange a border of Duchess Potatoes close to edge,
using a pastry bag and rose tube. Remove steak to plank, put in a hot
oven, and bake until steak is cooked and potatoes are browned. Spread
steak with butter, sprinkle with salt, pepper, and finely chopped
parsley. Garnish top of steak with sautéd mushroom caps, and put around
steak at equal distances halves of small tomatoes sautéd in butter, and
on top of each tomato a circular slice of cucumber.



\needspace{15\baselineskip}
\section*{Broiled Fillets Of Beef}

Slices cut from the tenderloin are called sliced fillets of beef. Wipe
sliced fillets, place in greased broiler, and broil four or five minutes
over a clear fire. These may be served with Maître d'Hôtel Butter or
Mushroom Sauce.



\needspace{15\baselineskip}
\section*{Cutlets Of Tenderloin With Chestnut Purée}

Shape slices of tenderloin, one inch thick, in circular pieces. Broil
five minutes. Spread with butter, sprinkle with salt and pepper. Arrange
on platter around a mound of Chestnut Purée.



\needspace{15\baselineskip}
\section*{Sautéd Mignon Fillets Of Beef With Sauce Figaro}

Wipe and sauté small fillets in hot omelet pan. Arrange in a circle on
platter with cock's-comb shaped croûtons between, and pour sauce in the
centre. Serve as a luncheon dish with Brussels Sprouts or String Beans.



\needspace{15\baselineskip}
\section*{Sautéd Mignon Fillets Of Beef With Sauce Trianon}

Wipe and sauté small fillets in hot omelet pan. Arrange in a circle
around a mound of fried potato balls sprinkled with parsley. Put Sauce
Trianon on each fillet.



\needspace{15\baselineskip}
\section*{Sautéd Fillets Of Beef À La Moelle}

Cut beef tenderloin in slices one inch thick, and trim into circular
shapes. Season with salt and pepper, and broil six minutes in hot
buttered frying-pan. Remove marrow from a marrow-bone, cut in one-third
inch slices, poach in boiling water, and drain. Put a slice of marrow on
each fillet. To liquor in pan add one tablespoon butter, two tablespoons
flour, and one cup Brown Stock. Season with salt, pepper, and Madeira
wine. Pour sauce around meat.



\needspace{15\baselineskip}
\section*{Sautéd Fillets Of Beef, Cherry Sauce}

Prepare and cook six fillets same as Sautéd Fillets of Beef à la Moelle.
Arrange on serving dish, sprinkle with salt and pepper, spread with
butter, and pour over.

\textbf{Cherry Sauce.} Soak one-fourth cup glacéd cherries fifteen minutes in
boiling water. Drain, cut in halves, cover with Sherry wine, and let
stand three hours.



\needspace{15\baselineskip}
\section*{Sautéd Fillets Of Beef With Stuffed Mushroom Caps}

Prepare and cook six fillets same as Sautéd Fillets of Beef à la Moelle.
Put a sautéd stuffed mushroom cap on each, sprinkle with buttered
crumbs, and bake until crumbs are browned. Remove to serving dish, pour
around Espagnole Sauce, and garnish caps with strips of red and green
pepper cut in fancy shapes.

\textbf{Stuffing for Mushroom Caps.} Clean and finely chop six mushroom caps;
add one tablespoon each of parsley and onion finely chopped, and one
tablespoon butter. Moisten with Espagnole Sauce (See p. 200).



\needspace{15\baselineskip}
\section*{Châteaubriand Of Beef}

Trim off fat and skin from three pounds of beef cut from centre of
fillet and flatten with a broad-bladed cleaver. Sprinkle with salt,
brush over with olive oil, and broil over a clear fire twenty minutes.
Remove to serving dish, garnish with red pepper cut in fancy shapes and
parsley. Serve with

\textbf{Espagnole Sauce.} To one and one-half cups rich brown sauce add
two-thirds teaspoon meat extract, one tablespoon lemon juice, and one
and one-half tablespoons finely chopped parsley. Just before serving add
one tablespoon butter and salt and pepper to taste.



\needspace{15\baselineskip}
\section*{Broiled Meat Cakes}

Chop finely lean raw beef, season with salt and pepper, shape in small
flat cakes, and broil in a greased broiler or frying-pan. Spread with
butter, or serve with Maître d'Hôtel Butter. In forming the cakes,
handle as little as possible; for if pressed too compactly, cakes will
be found solid.



\needspace{15\baselineskip}
\section*{Hamburg Steaks}

Chop finely one pound lean raw beef; season highly with salt, pepper,
and a few drops onion juice or one-half shallot finely chopped. Shape,
cook, and serve as Meat Cakes. A few gratings of nutmeg and one egg
slightly beaten may be added.






\needspace{15\baselineskip}
\section*{Cannelon Of Beef}


\begin{minipage}{1.0\textwidth}
{\setlength{\multicolsep}{0pt}\setlength{\columnsep}{2em}\raggedcolumns%
\begin{multicols}{2}
\begin{itemize}
\setlength{\itemsep}{0pt}
\setlength{\parsep}{0pt}
\item 2 lbs. lean beef, cut from round
\item Grated rind 1/2 lemon
\item 1 tablespoon finely chopped parsley
\item 1 egg
\item 1/2 teaspoon onion juice
\item 2 tablespoons melted butter
\item Few gratings nutmeg
\item 1 teaspoon salt
\item 1/4 teaspoon pepper
\end{itemize}
\end{multicols}}
\end{minipage}

\vspace{0.3em}
\noindent%
Chop meat finely, and add remaining ingredients in order given. Shape in
a roll six inches long, wrap in buttered paper, place on rack in
dripping-pan, and bake thirty minutes. Baste every five minutes with
one-fourth cup butter melted in one cup boiling water. Serve with Brown
Mushroom Sauce I.



\needspace{15\baselineskip}
\section*{Roast Beef}

The best cuts of beef for roasting are: tip or middle of sirloin, back
of rump, or first three ribs. Tip of sirloin roast is desirable for a
small family. Back of rump makes a superior roast for a large family,
and is more economical than sirloin. It is especially desirable where a
large quantity of dish gravy is liked, for in carving the meat juices
follow the knife. Rib roasts contain more fat than either of the others,
and are somewhat cheaper.

\textbf{To Roast Beef.} Wipe, put on a rack in dripping-pan, skin side down,
rub over with salt, and dredge meat and pan with flour. Place in hot
oven, that the surface may be quickly seared, thus preventing escape of
inner juices. After flour in pan is browned, reduce heat, and baste with
fat which has tried out; if meat is quite lean, it may be necessary to
put trimmings of fat in pan. Baste every ten minutes; if this rule is
followed, meat will be found more juicy. When meat is about half done,
turn it over and dredge with flour, that skin side may be uppermost for
final browning. For roasting, consult Time Table for Baking Meats, page
30.

If there is danger of flour burning in pan, add a small quantity of
water; this, however, is not desirable, and seldom need be done if size
of pan is adapted to size of roast. Beef to be well roasted should be
started in hot oven and heat decreased, so that when carved the slices
will be red throughout, with a crisp layer of golden brown fat on the
top. Beef roasted when temperature is so high that surface is hardened
before heat can penetrate to the centre is most unsatisfactory.

Sirloin or rib roasts may have the bones removed, and be rolled,
skewered, and tied in shape. Chicago Butt is cut from the most tender
part of back of rump. They are shipped from Chicago, our greatest beef
centre, and if fresh and from a heavy creature, make excellent roasts at
a small price.

\textbf{Roast Beef Gravy.} Remove some of the fat from pan, leaving four
tablespoons. Place on front of range, add four tablespoons flour, and
stir until well browned. The flour, dredged and browned in pan, should
give additional color to gravy. Add gradually one and one-half cups
boiling water, cook five minutes, season with salt and pepper, and
strain. If flour should burn in pan, gravy will be full of black
particles.

\textbf{To Carve a Roast of Beef.} Have roast placed on platter skin side up;
with a pointed, thin-bladed, sharp knife cut a sirloin or rib roast in
thin slices at right angles to the ribs, and cut slices from ribs. If
there is tenderloin, remove it from under the bone, and cut in thin
slices across grain of meat. Carve back of rump in thin slices with the
grain of meat; by so doing, some of the least tender muscle will be
served with that which is tender. By cutting across grain of meat, the
tenderest portion is sliced by itself, as is the less tender portion.



\needspace{15\baselineskip}
\section*{Yorkshire Pudding}


\begin{itemize}
\setlength{\itemsep}{0pt}
\setlength{\parsep}{0pt}
\item 1 cup milk
\item 1 cup flour
\item 2 eggs
\item 1/4 teaspoon salt
\item \textit{Miss C. J. Wills}
\end{itemize}

\vspace{-0.5em}
\noindent%
Mix salt and flour, and add milk gradually to form a smooth paste; then
add eggs beaten until very light. Cover bottom of hot pan with some of
beef fat tried out from roast, pour mixture in pan one-half inch deep.
Bake twenty minutes in hot oven, basting after well risen, with some of
the fat from pan in which meat is roasting. Cut in squares for serving.
Bake, if preferred, in greased, hissing hot iron gem pans.



\needspace{15\baselineskip}
\section*{Larded Fillet Of Beef}

The tenderloin of beef which lies under the loin and rump is called
fillet of beef. The fillet under the loin is known as the long fillet,
and when removed no porterhouse steaks can be cut; therefore it commands
a higher price than the short fillet lying under rump. Two short fillets
are often skewered together, and served in place of a long fillet.

Wipe, remove fat, veins, and any tendinous portions; skewer in shape,
and lard upper side with grain of meat, following directions for larding
on page 23. Place on a rack in small pan, sprinkle with salt and pepper,
dredge with flour, and put in bottom of pan small pieces of pork. Bake
twenty to thirty minutes in hot oven, basting three times. Take out
skewer, remove meat to hot platter, and garnish with watercress. Serve
with Mushroom, Figaro, or Horseradish Sauce I.



\needspace{15\baselineskip}
\section*{Fillet Of Beef With Vegetables}

Wipe a three-pound fillet, trim, and remove fat. Put one-half pound
butter in hot frying-pan and when melted add fillet, and turn frequently
until the entire surface is seared and well browned; then turn
occasionally until done, the time required being about thirty minutes.
Remove to serving dish and garnish with one cup each cooked peas and
carrots cut in fancy shapes, both well seasoned, one-half cup raisins
seeded and cooked in boiling water until soft, and the caps from
one-half pound fresh mushrooms sautéd in butter five minutes. Serve with

\textbf{Brown Mushroom Sauce.} Pour off one-fourth cup fat from frying-pan, add
five tablespoons flour, and stir until well browned; then add one cup
Brown Soup Stock, one-third cup mushroom liquor, and the caps from
one-half pound mushrooms cut in slices and sautéd in butter three
minutes. Season with salt and pepper, and just before serving add
gradually, while stirring constantly, the butter remaining in
frying-pan.

To obtain mushroom liquor, scrape stems of mushrooms, break in pieces,
cover with cold water, and cook slowly until liquid is reduced to
one-third cup.



\needspace{15\baselineskip}
\section*{Braised Beef}


\begin{minipage}{1.0\textwidth}
{\setlength{\multicolsep}{0pt}\setlength{\columnsep}{2em}\raggedcolumns%
\begin{multicols}{2}
\begin{itemize}
\setlength{\itemsep}{0pt}
\setlength{\parsep}{0pt}
\item 3 lbs. beef from lower part of round or face of rump
\item 2 thin slices fat salt pork
\item 1/2 teaspoon peppercorns
\item 1/4 cup carrot
\item 1/4 cup turnip
\item 1/4 cup onion
\item 1/4 cup celery
\item Salt and pepper
\end{itemize}
\end{multicols}}
\end{minipage}

\vspace{0.3em}
\noindent%
Fry out pork and remove scraps. Wipe meat, sprinkle with salt and
pepper, dredge with flour, and brown entire surface in pork fat. When
turning meat, avoid piercing with fork or skewer, which allows the inner
juices to escape. Place on trivet in deep granite pan or in earthen
pudding-dish, and surround with vegetables, peppercorns, and three cups
boiling water; cover closely, and bake four hours in very slow oven,
basting every half-hour, and turning after second hour. Throughout the
cooking, the liquid should be kept below the boiling-point. Serve with
Horseradish Sauce, or with sauce made from liquor in pan.



\needspace{15\baselineskip}
\section*{Beef À La Mode}

Insert twelve large lardoons in a four-pound piece of beef cut from the
round. Make incisions for lardoons by running through the meat a large
skewer. Season with salt and pepper, dredge with flour, and brown the
entire surface in pork fat. Put on a trivet in kettle, surround with
one-third cup each carrot, turnip, celery, and onion cut in dice, sprig
of parsley, bit of bay leaf, and water to half cover meat. Cover
closely, and cook slowly four hours, keeping liquor below the
boiling-point. Remove to hot platter. Strain liquor, thicken and season
to serve as a gravy. When beef is similarly prepared (with exception of
lardoons and vegetables), and cooked in smaller amount of water, it is
called Smothered Beef, or Pot Roast. A bean-pot (covered with a piece of
buttered paper, tied firmly down) is the best utensil to use for a Pot
Roast.



\needspace{15\baselineskip}
\section*{Pressed Beef Flank}

Wipe, remove superfluous fat, and roll a flank of beef. Put in a kettle,
cover with boiling water, and add one tablespoon salt, one-half teaspoon
peppercorns, a bit of bay leaf, and a bone or two which may be at hand.
Cook slowly until meat is in shreds; there should be but little liquor
in kettle when meat is done. Arrange meat in a deep pan, pour over
liquor, cover, and press with a heavy weight. Serve cold, thinly sliced.



\needspace{15\baselineskip}
\section*{Beef Stew With Dumplings}

               Aitchbone, weighing 5 lbs.

\begin{minipage}{1.0\textwidth}
{\setlength{\multicolsep}{0pt}\setlength{\columnsep}{2em}\raggedcolumns%
\begin{multicols}{2}
\begin{itemize}
\setlength{\itemsep}{0pt}
\setlength{\parsep}{0pt}
\item 4 cups potatoes, cut in 1/4 inch slices
\item 2/3 cup turnip
\item 2/3 cup carrot
\item 1/2 small onion, cut in thin slices
\item 1/4 cup flour
\item Salt
\item Pepper
\end{itemize}
\end{multicols}}
\end{minipage}

\vspace{0.3em}
\noindent%
Wipe meat, remove from bone, cut in one and one-half inch cubes,
sprinkle with salt and pepper, and dredge with flour. Cut some of the
fat in small pieces and try out in frying-pan. Add meat and stir
constantly, that the surface may be quickly seared; when well browned,
put in kettle, and rinse frying-pan with boiling water, that none of the
goodness may be lost. Add to meat remaining fat, and bone sawed in
pieces; cover with boiling water and boil five minutes, then cook at a
lower temperature until meat is tender (time required being about three
hours). Add carrot, turnip, and onion, with salt and pepper the last
hour of cooking. Parboil potatoes five minutes, and add to stew fifteen
minutes before taking from fire. Remove bones, large pieces of fat, and
then skim. Thicken with one-fourth cup flour, diluted with enough cold
water to pour easily. Pour in deep hot platter, and surround with
dumplings. Remnants of roast beef are usually made into a beef stew; the
meat having been once cooked, there is no necessity of browning it. If
gravy is left, it should be added to the stew.



\needspace{15\baselineskip}
\section*{Dumplings}


\begin{itemize}
\setlength{\itemsep}{0pt}
\setlength{\parsep}{0pt}
\item 2 cups flour
\item 4 teaspoons baking powder
\item 1/2 teaspoon salt
\item 2 teaspoons butter
\item 3/4 cup milk
\end{itemize}

\vspace{-0.5em}
\noindent%
Mix and sift dry ingredients. Work in butter with tips of fingers, and
add milk gradually, using a knife for mixing. Toss on a floured board,
pat, and roll out to one-half inch in thickness. Shape with
biscuit-cutter, first dipped in flour. Place closely together in a
buttered steamer, put over kettle of boiling water, cover closely, and
steam twelve minutes. A perforated tin pie plate may be used in place of
steamer. A little more milk may be used in the mixture, when it may be
taken up by spoonfuls, dropped and cooked on top of stew. In this case
some of the liquid must be removed, that dumplings may rest on meat and
potato, and not settle into liquid.



\needspace{15\baselineskip}
\section*{Corned Beef}

Corned beef has but little nutritive value. It is used to give variety
to our diet in summer, when fresh meats prove too stimulating. It is
eaten by the workingman to give bulk to his food. The best pieces of
corned beef are the rattle rand and fancy brisket. The fancy brisket
commands a higher price and may be easily told from the rattle rand by
the selvage on lower side and the absence of bones. The upper end of
brisket (butt end) is thick and composed mostly of lean meat, the middle
cut has more fat but is not well mixed, while the lower (navel end) has
a large quantity of fat. The rattle rand contains a thick lean end; the
second cut contains three distinct layers of meat and fat, and is
considered the best cut by those who prefer meat well streaked with fat.
The rattle rand has a thin end, which contains but one layer of lean
meat and much fat, consequently is not a desirable piece.

\textbf{To Boil Corned Beef.} Wipe the meat and tie securely in shape, if this
has not been already done at market. Put in kettle, cover with cold
water, and bring slowly to boiling-point. Boil five minutes, remove
scum, and cook at a lower temperature until tender. Cool slightly in
water in which it was cooked, remove to a dish, cover, and place on
cover a weight, that meat may be well pressed. The lean meat and fat may
be separated and put in alternate layers in a bread pan, then covered
and pressed.



\needspace{15\baselineskip}
\section*{Boiled Dinner}

A boiled dinner consists of warm impressed corned beef, served with
cabbage, beets, turnips, carrots, and potatoes. After removing meat from
water, skim off fat and cook vegetables (with exception of beets, which
require a long time for cooking) in this water. Carrots require a longer
time for cooking than cabbage or turnips. Carrots and turnips, if small,
may be cooked whole; if large, cut in pieces. Cabbage and beets are
served in separate dishes, other vegetables on same dish with meat.



\needspace{15\baselineskip}
\section*{Boiled Tongue}

A boiled corned tongue is cooked the same as Boiled Corned Beef. If very
salt, it should be soaked in cold water several hours, or over night,
before cooking. Take from water when slightly cooled and remove skin.



\needspace{15\baselineskip}
\section*{Braised Tongue}

A fresh tongue is necessary for braising. Put tongue in kettle, cover
with boiling water, and cook slowly two hours. Take tongue from water
and remove skin and roots. Place in deep pan and surround with one-third
cup each carrot, onion, and celery, cut in dice, and one sprig parsley;
then pour over four cups sauce. Cover closely, and bake two hours,
turning after the first hour. Serve on platter and strain around the
sauce.

\textbf{Sauce for Tongue.} Brown one-fourth cup butter, add one-fourth cup
flour and stir together until well browned. Add gradually four cups of
water in which tongue was cooked. Season with salt and pepper and add
one teaspoon Worcestershire Sauce. One and one-half cups stewed and
strained tomatoes may be used in place of some of the water.



\needspace{15\baselineskip}
\section*{Broiled Liver}

Cover with boiling water slices of liver cut one-half inch thick, let
stand five minutes to draw out the blood; drain, wipe, and remove the
thin outside skin and veins. Sprinkle with salt and pepper, place in a
greased wire broiler and broil five minutes, turning often. Remove to a
hot platter, spread with butter, and sprinkle with salt and pepper.



\needspace{15\baselineskip}
\section*{Liver And Bacon}

Prepare as for Broiled Liver, cut in pieces for serving, sprinkle with
salt and pepper, dredge with flour, and fry in bacon fat. Serve with
bacon.



\needspace{15\baselineskip}
\section*{Bacon I}

Place strips of thinly cut bacon on board, and with a broad-bladed knife
make strips as thin as possible. Put in hot frying-pan and cook until
bacon is crisp and brown, occasionally pouring off fat from pan, turning
frequently. Drain on brown paper.



\needspace{15\baselineskip}
\section*{Bacon II}

Place thin slices of bacon (from which the rind has been removed)
closely together in a fine wire broiler; place broiler over dripping-pan
and bake in a hot oven until bacon is crisp and brown, turning once.
Drain on brown paper. Fat which has dripped into the pan should be
poured out and used for frying liver, eggs, potatoes, etc.



\needspace{15\baselineskip}
\section*{Braised Liver}

Skewer, tie in shape, and lard upper side of calf's liver. Place in deep
pan, with trimmings from lardoons; surround with one-fourth cup each,
carrot, onion, and celery, cut in dice; one-fourth teaspoon peppercorns,
two cloves, bit of bay leaf, and two cups Brown Stock or water. Cover
closely and bake slowly two hours, uncovering the last twenty minutes.
Remove from pan, strain liquor, and use liquor for the making of a brown
sauce with one and one-half tablespoons butter and two tablespoons
flour. Pour sauce around liver for serving.



\needspace{15\baselineskip}
\section*{Calf's Liver, Stuffed And Larded}

Make a deep cut nearly the entire length of liver, beginning at thick
end, thus making a pouch for stuffing. Fill pouch. Skewer liver and lard
upper side. Put liver in baking-pan, pour around two cups Brown Sauce,
made of one tablespoon each butter and flour, and two cups Brown Stock,
salt, and pepper. Bake one and one-fourth hours, basting every twelve
minutes with sauce in pan. Remove to serving dish, strain sauce around
liver, and garnish with Glazed or French Fried Onions (see p. 296).

\textbf{Stuffing.} Mix one-half pound chopped cooked cold ham, one-half cup
stale bread crumbs, one half small onion finely chopped, and one
tablespoon finely chopped parsley. Moisten with Brown Sauce; then add
one beaten egg, and season with salt and pepper.



\needspace{15\baselineskip}
\section*{Broiled Tripe}

Fresh honeycomb tripe is best for broiling. Wipe tripe as dry as
possible, dip in fine cracker dust and olive oil or melted butter,
draining off all fat that is possible, and again dip in cracker dust.
Place in a greased broiler and broil five minutes, cooking smooth side
of tripe the first three minutes. Place on a hot platter, honeycomb side
up, spread with butter, and sprinkle with salt and pepper. Broiled tripe
is at its best when cooked over a charcoal fire.



\needspace{15\baselineskip}
\section*{Tripe In Batter}

Wipe tripe and cut in pieces for serving. Sprinkle with salt and pepper,
dip in batter, fry in a small quantity of hot fat, and drain.

\textbf{Tripe Batter.} Mix one cup flour with one-fourth teaspoon salt; add
gradually one-half cup cold water, and when perfectly smooth add one egg
well beaten, one-half tablespoon vinegar, and one teaspoon olive oil or
melted butter.



\needspace{15\baselineskip}
\section*{Tripe Fried In Batter}

Cut pickled honeycomb tripe in pieces for serving; wash, cover with
boiling water, and simmer gently twenty minutes. Drain, and again cover,
using equal parts cold water and milk. Heat to boiling-point, again
drain, wipe as dry as possible, sprinkle with salt and pepper, brush
over with melted butter, dip in batter, fry in deep fat, and drain on
brown paper. Serve with slices of lemon and Chili Sauce.

\textbf{Batter.} Mix and sift one cup flour, one and one-half teaspoons baking
powder, one-fourth teaspoon salt, and a few grains pepper. Add one-third
cup milk and one egg well beaten.



\needspace{15\baselineskip}
\section*{Lyonnaise Tripe}

Cut honeycomb tripe in pieces two inches long by one-half inch wide,
having three cupfuls. Put in a pan and place in oven that water may be
drawn out. Cook one tablespoon finely chopped onion in two tablespoons
butter until slightly browned, add tripe drained from water, and cook
five minutes. Sprinkle with salt and pepper and finely chopped parsley.



\needspace{15\baselineskip}
\section*{Tripe À La Creole}

Cut, bake, and drain tripe as for Lyonnaise Tripe. Cook same quantity of
butter and onion, add one-eighth green pepper finely chopped, one
tablespoon flour, one-half cup stock, one-fourth cup drained tomatoes,
and one fresh mushroom cut in slices; then add tripe and cook five
minutes. Season with salt and pepper.



\needspace{15\baselineskip}
\section*{Tripe À La Provençale}

Add to Lyonnaise Tripe one tablespoon white wine. Cook until quite dry,
add one-third cup Tomato Sauce, cook two minutes, season with salt and
pepper, and serve.



\needspace{15\baselineskip}
\section*{Calf's Head À La Terrapin}

Wash and clean a calf's head, and cook until tender in boiling water to
cover. Cool, and cut meat from cheek in small cubes. To two cups meat
dice add one cup sauce made of two tablespoons butter, two tablespoons
flour, and one cup White Stock, seasoned with one-half teaspoon salt,
one-eighth teaspoon pepper, and a few grains cayenne. Add one-half cup
cream and yolks of two eggs slightly beaten; cook two minutes and add
two tablespoons Madeira wine.



\needspace{15\baselineskip}
\section*{Calves' Tongues}

Cook tongues until tender in boiling water to cover, with six slices
carrot, two stalks celery, one onion stuck with six cloves, one-half
teaspoon peppercorns and one-half tablespoon salt; take from water and
remove skin and roots. Split and pour over equal parts brown stock and
tomatoes boiled until thick.



\needspace{15\baselineskip}
\section*{Calves' Tongues, Sauce Piquante}

Cook four tongues, until tender, in boiling water, to cover, with six
slices carrot, two stalks celery, one onion stuck with eight cloves, one
teaspoon peppercorns, and one-half tablespoon salt. Take tongues from
water, and remove skin and roots. Cut in halves lengthwise and reheat in

\textbf{Sauce Piquante.} Brown one-fourth cup butter, add six tablespoons
flour, and stir until well browned; then add two cups Brown Stock and
cook three minutes. Season with two-thirds teaspoon salt, one-half
teaspoon paprika, few grains of cayenne, one tablespoon vinegar,
one-half tablespoon capers, and one cucumber pickle thinly sliced.
Served garnished with cucumber pickles, and cold cooked beets cut in
fancy shapes.



\needspace{15\baselineskip}
\section*{Calf's Heart}

Wash a calf's heart, remove veins, arteries, and clotted blood. Stuff
(using half quantity of Fish Stuffing I on page 164, seasoned highly
with sage) and sew. Sprinkle with salt and pepper, roll in flour, and
brown in hot fat. Place in small, deep baking-pan, half cover it with
boiling water, cover closely, and bake slowly two hours, basting every
fifteen minutes. It may be necessary to add more water. Remove heart
from pan, and thicken the liquor with flour diluted with a small
quantity of cold water. Season with salt and pepper, and pour around the
heart before serving.



\needspace{15\baselineskip}
\section*{Stuffed Hearts With Vegetables}

Clean and wash calves' hearts, stuff, skewer into shape, lard, season
with salt and pepper, dredge with flour, and sauté in pork fat, adding
to fat one stalk celery, one tablespoon chopped onion, two sprigs
parsley, four slices carrot cut in pieces, half the quantity of turnip,
a bit of bay leaf, two cloves, and one-fourth teaspoon peppercorns. Turn
hearts occasionally until well browned, then add one and one-half cups
Brown Stock, cover, and cook slowly one and one-half hours. Serve with
cooked carrots and turnips cut in strips or fancy shapes.



\needspace{15\baselineskip}
\section*{Braised Ox Joints}

Cut ox-tail at joints, parboil five minutes, wash thoroughly, dredge
with flour, and sauté in butter (to which has been added a sliced onion)
until well browned. Add one-fourth cup flour, two cups each brown stock,
water, and canned tomatoes, one teaspoon salt, and one-fourth teaspoon
pepper. Turn into an earthen pudding-dish, cover, and cook slowly three
and one-half hours. Remove ox-tail, strain sauce, and return ox-tail and
sauce to oven to finish cooking. Add two-thirds cup each carrot and
turnip (shaped with a vegetable cutter in pieces one-inch long, and
about as large around as macaroni) parboiled in boiled salted water five
minutes. As soon as vegetables are soft, add Sherry wine to taste, and
more salt and pepper, if needed. The wine may be omitted.



\needspace{15\baselineskip}
\section*{Ways Of Warming Over Beef}


\needspace{15\baselineskip}
\subsection*{Roast Beef with Gravy}

Cut cold roast beef in thin slices, place on a warm platter, and pour
over some of the gravy reheated to the boiling-point. If meat is allowed
to stand in gravy on the range, it becomes hard and tough.



\needspace{15\baselineskip}
\subsection*{Roast Beef, Mexican Sauce}

Reheat cold roast beef cut in thin slices, in

\textbf{Mexican Sauce.} Cook one onion, finely chopped, in two tablespoons
butter five minutes. Add one red pepper, one green pepper, and one clove
of garlic, each finely chopped, and two tomatoes peeled and cut in
pieces. Cook fifteen minutes, add one teaspoon Worcestershire Sauce,
one-fourth teaspoon celery salt, and salt to taste.



\needspace{15\baselineskip}
\subsection*{Cottage Pie}

Cover bottom of a small greased baking-dish with hot mashed potato, add
a thick layer of roast beef, chopped or cut in small pieces (seasoned
with salt, pepper, and a few drops onion juice) and moistened with some
of the gravy; cover with a thin layer of mashed potato, and bake in a
hot oven long enough to heat through.



\needspace{15\baselineskip}
\subsection*{Beefsteak Pie}

Cut remnants of cold broiled steak or roast beef in one-inch cubes.
Cover with boiling water, add one-half onion, and cook slowly one hour.
Remove onion, thicken gravy with flour diluted with cold water, and
season with salt and pepper. Add potatoes cut in one-fourth inch slices,
which have been parboiled eight minutes in boiling salted water. Put in
a buttered pudding-dish, cool, cover with baking-powder biscuit mixture
or pie-crust. Bake in a hot oven. If covered with pie crust, make
several incisions in crust that gases may escape.



\needspace{15\baselineskip}
\subsection*{Cecils with Tomato Sauce}


\begin{minipage}{1.0\textwidth}
{\setlength{\multicolsep}{0pt}\setlength{\columnsep}{2em}\raggedcolumns%
\begin{multicols}{2}
\begin{itemize}
\setlength{\itemsep}{0pt}
\setlength{\parsep}{0pt}
\item 1 cup cold roast beef or rare   steak finely chopped
\item Salt
\item Pepper
\item Onion juice
\item Worcestershire Sauce
\item 2 tablespoons bread crumbs
\item 1 tablespoon melted butter
\item Yolk 1 egg slightly beaten
\end{itemize}
\end{multicols}}
\end{minipage}

\vspace{0.3em}
\noindent%
Season beef with salt, pepper, onion juice, and Worcestershire Sauce;
add remaining ingredients, shape after the form of small croquettes,
pointed at ends. Roll in flour, egg, and crumbs, fry in deep fat, drain,
and serve with Tomato Sauce.



\needspace{15\baselineskip}
\subsection*{Corned Beef Hash}

Remove skin and gristle from cooked corned beef, then chop the meat.
When meat is very fat, discard most of the fat. To chopped meat add an
equal quantity of cold boiled chopped potatoes. Season with salt and
pepper, put into a hot buttered frying-pan, moisten with milk or cream,
stir until well mixed, spread evenly, then place on a part of the range
where it may slowly brown underneath. Turn, and fold on a hot platter.
Garnish with sprig of parsley in the middle.



\needspace{15\baselineskip}
\subsection*{Corned Beef Hash with Beets}

When preparing Corned Beef hash, add one-half as much finely chopped
cooked beets as potatoes. Cold roast beef or one-half roast beef and
one-half corned beef may be used.



\needspace{15\baselineskip}
\subsection*{Dried Beef with Cream}


\begin{itemize}
\setlength{\itemsep}{0pt}
\setlength{\parsep}{0pt}
\item 1/4 lb. smoked dried beef, thinly sliced
\item 1 cup scalded cream
\item 1 1/2 tablespoons flour
\end{itemize}

\vspace{-0.5em}
\noindent%
Remove skin and separate meat in pieces, cover with hot water, let stand
ten minutes, and drain. Dilute flour with enough cold water to pour
easily, making a smooth paste; add to cream, and cook in double boiler
ten minutes. Add beef, and reheat. One cup White Sauce I may be used in
place of cream, omitting the salt.





\chapter{Lamb And Mutton}



Lamb is the name given to the meat of lambs; mutton, to the meat of
sheep. Lamb, coming as it does from the young creature, is immature, and
less nutritious than mutton. The flesh of mutton ranks with the flesh of
beef in nutritive value and digestibility. The fat of mutton, on account
of its larger percentage of stearic acid, is more difficult of digestion
than the fat of beef.

Lamb may be eaten soon after the animal is killed and dressed; mutton
must hang to ripen. Good mutton comes from a sheep about three years
old, and should hang from two to three weeks. The English South Down
Mutton is cut from creatures even older than three years. Young lamb,
when killed from six weeks to three months old, is called spring lamb,
and appears in the market as early as the last of January, but is very
scarce until March. Lamb one year old is called a yearling. Many object
to the strong flavor of mutton; this is greatly overcome by removing the
pink skin and trimming off superfluous fat.

Lamb and mutton are divided into two parts by cutting through entire
length of backbone; then subdivided into fore and hind quarter, eight
ribs being left on hind quarter,--while in beef but three ribs are left
on hind-quarter. These eight ribs are cut into chops and are known as
\textit{rib chops}. The meat which lies between these ribs and the leg, cut
into chops, is known as \textit{loin} or \textit{kidney chops}.

Lamb and mutton chops cut from loin have a small piece of tenderloin on
one side of bone, and correspond to porterhouse steaks in the beef
creature. Rib chops which have the bone cut short and scraped clean,
nearly to the lean meat, are called \textit{French chops}.

\textit{The leg} is sold whole for boiling or roasting. The fore-quarter may be
boned, stuffed, rolled, and roasted, but is more often used for broth,
stew, or fricassee.

For \textit{a saddle of mutton} the loin is removed whole before splitting the
creature. Some of the bones are removed and the flank ends are rolled,
fastened with wooden skewers, and securely tied to keep skewers in
place.

Good quality mutton should be fine-grained and of bright pink color; the
fat white, hard, and flaky. If the outside skin comes off easily, mutton
is sure to be good. Lamb chops may be easily distinguished from mutton
chops by the red color of bone. As lamb grows older, blood recedes from
bones; therefore in mutton the bone is white. In \textit{leg of lamb} the bone
at joint is serrated, while in leg of mutton the bone at joint is smooth
and rounded. Good mutton contains a larger proportion of fat than good
beef. Poor mutton is often told by the relatively small proportion of
fat and lean as compared to bone.

Lamb is usually preferred well done; mutton is often cooked rare.



\needspace{15\baselineskip}
\section*{Broiled Lamb Or Mutton Chops}

Wipe chops, remove superfluous fat, and place in a broiler greased with
some of mutton fat. In loin chops, flank may be rolled and fastened with
a small wooden skewer. Follow directions for Broiling Beefsteak on page
196.



\needspace{15\baselineskip}
\section*{Pan-Broiled Chops}

Chops for pan broiling should have flank and most of fat removed. Wipe
chops and put in hissing hot frying-pan.

Turn as soon as under surface is seared, and sear other side. Turn
often, using knife and fork that the surface may not be pierced, as
would be liable if fork alone were used. Cook six minutes if liked rare,
eight to ten minutes if liked well done. Let stand around edge of
frying-pan to brown the outside fat. When half cooked, sprinkle with
salt. Drain on brown paper, put on hot platter, and spread with butter
or serve with Tomato or Soubise Sauce.



\needspace{15\baselineskip}
\section*{Breaded Mutton Chops}

Wipe and trim chops, sprinkle with salt and pepper, dip in crumbs, egg,
and crumbs, fry in deep fat from five to eight minutes, and drain. Serve
with Tomato Sauce, or stack around a mound of mashed potatoes, fried
potato balls, or green peas. Never fry but four at a time, and allow fat
to reheat between fryings. After testing fat for temperature, put in
chops and place kettle on back of range, that surface of chops may not
be too brown while the inside is still underdone.



\needspace{15\baselineskip}
\section*{Chops À La Signora}

Gash French Chops on outer edge, extending cut half-way through lean
meat. Insert in each gash a slice of truffle, sprinkle with salt and
pepper, wrap in calf's caul. Roll in flour, dip in egg, then in stale
bread crumbs, and sauté in butter eight minutes, turning often. Place in
oven four minutes to finish cooking. Arrange on hot platter for serving,
and place on top of each a fresh broiled mushroom or mushroom baked in
cream. To fat in pan add a small quantity of boiling water and pour
around chops. This is a delicious way of cooking chops for a dinner
party.



\needspace{15\baselineskip}
\section*{Lamb Chops À La Marseilles}

Pan broil, on one side, six French chops, cover cooked side with
Mushroom Sauce, place in a buttered baking-dish, and bake in a hot oven
eight minutes. Remove to serving dish, place a paper frill on each chop,
and garnish with parsley.

\textbf{Mushroom Sauce.} Brown one and one-half tablespoons butter, add three
tablespoons flour, and stir until well browned; then add one-half cup
highly seasoned Brown Stock. Add one-fourth cup chopped canned
mushrooms, and season with salt and pepper.



\needspace{15\baselineskip}
\section*{Chops À La Castillane}

Broil six lamb chops, arrange on slices of fried eggplant, and pour
around the following sauce: Brown three tablespoons butter, add three
and one-half tablespoons flour, and stir until well browned; then add,
gradually, one cup rich Brown Stock. Cook three tablespoons lean raw ham
cut in small cubes in one-half tablespoon butter two minutes. Moisten
with two tablespoons Sherry wine, and add to sauce with two tablespoons
finely shredded green pepper. Season with salt and pepper.



\needspace{15\baselineskip}
\section*{Chops En Papillote}

Finely chop the whites of three “hard-boiled” eggs and force yolks
through potato ricer, mix, and add to three common crackers, rolled and
sifted; then add three tablespoons melted butter, salt, pepper, and
onion juice, to taste. Add enough cream to make of right consistency to
spread. Cover chops thinly with mixture and wrap in buttered paper
cases. Bake twenty-five minutes in hot oven. Remove from cases, place on
hot platter, and garnish with parsley.



\needspace{15\baselineskip}
\section*{Mutton Cutlets À La Maintenon}

Wipe six French Chops, cut one and one-half inches thick. Split meat in
halves, cutting to bone. Cook two and one-half tablespoons butter and
one tablespoon onion five minutes; remove onion, add one-half cup
chopped mushrooms, and cook five minutes; then add two tablespoons
flour, three tablespoons stock, one teaspoon finely chopped parsley,
one-fourth teaspoon salt, and a few grains cayenne. Spread mixture
between layers of chops, press together lightly, wrap in buttered paper
cases, and broil ten minutes. Serve with Spanish Sauce.



\needspace{15\baselineskip}
\section*{Boiled Leg Of Mutton}

Wipe meat, place in a kettle, and cover with boiling water. Bring
quickly to boiling-point, boil five minutes, and skim. Set on back of
range and simmer until meat is tender. When half done, add one
tablespoon salt. Serve with Caper Sauce, or add to two cups White Sauce
(made of one-half milk and one-half Mutton Stock), two “hard-boiled”
eggs cut in slices.



\needspace{15\baselineskip}
\section*{Braised Leg Of Mutton}

Order a leg of mutton boned. Wipe, stuff, sew, and place in deep pan.
Cook five minutes in one-fourth cup butter, a slice each of onion,
carrot, and turnip cut in dice, one-half bay leaf, and a sprig each of
thyme and parsley. Add three cups hot water, one and one-half teaspoons
salt, and twelve peppercorns; pour over mutton. Cover closely, and cook
slowly three hours, uncovering for the last half-hour. Remove from pan
to hot platter. Brown three tablespoons butter, add four tablespoons
flour, and stir until well browned; then pour on slowly the strained
liquor; there should be one and three-fourths cups.



\needspace{15\baselineskip}
\section*{Stuffing}


\begin{minipage}{1.0\textwidth}
{\setlength{\multicolsep}{0pt}\setlength{\columnsep}{2em}\raggedcolumns%
\begin{multicols}{2}
\begin{itemize}
\setlength{\itemsep}{0pt}
\setlength{\parsep}{0pt}
\item 1 cup cracker crumbs
\item 1/4 cup melted butter
\item 1/4 teaspoon salt
\item 1/8 teaspoon pepper
\item 1/2 tablespoon Poultry Seasoning
\item 1/4 cup boiling water
\end{itemize}
\end{multicols}}
\end{minipage}

\vspace{0.3em}
\noindent%
                               Roast Lamb

A leg of lamb is usually sent from market wrapped in caul; remove caul,
wipe meat, sprinkle with salt and pepper, place on rack in dripping-pan,
and dredge meat and bottom of pan with flour. Place in hot oven, and
baste as soon as flour in pan is brown, and every fifteen minutes
afterwards until meat is done, which will take about one and
three-fourths hours. It may be necessary to put a small quantity of
water in pan while meat is cooking. Leg of lamb may be boned and stuffed
for roasting. See Stuffing, under Braised Mutton.

Make gravy, following directions for Roast Beef Gravy on page 202, or
serve with Currant Jelly Sauce.

\textbf{To Carve a Leg of Lamb.} Cut in thin slices across grain of meat to the
bone, beginning at top of the leg.



\needspace{15\baselineskip}
\section*{Lamb Bretonne}

Serve hot thinly sliced roast lamb with

\textbf{Beans Bretonne.} Soak one and one-half cups pea beans over night in
cold water to cover, drain, and parboil until soft; again drain, put in
earthen-ware dish or bean-pot, add tomato sauce, cover, and cook until
beans have nearly absorbed sauce.





\textbf{Tomato Sauce.} Mix one cup stewed and strained tomatoes, one cup white
stock, six canned pimentoes rubbed through a sieve, one onion finely
chopped, two cloves garlic finely chopped, one-fourth cup butter, and
two teaspoons salt.



\needspace{15\baselineskip}
\section*{Saddle Of Mutton}

Mutton for a saddle should always be dressed at market. Wipe meat,
sprinkle with salt and pepper, place on rack in dripping-pan, and dredge
meat and bottom of pan with flour. Bake in hot oven one and one-fourth
hours, basting every fifteen minutes. Serve with Currant Jelly Sauce.

\textbf{To Carve a Saddle of Mutton}, cut thin slices parallel with backbone,
then slip the knife under and separate slices from ribs.



\needspace{15\baselineskip}
\section*{Saddle Of Mutton, Currant Mint Sauce}

Follow directions for Saddle of Mutton, and serve with

\textbf{Currant Mint Sauce.} Separate two-thirds tumbler of currant jelly in
pieces, but do not beat it. Add one and one-half tablespoons finely
chopped mint leaves and shavings from the rind of one-fourth orange.



\needspace{15\baselineskip}
\section*{Saddle Of Lamb À L'Estragnon}

Wipe meat, sprinkle with salt and pepper, place on rack in dripping-pan,
and dredge meat and bottom of pan with flour. Bake in hot oven one and
one-fourth hours, basting every fifteen minutes. Remove to hot serving
dish and pour around

\textbf{Estragnon Sauce.} Brown four tablespoons butter, add four tablespoons
flour (which has been previously browned), and pour on gradually, while
stirring constantly, two cups bouillon, and one-half cup stock which has
infused with one tablespoon tarragon one hour.



\needspace{15\baselineskip}
\section*{Crown Of Lamb}

Select parts from two loins containing ribs, scrape flesh from bone
between ribs, as far as lean meat, and trim off backbone. Shape each
piece in a semicircle, having ribs outside, and sew pieces together to
form a crown. Trim ends of bones evenly, care being taken that they are
not left too long, and wrap each bone in a thin strip of fat salt pork
or insert in cubes of fat salt pork to prevent bone from burning; then
cover with buttered paper. Roast one and one-fourth hours.

Remove pork from bones before serving, and fill centre with Purée of
Chestnuts.



\needspace{15\baselineskip}
\section*{Lamb En Casserole}

Wipe two slices of lamb cut one and one-fourth inches thick from centre
of leg. Put in hot frying-pan, and turn frequently until seared and
browned on both sides. Brush over with melted butter, season with salt
and pepper, and bake in casserole dish twenty minutes or until tender.
Parboil three-fourths cup carrot, cut in strips, fifteen minutes; drain,
and sauté in one tablespoon bacon fat to which has been added one
tablespoon finely chopped onion. Add to lamb, with one cup potato balls,
two cups thin Brown Sauce, three tablespoons Sherry wine, and pepper to
taste. Cook until potatoes are soft, then add twelve small onions cooked
until soft, then drained and sautéd in butter to which is added one
tablespoon sugar. Onions need not be sautéd unless they are desired
glazed. Serve from casserole dish.



\needspace{15\baselineskip}
\section*{Mutton Curry}

Wipe and cut meat from fore-quarter of mutton in one-inch pieces; there
should be three cupfuls. Put in kettle, cover with cold water, and bring
quickly to boiling-point; drain in colander and pour over one quart cold
water. Return meat to kettle, cover with one quart boiling water, add
three onions cut in slices, one-half teaspoon peppercorns, and a sprig
each of thyme and parsley. Simmer until meat is tender, remove meat,
strain liquor, and thicken with one-fourth cup each of butter and flour
cooked together; to the flour add one-half teaspoon curry powder,
one-half teaspoon salt, and one-eighth teaspoon pepper. Add meat to
gravy, reheat, and serve with border of steamed rice.



\needspace{15\baselineskip}
\section*{Fricassee Of Lamb With Brown Gravy}

Order three pounds lamb from the fore-quarter, cut in pieces for
serving. Wipe meat, put in kettle, cover with boiling water, and cook
slowly until meat is tender. Remove from water, cool, sprinkle with salt
and pepper, dredge with flour, and sauté in butter or mutton fat.
Arrange on platter, and pour around one and one-half cups Brown Sauce
made from liquor in which meat was cooked after removing all fat. It is
better to cook meat the day before serving, as then fat may be more
easily removed.



\needspace{15\baselineskip}
\section*{Mutton Broth}


\begin{minipage}{1.0\textwidth}
{\setlength{\multicolsep}{0pt}\setlength{\columnsep}{2em}\raggedcolumns%
\begin{multicols}{2}
\begin{itemize}
\setlength{\itemsep}{0pt}
\setlength{\parsep}{0pt}
\item 3 lbs. mutton (from the neck)
\item 2 quarts cold water
\item 1 teaspoon salt
\item Few grins pepper
\item 3 tablespoons rice or
\item 3 tablespoons barley
\end{itemize}
\end{multicols}}
\end{minipage}

\vspace{0.3em}
\noindent%
Wipe meat, remove skin and fat, and cut in small pieces. Put into kettle
with bones, and cover with cold water. Heat gradually to boiling-point,
skim, then season with salt and pepper. Cook slowly until meat is
tender, strain, and remove fat. Reheat to boiling-point, add rice or
barley, and cook until rice or barley is tender. If barley is used, soak
over night in cold water. Some of the meat may be served with the broth.



\needspace{15\baselineskip}
\section*{Irish Stew With Dumplings}

Wipe and cut in pieces three pounds lamb from the fore-quarter. Put in
kettle, cover with boiling water, and cook slowly two hours or until
tender. After cooking one hour add one-half cup each carrot and turnip
cut in one-half inch cubes, and one onion cut in slices. Fifteen minutes
before serving add four cups potatoes cut in one-fourth inch slices,
previously parboiled five minutes in boiling water. Thicken with
one-fourth cup flour, diluted with enough cold water to form a thin
smooth paste. Season with salt and pepper, serve with Dumplings. (See p.
205.)



\needspace{15\baselineskip}
\section*{Scotch Broth}

Wipe three pounds mutton cut from fore-quarter. Cut lean meat in
one-inch cubes, put in kettle, cover with three pints cold water, bring
quickly to boiling-point, skim, and add one-half cup barley which has
been soaked in cold water over night; simmer one and one-half hours, or
until meat is tender. Put bones in a second kettle, cover with cold
water, heat slowly to boiling-point, skim, and boil one and one-half
hours. Strain water from bones and add to meat. Fry five minutes in two
tablespoons butter, one-fourth cup each of carrot, turnip, onion, and
celery, cut in one-half inch dice, add to soup with salt and pepper to
taste, and cook until vegetables are soft. Thicken with two tablespoons
each of butter and flour cooked together. Add one-half tablespoon finely
chopped parsley just before serving. Rice may be used in place of
barley.



\needspace{15\baselineskip}
\section*{Lambs' Kidneys I}

Soak, pare, and cut in slices six kidneys, and sprinkle with salt and
pepper. Melt two tablespoons butter in hot frying-pan, put in kidneys,
and cook five minutes; dredge thoroughly with flour, and add two-thirds
cup boiling water or hot Brown Stock. Cook five minutes, add more salt
and pepper if needed. Lemon juice, onion juice, or Madeira wine may be
used for additional flavor. Kidneys must be cooked a short time, or for
several hours; they are tender after a few minutes' cooking, but soon
toughen, and need hours of cooking to again make them tender.



\needspace{15\baselineskip}
\section*{Lambs' Kidneys II}

Soak, pare, trim, and slice six kidneys. Sprinkle with salt and pepper,
sauté in butter, and remove to a hot dish. Cook one-half tablespoon
finely chopped onion in two tablespoons butter until brown; add three
tablespoons flour, and pour on slowly one and one-half cups hot stock.
Season with salt and pepper, strain, add kidneys, and one tablespoon
Madeira wine.



\needspace{15\baselineskip}
\section*{Ragout Of Kidneys}

Soak lambs' kidneys one hour in lukewarm water. Drain, clean, cut in
slices, season with salt and pepper, dredge with flour, and sauté in
butter. Fry one sliced onion and one-half shallot, finely chopped, in
three tablespoons butter until yellow; add three tablespoons flour and
one and one-fourth cups Brown Stock. Cook five minutes, strain, and add
one-half cup mushroom caps peeled and cut in quarters; season with salt
and pepper, add kidneys, and serve as soon as heated. White wine may be
added if desired.



\needspace{15\baselineskip}
\section*{Kidney Rolls}

Mix one-half cup stale bread crumbs, one-half small onion, finely
chopped, and one-half tablespoon finely chopped parsley. Season with
salt and pepper and moisten with beaten egg. Spread mixture on thin
slices of bacon, fasten around pieces of lambs' kidney, using skewers.
Bake in a hot oven twenty minutes.



\needspace{15\baselineskip}
\section*{Ways Of Warming Over Mutton And Lamb}


\needspace{15\baselineskip}
\subsection*{Minced Lamb on Toast}

Remove dry pieces of skin and gristle from remnants of cold roast lamb,
then chop meat. Heat in well-buttered frying-pan, season with salt,
pepper, and celery salt, and moisten with a little hot water or stock;
or, after seasoning, dredge well with flour, stir, and add enough stock
to make thin gravy. Pour over small slices of buttered toast.



\needspace{15\baselineskip}
\subsection*{Scalloped Lamb}

Remove skin and fat from thin slices of cold roast lamb, and sprinkle
with salt and pepper. Cover bottom of a buttered baking-dish with
buttered cracker crumbs; cover meat with boiled macaroni, and add
another layer of meat and macaroni. Pour over Tomato Sauce, and cover
with buttered cracker crumbs. Bake in hot oven until crumbs are brown.
Cold boiled rice may be used in place of macaroni.



\needspace{15\baselineskip}
\subsection*{Blanquette of Lamb}

Cut remnants of cooked lamb in cubes or strips. Reheat two cups meat in
two cups sauce,--sauce made of one-fourth cup each of butter and flour,
one cup White Stock, and one cup of milk which has been scalded with two
blades of mace. Season with salt and pepper, and add one tablespoon
Mushroom Catsup, or any other suitable table sauce. Garnish with large
croûtons, serve around green peas, or in a potato border, sprinkle with
finely chopped parsley.



\needspace{15\baselineskip}
\subsection*{Barbecued Lamb}

Cut cold roast lamb in thin slices and reheat in sauce made by melting
two tablespoons butter, adding three-fourths tablespoon vinegar,
one-fourth cup currant jelly, one-fourth teaspoon French mustard, and
salt and cayenne to taste.



\needspace{15\baselineskip}
\subsection*{Rechauffé of Lamb}

Brown two tablespoons butter, add two and one-half tablespoons flour,
and stir until well browned; then add one-fourth teaspoon, each, curry
powder, mustard, and salt, and one-eighth teaspoon paprika. Add,
gradually, one cup brown stock and two tablespoons sherry wine. Reheat
cold roast lamb cut in thin slices in sauce.



\needspace{15\baselineskip}
\subsection*{Salmi of Lamb}

Cut cold roast lamb in thin slices. Cook five minutes two tablespoons
butter with one-half tablespoon finely chopped onion. Add lamb, sprinkle
with salt and pepper, and cover with one cup Brown Sauce, or one cup
cold lamb gravy seasoned with Worcestershire, Harvey, or Elizabeth
Sauce. Cook until thoroughly heated. Arrange slices overlapping one
another lengthwise of platter, pour around sauce, and garnish with toast
points. A few sliced mushrooms or stoned olives improve this sauce.



\needspace{15\baselineskip}
\subsection*{Casserole of Rice and Meat}

Line a mould, slightly greased, with steamed rice. Fill the centre with
two cups cold, finely chopped, cooked mutton, highly seasoned with salt,
pepper, cayenne, celery salt, onion juice and lemon juice; then add
one-fourth cup cracker crumbs, one egg slightly beaten, and enough hot
stock or water to moisten. Cover meat with rice, cover rice with
buttered paper to keep out moisture while steaming, and steam forty-five
minutes. Serve on a platter surrounded with Tomato Sauce. Veal may be
used in place of mutton.



\needspace{15\baselineskip}
\subsection*{Breast of Lamb}

Wipe a breast of lamb, put in kettle with bouquet of sweet herbs, a
small onion stuck with six cloves, one-half tablespoon salt, one-half
teaspoon peppercorns, and one-fourth cup each carrot and turnip cut in
dice. Cover with boiling water, and simmer until bones will slip out
easily. Take meat from water, remove bones, and press under weight. When
cool, trim in shape, dip in crumbs, egg, and crumbs, fry in deep fat,
and drain. Serve with Spanish Sauce. Small pieces of cold lamb may be
sprinkled with salt and pepper, dipped in crumbs, egg, and crumbs, and
fried in deep fat.





\chapter{Veal}



Veal is the meat obtained from a young calf killed when six to eight
weeks old. Veal from a younger animal is very unwholesome, and is liable
to provoke serious gastric disturbances. Veal contains a much smaller
percentage of fat than beef or mutton, is less nutritious, and (though
from a young creature) more difficult of digestion. Like lamb, it is not
improved by long hanging, but should be eaten soon after killing and
dressing. It should always be remembered that the flesh of young animals
does not keep fresh as long as that of older ones. Veal is divided in
same manner as lamb, into fore and hind quarters. The fore-quarter is
subdivided into breast, shoulder, and neck; the hind-quarter into loin,
leg, and knuckle. Cutlets, fillets (cushion), and fricandeau are cut
from the thick part of leg.

Good veal may be known by its pinkish-colored flesh and white fat; when
the flesh lacks color, it has been taken from a creature which was too
young to kill for food, or, if of the right age, was bled before
killing. Veal may be obtained throughout the year, but is in season
during the spring. Veal should be thoroughly cooked; being deficient in
fat and having but little flavor, pork or butter should be added while
cooking, and more seasoning is required than for other meats.



\needspace{15\baselineskip}
\section*{Veal Cutlets}

Use slices of veal from leg cut one-half inch thick. Wipe, remove bone
and skin, then cut in pieces for serving. The long, irregular-shaped
pieces may be rolled, and fastened with small wooden skewers. Sprinkle
with salt and pepper; dip in flour, egg, and crumbs; fry slowly, until
well browned, in salt pork fat or butter; then remove cutlets to stewpan
and pour over one and one-half cups Brown Sauce. Place on back of range
and cook slowly forty minutes, or until cutlets are tender.

Veal may be cooked first in boiling water until tender, then crumbed and
fried. The water in which veal was cooked may be used for sauce. Arrange
on hot platter, strain sauce and pour around cutlets, and garnish with
parsley.

\textbf{Brown Sauce.} Brown three tablespoons butter, add three tablespoons
flour, and stir until well browned. Add gradually one and one-half cups
stock or water, or half stock and half stewed and strained tomatoes.
Season with salt, pepper, lemon juice, and Worcestershire Sauce. The
trimmings from veal (including skin and bones) may be covered with one
and one-half cups cold water, allowed to heat slowly to boiling-point,
then cooked, strained, and used for sauce.



\needspace{15\baselineskip}
\section*{Veal Chops Bavarian}

Wipe six loin chops and put in a stewpan with one-half onion, eight
slices carrot, two stalks celery, one-half teaspoon peppercorns, four
cloves, and two tablespoons butter. Cover with boiling water and cook
until tender. Drain, season with salt and pepper, dip in flour, egg, and
crumbs, fry in deep fat, and drain on brown paper. Arrange chops on hot
serving dish and surround with boiled flat macaroni to which Soubise
Sauce (see p. 267) is added.



\needspace{15\baselineskip}
\section*{Fricassee Of Veal}

Wipe two pounds sliced veal, cut from loin, and cover with boiling
water; add one small onion, two stalks celery, and six slices carrot.
Cook slowly until meat is tender. Remove meat, sprinkle with salt and
pepper, dredge with flour, and sauté in pork fat. Strain liquor (there
should be two cups). Melt four tablespoons butter, add four tablespoons
flour and strained liquor. Bring to boiling-point, season with salt and
pepper, and pour around meat. Garnish with parsley.



\needspace{15\baselineskip}
\section*{Minuten Fleisch}


\begin{minipage}{1.0\textwidth}
{\setlength{\multicolsep}{0pt}\setlength{\columnsep}{2em}\raggedcolumns%
\begin{multicols}{2}
\begin{itemize}
\setlength{\itemsep}{0pt}
\setlength{\parsep}{0pt}
\item 1 1/2 lbs. veal cut in thin slices
\item Salt and pepper
\item 2/3 cup claret wine
\item Flour
\item 1 1/3 cups Brown Stock
\item Juice 1 lemon
\item 2 sprigs parsley
\end{itemize}
\end{multicols}}
\end{minipage}

\vspace{0.3em}
\noindent%
Pound veal until one-fourth inch thick and cut in pieces for serving.
Sprinkle with salt and pepper, put in baking-pan, pour over wine, and
let stand thirty minutes. Drain, dip in flour, arrange in two buttered
pans, and pour over remaining ingredients and wine which was drained
from meat. Cover, and cook slowly until meat is tender. Remove to
serving dish and pour over sauce remaining in pan.



\needspace{15\baselineskip}
\section*{Loin Of Veal À La Jardinière}

Wipe four pounds loin of veal, sprinkle with salt and pepper, and dredge
with flour. Put one-fourth cup butter in deep stewpan; when melted, add
veal and brown entire surface of meat, watching carefully and turning
often, that it may not burn. Add one cup hot water, cover closely, and
cook slowly two hours, or until meat is tender, adding more water as
needed, using in all about three cups. Remove meat, thicken stock
remaining in pan with flour diluted with enough cold water to pour
easily. Surround the meat with two cups each boiled turnips and carrots,
cut in half-inch cubes, and potatoes cut in balls. Serve gravy in a
tureen.



\needspace{15\baselineskip}
\section*{Braised Shoulder Of Veal}

Bone, stuff, and sew in shape five pounds shoulder of veal; then cook
same as Braised Beef, adding with vegetables two sprigs thyme and one of
marjoram.



\needspace{15\baselineskip}
\section*{English Meat Pie}


\begin{minipage}{1.0\textwidth}
{\setlength{\multicolsep}{0pt}\setlength{\columnsep}{2em}\raggedcolumns%
\begin{multicols}{2}
\begin{itemize}
\setlength{\itemsep}{0pt}
\setlength{\parsep}{0pt}
\item 1 knuckle of veal
\item 1 slice onion
\item 1 slice carrot
\item Bit of bay leaf
\item Sprig of parsley
\item 12 peppercorns
\item Blade of mace
\item 2 teaspoons salt
\item 1/2 lb. lean raw ham
\item 4 tablespoons flour
\item 4 tablespoons butter
\item 2 doz. bearded oysters
\end{itemize}
\end{multicols}}
\end{minipage}

\vspace{0.3em}
\noindent%
Remove meat from bones. Cover bones with cold water, add vegetables and
seasonings, and heat slowly to boiling-point. Add meat, boil five
minutes, and let simmer until meat is tender; remove meat and reduce
stock to two cups. Put ham in frying-pan, cover with lukewarm water, and
let stand on back of range one hour. Brown butter, add flour, and when
well browned add stock; then add veal and ham each cut into cubes. Let
simmer twenty minutes and add oysters. Put in serving dish and cover
with top made of puff paste. It is much better to bake the paste
separately and cover pie just before sending to table.



\needspace{15\baselineskip}
\section*{Roast Veal}

The leg, cushion (thickest part of leg), and loin, are suitable pieces
for roasting. When leg is to be used, it should be boned at market. Wipe
meat, sprinkle with salt and pepper, stuff, and sew in shape. Place on
rack in dripping-pan, dredge meat and bottom of pan with flour, and
place around meat strips of fat salt pork. Bake three or four hours in
moderate oven, basting every fifteen minutes with one-third cup butter
melted in one-half cup boiling water, until used, then baste with fat in
pan. Serve with brown gravy.



\needspace{15\baselineskip}
\section*{Fricandeau Of Veal}

Lard a cushion of veal and roast or braise.



\needspace{15\baselineskip}
\section*{India Curry}

Wipe a slice of veal one-half inch thick, weighing one and one-half
pounds, and cook in frying-pan without butter, quickly searing one side,
then the other. Place on a board and cut in one and one-half inch
pieces. Fry two sliced onions in one-half cup butter until brown, remove
onions, and add to the butter, meat, and one-half tablespoon curry
powder, then cover with boiling water. Cook slowly until meat is tender.
Thicken with flour diluted with enough cold water to pour easily; then
add one teaspoon vinegar. Serve with a border of steamed rice.



\needspace{15\baselineskip}
\section*{Veal Birds}

Wipe slices of veal from leg, cut as thinly as possible, then remove
bone, skin, and fat. Pound until one-fourth inch thick and cut in pieces
two and one-half inches long by one and one-half inches wide, each piece
making a bird. Chop trimmings of meat, adding for every three birds a
piece of fat salt pork cut one inch square and one-fourth inch thick;
pork also to be chopped. Add to trimmings and pork one-half their
measure of fine cracker crumbs, and season highly with salt, pepper,
cayenne, poultry seasoning, lemon juice, and onion juice. Moisten with
beaten egg and hot water or stock. Spread each piece with thin layer of
mixture and avoid having mixture come close to edge. Roll, and fasten
with skewers. Sprinkle with salt and pepper, dredge with flour, and fry
in hot butter until a golden brown. Put in stewpan, add cream to half
cover meat, cook slowly twenty minutes or until tender. Serve on small
pieces of toast, straining cream remaining in pan over birds and toast,
and garnish with parsley. A Thin White Sauce in place of cream may be
served around birds.



\needspace{15\baselineskip}
\section*{Veal Loaf I}

Separate a knuckle of veal in pieces by sawing through bone. Wipe, put
in kettle with one pound lean veal and one onion; cover with boiling
water, and cook slowly until veal is tender. Drain, chop meat finely,
and season highly with salt and pepper. Garnish bottom of a mould with
slices of “hard-boiled” eggs and parsley. Put in layer of meat, layer of
thinly sliced “hard-boiled” eggs, sprinkle with finely chopped parsley,
and cover with remaining meat. Pour over liquor, which should be reduced
to one cupful. Press and chill, turn on a dish, and garnish with
parsley.



\needspace{15\baselineskip}
\section*{Veal Loaf II}

Wipe three pounds lean veal, and remove skin and membrane. Chop finely
or force through meat chopper, then add one-half pound fat salt pork
(also finely chopped), six common crackers (rolled), four tablespoons
cream, two tablespoons lemon juice, one tablespoon salt, one-half
tablespoon pepper, and a few drops onion juice. Pack in a small bread
pan, smooth evenly on top, brush with white of egg, and bake slowly
three hours, basting with one-fourth cup pork fat. Prick frequently
while baking, that pork fat may be absorbed by meat. Cool, remove from
pan, and cut in thin slices for serving.



\needspace{15\baselineskip}
\section*{Broiled Veal Kidneys}

Order veal kidneys with the suet left on. Trim, split, and broil ten
minutes. Arrange on pieces of toast and pour over melted butter seasoned
with salt, cayenne, and lemon juice.



\needspace{15\baselineskip}
\section*{Veal Kidneys À La Canfield}

Trim kidneys, cook in Brown Stock ten minutes, drain, and cut in slices.
Arrange alternate slices of kidney and thinly sliced bacon on skewers
with a fresh mushroom cap at either end of each skewer. Broil until
bacon is crisp and arrange on pieces of toast. Pour over sauce made from
stock in which kidneys were cooked, seasoned with salt, cayenne, and
Madeira wine.



\needspace{15\baselineskip}
\section*{Ways Of Warming Over Veal}


\needspace{15\baselineskip}
\subsection*{Minced Veal on Toast}

Prepare as Minced Lamb on Toast, using veal in place of lamb.



\needspace{15\baselineskip}
\subsection*{Blanquette of Veal}

Reheat two cups cold roast veal, cut in small strips, in one and
one-half cups White Sauce I. Serve in a potato border and sprinkle over
all finely chopped parsley.



\needspace{15\baselineskip}
\subsection*{Ragoût of Veal}

Reheat two cups cold roast veal, cut in cubes, in one and one-half cups
Brown Sauce seasoned with one teaspoon Worcestershire Sauce, few drops
of onion juice, and a few grains of cayenne.





\chapter{Sweetbreads}



A sweetbread is the thymus gland of lamb or calf, but in cookery, veal
sweetbreads only are considered. It is prenatally developed, of unknown
function, and as soon as calf is taken from liquid food it gradually
disappears. Pancreas, stomach sweetbread, is sold in some sections of
the country, but in our markets this custom is not practised.
Sweetbreads are a reputed table delicacy, and a valuable addition to the
menu of the convalescent.

A sweetbread consists of two parts, connected by tubing and membranes.
The round, compact part is called the heart sweetbread, as its position
is nearer the heart; the other part is called the throat sweetbread.
When sweetbread is found in market separated, avoid buying two of the
throat sweetbreads, as the heart sweetbread is more desirable.

Sweetbreads spoil very quickly. They should be removed from paper as
soon as received from market, plunged into cold water and allowed to
stand one hour, drained, and put into acidulated salted boiling water
then allowed to cook slowly twenty minutes; again drained, and plunged
into cold water, that they may be kept white and firm. Sweetbreads are
always parboiled in this manner for subsequent cooking.



\needspace{15\baselineskip}
\section*{Broiled Sweetbread}

Parboil a sweetbread, split crosswise, sprinkle with salt and pepper,
and broil five minutes. Serve with Lemon Butter.



\needspace{15\baselineskip}
\section*{Creamed Sweetbread}

Parboil a sweetbread, and cut in one-half inch cubes, or separate in
small pieces. Reheat in one cup White Sauce II. Creamed Sweetbread may
be served on toast, or used as filling for patty cases or Swedish
Timbales.



\needspace{15\baselineskip}
\section*{Creamed Sweetbread And Chicken}

Reheat equal parts of cold cooked chicken, and sweetbread cut in dice,
in White Sauce II.



\needspace{15\baselineskip}
\section*{Sweetbread À La Poulette}

Reheat sweetbread, cut in cubes, in one cup Béchamel Sauce.



\needspace{15\baselineskip}
\section*{Sweetbreads, Country Style}

Parboil sweetbreads, sprinkle with salt and pepper, and dredge with
flour. Arrange in baking-dish, brush over with melted butter, allowing
two tablespoons to each pair of sweetbreads, and cover with thin slices
fat salt pork. Bake in a hot oven over twenty-five minutes, basting
twice during the cooking, and remove pork during the last five minutes
of the cooking.



\needspace{15\baselineskip}
\section*{Larded Sweetbread}

Parboil a sweetbread, lard the upper side, and bake until well browned,
basting with Meat Glaze.



\needspace{15\baselineskip}
\section*{Sweetbreads À La Napoli}

Parboil a large sweetbread and cut in eight pieces. Cook in hot
frying-pan with a small quantity of butter, adding enough beef extract
to give sweetbread a glazed appearance. Cut bread in slices, shape with
a circular cutter three and one-half inches in diameter, and toast.
Spread each piece with two tablespoons grated Parmesan cheese seasoned
with salt and paprika and moistened with two tablespoons heavy cream.
Arrange one piece of sweetbread on each piece of toast. Put in
individual glass-covered dishes, having two tablespoons cream in each
dish. Cover each piece of sweetbread with sautéd mushroom cap, put on
glass covers, and bake in a moderate oven eight minutes.



\needspace{15\baselineskip}
\section*{Braised Sweetbreads Eugénie}

Parboil a sweetbread in Sherry wine twelve minutes. Drain, cool, cut in
four pieces, and lard. Cook in frying-pan same as Sweetbreads à la
Napoli. Peel mushroom caps, cover with Sherry wine, let stand one hour,
drain, and sauté in butter. Arrange on circular pieces of toast, over
each of which has been poured one teaspoon wine drained from mushroom
caps. Pile five or six mushroom caps on each piece of sweetbread, add
two tablespoons heavy cream, and bake in a moderate oven, eight minutes.
Cook in individual glass-covered dishes.



\needspace{15\baselineskip}
\section*{Sweetbread Cutlets With Asparagus Tips}

Parboil a sweetbread, split, and cut in pieces shaped like a small
cutlet, or cut in circular pieces. Sprinkle with salt and pepper, dip in
crumbs, egg, and crumbs, and sauté in butter. Arrange in a circle around
Creamed Asparagus Tips.



\needspace{15\baselineskip}
\section*{Sweetbread With Tomato Sauce}

Prepare as Sweetbread Cutlets with Asparagus Tips, sauté in butter or
fry in deep fat, and serve with Tomato Sauce.



\needspace{15\baselineskip}
\section*{Sweetbread And Bacon}

Parboil a sweetbread, cut in small pieces, dip in flour, egg, and
crumbs, and arrange alternately with pieces of bacon on small skewers,
having four pieces sweetbread and three of bacon on each skewer. Fry in
deep fat, and drain. Arrange in a circle around mound of green peas.









\chapter{Pork}



Pork is the flesh and fat of pig or hog. Different parts of the
creature, when dressed, take different names.

The chine and spareribs, which correspond to the loin in lamb and veal,
are used for roasts or steaks. Two ribs are left on the chine. The hind
legs furnish \textit{hams}. These are cured, salted, and smoked. Sugar-cured
hams are considered the best. Pickle, to which is added light brown
sugar, molasses, and saltpetre, is introduced close to bone; hams are
allowed to hang one week, then smoked with hickory wood. \textit{Shoulders} are
usually corned, or salted and smoked, though sometimes cooked fresh.
\textit{Pigs' feet} are boiled until tender, split, and covered with vinegar
made from white wine. \textit{Hocks}, the part just above the feet, are corned,
and much used by Germans. \textit{Heads} are soused, and cooked by boiling. The
flank, which lies just below the ribs, is salted and smoked, and
furnishes \textit{bacon}. The best pieces of fat salt pork come from the back,
on either side of backbone.

Fat, when separated from flesh and membrane, is tried out and called
lard. \textit{Leaf lard} is the best, and is tried out from the leaf shaped
pieces of solid fat which lie inside the flank. \textit{Sausages} are trimmings
of lean and fat meat, minced, highly seasoned, and forced into thin
casings made of the prepared entrails. \textit{Little pigs} (four weeks old)
are sometimes killed; dressed, and roasted whole.

Pork contains the largest percentage of fat of any meat. When eaten
fresh it is the most difficult of digestion, and although found in
market through the entire year, it should be but seldom served, and then
only during the winter months. By curing, salting, and smoking, pork is
rendered more wholesome. \textit{Bacon}, next to butter and cream, is the most
easily assimilated of all fatty foods.



\needspace{15\baselineskip}
\section*{Pork Chops}

Wipe chops, sprinkle with salt and pepper, place in a hot frying-pan,
and cook slowly until tender, and well browned on each side.



\needspace{15\baselineskip}
\section*{Pork Chops With Fried Apples}

Arrange Pork Chops on a platter, and surround with slices of apples, cut
one-half inch thick, fried in the fat remaining in pan.



\needspace{15\baselineskip}
\section*{Roast Pork}

Wipe pork, sprinkle with salt and pepper, place on a rack in a
dripping-pan, and dredge meat and bottom of pan with flour. Bake in a
moderate oven three or four hours, basting every fifteen minutes with
fat in pan. Make a gravy as for other roasts.



\needspace{15\baselineskip}
\section*{Pork Tenderloins With Sweet Potatoes}

Wipe tenderloins, put in a dripping-pan, and brown quickly in a hot
oven; then sprinkle with salt and pepper, and bake forty-five minutes,
basting every fifteen minutes.

\textbf{Sweet Potatoes.} Pare six potatoes and parboil ten minutes, drain, put
in pan with meat, and cook until soft, basting when basting meat.



\needspace{15\baselineskip}
\section*{Breakfast Bacon}

See Liver and Bacon, page 207.



\needspace{15\baselineskip}
\section*{Fried Salt Pork With Codfish}

Cut fat salt pork in one-fourth inch slices, cut gashes one-third inch
apart in slices, nearly to rind. Try out in a hot frying-pan until brown
and crisp, occasionally turning off fat from pan. Serve around strips of
codfish which have been soaked in pan of lukewarm water and allowed to
stand on back of range until soft. Serve with Drawn Butter Sauce, boiled
potatoes, and beets.



\needspace{15\baselineskip}
\section*{Broiled Ham}

Soak thin slices of ham one hour in lukewarm water. Drain, wipe, and
broil three minutes.



\needspace{15\baselineskip}
\section*{Fried Ham And Eggs}

Wipe ham, remove one-half outside layer of fat, and place in frying-pan.
Cover with tepid water and let stand on back of range thirty minutes;
drain, and dry on a towel. Heat pan, put in ham, brown quickly on one
side, turn and brown other side; or soak ham over night, dry, and cook
in hot frying-pan. If cooked too long, ham will become hard and dry.
Serve with fried eggs cooked in the dried-out ham fat.



\needspace{15\baselineskip}
\section*{Barbecued Ham}

Soak thin slices of ham one hour in lukewarm water; drain, wipe, and
cook in a hot frying-pan until slightly browned. Remove to serving dish
and add to fat in pan three tablespoons vinegar mixed with one and
one-half teaspoons mustard, one-half teaspoon sugar, and one-eighth
teaspoon paprika. When thoroughly heated pour over ham and serve at
once.



\needspace{15\baselineskip}
\section*{Boiled Ham}

Soak several hours or over night in cold water to cover. Wash
thoroughly, trim off hard skin near end of bone, put in a kettle, cover
with cold water, heat to boiling-point, and cook slowly until tender.
See Time Table for Cooking, page 28. Remove kettle from range and set
aside, that ham may partially cool; then take from water, remove outside
skin, sprinkle with sugar and fine cracker crumbs, and stick with cloves
one-half inch apart. Bake one hour in a slow oven. Serve cold, thinly
sliced.



\needspace{15\baselineskip}
\section*{Roast Ham With Champagne Sauce}

Place a whole baked ham in the oven fifteen minutes before serving time,
that outside fat may be heated. Remove to a hot platter, garnish bone
end with a paper ruffle, and serve with Champagne Sauce.



\needspace{15\baselineskip}
\section*{Westphalian Ham}

These hams are imported from Germany, and need no additional cooking.
Cut in very thin slices for serving.



\needspace{15\baselineskip}
\section*{Broiled Pigs' Feet}

Wipe, sprinkle with salt and pepper, and broil six to eight minutes.
Serve with Maître d'Hôtel Butter or Sauce Piquante.



\needspace{15\baselineskip}
\section*{Fried Pigs' Feet}

Wipe, sprinkle with salt and pepper, dip in crumbs, egg, and crumbs, fry
in deep fat, and drain.



\needspace{15\baselineskip}
\section*{Sausages}

Cut apart a string of sausages. Pierce each sausage several times with a
carving fork. Put in frying-pan, cover with boiling water, and cook
fifteen minutes; drain, return to frying-pan, and fry until well
browned. Serve with fried apples. Sausages are often broiled same as
bacon and apples baked in pan under them.



\needspace{15\baselineskip}
\section*{Boston Baked Beans}

Pick over one quart pea beans, cover with cold water, and soak over
night. In morning, drain, cover with fresh water, heat slowly (keeping
water below boiling-point), and cook until skins will burst,--which is
best determined by taking a few beans on the tip of a spoon and blowing
on them, when skins will burst if sufficiently cooked. Beans thus tested
must, of course, be thrown away. Drain beans, throwing bean-water out of
doors, not in sink. Scald rind of three-fourths pound fat salt pork,
scrape, remove one-fourth inch slice and put in bottom of bean-pot. Cut
through rind of remaining pork every one-half inch, making cuts one inch
deep. Put beans in pot and bury pork in beans, leaving rind exposed. Mix
one tablespoon salt, one tablespoon molasses, and three tablespoons
sugar; add one cup boiling water, and pour over beans; then add enough
more boiling water to cover beans. Cover bean-pot, put in oven, and bake
slowly six or eight hours, uncovering the last hour of cooking, that
rind may become brown and crisp. Add water as needed. Many feel sure
that by adding with seasonings one-half tablespoon mustard, the beans
are more easily digested. If pork mixed with lean is preferred, use less
salt.

The fine reputation which Boston Baked Beans have gained has been
attributed to the earthen bean-pot with small top and bulging sides in
which they are supposed to be cooked. Equally good beans have often been
eaten where a five-pound lard pail was substituted for the broken
bean-pot.

Yellow-eyed beans are very good when baked.





\chapter{Poultry And Game}



Poultry includes all domestic birds suitable for food except pigeon and
squab. Examples: chicken, fowl, turkey, duck, goose, etc. Game includes
such birds and animals suitable for food as are pursued and taken in
field and forest. Examples: quail, partridge, wild duck, plover, deer,
etc.

The flesh of chicken, fowl, and turkey has much shorter fibre than that
of ruminating animals, and is not intermingled with fat,--the fat always
being found in layers directly under the skin, and surrounding the
intestines. Chicken, fowl, and turkey are nutritious, and chicken is
specially easy of digestion. The white meat found on breast and wing is
more readily digested than the dark meat. The legs, on account of
constant motion, are of a coarser fibre and darker color.

Since incubators have been so much used for hatching chickens, small
birds suitable for broiling may be always found in market. Chickens
which appear in market during January weighing about one and one-half
pounds are called \textit{spring chickens}.

Fowl is found in market throughout the year, but is at its best from
March until June.

Philadelphia, until recently, furnished our market with Philadelphia
chickens and \textit{capons}, but now Massachusetts furnishes equally good
ones, which are found in market from December to June. They are very
large, plump, and superior eating. At an early age they are deprived of
the organs of reproduction, penned, and specially fatted for killing.
They are recognized by the presence of head, tail, and wing feathers.

Turkeys are found in market throughout the year, but are best during the
winter months. Tame ducks and geese are very indigestible on account of
the large quantity of fat they contain. Goose meat is thoroughly
infiltrated with fat, containing sometimes forty to forty-five per cent.
Pigeons, being old birds, need long, slow cooking to make them tender.
Squabs (young pigeons) make a delicious tidbit for the convalescent, and
are often the first meat allowed a patient by the physician.

The flesh of game, with the exception of wild duck and wild geese, is
tender, contains less fat than poultry, is of fine though strong flavor,
and easy of digestion. Game meat is usually of dark color, partridge and
quail being exceptions, and is usually cooked rare. Venison, the flesh
of deer, is short-fibred, dark-colored, highly savored, tender, and easy
of digestion; being highly savored, it often disagrees with those of
weak digestion.

Geese are in market throughout the year; Massachusetts and Rhode Island
furnishing specially good ones. A goose twelve weeks old is known as a
\textit{green goose}. They may be found in market from May to September. Young
geese which appear in market September first and continue through
December are called \textit{goslings}. They have been hatched during May and
June, and then fatted for market.

Young ducks, found in market about March first, are called \textit{ducklings}.
Canvasback Ducks have gained a fine reputation throughout the country,
and are found in market from the last of November until March. Redhead
Ducks are in season two weeks earlier, and are about as good eating as
Canvasback Ducks, and much less in price. The distinctive flavor of both
is due to the wild celery on which they feed. Many other kinds of ducks
are found in market during the fall and winter. Examples: Widgeon,
Mallard, Lake Erie Teal, Black Ducks, and Butterballs.

Fresh quail are in market from October fifteenth to January first, the
law forbidding their being killed at any other time in the year. The
same is true of partridge, but both are frozen and kept in cold storage
several months. California sends frozen quail in large numbers to
Eastern markets. Grouse (\textit{prairie chicken}) are always obtainable,--fresh
ones in the fall; later, those kept in cold storage. Plover may be
bought from April until December.

\textbf{To Select Poultry and Game.} A \textit{chicken} is known by soft feet, smooth
skin, and soft cartilage at end of breastbone. An abundance of
pinfeathers always indicates a young bird, while the presence of long
hairs denotes age. In a \textit{fowl} the feet have become hard and dry with
coarse scales, and cartilage at end of breastbone has ossified. \_Cock
turkeys\_ are usually better eating than hen turkeys, unless hen turkey
is young, small, and plump. A good turkey should be plump, have smooth
dark legs, and cartilage at end of breastbone soft and pliable. Good
geese abound in pinfeathers. Small birds should be plump, have soft feet
and pliable bills.

\textbf{To Dress and Clean Poultry.} Remove hairs and down by holding the bird
over a flame (from gas, alcohol, or burning paper) and constantly
changing position until all parts of surface have been exposed to flame;
this is known as \textit{singeing}. Cut off the head and draw out pinfeathers,
using a small pointed knife. Cut through the skin around the leg one and
one-half inches below the leg joint, care being taken not to cut
tendons; place leg at this cut over edge of board, press downward to
snap the bone, then take foot in right hand, holding bird firmly in left
hand, and pull off foot, and with it the tendons. In old birds the
tendons must be drawn separately, which is best accomplished by using a
steel skewer. Make an incision through skin below breastbone, just large
enough to admit the hand. With the hand remove entrails, gizzard, heart,
and liver; the last three named constitute what is known as \textit{giblets}.
The gall-bladder, lying on the under surface of the right lobe of the
liver, is removed with liver, and great care must be taken that it is
not broken, as a small quantity of the bile which it contains would
impart a bitter flavor to the parts with which it came in contact.
Enclosed by the ribs, on either side of backbone, may be found the
lungs, of spongy consistency and red color. Care must be taken that
every part of them is removed. Kidneys, lying in the hollow near end of
backbone, must also be removed. By introducing first two fingers under
skin close to neck, the windpipe may be easily found and withdrawn; also
the crop, which will be found adhering to skin close to breast. Draw
down neck skin, and cut off neck close to body, leaving skin long enough
to fasten under the back. Remove oil bag, and wash bird by allowing cold
water to run through it, not allowing bird to soak in cold water. Wipe
inside and outside, looking carefully to see that everything has been
withdrawn. If there is disagreeable odor, suggesting that fowl may have
been kept too long, clean at once, wash inside and out with soda water,
and sprinkle inside with charcoal and place some under wings.

Poultry dressed at market seldom have tendons removed unless so ordered.
It is always desirable to have them withdrawn, as they become hard and
bony during cooking. It is the practice of market-men to cut a gash
through the skin, to easier reach crop and windpipe. This gash must be
sewed before stuffing, and causes the bird to look less attractive when
cooked.

\textbf{To Cut up a Fowl.} Singe, draw out pinfeathers, cut off head, remove
tendons and oil bag. Cut through skin between leg and body close to
body, bend back leg (thus breaking ligaments), cut through flesh, and
separate at joint. Separate the upper part of leg, \textit{second joint}, from
lower part of leg, \textit{drumstick}, as leg is separated from body. Remove
wing by cutting through skin and flesh around upper wing joint which
lies next to body, then disjoint from body. Cut off tip of wing and
separate wing at middle joint. Remove leg and wing from other side.
Separate breast from back by cutting through skin, beginning two inches
below breastbone and passing knife between terminus of small ribs on
either side and extending cut to collar-bone. Before removing entrails,
gizzard, heart, liver, lungs, kidneys, crop, and windpipe, observe their
position, that the anatomy of the bird may be understood. The back is
sometimes divided by cutting through the middle crosswise. The wishbone,
with adjoining meat, is frequently removed, and the breast meat may be
separated in two parts by cutting through flesh close to breastbone with
cleaver. Wipe pieces, excepting back, with cheese-cloth wrung out of
cold water. Back piece needs thorough washing.

\textbf{To Clean Giblets.} Remove thin membrane, arteries, veins, and clotted
blood around heart. Separate gall-bladder from liver, cutting off any of
liver that may have a greenish tinge. Cut fat and membranes from
gizzard. Make a gash through thickest part of gizzard, and cut as far as
inner lining, being careful not to pierce it. Remove the inner sack and
discard. Wash giblets and cook until tender, with neck and tips of
wings, putting them in cold water and heating water quickly that some of
the flavor may be drawn out into stock, which is to be used for making
gravy.

\textbf{To Stuff Poultry.} Put stuffing by spoonfuls in neck end, using enough
to sufficiently fill the skin, that bird may look plump when served.
Where cracker stuffing is used, allowance must be made for the swelling
of crackers, otherwise skin may burst during cooking. Put remaining
stuffing in body; if the body is full, sew skin; if not full, bring skin
together with a skewer.

\textbf{To Truss Fowl.} Draw thighs close to body and hold by inserting a steel
skewer under middle joint running it through body, coming out under
middle joint on other side. Cut piece three-fourths inch wide from neck
skin, and with it fasten legs together at ends; or cross drumsticks, tie
securely with a long string, and fasten to tail. Place wings close to
body and hold them by inserting a second skewer through wing, body, and
wing on opposite side. Draw neck skin under back and fasten with a small
wooden skewer. Turn bird on its breast. Cross string attached to tail
piece and draw it around each end of lower skewer; again cross string
and draw around each end of upper skewer; fasten string in a knot and
cut off ends. In birds that are not stuffed legs are often passed
through incisions cut in body under bones near tail.

\textbf{To Dress Birds for Broiling.} Singe, wipe, and with a sharp-pointed
knife, beginning at back of neck, make a cut through backbone the entire
length of bird. Lay open the bird and remove contents from inside. Cut
out rib bones on either side of backbone, remove from breastbone, then
cut through tendons at joints.

\textbf{To Fillet a Chicken.} Remove skin from breast, and with a small sharp
knife begin at end of collar-bone and cut through flesh, following close
to wish and breast bones the entire length of meat. Raise flesh with
fingers, and with knife free the piece of meat from bones which lie
under it. Cut meat away from wing joint; this solid piece of breast is
meat known as a \textit{fillet}. This meat is easily separated in two parts.
The upper, larger part is called the \textit{large fillet}; the lower part the
\textit{mignon fillet}. The tough skin on the outside of large fillet should be
removed, also the sinew from mignon fillet. To remove tough skin, place
large fillet on a board, upper side down, make an incision through flesh
at top of fillet, and cut entire length of fillet, holding knife as
close to skin as possible. Trim edges, that fillet may look shapely.



\needspace{15\baselineskip}
\section*{Broiled Chicken}

Dress for broiling, following directions on page 244. Sprinkle with salt
and pepper, and place in a well-greased broiler. Broil twenty minutes
over a clear fire, watching carefully and turning broiler so that all
parts may be equally browned. The flesh side must be exposed to the fire
the greater part of time, as the skin side will brown quickly. Remove to
a hot platter, spread with soft butter, and sprinkle with salt and
pepper. Chickens are so apt to burn while broiling that many prefer to
partially cook in oven. Place chicken in dripping-pan, skin side down,
sprinkle with salt and pepper, dot over with butter, and bake fifteen
minutes in hot oven; then broil to finish cooking. \textit{Guinea chickens} are
becoming popular cooked in this way.



\needspace{15\baselineskip}
\section*{Boiled Fowl}

Dress, clean, and truss a four-pound fowl, tie in cheese-cloth, place on
trivet in a kettle, half surround with boiling water, cover, and cook
slowly until tender, turning occasionally. Add salt the last hour of
cooking. Serve with Egg, Oyster, or Celery Sauce. It is not desirable to
stuff a boiled fowl.



\needspace{15\baselineskip}
\section*{Boiled Capon With Cauliflower Sauce}

Prepare and cook a capon same as Boiled Fowl, and serve surrounded with
Cauliflower Sauce and garnished with parsley.



\needspace{15\baselineskip}
\section*{Chicken À La Providence}

Prepare and boil a chicken, following recipe for Boiled Fowl. The liquor
should be reduced to two cups, and used for making sauce, with two
tablespoons each butter and flour cooked together. Add to sauce one-half
cup each of cooked carrot (cut in fancy shapes) and green peas, one
teaspoon lemon juice, yolks two eggs, salt and pepper. Place chicken on
hot platter, surround with sauce, and sprinkle chicken and sauce with
one-half tablespoon finely chopped parsley.



\needspace{15\baselineskip}
\section*{Stewed Chicken With Onions}

Dress, clean, and cut in pieces for serving, two chickens. Cook in a
small quantity of water with eighteen tiny young onions. Remove chicken
to serving dish as soon as tender, and when onions are soft drain from
stock and reduce stock to one and one-half cups. Make sauce of three
tablespoons butter, four tablespoons flour, stock, and one-half cup
heavy cream; then add yolks three eggs, salt, pepper, and lemon juice to
taste. Pour sauce over chicken and onions.



\needspace{15\baselineskip}
\section*{Chicken À La Stanley}

Melt one-fourth cup butter, add one large onion thinly sliced, and two
broilers cut in pieces for serving; cover, and cook slowly ten minutes;
then add one cup Chicken Stock, and cook until meat is tender. Remove
chickens, rub stock and onions through a sieve, and add one and one-half
tablespoons each butter and flour cooked together. Add cream to make
sauce of the right consistency. Season with salt and pepper. Arrange
chicken on serving dish, pour around sauce, and garnish dish with
bananas cut in diagonal slices dipped in flour and sautéd in butter.



\needspace{15\baselineskip}
\section*{Chili Con Carni}

Clean, singe, and cut in pieces for serving, two young chickens. Season
with salt and pepper, and sauté in butter. Remove seeds and veins from
eight red peppers, cover with boiling water, and cook until soft; mash,
and rub through a sieve. Add one teaspoon salt, one onion finely
chopped, two cloves of garlic finely chopped, the chicken, and boiling
water to cover. Cook until chicken is tender. Remove to serving dish,
and thicken sauce with three tablespoons each butter and flour cooked
together; there should be one and one-half cups sauce. Canned pimentoes
may be used in place of red peppers.



\needspace{15\baselineskip}
\section*{Roast Chicken}

Dress, clean, stuff, and truss a chicken. Place on its back on rack in a
dripping-pan, rub entire surface with salt, and spread breast and legs
with three tablespoons butter, rubbed until creamy and mixed with two
tablespoons flour. Dredge bottom of pan with flour. Place in a hot oven,
and when flour is well browned, reduce the heat, then baste. Continue
basting every ten minutes until chicken is cooked. For basting, use
one-fourth cup butter, melted in two-thirds cup boiling water, and after
this is gone, use fat in pan, and when necessary to prevent flour
burning, add one cup boiling water. During cooking, turn chicken
frequently, that it may brown evenly. If a thick crust is desired,
dredge bird with flour two or three times during cooking. If a glazed
surface is preferred, spread bird with butter, omitting flour, and do
not dredge during baking. When breast meat is tender, bird is
sufficiently cooked. A four-pound chicken requires about one and
one-half hours.



\needspace{15\baselineskip}
\section*{Stuffing I}


\begin{itemize}
\setlength{\itemsep}{0pt}
\setlength{\parsep}{0pt}
\item 1 cup cracker crumbs
\item 1/3 cup butter
\item 1/3 cup boiling water
\item Salt and Pepper
\item Powdered sage, summer savory, or marjoram
\end{itemize}

\vspace{-0.5em}
\noindent%
Melt butter in water, and pour over crackers, to which seasonings have
been added.



\needspace{15\baselineskip}
\section*{Stuffing II}


\begin{minipage}{1.0\textwidth}
{\setlength{\multicolsep}{0pt}\setlength{\columnsep}{2em}\raggedcolumns%
\begin{multicols}{2}
\begin{itemize}
\setlength{\itemsep}{0pt}
\setlength{\parsep}{0pt}
\item 1 cup cracker crumbs
\item 1/4 cup melted butter
\item Sage of Poultry Seasoning
\item Salt
\item Pepper
\item 2/3 cup scalded milk
\end{itemize}
\end{multicols}}
\end{minipage}

\vspace{0.3em}
\noindent%
Make same as Stuffing I.



\needspace{15\baselineskip}
\section*{Gravy}

Pour off liquid in pan in which chicken has been roasted. From liquid
skim off four tablespoons fat; return fat to pan, and brown with four
tablespoons flour; add two cups stock in which giblets, neck, and tips
of wings have been cooked. Cook five minutes, season with salt and
pepper, then strain. The remaining fat may be used, in place of butter,
for frying potatoes, or for basting when roasting another chicken.

For \textbf{Giblet Gravy}, add to the above, giblets (heart, liver, and
gizzard) finely chopped.



\needspace{15\baselineskip}
\section*{Braised Chicken}

Dress, clean, and truss a four-pound fowl. Try out two slices fat salt
pork cut one-fourth inch thick; remove scraps, and add to fat five
slices carrot cut in small cubes, one-half sliced onion, two sprigs
thyme, one sprig parsley, and one bay leaf, then cook ten minutes; add
two tablespoons butter, and fry fowl, turning often until surface is
well browned. Place on trivet in a deep pan, pour over fat, and add two
cups boiling water or Chicken Stock. Cover, and bake in slow oven until
tender, basting often, and adding more water if needed. Serve with a
sauce made from stock in pan, first straining and removing the fat.



\needspace{15\baselineskip}
\section*{Chicken Fricassee}

Dress, clean, and cut up a fowl. Put in a kettle, cover with boiling
water, and cook slowly until tender, adding salt to water when chicken
is about half done. Remove from water, sprinkle with salt and pepper,
dredge with flour, and sauté in butter or pork fat. Arrange chicken on
pieces of dry toast placed on a hot platter, having wings and second
joints opposite each other, breast in centre of platter, and drumsticks
crossed just below second joints. Pour around White or Brown Sauce.
Reduce stock to two cups, strain, and remove the fat. Melt three
tablespoons butter, add four tablespoons flour, and pour on gradually
one and one-half cups stock. Just before serving, add one-half cup
cream, and salt and pepper to taste; or make a sauce by browning butter
and flour and adding two cups stock, then seasoning with salt and
pepper.

Fowls, which are always made tender by long cooking, are frequently
utilized in this way. If chickens are employed, they are sautéd without
previous boiling, and allowed to simmer fifteen to twenty minutes in the
sauce.



\needspace{15\baselineskip}
\section*{Fried Chicken}

Fried chicken is prepared and cooked same as Chicken Fricassee, with
Brown Sauce, chicken always being used, never fowl.



\needspace{15\baselineskip}
\section*{Fried Chicken (Southern Style)}

Clean, singe, and cut in pieces for serving, two young chickens. Plunge
in cold water, drain but do not wipe. Sprinkle with salt and pepper, and
coat thickly with flour, having as much flour adhere to chicken as
possible. Try out one pound fat salt pork cut in pieces, and cook
chicken slowly in fat until tender and well browned. Serve with White
Sauce made of half milk and half cream.



\needspace{15\baselineskip}
\section*{Maryland Chicken}

Dress, clean, and cut up two chickens. Sprinkle with salt and pepper,
dip in flour, egg, and crumbs, place in a well-greased dripping-pan, and
bake thirty minutes in a hot oven, basting after first five minutes of
cooking with one-third cup melted butter. Arrange on platter and pour
over two cups Cream Sauce.



\needspace{15\baselineskip}
\section*{Blanketed Chicken}

Split and clean two broilers. Place in dripping-pan and sprinkle with
salt, pepper, two tablespoons green pepper finely chopped, and one
tablespoon chives finely cut. Cover with strips of bacon thinly cut, and
bake in a hot oven until chicken is tender. Remove to serving dish and
pour around the following sauce:

To three tablespoons fat, taken from dripping-pan, add four tablespoons
flour and one and one-half cups thin cream, or half chicken stock and
half cream may be used. Season with salt and pepper.



\needspace{15\baselineskip}
\section*{Chicken À La Merango}

Dress, clean, and cut up a chicken. Sprinkle with salt and pepper,
dredge with flour, and sauté in salt pork fat. Put in a stewpan, cover
with sauce, and cook slowly until chicken is tender. Add one-half can
mushrooms cut in quarters, and cook five minutes. Arrange chicken on
serving dish and pour around sauce; garnish with parsley.



\needspace{15\baselineskip}
\section*{Sauce}


\begin{minipage}{1.0\textwidth}
{\setlength{\multicolsep}{0pt}\setlength{\columnsep}{2em}\raggedcolumns%
\begin{multicols}{2}
\begin{itemize}
\setlength{\itemsep}{0pt}
\setlength{\parsep}{0pt}
\item 1/4 cup butter
\item 1 tablespoon finely chopped onion
\item 1 slice carrot, cut in cubes
\item 1 slice turnip, cut in cubes
\item 1/4 cup flour
\item 2 cups boiling water
\item 1/2 cup stewed and strained tomato
\item 1 teaspoon salt
\item 1/8 teaspoon pepper
\item Few grains cayenne
\end{itemize}
\end{multicols}}
\end{minipage}

\vspace{0.3em}
\noindent%
Cook butter five minutes with vegetables. Add flour, with salt, pepper,
and cayenne, and cook until flour is well browned. Add gradually water
and tomato; cook five minutes, then strain.



\needspace{15\baselineskip}
\section*{Baked Chicken}

Dress, clean, and cut up two chickens. Place in a dripping-pan, sprinkle
with salt and pepper, dredge with flour, and dot over with one-fourth
cup butter. Bake thirty minutes in a hot oven, basting every five
minutes with one-fourth cup butter melted in one-fourth cup boiling
water. Serve with gravy made by using fat in pan, one-fourth cup flour,
one cup each Chicken Stock and cream, salt and pepper.



\needspace{15\baselineskip}
\section*{Planked Chicken}


\begin{minipage}{1.0\textwidth}
{\setlength{\multicolsep}{0pt}\setlength{\columnsep}{2em}\raggedcolumns%
\begin{multicols}{2}
\begin{itemize}
\setlength{\itemsep}{0pt}
\setlength{\parsep}{0pt}
\item 1/4 cup butter
\item 1/4 tablespoon red pepper
\item 1/4 tablespoon green pepper
\item 1/4 tablespoon parsley
\item Duchess potatoes
\item 1 teaspoon finely chopped onion
\item 1/2 clove garlic, finely chopped
\item 1 teaspoon lemon juice
\item 8 mushroom caps
\end{itemize}
\end{multicols}}
\end{minipage}

\vspace{0.3em}
\noindent%
Cream the butter, add pepper, parsley, onion, garlic, and lemon juice.
Split a young chicken as for broiling, place in dripping-pan, sprinkle
with salt and pepper, dot over with butter, and bake in a hot oven until
nearly cooked. Butter plank, arrange a border of Duchess Potatoes (see
p. 312) close to edge of plank, and remove chicken to plank. Clean,
peel, and sauté mushroom caps, place on chicken, spread over prepared
butter, and put in a very hot oven to brown potatoes and finish cooking
chicken. Serve on the plank.



\needspace{15\baselineskip}
\section*{Chicken Gumbo}

Dress, clean, and cut up a chicken. Sprinkle with salt and pepper,
dredge with flour, and sauté in pork fat. Fry one-half finely chopped
onion in fat remaining in frying-pan. Add four cups sliced okra, sprig
of parsley, and one-fourth red pepper finely chopped, and cook slowly
fifteen minutes. Add to chicken, with one and one-half cups tomato,
three cups boiling water, and one and one-half teaspoons salt. Cook
slowly until chicken is tender, then add one cup boiled rice.



\needspace{15\baselineskip}
\section*{Chicken Stew}

Dress, clean, and cut up a fowl. Put in a stewpan, cover with boiling
water, and cook slowly until tender, adding one-half tablespoon salt and
one-eighth teaspoon pepper when fowl is about half cooked. Thicken stock
with one-third cup flour diluted with enough cold water to pour easily.
Serve with Dumplings. If desired richer, butter may be added.



\needspace{15\baselineskip}
\section*{Chicken Pie}

Dress, clean, and cut up two fowls or chickens. Put in a stewpan with
one-half onion, sprig of parsley, and bit of bay leaf; cover with
boiling water, and cook slowly until tender. When chicken is half
cooked, add one-half tablespoon salt and one-eighth teaspoon pepper.
Remove chicken, strain stock, skim off fat, and then cook until reduced
to four cups. Thicken stock with one-third cup flour diluted with enough
cold water to pour easily. Place a small cup in centre of baking-dish,
arrange around it pieces of chicken, removing some of the larger bones;
pour over gravy, and cool. Cover with pie-crust in which several
incisions have been made, that there may be an outlet for escape of
steam and gases. Wet edge of crust and put around a rim, having rim come
close to edge. Bake in a moderate oven until crust is well risen and
browned. Roll remnants of pastry and cut in diamond-shaped pieces, bake,
and serve with pie when reheated. If puff paste is used, it is best to
bake top separately.



\needspace{15\baselineskip}
\section*{Chicken Curry}


\begin{minipage}{1.0\textwidth}
{\setlength{\multicolsep}{0pt}\setlength{\columnsep}{2em}\raggedcolumns%
\begin{multicols}{2}
\begin{itemize}
\setlength{\itemsep}{0pt}
\setlength{\parsep}{0pt}
\item 3 lb. chicken
\item 1/3 cup butter
\item 2 onions
\item 1 tablespoon curry powder
\item 2 teaspoons salt
\item 1 teaspoon vinegar
\end{itemize}
\end{multicols}}
\end{minipage}

\vspace{0.3em}
\noindent%
Clean, dress, and cut chicken in pieces for serving. Put butter in a hot
frying-pan, add chicken, and cook ten minutes; then add liver and
gizzard and cook ten minutes longer. Cut onions in thin slices, and add
to chicken with curry powder and salt. Add enough boiling water to
cover, and simmer until chicken is tender. Remove chicken; strain, and
thicken liquor with flour diluted with enough cold water to pour easily.
Pour gravy over chicken, and serve with a border of rice or Turkish
Pilaf.



\needspace{15\baselineskip}
\section*{Chicken En Casserole}

Cut two small, young chickens in pieces for serving. Season with salt
and pepper, brush over with melted butter, and bake in a casserole dish
twelve minutes. Parboil one-third cup carrots cut in strips five
minutes, drain, and fry with one tablespoon finely chopped onion and
four thin slices bacon cut in narrow strips. Add one and one-third cups
Brown Sauce and two-thirds cup potato balls. Add to chicken, with three
tablespoons Sherry wine, salt and pepper to taste. Cook in a moderate
oven twenty minutes, or until chicken is tender. If small casserole
dishes are used allow but one chicken to each dish.



\needspace{15\baselineskip}
\section*{Breslin Potted Chicken}

Dress, clean, and truss a broiler. Put in a casserole dish, brush over
with two and one-half tablespoons melted butter, put on cover, and bake
twenty minutes; then add one cup stock and cook until chicken is tender.
Thicken stock with one tablespoon, each, butter and flour cooked
together, and add one-half cup cooked potato balls, one-third cup canned
string beans, cut in small pieces, one-third cup cooked carrot, cut in
fancy shapes, and six sautéd mushroom caps.



\needspace{15\baselineskip}
\section*{Jellied Chicken}

Dress, clean, and cut up a four-pound fowl. Put in a stewpan with two
slices onion, cover with boiling water, and cook slowly until meat falls
from bones. When half cooked, add one-half tablespoon salt. Remove
chicken; reduce stock to three-fourths cup, strain, and skim off fat.
Decorate bottom of a mould with parsley and slices of hard-boiled eggs.
Pack in meat freed from skin and bone and sprinkled with salt and
pepper. Pour on stock and place mould under heavy weight. Keep in a cold
place until firm. In summer it is necessary to add one teaspoon
dissolved granulated gelatine to stock.



\needspace{15\baselineskip}
\section*{Chickens' Livers With Madeira Sauce}

Clean and separate livers, sprinkle with salt and pepper, dredge with
flour, and sauté in butter. Brown two tablespoons butter, add two and
one-half tablespoons flour, and when well browned add gradually one cup
Brown Stock; then add two tablespoons Madeira wine, and reheat livers in
sauce.

                  Chickens' Livers with Bacon

Clean livers and cut each liver in six pieces. Wrap a thin slice of
bacon around each piece and fasten with a small skewer. Put in a
broiler, place over a dripping-pan, and bake in a hot oven until bacon
is crisp, turning once during cooking.



\needspace{15\baselineskip}
\section*{Sautéd Chickens' Livers}

Cut one slice bacon in small pieces and cook five minutes with two
tablespoons butter. Remove bacon, add one finely chopped shallot, and
fry two minutes; then add six chickens' livers cleaned and separated,
and cook two minutes. Add two tablespoons flour, one cup Brown Stock,
one teaspoon lemon juice, and one-fourth cup sliced mushrooms. Cook two
minutes, turn into a serving dish, and sprinkle with finely chopped
parsley.



\needspace{15\baselineskip}
\section*{Chickens' Livers With Curry}

Clean and separate livers. Dip in seasoned crumbs, egg, and crumbs, and
sauté in butter. Remove livers, and to fat in pan add two tablespoons
butter, one-half tablespoon finely chopped onion, and cook five minutes.
Add two tablespoons flour mixed with one-half teaspoon curry powder and
one cup stock. Strain sauce over livers, and serve around livers Rice
Timbales.



\needspace{15\baselineskip}
\section*{Boiled Turkey}

Prepare and cook same as Boiled Fowl. Serve with Oyster or Celery Sauce.



\needspace{15\baselineskip}
\section*{Roast Turkey}

Dress, clean, stuff, and truss a ten-pound turkey (see pages 242--244).
Place on its side on rack in a dripping-pan, rub entire surface with
salt, and spread breast, legs, and wings with one-third cup butter,
rubbed until creamy and mixed with one-fourth cup flour. Dredge bottom
of pan with flour. Place in a hot oven, and when flour on turkey begins
to brown, reduce heat, baste with fat in pan, and add two cups boiling
water. Continue basting every fifteen minutes until turkey is cooked,
which will require about three hours. For basting, use one-half cup
butter melted in one-half cup boiling water, and after this is used
baste with fat in pan. During cooking turn turkey frequently, that it
may brown evenly. If turkey is browning too fast, cover with buttered
paper to prevent burning. Remove string and skewers before serving.
Garnish with parsley, or celery tips, or curled celery and rings and
discs of carrots strung on fine wire.

For stuffing, use double the quantities given in recipes under Roast
Chicken. If stuffing is to be served cold, add one beaten egg. Turkey is
often roasted with Chestnut Stuffing, Oyster Stuffing, or Turkey
Stuffing (Swedish Style).



\needspace{15\baselineskip}
\section*{Chestnut Stuffing}


\begin{minipage}{1.0\textwidth}
{\setlength{\multicolsep}{0pt}\setlength{\columnsep}{2em}\raggedcolumns%
\begin{multicols}{2}
\begin{itemize}
\setlength{\itemsep}{0pt}
\setlength{\parsep}{0pt}
\item 3 cups French chestnuts
\item 1/2 cup butter
\item 1 teaspoon salt
\item 1/8 teaspoon pepper
\item 1/4 cup cream
\item 1 cup cracker crumbs
\end{itemize}
\end{multicols}}
\end{minipage}

\vspace{0.3em}
\noindent%
Shell and blanch chestnuts. Cook in boiling salted water until soft.
Drain and mash, using a potato ricer. Add one-half the butter, salt,
pepper, and cream. Melt remaining butter, mix with cracker crumbs, then
combine mixtures.



\needspace{15\baselineskip}
\section*{Oyster Stuffing}


\begin{itemize}
\setlength{\itemsep}{0pt}
\setlength{\parsep}{0pt}
\item 3 cups stale bread crumbs
\item 1/2 cup melted butter
\item Salt and pepper
\item Few drops onion juice
\item 1 pint oysters
\end{itemize}

\vspace{-0.5em}
\noindent%
Mix ingredients in the order given, add oysters, cleaned and drained
from their liquor.



\needspace{15\baselineskip}
\section*{Turkey Stuffing (Swedish Style)}


\begin{minipage}{1.0\textwidth}
{\setlength{\multicolsep}{0pt}\setlength{\columnsep}{2em}\raggedcolumns%
\begin{multicols}{2}
\begin{itemize}
\setlength{\itemsep}{0pt}
\setlength{\parsep}{0pt}
\item 2 cups stale bread crumbs
\item 2/3 cup melted butter
\item 1/2 cup raisins, seeded and cut in pieces
\item 1/2 cup English walnut meats, broken in pieces
\item Salt and pepper
\item Sage
\end{itemize}
\end{multicols}}
\end{minipage}

\vspace{0.3em}
\noindent%
Mix ingredients in the order given.



\needspace{15\baselineskip}
\section*{Gravy}

Pour off liquid in pan in which turkey has been roasted. From liquid
skim off six tablespoons fat; return fat to pan and brown with six
tablespoons flour; pour on gradually three cups stock in which giblets,
neck, and tips of wings have been cooked, or use liquor left in pan.
Cook five minutes, season with salt and pepper; strain. For Giblet Gravy
add to the above, giblets (heart, liver, and gizzard) finely chopped.



\needspace{15\baselineskip}
\section*{Chestnut Gravy}

To two cups thin Turkey Gravy add three-fourths cup cooked and mashed
chestnuts.



\needspace{15\baselineskip}
\section*{To Carve Turkey}

Bird should be placed on back, with legs at right of platter for
carving. Introduce carving fork across breastbone, hold firmly in left
hand, and with carving knife in right hand cut through skin between leg
and body, close to body. With knife pull back leg and disjoint from
body. Then cut off wing. Remove leg and wing from other side. Separate
second joints from drumsticks and divide wings at joints. Carve breast
meat in thin crosswise slices. Under back on either side of backbone may
be found two small, oyster-shaped pieces of dark meat, which are dainty
tidbits. Chicken and fowl are carved in the same way. For a small family
carve but one side of a turkey, that remainder may be left in better
condition for second serving.



\needspace{15\baselineskip}
\section*{Roast Goose With Potato Stuffing}

Singe, remove pinfeathers, wash and scrub a goose in hot soapsuds; then
draw (which is removing inside contents). Wash in cold water and wipe.
Stuff, truss, sprinkle with salt and pepper, and lay six thin strips fat
salt pork over breast. Place on rack in dripping-pan, put in hot oven,
and bake two hours. Baste every fifteen minutes with fat in pan. Remove
pork last half-hour of cooking. Place on platter, cut string, and remove
string and skewers. Garnish with watercress and bright red cranberries.
Serve with Apple Sauce.



\needspace{15\baselineskip}
\section*{Potato Stuffing}


\begin{minipage}{1.0\textwidth}
{\setlength{\multicolsep}{0pt}\setlength{\columnsep}{2em}\raggedcolumns%
\begin{multicols}{2}
\begin{itemize}
\setlength{\itemsep}{0pt}
\setlength{\parsep}{0pt}
\item 2 cups hot mashed potato
\item 1 1/4 cups soft stale bread crumbs
\item 1/4 cup finely chopped fat salt pork
\item 1 finely chopped onion
\item 1/3 cup butter
\item 1 egg
\item 1 1/2 teaspoons salt
\item 1 teaspoon sage
\end{itemize}
\end{multicols}}
\end{minipage}

\vspace{0.3em}
\noindent%
Add to potato, bread crumbs, butter, egg, salt, and sage; then add pork
and onion.



\needspace{15\baselineskip}
\section*{Goose Stuffing (Chestnut)}


\begin{minipage}{1.0\textwidth}
{\setlength{\multicolsep}{0pt}\setlength{\columnsep}{2em}\raggedcolumns%
\begin{multicols}{2}
\begin{itemize}
\setlength{\itemsep}{0pt}
\setlength{\parsep}{0pt}
\item 1/2 tablespoon finely chopped shallot
\item 3 tablespoons butter
\item 1/4 lb. sausage meat
\item 12 canned mushrooms, finely chopped
\item 1 cup chestnut purée
\item 1/3 cup stale bread crumbs
\item 1/2 tablespoon finely chopped parsley
\item 24 French chestnuts cooked and left whole
\item Salt and pepper
\end{itemize}
\end{multicols}}
\end{minipage}

\vspace{0.3em}
\noindent%
Cook shallot with butter five minutes, add sausage meat, and cook two
minutes, then add mushrooms, chestnut purée, parsley, and salt and
pepper. Heat to boiling-point, add bread crumbs and whole chestnuts.
Cool mixture before stuffing goose.







\needspace{15\baselineskip}
\section*{To Truss A Goose}

A goose, having short legs, is trussed differently from chicken, fowl,
and turkey. After inserting skewers, wind string twice around one leg
bone, then around other leg bone, having one inch space of string
between legs. Draw legs with both ends of string close to back, cross
string under back, then fasten around skewers and tie in a knot.



\needspace{15\baselineskip}
\section*{Roast Wild Duck}

Dress and clean a wild duck and truss as goose. Place on rack in
dripping-pan, sprinkle with salt and pepper, and cover breast with two
very thin slices fat salt pork. Bake twenty to thirty minutes in a very
hot oven, basting every five minutes with fat in pan; cut string and
remove string and skewers. Serve with Orange or Olive Sauce. Currant
jelly should accompany a duck course. Domestic ducks should always be
well cooked, requiring little more than twice the time allowed for wild
ducks.

Ducks are sometimes stuffed with apples, pared, cored, and cut in
quarters, or three small onions may be put in body of duck to improve
flavor. Neither apples nor onions are to be served. If a stuffing to be
eaten is desired, cover pieces of dry bread with boiling water; as soon
as bread has absorbed water, press out the water; season bread with
salt, pepper, melted butter, finely chopped onion, or use



\needspace{15\baselineskip}
\section*{Duck Stuffing (Peanut)}


\begin{minipage}{1.0\textwidth}
{\setlength{\multicolsep}{0pt}\setlength{\columnsep}{2em}\raggedcolumns%
\begin{multicols}{2}
\begin{itemize}
\setlength{\itemsep}{0pt}
\setlength{\parsep}{0pt}
\item 3/4 cup cracker crumbs
\item 1/2 cup shelled peanuts, finely chopped
\item 1/2 cup heavy cream
\item 2 tablespoons butter
\item Few drops onion juice
\item Salt and pepper
\item Cayenne
\end{itemize}
\end{multicols}}
\end{minipage}

\vspace{0.3em}
\noindent%
Mix ingredients in the order given.



\needspace{15\baselineskip}
\section*{Braised Duck}

Tough ducks are sometimes steamed one hour, and then braised in same
manner as chicken.



\needspace{15\baselineskip}
\section*{Broiled Quail}

Follow recipe for Broiling Chicken, allowing eight minutes for cooking.
Serve on pieces of toast, and garnish with parsley and thin slices of
lemon. Currant jelly or Rice Croquettes with Jelly should accompany this
course.



\needspace{15\baselineskip}
\section*{Roast Quail}

Dress, clean, lard, and truss a quail. Bake same as Larded Grouse,
allowing fifteen to twenty minutes for cooking.



\needspace{15\baselineskip}
\section*{Larded Grouse}

Clean, remove pinions, and if it be tough the skin covering breast. Lard
breast and insert two lardoons in each leg. Truss, and place on trivet
in small shallow pan; rub with salt, brush over with melted butter,
dredge with flour, and surround with trimmings of fat salt pork. Bake
twenty to twenty-five minutes in a hot oven, basting three times.
Arrange on platter, remove string and skewers, pour around Bread Sauce,
and sprinkle bird and sauce with coarse brown bread crumbs. Garnish with
parsley.



\needspace{15\baselineskip}
\section*{Breast Of Grouse Sauté Chasseur}

Remove breasts from pair of grouse, and sauté in butter. When partially
cooked, season with salt and pepper. Break carcasses in pieces, cover
with cold water, add carrot, celery, onion, parsley, and bay leaf, and
cook until stock is reduced to three-fourths cup. Arrange grouse on a
serving dish, and pour around a sauce made of three tablespoons butter,
four and one-half tablespoons flour, stock made from grouse, and
three-fourths cup stewed and strained tomatoes. Season with salt,
cayenne, and lemon juice, and add one teaspoon finely chopped parsley,
and one-half cup canned mushrooms cut in slices.



\needspace{15\baselineskip}
\section*{Broiled Or Roasted Plover}

Plover is broiled or roasted same as quail.



\needspace{15\baselineskip}
\section*{Potted Pigeons}

Clean, stuff, and truss six pigeons, place upright in a stewpan, and add
one quart boiling water in which celery has been cooked. Cover, and cook
slowly three hours or until tender; or cook in oven in a covered earthen
dish. Remove from water, cool slightly, sprinkle with salt and pepper,
dredge with flour, and brown entire surface in pork fat. Make a sauce
with one-fourth cup, each, butter and flour cooked together and stock
remaining in pan; there should be two cups. Place each bird on a slice
of dry toast, and pour gravy over all. Garnish with parsley.



\needspace{15\baselineskip}
\section*{Stuffing}


\begin{minipage}{1.0\textwidth}
{\setlength{\multicolsep}{0pt}\setlength{\columnsep}{2em}\raggedcolumns%
\begin{multicols}{2}
\begin{itemize}
\setlength{\itemsep}{0pt}
\setlength{\parsep}{0pt}
\item 1 cup hot riced potatoes
\item 1/4 teaspoon salt
\item 1/8 teaspoon pepper
\item 1/4 teaspoon marjoram or summer savory
\item Few drops onion juice
\item 1 tablespoon butter
\item 1/4 cup soft stale bread crumbs soaked in some of the celery water and
\item wrung in cheese-cloth
\item Yolk 1 egg
\end{itemize}
\end{multicols}}
\end{minipage}

\vspace{0.3em}
\noindent%
Mix ingredients in order given.



\needspace{15\baselineskip}
\section*{Broiled Venison Steak}

Follow recipe for Broiled Beefsteak. Serve with Maître d'Hôtel Butter.
Venison should always be cooked rare.



\needspace{15\baselineskip}
\section*{Venison Steaks, Sautéd, Cumberland Sauce}

Cut venison steaks in circular pieces and use trimmings for the making
of stock. Sauté steaks in hot buttered frying-pan and serve with

\textbf{Cumberland Sauce.} Soak two tablespoons citron, cut in julienne-shaped
pieces, two tablespoons glacéd cherries, and one tablespoon Sultana
raisins, in Port wine for several hours. Drain and cook fruit five
minutes in one-third cup Port wine. Add one-half tumbler currant jelly,
and, as soon as jelly is dissolved, add one and one-third cups Brown
Sauce, and two tablespoons shredded almonds.



\needspace{15\baselineskip}
\section*{Venison Steak, Chestnut Sauce}

Wipe steak, sprinkle with salt and pepper, place on a greased broiler,
and broil five minutes. Remove to hot platter and pour over

\textbf{Chestnut Sauce.} Fry one-half onion and six slices carrot, cut in small
pieces, in two tablespoons butter, five minutes, add three tablespoons
flour, and stir until well browned; then add one and one-half cups Brown
Stock, a sprig of parsley, a bit of bay leaf, eight peppercorns, and one
teaspoon salt. Let simmer twenty minutes, strain, then add three
tablespoons Madeira wine, one cup boiled French chestnuts, and one
tablespoon butter.



\needspace{15\baselineskip}
\section*{Venison Cutlets}

Clean and trim slices of venison cut from loin. Sprinkle with salt and
pepper, brush over with melted butter or olive oil, and roll in soft
stale bread crumbs. Place in a broiler and broil five minutes, or sauté
in butter. Serve with Port Wine Sauce.



\needspace{15\baselineskip}
\section*{Roast Leg Of Venison}

Prepare and cook as Roast Lamb, allowing less time that it may be cooked
rare.



\needspace{15\baselineskip}
\section*{Saddle Of Venison}

Clean and lard a saddle of venison. Cook same as Saddle of Mutton. Serve
with Currant Jelly Sauce.



\needspace{15\baselineskip}
\section*{Belgian Hare À La Maryland}

Follow directions for Chicken à la Maryland (see p. 249). Bake forty
minutes, basting with bacon fat in place of butter.



\needspace{15\baselineskip}
\section*{Belgian Hare, Sour Cream Sauce}

Clean and split a hare. Lard back and hind legs, and season with salt
and pepper. Cook eight slices carrot cut in small pieces and one-half
small onion in two tablespoons bacon fat five minutes. Add one cup Brown
Stock, and pour around hare in pan. Bake forty-five minutes, basting
often. Add one cup heavy cream and the juice of one lemon. Cook fifteen
minutes longer, and baste every five minutes. Remove to serving dish,
strain sauce, thicken, season with salt and pepper, and pour around
hare.



\needspace{15\baselineskip}
\section*{Ways Of Warming Over Poultry And Game}


\needspace{15\baselineskip}
\subsection*{Creamed Chicken}


\begin{itemize}
\setlength{\itemsep}{0pt}
\setlength{\parsep}{0pt}
\item 1 1/2 cups cold cooked chicken, cut in dice 
\item 1 cup White Sauce II 
\item 1/8 teaspoon celery salt. 
\end{itemize}

\vspace{-0.5em}
\noindent%
Heat chicken dice in sauce, to which celery salt has been added.



\needspace{15\baselineskip}
\subsection*{Creamed Chicken with Mushrooms}

Add to Creamed Chicken one-fourth cup mushrooms cut in slices.



\needspace{15\baselineskip}
\subsection*{Chicken with Potato Border}

Serve Creamed Chicken in Potato Border.



\needspace{15\baselineskip}
\subsection*{Chicken in Baskets}

To three cups hot mashed potatoes add three tablespoons butter, one
teaspoon salt, yolks of three eggs slightly beaten, and enough milk to
moisten. Shape in form of small baskets, using a pastry bag and tube.
Brush over with white of egg slightly beaten, and brown in oven. Fill
with Creamed Chicken. Form handles for baskets of parsley.



\needspace{15\baselineskip}
\subsection*{Chicken and Oysters à la Métropole}


\begin{minipage}{1.0\textwidth}
{\setlength{\multicolsep}{0pt}\setlength{\columnsep}{2em}\raggedcolumns%
\begin{multicols}{2}
\begin{itemize}
\setlength{\itemsep}{0pt}
\setlength{\parsep}{0pt}
\item 1/4 cup butter
\item 1/4 cup flour
\item 1/2 teaspoon salt
\item 1/8 teaspoon pepper
\item 2 cups cream
\item 2 cups cold cooked chicken, cut in dice
\item 1 pint oysters, cleaned and  rained
\item 1/3 cup finely chopped celery
\end{itemize}
\end{multicols}}
\end{minipage}

\vspace{0.3em}
\noindent%
Make a sauce of first five ingredients, add chicken dice and oysters;
cook until oysters are plump. Serve sprinkled with celery.



\needspace{15\baselineskip}
\subsection*{Luncheon Chicken}


\begin{minipage}{1.0\textwidth}
{\setlength{\multicolsep}{0pt}\setlength{\columnsep}{2em}\raggedcolumns%
\begin{multicols}{2}
\begin{itemize}
\setlength{\itemsep}{0pt}
\setlength{\parsep}{0pt}
\item 1 1/2 cups cold cooked chicken, cut in small dice
\item 2 tablespoons butter
\item 1 slice carrot, cut in small cubes
\item 1 slice onion
\item 2 tablespoons flour
\item 1 cup Chicken Stock
\item Salt
\item Pepper
\item 2/3 cup buttered cracker crumbs
\item 4 eggs
\end{itemize}
\end{multicols}}
\end{minipage}

\vspace{0.3em}
\noindent%
Cook butter five minutes with vegetables, add flour, and gradually the
stock. Strain, add chicken dice, and season with salt and pepper. Turn
on a slightly buttered platter and sprinkle with cracker crumbs. Make
four nests, and in each nest slip an egg; cover eggs with crumbs, and
bake in a moderate oven until whites of eggs are firm.



\needspace{15\baselineskip}
\subsection*{Blanquette of Chicken}


\begin{itemize}
\setlength{\itemsep}{0pt}
\setlength{\parsep}{0pt}
\item 2 cups cold cooked chicken, cut in strips
\item 1 cup White Sauce II
\item 1 tablespoon finely chopped parsley
\item 2 egg yolks
\item 2 tablespoons milk
\end{itemize}

\vspace{-0.5em}
\noindent%
Add chicken to sauce; when well heated, add yolks of eggs slightly
beaten, diluted with milk. Cook two minutes, then add parsley.



\needspace{15\baselineskip}
\subsection*{Scalloped Chicken}

Butter a baking-dish. Arrange alternate layers of cold, cooked sliced
chicken and boiled macaroni or rice. Pour over White, Brown, or Tomato
Sauce, cover with buttered cracker crumbs, and bake in a hot oven until
crumbs are brown.



\needspace{15\baselineskip}
\subsection*{Mock Terrapin}


\begin{minipage}{1.0\textwidth}
{\setlength{\multicolsep}{0pt}\setlength{\columnsep}{2em}\raggedcolumns%
\begin{multicols}{2}
\begin{itemize}
\setlength{\itemsep}{0pt}
\setlength{\parsep}{0pt}
\item 1 1/2 cups cold cooked chicken or veal, cut in dice
\item 1 cup White Sauce I
\item Yolks 2 “hard-boiled” eggs, finely chopped
\item Whites 2 “hard-boiled” eggs, chopped
\item 3 tablespoons Sherry wine
\item 1/4 teaspoon salt
\item Few grains cayenne
\end{itemize}
\end{multicols}}
\end{minipage}

\vspace{0.3em}
\noindent%
Add to sauce, chicken, yolks and whites of eggs, salt, and cayenne; cook
two minutes, and add wine.



\needspace{15\baselineskip}
\subsection*{Chicken Soufflé}


\begin{minipage}{1.0\textwidth}
{\setlength{\multicolsep}{0pt}\setlength{\columnsep}{2em}\raggedcolumns%
\begin{multicols}{2}
\begin{itemize}
\setlength{\itemsep}{0pt}
\setlength{\parsep}{0pt}
\item 2 cups scalded milk
\item 1/8 cup butter
\item 1/8 cup flour
\item 1 teaspoon salt
\item 1/8 teaspoon pepper
\item 1/2 cup stale soft bread crumbs
\item 2 cups cold cooked chicken, finely chopped
\item 3 egg yolks, well beaten
\item 1 tablespoon finely chopped parsley
\item 3 egg whites, beaten stiff
\end{itemize}
\end{multicols}}
\end{minipage}

\vspace{0.3em}
\noindent%
Make a sauce of first five ingredients, add bread crumbs, and cook two
minutes; remove from fire, add chicken, yolks of eggs, and parsley, then
fold in whites of eggs. Turn in a buttered pudding-dish, and bake
thirty-five minutes in a slow oven. Serve with White Mushroom Sauce.
Veal may be used in place of chicken.



\needspace{15\baselineskip}
\subsection*{Chicken Hollandaise}


\begin{minipage}{1.0\textwidth}
{\setlength{\multicolsep}{0pt}\setlength{\columnsep}{2em}\raggedcolumns%
\begin{multicols}{2}
\begin{itemize}
\setlength{\itemsep}{0pt}
\setlength{\parsep}{0pt}
\item 1 1/2 tablespoons butter
\item 1 teaspoon finely chopped onion
\item 2 tablespoons corn-starch
\item 1 cup chicken stock
\item 1 teaspoon lemon juice
\item 1/3 cup finely chopped celery
\item 1/4 teaspoon salt
\item Few grains paprika
\item 1 cup cold cooked chicken, cut in small cubes
\item Yolk 1 egg
\end{itemize}
\end{multicols}}
\end{minipage}

\vspace{0.3em}
\noindent%
Cook butter and onion five minutes, add corn-starch and stock gradually.
Add lemon juice, celery, salt, paprika, and chicken; when well heated,
add yolk of egg slightly beaten, and cook one minute. Serve with
buttered Graham toast.



\needspace{15\baselineskip}
\subsection*{Chicken Chartreuse}

Prepare and cook same as Casserole of Rice and Meat, using chicken in
place of lamb or veal. Season chicken with salt, pepper, celery salt,
onion juice, and one-half teaspoon finely chopped parsley.



\needspace{15\baselineskip}
\subsection*{Scalloped Turkey}

Make one cup of sauce, using two tablespoons butter, two tablespoons
flour, one-fourth teaspoon salt, few grains of pepper, and one cup stock
(obtained by cooking in water bones and skin of a roast turkey). Cut
remnants of cold roast turkey in small pieces; there should be one and
one-half cups. Sprinkle bottom of buttered baking-dish with seasoned
cracker crumbs, add turkey meat, pour over sauce, and sprinkle with
buttered cracker crumbs. Bake in a hot oven until crumbs are brown.
Turkey, chicken, or veal may be used separately or in combination.



\needspace{15\baselineskip}
\subsection*{Minced Turkey}

To one cup cold roast turkey, cut in small dice, add one-third cup soft
stale bread crumbs. Make one cup sauce, using two tablespoons butter,
two tablespoons flour, and one cup stock (obtained by cooking bones and
skin of a roast turkey). Season with salt, pepper, and onion juice. Heat
turkey and bread crumbs in sauce. Serve on small pieces of toast, and
garnish with poached eggs and toast points.



\needspace{15\baselineskip}
\subsection*{Salmi of Duck}

Cut cold roast duck in pieces for serving. Reheat in Spanish Sauce.

\textbf{Spanish Sauce.} Melt one-fourth cup butter, add one tablespoon finely
chopped onion, a stalk of celery, two slices carrot cut in pieces, and
two tablespoons finely chopped lean raw ham. Cook until butter is brown,
then add one-fourth cup flour, and when well browned add two cups
Consommé, bit of bay leaf, sprig of parsley, blade of mace, two cloves,
one-half teaspoon salt, and one-eighth teaspoon pepper; cook five
minutes. Strain, add duck, and when reheated add Sherry wine, stoned
olives, and mushrooms cut in quarters. Arrange on dish for serving, and
garnish with olives and mushrooms. Grouse may be used in place of duck.





\chapter{Fish And Meat Sauces}



The French chef keeps always on hand four sauces,--White, Brown,
Béchamel, and Tomato,--and with these as a basis is able to make kinds
innumerable. Butter and flour are usually cooked together for thickening
sauces. When not browned, it is called \textit{roux}; when browned, \_brown
roux\_. The French mix butter and flour together, put in saucepan, place
over fire, stir for five minutes; set aside to cool, again place over
fire, and add liquid, stirring constantly until thick and smooth. Butter
and flour for brown sauces are cooked together much longer, and watched
carefully lest butter should burn. The American cook makes sauce by
stirring butter in saucepan until melted and bubbling, adds flour and
continues stirring, then adds liquid, gradually stirring or beating
until the boiling-point is reached. For Brown Sauce, butter should be
stirred until well browned; flour should be added and stirred with
butter until both are browned before the addition of liquid. The secret
in making a Brown Sauce is to have butter and flour well browned before
adding liquid.

It is well worth remembering that a sauce of average thickness is made
by allowing two tablespoons each of butter and flour to one cup liquid,
whether it be milk, stock, or tomato. For Brown Sauce a slightly larger
quantity of flour is necessary, as by browning flour its thickening
property is lessened, its starch being changed to dextrine. When sauces
are set away, put a few bits of butter on top to prevent crust from
forming.



\needspace{15\baselineskip}
\section*{Thin White Sauce}


\begin{itemize}
\setlength{\itemsep}{0pt}
\setlength{\parsep}{0pt}
\item 2 tablespoons butter
\item 1 1/2 tablespoons flour
\item 1 cup scalded milk
\item 1/4 teaspoon salt
\item Few grains pepper
\end{itemize}

\vspace{-0.5em}
\noindent%
Put butter in saucepan, stir until melted and bubbling; add flour mixed
with seasonings, and stir until thoroughly blended. Pour on gradually
the milk, adding about one-third at a time, stirring until well mixed,
then beating until smooth and glossy. If a wire whisk is used, all the
milk may be added at once.



\needspace{15\baselineskip}
\section*{Cream Sauce}

Make same as Thin White Sauce, using cream instead of milk.



\needspace{15\baselineskip}
\section*{White Sauce I}


\begin{itemize}
\setlength{\itemsep}{0pt}
\setlength{\parsep}{0pt}
\item 2 tablespoons butter
\item 2 tablespoons flour
\item 1 cup milk
\item 1/4 teaspoon salt
\item Few grains pepper
\end{itemize}

\vspace{-0.5em}
\noindent%
Make same as Thin White Sauce.



\needspace{15\baselineskip}
\section*{White Sauce II}


\begin{itemize}
\setlength{\itemsep}{0pt}
\setlength{\parsep}{0pt}
\item 2 tablespoons butter
\item 3 tablespoons flour
\item 1 cup milk
\item 1/4 teaspoon salt
\item Few grains pepper
\end{itemize}

\vspace{-0.5em}
\noindent%
Make same as Thin White Sauce.



\needspace{15\baselineskip}
\section*{Thick White Sauce (For Cutlets And Croquets)}


\begin{minipage}{1.0\textwidth}
{\setlength{\multicolsep}{0pt}\setlength{\columnsep}{2em}\raggedcolumns%
\begin{multicols}{2}
\begin{itemize}
\setlength{\itemsep}{0pt}
\setlength{\parsep}{0pt}
\item 2 1/2 tablespoons butter
\item 1/4 cup corn-starch or
\item 1/3 cup flour
\item 1 cup milk
\item 1/4 teaspoon salt
\item Few grains pepper
\end{itemize}
\end{multicols}}
\end{minipage}

\vspace{0.3em}
\noindent%
Make same as Thin White Sauce.



\needspace{15\baselineskip}
\section*{Velouté Sauce}


\begin{itemize}
\setlength{\itemsep}{0pt}
\setlength{\parsep}{0pt}
\item 2 tablespoons butter
\item 2 tablespoons flour
\item 1 cup White Stock
\item 1/4 teaspoon salt
\item Few grains pepper
\end{itemize}

\vspace{-0.5em}
\noindent%
Make same as Thin White Sauce.



\needspace{15\baselineskip}
\section*{Sauce Allemande}

To Velouté Sauce add one teaspoon lemon juice and yolk one egg.



\needspace{15\baselineskip}
\section*{Soubise Sauce}


\begin{itemize}
\setlength{\itemsep}{0pt}
\setlength{\parsep}{0pt}
\item 2 cups sliced onions
\item 1 cup Velouté Sauce
\item 1/2 cup cream or milk
\item Salt and pepper
\end{itemize}

\vspace{-0.5em}
\noindent%
Cover onions with boiling water, cook five minutes, drain, again cover
with boiling water, and cook until soft; drain, and rub through a sieve.
Add to sauce with cream. Season with salt and pepper. Serve with mutton,
pork chops, or “hard-boiled” eggs.



\needspace{15\baselineskip}
\section*{Drawn Butter Sauce}


\begin{itemize}
\setlength{\itemsep}{0pt}
\setlength{\parsep}{0pt}
\item 1/3 cup butter
\item 3 tablespoons flour
\item 1 1/2 cups hot water
\item 1/2 teaspoon salt
\item 1/8 teaspoon pepper
\end{itemize}

\vspace{-0.5em}
\noindent%
Melt one-half the butter, add flour with seasonings, and pour on
gradually hot water. Boil five minutes, and add remaining butter in
small pieces. To be served with boiled or baked fish.



\needspace{15\baselineskip}
\section*{Shrimp Sauce}

To Drawn Butter Sauce add one egg yolk and one-half can shrimps cleaned
and cut in pieces.



\needspace{15\baselineskip}
\section*{Caper Sauce}

To Drawn Butter Sauce add one-half cup capers drained from their liquor.
Serve with boiled mutton.



\needspace{15\baselineskip}
\section*{Egg Sauce I}

To Drawn Butter Sauce add two “hard-boiled” eggs cut in one-fourth inch
slices.



\needspace{15\baselineskip}
\section*{Egg Sauce II}

To Drawn Butter Sauce add beaten yolks of two eggs and one teaspoon
lemon juice.



\needspace{15\baselineskip}
\section*{Brown Sauce I}


\begin{minipage}{1.0\textwidth}
{\setlength{\multicolsep}{0pt}\setlength{\columnsep}{2em}\raggedcolumns%
\begin{multicols}{2}
\begin{itemize}
\setlength{\itemsep}{0pt}
\setlength{\parsep}{0pt}
\item 2 tablespoons butter
\item 1/2 slice onion
\item 3 tablespoons flour
\item 1 cup Brown Stock
\item 1/4 teaspoon salt
\item 1/8 teaspoon pepper
\end{itemize}
\end{multicols}}
\end{minipage}

\vspace{0.3em}
\noindent%
Cook onion in butter until slightly browned; remove onion and stir
butter constantly until well browned; add flour mixed with seasonings,
and brown the butter and flour; then add stock gradually.



\needspace{15\baselineskip}
\section*{Brown Sauce II (Espagnole)}


\begin{minipage}{1.0\textwidth}
{\setlength{\multicolsep}{0pt}\setlength{\columnsep}{2em}\raggedcolumns%
\begin{multicols}{2}
\begin{itemize}
\setlength{\itemsep}{0pt}
\setlength{\parsep}{0pt}
\item 1/4 cup butter
\item 1 slice carrot
\item 1 slice onion
\item Bit of bay leaf
\item Sprig of thyme
\item Sprig of parsley
\item 6 peppercorns
\item 5 tablespoons flour
\item 2 cups Brown Stock
\item Salt and pepper
\end{itemize}
\end{multicols}}
\end{minipage}

\vspace{0.3em}
\noindent%
Cook butter with carrot, onion, bay leaf, thyme, parsley, and
peppercorns, until brown, stirring constantly, care being taken that
butter is not allowed to burn; add flour, and when well browned, add
stock gradually. Bring to boiling-point, strain, and season with salt
and pepper.



\needspace{15\baselineskip}
\section*{Brown Mushroom Sauce I}

To one cup Brown Sauce add one-fourth can mushrooms, drained, rinsed,
and cut in quarters or slices.



\needspace{15\baselineskip}
\section*{Brown Mushroom Sauce II}


\begin{minipage}{1.0\textwidth}
{\setlength{\multicolsep}{0pt}\setlength{\columnsep}{2em}\raggedcolumns%
\begin{multicols}{2}
\begin{itemize}
\setlength{\itemsep}{0pt}
\setlength{\parsep}{0pt}
\item 1 can mushrooms
\item 1/4 cup butter
\item 1/2 tablespoon lemon juice
\item 1/4 cup flour
\item 2 cups Consommé or Brown Stock
\item Salt and pepper
\end{itemize}
\end{multicols}}
\end{minipage}

\vspace{0.3em}
\noindent%
Drain and rinse mushrooms and chop finely one-half of same. Cook five
minutes with butter and lemon juice; drain; brown the butter, add flour,
and when well browned, add gradually Consommé. Cook fifteen minutes,
skim, add remaining mushrooms cut in quarters or slices, and cook two
minutes. Season with salt and pepper. Use fresh mushrooms in place of
canned ones when possible.



\needspace{15\baselineskip}
\section*{Sauce Piquante}

To one cup Brown Sauce add one tablespoon vinegar, one-half small
shallot finely chopped, one tablespoon each chopped capers and pickle,
and a few grains of cayenne.



\needspace{15\baselineskip}
\section*{Olive Sauce}

Remove stones from ten olives, leaving meat in one piece. Cover with
boiling water and cook five minutes. Drain olives, and add to two cups
Brown Sauce I or II.



\needspace{15\baselineskip}
\section*{Orange Sauce}


\begin{minipage}{1.0\textwidth}
{\setlength{\multicolsep}{0pt}\setlength{\columnsep}{2em}\raggedcolumns%
\begin{multicols}{2}
\begin{itemize}
\setlength{\itemsep}{0pt}
\setlength{\parsep}{0pt}
\item 1/4 cup butter
\item 1/4 cup flour
\item 1 1/3 cups Brown Stock
\item 1/2 teaspoon salt
\item Few grains cayenne
\item Juice 2 oranges
\item 2 tablespoons Sherry wine
\item Rind of 1 orange, cut in fancy shapes
\end{itemize}
\end{multicols}}
\end{minipage}

\vspace{0.3em}
\noindent%
Brown the butter, add flour, with salt and cayenne, and stir until well
browned. Add stock gradually, and just before serving, orange juice,
Sherry, and pieces of rind.



\needspace{15\baselineskip}
\section*{Sauce À L'Italienne}


\begin{minipage}{1.0\textwidth}
{\setlength{\multicolsep}{0pt}\setlength{\columnsep}{2em}\raggedcolumns%
\begin{multicols}{2}
\begin{itemize}
\setlength{\itemsep}{0pt}
\setlength{\parsep}{0pt}
\item 2 tablespoons onion
\item 2 tablespoons carrot
\item 2 tablespoons lean raw ham
\item 12 peppercorns
\item 2 cloves
\item Sprig marjoram
\item 2 tablespoons butter
\item 2 1/2 tablespoons flour
\item 1 cup Brown Stock
\item 1 1/4 cups white wine
\item 1/2 tablespoon finely chopped parsley
\end{itemize}
\end{multicols}}
\end{minipage}

\vspace{0.3em}
\noindent%
Cook first six ingredients with butter five minutes, add flour, and stir
until well browned; then add gradually stock and wine. Strain, reheat,
and after pouring around fish sprinkle with parsley.



\needspace{15\baselineskip}
\section*{Champagne Sauce}

Simmer two cups Espagnole Sauce until reduced to one and one-half cups.
Add two tablespoons mushroom liquor, one-half cup champagne, and one
tablespoon powdered sugar.



\needspace{15\baselineskip}
\section*{Tomato Sauce I (Without Stock)}


\begin{minipage}{1.0\textwidth}
{\setlength{\multicolsep}{0pt}\setlength{\columnsep}{2em}\raggedcolumns%
\begin{multicols}{2}
\begin{itemize}
\setlength{\itemsep}{0pt}
\setlength{\parsep}{0pt}
\item 1/2 can tomatoes or
\item 1 3/4 cups fresh stewed tomatoes
\item 1 slice onion
\item 3 tablespoons butter
\item 3 tablespoons flour
\item 1/4 teaspoon salt
\item 1/8 teaspoon pepper
\end{itemize}
\end{multicols}}
\end{minipage}

\vspace{0.3em}
\noindent%
Cook onion with tomatoes fifteen minutes, rub through a strainer, and
add to butter and flour (to which seasonings have been added) cooked
together. If tomatoes are very acid, add a few grains of soda. If
tomatoes are to retain their red color it is necessary to brown butter
and flour together before adding the tomatoes.



\needspace{15\baselineskip}
\section*{Tomato Sauce II}


\begin{minipage}{1.0\textwidth}
{\setlength{\multicolsep}{0pt}\setlength{\columnsep}{2em}\raggedcolumns%
\begin{multicols}{2}
\begin{itemize}
\setlength{\itemsep}{0pt}
\setlength{\parsep}{0pt}
\item 1/2 can tomatoes
\item 1 teaspoon sugar
\item 8 peppercorns
\item Bit of bay leaf
\item 1/2 teaspoon salt
\item 4 tablespoons butter
\item 4 tablespoons flour
\item 1 cup Brown Stock
\end{itemize}
\end{multicols}}
\end{minipage}

\vspace{0.3em}
\noindent%
Cook tomatoes twenty minutes with sugar, peppercorns, bay leaf, and
salt; rub through a strainer, and add stock. Brown the butter, add
flour, and when well browned, gradually add hot liquid.



\needspace{15\baselineskip}
\section*{Tomato Sauce III}


\begin{minipage}{1.0\textwidth}
{\setlength{\multicolsep}{0pt}\setlength{\columnsep}{2em}\raggedcolumns%
\begin{multicols}{2}
\begin{itemize}
\setlength{\itemsep}{0pt}
\setlength{\parsep}{0pt}
\item 1/4 cup butter
\item 1 slice carrot
\item 1 slice onion
\item Bit of bay leaf
\item Sprig of thyme
\item Sprig of parsley
\item 1 cup stewed and strained tomatoes
\item 1 cup Brown Stock
\item Salt and pepper
\item 1/4 cup flour
\end{itemize}
\end{multicols}}
\end{minipage}

\vspace{0.3em}
\noindent%
Brown the butter with carrot, onion, bay leaf, thyme, and parsley;
remove seasonings, add flour, stir until well browned, then add tomatoes
and stock. Bring to boiling-point, and strain.



\needspace{15\baselineskip}
\section*{Tomato And Mushroom Sauce}


\begin{minipage}{1.0\textwidth}
{\setlength{\multicolsep}{0pt}\setlength{\columnsep}{2em}\raggedcolumns%
\begin{multicols}{2}
\begin{itemize}
\setlength{\itemsep}{0pt}
\setlength{\parsep}{0pt}
\item 2 slices chopped bacon or small quantity uncooked ham
\item 1 slice onion
\item 6 slices carrot
\item 1 bay leaf
\item 2 sprigs thyme
\item Sprig of parsley
\item 2 cloves
\item 1/2 teaspoon peppercorns
\item Few gratings nutmeg
\item 3 tablespoons flour
\item 1/2 can tomatoes
\item 1 1/2 cups Brown Stock
\item Salt and pepper
\item 1/2 can mushrooms
\end{itemize}
\end{multicols}}
\end{minipage}

\vspace{0.3em}
\noindent%
Cook bacon, onion, and carrot five minutes; add bay leaf, thyme,
parsley, cloves, peppercorns, nutmeg, and tomatoes, and cook five
minutes. Add flour diluted with enough cold water to pour; as it
thickens, dilute with stock. Cover, and cook in oven one hour. Strain,
add salt and pepper to taste, and one-half can mushrooms, drained from
their liquor, rinsed, and cut in quarters; then cook two minutes. Use
fresh mushrooms in place of canned ones when possible.



\needspace{15\baselineskip}
\section*{Tomato Cream Sauce}


\begin{minipage}{1.0\textwidth}
{\setlength{\multicolsep}{0pt}\setlength{\columnsep}{2em}\raggedcolumns%
\begin{multicols}{2}
\begin{itemize}
\setlength{\itemsep}{0pt}
\setlength{\parsep}{0pt}
\item 1/2 can tomatoes
\item Sprig of thyme
\item 1 stalk celery
\item 1 slice onion
\item Bit of bay leaf
\item 1 cup White Sauce I
\item 1/2 teaspoon salt
\item Few grains cayenne
\item 1/4 teaspoon soda
\end{itemize}
\end{multicols}}
\end{minipage}

\vspace{0.3em}
\noindent%
Cook tomatoes twenty minutes with seasonings; rub through a strainer,
add soda, then White Sauce. Serve with Baked Fish or Lobster Cutlets.



\needspace{15\baselineskip}
\section*{Spanish Sauce}


\begin{minipage}{1.0\textwidth}
{\setlength{\multicolsep}{0pt}\setlength{\columnsep}{2em}\raggedcolumns%
\begin{multicols}{2}
\begin{itemize}
\setlength{\itemsep}{0pt}
\setlength{\parsep}{0pt}
\item 2 tablespoons finely chopped lean raw ham
\item 2 tablespoons chopped celery
\item 2 tablespoons chopped carrot
\item 1 tablespoon chopped onion
\item 1/4 cup butter
\item 1/4 cup flour
\item 1 1/3 cups Brown Stock
\item 2/3 cup stewed and strained tomatoes
\item Salt and pepper
\end{itemize}
\end{multicols}}
\end{minipage}

\vspace{0.3em}
\noindent%
Cook ham and vegetables with butter until butter is well browned; add
flour, stock, and tomatoes; cook five minutes, then strain. Season with
salt and pepper.



\needspace{15\baselineskip}
\section*{Béchamel Sauce}


\begin{minipage}{1.0\textwidth}
{\setlength{\multicolsep}{0pt}\setlength{\columnsep}{2em}\raggedcolumns%
\begin{multicols}{2}
\begin{itemize}
\setlength{\itemsep}{0pt}
\setlength{\parsep}{0pt}
\item 1 1/2 cups White Stock
\item 1 slice onion
\item 1 slice carrot
\item Bit of bay leaf
\item Sprig of parsley
\item 6 peppercorns
\item 1/4 cup butter
\item 1/4 cup flour
\item 1 cup scalded milk
\item 1/2 teaspoon salt
\item 1/8 teaspoon pepper
\end{itemize}
\end{multicols}}
\end{minipage}

\vspace{0.3em}
\noindent%
Cook stock twenty minutes with onion, carrot, bay leaf, parsley, and
peppercorns, then strain; there should be one cupful. Melt the butter,
add flour, and gradually hot stock and milk. Season with salt and
pepper.



\needspace{15\baselineskip}
\section*{Yellow Béchamel Sauce}

To two cups Béchamel Sauce add yolks of three eggs slightly beaten,
first diluting eggs with small quantity of hot sauce, then adding
gradually to remaining sauce. This prevents the sauce from having a
curdled appearance.



\needspace{15\baselineskip}
\section*{Olive And Almond Sauce}


\begin{minipage}{1.0\textwidth}
{\setlength{\multicolsep}{0pt}\setlength{\columnsep}{2em}\raggedcolumns%
\begin{multicols}{2}
\begin{itemize}
\setlength{\itemsep}{0pt}
\setlength{\parsep}{0pt}
\item 3 tablespoons butter
\item 3 tablespoons flour
\item 1 cup White Stock
\item 1/2 cup cream
\item 1/4 cup shredded almonds
\item 1 teaspoon beef extract
\item 8 olives (stoned and cut in quarters)
\item 1/2 tablespoon lemon juice
\item 1/4 teaspoon salt
\item Few grains cayenne
\end{itemize}
\end{multicols}}
\end{minipage}

\vspace{0.3em}
\noindent%
Melt butter, add flour, and pour on gradually White Stock. Just before
serving add remaining ingredients. Serve with boiled or steamed fish.



\needspace{15\baselineskip}
\section*{Oyster Sauce}


\begin{minipage}{1.0\textwidth}
{\setlength{\multicolsep}{0pt}\setlength{\columnsep}{2em}\raggedcolumns%
\begin{multicols}{2}
\begin{itemize}
\setlength{\itemsep}{0pt}
\setlength{\parsep}{0pt}
\item 1 pint oysters
\item 1/4 cup butter
\item 1/4 cup flour
\item 1 cup milk or Chicken Stock
\item Salt
\item Pepper
\item Oyster liquor
\end{itemize}
\end{multicols}}
\end{minipage}

\vspace{0.3em}
\noindent%
Wash oysters, reserve liquor, heat, strain, add oysters, and cook until
plump. Remove oysters, and make a sauce of butter, flour, oyster liquor,
and milk. Add oysters, and season with salt and pepper.



\needspace{15\baselineskip}
\section*{Cucumber Sauce I}

Grate two cucumbers, drain, and season with salt, pepper, and vinegar.
Serve with Broiled Fish.



\needspace{15\baselineskip}
\section*{Cucumber Sauce II}

Beat one-half cup heavy cream until stiff, and add one-fourth teaspoon
salt, few grains pepper, and gradually two tablespoons vinegar; then add
one cucumber, pared, chopped, and drained.



\needspace{15\baselineskip}
\section*{Celery Sauce}


\begin{itemize}
\setlength{\itemsep}{0pt}
\setlength{\parsep}{0pt}
\item 3 cups celery, cut in thin slices
\item 2 cups Thin White Sauce
\end{itemize}

\vspace{-0.5em}
\noindent%
Wash and scrape celery before cutting into pieces. Cook in boiling
salted water until soft, drain, rub through a sieve, and add to sauce.
Celery sauce is often made from the stock in which fowl or turkey has
been boiled, or with one-half stock and one-half milk.



\needspace{15\baselineskip}
\section*{Suprême Sauce}


\begin{minipage}{1.0\textwidth}
{\setlength{\multicolsep}{0pt}\setlength{\columnsep}{2em}\raggedcolumns%
\begin{multicols}{2}
\begin{itemize}
\setlength{\itemsep}{0pt}
\setlength{\parsep}{0pt}
\item 1/4 cup butter
\item 1/4 cup flour
\item 1 1/2 cups hot Chicken Stock
\item 1/2 cup hot cream
\item 1 tablespoon mushroom liquor
\item 3/4 teaspoon lemon juice
\item Salt and pepper
\end{itemize}
\end{multicols}}
\end{minipage}

\vspace{0.3em}
\noindent%
Make same as Thin White Sauce, and add seasonings.



\needspace{15\baselineskip}
\section*{Maître D'Hôtel Butter}


\begin{itemize}
\setlength{\itemsep}{0pt}
\setlength{\parsep}{0pt}
\item 1/4 cup butter
\item 1/2 teaspoon salt
\item 1/8 teaspoon pepper
\item 1/2 tablespoon finely chopped parsley
\item 3/4 tablespoon lemon juice
\end{itemize}

\vspace{-0.5em}
\noindent%
Put butter in a bowl, and with small wooden spoon work until creamy. Add
salt, pepper, and parsley, then lemon juice very slowly.



\needspace{15\baselineskip}
\section*{Tartar Sauce}


\begin{minipage}{1.0\textwidth}
{\setlength{\multicolsep}{0pt}\setlength{\columnsep}{2em}\raggedcolumns%
\begin{multicols}{2}
\begin{itemize}
\setlength{\itemsep}{0pt}
\setlength{\parsep}{0pt}
\item 1 tablespoon vinegar
\item 1 teaspoon lemon juice
\item 1/4 teaspoon salt
\item 1 tablespoon Worcestershire Sauce
\item 1/3 cup butter
\item \textit{The Boston Cook Book}
\end{itemize}
\end{multicols}}
\end{minipage}

\vspace{0.3em}
\noindent%
Mix vinegar, lemon juice, salt, and Worcestershire Sauce in a small
bowl, and heat over hot water. Brown the butter in an omelet pan, and
strain into first mixture.



\needspace{15\baselineskip}
\section*{Lemon Butter}


\begin{itemize}
\setlength{\itemsep}{0pt}
\setlength{\parsep}{0pt}
\item 1/4 cup butter
\item 1 tablespoon lemon juice
\end{itemize}

\vspace{-0.5em}
\noindent%
Cream the butter, and add slowly lemon juice.



\needspace{15\baselineskip}
\section*{Anchovy Butter}


\begin{itemize}
\setlength{\itemsep}{0pt}
\setlength{\parsep}{0pt}
\item 1/4 cup butter
\item Anchovy essence
\end{itemize}

\vspace{-0.5em}
\noindent%
Cream the butter and add Anchovy essence to taste.



\needspace{15\baselineskip}
\section*{Lobster Butter}


\begin{itemize}
\setlength{\itemsep}{0pt}
\setlength{\parsep}{0pt}
\item 1/4 cup butter
\item Lobster coral
\end{itemize}

\vspace{-0.5em}
\noindent%
Clean, wipe, and force coral through a fine sieve. Put in a mortar with
butter, and pound until well blended. This butter is used in Lobster
Soup and Sauces to give color and richness.



\needspace{15\baselineskip}
\section*{Hollandaise Sauce I}


\begin{minipage}{1.0\textwidth}
{\setlength{\multicolsep}{0pt}\setlength{\columnsep}{2em}\raggedcolumns%
\begin{multicols}{2}
\begin{itemize}
\setlength{\itemsep}{0pt}
\setlength{\parsep}{0pt}
\item 1/2 cup butter
\item 4 egg yolks
\item 1 tablespoon lemon juice
\item 1/4 teaspoon salt
\item Few grains cayenne
\item 1/3 cup boiling water
\end{itemize}
\end{multicols}}
\end{minipage}

\vspace{0.3em}
\noindent%
Put butter in a bowl, cover with cold water, and wash, using a spoon.
Divide in three pieces; put one piece in a saucepan with yolks of eggs
and lemon juice, place saucepan in a larger one containing boiling
water, and stir constantly with a wire whisk until butter is melted;
then add second piece of butter, and, as it thickens, third piece. Add
water, cook one minute, and season with salt and cayenne.



\needspace{15\baselineskip}
\section*{Hollandaise Sauce II}


\begin{itemize}
\setlength{\itemsep}{0pt}
\setlength{\parsep}{0pt}
\item 1/2 cup butter
\item 1/2 tablespoon vinegar or
\item 1 tablespoon lemon juice
\item 4 egg yolks
\item 1/4 teaspoon salt
\end{itemize}

\vspace{-0.5em}
\noindent%
                      Few grains cayenne.

Wash butter, divide in three pieces; put one piece in a saucepan with
vinegar or lemon juice and egg yolks; place saucepan in a larger one
containing boiling water, and stir constantly with a wire whisk. Add
second piece of butter, and, as it thickens, third piece. Remove from
fire, and add salt and cayenne. If left over fire a moment too long it
will separate. If a richer sauce is desired, add one-half teaspoon hot
water and one-half tablespoon heavy cream.



\needspace{15\baselineskip}
\section*{Anchovy Sauce}

Season Brown, Drawn Butter, or Hollandaise Sauce with Anchovy essence.



\needspace{15\baselineskip}
\section*{Horseradish Hollandaise Sauce}

To Hollandaise Sauce II add one-fourth cup grated horseradish root.



\needspace{15\baselineskip}
\section*{Lobster Sauce I}

To Hollandaise Sauce I add one-third cup lobster meat cut in small dice.



\needspace{15\baselineskip}
\section*{Lobster Sauce II}


\begin{minipage}{1.0\textwidth}
{\setlength{\multicolsep}{0pt}\setlength{\columnsep}{2em}\raggedcolumns%
\begin{multicols}{2}
\begin{itemize}
\setlength{\itemsep}{0pt}
\setlength{\parsep}{0pt}
\item 1 1/4 lb. lobster
\item 1/4 cup butter
\item 1/4 cup flour
\item 1/2 teaspoon salt
\item Few grains cayenne
\item 1/2 tablespoon lemon juice
\item 3 cups cold water
\end{itemize}
\end{multicols}}
\end{minipage}

\vspace{0.3em}
\noindent%
Remove meat from lobster, and cut tender claw meat in one-half inch
dice. Chop remaining meat, add to body bones, and cover with water; cook
until stock is reduced to two cups, strain, and add gradually to butter
and flour cooked together, then add salt, cayenne, lemon juice, and
lobster dice.

If the lobster contains coral, prepare Lobster Butter, add flour, and
thicken sauce therewith.



\needspace{15\baselineskip}
\section*{Sauce Béarnaise}

To Hollandaise Sauce II add one teaspoon each of finely chopped parsley
and fresh tarragon.

Served with mutton chops, steaks, broiled squabs, smelts, or boiled
salmon.



\needspace{15\baselineskip}
\section*{Sauce Trianon}

To Hollandaise Sauce II add gradually, while cooking, one and one-half
tablespoons Sherry wine.



\needspace{15\baselineskip}
\section*{Sauce Figaro}

To Hollandaise Sauce II add two tablespoons tomato purée (tomatoes
stewed, strained, and cooked until reduced to a thick pulp), one
teaspoon finely chopped parsley, and a few grains cayenne.



\needspace{15\baselineskip}
\section*{Horseradish Sauce I}


\begin{itemize}
\setlength{\itemsep}{0pt}
\setlength{\parsep}{0pt}
\item 3 tablespoons grated horseradish root
\item 1 tablespoon vinegar
\item 1/4 teaspoon salt
\item Few grains cayenne
\item 4 tablespoons heavy cream
\end{itemize}

\vspace{-0.5em}
\noindent%
Mix first four ingredients, and add cream beaten stiff.



\needspace{15\baselineskip}
\section*{Horseradish Sauce II}


\begin{minipage}{1.0\textwidth}
{\setlength{\multicolsep}{0pt}\setlength{\columnsep}{2em}\raggedcolumns%
\begin{multicols}{2}
\begin{itemize}
\setlength{\itemsep}{0pt}
\setlength{\parsep}{0pt}
\item 3 tablespoons cracker crumbs
\item 1/3 cup grated horseradish root
\item 1 1/2 cups milk
\item 3 tablespoons butter
\item 1/2 teaspoon salt
\item 1/8 teaspoon pepper
\end{itemize}
\end{multicols}}
\end{minipage}

\vspace{0.3em}
\noindent%
Cook first three ingredients twenty minutes in double boiler. Add
butter, salt, and pepper.



\needspace{15\baselineskip}
\section*{Bread Sauce}


\begin{minipage}{1.0\textwidth}
{\setlength{\multicolsep}{0pt}\setlength{\columnsep}{2em}\raggedcolumns%
\begin{multicols}{2}
\begin{itemize}
\setlength{\itemsep}{0pt}
\setlength{\parsep}{0pt}
\item 2 cups milk
\item 1/2 cup fine stale bread crumbs
\item 1 onion
\item 6 cloves
\item 1/2 teaspoon salt
\item Few grains cayenne
\item 3 tablespoons butter
\item 1/2 cup coarse stale bread crumbs
\end{itemize}
\end{multicols}}
\end{minipage}

\vspace{0.3em}
\noindent%
Cook milk thirty minutes in double boiler, with fine bread crumbs and
onion stuck with cloves. Remove onion, add salt, cayenne, and two
tablespoons butter. Usually served poured around roast partridge or
grouse, and sprinkled with coarse crumbs browned in remaining butter.



\needspace{15\baselineskip}
\section*{Rice Sauce}


\begin{minipage}{1.0\textwidth}
{\setlength{\multicolsep}{0pt}\setlength{\columnsep}{2em}\raggedcolumns%
\begin{multicols}{2}
\begin{itemize}
\setlength{\itemsep}{0pt}
\setlength{\parsep}{0pt}
\item 3 tablespoons rice
\item 2 cups milk
\item 1/2 onion
\item 3 cloves
\item 2 tablespoons butter
\item Salt and pepper
\end{itemize}
\end{multicols}}
\end{minipage}

\vspace{0.3em}
\noindent%
Wash rice, add to milk, and cook in double boiler until soft. Rub
through a fine strainer, return to double boiler, add onion stuck with
cloves, and cook fifteen minutes. Remove onion, add butter, salt, and
pepper.



\needspace{15\baselineskip}
\section*{Cauliflower Sauce}


\begin{minipage}{1.0\textwidth}
{\setlength{\multicolsep}{0pt}\setlength{\columnsep}{2em}\raggedcolumns%
\begin{multicols}{2}
\begin{itemize}
\setlength{\itemsep}{0pt}
\setlength{\parsep}{0pt}
\item 1/4 cup butter
\item 1/4 cup flour
\item 1 cup White Stock III
\item 1 cup scalded milk
\item Cooked flowerets from a small cauliflower
\item Salt
\item Pepper
\end{itemize}
\end{multicols}}
\end{minipage}

\vspace{0.3em}
\noindent%
Make same as Thin White Sauce and add flowerets.



\needspace{15\baselineskip}
\section*{Mint Sauce}


\begin{itemize}
\setlength{\itemsep}{0pt}
\setlength{\parsep}{0pt}
\item 1/4 cup finely chopped mint leaves
\item 1/2 cup vinegar
\item 1 tablespoon powdered sugar
\end{itemize}

\vspace{-0.5em}
\noindent%
Add sugar to vinegar; when dissolved, pour over mint and let stand
thirty minutes on back of range to infuse. If vinegar is very strong,
dilute with water.



\needspace{15\baselineskip}
\section*{Currant Jelly Sauce}

To one cup Brown Sauce, from which onion has been omitted, add
one-fourth tumbler currant jelly and one tablespoon Sherry wine; or, add
currant jelly to one cup gravy made to serve with roast lamb. Currant
Jelly Sauce is suitable to serve with lamb.



\needspace{15\baselineskip}
\section*{Port Wine Sauce}

To one cup Brown Sauce, from which onion has been omitted, add
one-eighth tumbler currant jelly, two tablespoons Port wine, and a few
grains cayenne.



\needspace{15\baselineskip}
\section*{Vinaigrette Sauce}


\begin{minipage}{1.0\textwidth}
{\setlength{\multicolsep}{0pt}\setlength{\columnsep}{2em}\raggedcolumns%
\begin{multicols}{2}
\begin{itemize}
\setlength{\itemsep}{0pt}
\setlength{\parsep}{0pt}
\item 1 teaspoon salt
\item 1/4 teaspoon paprika
\item Few grains pepper
\item 1 tablespoon tarragon vinegar
\item 2 tablespoons cider vinegar
\item 6 tablespoons olive oil
\item 1 tablespoon chopped pickles
\item 1 tablespoon chopped green pepper
\item 1 teaspoon chopped parsley
\item 1 teaspoon chopped chives
\end{itemize}
\end{multicols}}
\end{minipage}

\vspace{0.3em}
\noindent%
Mix ingredients in order given.



\needspace{15\baselineskip}
\section*{Sauce Tartare}


\begin{minipage}{1.0\textwidth}
{\setlength{\multicolsep}{0pt}\setlength{\columnsep}{2em}\raggedcolumns%
\begin{multicols}{2}
\begin{itemize}
\setlength{\itemsep}{0pt}
\setlength{\parsep}{0pt}
\item 1/2 teaspoon mustard
\item 1 teaspoon powdered sugar
\item 1/2 teaspoon salt
\item Few grains cayenne
\item 4 egg yolks
\item 1/2 cup olive oil
\item 1 1/2 tablespoons vinegar
\item 1/2 tablespoon capers
\item 1/2 tablespoon pickles
\item 1/2 tablespoon olives
\item 1/2 tablespoon parsley
\item 1/2 shallot, finely chopped
\item 1/4 teaspoon powdered tarragon
\end{itemize}
\end{multicols}}
\end{minipage}

\vspace{0.3em}
\noindent%
Mix mustard, sugar, salt, and cayenne; add yolks of eggs, and stir until
thoroughly mixed, setting bowl in pan of ice-water. Add oil, at first
drop by drop, stirring with a wooden spoon or wire whisk. As mixture
thickens, dilute with vinegar, when oil may be added more rapidly. Keep
in cool place until ready to serve, then add remaining ingredients.



\needspace{15\baselineskip}
\section*{Hot Sauce Tartare}


\begin{minipage}{1.0\textwidth}
{\setlength{\multicolsep}{0pt}\setlength{\columnsep}{2em}\raggedcolumns%
\begin{multicols}{2}
\begin{itemize}
\setlength{\itemsep}{0pt}
\setlength{\parsep}{0pt}
\item 1/2 cup White Sauce I
\item 1/3 cup Mayonnaise
\item 1/2 shallot, finely chopped
\item 1/2 teaspoon vinegar
\item 1/2 tablespoon capers
\item 1/2 tablespoon pickles
\item 1/2 tablespoon olives
\item 1/2 tablespoon parsley
\end{itemize}
\end{multicols}}
\end{minipage}

\vspace{0.3em}
\noindent%
To white sauce add remaining ingredients. Stir constantly until mixture
is thoroughly heated, but do not let it come to the boiling-point.
Served with boiled, steamed, or fried fish.



\needspace{15\baselineskip}
\section*{Hot Mayonnaise}


\begin{minipage}{1.0\textwidth}
{\setlength{\multicolsep}{0pt}\setlength{\columnsep}{2em}\raggedcolumns%
\begin{multicols}{2}
\begin{itemize}
\setlength{\itemsep}{0pt}
\setlength{\parsep}{0pt}
\item 4 egg yolks
\item 2 tablespoons olive oil
\item 1 tablespoon vinegar
\item 1/4 cup hot water
\item Salt
\item Few grains cayenne
\item 1 teaspoon finely chopped parsley
\end{itemize}
\end{multicols}}
\end{minipage}

\vspace{0.3em}
\noindent%
Add oil slowly to egg yolks, then pour on gradually vinegar and water.
Cook over boiling water until mixture thickens, then add seasonings and
parsley.



\needspace{15\baselineskip}
\section*{Sauce Tyrolienne}

To three-fourths cup Mayonnaise add one-half tablespoon each finely
chopped capers and parsley, one finely chopped gherkin, and one-half can
tomatoes, stewed, strained, and cooked until reduced to two tablespoons.
Serve with any kind of fried fish.



\needspace{15\baselineskip}
\section*{Creole Sauce}


\begin{minipage}{1.0\textwidth}
{\setlength{\multicolsep}{0pt}\setlength{\columnsep}{2em}\raggedcolumns%
\begin{multicols}{2}
\begin{itemize}
\setlength{\itemsep}{0pt}
\setlength{\parsep}{0pt}
\item 2 tablespoons chopped onion
\item 4 tablespoons green pepper, finely chopped
\item 2 tablespoons butter
\item 2 tomatoes
\item 1/4 cup sliced mushrooms
\item 6 olives, stoned
\item 1 1/3 cups Brown Sauce
\item Salt and pepper
\item Sherry wine
\end{itemize}
\end{multicols}}
\end{minipage}

\vspace{0.3em}
\noindent%
Cook onion and pepper with butter five minutes; add tomatoes, mushrooms,
and olives, and cook two minutes, then add Brown Sauce. Bring to
boiling-point, and add wine to taste. Serve with broiled beefsteak or
fillet of beef. Boiled rice should accompany the beef, and be served on
same platter.



\needspace{15\baselineskip}
\section*{Russian Sauce}


\begin{minipage}{1.0\textwidth}
{\setlength{\multicolsep}{0pt}\setlength{\columnsep}{2em}\raggedcolumns%
\begin{multicols}{2}
\begin{itemize}
\setlength{\itemsep}{0pt}
\setlength{\parsep}{0pt}
\item 3 tablespoons butter
\item 2 tablespoons flour
\item 1 cup White Stock III
\item 1/4 teaspoon salt
\item Few grains pepper
\item 1/2 teaspoon finely chopped chives
\item 1/2 teaspoon made mustard
\item 1 teaspoon grated horseradish
\item 1/4 cup cream
\item 1 teaspoon lemon juice
\end{itemize}
\end{multicols}}
\end{minipage}

\vspace{0.3em}
\noindent%
Melt butter, add flour, and pour on gradually White Stock; then add
salt, pepper, mustard, chives, and horseradish. Cook two minutes,
strain, add cream and lemon juice. Reheat before serving. Serve with
Beef Tenderloins or Hamburg Steaks.



\needspace{15\baselineskip}
\section*{Sauce Finiste}


\begin{minipage}{1.0\textwidth}
{\setlength{\multicolsep}{0pt}\setlength{\columnsep}{2em}\raggedcolumns%
\begin{multicols}{2}
\begin{itemize}
\setlength{\itemsep}{0pt}
\setlength{\parsep}{0pt}
\item 3 tablespoons butter
\item 1/2 teaspoon mustard
\item Few grains cayenne
\item 1 teaspoon lemon juice
\item 1 1/2 teaspoons Worcestershire Sauce
\item 3/4 cup stewed and strained tomatoes
\end{itemize}
\end{multicols}}
\end{minipage}

\vspace{0.3em}
\noindent%
Cook butter until well browned, and add remaining ingredients.





\chapter{Vegetables}




\needspace{15\baselineskip}
\section*{Table Showing Composition Of Vegetables}


\begin{tabular}{p{2in}ccccc}
\hline
Item & Proteid & Fat & Carbohydrates & Mineral matter & Water \\
\hline
Artichokes & 2.6 & .2 & 16.7 & 1. & 79.5 \\
\arrayrulecolor{tablerowgray}\hline
Asparagus & 1.8 & .2 & 3.3 & 1. & 94. \\
\arrayrulecolor{tablerowgray}\hline
Beans, Lima, green & 7.1 & .7 & 22. & 1.7 & 68.5 \\
\arrayrulecolor{tablerowgray}\hline
Beans, green string & 2.2 & .4 & 9.4 & .7 & 87.3 \\
\arrayrulecolor{tablerowgray}\hline
Beets & 1.6 & .1 & 9.6 & 1.1 & 87.6 \\
\arrayrulecolor{tablerowgray}\hline
Brussels sprouts & 4.7 & 1.1 & 4.3 & 1.7 & 88.2 \\
\arrayrulecolor{tablerowgray}\hline
Cabbage & 2.1 & .4 & 5.8 & 1.4 & 90.3 \\
\arrayrulecolor{tablerowgray}\hline
Carrots & 1.1 & .4 & 9.2 & 1.1 & 88.2 \\
\arrayrulecolor{tablerowgray}\hline
Cauliflower & 1.6 & .8 & 6. & .8 & 90.8 \\
\arrayrulecolor{tablerowgray}\hline
Celery & 1.4 & .1 & 3. & 1.1 & 94.4 \\
\arrayrulecolor{tablerowgray}\hline
Corn, green, sweet & 2.8 & 1.1 & 14.1 & .7 & 81.3 \\
\arrayrulecolor{tablerowgray}\hline
Cucumbers & .8 & .2 & 2.5 & .5 & 96. \\
\arrayrulecolor{tablerowgray}\hline
Eggplant & 1.2 & .3 & 5.1 & .5 & 92.9 \\
\arrayrulecolor{tablerowgray}\hline
Kohl-rabi & 2. & .1 & 5.5 & 1.3 & 91.1 \\
\arrayrulecolor{tablerowgray}\hline
Lettuce & 1.3 & .4 & 3.3 & 1. & 94. \\
\arrayrulecolor{tablerowgray}\hline
Okra & 2. & .4 & 9.5 & .7 & 87.4 \\
\arrayrulecolor{tablerowgray}\hline
Onions & 4.4 & .8 & .5 & 1.2 & 93.5 \\
\arrayrulecolor{tablerowgray}\hline
Parsnips & 1.7 & .6 & 16.1 & 1.7 & 79.9 \\
\arrayrulecolor{tablerowgray}\hline
Peas, green & 4.4 & .5 & 16.1 & .9 & 78.1 \\
\arrayrulecolor{tablerowgray}\hline
Potatoes, sweet & 1.8 & .7 & 27.1 & 1.1 & 69.3 \\
\arrayrulecolor{tablerowgray}\hline
Potatoes, white & 2.1 & .1 & 18. & .9 & 78.9 \\
\arrayrulecolor{tablerowgray}\hline
Spinach & 2.1 & .5 & 3.1 & 1.9 & 92.4 \\
\arrayrulecolor{tablerowgray}\hline
Squash & 1.6 & .6 & 10.4 & .9 & 86.5 \\
\arrayrulecolor{tablerowgray}\hline
Tomatoes & .8 & .4 & 3.9 & .5 & 94.4 \\
\arrayrulecolor{tablerowgray}\hline
Turnips & 1.4 & .2 & 8.7 & .8 & 88.9 \\
\arrayrulecolor{black}
\hline
\end{tabular}

                                               \textit{W. O. Atwater, Ph.D.}

Vegetables include, commonly though not botanically speaking, all plants
used for food except grains and fruits. With exception of beans, peas,
and lentils, which contain a large amount of protein, they are chiefly
valuable for their potash salts, and should form a part of each day's
dietary. Many contain much cellulose, which gives needed bulk to the
food. The legumes, peas, beans, and lentils may be used in place of
flesh food.

For the various vegetables different parts of the plant are used. Some
are eaten in the natural state, others are cooked.


\arrayrulecolor{tablerowgray}
\begin{tabular}{|p{1in}|p{3.5in}|}
\hline
Tubers & White potatoes and Jerusalem artichokes \\
\hline
Roots & Beets, carrots, parsnips, radishes, sweet potatoes, salsify or oyster plant, and turnips \\
\hline
Bulbs & Garlic, onions, and shallots \\
\hline
Stems & Asparagus, celery, and chives \\
\hline
Leaves & Brussels sprouts, beet greens, cabbages, dandelions, lettuce, sorrel, spinach, and watercress \\
\hline
Flowers & Cauliflower \\
\hline
Fruit & Beans, corn, cucumbers, okra, eggplant, peas, lentils, squash, and tomatoes. \\
\hline
\arrayrulecolor{black}
\end{tabular}

Young, tender vegetables,--as lettuce, radishes, cucumbers, watercress,
and tomatoes,--eaten uncooked, served separately or combined in salads,
help to stimulate a flagging appetite, and when dressed with oil furnish
considerable nutriment.

Beans, and peas when old, should be employed in making purées and soups;
by so doing, the outer covering of cellulose, so irritating to the
stomach, is removed.



\needspace{15\baselineskip}
\section*{Care Of Vegetables}

Summer vegetables should be cooked as soon after gathering as possible;
in case they must be kept, spread on bottom of cool, dry,
well-ventilated cellar, or place in ice-box. Lettuce may be best kept by
sprinkling with cold water and placing in a tin pail closely covered.
Wilted vegetables may be freshened by allowing to stand in cold water.
Vegetables which contain sugar lose some of their sweetness by standing;
corn and peas are more quickly affected than others. Winter vegetables
should be kept in a cold, dry place. Beets, carrots, turnips, potatoes,
etc., should be put in barrels or piled in bins, to exclude as much air
as possible. Squash should be spread, and needs careful watching; when
dark spots appear, cook at once.

In using canned goods, empty contents from can as soon as opened, lest
the acid therein act on the tin to produce poisonous compounds, and let
stand one hour, that it may become reoxygenated. Beans, peas, asparagus,
etc., should be emptied into a strainer, drained, and cold water poured
over them and allowed to run through. In using dried vegetables, soak in
cold water several hours before cooking. A few years ago native
vegetables were alone sold; but now our markets are largely supplied
from the Southern States and California, thus allowing us fresh
vegetables throughout the year.



\needspace{15\baselineskip}
\section*{Cooking Of Vegetables}

A small scrubbing-brush, which may be bought for five cents, and two
small pointed knives for preparing vegetables should be found in every
kitchen.

Vegetables should be washed in cold water, and cooked until soft in
boiling salted water; if cooked in an uncovered vessel, their color is
better kept. For peas and beans add salt to water last half hour of
cooking. Time for cooking the same vegetable varies according to
freshness and age, therefore time tables for cooking serve only as
guides.



\needspace{15\baselineskip}
\section*{Mushrooms And Truffles}

These are classed among vegetables. Mushrooms, which grow about us
abundantly, may be easily gathered, and as they contain considerable
nutriment, should often be found on the table. While there are hundreds
of varieties, one by a little study may acquaint herself with a dozen or
more of the most common ones which are valuable as food. Consult W.
Hamilton Gibson, “Our Edible Toadstools and Mushrooms.” Many might cause
illness, but only a few varieties of the \textit{Amanita} family are deadly
poison. Mushrooms require heat and moisture,--a severe drought or very
wet soil being unfavorable for their growth. Never gather mushrooms in
the vicinity of decaying matter. They appear the middle of May, and last
until frost comes. \textit{Campestris} is the variety always found in market;
French canned are of this family. \textit{Boleti} are dried, canned, and sold
as \textit{cepes}.



\needspace{15\baselineskip}
\section*{Truffles}

Truffles belong to the same family as mushrooms, and are grown
underground. France is the most famous field for their production, from
which country they are exported in tin cans, and are too expensive for
ordinary use.



\needspace{15\baselineskip}
\section*{Artichokes}

French artichokes, imported throughout the year, are the ones
principally used. They retail from thirty to forty cents each, and are
cheapest and best in November, December, and January. Artichokes are
appearing in market from California and are somewhat cheaper in price
than the French Artichoke. Jerusalem artichokes are employed for
pickling, and can be bought for fifteen cents per quart.



\needspace{15\baselineskip}
\section*{Boiled Artichokes}

Cut off stem close to leaves, remove outside bottom leaves, trim
artichoke, cut off one inch from top of leaves, and with a sharp knife
remove choke; then tie artichoke with a string to keep its shape. Soak
one-half hour in cold water. Drain, and cook thirty to forty-five
minutes in boiling, salted, acidulated water. Remove from water, place
upside down to drain, then take off string. Serve with Béchamel or
Hollandaise Sauce. Boiled Artichokes often constitute a course at
dinner. Leaves are drawn out separately with fingers, dipped in sauce,
and fleshy ends only eaten, although the bottom is edible. Artichokes
may be cut in quarters, cooked, drained, and served with Sauce
Bearnaise. When prepared in this way they are served with mutton.



\needspace{15\baselineskip}
\section*{Fried Artichokes}

Sprinkle Boiled Artichokes cut in quarters with salt, pepper, and finely
chopped parsley. Dip in Batter I, fry in deep fat, and drain. In
preparing artichokes, trim off tops of leaves closer than when served as
Boiled Artichokes.



\needspace{15\baselineskip}
\section*{Artichoke Bottoms}

Remove all leaves and the choke. Trim bottoms in shape, and cook until
soft in boiling, salted, acidulated water. Serve with Hollandaise or
Béchamel Sauce.



\needspace{15\baselineskip}
\section*{Stuffed Artichokes}

Prepare and cook as Boiled Artichokes, having them slightly underdone.
Fill with Chicken Force-meat I or II, and bake thirty minutes in a
moderate oven, basting twice with Thin White Sauce. Remove to serving
dish and pour around Thin White Sauce.



\needspace{15\baselineskip}
\section*{Asparagus}

Hothouse asparagus is found in market during winter, but is not very
satisfactory, and is sold for about one dollar per bunch. Oyster Bay
(white asparagus) appears first of May, and commands a very high price.
Large and small green stalk asparagus is in season from first of June to
middle of July, and cheapest the middle of June.



\needspace{15\baselineskip}
\section*{Boiled Asparagus}

Cut off lower parts of stalks as far down as they will snap, untie
bunches, wash, remove scales, and retie. Cook in boiling salted water
fifteen minutes or until soft, leaving tips out of water first ten
minutes. Drain, remove string, and spread with soft butter, allowing one
and one-half tablespoons butter to each bunch asparagus. Asparagus is
often broken or cut in inch pieces for boiling, cooking tips a shorter
time than stalks.



\needspace{15\baselineskip}
\section*{Asparagus On Toast}

Serve Boiled Asparagus on Buttered or Milk Toast.



\needspace{15\baselineskip}
\section*{Asparagus In White Sauce}

Boil asparagus cut in one-inch pieces, drain, and add to White Sauce I,
allowing one cup sauce to each bunch asparagus. Serve in Croustades of
Bread for a vegetable course.



\needspace{15\baselineskip}
\section*{Asparagus À La Hollandaise}

Pour Hollandaise Sauce I over Boiled Asparagus.



\needspace{15\baselineskip}
\section*{Asparagus In Crusts}

Remove centres from small rolls, fry shells in deep fat, drain, and fill
with Asparagus in White Sauce.



\needspace{15\baselineskip}
\section*{Beans}

\textit{String Beans} that are obtainable in winter come from California;
natives appear in market the last of June and continue until the last of
September. There are two varieties, green (pole cranberry being best
flavored) and yellow (butter bean).

\textit{Shell Beans}, including horticultural and sieva, are sold in the pod or
shelled, five quarts in pod making one quart shelled. They are found in
market during July and August. Common lima and improved lima shell beans
are in season in August and September. Dried lima beans are procurable
throughout the year.



\needspace{15\baselineskip}
\section*{String Beans}

Remove strings, and snap or cut in one-inch pieces; wash, and cook in
boiling water from one to three hours, adding salt last half-hour of
cooking. Drain, season with butter and salt.



\needspace{15\baselineskip}
\section*{Shell Beans}

Wash, and cook in boiling water from one to one and a half hours, adding
salt last half-hour of cooking. Cook in sufficiently small quantity of
water, that there may be none left to drain off when beans are cooked.
Season with butter and salt.



\needspace{15\baselineskip}
\section*{Cream Of Lima Beans}

Soak one cup dried beans over night, drain, and cook in boiling salted
water until soft; drain, add three-fourths cup cream, and season with
butter and salt. Reheat before serving.



\needspace{15\baselineskip}
\section*{Boiled Beets}

Wash, and cook whole in boiling water until soft; time required being
from one to four hours. Old beets will never be tender, no matter how
long they may be cooked. Drain, and put in cold water that skins may be
easily removed. Serve cut in quarters or slices.



\needspace{15\baselineskip}
\section*{Sugared Beets}


\begin{itemize}
\setlength{\itemsep}{0pt}
\setlength{\parsep}{0pt}
\item 4 hot boiled beets
\item 3 tablespoons butter
\item 1 1/2 tablespoons sugar
\item 1/2 teaspoon salt
\end{itemize}

\vspace{-0.5em}
\noindent%
Cut beets in one-fourth inch slices, add butter, sugar, and salt; reheat
for serving.



\needspace{15\baselineskip}
\section*{Pickled Beets}

Slice cold boiled beets and cover with vinegar.



\needspace{15\baselineskip}
\section*{Beets, Sour Sauce}

Wash beets, and cook in boiling salted water until soft. Drain, and
reserve one-half cup water in which beets were cooked. Plunge into cold
water, rub off skins and cut into cubes. Reheat in

\textbf{Sour Sauce.} Melt two tablespoons butter, add two tablespoons flour,
and pour on the beet water. Add one-fourth cup, each, vinegar and cream,
one teaspoon sugar, one-half teaspoon salt, and a few grains pepper.



\needspace{15\baselineskip}
\section*{Harvard Beets}

Wash twelve small beets, cook in boiling water until soft, remove skins,
and cut beets in thin slices, small cubes, or fancy shapes, using French
vegetable cutter. Mix one-half cup sugar and one-half tablespoon
corn-starch. Add one-half cup vinegar and let boil five minutes. Pour
over beets, and let stand on back of range one-half hour. Just before
serving add two tablespoons butter.



\needspace{15\baselineskip}
\section*{Brussels Sprouts}

Brussels sprouts belong to the same family as cabbage, and the small
heads grow from one to two inches apart, on the axis of the entire stem,
one root yielding about two quarts. They are imported, and also grow in
this country, being cheapest and best in December and January.



\needspace{15\baselineskip}
\section*{Brussels Sprouts In White Sauce}

Pick over, remove wilted leaves, and soak in cold water fifteen minutes.
Cook in boiling salted water twenty minutes, or until easily pierced
with a skewer. Drain, and to each pint add one cup White Sauce I.



\needspace{15\baselineskip}
\section*{Scalloped Brussels Sprouts}

Pick over, remove wilted leaves, and soak in cold water one quart
sprouts. Cook in boiling salted water until soft, then drain. Wash
celery and cut in pieces; there should be one and one-half cups. Melt
three tablespoons butter, add celery, cook two minutes, add three
tablespoons flour, and pour on gradually one and one-half cups scalded
milk; add sprouts and turn mixture into a baking-dish. Cover with
buttered crumbs and bake in a hot oven until crumbs are brown.



\needspace{15\baselineskip}
\section*{Cabbage}

There are four kinds of cabbage in the market,--drum-head, sugar-loaf,
Savoy, and purple; and some variety may be found throughout the year.
The Savoy is best for boiling; drum-head and purple for Cole-Slaw. In
buying, select heavy cabbages.



\needspace{15\baselineskip}
\section*{Boiled Cabbage}

Take off outside leaves, cut in quarters, and remove tough stalk. Soak
in cold water and cook in an uncovered vessel in boiling salted water,
to which is added one-fourth teaspoon soda; this prevents disagreeable
odor during cooking. Cook from thirty minutes to one hour, drain, and
serve; or chop, and season with butter, salt, and pepper.



\needspace{15\baselineskip}
\section*{Escalloped Cabbage}

Cut one-half boiled cabbage in pieces; put in buttered baking-dish,
sprinkle with salt and pepper, and add one cup White Sauce I. Lift
cabbage with fork, that it may be well mixed with sauce, cover with
buttered crumbs, and bake until crumbs are brown.



\needspace{15\baselineskip}
\section*{German Cabbage}

Slice red cabbage and soak in cold water. Put one quart in stewpan with
two tablespoons butter, one-half teaspoon salt, one tablespoon finely
chopped onion, few gratings of nutmeg, and few grains cayenne; cover,
and cook until cabbage is tender. Add two tablespoons vinegar and
one-half tablespoon sugar, and cook five minutes.



\needspace{15\baselineskip}
\section*{Cole-Slaw}

Select a small, heavy cabbage, take off outside leaves, and cut in
quarters; with a sharp knife slice very thinly. Soak in cold water until
crisp, drain, dry between towels, and mix with Cream Salad Dressing.



\needspace{15\baselineskip}
\section*{Hot Slaw}

Slice cabbage as for Cole-Slaw, using one-half cabbage. Heat in a
dressing made of yolks of two eggs slightly beaten, one-fourth cup cold
water, one tablespoon butter, one-fourth cup hot vinegar, and one-half
teaspoon salt, stirred over hot water until thickened.



\needspace{15\baselineskip}
\section*{Carrots}

Carrots may always be found in market. New carrots appear last of April,
and are sold in bunches; these may be boiled and served, but carrots are
chiefly used for flavoring soups, and for garnishing, on account of
their bright color. To prepare carrots for cooking, wash and scrape, as
best flavor and brightest color are near the skin.



\needspace{15\baselineskip}
\section*{Carrots And Peas}

Wash, scrape, and cut young carrots in small cubes or fancy shapes; cook
until soft in boiling salted water or stock. Drain, add an equal
quantity of cooked green peas, and season with butter, salt, and pepper.



\needspace{15\baselineskip}
\section*{Carrots, Poulette Sauce}

Wash, scrape, and cut carrots in strips, cubes, or fancy shapes, cover
with boiling water, let stand five minutes; drain, and cook in boiling
salted water, to which is added one-half tablespoon butter, until soft.
Add to recipe for sauce given under Macédoine of Vegetables à la
Poulette (see p. 308).



\needspace{15\baselineskip}
\section*{Cauliflower}

Cauliflowers comprise the stalks and flowerets of a plant which belongs
to the same family as Brussels sprouts and cabbage; they may be obtained
throughout the year, but are cheapest and best in September and October.
In selecting cauliflowers, choose those with white heads and fresh green
leaves; if dark spots are on the heads, they are not fresh.



\needspace{15\baselineskip}
\section*{Creamed Cauliflower}

Remove leaves, cut off stalk, and soak thirty minutes (head down) in
cold water to cover. Cook (head up) twenty minutes or until soft in
boiling salted water; drain, separate flowerets, and reheat in one and
one-half cups White Sauce I.



\needspace{15\baselineskip}
\section*{Cauliflower À La Hollandaise}

Prepare as for Creamed Cauliflower, using Hollandaise Sauce I instead of
White Sauce.



\needspace{15\baselineskip}
\section*{Cauliflower Au Gratin}

Place a whole cooked cauliflower on a dish for serving, cover with
buttered crumbs, and place on oven grate to brown crumbs; remove from
oven and pour one cup Thin White Sauce around cauliflower.



\needspace{15\baselineskip}
\section*{Cauliflower À La Parmesan}

Prepare as Cauliflower au Gratin. Sprinkle with grated cheese before
covering with crumbs.



\needspace{15\baselineskip}
\section*{Cauliflower À La Huntington}

Prepare cauliflower as for boiled cauliflower, and steam until soft.
Separate in pieces and pour over the following sauce:

Mix one and one-half teaspoons mustard, one and one-fourth teaspoons
salt, one teaspoon powdered sugar, and one-fourth teaspoon paprika. Add
yolks three eggs slightly beaten, one-fourth cup olive oil, and one-half
cup vinegar in which one-half teaspoon finely chopped shallot has
infused five minutes. Cook over hot water until mixture thickens. Remove
from range, and add one-half tablespoon curry powder, two tablespoons
melted butter, and one teaspoon finely chopped parsley.



\needspace{15\baselineskip}
\section*{Celery}

Celery may be obtained from last of July until April. It is best and
cheapest in December. Celery stalks are green while growing; but the
white celery seen in market has been bleached, with the exception of
Kalamazoo variety, which grows white. To prepare celery for table, cut
off roots and leaves, separate stalks, wash, scrape, and chill in
ice-water. By adding a slice of lemon to ice-water celery is kept white
and made crisp. If tops of stalks are gashed several times before
putting in water, they will curl back and make celery look more
attractive.



\needspace{15\baselineskip}
\section*{Celery In White Sauce}

Wash, scrape, and cut celery stalks in one-inch pieces; cook twenty
minutes or until soft in boiling salted water; drain, and to two cups
celery add one cup White Sauce I. This is a most satisfactory way of
using the outer stalks of celery.



\needspace{15\baselineskip}
\section*{Fried Celery, Tomato Sauce}

Wash and scrape celery, cut in three-inch pieces, dip in batter, fry in
deep fat, and drain on brown paper. Serve with Tomato Sauce.

\textbf{Batter.} Mix one-half cup bread flour, one-fourth teaspoon salt, a few
grains pepper, one-third cup milk, and one egg well beaten.



\needspace{15\baselineskip}
\section*{Chiccory Or Endive}

Chiccory or endive may be obtained throughout the year, but during
January, February, March, and April supply is imported. It is used only
for salads.



\needspace{15\baselineskip}
\section*{Corn}

Corn may be found in market from first of June to first of October.
Until native corn appears it is the most unsatisfactory vegetable.
Native corn is obtainable the last of July, but is most abundant and
cheapest in August. Among the best varieties are Crosby for early corn
and Evergreen for late corn.



\needspace{15\baselineskip}
\section*{Boiled Green Corn}

Remove husks and silky threads. Cook ten to twenty minutes in boiling
water. Place on platter covered with napkin; draw corners of napkin over
corn; or cut from cob and season with butter and salt.



\needspace{15\baselineskip}
\section*{Succotash}

Cut hot boiled corn from cob, add equal quantity of hot boiled shelled
beans; season with butter and salt; reheat before serving.



\needspace{15\baselineskip}
\section*{Corn Oysters}

Grate raw corn from cobs. To one cup pulp add one well-beaten egg,
one-fourth cup flour, and season highly with salt and pepper. Drop by
spoonfuls and fry in deep fat, or cook on a hot, well-greased griddle.
They should be made about the size of large oysters.



\needspace{15\baselineskip}
\section*{Corn Fritters}


\begin{minipage}{1.0\textwidth}
{\setlength{\multicolsep}{0pt}\setlength{\columnsep}{2em}\raggedcolumns%
\begin{multicols}{2}
\begin{itemize}
\setlength{\itemsep}{0pt}
\setlength{\parsep}{0pt}
\item 1 can corn
\item 1 cup flour
\item 1 teaspoon baking powder
\item 2 teaspoons salt
\item 1/4 teaspoon paprika
\item 2 eggs
\end{itemize}
\end{multicols}}
\end{minipage}

\vspace{0.3em}
\noindent%
Chop corn, and add dry ingredients mixed and sifted, then add yolks of
eggs beaten until thick, and fold in whites of eggs beaten stiff. Cook
in a frying-pan in fresh hot lard. Drain on paper.



\needspace{15\baselineskip}
\section*{Corn À La Southern}

To one can chopped corn add two eggs slightly beaten, one teaspoon salt,
one-eighth teaspoon pepper, one and one-half tablespoons melted butter,
and one pint scalded milk; turn into a buttered pudding-dish and bake in
slow oven until firm.



\needspace{15\baselineskip}
\section*{Chestnuts}

French and Italian chestnuts are served in place of vegetables.



\needspace{15\baselineskip}
\subsection*{Chestnut Purée}

Remove shells from chestnuts, cook until soft in boiling salted water;
drain, mash, moisten with scalded milk, season with salt and pepper, and
beat until light. Chestnuts are often boiled, riced, and piled lightly
in centre of dish, then surrounded by meat.



\needspace{15\baselineskip}
\section*{Baked Chestnuts}

Remove shells from one pint chestnuts, put in a baking-dish, cover with
Chicken Stock highly seasoned with salt and cayenne, and bake until
soft, keeping covered until nearly done. There should be a small
quantity of stock in pan to serve with chestnuts.



\needspace{15\baselineskip}
\section*{Cucumbers}

Cucumbers may be obtained throughout the year, and are generally served
raw. During the latter part of the summer they are gathered and pickled
for subsequent use. Small pickled cucumbers are called gherkins.



\needspace{15\baselineskip}
\section*{Sliced Cucumbers}

Remove thick slices from both ends and cut off a thick paring, as the
cucumber contains a bitter principle, a large quantity of which lies
near the skin and stem end. Cut in thin slices and keep in cold water
until ready to serve. Drain, and cover with crushed ice for serving.



\needspace{15\baselineskip}
\section*{Boiled Cucumbers}

Old cucumbers may be pared, cut in pieces, cooked until soft in boiling
salted water, drained, mashed, and seasoned, with butter, salt, and
pepper.



\needspace{15\baselineskip}
\section*{Fried Cucumbers}

Pare cucumbers and cut lengthwise in one-third inch slices. Dry between
towels, sprinkle with salt and pepper, dip in crumbs, egg, and crumbs
again, fry in deep fat, and drain.



\needspace{15\baselineskip}
\section*{Stuffed Cucumbers}

Pare three cucumbers, cut in halves crosswise, remove seeds, and let
stand in cold water thirty minutes. Drain, wipe, and fill with
force-meat, using recipe for Chicken Force-meat I or II, substituting
veal for chicken. Place upright on a trivet in a saucepan. Half surround
with White Stock, cover, and cook forty minutes. Place on thin slices of
dry toast, cut in circular shapes, and pour around one and one-half cups
Béchamel Sauce. Serve as a vegetable course or an entrée.



\needspace{15\baselineskip}
\section*{Fried Eggplant I}

Pare an eggplant and cut in very thin slices. Sprinkle slices with salt
and pile on a plate. Cover with a weight to express the juice, and let
stand one and one-half hours. Dredge with flour and sauté slowly in
butter until crisp and brown. Eggplant is in season from September to
February.



\needspace{15\baselineskip}
\section*{Fried Eggplant II}

Pare an eggplant, cut in one-fourth inch slices, and soak over night in
cold salted water. Drain, let stand in cold water one-half hour, drain
again, and dry between towels. Sprinkle with salt and pepper, dip in
batter, or dip in flour, egg, and crumbs, and fry in deep fat.



\needspace{15\baselineskip}
\section*{Stuffed Eggplant}

Cook eggplant fifteen minutes in boiling salted water to cover. Cut a
slice from top, and with a spoon remove pulp, taking care not to work
too closely to skin. Chop pulp, and add one cup soft stale bread crumbs.
Melt two tablespoons butter, add one-half tablespoon finely chopped
onion, and cook five minutes, or try out three slices of bacon, using
bacon fat in place of butter. Add to chopped pulp and bread, season with
salt and pepper, and if necessary moisten with a little stock or water;
cook five minutes, cool slightly, and add one beaten egg. Refill
eggplant, cover with buttered bread crumbs, and bake twenty-five minutes
in a hot oven.



\needspace{15\baselineskip}
\section*{Scalloped Eggplant}

Pare an eggplant and cut in two-thirds inch cubes. Cook in a small
quantity of boiling water until soft, then drain. Cook two tablespoons
butter with one-half onion, finely chopped, until yellow, add
three-fourths tablespoon finely chopped parsley and eggplant. Turn into
a buttered baking-dish. Cover with buttered crumbs and bake until crumbs
are brown.



\needspace{15\baselineskip}
\section*{Greens}

Hothouse beet greens and dandelions appear in market the first of March,
when they command a high price. Those grown out of doors are in season
from middle of May to first of July.



\needspace{15\baselineskip}
\section*{Boiled Beet Greens}

Wash thoroughly and scrape roots, cutting off ends. Drain, and cook one
hour or until tender in a small quantity boiling salted water. Season
with butter, salt, and pepper. Serve with vinegar.



\needspace{15\baselineskip}
\section*{Dandelions}

Wash thoroughly, remove roots, drain, and cook one hour or until tender
in boiling salted water. Allow two quarts water to one peck dandelions.
Season with butter, salt, and pepper. Serve with vinegar.



\needspace{15\baselineskip}
\section*{Lettuce}

Lettuce is obtainable all the year, and is especially valuable during
the winter and spring, when other green vegetables in market command a
high price. Although containing but little nutriment, it is useful for
the large quantity of water and potash salts that it contains, and
assists in stimulating the appetite. Curly lettuce is of less value than
Tennis Ball, but makes an effective garnish.

Lettuce should be separated by removing leaves from stalk (discarding
wilted outer leaves), washed, kept in cold water until crisp, drained,
and so placed on a towel that water may drop from leaves. A bag made
from white mosquito netting is useful for drying lettuce. Wash lettuce
leaves, place in bag, and hang in lower part of ice-box to drain. Wire
baskets are used for the same purpose. Arrange lettuce for serving in
nearly its original shape.



\needspace{15\baselineskip}
\section*{Leeks On Toast}

Wash and trim leeks, cook in boiling salted water until soft, and drain.
Arrange on pieces of buttered toast and pour over melted butter,
seasoned with salt and pepper.



\needspace{15\baselineskip}
\section*{Onions}

The onion belongs to the same family (Lily) as do \textit{shallot}, \textit{garlic},
\textit{leek}, and \textit{chive}. Onions are cooked and served as a vegetable. They
are wholesome, and contain considerable nutriment, but are objectionable
on account of the strong odor they impart to the breath, due to volatile
substances absorbed by the blood, and by the blood carried to the lungs,
where they are set free. The common garden onion is obtainable
throughout the year, the new ones appearing in market about the first of
June. In large centres Bermuda and Spanish onions are procurable from
March 1st to June 1st, and are of delicate flavor.

Shallot, leek, garlic, and chive are principally used to give additional
flavor to food. Shallot, garlic, and chive are used, to some extent, in
making salads.



\needspace{15\baselineskip}
\section*{Boiled Onions}

Put onions in cold water and remove skins while under water. Drain, put
in a saucepan, and cover with boiling salted water; boil five minutes,
drain, and again cover with boiling salted water. Cook one hour or until
soft, but not broken. Drain, add a small quantity of milk, cook five
minutes, and season with butter, salt, and pepper.



\needspace{15\baselineskip}
\section*{Onions In Cream}

Prepare and cook as Boiled Onions, changing the water twice during
boiling; drain, and cover with Cream or Thin White Sauce.



\needspace{15\baselineskip}
\section*{Scalloped Onions}

Cut Boiled Onions in quarters. Put in a buttered baking-dish, cover with
White Sauce I, sprinkle with buttered cracker crumbs, and place on
centre grate in oven to brown crumbs.



\needspace{15\baselineskip}
\section*{Glazed Onions}

Peel small silver skinned onions, and cook in boiling water fifteen
minutes. Drain, dry on cheese-cloth, put in a buttered baking-dish, add
highly seasoned brown stock to cover bottom of dish, sprinkle with
sugar, and bake until soft, basting with stock in pan.



\needspace{15\baselineskip}
\section*{Fried Onions}

Remove skins from four medium-sized onions. Cut in thin slices and put
in a hot omelet pan with one and one-half tablespoons butter. Cook until
brown, occasionally shaking pan that onions may not burn, or turn
onions, using a fork. Sprinkle with salt one minute before taking from
fire.



\needspace{15\baselineskip}
\section*{French Fried Onions}

Peel onions, cut in one-fourth inch slices, and separate into rings. Dip
in milk, drain, and dip in flour. Fry in deep fat, drain on brown paper,
and sprinkle with salt.



\needspace{15\baselineskip}
\section*{Stuffed Onions}

Remove skins from onions, and parboil ten minutes in boiling salted
water to cover. Turn upside down to cool, and remove part of centres.
Fill cavities with equal parts of finely chopped cooked chicken, stale
soft bread crumbs, and finely chopped onion which was removed, seasoned
with salt and pepper, and moistened with cream or melted butter. Place
in buttered shallow baking-pan, sprinkle with buttered crumbs, and bake
in a moderate oven until onions are soft.



\needspace{15\baselineskip}
\section*{Creamed Oyster Plant (Salsify)}

Wash, scrape, and put at once into cold acidulated water to prevent
discoloration. Cut in inch slices, cook in boiling salted water until
soft, drain, and add to White Sauce I. Oyster plant is in season from
October to March.



\needspace{15\baselineskip}
\section*{Salsify Fritters}

Cook oyster plant as for Creamed Oyster Plant. Mash, season with butter,
salt, and pepper. Shape in small flat cakes, roll in flour, and sauté in
butter.



\needspace{15\baselineskip}
\section*{Parsnips}

Parsnips are not so commonly served as other vegetables; however, they
often accompany a boiled dinner. They are raised mostly for feeding
cattle. Unless young they contain a large amount of woody fibre, which
extends through centre of roots and makes them undesirable as food.



\needspace{15\baselineskip}
\section*{Parsnips With Drawn Butter Sauce}

Wash and scrape parsnips, and cut in pieces two inches long and one-half
inch wide and thick. Cook five minutes in boiling salted water, or until
soft. Drain, and to two cups add one cup Drawn Butter Sauce.



\needspace{15\baselineskip}
\section*{Parsnip Fritters}

Wash parsnips and cook forty-five minutes in boiling salted water.
Drain, plunge into cold water, when skins will be found to slip off
easily. Mash, season with butter, salt, and pepper, shape in small flat
round cakes, roll in flour, and sauté in butter.



\needspace{15\baselineskip}
\section*{Peas}

Peas contain, next to beans, the largest percentage of protein of any of
the vegetables, and when young are easy of digestion. They appear in
market as early as April, coming from Florida and California, and
although high in price are hardly worth buying, having been picked so
long. Native peas may be obtained the middle of June, and last until the
first of September. The early June are small peas, contained in a small
pod. McLean, the best peas, are small peas in large flat pods. Champion
peas are large, and the pods are well filled, but they lack sweetness.
Marrowfat peas are the largest in the market, and are usually sweet.



\needspace{15\baselineskip}
\section*{Boiled Peas}

Remove peas from pods, cover with cold water, and let stand one-half
hour. Skim off undeveloped peas which rise to top of water, and drain
remaining peas. Cook until soft in a small quantity of boiling water,
adding salt the last fifteen minutes of cooking. (Consult Time Table for
Cooking, p. 28). There should be but little, if any, water to drain from
peas when they are cooked. Season with butter, salt, and pepper. If peas
have lost much of their natural sweetness, they are improved by the
addition of a small amount of sugar.



\needspace{15\baselineskip}
\section*{Creamed Peas}

Drain Boiled Peas, and to two cups peas add three-fourths cup White
Sauce II. Canned peas are often drained, rinsed, and reheated in this
way.



\needspace{15\baselineskip}
\section*{Pea Timbales}

Drain and rinse one can peas, and rub through a sieve. To one cup pea
pulp add two beaten eggs, two tablespoons melted butter, two-thirds
teaspoon salt, one-eighth teaspoon pepper, few grains cayenne, and few
drops onion juice. Turn into buttered moulds, set in pan of hot water,
cover with buttered paper, and bake until firm. Serve with one cup white
sauce to which is added one-third cup canned peas drained and rinsed.



\needspace{15\baselineskip}
\section*{Stuffed Peppers I}


\begin{minipage}{1.0\textwidth}
{\setlength{\multicolsep}{0pt}\setlength{\columnsep}{2em}\raggedcolumns%
\begin{multicols}{2}
\begin{itemize}
\setlength{\itemsep}{0pt}
\setlength{\parsep}{0pt}
\item 6 green peppers
\item 1 onion, finely chopped
\item 2 tablespoons butter
\item 4 tablespoons chopped mushrooms
\item 1/3 cup Brown Sauce
\item 3 tablespoons bread crumbs
\item Salt and pepper
\item Buttered bread crumbs
\item 4 tablespoons lean raw ham, finely chopped
\end{itemize}
\end{multicols}}
\end{minipage}

\vspace{0.3em}
\noindent%
Cut a slice from stem end of each pepper, remove seeds, and parboil
peppers, fifteen minutes.

Cook onion in butter three minutes; add mushrooms and ham, and cook one
minute, then add Brown Sauce and bread crumbs. Cool mixture, sprinkle
peppers with salt, fill with cooked mixture, cover with buttered bread
crumbs and bake ten minutes. Serve on toast with Brown Sauce.



\needspace{15\baselineskip}
\section*{Stuffed Peppers II}

Prepare peppers as for Stuffed Peppers I. Fill with equal parts of
finely chopped cold cooked chicken or veal, and softened bread crumbs,
seasoned with onion juice, salt, and pepper.



\needspace{15\baselineskip}
\section*{Pumpkins}

Pumpkins are boiled or steamed same as squash, but require longer
cooking. They are principally used for making pies.



\needspace{15\baselineskip}
\section*{Radishes}

Radishes may be obtained throughout the year. There are round and long
varieties, the small round ones being considered best. They are bought
in bunches, six or seven constituting a bunch. Radishes are used merely
for a relish, and are served uncooked. To prepare radishes for table,
remove leaves, stems, and tip end of root, scrape roots, and serve on
crushed ice. Round radishes look very attractive cut to imitate tulips,
when they should not be scraped; to accomplish this, begin at root end
and make six incisions through skin running three-fourths length of
radish. Pass knife under sections of skin, and cut down as far as
incisions extend. Place in cold water, and sections of skin will fold
back, giving radish a tulip-like appearance.



\needspace{15\baselineskip}
\section*{Spinach}

Spinach is cheapest and best in early summer, but is obtainable
throughout the year. It gives variety to winter diet, when most green
vegetables are expensive and of inferior quality.



\needspace{15\baselineskip}
\section*{Boiled Spinach}

Remove roots, carefully pick over (discarding wilted leaves), and wash
in several waters to be sure that it is free from all sand. When young
and tender put in a stewpan, allow to heat gradually, and boil
twenty-five minutes, or until tender, in its own juices. Old spinach is
better cooked in boiling salted water, allowing two quarts water to one
peck spinach. Drain thoroughly, chop finely, reheat, and season with
butter, salt, and pepper. Mound on a serving dish and garnish with
slices of “hard-boiled” eggs and toast points. The green color of
spinach is better retained by cooking in a large quantity of water in an
uncovered vessel.



\needspace{15\baselineskip}
\section*{Spinach À La Béchamel}

Prepare one-half peck Boiled Spinach. Put three tablespoons butter in
hot omelet pan; when melted, add chopped spinach, cook three minutes.
Sprinkle with two tablespoons flour, stir thoroughly, and add gradually
three-fourths cup milk; cook five minutes.



\needspace{15\baselineskip}
\section*{Purée Of Spinach}

Wash and pick over one-half peck spinach. Cook in an uncovered vessel
with a large quantity of boiling salted water to which is added
one-third teaspoon soda and one-half teaspoon sugar. Drain, chop finely,
and rub through a sieve. Reheat, add three tablespoons butter, one
tablespoon flour, and one-half cup cream. Arrange on serving dish and
garnish with yolk and white of “hard-boiled” egg and fried bread cut in
fancy shapes.



\needspace{15\baselineskip}
\section*{Spinach (French Style)}

Pick over and wash one peck spinach, and cook in boiling salted water
twenty-five minutes. Drain, and finely chop. Reheat in hot pan with four
tablespoons butter to which have been added three tablespoons flour and
two-thirds cup Chicken Stock. Season with one teaspoon powdered sugar,
salt, pepper, and a few gratings each of nutmeg and lemon rind.



\needspace{15\baselineskip}
\section*{Squash}

Summer squash, which are in market during the summer months, should be
young, tender, and thin skinned. The common varieties are the white
round and yellow crook-neck. Some of the winter varieties appear in
market as early as the middle of August; among the most common are
Marrow, Turban, and Hubbard. Turban and Hubbard are usually drier than
Marrow. Marrow and Turban have a thin shell, which may be pared off
before cooking. Hubbard Squash has a very hard shell, which must be
split in order to separate squash in pieces, and squash then cooked in
the shell. In selecting winter squash, see that it is heavy in
proportion to its size.



\needspace{15\baselineskip}
\section*{Boiled Summer Squash}

Wash squash and cut in thick slices or quarters. Cook twenty minutes in
boiling salted water, or until soft. Turn in a cheese-cloth placed over
a colander, drain, and wring in cheese-cloth. Mash, and season with
butter, salt, and pepper.



\needspace{15\baselineskip}
\section*{Fried Summer Squash I}

Wash, and cut in one-half inch slices. Sprinkle with salt and pepper,
dip in crumbs, egg, and crumbs again, fry in hot fat, and drain.



\needspace{15\baselineskip}
\section*{Fried Summer Squash II}

Follow recipe for Fried Eggplant I.



\needspace{15\baselineskip}
\section*{Steamed Winter Squash}

Cut in pieces, remove seeds and stringy portion, and pare. Place in a
strainer and cook thirty minutes, or until soft, over boiling water.
Mash, and season with butter, salt, and pepper. If lacking in sweetness,
add a small quantity of sugar.



\needspace{15\baselineskip}
\section*{Boiled Winter Squash}

Prepare as for Steamed Winter Squash. Cook in boiling salted water,
drain, mash, and season. Unless squash is very dry, it is much better
steamed than boiled.



\needspace{15\baselineskip}
\section*{Baked Winter Squash I}

Cut in pieces two inches square, remove seeds and stringy portion, place
in a dripping-pan, sprinkle with salt and pepper, and allow for each
square one-half teaspoon molasses and one-half teaspoon melted butter.
Bake fifty minutes, or until soft, in a moderate oven, keeping covered
the first half-hour of cooking. Serve in the shell.



\needspace{15\baselineskip}
\section*{Baked Winter Squash II}

Cut squash in halves, remove seeds and stringy portion, place in a
dripping-pan, cover, and bake two hours, or until soft, in a slow oven.
Remove from shell, mash, and season with butter, salt, and pepper.



\needspace{15\baselineskip}
\section*{Tomatoes}

Tomatoes are obtainable throughout the year, but are cheapest and best
in September. Hothouse tomatoes are in market during the winter, and
command a very high price, sometimes retailing for one and one-half
dollars a pound.

Southern tomatoes appear as early as May 1st, and although of good
color, lack flavor. Of the many varieties of tomatoes, Acme is among the
best.



\needspace{15\baselineskip}
\section*{Sliced Tomatoes}

Wipe, and cover with boiling water; let stand one minute, when they may
be easily skinned. Chill thoroughly, and cut in one-third inch slices.



\needspace{15\baselineskip}
\section*{Stewed Tomatoes}

Wipe, pare, cut in pieces, put in stewpan, and cook slowly twenty
minutes, stirring occasionally. Season with butter, salt, and pepper.

Scalloped Tomatoes

Remove contents from one can tomatoes and drain tomatoes from some of
their liquor. Season with salt, pepper, a few drops of onion juice, and
sugar if preferred sweet. Cover the bottom of a buttered baking-dish
with buttered cracker crumbs, cover with tomatoes, and sprinkle top
thickly with buttered crumbs. Bake in a hot oven until crumbs are brown.



\needspace{15\baselineskip}
\section*{Broiled Tomatoes}

Wipe and cut in halves crosswise, cut off a thin slice from rounding
part of each half. Sprinkle with salt and pepper, dip in crumbs, egg,
and crumbs again, place in a well-buttered broiler, and broil six to
eight minutes.



\needspace{15\baselineskip}
\section*{Tomatoes À La Crême}

Wipe, peel, and slice three tomatoes. Sprinkle with salt and pepper,
dredge with flour, and sauté in butter. Place on a hot platter and pour
over them one cup White Sauce I.



\needspace{15\baselineskip}
\section*{Devilled Tomatoes}


\begin{minipage}{1.0\textwidth}
{\setlength{\multicolsep}{0pt}\setlength{\columnsep}{2em}\raggedcolumns%
\begin{multicols}{2}
\begin{itemize}
\setlength{\itemsep}{0pt}
\setlength{\parsep}{0pt}
\item 3 tomatoes
\item Salt and pepper
\item Flour
\item Butter for sautéing
\item 4 tablespoons butter
\item 2 teaspoons powdered sugar
\item 1 teaspoon mustard
\item 1/4 teaspoon salt
\item Few grains cayenne
\item Yolk 1 “hard-boiled” egg
\item 1 egg
\item 2 tablespoons vinegar
\end{itemize}
\end{multicols}}
\end{minipage}

\vspace{0.3em}
\noindent%
Wipe, peel, and cut tomatoes in slices. Sprinkle with salt and pepper,
dredge with flour, and sauté in butter. Place on a hot platter and pour
over the dressing made by creaming the butter, adding dry ingredients,
yolk of egg rubbed to a paste, egg beaten slightly, and vinegar, then
cooking over hot water, stirring constantly until it thickens.



\needspace{15\baselineskip}
\section*{Baked Tomatoes I}

Wipe, and remove a thin slice from stem end of six smooth, medium-sized
tomatoes. Take out seeds and pulp, and drain off most of the liquid. Add
an equal quantity of cracker crumbs, season with salt, pepper, and a few
drops onion juice, and refill tomatoes with mixture. Place in a buttered
pan, sprinkle with buttered crumbs, and bake twenty minutes in a hot
oven.



\needspace{15\baselineskip}
\section*{Baked Tomatoes II}

Wipe six small, selected tomatoes and make two one-inch gashes on
blossom end of each, having gashes cross each other at right angles.
Place in granite-ware pan and bake until thoroughly heated. Serve with
sauce for Devilled Tomatoes, adding, just before serving, one tablespoon
heavy cream.



\needspace{15\baselineskip}
\section*{Stuffed Tomatoes}

Wipe, and remove thin slices from stem end of six medium-sized tomatoes.
Take out seeds and pulp, sprinkle inside of tomatoes with salt, invert,
and let stand one-half hour. Cook five minutes two tablespoons butter
with one-half tablespoon finely chopped onion. Add one-half cup finely
chopped cold cooked chicken or veal, one-half cup stale soft bread
crumbs, tomato pulp, and salt and pepper to taste. Cook five minutes,
then add one egg slightly beaten, cook one minute, and refill tomatoes
with mixture. Place in buttered pan, sprinkle with buttered cracker
crumbs, and bake twenty minutes in a hot oven.



\needspace{15\baselineskip}
\section*{Turnips}

Turnips are best during the fall and winter; towards spring they become
corky, and are then suitable only for stews and flavoring. The
Ruta-baga, a large yellow turnip, is one of the best varieties; the
large white French turnip and the small flat Purple Top are also used.



\needspace{15\baselineskip}
\section*{Mashed Turnip}

Wash and pare turnips, cut in slices or quarters, and cook in boiling
salted water until soft. Drain, mash, and season with butter, salt, and
pepper.



\needspace{15\baselineskip}
\section*{Creamed Turnip}

Wash turnips, and cut in one-half inch cubes. Cook three cups cubes in
boiling salted water twenty minutes, or until soft. Drain, and add one
cup White Sauce I.



\needspace{15\baselineskip}
\section*{Turnip Croquettes}

Wash, pare, and cut in quarters new French turnips. Steam until tender,
mash, pressing out all water that is possible. This is best accomplished
by wringing in cheese-cloth. Season one and one-fourth cups with salt
and pepper, then add yolks of two eggs slightly beaten. Cool, shape in
small croquettes, dip in crumbs, egg, and crumbs again, fry in deep fat,
and drain.



\needspace{15\baselineskip}
\section*{Stewed Mushrooms}

Wash one-half pound mushrooms. Remove stems, scrape, and cut in pieces.
Peel caps, and break in pieces. Melt three tablespoons of butter, add
mushrooms, cook two minutes; sprinkle with salt and pepper, dredge with
flour, and add one-half cup hot water or stock. Cook slowly five
minutes.



\needspace{15\baselineskip}
\section*{Stewed Mushrooms In Cream}

Prepare mushrooms as for Stewed Mushrooms. Cook with three-fourths cup
cream instead of using water or stock. Add a slight grating of nutmeg,
pour over small finger-shaped pieces of dry toast, and garnish with
toast points and parsley.



\needspace{15\baselineskip}
\section*{Broiled Mushrooms}

Wash mushrooms, remove stems, and place caps in a buttered broiler and
broil five minutes, having cap side down first half of broiling. Serve
on circular pieces of buttered dry toast. Put a small piece of butter in
each cap, sprinkle with salt and pepper, and serve as soon as butter has
melted. Care must be taken, in removing from broiler, to keep mushrooms
cap side up, to prevent loss of juices.



\needspace{15\baselineskip}
\section*{Baked Mushrooms In Cream}

Wash twelve large mushrooms. Remove stems, and peel caps. Put in a
shallow buttered pan, cap side up. Sprinkle with salt and pepper, and
dot over with butter; add two-thirds cup cream. Bake ten minutes in a
hot oven. Place on pieces of dry toast, and pour over them cream
remaining in pan.



\needspace{15\baselineskip}
\section*{Sautéd Mushrooms}

Wash, remove stems, peel caps, and break in pieces; there should be one
cup of mushrooms. Put two tablespoons butter in a hot omelet pan; when
melted, add mushrooms which have been dredged with flour, few drops
onion juice, one-fourth teaspoon salt, a few grains pepper, and cook
five minutes. Add one teaspoon finely chopped parsley and one-fourth cup
boiling water. Cook two minutes, and serve on dry toast.



\needspace{15\baselineskip}
\section*{Mushrooms À La Sabine}

Wash one-half pound mushrooms, remove stems, and peel caps. Sprinkle
with salt and pepper, dredge with flour, and cook three minutes in a hot
frying-pan, with two tablespoons butter. Add one and one-third cups
Brown Sauce, and cook slowly five minutes. Sprinkle with three
tablespoons grated cheese. As soon as cheese is melted, arrange
mushrooms on pieces of toast, and pour over sauce. Garnish with parsley.



\needspace{15\baselineskip}
\section*{Mushrooms À L'Algonquin}

Wash large selected mushrooms. Remove stems, peel caps, and sauté caps
in butter. Place in a small buttered shallow pan, cap side being up;
place on each a large oyster, sprinkle with salt and pepper, and place
on each a bit of butter. Cook in a hot oven until oysters are plump.
Serve with Brown or Béchamel Sauce.



\needspace{15\baselineskip}
\section*{Mushrooms Allamande}

Clean mushroom caps and sauté in butter. Put together in pairs, cover
with Allamande Sauce, dip in crumbs, egg, and crumbs again, fry in deep
fat, and drain on brown paper.

\textbf{Allamande Sauce.} Melt three tablespoons butter, add one-third cup
flour, and pour on gradually one cup White Stock; then add one egg yolk
and season with salt, pepper, and lemon juice.



\needspace{15\baselineskip}
\section*{Stuffed Mushrooms}

Wash twelve large mushrooms. Remove stems, chop finely, and peel caps.
Melt three tablespoons butter, add one-half tablespoon finely chopped
shallot and chopped stems, then cook ten minutes. Add one and one-half
tablespoons flour, chicken stock to moisten, a slight grating of nutmeg,
one-half teaspoon finely chopped parsley, and salt and pepper to taste.
Cool mixture and fill caps, well rounding over top. Cover with buttered
cracker crumbs, and bake fifteen minutes in a hot oven.



\needspace{15\baselineskip}
\section*{Mushrooms Under Glass I}

Cover the bottom of an individual baking-dish with circular pieces of
toasted bread. Arrange mushroom caps on toast, sprinkle with salt and
pepper, dot over with butter, and pour over a small quantity of hot
cream. Cover, and bake twenty minutes.

Individual dishes with bell-shaped glass covers may be bought at
first-class kitchen furnishers. These dishes are sent to table with
covers left on, that the fine flavor of the prepared viand may all be
retained.



\needspace{15\baselineskip}
\section*{Mushrooms Under Glass II}


\begin{minipage}{1.0\textwidth}
{\setlength{\multicolsep}{0pt}\setlength{\columnsep}{2em}\raggedcolumns%
\begin{multicols}{2}
\begin{itemize}
\setlength{\itemsep}{0pt}
\setlength{\parsep}{0pt}
\item 2 tablespoons butter
\item 1/2 tablespoon lemon juice
\item 1/4 teaspoon salt
\item Few grains pepper
\item 1/4 teaspoon finely chopped parsley
\item Bread
\item 1/4 cup heavy cream
\item Sherry wine
\item Mushrooms
\end{itemize}
\end{multicols}}
\end{minipage}

\vspace{0.3em}
\noindent%
Cream the butter, add lemon juice drop by drop, salt, pepper, and
parsley. Cut bread in circular pieces three-eighths inch thick, then
toast. Put one-half of the sauce on the under side of toast; put toast
on a small baking-dish, pile mushroom caps cleaned and peeled in conical
shape on toast, and pour over cream. Cover with glass and bake about
twenty-five minutes, adding more cream if necessary. Just before serving
add one teaspoon Sherry wine.



\needspace{15\baselineskip}
\section*{Vegetable Soufflé}


\begin{minipage}{1.0\textwidth}
{\setlength{\multicolsep}{0pt}\setlength{\columnsep}{2em}\raggedcolumns%
\begin{multicols}{2}
\begin{itemize}
\setlength{\itemsep}{0pt}
\setlength{\parsep}{0pt}
\item 1/4 cup butter
\item 1/4 cup flour
\item 1/3 cup cream
\item 1/3 cup water in which vegetables were cooked
\item 1 cup cooked vegetables rubbed through a sieve,--carrots, turnips, or
\item onions
\item 3 egg yolks
\item 3 egg whites
\item Salt and pepper
\end{itemize}
\end{multicols}}
\end{minipage}

\vspace{0.3em}
\noindent%
Melt butter, add flour, and pour on gradually cream and water; add
vegetable, yolks of eggs beaten until thick and lemon-colored, and fold
in whites of eggs beaten until stiff; then add seasonings. Turn in a
buttered baking-dish and bake in a slow oven.



\needspace{15\baselineskip}
\section*{Curried Vegetables}

Cook one cup each potatoes and carrots, and one-half cup turnip, cut in
fancy shapes, in boiling salted water until soft. Drain, add one-half
cup canned peas, and pour over a sauce made by cooking two tablespoons
butter with two slices onion five minutes, removing onion, adding two
tablespoons flour, three-fourths teaspoon salt, one-half teaspoon curry
powder, one-fourth teaspoon pepper, few grains celery salt, and pouring
on gradually one cup scalded milk. Sprinkle with finely chopped parsley.



\needspace{15\baselineskip}
\section*{Macedoine Of Vegetables À La Poulette}

Clean carrots and turnips and cut into strips or fancy shapes; there
should be one and one-fourth cups carrots and one-half cup turnips. Cook
separately in boiling salted water until soft. Drain, and add one and
one-fourth cups cooked peas. Reheat in a sauce made of three tablespoons
butter, three tablespoons flour, one cup chicken stock, and one-half cup
cream. Season to taste with pepper and salt, and just before serving add
yolks two eggs and one-half tablespoon lemon juice.








\chapter{Potatoes}




\begin{center}
\begin{tabular}{|l|r|}
\hline
\multicolumn{2}{|l|}{\textbf{Composition}} \\
\hline
Water & 78.9\% \\
\hline
Starch & 18\% \\
\hline
Protein & 2.1\% \\
\hline
Mineral matter & .9\% \\
\hline
Fat & 1\% \\
\hline
\end{tabular}
\end{center}

\vspace{10pt}

\noindent
Potatoes stand pre-eminent among the vegetables used for food. They are
tubers belonging to the Nightshade family; their hardy growth renders
them easy of cultivation in almost any soil or climate, and, resisting
early frosts, they may be raised in a higher latitude than the cereals.

They give needed bulk to food rather than nutriment, and, lacking in
protein, should be used in combination with meat, fish, or eggs.

Potatoes contain an acrid juice, the greater part of which lies near the
skin; it passes into the water during boiling of potatoes, and escapes
with the steam from a baked potato.

Potatoes are best in the fall, and keep well through the winter. By
spring the starch is partially changed to dextrin, giving the potatoes a
sweetness, and when cooked a waxiness. The same change takes place when
potatoes are frozen. To prevent freezing, keep a pail of cold water
standing near them.

Potatoes keep best in a cool dry cellar, in barrels or piled in a bin.
When sprouts appear they should be removed; receiving their nourishment
from the starch, they deteriorate the potato.

New potatoes may be compared to unripe fruit, the starch-grains not
having reached maturity; therefore they should not be given to children
or invalids.



\needspace{15\baselineskip}
\section*{Sweet Potatoes}

Sweet potatoes, although analogous to white potatoes, are fleshy roots
of the plant, belong to a different family (Convolvulus), and contain a
much larger percentage of sugar. Our own country produces large
quantities of sweet potatoes, which may be grown as far north as New
Jersey and Southern Michigan. Kiln-dried sweet potatoes are the best, as
they do not so quickly spoil.



\needspace{15\baselineskip}
\section*{Baked Potatoes}

Select smooth, medium-sized potatoes. Wash, using a vegetable brush, and
place in dripping-pan. Bake in hot oven forty minutes or until soft,
remove from oven, and serve at once. If allowed to stand, unless the
skin is ruptured for escape of steam, they become soggy. Properly baked
potatoes are more easily digested than potatoes cooked in any other way,
as some of the starch is changed to dextrin by the intense heat. They
are better cooked in boiling water than baked in a slow oven.



\needspace{15\baselineskip}
\section*{Boiled Potatoes}

Select potatoes of uniform size. Wash, pare, and drop at once in cold
water to prevent discoloration; soak one-half hour in the fall, and one
to two hours in winter and spring. Cook in boiling salted water until
soft, which is easily determined by piercing with a skewer. For seven
potatoes allow one tablespoon salt, and boiling water to cover. Drain
from water, and keep uncovered in warm place until serving time. Avoid
sending to table in a covered vegetable dish. In boiling large potatoes,
it often happens that outside is soft, while centre is underdone. To
finish cooking without potatoes breaking apart, add one pint cold water,
which drives heat to centre, thus accomplishing the cooking.



\needspace{15\baselineskip}
\section*{Riced Potatoes}

Force hot boiled potatoes through a potato ricer or coarse strainer.
Serve lightly piled in a hot vegetable dish.



\needspace{15\baselineskip}
\section*{Mashed Potatoes}

To five riced potatoes add three tablespoons butter, one teaspoon salt,
few grains pepper, and one-third cup hot milk; beat with fork until
creamy, reheat, and pile lightly in hot dish.



\needspace{15\baselineskip}
\section*{Potato Omelet}

Prepare Mashed Potatoes, turn in hot omelet pan greased with one
tablespoon butter, spread evenly, cook slowly until browned underneath,
and fold as an omelet.



\needspace{15\baselineskip}
\section*{Potato Border}

Place a buttered mould on platter, build around it a wall of hot Mashed
Potatoes, three and one-half inches high by one inch deep, smooth, and
crease with case knife. Remove mould, fill with creamed meat or fish,
and reheat in oven before serving.



\needspace{15\baselineskip}
\section*{Escalloped Potatoes}

Mash, pare, soak, and cut four potatoes in one-fourth inch slices. Put a
layer in buttered baking-dish, sprinkle with salt and pepper, dredge
with flour, and dot over with one-half tablespoon butter; repeat. Add
hot milk until it may be seen through top layer, bake one and one-fourth
hours or until potato is soft.



\needspace{15\baselineskip}
\section*{Potatoes À La Hollandaise}

Mash, pare, soak, and cut potatoes in one-fourth inch slices, shape with
French vegetable cutters; or cut in one-half inch cubes. Cover three
cups potato with White Stock, cook until soft, and drain. Cream
one-third cup butter, add one tablespoon lemon juice, one-half teaspoon
salt, and few grains of cayenne. Add to potatoes, cook three minutes,
and add one-half tablespoon finely chopped parsley.



\needspace{15\baselineskip}
\section*{Chambery Potatoes}

Wash, pare, and thinly slice potatoes, using vegetable slicer. Let stand
one-half hour in cold water, then drain, and dry between towels. Arrange
in layers in a well buttered iron frying-pan, having pan three-fourths
full, seasoning each layer with salt and pepper, and brushing over with
melted butter. Cook in a moderate oven until soft and well browned.



\needspace{15\baselineskip}
\section*{Potatoes Baked In Half Shell}

Select six medium-sized potatoes and bake, following recipe for Baked
Potatoes. Remove from oven, cut slice from top of each, and scoop out
inside. Mash, add two tablespoons butter, salt, pepper, and three
tablespoons hot milk; then add whites two eggs well beaten. Refill
skins, and bake five to eight minutes in very hot oven. Potatoes may be
sprinkled with grated cheese before putting in oven.



\needspace{15\baselineskip}
\section*{Duchess Potatoes}

To two cups hot riced potatoes add two tablespoons butter, one-half
teaspoon salt, and yolks of three eggs slightly beaten. Shape, using
pastry bag and tube, in form of baskets, pyramids, crowns, leaves,
roses, etc. Brush over with beaten egg diluted with one teaspoon water,
and brown in a hot oven.



\needspace{15\baselineskip}
\section*{Maître D'Hôtel Potatoes}

Wash, pare, and shape potatoes in balls, using a French vegetable
cutter, or cut potatoes in one-half inch cubes. There should be two
cups. Soak fifteen minutes in cold water, and cook in boiling salted
water to cover until soft. Drain, and add Maître d'Hôtel Butter.



\needspace{15\baselineskip}
\section*{Maître D'Hôtel Butter}

Cream three tablespoons butter, add one teaspoon lemon juice, one-half
teaspoon salt, one-eighth teaspoon pepper, and one-half tablespoon
finely chopped parsley.



\needspace{15\baselineskip}
\section*{Franconia Potatoes}

Prepare as for Boiled Potatoes, and parboil ten minutes; drain, and
place in pan in which meat is roasting; bake until soft, basting with
fat in pan when basting meat. Time required for baking about forty
minutes. Sweet potatoes may be prepared in the same way.



\needspace{15\baselineskip}
\section*{Brabant Potatoes}

Prepare same as for Boiled Potatoes, using small potatoes, and trim
egg-shaped; parboil ten minutes, drain, and place in baking-pan and bake
until soft, basting three times with melted butter.



\needspace{15\baselineskip}
\section*{Anna Potatoes}

Wash and pare medium-sized potatoes. Cut lengthwise in one-fourth inch
slices, and fasten in fan shapes, with small wooden skewers, allowing
five slices of potato to each skewer. Parboil ten minutes, drain, then
place in a dripping-pan, and bake in a hot oven until soft, basting
every three minutes with butter or some other fat.



\needspace{15\baselineskip}
\section*{Persillade Potatoes}

Wash and pare small potatoes, and cut in shapes of large olives. Cook in
boiling salted water until soft. Drain, and let stand to dry off. Turn
into hot serving dish, pour over clarified butter, sprinkle generously
with paprika, and send to table at once.



\needspace{15\baselineskip}
\section*{Potato Balls}

Select large potatoes, wash, pare, and soak. Shape in balls with a
French vegetable cutter. Cook in boiling salted water until soft; drain,
and to one pint potatoes add one cup Thin White Sauce. Turn into hot
dish, and sprinkle with finely chopped parsley.



\needspace{15\baselineskip}
\section*{Hongroise Potatoes}

Wash, pare, and cut potatoes in one-third inch cubes,--there should be
three cups; parboil three minutes, and drain. Add one-third cup butter,
and cook on back of range until potatoes are soft and slightly browned.
Melt two tablespoons butter, add a few drops onion juice, two
tablespoons flour, and pour on gradually one cup hot milk. Season with
salt and paprika, then add one egg yolk. Pour sauce over potatoes, and
sprinkle with finely chopped parsley.



\needspace{15\baselineskip}
\section*{Fried Potatoes}


\needspace{15\baselineskip}
\subsection*{Shadow Potatoes (Saratoga Chips)}

Wash and pare potatoes. Slice thinly (using vegetable slicer) into a
bowl of cold water. Let stand two hours, changing water twice. Drain,
plunge in a kettle of boiling water, and boil one minute. Drain again,
and cover with cold water. Take from water and dry between towels. Fry
in deep fat until light brown, keeping in motion with a skimmer. Drain
on brown paper and sprinkle with salt.



\needspace{15\baselineskip}
\subsection*{Shredded Potatoes}

Wash, pare, and cut potatoes in one-eighth inch slices. Cut slices in
one-eighth inch strips. Soak one hour in cold water. Take from water,
dry between towels, and fry in deep fat. Drain on brown paper and
sprinkle with salt. Serve around fried or baked fish.



\needspace{15\baselineskip}
\subsection*{Lattice Potatoes}

Wash and pare potatoes. Slice, using a vegetable slicer which comes for
this purpose, and let stand in a bowl of cold water two hours. Drain,
and dry between towels. Fry in deep fat, drain on brown paper, and
sprinkle with salt.



\needspace{15\baselineskip}
\subsection*{Potato Nests}

Wash, pare, and cut potatoes in thin strips, using same slicer as for
Lattice Potatoes. Soak in cold water fifteen minutes, drain, and dry
between towels. Line a fine wire strainer of four-inch diameter, and
having a wire handle, with potatoes, place a similar strainer, having a
two and one-half inch diameter, in larger strainer, thus holding
potatoes in nest shapes. Fry in deep fat, taking care that the fat does
not reach too high a temperature at first. Keep the small strainer in
place during frying with a long handled spoon. Carefully remove nests
from strainers. Drain on brown paper, and sprinkle with salt. Fill with
small fillets of fried fish or fried smelts.



\needspace{15\baselineskip}
\subsection*{French Fried Potatoes}

Wash and pare small potatoes, cut in eighths lengthwise, and soak one
hour in cold water. Take from water, dry between towels, and fry in deep
fat. Drain on brown paper and sprinkle with salt.

Care must be taken that fat is not too hot, as potatoes must be cooked
as well as browned.



\needspace{15\baselineskip}
\subsection*{O'Brion Potatoes}

Fry three cups potato cubes or balls in deep fat, drain on brown paper,
and sprinkle with salt. Cook one slice onion in one and one-half
tablespoons butter three minutes, remove onion, and add to butter three
canned pimentoes cut in small pieces. When thoroughly heated add
potatoes; stir until well mixed, turn into serving dish, and sprinkle
with finely chopped parsley.



\needspace{15\baselineskip}
\subsection*{Potato Marbles}

Wash and pare potatoes. Shape in balls, using a French vegetable cutter.
Soak fifteen minutes in cold water; take from water and dry between
towels. Fry in deep fat, drain, and sprinkle with salt.



\needspace{15\baselineskip}
\subsection*{Fried Potato Balls}

To one cup hot riced potatoes add one tablespoon butter, one-fourth
teaspoon salt, one-eighth teaspoon celery salt, and few grains cayenne.
Cool slightly, and add one-half beaten egg and one-half teaspoon finely
chopped parsley. Shape in small balls, roll in flour, fry in deep fat,
and drain.



\needspace{15\baselineskip}
\subsection*{Potatoes, Somerset Style}

To two cups hot riced potatoes add two tablespoons butter, one-half cup
grated mild cheese, yolks three eggs, slightly beaten, one-half teaspoon
salt, and a few grains cayenne. Shape in form of birds, dip in crumbs,
egg, and crumbs, insert slices of raw potato cut to represent wings and
tail, and cloves to represent eyes. Fry in deep fat and drain on brown
paper.



\needspace{15\baselineskip}
\subsection*{Potato Fritters}


\begin{minipage}{1.0\textwidth}
{\setlength{\multicolsep}{0pt}\setlength{\columnsep}{2em}\raggedcolumns%
\begin{multicols}{2}
\begin{itemize}
\setlength{\itemsep}{0pt}
\setlength{\parsep}{0pt}
\item 2 cups hot riced potatoes
\item 2 tablespoons cream
\item 2 tablespoons wine
\item 1 teaspoon salt
\item Few gratings nutmeg
\item Few grains cayenne
\item 3 eggs
\item 4 egg yolks
\item 1/2 cup flour
\end{itemize}
\end{multicols}}
\end{minipage}

\vspace{0.3em}
\noindent%
Add cream, wine, and seasonings to potatoes; then add eggs well beaten,
having bowl containing mixture in pan of ice-water, and beat until cold.
Add flour, and when well mixed, drop by spoonfuls in deep fat, fry until
delicately browned, and drain on brown paper.



\needspace{15\baselineskip}
\subsection*{Potato Curls}

Wash and pare large long potatoes. Shape with a potato curler, soak one
hour in cold water, drain, dry between towels, fry in deep fat, drain,
and sprinkle with salt.



\needspace{15\baselineskip}
\subsection*{Potato Croquettes}


\begin{minipage}{1.0\textwidth}
{\setlength{\multicolsep}{0pt}\setlength{\columnsep}{2em}\raggedcolumns%
\begin{multicols}{2}
\begin{itemize}
\setlength{\itemsep}{0pt}
\setlength{\parsep}{0pt}
\item 2 cups hot riced potatoes
\item 2 tablespoons butter
\item 1/2 teaspoon salt
\item 1/8 teaspoon pepper
\item 1/4 teaspoon celery salt
\item Few grains cayenne
\item Few drops onion juice
\item Yolk 1 egg
\item 1 teaspoon finely chopped parsley
\end{itemize}
\end{multicols}}
\end{minipage}

\vspace{0.3em}
\noindent%
Mix ingredients in order given, and beat thoroughly. Shape, dip in
crumbs, egg, and crumbs again, fry one minute in deep fat, and drain on
brown paper. Croquettes are shaped in a variety of forms. The most
common way is to first form a smooth ball by rolling one rounding
tablespoon of mixture between hands. Then roll on a board until of
desired length, and flatten ends.



\needspace{15\baselineskip}
\subsection*{French Potato Croquettes}


\begin{itemize}
\setlength{\itemsep}{0pt}
\setlength{\parsep}{0pt}
\item 2 cups hot riced potatoes
\item 2 tablespoons butter
\item 3 egg yolks
\item 1/2 teaspoon salt
\item Few grains cayenne
\end{itemize}

\vspace{-0.5em}
\noindent%
Mix ingredients in order given, and beat thoroughly. Shape in balls,
then in rolls, pointed at ends. Roll in flour, mark in three places on
top of each with knife-blade to represent a small French loaf. Fry in
deep fat, and drain on brown paper.







\needspace{15\baselineskip}
\subsection*{Potato Apples}


\begin{minipage}{1.0\textwidth}
{\setlength{\multicolsep}{0pt}\setlength{\columnsep}{2em}\raggedcolumns%
\begin{multicols}{2}
\begin{itemize}
\setlength{\itemsep}{0pt}
\setlength{\parsep}{0pt}
\item 2 cups hot riced potatoes
\item 2 tablespoons butter
\item 1/3 cup grated cheese
\item 1/2 teaspoon salt
\item Few grains cayenne
\item Slight grating nutmeg
\item 2 tablespoons thick cream
\item 4 egg yolks
\end{itemize}
\end{multicols}}
\end{minipage}

\vspace{0.3em}
\noindent%
Mix ingredients in order given, and beat thoroughly. Shape in form of
small apples, roll in flour, egg, and crumbs, fry in deep fat, and drain
on brown paper. Insert a clove at both stem and blossom end of each
apple.



\needspace{15\baselineskip}
\subsection*{Potatoes en Surprise}

Make Potato Croquette mixture, omitting parsley. Shape in small nests
and fill with Creamed Chicken, shrimp, or peas. Cover nests with
Croquette mixture, then roll in form of croquettes. Dip in crumbs, egg,
and crumbs again; fry in deep fat, and drain on brown paper.



\needspace{15\baselineskip}
\section*{Sweet Potatoes}


\needspace{15\baselineskip}
\subsection*{Baked Sweet Potatoes}

Prepare and bake same as white potatoes.



\needspace{15\baselineskip}
\subsection*{Sweet Potatoes, Southern Style}

Bake six medium-sized sweet potatoes, remove from oven, cut in halves
lengthwise, and scoop out inside. Mash, add two tablespoons butter, and
cream to moisten. Season with salt and Sherry wine. Refill skins and
bake five minutes in a hot oven.



\needspace{15\baselineskip}
\subsection*{Boiled Sweet Potatoes}

Select potatoes of uniform size. Wash, pare, and cook twenty minutes in
boiling salted water to cover. Many boil sweet potatoes with the skins
on.



\needspace{15\baselineskip}
\subsection*{Mashed Sweet Potatoes}

To two cups riced sweet potatoes add three tablespoons butter, one-half
teaspoon salt, and hot milk to moisten. Beat until light, and pile on a
vegetable dish.



\needspace{15\baselineskip}
\subsection*{Sweet Potatoes, Georgian Style}

Season mashed boiled sweet potatoes with butter, salt, pepper, and
Sherry wine. Moisten with cream, and beat five minutes. Put in a
buttered baking-dish, leaving a rough surface. Pour over a syrup made by
boiling two tablespoons molasses and one teaspoon butter five minutes.
Bake in the oven until delicately browned.



\needspace{15\baselineskip}
\subsection*{Glazed Sweet Potatoes}

Wash and pare six medium-sized potatoes. Cook ten minutes in boiling
salted water. Drain, cut in halves lengthwise, and put in a buttered
pan. Make a syrup by boiling three minutes one-half cup sugar and four
tablespoons water; add one tablespoon butter. Brush potatoes with syrup
and bake fifteen minutes, basting twice with remaining syrup.



\needspace{15\baselineskip}
\subsection*{Sweet Potatoes au Gratin}

Cut five medium-sized cold boiled sweet potatoes in one-third inch
slices. Put a layer in buttered baking-dish, sprinkle with salt, pepper,
and three tablespoons brown sugar, dot over with one tablespoon butter.
Repeat, cover with buttered cracker crumbs, and bake until the crumbs
are brown.



\needspace{15\baselineskip}
\subsection*{Sweet Potatoes en Brochette}

Wash and pare potatoes, and cut in one-third inch slices. Arrange on
skewers in groups of three or four, parboil six minutes, and drain.
Brush over with melted butter, sprinkle with brown sugar, and bake in a
hot oven until well browned.



\needspace{15\baselineskip}
\subsection*{Sweet Potato Balls}

To two cups hot riced sweet potatoes add three tablespoons butter,
one-half teaspoon salt, few grains pepper, and one beaten egg. Shape in
small balls, roll in flour, fry in deep fat, and drain. If potatoes are
very dry, it will be necessary to add hot milk to moisten.



\needspace{15\baselineskip}
\subsection*{Sweet Potato Croquettes}

Prepare mixture for Sweet Potato Balls. Shape in croquettes, dip in
crumbs, egg, and crumbs again, fry in deep fat, and drain.



\needspace{15\baselineskip}
\section*{Warmed Over Potatoes}


\needspace{15\baselineskip}
\subsection*{Potato Cakes}

Shape cold mashed potato in small cakes, and roll in flour. Butter hot
omelet pan, put in cakes, brown one side, turn and brown other side,
adding butter as needed to prevent burning; or pack potato in small
buttered pan as soon as it comes from table, and set aside until ready
for use. Turn from pan, cut in pieces, roll in flour, and cook same as
Potato Cakes.



\needspace{15\baselineskip}
\subsection*{Creamed Potatoes}

Reheat two cups cold boiled potatoes, cut in dice, in one and one-fourth
cups White Sauce I.



\needspace{15\baselineskip}
\subsection*{Potatoes au Gratin}

Put Creamed Potatoes in buttered baking-dish, cover with buttered
crumbs, and bake on centre grate until crumbs are brown.



\needspace{15\baselineskip}
\subsection*{Delmonico Potatoes}

To Potatoes au Gratin add one-third cup grated mild cheese, arranging
potatoes and cheese in alternate layers before covering with crumbs.



\needspace{15\baselineskip}
\subsection*{Potatoes à l'Antlers}

Cook potatoes with jackets on, drain, and let stand twenty-four hours.
Peel, and cut in small cubes. Put into a saucepan with two tablespoons
butter to each two cups potatoes. Sprinkle with salt, and generously
with paprika. Add cream to cover, and cook slowly, forty minutes.



\needspace{15\baselineskip}
\subsection*{Hashed Brown Potatoes}

Try out fat salt pork cut in small cubes, remove scraps; there should be
about one-third cup of fat. Add two cups cold boiled potatoes finely
chopped, one-eighth teaspoon pepper, and salt if needed. Mix potatoes
thoroughly with fat; cook three minutes, stirring constantly; let stand
to brown underneath. Fold as an omelet and turn on hot platter.



\needspace{15\baselineskip}
\subsection*{Sautéd Potatoes}

Cut cold boiled potatoes in one-fourth inch slices, season with salt and
pepper, put in a hot, well-greased frying-pan, brown on one side, turn
and brown on other side.



\needspace{15\baselineskip}
\subsection*{Chartreuse Potatoes}

Cut cold boiled potatoes in one-fourth inch slices, sprinkle with salt,
pepper, and a few drops onion juice, put together in pairs, dip in
Batter I, fry in deep fat, and drain on brown paper.



\needspace{15\baselineskip}
\subsection*{Lyonnaise Potatoes I}

Cook five minutes three tablespoons butter with one small onion cut in
thin slices; add three cold boiled potatoes cut in one-fourth inch
slices and sprinkled with salt and pepper; stir until well mixed with
onion and butter; let stand until potato is brown underneath, fold, and
turn on a hot platter. This dish is much improved and potatoes brown
better by addition of two tablespoons Brown Stock. Sprinkle with finely
chopped parsley if desired.



\needspace{15\baselineskip}
\subsection*{Lyonnaise Potatoes II}

Slice cold boiled potatoes to make two cups. Cook five minutes one and
one-half tablespoons butter with one tablespoon finely chopped onion.
Melt two tablespoons butter, season with salt and pepper, add potatoes,
and cook until potatoes have absorbed butter, occasionally shaking pan.
Add butter and onion, and when well mixed, add one-half tablespoon
finely chopped parsley.



\needspace{15\baselineskip}
\subsection*{Oak Hill Potatoes}

Cut four cold boiled potatoes and six “hard-boiled” eggs in one-fourth
inch slices. Put layer of potatoes in buttered baking-dish, sprinkle
with salt and pepper, cover with layer of eggs; repeat, and pour over
two cups Thin White Sauce. Cover with buttered cracker crumbs and bake
until the crumbs are brown.



\needspace{15\baselineskip}
\subsection*{Curried Potatoes}

Cook one-fourth cup butter with one small onion, finely chopped, until
yellow; add three cups cold boiled potato cubes, and cook until potatoes
have absorbed butter, then add from one-half to three-fourths cup White
Stock, one half tablespoon each curry powder and lemon juice, and salt
and pepper to taste. Cook until potatoes have absorbed stock.





\chapter{Salads And Salad Dressings}



Salads, which constitute a course in almost every dinner, but a few
years since seldom appeared on the table. They are now made in an
endless variety of ways, and are composed of meat, fish, vegetables
(alone or in combination) or fruits, with the addition of a dressing.
The salad plants, lettuce, watercress, chiccory, cucumbers, etc.,
contain but little nutriment, but are cooling, refreshing, and assist in
stimulating the appetite. They are valuable for the water and potash
salts they contain. The olive oil, which usually forms the largest part
of the dressing, furnishes nutriment, and is of much value to the
system.

Salads made of greens should always be served crisp and cold. The
vegetables should be thoroughly washed, allowed to stand in cold or
ice-water until crisp, then drained and spread on a towel and set aside
in a cold place until serving time. See Lettuce, page 294. Dressing may
be added at table or just before sending to table. If greens are allowed
to stand in dressing they will soon wilt. It should be remembered that
winter greens are raised under glass and should be treated as any other
hothouse plant. Lettuce will be affected by a change of temperature and
wilt just as quickly as delicate flowers.

Canned or cold cooked left-over vegetables are well utilized in salads,
but are best mixed with French Dressing and allowed to stand in a cold
place one hour before serving. Where several vegetables are used in the
same salad they should be marinated separately, and arranged for serving
just before sending to table.

Meat for salads should be freed from skin and gristle, cut in small
cubes, and allowed to stand mixed with French Dressing before combining
with vegetables. Fish should be flaked or cut in cubes.

Where salads are dressed at table, first sprinkle with salt and pepper,
add oil, and lastly vinegar. If vinegar is added before oil, the greens
will become wet, and oil will not cling, but settle to bottom of bowl.

\textbf{A Chapon.} Remove a small piece from end of French loaf and rub over
with a clove of garlic, first dipped in salt. Place in bottom of salad
bowl before arranging salad. A chapon is often used in vegetable salads,
and gives an agreeable additional flavor.

\textbf{To Marinate.} The word marinate, as used in cookery, means to add salt,
pepper, oil, and vinegar to a salad ingredient or mixture, then allow to
let stand until well seasoned.



\needspace{15\baselineskip}
\section*{Salad Dressings}


\needspace{15\baselineskip}
\subsection*{French Dressing}


\begin{itemize}
\setlength{\itemsep}{0pt}
\setlength{\parsep}{0pt}
\item 1/2 teaspoon salt
\item 1/4 teaspoon pepper
\item 2 tablespoons vinegar
\item 4 tablespoons olive oil
\end{itemize}

\vspace{-0.5em}
\noindent%
Mix ingredients and stir until well blended. Some prefer the addition of
a few drops onion juice. French Dressing is more easily prepared and
largely used than any other dressing.



\needspace{15\baselineskip}
\subsection*{Parisian French Dressing}


\begin{minipage}{1.0\textwidth}
{\setlength{\multicolsep}{0pt}\setlength{\columnsep}{2em}\raggedcolumns%
\begin{multicols}{2}
\begin{itemize}
\setlength{\itemsep}{0pt}
\setlength{\parsep}{0pt}
\item 1/2 cup olive oil
\item 5 tablespoons vinegar
\item 1/2 teaspoon powdered sugar
\item 1 tablespoon finely chopped Bermuda onion
\item 2 tablespoons finely chopped parsley
\item 4 red peppers
\item 8 green peppers
\item 1 teaspoon salt
\end{itemize}
\end{multicols}}
\end{minipage}

\vspace{0.3em}
\noindent%
Mix ingredients in the order given. Let stand one hour, then stir
vigorously for five minutes. This is especially fine with lettuce,
romaine, chiccory, or endive. The red and green peppers are the small
ones found in pepper sauce.



\needspace{15\baselineskip}
\subsection*{Club French Dressing}


\begin{itemize}
\setlength{\itemsep}{0pt}
\setlength{\parsep}{0pt}
\item 1/2 teaspoon salt
\item 1/4 teaspoon pepper
\item 2 tablespoons brandy
\item 2 tablespoons Tarragon vinegar
\item 2 tablespoons olive oil
\end{itemize}

\vspace{-0.5em}
\noindent%
Mix ingredients and stir until well blended.



\needspace{15\baselineskip}
\subsection*{Curry Dressing}


\begin{itemize}
\setlength{\itemsep}{0pt}
\setlength{\parsep}{0pt}
\item 3/4 teaspoon salt
\item 1/4 teaspoon curry powder
\item 1/4 teaspoon pepper
\item 5 tablespoons olive oil
\item 3 tablespoons vinegar
\end{itemize}

\vspace{-0.5em}
\noindent%
Mix ingredients in order given and stir until well blended.



\needspace{15\baselineskip}
\subsection*{Cream Dressing I}


\begin{minipage}{1.0\textwidth}
{\setlength{\multicolsep}{0pt}\setlength{\columnsep}{2em}\raggedcolumns%
\begin{multicols}{2}
\begin{itemize}
\setlength{\itemsep}{0pt}
\setlength{\parsep}{0pt}
\item 1/2 tablespoon salt
\item 1/2 tablespoon mustard
\item 3/4 tablespoon sugar
\item 1 egg slightly beaten
\item 2 1/2 tablespoons melted butter
\item 3/4 cup cream
\item 1/4 cup vinegar
\end{itemize}
\end{multicols}}
\end{minipage}

\vspace{0.3em}
\noindent%
Mix ingredients in order given, adding vinegar very slowly. Cook over
boiling water, stirring constantly until mixture thickens, strain and
cool.



\needspace{15\baselineskip}
\subsection*{Cream Dressing II}


\begin{minipage}{1.0\textwidth}
{\setlength{\multicolsep}{0pt}\setlength{\columnsep}{2em}\raggedcolumns%
\begin{multicols}{2}
\begin{itemize}
\setlength{\itemsep}{0pt}
\setlength{\parsep}{0pt}
\item 1 teaspoon mustard
\item 1 teaspoon salt
\item 2 teaspoons flour
\item 1 1/2 teaspoons powdered sugar
\item Few grains cayenne
\item 1 teaspoon melted butter
\item Yolk 1 egg
\item 1/3 cup hot vinegar
\item 1/2 cup thick cream
\end{itemize}
\end{multicols}}
\end{minipage}

\vspace{0.3em}
\noindent%
Mix dry ingredients, add butter, egg, and vinegar slowly. Cook over
boiling water, stirring constantly, until mixture thickens; cool, and
add to heavy cream, beaten until stiff.



\needspace{15\baselineskip}
\subsection*{Boiled Dressing I}


\begin{minipage}{1.0\textwidth}
{\setlength{\multicolsep}{0pt}\setlength{\columnsep}{2em}\raggedcolumns%
\begin{multicols}{2}
\begin{itemize}
\setlength{\itemsep}{0pt}
\setlength{\parsep}{0pt}
\item 1/2 tablespoon salt
\item 1 teaspoon mustard
\item 1 1/2 tablespoons sugar
\item Few grains cayenne
\item 1/2 tablespoon flour
\item 4 egg yolks
\item 1 1/2 tablespoons melted butter
\item 3/4 cup milk
\item 1/4 cup vinegar
\end{itemize}
\end{multicols}}
\end{minipage}

\vspace{0.3em}
\noindent%
Mix dry ingredients, add yolks of eggs slightly beaten, butter, milk,
and vinegar very slowly. Cook over boiling water until mixture thickens;
strain and cool.



\needspace{15\baselineskip}
\subsection*{Boiled Dressing II}


\begin{minipage}{1.0\textwidth}
{\setlength{\multicolsep}{0pt}\setlength{\columnsep}{2em}\raggedcolumns%
\begin{multicols}{2}
\begin{itemize}
\setlength{\itemsep}{0pt}
\setlength{\parsep}{0pt}
\item 4 egg yolks
\item 1/2 cup olive oil
\item 4 tablespoons vinegar
\item 1 tablespoon lemon juice
\item 1 1/2 teaspoons salt
\item 3 teaspoons powdered sugar
\item 1 pint whipped cream
\end{itemize}
\end{multicols}}
\end{minipage}

\vspace{0.3em}
\noindent%
Beat yolks of eggs slightly, add gradually one-half of the oil and lemon
juice. Cook in double boiler until mixture thickens; chill, and add
gradually remaining oil, salt, and sugar. Just before serving add cream.



\needspace{15\baselineskip}
\subsection*{German Dressing}


\begin{itemize}
\setlength{\itemsep}{0pt}
\setlength{\parsep}{0pt}
\item 1/2 cup thick cream
\item 3 tablespoons vinegar
\item 1/4 teaspoon salt
\item Few grains pepper
\end{itemize}

\vspace{-0.5em}
\noindent%
Beat cream until stiff, using Dover Egg-beater. Add salt, pepper, and
vinegar very slowly, continuing the beating.



\needspace{15\baselineskip}
\subsection*{Chicken Salad Dressing}


\begin{minipage}{1.0\textwidth}
{\setlength{\multicolsep}{0pt}\setlength{\columnsep}{2em}\raggedcolumns%
\begin{multicols}{2}
\begin{itemize}
\setlength{\itemsep}{0pt}
\setlength{\parsep}{0pt}
\item 1/2 cup rich chicken stock
\item 1/2 cup vinegar
\item 5 egg yolks
\item 2 tablespoons mixed mustard
\item 1 teaspoon salt
\item 1/4 teaspoon pepper
\item Few grains cayenne
\item 1/2 cup thick cream
\item 1/3 cup melted butter
\end{itemize}
\end{multicols}}
\end{minipage}

\vspace{0.3em}
\noindent%
Reduce stock in which a fowl has been cooked to one-half cupful. Add
vinegar, yolks of eggs slightly beaten, mustard, salt, pepper, and
cayenne. Cook over boiling water, stirring constantly until mixture
thickens. Strain, add cream and melted butter, then cool.



\needspace{15\baselineskip}
\subsection*{Oil Dressing I}


\begin{minipage}{1.0\textwidth}
{\setlength{\multicolsep}{0pt}\setlength{\columnsep}{2em}\raggedcolumns%
\begin{multicols}{2}
\begin{itemize}
\setlength{\itemsep}{0pt}
\setlength{\parsep}{0pt}
\item 4 “hard-boiled” eggs
\item 4 tablespoons oil
\item 4 tablespoons vinegar
\item 1/2 tablespoon sugar
\item 1/2 teaspoon mustard
\item 1/2 teaspoon salt
\item Few grains cayenne
\item 1 egg white
\end{itemize}
\end{multicols}}
\end{minipage}

\vspace{0.3em}
\noindent%
Force yolks of “hard-boiled” eggs through a strainer, then work, using a
silver or wooden spoon, until smooth. Add sugar, mustard, salt, and
cayenne, and when well blended add gradually oil and vinegar, stirring
and beating until thoroughly mixed; then cut and fold in white of egg
beaten until stiff.



\needspace{15\baselineskip}
\subsection*{Oil Dressing II}


\begin{minipage}{1.0\textwidth}
{\setlength{\multicolsep}{0pt}\setlength{\columnsep}{2em}\raggedcolumns%
\begin{multicols}{2}
\begin{itemize}
\setlength{\itemsep}{0pt}
\setlength{\parsep}{0pt}
\item 1 1/2 teaspoons mustard
\item 1 teaspoon salt
\item 2 teaspoons powdered sugar
\item Few grains cayenne
\item 2 tablespoons oil
\item 1/3 cup vinegar diluted with cold water to make one-half cup
\item 2 eggs, slightly beaten
\end{itemize}
\end{multicols}}
\end{minipage}

\vspace{0.3em}
\noindent%
Mix dry ingredients, add egg and oil gradually, stirring constantly
until thoroughly mixed; then add diluted vinegar. Cook over boiling
water until mixture thickens; strain and cool.



\needspace{15\baselineskip}
\subsection*{Mayonnaise Dressing I}


\begin{minipage}{1.0\textwidth}
{\setlength{\multicolsep}{0pt}\setlength{\columnsep}{2em}\raggedcolumns%
\begin{multicols}{2}
\begin{itemize}
\setlength{\itemsep}{0pt}
\setlength{\parsep}{0pt}
\item 1 teaspoon mustard
\item 1 teaspoon salt
\item 1 teaspoon powdered sugar
\item Few grains cayenne
\item 4 egg yolks
\item 2 tablespoons lemon juice
\item 2 tablespoons vinegar
\item 1 1/2 cups olive oil
\end{itemize}
\end{multicols}}
\end{minipage}

\vspace{0.3em}
\noindent%
Mix dry ingredients, add egg yolks, and when well mixed add one-half
teaspoon of vinegar. Add oil gradually, at first drop by drop, and stir
constantly. As mixture thickens, thin with vinegar or lemon juice. Add
oil, and vinegar or lemon juice alternately, until all is used, stirring
or beating constantly. If oil is added too rapidly, dressing will have a
curdled appearance. A smooth consistency may be restored by taking yolk
of another egg and adding curdled mixture slowly to it. It is desirable
to have bowl containing mixture placed in a larger bowl of crushed ice,
to which a small quantity of water has been added. Olive oil for making
Mayonnaise should always be thoroughly chilled. A silver fork, wire
whisk, small wooden spoon, or Dover Egg-beater may be used as preferred.
If one has a Keystone Egg-beater, dressing may be made very quickly by
its use. Mayonnaise should be stiff enough to hold its shape. It soon
liquefies when added to meat or vegetables; therefore it should be added
just before serving time.



\needspace{15\baselineskip}
\subsection*{Mayonnaise Dressing II}

Use same ingredients as for Mayonnaise Dressing I, adding mashed yolk of
a “hard-boiled” egg to dry ingredients.




\needspace{15\baselineskip}
\subsection*{Cream Mayonnaise Dressing}

To Mayonnaise Dressing I or II add one-third cup thick cream, beaten
until stiff. This recipe should be used only when dressing is to be
eaten the day it is made.



\needspace{15\baselineskip}
\subsection*{Green Mayonnaise}

Color Mayonnaise Dressing 1 with juices expressed from parsley and
watercress, using one-half as much parsley as watercress. To obtain
coloring, break greens in pieces, pound in a mortar until thoroughly
macerated, then squeeze through cheese-cloth. Lobster coral, rubbed
through a fine sieve, added to Mayonnaise, makes \textit{Red Mayonnaise}.



\needspace{15\baselineskip}
\subsection*{Potato Mayonnaise}


\begin{minipage}{1.0\textwidth}
{\setlength{\multicolsep}{0pt}\setlength{\columnsep}{2em}\raggedcolumns%
\begin{multicols}{2}
\begin{itemize}
\setlength{\itemsep}{0pt}
\setlength{\parsep}{0pt}
\item 1 Very small baked potato
\item 1 teaspoon mustard
\item 1 teaspoon salt
\item 1 teaspoon powdered sugar
\item 2 tablespoons vinegar
\item 3/4 cup olive oil
\end{itemize}
\end{multicols}}
\end{minipage}

\vspace{0.3em}
\noindent%
Remove and mash the inside of potato. Add mustard, salt, and powdered
sugar; add one tablespoon vinegar, and rub mixture through a fine sieve.
Add slowly oil and remaining vinegar. By the taste one would hardly
realize eggs were not used in the making.



\needspace{15\baselineskip}
\section*{Salads}


\needspace{15\baselineskip}
\subsection*{Dressed Lettuce}

Prepare lettuce as directed on page 294. Serve with French Dressing.



\needspace{15\baselineskip}
\subsection*{Lettuce and Cucumber Salad}

Place a chapon in bottom of salad bowl. Wash, drain, and dry one head
lettuce, arrange in bowl, and place between leaves one cucumber cut in
thin slices. Serve with French Dressing.



\needspace{15\baselineskip}
\subsection*{Lettuce and Radish Salad}

Prepare and arrange as for Dressed Lettuce. Place between leaves six
radishes which have been washed, scraped, and cut in thin slices.
Garnish with round radishes cut to represent tulips. See page 299. Serve
with French Dressing.



\needspace{15\baselineskip}
\subsection*{Lettuce and Tomato Salad}

Peel and chill three tomatoes. Cut in halves crosswise, arrange each
half on a lettuce leaf. Garnish with Mayonnaise Dressing forced through
a pastry bag and tube. If tomatoes are small, cut in quarters, and allow
one tomato to each lettuce leaf.



\needspace{15\baselineskip}
\subsection*{Dressed Watercress}

Wash, remove roots, drain, and chill watercress. Arrange in salad dish,
and serve with French Dressing.



\needspace{15\baselineskip}
\subsection*{Cucumber Salad}

Remove thick slices from both ends of a cucumber, cut off a thick
paring, and with a sharp-pointed knife cut five parallel grooves
lengthwise of cucumber at equal distances; then cut in thin parallel
slices crosswise, keeping cucumber in its original shape. Arrange on
lettuce leaves, and pour over Parisian French Dressing. Serve with fish
course.



\needspace{15\baselineskip}
\subsection*{Watercress and Cucumber Salad}

Prepare watercress and add one cucumber, pared, chilled, and cut in
one-half inch dice. Serve with French Dressing.



\needspace{15\baselineskip}
\subsection*{Cucumber and Tomato Salad}

Arrange sliced tomatoes on a bed of lettuce leaves. Pile on each slice,
cucumber cubes cut one-half inch square. Serve with French or Mayonnaise
Dressing.



\needspace{15\baselineskip}
\subsection*{Cucumber Cups with Lettuce}

Pare cucumbers, cut in quarters crosswise, remove centres from pieces,
arrange on lettuce leaves, and fill cups with Sauce Tartare (see p.
277).



\needspace{15\baselineskip}
\subsection*{Cucumber Baskets}

Select three long, regular-shaped cucumbers; cut a piece from both the
stem and blossom end of each; then cut in halves crosswise. Cut two
pieces from each section, leaving remaining piece in shape of basket
with handle. Remove pulp and seeds, in sufficiently large pieces to cut
in cubes for refilling one-half the baskets, the remaining half being
filled with pieces of tomatoes. Arrange baskets on lettuce leaves,
alternating the fillings, and pour over French Dressing.



\needspace{15\baselineskip}
\subsection*{Dressed Celery}

Wash, scrape, and cut stalks of celery in thin slices. Mix with Cream
Dressing I.



\needspace{15\baselineskip}
\subsection*{Celery and Cabbage Salad}

Remove outside leaves from a small solid white cabbage, and cut off
stalk close to leaves. Cut out centre, and with a sharp knife shred
finely. Let stand one hour in cold or ice water. Drain, wring in double
cheese-cloth, to make as dry as possible. Mix with equal parts celery
cut in small pieces. Moisten with Cream Dressing and refill cabbage.
Arrange on a folded napkin and garnish with celery tips and parsley
between folds of napkin and around top of cabbage.



\needspace{15\baselineskip}
\subsection*{Asparagus Salad}

Drain and rinse stalks of canned asparagus. Cut rings from a bright red
pepper one-third inch wide. Place three or four stalks in each ring.
Arrange on lettuce leaves and serve with French Dressing, to which has
been added one-half tablespoon tomato catsup.



\needspace{15\baselineskip}
\subsection*{Corn Salad}

Drain one can corn and season with mustard and onion juice. Marinate
with French Dressing, let stand one hour, then drain. Arrange on a bed
of lettuce or chiccory.



\needspace{15\baselineskip}
\subsection*{String Bean Salad}

Marinate two cups cold string beans with French Dressing. Add one
teaspoon finely cut chives. Pile in centre of salad dish and arrange
around base thin slices of radishes overlapping one another. Garnish top
with radish cut to represent a tulip.



\needspace{15\baselineskip}
\subsection*{Potato Salad I}

Cut cold boiled potatoes in one-half inch cubes. Sprinkle four cupfuls
with one-half tablespoon salt and one-fourth teaspoon pepper. Add four
tablespoons oil and mix thoroughly; then add two tablespoons vinegar. A
few drops of onion juice may be added, or one-half tablespoon chives
finely cut. Arrange in a mound and garnish with whites and yolks of two
“hard-boiled” eggs, cold boiled red beets, and parsley. Chop whites and
arrange on one-fourth of the mound; chop beets finely, mix with one
tablespoon vinegar, and let stand fifteen minutes; then arrange on
fourths of mounds next to whites. Arrange on remaining fourth of mound
yolks chopped or forced through a potato ricer. Put small sprigs of
parsley in lines dividing beets from eggs; also garnish with parsley at
base.



\needspace{15\baselineskip}
\subsection*{Potato Salad II}

Mix two cups cold boiled riced potatoes and one cup pecan nut meats
broken in pieces. Marinate with French Dressing, and arrange on a bed of
watercress.



\needspace{15\baselineskip}
\subsection*{Hot Potato Salad}

Wash six medium-sized potatoes, and cook in boiling salted water until
soft. Cool, remove skins, and cut in very thin slices. Cover bottom of
baking-dish with potatoes, season with salt and pepper, sprinkle with
finely chopped celery, then with finely chopped parsley. Mix two
tablespoons each tarragon and cider vinegar and four tablespoons olive
oil, and add one slice lemon cut one-third inch thick. Bring to
boiling-point, pour over potatoes, cover, and let stand in oven until
thoroughly warmed.



\needspace{15\baselineskip}
\subsection*{Potato and Celery Salad}

To two cups boiled potatoes cut in one-half inch cubes add one-half cup
finely cut celery and a medium-sized apple, pared, cut in eighths, then
eighths cut in thin slices. Marinate with French Dressing. Arrange in a
mound and garnish with celery tip and sections of bright red apple.



\needspace{15\baselineskip}
\subsection*{Bolivia Salad}

Cut cold boiled potatoes in one-half inch cubes; there should be one and
one-half cups. Add three “hard-boiled” eggs finely chopped, one and
one-half tablespoons finely chopped red peppers, and one-half tablespoon
chopped chives. Pour over Cream Dressing I (see p. 324) and serve in
nests of lettuce leaves.







\needspace{15\baselineskip}
\subsection*{Lettuce Salad}

Wash one head romaine and cut in pieces, using scissors. Mix two cups
cold riced potatoes, one-half pound white mushroom caps peeled and cut
in eighths, and one pound Brazil nut meats (from which skins have been
removed) cut in pieces. Moisten with French Dressing, made by allowing
one-third tarragon vinegar to two-thirds olive oil. Arrange on salad
dish, surround with romaine, and garnish with three peeled mushroom caps
and six Brazil nut meats.



\needspace{15\baselineskip}
\subsection*{Macédoine Salad}

Marinate separately cold cooked cauliflower, peas, and carrots cut in
small cubes, and outer stalks of celery finely cut. Arrange peas and
carrots in alternate piles in centre of a salad dish. Pile cauliflower
on top. Arrange celery in four piles at equal distances. At top of each
pile place a small gherkin cut lengthwise in very thin slices, beginning
at blossom end and cutting nearly to stem end. Open slices to represent
a fan. Place between piles of celery a slice of tomato.

Almost any cold cooked vegetables on hand may be used for a Macédoine
Salad, and if care is taken in arrangement, they make a very attractive
dish.



\needspace{15\baselineskip}
\subsection*{Russian Salad}

Mix one cup each cold cooked carrot cubes and potato cubes, one cup cold
cooked peas, and one cup cold cooked beans, and marinate with French
Dressing. Arrange on lettuce leaves in four sections, and cover each
section with Mayonnaise Dressing. Garnish two sections with small pieces
of smoked salmon, one section with finely chopped whites of “hard
boiled” eggs, and one section with yolks of “hard-boiled” eggs forced
through a strainer. Put small sprigs of parsley or shrimps in lines
dividing sections.



\needspace{15\baselineskip}
\subsection*{Tomatoes Stuffed with Pineapple}

Peel medium-sized tomatoes. Remove thin slice from top of each, and take
out seeds and some of pulp. Sprinkle inside with salt, invert, and let
stand one-half hour. Fill tomatoes with fresh pineapple cut in small
cubes or shredded, and nut meats, using two-thirds pineapple and
one-third nut meats. Mix with Mayonnaise Dressing, garnish with
Mayonnaise, halves of nut meats, and slices cut from tops cut square.
Serve on a bed of lettuce leaves.



\needspace{15\baselineskip}
\subsection*{Stuffed Tomato Salad I}

Peel medium-sized tomatoes. Remove thin slice from top of each and take
out seeds and some of pulp. Sprinkle inside with salt, invert, and let
stand one-half hour. Fill tomatoes with cucumbers cut in small cubes and
mixed with Mayonnaise Dressing. Arrange on lettuce leaves, and garnish
top of each with Mayonnaise Dressing forced through a pastry bag and
tube.



\needspace{15\baselineskip}
\subsection*{Stuffed Tomato Salad II}

Prepare tomatoes same as for Tomatoes Stuffed with Pineapple. Refill
with finely cut celery and apple, using equal parts. Serve with
Mayonnaise, and garnish with shredded lettuce.



\needspace{15\baselineskip}
\subsection*{Stuffed Tomato Salad (German Style)}

Prepare tomatoes same as Tomatoes Stuffed with Pineapple. Shred finely
one-half a cabbage. Let stand two hours in salted water, allowing two
tablespoons salt to one quart water. Cook slowly thirty minutes one-half
cup each cold water and vinegar, with a bit of bay leaf, one-half
teaspoon peppercorns, one-fourth teaspoon mustard seed, and six cloves.
Strain, and pour over cabbage drained from salt water. Let stand two
hours, again drain, and refill tomatoes.



\needspace{15\baselineskip}
\subsection*{Tomato and Horseradish Salad}

Peel and chill tomatoes, cut in halves crosswise, arrange on lettuce
leaves, and garnish with Horseradish Sauce I.



\needspace{15\baselineskip}
\subsection*{Hindoo Salad}

Arrange four slices tomato on a bed of shredded lettuce. On two of the
slices pile shaved celery, on the opposite slices, finely cut
watercress. Garnish with small pieces of tomato shaped with circular
cutter, and serve with French Dressing.



\needspace{15\baselineskip}
\subsection*{Tomato Ciboulettes}

Remove skins from four small tomatoes, and cut in halves crosswise.
Cover with Mayonnaise, and sprinkle with finely chopped chives. Serve on
lettuce leaves.



\needspace{15\baselineskip}
\subsection*{Tomato and Watercress Salad}

Peel and chill large tomatoes, cut in slices one-third inch thick, and
slices in strips one-third inch wide. Arrange on a flat dish to
represent lattice work, and fill in the spaces with watercress. Serve
with French Dressing.



\needspace{15\baselineskip}
\subsection*{Tomato and Cucumber Salad}

Arrange alternate slices of tomato and cucumber until six slices have
been piled one on top of another. Place on lettuce leaves, garnish with
strips of red and green peppers. Serve with French and Mayonnaise
Dressing. Remove seeds from peppers and parboil two minutes before
using.



\needspace{15\baselineskip}
\subsection*{Salad Chiffonade}

Cook two green peppers in boiling water one minute; cool, and shred.
Shred one head of romaine, remove pulp from one large grape fruit, and
cut three small ripe tomatoes in quarters lengthwise. Arrange in salad
dish and serve with French Dressing.



\needspace{15\baselineskip}
\subsection*{Wiersbick's Salad}

Peel small tomatoes of uniform size and scoop out a portion of centres.
Arrange in nests of lettuce leaves and garnish top of each with a slice
of cucumber, slice of truffle cut in fancy shape, and ring of green
pepper. Serve with the following dressing:

Mix three tablespoons Louit Frères mustard, one-fourth teaspoon salt,
one-eighth teaspoon paprika, one tablespoon vinegar, and one-half
teaspoon Worcestershire Sauce; then add slowly, while stirring
constantly, one-half cup olive oil.



\needspace{15\baselineskip}
\subsection*{Tomato and Cheese Salad}

Peel six medium-sized tomatoes, chill, and scoop out a small quantity of
pulp from the centre of each. Fill cavities, using equal parts of
Roquefort and Neufchâtel cheese worked together and moistened with
French Dressing. Arrange on lettuce leaves and serve with French
Dressing.



\needspace{15\baselineskip}
\subsection*{Tomato Jelly Salad}

To one can stewed and strained tomatoes add one teaspoon each of salt
and powdered sugar, and two-thirds box gelatine which has soaked fifteen
minutes in one-half cup cold water. Pour into small cups, and chill. Run
a knife around inside of moulds, so that when taken out shapes may have
a rough surface, suggesting a fresh tomato. Place on lettuce leaves and
garnish top of each with Mayonnaise Dressing.



\needspace{15\baselineskip}
\subsection*{Frozen Tomato Salad}

Open one quart can tomatoes, turn from can, and let stand one hour that
they may be reoxygenated. Add three tablespoons sugar, and season highly
with salt and cayenne; then rub through a sieve. Turn into one-half
pound breakfast-cocoa boxes, cover tightly, pack in salt and ice, using
equal parts, and let stand three hours. Remove from mould, arrange on
lettuce leaves, and serve with Mayonnaise Dressing.



\needspace{15\baselineskip}
\subsection*{Salad à la Russe}

Peel six tomatoes, remove thin slices from top of each, and take out
seeds and pulp. Sprinkle inside with salt, invert, and let stand
one-half hour. Place seeds and pulp removed from tomatoes in a strainer
to drain. Mix one-third cup cucumbers cut in dice, one-third cup cold
cooked peas, one-fourth cup pickles finely chopped, one-third cup tomato
pulp, and two tablespoons capers. Season with salt, pepper, and vinegar.
Put in a cheese-cloth and squeeze; then add one-half cup cold cooked
chicken cut in very small dice. Mix with Mayonnaise Dressing, refill
tomatoes, sprinkle with finely chopped parsley, and place each on a
lettuce leaf.



\needspace{15\baselineskip}
\subsection*{Spinach Salad}

Pick over, wash, and cook one-half peck spinach. Drain, and chop finely.
Season with salt, pepper, and lemon juice, and add one tablespoon melted
butter. Butter slightly small tin moulds and pack solidly with mixture.
Chill, remove from moulds, and arrange on thin slices of cold boiled
tongue cut in circular pieces. Garnish base of each with a wreath of
parsley, and serve on top of each Sauce Tartare.



\needspace{15\baselineskip}
\subsection*{Moulded Russian Salad}

Reduce strong consommé so that when cold it will be jelly-like in
consistency. Set individual moulds in pan of ice-water, pour in consommé
one-fourth inch deep; when firm, decorate bottom and sides of moulds
with cold cooked carrots, beets and potatoes cut in fancy shapes. Add
consommé to cover vegetables, and as soon as firm fill moulds two-thirds
full of any cooked vegetable that may be at hand. Add consommé by
spoonfuls, allowing it to become firm between the additions, and put in
enough to cover vegetables. Chill thoroughly, remove from moulds, and
arrange on lettuce leaves. Serve with Mayonnaise Dressing.



\needspace{15\baselineskip}
\subsection*{Mexican Jelly}

Peel four large cucumbers and cut in thin slices. Put in saucepan with
one cup cold water, bring to boiling-point, and cook slowly until soft;
then force through a purée strainer. Add two and one-half tablespoons
granulated gelatine dissolved in three-fourths cup boiling water, few
drops onion juice, one tablespoon vinegar, few grains cayenne, and salt
and pepper to taste. Color with leaf green, strain through cheese-cloth,
and mould same as Fruit Chartreuse (see p. 423). After removing small
mould fill space with Tomato Mayonnaise. Garnish sides of mould with
thin slices of cucumber shaped with a small round fluted cutter, and on
the centre of each slice place a circular piece of truffle. Garnish
around base of mould with small tomatoes peeled, chilled, and cut in
halves crosswise. On each slice of tomato place a circular fluted slice
of cucumber, and over all a circular piece of truffle. Serve with

\textbf{Tomato Mayonnaise.} Color mayonnaise red with tomato purée.



\needspace{15\baselineskip}
\subsection*{Egg Salad I}

Cut six “hard-boiled” eggs in halves crosswise, keeping whites in pairs.
Remove yolks, and mash or put through a potato ricer. Add slowly enough
Oil Dressing II to moisten. Make into balls the size of original yolks
and refill whites. Arrange on a bed of lettuce, and pour Oil Dressing
No. II around eggs.



\needspace{15\baselineskip}
\subsection*{Egg Salad II}

Cut four “hard-boiled” eggs in halves crosswise in such a way that tops
of halves may be cut in small points. Remove yolks, mash, and add an
equal amount of finely chopped cooked chicken. Moisten with Oil Dressing
I, shape in balls size of original yolks, and refill whites. Arrange on
lettuce leaves, garnish with radishes cut in fancy shapes, and serve
with Oil Dressing I.



\needspace{15\baselineskip}
\subsection*{Lenten Salad}

Separate yolks and whites of four “hard-boiled” eggs. Chop whites
finely, marinate with French Dressing, and arrange on lettuce leaves.
Force yolks through a potato ricer and pile on the centre of whites.
Serve with French Dressing.



\needspace{15\baselineskip}
\subsection*{Crackers and Cheese}

Mash a cream cheese, season, and shape in balls, then flatten balls, and
serve on butter-thin crackers.

NOTE. Cream cheese is very acceptable served with zephyrettes or
butter-thins and Bar-le-Duc currants.



\needspace{15\baselineskip}
\subsection*{Cottage Cheese I}

Heat one quart sour milk to 100deg F., and turn into a strainer lined with
cheese-cloth. Pour over one quart hot water, and as soon as water has
drained through, pour over another quart; then repeat. Gather
cheese-cloth around curd to form a bag and let hang until curd is free
from whey. Moisten with melted butter and heavy cream, and add salt to
taste. Shape into small balls.



\needspace{15\baselineskip}
\subsection*{Cottage Cheese II}

Heat one quart sweet milk to 100deg F., and add one junket tablet reduced
to a powder. Let stand in warm place until set. Beat with a fork to
break curd, turn into a bag made of cheese-cloth, and let hang until
whey has drained from curd; then proceed as with Cottage Cheese I.



\needspace{15\baselineskip}
\subsection*{Cheese Salad}

Arrange one head lettuce on salad dish, sprinkle with Edam cheese broken
in small pieces, and pour over French Dressing.



\needspace{15\baselineskip}
\subsection*{Neufchâtel Salad I}

Cut cheese in dice, arrange on lettuce leaves, and garnish with
radishes. Serve with French Dressing.



\needspace{15\baselineskip}
\subsection*{Neufchâtel Salad II}

Mash one Neufchâtel cheese and moisten with milk or cream. Shape into
forms the size of robins' eggs. Sprinkle with finely chopped parsley,
which has been dried. Arrange in nests of lettuce leaves, and garnish
with radishes. Serve with French Dressing.



\needspace{15\baselineskip}
\subsection*{Cheese and Olive Salad}

Mash a cream cheese, moisten with cream, and season with salt and
cayenne. Add six olives finely chopped, lettuce finely cut, and one-half
a can pimento cut in strips. Press in original shape of cheese and let
stand two hours. Cut in slices, separate in pieces, and serve on lettuce
leaves with Mayonnaise Dressing.



\needspace{15\baselineskip}
\subsection*{Cheese and Currant Salad}

Mash a cream cheese and mix with finely chopped lettuce. Shape in balls,
arrange on lettuce leaves, pour over French Dressing, and over all
Bar-le-Duc currants.



\needspace{15\baselineskip}
\subsection*{East India Salad}

Work two ten cent cream cheeses until smooth. Moisten with milk and
cream, using equal parts. Add one-half cup grated Young America cheese,
one cup whipped cream, and three-fourths tablespoon granulated gelatine
soaked in one tablespoon cold water and dissolved in one tablespoon
boiling water. Season highly with salt and paprika, and turn into a
border mould. Chill, remove from mould, arrange on lettuce leaves, fill
centre with lettuce leaves, and serve with Curry Dressing (see p. 324).



\needspace{15\baselineskip}
\subsection*{Nut Salad}

Mix one cup chopped English walnut meat and two cups shredded lettuce.
Arrange on lettuce leaves and garnish with Mayonnaise Dressing.



\needspace{15\baselineskip}
\subsection*{Nut and Celery Salad I}

Mix equal parts of English walnut or pecan nut meat cut in pieces, and
celery cut in small pieces. Marinate with French Dressing. Serve with a
border of shredded lettuce.



\needspace{15\baselineskip}
\subsection*{Nut and Celery Salad II}

Mix one and one-half cups finely cut celery, one cup pecan nut meats
broken in pieces, and one cup shredded cabbage. Moisten with Cream
Dressing, and serve in a salad bowl made of a small white cabbage.



\needspace{15\baselineskip}
\subsection*{Banana Salad}

Remove one section of skin from each of four bananas. Take out fruit,
scrape, and cut fruit from one banana in thin slices, fruit from other
three bananas in one-half inch cubes. Marinate cubes with French
Dressing. Refill skins and garnish each with slices of banana. Stack
around a mound of lettuce leaves.



\needspace{15\baselineskip}
\section*{Orange Salad}

Cut five thin-skinned sour oranges in very thin slices, and slices in
quarters. Marinate with a dressing made by mixing one-third cup olive
oil, one and one-half tablespoons each lemon juice and vinegar,
one-third teaspoon salt, one-fourth teaspoon paprika, and a few grains
mustard. Serve on a bed of watercress.



\needspace{15\baselineskip}
\subsection*{Orange Mint Salad}

Remove pulp from four large oranges, by cutting fruit in halves
crosswise and using a spoon. Sprinkle with two tablespoons powdered
sugar, and add two tablespoons finely chopped mint, and one tablespoon
each lemon juice and Sherry wine. Chill thoroughly, serve in glasses,
and garnish each with a sprig of mint. Should the oranges be very juicy,
pour off a portion of the juice before turning the mixture into glasses.



\needspace{15\baselineskip}
\subsection*{French Fruit Salad}


\begin{minipage}{1.0\textwidth}
{\setlength{\multicolsep}{0pt}\setlength{\columnsep}{2em}\raggedcolumns%
\begin{multicols}{2}
\begin{itemize}
\setlength{\itemsep}{0pt}
\setlength{\parsep}{0pt}
\item 2 oranges
\item 3 bananas
\item 1/2 lb. Malaga grapes
\item 12 English walnut meats
\item 1 head lettuce
\item French Dressing
\end{itemize}
\end{multicols}}
\end{minipage}

\vspace{0.3em}
\noindent%
Peel oranges, and remove pulp separately from each section. Peel
bananas, and cut in one-fourth inch slices. Remove skins and seeds from
grapes. Break walnut meats in pieces. Mix prepared ingredients and
arrange on lettuce leaves. Serve with French Dressing.



\needspace{15\baselineskip}
\subsection*{Hungarian Salad}

Mix equal parts shredded fresh pineapple, bananas cut in pieces, and
sections of tangerines, and marinate with French dressing. Fill banana
skins with mixture, sprinkle generously with paprika, and arrange on
lettuce leaves.



\needspace{15\baselineskip}
\subsection*{Waldorf Salad}

Mix equal quantities of finely cut apple and celery, and moisten with
Mayonnaise Dressing. Garnish with curled celery and canned pimentoes cut
in strips or fancy shapes. An attractive way of serving this salad is to
remove tops from red or green apples, scoop out inside pulp, leaving
just enough adhering to skin to keep apples in shape. Refill shells thus
made with the salad, replace tops, and serve on lettuce leaves.



\needspace{15\baselineskip}
\subsection*{Malaga Salad}

Remove skins and seeds from white grapes; add an equal quantity of
English walnut meats, blanched and broken in pieces. Marinate with
French Dressing. Serve on lettuce leaves and garnish with Maraschino
cherries.



\needspace{15\baselineskip}
\subsection*{Brazilian Salad}

Remove skin and seeds from white grapes and cut in halves lengthwise.
Add an equal quantity of shredded fresh pineapple, apples pared, cored,
and cut in small pieces, and celery cut in small pieces; then add
one-fourth the quantity of Brazil nuts broken in pieces. Mix thoroughly,
and season with lemon juice. Moisten with Cream Mayonnaise Dressing (see
p. 327).



\needspace{15\baselineskip}
\subsection*{De John's Salad}

Pare six Bartlett pears, care being taken not to remove stems. Cut in
thin slices, and serve in original shapes on lettuce leaves. Serve with
French Dressing.



\needspace{15\baselineskip}
\subsection*{Pear Salad}

Wipe, pare, and cut pears in eighths lengthwise; then remove seeds.
Arrange on lettuce leaves, pour over French dressing, and garnish with
ribbons of red pepper. See Canned Red Peppers p. 581.



\needspace{15\baselineskip}
\subsection*{Game Salad}

Drain the syrup from one can peaches. Arrange halves of fruit on lettuce
leaves, and pour over all a dressing made by mixing two teaspoons sugar,
one teaspoon celery salt, one-fourth teaspoon salt, one-eighth teaspoon
pepper, a few grains cayenne, five drops Tabasco, and adding gradually
four tablespoons olive oil and two tablespoons fresh lime juice. Use
fresh fruit when in season.



\needspace{15\baselineskip}
\subsection*{Pepper and Grape Fruit Salad}

Cut slices from stem ends of six green peppers, and remove seeds. Refill
with grape fruit pulp, finely cut celery, and English walnut meats
broken in pieces, allowing twice as much grape fruit as celery, and two
nut meats to each pepper. Arrange on chicory or lettuce leaves, and
serve with Mayonnaise Dressing.



\needspace{15\baselineskip}
\subsection*{Grape Fruit and Celery Salad}

Cut medium-sized grape fruits in thirds lengthwise. Remove the pulp, and
add to it an equal quantity of finely cut celery. Refill sections with
mixture, mask with Mayonnaise Dressing, and garnish with celery tips or
curled celery and canned pimentoes cut in strips.



\needspace{15\baselineskip}
\subsection*{Monte Carlo Salad}

Remove pulp from four large grape fruits, and drain. Add an equal
quantity of finely cut celery, and apple cut in small pieces. Moisten
with Mayonnaise, pile on a shallow salad dish, arrange around a border
of lettuce leaves, and mask with Mayonnaise. Outline, using green
Mayonnaise, four oblongs to represent playing cards, and denote spots on
cards by canned pimentoes or truffles; pimentoes cut in shapes of hearts
and diamonds, truffles cut in shapes of spades and clubs. Garnish with
cold cooked carrot and turnip, shaped with a small round cutter to
suggest gold and silver coin.



\needspace{15\baselineskip}
\subsection*{Salmon Salad}

Flake remnants of cold boiled salmon. Mix with French Mayonnaise, or
Cream Dressing. Arrange on nests of lettuce leaves. Garnish with the
yolk of a “hard-boiled” egg forced through a potato ricer, and white of
egg cut in strips.



\needspace{15\baselineskip}
\subsection*{Shrimp Salad}

Remove shrimps from can, cover with cold or ice water, and let stand
twenty minutes. Drain, dry between towels, remove intestinal veins, and
break in pieces, reserving six of the finest. Moisten with Cream
Dressing II, and arrange on nests of lettuce leaves. Put a spoonful of
dressing on each, and garnish with a whole shrimp, capers, and an olive
cut in quarters.



\needspace{15\baselineskip}
\subsection*{Sardine Salad}

Remove skin and bones from sardines, and mix with an equal quantity of
the mashed yolks of “hard-boiled” eggs. Arrange in nests of lettuce
leaves and serve with Mayonnaise Dressing.



\needspace{15\baselineskip}
\subsection*{Lobster Salad I}

Remove lobster meat from shell, cut in one-half inch cubes, and marinate
with a French Dressing. Mix with a small quantity of Mayonnaise Dressing
and arrange in nests of lettuce leaves. Put a spoonful of Mayonnaise on
each, and sprinkle with lobster coral rubbed through a fine sieve.
Garnish with small lobster claws around outside of dish. Cream Dressing
I or II may be used in place of Mayonnaise Dressing.



\needspace{15\baselineskip}
\subsection*{Lobster Salad II}

Prepare lobster as for Lobster Salad I. Add an equal quantity of celery
cut in small pieces, kept one hour in cold or ice water, then drained
and dried in a towel. Moisten with any cream or oil dressing. Arrange on
a salad dish, pile slightly in centre, cover with dressing, sprinkle
with lobster coral forced through a fine sieve, and garnish with a
border of curled celery.

\textbf{To Curl Celery.} Cut thick stalks of celery in two-inch pieces. With a
sharp knife, beginning at outside of stalks, make five cuts parallel
with each other, extending one-third the length of pieces. Make six cuts
at right angles to cuts already made. Put pieces in cold or ice water
and let stand over night or for several hours, when they will curl back
and celery will be found very crisp. Both ends of celery may be curled
if one cares to take the trouble.



\needspace{15\baselineskip}
\subsection*{Lobster Salad III}

Remove large claws and split a lobster in two lengthwise by beginning
the cut on inside of tail end and cutting through entire length of tail
and body. Open lobster, remove tail meat, liver, and coral, and set
aside. Discard intestinal vein, stomach, and fat, and wipe inside
thoroughly with cloth wrung out of cold water. Body meat and small claws
are left on shell. Remove meat from upper parts of large claws and cut
off (using scissors or can opener) one-half the shell from lower parts,
taking out meat and leaving the parts in suitable condition to refill.
Cut lobster meat in one-half inch cubes and mix with an equal quantity
of finely cut celery. Season with salt, pepper, and vinegar, and moisten
with Mayonnaise Dressing. Refill tail, body, and under half of large
claw shells. Mix liver and coral, rub through a sieve, add one
tablespoon Mayonnaise Dressing and a few drops anchovy essence with
enough more Mayonnaise Dressing to cover lobster already in shell.
Arrange on a bed of lettuce leaves.







\needspace{15\baselineskip}
\subsection*{Fish Salad with Cucumbers}

Season one and one-half cups cold cooked flaked halibut, haddock, or
cod, with salt, cayenne, and lemon juice. Cover, and let stand one hour.
To Cream Dressing II (see p. 324) add one-third tablespoon granulated
gelatine soaked in one and one-half tablespoons cold water. As soon as
dressing begins to thicken, add one-half cup heavy cream beaten until
stiff, then fold in the fish. Turn into individual moulds, chill, remove
from moulds, arrange on lettuce leaves, garnish each with a thin slice
of cucumber, and serve with

\textbf{Cucumber Sauce.} Pare two cucumbers, chop, drain off most of liquor,
and season with salt, pepper, and vinegar.



\needspace{15\baselineskip}
\subsection*{Crab and Tomato Salad}

Remove meat from hard-shelled crabs; there should be one cup. Add
two-thirds cup celery, cut in small pieces, and six small tomatoes
peeled, chilled, and cut in quarters. Moisten with Mayonnaise. Serve on
lettuce leaves, and garnish with Mayonnaise, curled celery, and small
pieces of tomato.



\needspace{15\baselineskip}
\subsection*{Scallop and Tomato Salad}

Clean one pint scallops, parboil, and drain. Add juice of one lemon,
cover, and let stand one hour. Drain, dry between towels, sprinkle with
salt and pepper, dip in flour, egg, and stale bread crumbs, fry in deep
fat, and drain on brown paper. Cool, cut in halves, marinate with
dressing, and serve garnished with sliced tomatoes and watercress.

\textbf{Dressing.} Mix one teaspoon finely chopped shallot, three-fourths
teaspoon salt, one-eighth teaspoon paprika, two tablespoons lemon juice,
and four tablespoons olive oil.



\needspace{15\baselineskip}
\subsection*{Salmon à la Martin, Ravigôte Mayonnaise}

Drain one can salmon, rinse, dry, and separate in flakes. Moisten with
Ravigôte Mayonnaise, arrange on a bed of lettuce, mask with mayonnaise,
and garnish with canned pimentoes cut in triangles, and truffles cut in
fancy shapes.

\textbf{Ravigôte Mayonnaise.} Mix two tablespoons cooked spinach, one
tablespoon capers, one-half shallot finely chopped, three anchovies,
one-third cup parsley, and one-half cup watercress. Pound in mortar
until thoroughly macerated, then force through a very fine strainer. Add
to one-half the recipe for Mayonnaise Dressing I (see p. 326).



\needspace{15\baselineskip}
\subsection*{Oyster and Grape Fruit Salad}

Parboil one and one-half pints oysters, drain, cool, and remove tough
muscles. Cut three grape fruits in halves crosswise, remove pulp, and
drain. Mix oysters with pulp, and season with six tablespoons tomato
catsup, four tablespoons grape fruit juice, one tablespoon
Worcestershire Sauce, eight drops Tabasco sauce, and one-half teaspoon
salt. Refill grape fruit skins with mixture, and garnish with curled
celery.



\needspace{15\baselineskip}
\subsection*{Chicken Salad I}

Cut cold boiled fowl or remnants of roast chicken in one-half inch
cubes, and marinate with French Dressing. Add an equal quantity of
celery, washed, scraped, cut in small pieces, chilled in cold or
ice-water, drained, and dried in a towel. Just before serving moisten
with Cream, Oil, or Mayonnaise Dressing. Mound on a salad dish, and
garnish with yolks of “hard-boiled” eggs forced through a potato ricer,
capers, and celery tips.



\needspace{15\baselineskip}
\subsection*{Chicken Salad II}

Cut cold boiled fowl or remnants of roast chicken in one-half inch dice.
To two cups add one and one-half cups celery cut in small pieces, and
moisten with Cream Dressing II. Mound on a salad dish, cover with
dressing, and garnish with capers, thin slices cut from small pickles,
and curled celery.



\needspace{15\baselineskip}
\subsection*{Individual Chicken Salads in Aspic}

Cover bottom of individual moulds set in ice-water with aspic jelly
mixture. When jelly is firm decorate with yolks and whites of
“hard-boiled” eggs cooked as for Harlequin Slices (see p. 147) and
truffles cut in fancy shapes, or pistachio nuts blanched and cut in
halves. Cover decorations with aspic mixture, being careful not to
disarrange the designs. Finely chop cold cooked fowl (preferably breast
meat), moisten with Mayonnaise to which is added a small quantity of
dissolved granulated gelatine, shape in balls, put a ball in each mould,
and add gradually aspic mixture to fill moulds. Chill thoroughly, remove
to lettuce leaves, and arrange around a dish of Mayonnaise Dressing.



\needspace{15\baselineskip}
\subsection*{Swiss Salad}

Mix one cup cold cooked chicken cut in cubes, one cucumber pared and cut
in cubes, one cup chopped English walnut meats, and one cup French peas.
Marinate with French Dressing, arrange on serving dish, and garnish with
Mayonnaise Dressing.



\needspace{15\baselineskip}
\subsection*{Nile Salad}

Cut cold boiled or roasted chicken in cubes (there should be one and
one-half cups). Put one-half cup English walnut meats in pan, sprinkle
sparingly with salt, and add three-fourths tablespoon butter. Cook in a
slow oven until browned and thoroughly heated, stirring occasionally;
remove from oven and break in pieces.

Mix chicken and nuts and marinate with French Dressing. Add
three-fourths cup celery cut in small pieces. Arrange on a bed of
lettuce, and mask with Ravigôte Mayonnaise (see p. 344).



\needspace{15\baselineskip}
\subsection*{Berkshire Salad in Boxes}

Marinate one cup cold boiled fowl cut into dice and one cup cooked
French chestnuts broken in pieces with French Dressing. Add one grated
red pepper from which seeds have been removed, one cup celery cut into
small pieces, and Mayonnaise to moisten. Trim crackers (four inches long
by one inch wide, slightly salted) at ends, using a sharp knife; arrange
on plate in form of box, keep in place with red ribbon one-half inch
wide, and fasten at one corner by tying ribbon in a bow. Garnish
opposite corner with a sprig of holly berries. Line box with lettuce
leaves, put in a spoonful of salad, and mask with Mayonnaise. Any
colored ribbon may be used, and flowers substituted for berries.



\needspace{15\baselineskip}
\subsection*{Chicken and Oyster Salad}

Clean, parboil, and drain one pint oysters. Remove tough muscles, and
mix soft parts with an equal quantity of cold boiled fowl cut in
one-half inch dice. Moisten with any salad dressing, and serve on a bed
of lettuce leaves.



\needspace{15\baselineskip}
\subsection*{Sweetbread and Cucumber Salad I}

Parboil a pair of sweetbreads twenty minutes; drain, cool, and cut in
one-half inch cubes. Mix with an equal quantity of cucumber cut in
one-half inch dice. Season with salt and pepper, and moisten with German
Dressing. Arrange in nests of lettuce leaves or in cucumber cups, and
garnish with watercress. To prepare cucumber cups, pare cucumbers,
remove thick slices from each end, and cut in halves crosswise. Take out
centres, put cups in cold water, and let stand until crisp; drain, and
dry for refilling. Small cucumbers may be pared, cut in halves
lengthwise, centres removed, and cut pointed at ends to represent a
boat.



\needspace{15\baselineskip}
\subsection*{Sweetbread and Cucumber Salad II}

Parboil a sweetbread, adding to water a bit of bay leaf, a slice of
onion, and a blade of mace. Cool, and cut in small cubes; there should
be three-fourths cup. Add an equal quantity of cucumber cubes. Beat
one-half cup thick cream until stiff; add one-fourth tablespoon
granulated gelatine soaked in one-half tablespoon cold water and
dissolved in one and one-half tablespoons boiling water, then add one
and one-half tablespoons vinegar. Add sweetbread and cucumber, mould,
and chill. Arrange on lettuce leaves, and serve with French Dressing.



\needspace{15\baselineskip}
\subsection*{Sweetbread and Celery Salad}

Mix equal parts of parboiled sweetbreads cut in one-half inch cubes and
celery finely cut. Moisten with Cream Dressing, and arrange on lettuce
leaves.



\needspace{15\baselineskip}
\subsection*{Harvard Salad}

Make lemon baskets, following directions for Orange Baskets (see p.
429). With a small wooden skewer make an incision in centre of each
handle and insert a small sprig of parsley. Fill baskets with equal
parts of cold cooked sweetbread and cucumber cut in small cubes, and
one-fourth the quantity of finely cut celery, moistened with Cream
Dressing II (see p. 324). Pare round red radishes as thinly as possible
and finely chop parings. Smooth top of baskets and cover with dressing.
Sprinkle top of one-half the baskets with chopped parings, the remaining
half with finely chopped parsley. Arrange red and green baskets
alternately on serving dish, and garnish with watercress.





\chapter{Entrées}




\needspace{15\baselineskip}
\section*{Batter I}


\begin{itemize}
\setlength{\itemsep}{0pt}
\setlength{\parsep}{0pt}
\item 1 cup bread flour
\item 1/2 teaspoon salt
\item Few grains pepper
\item 2/3 cup milk
\item 2 eggs
\end{itemize}

\vspace{-0.5em}
\noindent%
Mix flour, salt, and pepper. Add milk gradually, and eggs well beaten.



\needspace{15\baselineskip}
\section*{Batter II}


\begin{minipage}{1.0\textwidth}
{\setlength{\multicolsep}{0pt}\setlength{\columnsep}{2em}\raggedcolumns%
\begin{multicols}{2}
\begin{itemize}
\setlength{\itemsep}{0pt}
\setlength{\parsep}{0pt}
\item 1 cup bread flour
\item 1 tablespoon sugar
\item 1/4 teaspoon salt
\item 2/3 cup water
\item 1/2 tablespoon olive oil
\item 1 egg white
\end{itemize}
\end{multicols}}
\end{minipage}

\vspace{0.3em}
\noindent%
Mix flour, sugar, and salt. Add water gradually, then olive oil and
white of egg beaten until stiff.



\needspace{15\baselineskip}
\section*{Batter III}


\begin{itemize}
\setlength{\itemsep}{0pt}
\setlength{\parsep}{0pt}
\item 1 1/3 cups flour
\item 2 teaspoons baking powder
\item 1/4 teaspoon salt
\item 2/3 cup milk
\item 1 egg
\end{itemize}

\vspace{-0.5em}
\noindent%
Mix and sift dry ingredients, add milk gradually, and egg well beaten.



\needspace{15\baselineskip}
\section*{Batter IV}


\begin{minipage}{1.0\textwidth}
{\setlength{\multicolsep}{0pt}\setlength{\columnsep}{2em}\raggedcolumns%
\begin{multicols}{2}
\begin{itemize}
\setlength{\itemsep}{0pt}
\setlength{\parsep}{0pt}
\item 1 cup flour
\item 1 1/2 teaspoons baking powder
\item 3 tablespoons powdered sugar
\item 1/4 teaspoon salt
\item 1/3 cup milk
\item 1 egg
\end{itemize}
\end{multicols}}
\end{minipage}

\vspace{0.3em}
\noindent%
Mix and sift dry ingredients, add milk gradually, and egg well beaten.



\needspace{15\baselineskip}
\section*{Batter V}


\begin{minipage}{1.0\textwidth}
{\setlength{\multicolsep}{0pt}\setlength{\columnsep}{2em}\raggedcolumns%
\begin{multicols}{2}
\begin{itemize}
\setlength{\itemsep}{0pt}
\setlength{\parsep}{0pt}
\item 1 cup flour
\item 1/4 teaspoon salt
\item 2/3 cup milk or water
\item 4 egg yolks
\item 2 egg whites
\item 1 tablespoon melted butter or olive oil
\end{itemize}
\end{multicols}}
\end{minipage}

\vspace{0.3em}
\noindent%
Mix salt and flour, add milk gradually, yolks of eggs beaten until
thick, butter, and whites of eggs beaten until stiff.



\needspace{15\baselineskip}
\section*{Apple Fritters I}


\begin{itemize}
\setlength{\itemsep}{0pt}
\setlength{\parsep}{0pt}
\item 2 medium-sized sour apples
\item Batter III
\item Powdered sugar
\end{itemize}

\vspace{-0.5em}
\noindent%
Pare, core, and cut apples in eighths, then cut eighths in slices, and
stir into batter. Drop by spoonfuls and fry in deep fat (see Rules for
Testing Fat, page 21). Drain on brown paper, and sprinkle with powdered
sugar. Serve hot on a folded napkin.



\needspace{15\baselineskip}
\section*{Apple Fritters II}


\begin{itemize}
\setlength{\itemsep}{0pt}
\setlength{\parsep}{0pt}
\item 2 medium-sized sour apples
\item Batter IV
\end{itemize}

\vspace{-0.5em}
\noindent%
Prepare and cook as Apple Fritters I.



\needspace{15\baselineskip}
\section*{Apple Fritters III}

                             Sour apples
                             Powdered sugar
                             Lemon juice
                             Batter II

Core, pare, and cut apples in one-third inch slices. Sprinkle with
powdered sugar and few drops lemon juice; cover, and let stand one-half
hour. Drain, dip pieces in batter, fry in deep fat, and drain. Arrange
on a folded napkin in form of a circle, and serve with Sabyon or Hard
Sauce.



\needspace{15\baselineskip}
\section*{Banana Fritters I}


\begin{itemize}
\setlength{\itemsep}{0pt}
\setlength{\parsep}{0pt}
\item 4 bananas
\item Powdered sugar
\item 1/2 tablespoon lemon juice
\item 3 tablespoons Sherry wine
\item Batter V
\end{itemize}

\vspace{-0.5em}
\noindent%
Remove skins from bananas. Scrape bananas, cut in halves lengthwise, and
cut halves in two pieces crosswise. Sprinkle with powdered sugar, lemon
juice, and wine; cover, and let stand thirty minutes; drain, dip in
batter, fry in deep fat, and drain on brown paper. Sprinkle with
powdered sugar, and serve on a folded napkin.



\needspace{15\baselineskip}
\section*{Banana Fritters II}


\begin{minipage}{1.0\textwidth}
{\setlength{\multicolsep}{0pt}\setlength{\columnsep}{2em}\raggedcolumns%
\begin{multicols}{2}
\begin{itemize}
\setlength{\itemsep}{0pt}
\setlength{\parsep}{0pt}
\item 3 bananas
\item 1 cup bread flour
\item 2 teaspoons baking powder
\item 1 tablespoon powdered sugar
\item 1/4 teaspoon salt
\item 1/4 cup milk
\item 1 egg
\item 1 tablespoon lemon juice
\end{itemize}
\end{multicols}}
\end{minipage}

\vspace{0.3em}
\noindent%
Mix and sift dry ingredients. Beat egg until light, add milk, and
combine mixtures; then add lemon juice and banana fruit forced through a
sieve. Drop by spoonfuls, fry in deep fat, and drain. Serve with Lemon
Sauce.



\needspace{15\baselineskip}
\section*{Orange Fritters}

Peel two oranges and separate into sections. Make an opening in each
section just large enough to admit of passage for seeds, which should be
removed. Dip sections in Batter II, III, IV, or V, and fry and serve
same as other fritters.



\needspace{15\baselineskip}
\section*{Fruit Fritters}

Fresh peaches, apricots, or pears may be cut in pieces, dipped in
batter, and fried same as other fritters. Canned fruits may be used,
after draining from their syrup.



\needspace{15\baselineskip}
\section*{Cauliflower Fritters}

                        Cold cooked cauliflower
                        Batter V
                        Salt and pepper

Sprinkle pieces of cauliflower with salt and pepper and dip in Batter I
or V. Fry in deep fat, and drain on brown paper.



\needspace{15\baselineskip}
\section*{Fried Celery}

                    Celery cut in three-inch pieces

\begin{itemize}
\setlength{\itemsep}{0pt}
\setlength{\parsep}{0pt}
\item Salt and pepper
\item Batter I, III, or V
\end{itemize}

\vspace{-0.5em}
\noindent%
Parboil celery until soft, drain, sprinkle with salt and pepper, dip in
batter, fry in deep fat, and drain on brown paper. Serve with Tomato
Sauce.



\needspace{15\baselineskip}
\section*{Sardines Fried In Batter}

Drain fish and pour over boiling water to free from oil, then remove
skins. Dip in Batter III, fry in deep fat, and drain on brown paper.
Serve with Hot Tartare Sauce.



\needspace{15\baselineskip}
\section*{Tomato Fritters}


\begin{minipage}{1.0\textwidth}
{\setlength{\multicolsep}{0pt}\setlength{\columnsep}{2em}\raggedcolumns%
\begin{multicols}{2}
\begin{itemize}
\setlength{\itemsep}{0pt}
\setlength{\parsep}{0pt}
\item 1 can tomatoes
\item 6 cloves
\item 1/8 cup sugar
\item 3 slices onion
\item 1 teaspoon salt
\item Few grains cayenne
\item 1/4 cup butter
\item 1/2 cup corn-starch
\item 1 egg
\end{itemize}
\end{multicols}}
\end{minipage}

\vspace{0.3em}
\noindent%
Cook first four ingredients twenty minutes, rub all through a sieve
except seeds, and season with salt and pepper. Melt butter, and when
bubbling, add corn-starch and tomato gradually; cook two minutes, then
add egg slightly beaten. Pour into a buttered shallow tin, and cool.
Turn on a board, cut in squares, diamonds, or strips. Roll in crumbs,
egg, and crumbs again, fry in deep fat, and drain.



\needspace{15\baselineskip}
\section*{Cherry Fritters}


\begin{minipage}{1.0\textwidth}
{\setlength{\multicolsep}{0pt}\setlength{\columnsep}{2em}\raggedcolumns%
\begin{multicols}{2}
\begin{itemize}
\setlength{\itemsep}{0pt}
\setlength{\parsep}{0pt}
\item 2 cups scalded milk
\item 1/4 cup corn-starch
\item 1/4 cup flour
\item 1/2 cup sugar
\item 1/4 teaspoon salt
\item 1/4 cup cold milk
\item 3 egg yolks
\item 1/2 cup Maraschino cherries, cut in halves
\end{itemize}
\end{multicols}}
\end{minipage}

\vspace{0.3em}
\noindent%
Mix corn-starch, flour, sugar, and salt. Dilute with cold milk and add
beaten yolks; then add gradually to scalded milk and cook fifteen
minutes in double boiler. Add cherries, pour into a buttered shallow
tin, and cool. Turn on a board, cut in squares, dip in flour, egg, and
crumbs, fry in deep fat, and drain. Serve with Maraschino Sauce.



\needspace{15\baselineskip}
\section*{Maraschino Sauce}


\begin{minipage}{1.0\textwidth}
{\setlength{\multicolsep}{0pt}\setlength{\columnsep}{2em}\raggedcolumns%
\begin{multicols}{2}
\begin{itemize}
\setlength{\itemsep}{0pt}
\setlength{\parsep}{0pt}
\item 2/3 cup boiling water
\item 1/3 cup sugar
\item 2 tablespoons corn-starch
\item 1/4 cup Maraschino cherries, cut in halves
\item 1/2 cup Maraschino syrup
\item 1/2 tablespoon butter
\end{itemize}
\end{multicols}}
\end{minipage}

\vspace{0.3em}
\noindent%
Mix sugar and corn-starch, add gradually to boiling water, stirring
constantly. Boil five minutes, and add cherries, syrup, and butter.



\needspace{15\baselineskip}
\section*{Farina Cakes With Jelly}


\begin{itemize}
\setlength{\itemsep}{0pt}
\setlength{\parsep}{0pt}
\item 2 cups scalded milk
\item 1/2 cup farina (scant)
\item 1/4 cup sugar
\item 1/2 teaspoon salt
\item 1 egg
\end{itemize}

\vspace{-0.5em}
\noindent%
Mix farina, sugar, and salt, add to milk, and cook in double boiler
twenty minutes, stirring constantly until mixture has thickened. Add egg
slightly beaten, pour into a buttered shallow pan, and brush over with
one egg slightly beaten and diluted with one tablespoon milk. Brown in a
moderate oven. Cut in squares, and serve with a cube of jelly on each
square.



\needspace{15\baselineskip}
\section*{Gnocchi À La Romaine}


\begin{minipage}{1.0\textwidth}
{\setlength{\multicolsep}{0pt}\setlength{\columnsep}{2em}\raggedcolumns%
\begin{multicols}{2}
\begin{itemize}
\setlength{\itemsep}{0pt}
\setlength{\parsep}{0pt}
\item 1/4 cup butter
\item 1/4 cup flour
\item 1/4 cup corn-starch
\item 1/2 teaspoon salt
\item 2 cups scalded milk
\item 4 egg yolks
\item 3/4 cup grated cheese
\end{itemize}
\end{multicols}}
\end{minipage}

\vspace{0.3em}
\noindent%
Melt butter, and when bubbling, add flour, corn-starch, salt, and milk,
gradually. Cook three minutes, stirring constantly. Add yolks of eggs
slightly beaten, and one-half cup cheese. Pour into a buttered shallow
pan, and cool. Turn on a board, cut in squares, diamonds, or strips.
Place on a platter, sprinkle with remaining cheese, and brown in oven.



\needspace{15\baselineskip}
\section*{Queen Fritters}


\begin{itemize}
\setlength{\itemsep}{0pt}
\setlength{\parsep}{0pt}
\item 1/4 cup butter (scant)
\item 1/2 cup boiling water
\item 1/2 cup flour
\item 2 eggs
\item Fruit preserve or marmalade
\end{itemize}

\vspace{-0.5em}
\noindent%
Put butter in small saucepan and pour on water. As soon as water again
reaches boiling-point, add flour all at once and stir until mixture
leaves sides of saucepan, cleaving to spoon. Remove from fire and add
eggs unbeaten, one at a time, beating mixture thoroughly between
addition of eggs. Drop by spoonfuls and fry in deep fat until well
puffed and browned. Drain, make an opening, and fill with preserve or
marmalade. Sprinkle with powdered sugar and serve on a folded napkin.



\needspace{15\baselineskip}
\section*{Chocolate Fritters With Vanilla Sauce}

Make Queen Fritters, fill with Chocolate Cream Filling, and serve with
Vanilla Sauce; filling to be cold and sauce warm.



\needspace{15\baselineskip}
\section*{Coffee Fritters, Coffee Cream Sauce}

Cut stale bread in one-half inch slices, remove crusts, and cut slices
in one-half inch strips. Mix three-fourths cup coffee infusion, two
tablespoons sugar, one-fourth teaspoon salt, one egg slightly beaten,
and one-fourth cup cream. Dip bread in mixture, crumbs, egg, and crumbs
again. Fry in deep fat and drain. Serve with

\textbf{Coffee Cream Sauce.} Beat yolks three eggs slightly, add four
tablespoons sugar and one-eighth teaspoon salt, then add gradually one
cup coffee infusion. Cook in double boiler until mixture thickens. Cool,
and fold in one-third cup heavy cream beaten until stiff.



\needspace{15\baselineskip}
\section*{Sponge Fritters}


\begin{minipage}{1.0\textwidth}
{\setlength{\multicolsep}{0pt}\setlength{\columnsep}{2em}\raggedcolumns%
\begin{multicols}{2}
\begin{itemize}
\setlength{\itemsep}{0pt}
\setlength{\parsep}{0pt}
\item 2 2/3 cups flour
\item 1/3 cup sugar
\item 7/8 cup scalded milk
\item 1/3 yeast cake, dissolved in 2 tablespoons lukewarm water
\item 1/3 cup melted butter
\item 1/4 teaspoon salt
\item 2 eggs
\item Grated rind 1/2 lemon
\item Quince marmalade
\item Currant jelly
\end{itemize}
\end{multicols}}
\end{minipage}

\vspace{0.3em}
\noindent%
Make a sponge of one-half the flour, sugar, milk, and dissolved yeast
cake; let rise to double its bulk. Add remaining ingredients and let
rise again. Toss on a floured board, roll to one-fourth inch thickness,
shape with a small biscuit-cutter (first dipped in flour), cover, and
let rise on board. Take each piece and hollow in centre to form a nest.
In one-half the pieces put one-half teaspoon of currant jelly and quince
marmalade mixed in the proportion of one part jelly to two parts
marmalade. Brush with milk edges of filled pieces. Cover with unfilled
pieces and press edges closely together with fingers first dipped in
flour. If this is not carefully done fritters will separate during
frying. Fry in deep fat, drain on brown paper, and sprinkle with
powdered sugar.



\needspace{15\baselineskip}
\section*{Calf's Brains Fritters}

Clean brains, and cook twenty minutes in boiling water, to which is
added one-half teaspoon salt, one tablespoon lemon juice, three cloves,
two slices onion, and one-half bay leaf. Remove from range, and let
stand in water until cold; drain, dry between towels, and separate into
pieces. Make a batter of one-half cup flour, one teaspoon baking powder,
one-fourth teaspoon salt, a few grains pepper, one egg well beaten, and
one-fourth cup milk. Add brains, and drop mixture by spoonfuls into
greased muffin rings, placed in a frying-pan in which there is a
generous supply of hot lard. Cook on one side until well browned, turn,
and cook other side. Arrange on serving dish and pour around Sauce
Finiste (see p. 279).



\needspace{15\baselineskip}
\section*{Clam Fritters}


\begin{minipage}{1.0\textwidth}
{\setlength{\multicolsep}{0pt}\setlength{\columnsep}{2em}\raggedcolumns%
\begin{multicols}{2}
\begin{itemize}
\setlength{\itemsep}{0pt}
\setlength{\parsep}{0pt}
\item 1 pint clams
\item 2 eggs
\item 1/3 cup milk
\item 1 1/3 cups flour
\item 2 teaspoons baking powder
\item Salt
\item Pepper
\end{itemize}
\end{multicols}}
\end{minipage}

\vspace{0.3em}
\noindent%
Clean clams, drain from their liquor, and chop. Beat eggs until light,
add milk and flour mixed and sifted with baking powder, then add chopped
clams, and season highly with salt and pepper. Drop by spoonfuls, and
fry in deep fat. Drain on brown paper, and serve at once on a folded
napkin.



\needspace{15\baselineskip}
\section*{Croquettes}

Before making Croquettes, consult Rules for Testing Fat for Frying, page
21; Egging and Crumbing, page 22; Uses for Stale Bread, page 69; and
Potato Croquettes, page 316.



\needspace{15\baselineskip}
\section*{Banana Croquettes}

Remove skins from bananas, scrape, using a silver knife to remove the
astringent principle which lies close to skin, and cut in halves
crosswise; then remove a slice from each end. Dip in crumbs, egg, and
crumbs again, fry in deep fat, and drain on brown paper.



\needspace{15\baselineskip}
\section*{Cheese Croquettes}


\begin{minipage}{1.0\textwidth}
{\setlength{\multicolsep}{0pt}\setlength{\columnsep}{2em}\raggedcolumns%
\begin{multicols}{2}
\begin{itemize}
\setlength{\itemsep}{0pt}
\setlength{\parsep}{0pt}
\item 3 tablespoons butter
\item 1/4 cup flour
\item 2/3 cup milk
\item 4 egg yolks
\item 1 cup mild cheese, cut in very small cubes
\item 1/2 cup grated Gruyère cheese
\item Salt and pepper
\item Few grains cayenne
\end{itemize}
\end{multicols}}
\end{minipage}

\vspace{0.3em}
\noindent%
Make a thick white sauce, using butter, flour, and milk, add yolks of
eggs without first beating, and stir until well mixed; then add grated
cheese. As soon as cheese melts, remove from fire, fold in cheese cubes,
and season with salt, pepper, and cayenne. Spread in a shallow pan, and
cool. Turn on a board, cut in small squares or strips, dip in crumbs,
egg, and crumbs again, fry in deep fat, and drain on brown paper. Serve
for a cheese course.



\needspace{15\baselineskip}
\section*{Chestnut Croquettes}


\begin{itemize}
\setlength{\itemsep}{0pt}
\setlength{\parsep}{0pt}
\item 1 cup mashed French chestnuts
\item 2 tablespoons thick cream
\item 4 egg yolks
\item 1 teaspoon sugar
\item 1/4 teaspoon vanilla
\end{itemize}

\vspace{-0.5em}
\noindent%
Mix ingredients in order given. Shape in balls, dip in crumbs, egg, and
crumbs again, fry in deep fat, and drain.



\needspace{15\baselineskip}
\section*{Chestnut Roulettes}


\begin{minipage}{1.0\textwidth}
{\setlength{\multicolsep}{0pt}\setlength{\columnsep}{2em}\raggedcolumns%
\begin{multicols}{2}
\begin{itemize}
\setlength{\itemsep}{0pt}
\setlength{\parsep}{0pt}
\item 1 cup chestnut purée
\item 2 eggs
\item Few drops onion juice
\item 2 tablespoons butter
\item 2 tablespoons heavy cream
\item 1/4 teaspoon salt
\item Few grains paprika
\end{itemize}
\end{multicols}}
\end{minipage}

\vspace{0.3em}
\noindent%
Mix ingredients in order given, cook two minutes, and cool. Shape a
little larger than French chestnuts, dip in crumbs, egg, and crumbs
again. Fry in deep fat, and drain on brown paper.



\needspace{15\baselineskip}
\section*{Lenten Croquettes}

Soak one-half cup lentils and one-fourth cup dried lima beans over
night, in cold water to cover; drain, add three pints water, one-half
small onion, one stalk celery, three slices carrot, and a sprig of
parsley. Cook until lentils are soft, remove seasonings, drain, and rub
through a sieve. To pulp add one-half cup stale bread crumbs, one egg
slightly beaten, and salt and pepper to taste. Melt one tablespoon
butter, add one tablespoon flour, and pour on gradually one-third cup
hot cream; combine mixtures, and cool. Shape, dip in crumbs, egg, and
crumbs again, fry in deep fat, and drain on brown paper. Serve with
Tomato Sauce I.



\needspace{15\baselineskip}
\section*{Rice Croquettes With Jelly}


\begin{minipage}{1.0\textwidth}
{\setlength{\multicolsep}{0pt}\setlength{\columnsep}{2em}\raggedcolumns%
\begin{multicols}{2}
\begin{itemize}
\setlength{\itemsep}{0pt}
\setlength{\parsep}{0pt}
\item 1/2 cup rice
\item 1/2 cup boiling water
\item 1 cup scalded milk
\item 1/2 teaspoon salt
\item 4 egg yolks
\item 1 tablespoon butter
\end{itemize}
\end{multicols}}
\end{minipage}

\vspace{0.3em}
\noindent%
Wash rice, add to water with salt, cover, and steam until rice has
absorbed water. Then add milk, stir lightly with a fork, cover, and
steam until rice is soft. Remove from fire, add egg yolks and butter;
spread on a shallow plate to cool. Shape in balls, roll in crumbs, then
shape in form of nests. Dip in egg, again in crumbs, fry in deep fat,
and drain. Put a cube of jelly in each croquette. Arrange on a folded
napkin, and garnish with parsley, or serve around game.



\needspace{15\baselineskip}
\section*{Sweet Rice Croquettes}

To rice croquette mixture add two tablespoons powdered sugar and grated
rind one-half lemon. Shape in cylinder forms, dip in crumbs, egg, and
crumbs again, fry in deep fat, and drain.



\needspace{15\baselineskip}
\section*{Rice And Tomato Croquettes}


\begin{minipage}{1.0\textwidth}
{\setlength{\multicolsep}{0pt}\setlength{\columnsep}{2em}\raggedcolumns%
\begin{multicols}{2}
\begin{itemize}
\setlength{\itemsep}{0pt}
\setlength{\parsep}{0pt}
\item 1/2 cup rice
\item 3/4 cup stock
\item 1/2 can tomatoes
\item 1 slice onion
\item 1 slice carrot
\item 1 sprig parsley
\item 1 sprig thyme
\item 2 cloves
\item 1/4 teaspoon peppercorns
\item 1 teaspoon sugar
\item 1 egg
\item 1/4 cup grated cheese
\item 1 tablespoon butter
\item 1/2 teaspoon salt
\item Few grains cayenne
\end{itemize}
\end{multicols}}
\end{minipage}

\vspace{0.3em}
\noindent%
Wash rice, and steam in stock until rice has absorbed stock; then add
tomatoes which have been cooked twenty minutes with onion, carrot,
parsley, thyme, cloves, peppercorns, and sugar, and then rubbed through
a strainer. Remove from fire, add egg slightly beaten, cheese, butter,
salt, and cayenne. Spread on a plate to cool. Shape in form of
cylinders, dip in crumbs, egg, and crumbs again, fry in deep fat, and
drain.



\needspace{15\baselineskip}
\section*{Oyster Crabs À La Newburg}


\begin{minipage}{1.0\textwidth}
{\setlength{\multicolsep}{0pt}\setlength{\columnsep}{2em}\raggedcolumns%
\begin{multicols}{2}
\begin{itemize}
\setlength{\itemsep}{0pt}
\setlength{\parsep}{0pt}
\item 1 cup oyster crabs
\item 1 cup mushroom caps
\item 1/3 cup Sherry wine
\item 1/4 cup butter
\item 1 tablespoon flour
\item Salt
\item Cayenne
\item Nutmeg
\item 3/4 cup cream
\item Yolks two eggs
\item 1 tablespoon brandy
\end{itemize}
\end{multicols}}
\end{minipage}

\vspace{0.3em}
\noindent%
Peel mushroom caps and break in pieces. Add oyster crabs and wine,
cover, and let stand one hour. Melt butter, add first mixture, and cook
eight minutes. Add flour, and cook two minutes. Season with salt,
cayenne, and nutmeg; then add heavy cream. Just before serving add egg
yolks, slightly beaten, and brandy.



\needspace{15\baselineskip}
\section*{Oyster And Macaroni Croquettes}


\begin{minipage}{1.0\textwidth}
{\setlength{\multicolsep}{0pt}\setlength{\columnsep}{2em}\raggedcolumns%
\begin{multicols}{2}
\begin{itemize}
\setlength{\itemsep}{0pt}
\setlength{\parsep}{0pt}
\item 1/3 cup macaroni, broken in 1/2 inch pieces
\item 1 pint oysters
\item 1 cup Thick White Sauce
\item Few grains cayenne
\item Few grains mace
\item 1/2 teaspoon lemon juice
\item 1/4 cup grated cheese
\end{itemize}
\end{multicols}}
\end{minipage}

\vspace{0.3em}
\noindent%
Cook macaroni in boiling salted water until soft, drain in a colander,
and pour over macaroni two cups cold water. Clean and parboil oysters,
remove tough muscles, and cut soft parts in pieces. Reserve one-half cup
oyster liquor and use in making Thick White Sauce in place of all milk.
Mix macaroni and oysters, add Thick White Sauce and seasonings. Spread
on a plate to cool. Shape, dip in crumbs, egg, and crumbs again, fry in
deep fat, and drain.



\needspace{15\baselineskip}
\section*{Oysters À La Somerset}


\begin{minipage}{1.0\textwidth}
{\setlength{\multicolsep}{0pt}\setlength{\columnsep}{2em}\raggedcolumns%
\begin{multicols}{2}
\begin{itemize}
\setlength{\itemsep}{0pt}
\setlength{\parsep}{0pt}
\item 1 pint selected oysters
\item 1 tablespoon chopped onion
\item 2 tablespoons chopped mushrooms
\item 3 tablespoons butter
\item 1/3 cup oyster liquor
\item 1/3 cup Chicken Stock
\item Salt
\item Pepper
\item Cayenne
\item 4 tablespoons flour
\end{itemize}
\end{multicols}}
\end{minipage}

\vspace{0.3em}
\noindent%
Parboil and drain oysters. Reserve liquor, strain, and set aside for
sauce. Cook onion and mushroom in butter five minutes, add flour, and
pour on gradually oyster liquor and chicken stock. Season with salt,
pepper, and cayenne. Remove tough muscles from oysters, and discard.
Shape oysters, cover with sauce, and cool on a plate covered with stale
bread crumbs. Dip in egg and stale bread crumbs, fry in deep fat, and
drain on brown paper.



\needspace{15\baselineskip}
\section*{Salmon Croquettes}


\begin{itemize}
\setlength{\itemsep}{0pt}
\setlength{\parsep}{0pt}
\item 1 3/4 cups cold flaked salmon
\item 1 cup Thick White Sauce
\item Few grains cayenne
\item 1 teaspoon lemon juice
\item Salt
\end{itemize}

\vspace{-0.5em}
\noindent%
Add sauce to salmon, then add seasonings. Spread on a plate to cool.
Shape, dip in crumbs, egg, and crumbs again, fry in deep fat, and drain.



\needspace{15\baselineskip}
\section*{Salmon Cutlets}

Mix equal parts of cold flaked salmon and hot mashed potatoes. Season
with salt and pepper. Shape in form of cutlets, dip in crumbs, egg, and
crumbs again, fry in deep fat, and drain. Arrange in a circle, having
cutlets overlap one another, on a folded napkin. Garnish with parsley.



\needspace{15\baselineskip}
\section*{Lobster Croquettes}


\begin{minipage}{1.0\textwidth}
{\setlength{\multicolsep}{0pt}\setlength{\columnsep}{2em}\raggedcolumns%
\begin{multicols}{2}
\begin{itemize}
\setlength{\itemsep}{0pt}
\setlength{\parsep}{0pt}
\item 2 cups chopped lobster meat
\item 1/2 teaspoon salt
\item 1/4 teaspoon mustard
\item Few grains cayenne
\item 1 teaspoon lemon
\item 1 cup Thick White Sauce
\end{itemize}
\end{multicols}}
\end{minipage}

\vspace{0.3em}
\noindent%
Add seasonings to lobster, then add Thick White Sauce. Cool, shape, dip
in crumbs, egg, and crumbs again, fry in deep fat, and drain. Serve with
Tomato Cream Sauce.



\needspace{15\baselineskip}
\section*{Lobster Cutlets}


\begin{minipage}{1.0\textwidth}
{\setlength{\multicolsep}{0pt}\setlength{\columnsep}{2em}\raggedcolumns%
\begin{multicols}{2}
\begin{itemize}
\setlength{\itemsep}{0pt}
\setlength{\parsep}{0pt}
\item 2 cups chopped lobster meat
\item 1/2 teaspoon salt
\item Few grains cayenne
\item Few gratings nutmeg
\item 1 teaspoon lemon juice
\item Yolk 1 egg
\item 1 teaspoon finely chopped parsley
\item 1 cup Thick White Sauce
\end{itemize}
\end{multicols}}
\end{minipage}

\vspace{0.3em}
\noindent%
Mix ingredients in order given, and cool. Shape in form of cutlets,
crumb, and fry same as croquettes. Make a cut at small end of each
cutlet, and insert in each the tip end of a small claw. Stack around a
mound of parsley. Serve with Sauce Tartare.



\needspace{15\baselineskip}
\section*{Beef And Rice Croquettes}


\begin{minipage}{1.0\textwidth}
{\setlength{\multicolsep}{0pt}\setlength{\columnsep}{2em}\raggedcolumns%
\begin{multicols}{2}
\begin{itemize}
\setlength{\itemsep}{0pt}
\setlength{\parsep}{0pt}
\item 1 cup chopped beef (cut from top of round)
\item 1/3 cup rice
\item 1/2 teaspoon salt
\item 1/4 teaspoon pepper
\item Few grains cayenne
\item Cabbage
\item Tomato Sauce
\end{itemize}
\end{multicols}}
\end{minipage}

\vspace{0.3em}
\noindent%
Mix beef and rice, and add salt, pepper, and cayenne. Cook cabbage
leaves two minutes in boiling water to cover. In each leaf put two
tablespoons mixture, and fold leaf to enclose mixture. Cook one hour in
Tomato Sauce.

\textbf{Tomato Sauce.} Brown four tablespoons butter, add five tablespoons
flour, and pour on gradually one and one-half cups each Brown Stock and
stewed and strained tomatoes. Add one slice onion, one slice carrot, a
bit of bay leaf, a sprig of parsley, four cloves, three-fourths teaspoon
salt, one-fourth teaspoon pepper, and a few grains cayenne. Cook ten
minutes, and strain.



\needspace{15\baselineskip}
\section*{Lamb Croquettes}


\begin{minipage}{1.0\textwidth}
{\setlength{\multicolsep}{0pt}\setlength{\columnsep}{2em}\raggedcolumns%
\begin{multicols}{2}
\begin{itemize}
\setlength{\itemsep}{0pt}
\setlength{\parsep}{0pt}
\item 1 tablespoon finely chopped onion
\item 2 tablespoons butter
\item 1/4 cup flour
\item 1 cup stock
\item 1 cup cold cooked lamb, cut in small cubes
\item 2/3 cup boiled potato cubes
\item Salt and pepper
\item 1 teaspoon finely chopped parsley
\end{itemize}
\end{multicols}}
\end{minipage}

\vspace{0.3em}
\noindent%
Fry onion in butter five minutes, then remove onion. To butter add flour
and stock, and cook two minutes. Add meat, potato, salt, and pepper.
Simmer until meat and potato have absorbed sauce. Add parsley, and
spread on a shallow dish to cool. Shape, dip in crumbs, egg, and crumbs
again, fry in deep fat, and drain. Serve with Tomato Sauce.



\needspace{15\baselineskip}
\section*{Veal Croquettes}


\begin{minipage}{1.0\textwidth}
{\setlength{\multicolsep}{0pt}\setlength{\columnsep}{2em}\raggedcolumns%
\begin{multicols}{2}
\begin{itemize}
\setlength{\itemsep}{0pt}
\setlength{\parsep}{0pt}
\item 2 cups chopped cold cooked veal
\item 1/2 teaspoon salt
\item 1/8 teaspoon pepper
\item Few grains cayenne
\item Few drops onion juice
\item Yolk 1 egg
\item 1 cup thick sauce made of White Soup Stock
\end{itemize}
\end{multicols}}
\end{minipage}

\vspace{0.3em}
\noindent%
Mix ingredients in order given. Cool, shape, crumb, and fry same as
other croquettes.



\needspace{15\baselineskip}
\section*{Chicken Croquettes I}


\begin{minipage}{1.0\textwidth}
{\setlength{\multicolsep}{0pt}\setlength{\columnsep}{2em}\raggedcolumns%
\begin{multicols}{2}
\begin{itemize}
\setlength{\itemsep}{0pt}
\setlength{\parsep}{0pt}
\item 1 3/4 cups chopped cold cooked fowl
\item 1/2 teaspoon salt
\item 1/4 teaspoon celery salt
\item Few grains cayenne
\item 1 teaspoon lemon juice
\item Few drops onion juice
\item 1 teaspoon finely chopped parsley
\item 1 cup Thick White Sauce
\end{itemize}
\end{multicols}}
\end{minipage}

\vspace{0.3em}
\noindent%
Mix ingredients in order given. Cool, shape, crumb, and fry same as
other croquettes.

White meat of fowl absorbs more sauce than dark meat. This must be
remembered if dark meat alone is used. Croquette mixtures should always
be as soft as can be conveniently handled, when croquettes will be soft
and creamy inside.



\needspace{15\baselineskip}
\section*{Chicken Croquettes II}

Clean and dress a four-pound fowl. Put into a kettle with six cups
boiling water, seven slices carrot, two slices turnip, one small onion,
one stalk celery, one bay leaf, and three sprigs thyme. Cook slowly
until fowl is tender. Remove fowl; strain liquor, cool, and skim off
fat. Make a thick sauce, using one-fourth cup butter, one-half cup
flour, one cup chicken stock, and one-third cup cream. Remove meat from
chicken, chop, and moisten with sauce. Season with salt, cayenne, and
slight grating of nutmeg; then add one beaten egg, cool, shape, crumb,
and fry same as other croquettes. Arrange around a mound of green peas,
and serve with Cream Sauce or Wine Jelly.



\needspace{15\baselineskip}
\section*{Chicken And Mushroom Croquettes}

Make as Chicken Croquettes I, using one and one-third cups chicken meat
and two-thirds cup chopped mushrooms.



\needspace{15\baselineskip}
\section*{Maryland Croquettes}

Season one cup chopped cold cooked fowl with salt, celery salt, cayenne,
lemon juice, and onion juice; moisten with sauce, and cool. Parboil one
pint selected oysters, drain, and cover each oyster with chicken
mixture. Dip in crumbs, egg, and crumbs; fry in deep fat, and drain.

\textbf{Sauce.} Melt one and one-half tablespoons butter, add three tablespoons
flour, and gradually one-third cup oyster liquor and two tablespoons
cream. Season with salt and cayenne.



\needspace{15\baselineskip}
\section*{Lincoln Croquettes}

Mix one cup each bread crumbs, walnut meats cut in pieces, and cold
cooked chicken cut in cubes. Moisten with a sauce made by melting one
and one-half tablespoons butter, adding one and one-half tablespoons
flour, and pouring on gradually, while stirring constantly, one-half cup
chicken stock. Season with salt, celery salt, paprika, nutmeg, and
Sherry wine. Shape in balls, dip in crumbs, egg, and crumbs, fry in deep
fat, and drain on brown paper. Serve with a sauce made of one-half
chicken stock and one-half cream and flavored with Sherry wine.



\needspace{15\baselineskip}
\section*{Cutlets Of Sweetbreads À La Victoria}


\begin{minipage}{1.0\textwidth}
{\setlength{\multicolsep}{0pt}\setlength{\columnsep}{2em}\raggedcolumns%
\begin{multicols}{2}
\begin{itemize}
\setlength{\itemsep}{0pt}
\setlength{\parsep}{0pt}
\item 2 pairs parboiled sweetbreads
\item 2 teaspoons lemon juice
\item 1/2 teaspoon salt
\item 1/8 teaspoon pepper
\item Slight grating nutmeg
\item 1 teaspoon finely chopped parsley
\item 1 egg
\item 1 cup Thick White Sauce
\end{itemize}
\end{multicols}}
\end{minipage}

\vspace{0.3em}
\noindent%
Chop the sweetbreads, of which there should be two cups; if not enough,
add chopped mushrooms to make two cups, then season. Add egg, slightly
beaten, to sauce, and combine mixtures. Cool, shape, crumb, and fry.
Make a cut in small end of each cutlet, and insert in each a piece of
cold boiled macaroni one and one-half inches long. Serve with Allemande
Sauce.



\needspace{15\baselineskip}
\section*{Epigrams Of Sweetbreads}

Parboil a sweetbread, drain, place in a small mould, cover, and press
with a weight. Cut in one-half inch slices, and spread with the
following mixture: Fry one-third teaspoon finely chopped shallot in one
and one-half tablespoons butter three minutes, add three tablespoons
chopped mushrooms, and cook three minutes; then add two and one-half
tablespoons flour, one-half cup stock, two tablespoons cream, one
tablespoon Sherry wine, one egg yolk, and salt and pepper to taste.
Cool, dip in crumbs, egg, and crumbs, fry in deep fat, and drain.



\needspace{15\baselineskip}
\section*{Swedish Timbales}


\begin{minipage}{1.0\textwidth}
{\setlength{\multicolsep}{0pt}\setlength{\columnsep}{2em}\raggedcolumns%
\begin{multicols}{2}
\begin{itemize}
\setlength{\itemsep}{0pt}
\setlength{\parsep}{0pt}
\item 3/4 cup flour
\item 1/2 teaspoon salt
\item 1 teaspoon sugar
\item 1/2 cup milk
\item 1 egg
\item 1 tablespoon olive oil
\end{itemize}
\end{multicols}}
\end{minipage}

\vspace{0.3em}
\noindent%
Mix dry ingredients, add milk gradually, and beaten egg; then add olive
oil. Shape, using a hot timbale iron, fry in deep fat until crisp and
brown; take from iron and invert on brown paper to drain.

\textbf{To Heat Timbale Iron.} Heat fat until nearly hot enough to fry uncooked
mixtures. Put iron into hot fat, having fat deep enough to more than
cover it, and let stand until heated. The only way of knowing when iron
is of right temperature is to take it from fat, shake what fat may drip
from it, lower in batter to three-fourths its depth, raise from batter,
then immerse in hot fat. If batter does not cling to iron, or drops from
iron as soon as immersed in fat, it is either too hot or not
sufficiently heated.

\textbf{To Form Timbales.} Turn timbale batter into a cup. Lower hot iron into
cup, taking care that batter covers iron to only three-fourths its
depth. When immersed in fat, mixture will rise to top of iron, and when
crisp and brown may be easily slipped off. If too much batter is used,
in cooking it will rise over top of iron, and in order to remove timbale
it must be cut around with a sharp knife close to top of iron. If the
cases are soft rather than crisp, batter is too thick and must be
diluted with milk.

Fill cases with Creamed Oysters, Chicken, Sweetbreads, or Chicken and
Sweetbreads in combination with Mushrooms.



\needspace{15\baselineskip}
\section*{Bunuelos}

Use recipe for and fry same as Swedish Timbales, using a Bunuelos iron.
Serve with cooked fruit and with or without whipped cream sweetened and
flavored.



\needspace{15\baselineskip}
\section*{Strawberry Baskets}

Fry Swedish Timbales, making cases one inch deep. Fill with selected
strawberries, sprinkled with powdered sugar. Serve as a first course at
a ladies' luncheon.



\needspace{15\baselineskip}
\section*{Rice Timbales}

Pack hot boiled rice in slightly buttered small tin moulds. Let stand in
hot water ten minutes. Use as a garnish for curried meat, fricassee, or
boiled fowl.



\needspace{15\baselineskip}
\section*{Macaroni Timbales}

Line slightly buttered Dario moulds with boiled macaroni. Cut strips the
length of height of mould, and place closely together around inside of
mould. Fill with Chicken, or Salmon Force meat. Put in a pan, half
surround with hot water, cover with buttered paper, and bake thirty
minutes in a moderate oven. Serve with Lobster, Béchamel, or Hollandaise
Sauce I.



\needspace{15\baselineskip}
\section*{Spaghetti Timbales}

Line bottom and sides of slightly buttered Dario moulds with long strips
of boiled spaghetti coiled around the inside. Fill and bake same as
Macaroni Timbales.



\needspace{15\baselineskip}
\section*{Pimento Timbales}

Line small timbale moulds with canned pimentoes. Fill with Chicken
Timbale II mixture (see p. 366), and bake until firm. Remove from
moulds, insert a sprig of parsley in top of each, and serve with



\needspace{15\baselineskip}
\section*{Brown Mushroom Sauce}


\begin{minipage}{1.0\textwidth}
{\setlength{\multicolsep}{0pt}\setlength{\columnsep}{2em}\raggedcolumns%
\begin{multicols}{2}
\begin{itemize}
\setlength{\itemsep}{0pt}
\setlength{\parsep}{0pt}
\item 3 tablespoons butter
\item Few drops onion juice
\item 3 1/2 tablespoons flour
\item 1 cup cream
\item 1/2 lb. mushrooms
\item 1 teaspoon beef extract
\item Salt
\item Paprika
\end{itemize}
\end{multicols}}
\end{minipage}

\vspace{0.3em}
\noindent%
Melt butter, add onion juice, and cook until slightly browned; then add
flour and continue the browning. Pour on, gradually, while stirring
constantly, the cream. Clean mushrooms, peal caps, cut in slices
lengthwise, and sauté in butter five minutes. Break stems in pieces,
cover with cold water, and cook slowly until liquor is reduced to
one-third cup; then strain. Dissolve beef extract in mushroom liquor.
Add to sauce, and season with salt and paprika. Just before serving, add
sautéd caps.



\needspace{15\baselineskip}
\section*{Halibut Timbales I}


\begin{minipage}{1.0\textwidth}
{\setlength{\multicolsep}{0pt}\setlength{\columnsep}{2em}\raggedcolumns%
\begin{multicols}{2}
\begin{itemize}
\setlength{\itemsep}{0pt}
\setlength{\parsep}{0pt}
\item 1 lb. halibut
\item 1/3 cup thick cream
\item 3/4 teaspoon salt
\item Few grains cayenne
\item 1 1/2 teaspoons lemon juice
\item 3 egg whites
\end{itemize}
\end{multicols}}
\end{minipage}

\vspace{0.3em}
\noindent%
Cook halibut in boiling salted water, drain, and rub through a sieve.
Season with salt, cayenne, and lemon juice; add cream beaten until
stiff, then beaten whites of eggs. Turn into small, slightly buttered
moulds, put in a pan, half surround with hot water, cover with buttered
paper, and bake twenty minutes in a moderate oven. Remove from moulds,
arrange on a serving dish, pour around Béchamel Sauce or Lobster Sauce
II, and garnish with parsley.



\needspace{15\baselineskip}
\section*{Halibut Timbales II}


\begin{minipage}{1.0\textwidth}
{\setlength{\multicolsep}{0pt}\setlength{\columnsep}{2em}\raggedcolumns%
\begin{multicols}{2}
\begin{itemize}
\setlength{\itemsep}{0pt}
\setlength{\parsep}{0pt}
\item 1 lb. halibut
\item 2/3 cup milk
\item Yolk 1 egg
\item 1 1/4 teaspoons salt
\item 1/4 teaspoon pepper
\item Few grains cayenne
\item 2/3 teaspoon corn-starch
\item 1/3 cup thick cream
\end{itemize}
\end{multicols}}
\end{minipage}

\vspace{0.3em}
\noindent%
Force fish through a meat chopper, then rub through a sieve or finely
chop. Add yolk of egg, seasonings, corn-starch, and cream beaten until
stiff. Cook same as Halibut Timbales I and serve with Cream or Lobster
Sauce.



\needspace{15\baselineskip}
\section*{Lobster Timbales I}

Sprinkle slightly buttered Dario or timbale moulds with lobster coral
rubbed through a strainer. Line moulds with Fish Force-meat I, fill
centres with Creamed Lobster, and cover with force-meat. Put in a pan,
half surround with hot water, place over moulds buttered paper, and bake
twenty minutes in a moderate oven. Serve with Lobster or Béchamel Sauce.



\needspace{15\baselineskip}
\section*{Lobster Timbales II}


\begin{minipage}{1.0\textwidth}
{\setlength{\multicolsep}{0pt}\setlength{\columnsep}{2em}\raggedcolumns%
\begin{multicols}{2}
\begin{itemize}
\setlength{\itemsep}{0pt}
\setlength{\parsep}{0pt}
\item 2 lb. live lobster
\item 1/4 cup stale bread crumbs
\item 1/2 cup heavy cream
\item 2 eggs
\item Sherry wine
\item Salt and pepper
\end{itemize}
\end{multicols}}
\end{minipage}

\vspace{0.3em}
\noindent%
Split lobster, remove intestinal vein, liver, and stomach. Crack claw
shells with mallet, then remove all meat, scraping as close to shell as
possible to obtain the color desired. Force meat through a sieve, add
bread crumbs, cream, eggs slightly beaten, and salt, pepper, and Sherry
wine to taste. Fill small timbale moulds two-thirds full, place in iron
frying-pan, and pour in boiling water to two-thirds the depths of the
moulds. Place over moulds buttered paper and cook on the range until
firm, keeping water below the boiling-point. Remove from moulds and
serve with Hot Mayonnaise (see p. 278).



\needspace{15\baselineskip}
\section*{Lobster Cream I}


\begin{minipage}{1.0\textwidth}
{\setlength{\multicolsep}{0pt}\setlength{\columnsep}{2em}\raggedcolumns%
\begin{multicols}{2}
\begin{itemize}
\setlength{\itemsep}{0pt}
\setlength{\parsep}{0pt}
\item 2 lb. lobster
\item 1/2 cup soft stale bread crumbs
\item 1/2 cup milk
\item 1/4 cup cream
\item 2 teaspoons Anchovy essence
\item 1/2 teaspoon salt
\item Few grains cayenne
\item 3 egg whites
\end{itemize}
\end{multicols}}
\end{minipage}

\vspace{0.3em}
\noindent%
Remove lobster meat from shell and chop finely. Cook bread and milk ten
minutes. Add cream, seasonings, and whites of eggs beaten until stiff.
Turn into one slightly buttered timbale mould and two slightly buttered
Dario moulds. Bake as Lobster Timbales. Remove to serving dish, having
larger mould in centre, smaller moulds one at either end. Pour around
Lobster Sauce I, sprinkle with coral rubbed through a sieve, and garnish
with pieces of lobster shell from tail, and parsley.



\needspace{15\baselineskip}
\section*{Lobster Cream II}


\begin{minipage}{1.0\textwidth}
{\setlength{\multicolsep}{0pt}\setlength{\columnsep}{2em}\raggedcolumns%
\begin{multicols}{2}
\begin{itemize}
\setlength{\itemsep}{0pt}
\setlength{\parsep}{0pt}
\item 1 cup chopped lobster meat
\item 1 tablespoon butter
\item 1 tablespoon flour
\item 1 teaspoon salt
\item 1/8 teaspoon paprika
\item Few drops onion juice
\item 2 egg yolks
\item 1/3 cup milk
\item 1/3 cup heavy cream
\item White one egg, beaten stiff
\end{itemize}
\end{multicols}}
\end{minipage}

\vspace{0.3em}
\noindent%
Cook lobster meat with butter five minutes. Add flour, seasonings, egg
yolks, milk, cream beaten until stiff, and white of egg. Fill buttered
timbale moulds three-fourths full, set in pan of hot water, cover with
buttered paper, and bake until firm. Serve with Lobster Sauce.



\needspace{15\baselineskip}
\section*{Chicken Timbales I}

Garnish slightly buttered Dario moulds with chopped truffles or slices
of truffles cut in fancy shapes. Line with Chicken Force-meat I, fill
centres with Creamed Chicken and Mushrooms, to which has been added a
few chopped truffles. Cover with Force-meat, and bake same as Lobster
Timbales Serve with Béchamel or Yellow Béchamel Sauce.



\needspace{15\baselineskip}
\section*{Chicken Timbales II}


\begin{minipage}{1.0\textwidth}
{\setlength{\multicolsep}{0pt}\setlength{\columnsep}{2em}\raggedcolumns%
\begin{multicols}{2}
\begin{itemize}
\setlength{\itemsep}{0pt}
\setlength{\parsep}{0pt}
\item 2 tablespoons butter
\item 1/4 cup stale bread crumbs
\item 2/3 cup milk
\item 1 cup chopped cooked chicken
\item 1/2 tablespoon chopped parsley
\item 2 eggs
\item Salt
\item Pepper
\end{itemize}
\end{multicols}}
\end{minipage}

\vspace{0.3em}
\noindent%
Melt butter, add bread crumbs and milk, and cook five minutes, stirring
constantly. Add chicken, parsley, and eggs slightly beaten. Season with
salt and pepper. Turn into buttered individual moulds, having moulds
two-thirds full set in pan of hot water, cover with buttered paper, and
bake twenty minutes. Serve with Béchamel Sauce.



\needspace{15\baselineskip}
\section*{Chicken Timbales III}

Soak one-half tablespoon granulated gelatine in one and one-half
tablespoons cold water, and dissolve in three-fourths cup chicken stock.
Add one cup chopped cooked chicken, and stir until the mixture begins to
thicken, then add one cup cream beaten until thick. Add one tablespoon
Sherry wine and a few grains cayenne. Mould, chill, and serve on lettuce
leaves.



\needspace{15\baselineskip}
\section*{Ham Timbales}

Make and bake same as Chicken Timbales II, using chopped cooked ham in
place of chicken. Serve with Béchamel Sauce.



\needspace{15\baselineskip}
\section*{Sweetbread And Mushroom Timbales}

Cook two tablespoons butter with one sliced onion five minutes. Add one
and one-half cups mushroom caps finely chopped, and one small parboiled
sweetbread, finely chopped; then add one cup White Sauce II, one-fourth
cup stale bread crumbs, one red pepper chopped, one-half teaspoon salt,
yolks two eggs, well beaten, and whites two eggs, beaten until stiff.
Fill buttered timbale moulds, set in pan of hot water, cover with
buttered paper, and bake fifteen minutes. Remove to serving dish and
pour around

\textbf{Mushroom Sauce.} Clean five large mushroom caps, cut in halves
crosswise, then in slices. Sauté in three tablespoons butter five
minutes; dredge with two tablespoons flour, add one-third cup cream and
one cup chicken stock, and cook two minutes. Season with salt and
paprika, and add one chopped truffle.



\needspace{15\baselineskip}
\section*{Sweetbread Mousse}

Parboil a sweetbread ten minutes, chop, and rub through sieve; there
should be one-half cup. Mix with one-third cup breast meat of a raw
chicken, and rub through sieve. Pound in mortar, add gradually white of
one egg, and work until smooth, then add three-fourths cup heavy cream.
Line buttered timbale moulds with mixture, fill centres, cover with
mixture, place in a pan of hot water, cover with buttered paper and bake
until firm. Remove to serving dish, and pour around sauce.

\textbf{Filling.} Melt one tablespoon butter, add one tablespoon corn-starch,
and pour on gradually one-fourth cup White Stock; then add one-third cup
parboiled sweetbread cut in cubes, one tablespoon Sherry wine, and salt
and pepper to taste.

\textbf{Sauce.} Melt three tablespoons butter, add three tablespoons flour, and
pour on one cup rich chicken stock and one-half cup heavy cream. Season
with one tablespoon Sherry wine, one-fourth teaspoon beef extract, and
salt and pepper to taste.



\needspace{15\baselineskip}
\section*{Suprême Of Chicken}

      Breast and second joints of uncooked chicken weighing 4 lbs.
      4 eggs
      1 1/3 cups thick cream
      Salt and pepper

Force chicken through a meat chopper, or chop very finely. Beat eggs
separately, add one at a time, stirring until mixture is smooth. Add
cream, and season with salt and pepper. Turn into slightly buttered
Dario moulds, and bake same as Lobster Timbales, allowing thirty minutes
for baking. Serve with Suprême or Béchamel Sauce.



\needspace{15\baselineskip}
\section*{Devilled Oysters}


\begin{minipage}{1.0\textwidth}
{\setlength{\multicolsep}{0pt}\setlength{\columnsep}{2em}\raggedcolumns%
\begin{multicols}{2}
\begin{itemize}
\setlength{\itemsep}{0pt}
\setlength{\parsep}{0pt}
\item 1 pint oysters
\item 1/4 cup butter
\item 1/4 cup flour
\item 2/3 cup milk
\item Yolk 1 egg
\item 1/2 tablespoon finely chopped
\item parsley
\item 1/2 teaspoon salt
\item Few grains cayenne
\item 1 teaspoon lemon juice
\item Buttered cracker crumbs
\end{itemize}
\end{multicols}}
\end{minipage}

\vspace{0.3em}
\noindent%
Clean, drain, and slightly chop oysters. Make a sauce of butter, flour,
and milk; add egg yolk, seasonings, and oysters. Arrange buttered
scallop shells in a dripping-pan, half fill with mixture, cover with
buttered crumbs, and bake twelve to fifteen minutes in a hot oven. Deep
oyster shells may be used in place of scallop shells.



\needspace{15\baselineskip}
\section*{Crab Meat, Indienne}


\begin{minipage}{1.0\textwidth}
{\setlength{\multicolsep}{0pt}\setlength{\columnsep}{2em}\raggedcolumns%
\begin{multicols}{2}
\begin{itemize}
\setlength{\itemsep}{0pt}
\setlength{\parsep}{0pt}
\item 2 tablespoons butter
\item 1 teaspoon finely chopped onion
\item 3 tablespoons flour
\item 2/3 tablespoon curry powder
\item 1 cup chicken stock
\item 1 cup crab meat
\item Salt
\end{itemize}
\end{multicols}}
\end{minipage}

\vspace{0.3em}
\noindent%
Cook butter with onion three minutes, add flour mixed with curry powder
and chicken stock. When boiling-point is reached add crab meat and
season with salt.



\needspace{15\baselineskip}
\section*{Devilled Crabs}


\begin{minipage}{1.0\textwidth}
{\setlength{\multicolsep}{0pt}\setlength{\columnsep}{2em}\raggedcolumns%
\begin{multicols}{2}
\begin{itemize}
\setlength{\itemsep}{0pt}
\setlength{\parsep}{0pt}
\item 1 cup chopped crab meat
\item 1/4 cup mushrooms, finely chopped
\item 2 tablespoons butter
\item 2 tablespoons flour
\item 2/3 cup White Stock
\item 4 egg yolks
\item 2 tablespoons Sherry wine
\item 1 teaspoon finely chopped parsley
\item Salt and pepper
\end{itemize}
\end{multicols}}
\end{minipage}

\vspace{0.3em}
\noindent%
Make a sauce of butter, flour, and stock; add yolks of eggs, seasonings
(except parsley), crab meat, and mushrooms. Cook three minutes, add
parsley, and cool mixture. Wash and trim crab shells, fill rounding with
mixture, sprinkle with stale bread crumbs mixed with a small quantity of
melted butter. Crease on top with a case knife, having three lines
parallel with each other across shell and three short lines branching
from outside parallel lines. Bake until crumbs are brown.



\needspace{15\baselineskip}
\section*{Devilled Scallops}


\begin{minipage}{1.0\textwidth}
{\setlength{\multicolsep}{0pt}\setlength{\columnsep}{2em}\raggedcolumns%
\begin{multicols}{2}
\begin{itemize}
\setlength{\itemsep}{0pt}
\setlength{\parsep}{0pt}
\item 1 quart scallops
\item 1/3 cup butter
\item 1/3 teaspoon made mustard
\item 1 teaspoon salt
\item Few grains cayenne
\item 2/3 cup buttered cracker crumbs
\end{itemize}
\end{multicols}}
\end{minipage}

\vspace{0.3em}
\noindent%
Clean scallops, drain, and heat to the boiling-point; drain again, and
reserve liquor. Cream the butter, add mustard, salt, cayenne, two-thirds
cup reserved liquor, and scallops chopped. Let stand one-half hour. Put
in a baking-dish, cover with crumbs, and bake twenty minutes.



\needspace{15\baselineskip}
\section*{Fried Oyster Crabs}

Wash and drain crabs. Roll in flour, and shake in a sieve to remove
superfluous flour. Fry in a basket in deep fat, having fat same
temperature as for cooked mixtures. Drain, and place on a napkin, and
garnish with parsley and slices of lemon. Serve with Sauce Tyrolienne.



\needspace{15\baselineskip}
\section*{Bouchées Of Oyster Crabs}

Pick over oyster crabs, dip in flour, cold milk, and crumbs, fry in deep
fat, and drain on brown paper. Fill bouchée cases with crabs.



\needspace{15\baselineskip}
\section*{Halibut Marguerites}

Line a buttered tablespoon with Fish Force-meat II. Fill with Creamed
Lobster, cover with force-meat, and garnish with force-meat, forced
through a pastry bag and tube, in the form of a marguerite, having the
centre colored yellow. Slip from spoon into boiling water, and cook
eight minutes. Serve with Béchamel or Lobster Sauce.



\needspace{15\baselineskip}
\section*{Cromesquis À La Russe}

Melt two tablespoons butter, add two tablespoons flour, and pour on
gradually one-half cup milk; then add one-half cup finnan haddie which
has been parboiled, drained, and separated into small pieces. Season
with cayenne, and spread on a plate to cook. Cut French pancakes in
pieces two by four inches. On lower halves of pieces put one tablespoon
mixture. Brush edges with beaten egg, fold over upper halves, press
edges firmly together, dip in crumbs, egg, and crumbs, fry in deep fat,
and drain. Serve garnished with parsley.

\textbf{French Pancakes.} To one-fourth cup bread flour add one-third cup milk,
one egg, and one-fourth teaspoon salt; beat thoroughly. Heat an omelet
pan, butter generously, cover bottom of pan with mixture, cook until
browned on one side, turn, and cook on other side.



\needspace{15\baselineskip}
\section*{Shad Roe With Celery}

Clean a shad roe, cook in boiling, salted, acidulated water twenty
minutes, and drain. Plunge into cold water, drain, remove membrane, and
separate roe into pieces. Melt three tablespoons butter, add roe, and
cook ten minutes; then add one tablespoon butter, one-half cup chopped
celery, few drops each onion and lemon juice, and salt and pepper. Serve
on pieces of toasted bread.



\needspace{15\baselineskip}
\section*{Stuffed Clams}

Cover bottom of dripping-pan with rock salt. Arrange two quarts
large-sized soft-shelled clams on salt, in such a manner that liquor
will not run into pan as clam-shells open. As soon as shells begin to
open, remove clams from shells, and chop. Reserve liquor, strain, and
use in making a thick sauce (follow directions for thick White Sauce for
Croquettes, p. 266), making one-half rule, and using one-fourth cup each
clam liquor and cream. Season highly with lemon juice and cayenne.
Moisten clams with sauce, fill shells, sprinkle with grated cheese,
cover with buttered soft stale bread crumbs, and bake in a hot oven
until crumbs are brown.



\needspace{15\baselineskip}
\section*{Crab Meat, Terrapin Style}


\begin{minipage}{1.0\textwidth}
{\setlength{\multicolsep}{0pt}\setlength{\columnsep}{2em}\raggedcolumns%
\begin{multicols}{2}
\begin{itemize}
\setlength{\itemsep}{0pt}
\setlength{\parsep}{0pt}
\item 1 cup crab meat
\item 2 tablespoons butter
\item 1/2 small onion, thinly sliced
\item 2 tablespoons Sherry wine
\item 1/3 cup heavy cream
\item 4 egg yolks
\item Salt and cayenne
\end{itemize}
\end{multicols}}
\end{minipage}

\vspace{0.3em}
\noindent%
Cook butter and onion until yellow; remove onion, add crab meat and
wine. Cook three minutes, add cream, yolks of eggs, salt, and cayenne.



\needspace{15\baselineskip}
\section*{Mock Crabs}


\begin{minipage}{1.0\textwidth}
{\setlength{\multicolsep}{0pt}\setlength{\columnsep}{2em}\raggedcolumns%
\begin{multicols}{2}
\begin{itemize}
\setlength{\itemsep}{0pt}
\setlength{\parsep}{0pt}
\item 4 tablespoons butter
\item 1/2 cup flour
\item 1 1/2 teaspoons salt
\item 3/4 teaspoon mustard
\item 1/4 teaspoon paprika
\item 1 1/2 cups scalded milk
\item 1 can Kornlet
\item 1 egg
\item 3 teaspoons Worcestershire Sauce
\item 1 cup buttered cracker crumbs
\end{itemize}
\end{multicols}}
\end{minipage}

\vspace{0.3em}
\noindent%
Melt butter, add flour mixed with dry seasonings, and pour on gradually
the milk. Add Kornlet, egg slightly beaten, and Worcestershire Sauce.
Pour into a buttered baking-dish, cover with crumbs, and bake until
crumbs are brown.



\needspace{15\baselineskip}
\section*{Martin's Specialty}


\begin{minipage}{1.0\textwidth}
{\setlength{\multicolsep}{0pt}\setlength{\columnsep}{2em}\raggedcolumns%
\begin{multicols}{2}
\begin{itemize}
\setlength{\itemsep}{0pt}
\setlength{\parsep}{0pt}
\item 1/2 tablespoon onion (finely chopped)
\item 2 tablespoons butter
\item 1 cup chopped cooked chicken or veal
\item 1 cup soft bread crumbs
\item Stock
\item 1 egg yolk
\item Salt and pepper
\item Lettuce
\end{itemize}
\end{multicols}}
\end{minipage}

\vspace{0.3em}
\noindent%
Cook onion in butter three minutes. Add meat and bread crumbs, moisten
with stock, and add egg yolk and seasonings. Wrap in lettuce leaves,
allowing two tablespoons mixture to each portion. Tie in cheese-cloth
and steam. Remove to serving dish and pour around Tomato Sauce.



\needspace{15\baselineskip}
\section*{Sweetbread Ramequins}

Clean and parboil a sweetbread and cut in cubes. Melt two tablespoons
butter, add three tablespoons flour, and pour on gradually one cup
chicken stock. Reheat sweetbread in sauce and add one-fourth cup heavy
cream and one and one-half teaspoons beef extract. Season with salt,
paprika, and lemon juice. Fill ramequin dishes, cover with buttered
crumbs, and bake until crumbs are brown.



\needspace{15\baselineskip}
\section*{Sweetbread À La Mont Vert}

Parboil a pair of sweetbreads, and gash. Decorate in gashes with
truffles cut in thin slices, and slice in fancy shapes. Melt three
tablespoons butter, add two slices onion, six slices carrot, and
sweetbreads; fry five minutes. Pour off butter, and add one-fourth cup
brown stock and two tablespoons Sherry wine. Cook in oven twenty-five
minutes, basting often until well glazed. Serve in nests of peas, and
pour around Mushroom Sauce.

\textbf{Nests.} Drain and rinse one can peas, and rub through a sieve. Add
three tablespoons butter, and salt and pepper to taste. Heat to
boiling-point, and shape in nests, using pastry bag and tube.

\textbf{Mushroom Sauce.} Clean three large mushroom caps, cut in halves
crosswise, then in slices. Sauté in two tablespoons butter five minutes.
Dredge with one tablespoon flour, and add one cup cream and liquor left
in pan in which sweetbreads were cooked. Cook two minutes.



\needspace{15\baselineskip}
\section*{Sweetbread In Peppers}

Parboil sweetbread, cool, and cut in small pieces; there should be one
cup. Melt two tablespoons butter, add two tablespoons flour, and pour on
gradually one-half cup chicken stock; then add two tablespoons heavy
cream, and one-third cup mushroom caps broken in small pieces. Season
with salt, paprika, and Worcestershire Sauce. Cut a slice from stem end
of six peppers, remove seeds, and parboil peppers fifteen minutes. Cool,
fill, cover with buttered crumbs, and bake until crumbs are brown. Break
stems of mushrooms, cover with cold water, and cook slowly twenty
minutes. Melt two tablespoons butter, add a few drops onion juice, two
tablespoons flour, and pour on gradually the water drained from mushroom
stems, and enough chicken stock to make one cup. Add one-fourth cup
heavy cream, and season with salt and paprika. Pour sauce around
peppers.



\needspace{15\baselineskip}
\section*{Cutlets Of Chicken}

Remove fillets from two chickens; for directions, see page 245. Make six
parallel slanting incisions in each mignon fillet and insert in each a
slice of truffle, having the part of truffle exposed cut in points on
edge. Arrange small fillets on large fillets. Garnish with truffles cut
in small shapes, and Chicken Force-meat forced through a pastry bag and
tube. Place in a greased pan, add one-third cup White Stock, cover with
buttered paper, and bake fifteen minutes in a hot oven. Serve with
Suprême or Béchamel Sauce.



\needspace{15\baselineskip}
\section*{Fillets Of Game}

Remove skin from breasts of three partridges. Cut off breasts, leaving
wing joints attached. Separate large from mignon fillets. Make five
parallel slanting incisions in each mignon fillet, and insert in each a
slice of truffle, having part of truffle exposed cut in points on edge.
Beginning at outer edge of large fillets make deep cuts, nearly
separating fillets in two parts, and stuff with Chicken Force-meat I or
II. Arrange small fillets on large fillets. Place in a greased
baking-pan, brush over with butter, add one tablespoon Madeira wine and
two tablespoons mushroom liquor. Cover with buttered paper, and bake
twelve minutes in a hot oven. Serve with Suprême Sauce.



\needspace{15\baselineskip}
\section*{Chicken Cutlets}

Remove fillets from two chickens; for directions, see page 245. Dip each
in thick cream, roll in flour, and sauté in lard three minutes. Place in
a pan, dot over with butter, and bake ten minutes. Serve with White
Sauce I, to which is added one tablespoon meat extract.



\needspace{15\baselineskip}
\section*{Russian Cutlets}

Cover bottom of cutlet moulds with Russian Pilaf and cover Pilaf with
Chicken Force-meat II (see p. 150), doubling the recipe and omitting
nutmeg. Set moulds in pan of hot water, cover with buttered paper, and
bake in a moderate oven fifteen minutes. Remove from moulds to serving
dish, surround with Brown Mushroom Sauce, and garnish with parsley.

\textbf{Russian Pilaf.} Wash one-half cup rice. Mix one cup highly seasoned
chicken stock with three-fourths cup stewed and strained tomato, and
heat to boiling-point. Add rice, and steam until rice is soft. Add two
tablespoons butter, stirring lightly with a fork that kernels may not be
broken, and season with salt.



\needspace{15\baselineskip}
\section*{Brown Mushroom Sauce}


\begin{minipage}{1.0\textwidth}
{\setlength{\multicolsep}{0pt}\setlength{\columnsep}{2em}\raggedcolumns%
\begin{multicols}{2}
\begin{itemize}
\setlength{\itemsep}{0pt}
\setlength{\parsep}{0pt}
\item 3 tablespoons butter
\item 1 slice carrot
\item 1 slice onion
\item 1 tablespoon lean raw ham, finely chopped
\item 5 tablespoons flour
\item 1 1/4 cups brown stock
\item 1/2 lb. mushrooms
\item 1 cup cold water
\item 1 teaspoon beef extract
\item Salt
\item Pepper
\end{itemize}
\end{multicols}}
\end{minipage}

\vspace{0.3em}
\noindent%
Cook butter with vegetables and ham until brown, add flour, and when
well browned add stock, gradually, then strain. Clean mushroom stems,
break in pieces, cover with water, and cook slowly until stock is
reduced to one-third cup. Strain, and add to sauce with beef extract and
seasonings. Just before serving add mushroom caps peeled, cut in slices
lengthwise, and sautéd in butter five minutes.



\needspace{15\baselineskip}
\section*{Chicken À La Mcdonald}


\begin{minipage}{1.0\textwidth}
{\setlength{\multicolsep}{0pt}\setlength{\columnsep}{2em}\raggedcolumns%
\begin{multicols}{2}
\begin{itemize}
\setlength{\itemsep}{0pt}
\setlength{\parsep}{0pt}
\item 1 cup cold cooked chicken, cut in strips
\item 3 cold boiled potatoes, cut in one-third inch slices
\item 1 truffle cut in strips
\item 3 tablespoons butter
\item 3 tablespoons flour
\item 1 1/2 cups scalded milk
\item Salt
\item Pepper
\end{itemize}
\end{multicols}}
\end{minipage}

\vspace{0.3em}
\noindent%
Make a sauce of butter, flour, and milk. Add chicken, potatoes, and
truffle, and, as soon as heated, add seasoning.



\needspace{15\baselineskip}
\section*{Chicken Mousse}

Make a chicken force-meat of one-half the breast of a raw chicken
pounded and forced through a purée strainer, the white of one egg
slightly beaten, one-half cup heavy cream, and salt, pepper, and cayenne
to taste. Add three-fourths cup cooked white chicken meat rubbed through
a sieve, the white of an egg slightly beaten, and one-half cup heavy
cream beaten until stiff. Decorate a buttered mould with truffles, turn
in mixture, set in pan of hot water, cover with buttered paper, and bake
until firm. Remove to platter, and pour around Cream or Béchamel Sauce.



\needspace{15\baselineskip}
\section*{Fillets Of Chicken, Sauce Suprême}

Remove fillets from three chickens, leaving wing joint and a piece of
bone attached to each fillet. Reserve mignon fillets for the making of
force-meat. Make a pocket in each large fillet, and stuff with one-half
tablespoon force-meat; close pockets, and fasten each with five pieces
of truffle, shaped to represent nails and drawn through with a larding
needle. Sprinkle with salt and pepper, put in small baking-pan, brush
over with cold water, add one-half cup Madeira wine, cover with buttered
paper, and bake in a hot oven ten minutes. Arrange cooked mushroom caps
overlapping one another the entire length of platter, put a chop frill
on bone of each fillet, and put three fillets on each side of mushrooms.
Garnish with celery tips and pour around





\textbf{Sauce Suprême.} Cook remaining chicken with one small sliced carrot,
one onion, one stalk celery, two sprigs parsley, and a bit of bay leaf,
with enough water to cover, one hour. Strain and cook stock until
reduced to one cup. Melt two tablespoons butter, add two tablespoons
flour, and pour on stock; cook slowly fifteen minutes. Add three-fourths
cup heavy cream and season with salt and pepper; then add twelve peeled
white mushroom caps and cook five minutes. Remove caps to platter and
add one-fourth cup heavy cream to sauce.

\textbf{Chicken Force-meat.} Put mignon fillets through a meat chopper, add
one-half the quantity of stale bread crumbs cooked with milk until
moisture has nearly evaporated. Cool and put through purée strainer;
then add one and one-half tablespoons melted butter, yolk one egg, two
tablespoons cream, and salt and pepper to taste.



\needspace{15\baselineskip}
\section*{Birds On Canapés}

Split five birds (quails or squabs), season with salt and pepper, and
spread with four tablespoons butter, rubbed until creamy, and mixed with
three tablespoons flour. Bake in a hot oven until well browned, basting
every four minutes with two tablespoons butter, melted in one-fourth cup
water. Chop six boiled chickens' livers, season with salt, pepper, and
onion juice, moisten with melted butter, and add one teaspoon finely
chopped parsley. Spread mixture on five pieces toasted bread, arrange a
bird on each canapé, and garnish with parsley.



\needspace{15\baselineskip}
\section*{Breast Of Quail Lucullus}

Remove breasts from six quail, lard, and bake in a hot oven twenty
minutes, basting every five minutes with a very rich brown stock, that
breasts may have a glazed appearance. Mould corn meal or hominy mush in
cone shape; when firm remove from mould and sprinkle with finely chopped
parsley. Arrange breasts on cone around base, and make six nests of
mashed seasoned sweet potato around base of cone at equal distances,
using a pastry bag and rose tube. Fill nests with creamed mushrooms and
sweetbread. Garnish between nests with toasted bread points, the tips of
which have been brushed with white of egg, then dipped in finely chopped
parsley. Insert a stab frill in each nest and one in top of cone.

Serve with one and one-half cups rich brown sauce seasoned with tomato
catsup and mashed sweet potato. A small amount of the sweet potato gives
a suggestion of chestnuts.



\needspace{15\baselineskip}
\section*{Pan Broiled Lamb Chops À La Lucullus}

Pan broil lamb chops and garnish same as Breast of Quail Lucullus.



\needspace{15\baselineskip}
\section*{Chickens' Livers En Brochette}

Cut each liver in four pieces. Alternate pieces of liver and pieces of
thinly sliced bacon on skewers, allowing one liver and five pieces of
bacon for each skewer. Balance skewers in upright positions on rack in
dripping-pan. Bake in a hot oven until bacon is crisp. Serve garnished
with watercress.



\needspace{15\baselineskip}
\section*{Chestnuts En Casserole}

Remove shells from three cups chestnuts, put in a casserole dish, and
pour over three cups highly seasoned chicken stock. Cover, and cook in a
slow oven three hours; then thicken chicken stock with two tablespoons
butter and one and one-half tablespoons flour cooked together. Send to
table in casserole dish.



\needspace{15\baselineskip}
\section*{Cheese Fondue}


\begin{minipage}{1.0\textwidth}
{\setlength{\multicolsep}{0pt}\setlength{\columnsep}{2em}\raggedcolumns%
\begin{multicols}{2}
\begin{itemize}
\setlength{\itemsep}{0pt}
\setlength{\parsep}{0pt}
\item 1 cup scalded milk
\item 1 cup soft stale bread crumbs
\item 1/4 lb. mild cheese, cut in small pieces
\item 1 tablespoon butter
\item 1/2 teaspoon salt
\item 3 egg yolks
\item 3 egg whites
\end{itemize}
\end{multicols}}
\end{minipage}

\vspace{0.3em}
\noindent%
Mix first five ingredients, add yolks of eggs beaten until
lemon-colored. Cut and fold in whites of eggs beaten until stiff. Pour
in a buttered baking-dish, and bake twenty minutes in a moderate oven.



\needspace{15\baselineskip}
\section*{Cheese Soufflé}


\begin{minipage}{1.0\textwidth}
{\setlength{\multicolsep}{0pt}\setlength{\columnsep}{2em}\raggedcolumns%
\begin{multicols}{2}
\begin{itemize}
\setlength{\itemsep}{0pt}
\setlength{\parsep}{0pt}
\item 2 tablespoons butter
\item 3 tablespoons flour
\item 1/2 cup scalded milk
\item 1/2 teaspoon salt
\item Few grains cayenne
\item 1/4 cup grated Old English or Young America cheese
\item 3 egg yolks
\item 3 egg whites
\end{itemize}
\end{multicols}}
\end{minipage}

\vspace{0.3em}
\noindent%
Melt butter, add flour, and when well mixed add gradually scalded milk.
Then add salt, cayenne, and cheese. Remove from fire; add yolks of eggs
beaten until lemon-colored. Cool mixture, and cut and fold in whites of
eggs beaten until stiff and dry. Pour into a buttered baking-dish, and
bake twenty minutes in a slow oven. Serve at once.



\needspace{15\baselineskip}
\section*{Ramequins Soufflés}

Bake Cheese Soufflé mixture in ramequin dishes. Serve for a course in a
dinner.



\needspace{15\baselineskip}
\section*{Cheese Balls}


\begin{minipage}{1.0\textwidth}
{\setlength{\multicolsep}{0pt}\setlength{\columnsep}{2em}\raggedcolumns%
\begin{multicols}{2}
\begin{itemize}
\setlength{\itemsep}{0pt}
\setlength{\parsep}{0pt}
\item 1 1/2 cups grated mild cheese
\item 1 tablespoon flour
\item 1/4 teaspoon salt
\item Few grains cayenne
\item 3 egg whites
\item Cracker dust
\end{itemize}
\end{multicols}}
\end{minipage}

\vspace{0.3em}
\noindent%
Mix cheese with flour and seasonings. Beat whites of eggs until stiff,
and add to first mixture. Shape in small balls, roll in cracker dust,
fry in deep fat, and drain on brown paper. Serve with salad course.



\needspace{15\baselineskip}
\section*{Compote Of Rice With Peaches}

Wash two-thirds cup rice, add one cup boiling water, and steam until
rice has absorbed water; then add one and one-third cups hot milk, one
teaspoon salt, and one-fourth cup sugar. Cook until rice is soft. Turn
into a slightly buttered round shallow mould. When shaped, remove from
mould to serving dish, and arrange on top sections of cooked peaches
drained from their syrup and dipped in macaroon dust. Garnish between
sections with candied cherries and angelica cut in leaf-shapes. Angelica
may be softened by dipping in hot water. Color peach syrup with fruit
red, and pour around mould.



\needspace{15\baselineskip}
\section*{Compote Of Rice And Pears}

Cook and mould rice as for Compote of Rice with Peaches. Arrange on top
quarters of cooked pears, and pour around pear syrup.



\needspace{15\baselineskip}
\section*{Croustades Of Bread}

Cut stale bread in diamonds, squares, or circles. Remove centres,
leaving cases. Fry in deep fat or brush over with melted butter, and
brown in oven. Fill with creamed vegetables, fish, or meat.



\needspace{15\baselineskip}
\section*{Rice Croustades}

Wash one cup rice, and steam in White Stock. Cool, and mix with
three-fourths cup Thick White Sauce, to which has been added beaten yolk
of one egg, slight grating of nutmeg, one-half teaspoon salt, and
one-eighth teaspoon pepper. Spread mixture in buttered pan two inches
thick, cover with buttered paper, and place weight on top. Let stand
until cold. Turn from pan, cut in rounds, remove centres, leaving cases;
dip in crumbs, egg, and crumbs, and fry in deep fat. Fill with creamed
fish.



\needspace{15\baselineskip}
\section*{Soufflé Au Rhum}


\begin{itemize}
\setlength{\itemsep}{0pt}
\setlength{\parsep}{0pt}
\item 4 egg yolks
\item 1/4 cup powdered sugar
\item 1 tablespoon rum
\item 4 egg whitess
\item Few grains salt
\end{itemize}

\vspace{-0.5em}
\noindent%
Beat yolks of eggs until lemon-colored. Add sugar, salt, and rum. Cut
and fold in whites of eggs beaten until stiff and dry. Butter a hot
omelet pan, pour in one-half mixture, brown underneath, fold gradually,
turn on a hot serving dish, and sprinkle with powdered sugar. Cook
remaining mixture in same way. Soufflé au Rhum should be slightly
underdone inside. At gentlemen's dinners rum is sometimes poured around
soufflé and lighted when sent to table.



\needspace{15\baselineskip}
\section*{Omelet Soufflé}


\begin{itemize}
\setlength{\itemsep}{0pt}
\setlength{\parsep}{0pt}
\item 4 egg yolks
\item 1/4 cup powdered sugar
\item 1/2 teaspoon vanilla
\item 4 egg whitess
\item Few grains salt
\end{itemize}

\vspace{-0.5em}
\noindent%
Prepare same as Soufflé au Rhum. Mound three-fourths of mixture on a
slightly buttered platter. Decorate mound with remaining mixture forced
through a pastry bag and tube. Sprinkle with powdered sugar, and bake
ten minutes in a moderate oven.



\needspace{15\baselineskip}
\section*{Patties}

Patty shells are filled with Creamed Oysters, Oysters in Brown Sauce,
Creamed Chicken, Creamed Chicken and Mushrooms, or Creamed Sweetbreads.
They are arranged on a folded napkin, and are served for a course at
dinner or luncheon.



\needspace{15\baselineskip}
\section*{Bouchées}

Small pastry shells filled with creamed meat are called bouchées.



\needspace{15\baselineskip}
\section*{Vol-Au-Vents}

Vol-au-vents are filled same as patty shells.



\needspace{15\baselineskip}
\section*{Rissoles}

Roll puff paste to one-eighth inch thickness, and cut in rounds. Place
one teaspoon finely chopped seasoned meat moistened with Thick White
Sauce on each round. Brush each piece with cold water half-way round
close to edge. Fold like a turnover, and press edges together. Dip in
egg slightly beaten and diluted with one tablespoon water. Roll in
gelatine, fry in deep fat, and drain. Granulated gelatine cannot be
used.

\textbf{Filling for Rissoles.} Mix one-half cup finely chopped cold cooked
chicken with one-fourth cup finely chopped cooked ham. Moisten with
Thick White Sauce, and season with salt and cayenne.



\needspace{15\baselineskip}
\section*{Cigarettes À La Prince Henry}

Roll puff paste very thin, and spread with Chicken Force-meat. Roll like
a jelly roll, and cut in pieces four inches long and a little larger
round than a cigarette. Brush over with egg, roll in crumbs, fry in deep
fat, and drain on brown paper. Arrange log-cabin fashion on a folded
doily, and serve while hot.



\needspace{15\baselineskip}
\section*{Zigaras À La Russe}

Make and fry same as Cigarettes à la Prince Henry, using cheese mixture
in place of Chicken Force-meat. Melt two tablespoons butter, add four
tablespoons flour, and pour on gradually one-half cup milk, then add one
tablespoon heavy cream, one egg yolk, and one-third cup grated cheese.
Season highly with salt and cayenne. Cool before spreading on paste.



\needspace{15\baselineskip}
\section*{Dresden Patties}

Cut stale bread in two-inch slices, shape with a round cutter three
inches in diameter, and remove centres, making cases. Dip cases in egg,
slightly beaten, diluted with milk and seasoned with salt, allowing two
tablespoons milk to each egg. When bread is thoroughly soaked, drain,
and fry in deep fat. Fill with any mixture suitable for patty cases.



\needspace{15\baselineskip}
\section*{Russian Patties}


\begin{minipage}{1.0\textwidth}
{\setlength{\multicolsep}{0pt}\setlength{\columnsep}{2em}\raggedcolumns%
\begin{multicols}{2}
\begin{itemize}
\setlength{\itemsep}{0pt}
\setlength{\parsep}{0pt}
\item 1 pint oysters
\item 3 tablespoons butter
\item 4 1/2 tablespoons flour
\item 1/2 cup chicken stock
\item 1/2 cup cream
\item 1/2 tablespoon vinegar
\item 3/4 tablespoon lemon juice
\item 4 egg yolks
\item 1 tablespoon grated horseradish
\item 2 tablespoons capers
\item Salt and pepper
\end{itemize}
\end{multicols}}
\end{minipage}

\vspace{0.3em}
\noindent%
Parboil oysters, drain, and reserve liquor; there should be one-half
cup. Make sauce of butter, flour, stock, oyster liquor, and cream; add
yolks of eggs, seasonings, and salt and pepper to taste. Add oysters,
and as soon as oysters are heated, fill patty shells.



\needspace{15\baselineskip}
\section*{Cheese Soufflé With Pastry}


\begin{minipage}{1.0\textwidth}
{\setlength{\multicolsep}{0pt}\setlength{\columnsep}{2em}\raggedcolumns%
\begin{multicols}{2}
\begin{itemize}
\setlength{\itemsep}{0pt}
\setlength{\parsep}{0pt}
\item 2 eggs
\item 2/3 cup thick cream
\item 1/2 cup Swiss cheese, cut in small dice
\item 1/2 cup grated American cheese
\item 1/3 cup grated Parmesan cheese
\item Salt and pepper
\item Few grains cayenne
\item Few gratings nutmeg
\end{itemize}
\end{multicols}}
\end{minipage}

\vspace{0.3em}
\noindent%
Add eggs to cream and beat slightly, then add cheese and seasonings.
Line the sides of ramequin dishes with strips of puff paste. Fill dishes
with mixture until two-thirds full. Bake fifteen minutes in a hot oven.



\needspace{15\baselineskip}
\section*{Lamb Rissoles À L'Indienne}

Roll puff paste one-eighth inch thick and shape, using circular cutters
of different sizes. On the centres of smaller pieces put one tablespoon
prepared lamb mixture, wet edges, cover with large pieces, press edges
firmly together, prick upper paste in several places, brush over with
yolk of egg diluted with one teaspoon cold water, and bake in hot oven.

\textbf{Lamb Filling.} Cook three tablespoons butter, with a few drops onion
juice, until well browned, add one-fourth cup flour, and brown butter
and flour, then add one cup lamb stock. Season highly with salt,
paprika, and curry powder. To one-half the sauce, add two-thirds cup
cold roast lamb cut in one-third inch cubes. Add stock to remaining
sauce, and pour around rissoles just before sending to table.



\needspace{15\baselineskip}
\section*{Quail Pies}


\begin{minipage}{1.0\textwidth}
{\setlength{\multicolsep}{0pt}\setlength{\columnsep}{2em}\raggedcolumns%
\begin{multicols}{2}
\begin{itemize}
\setlength{\itemsep}{0pt}
\setlength{\parsep}{0pt}
\item 6 quails
\item 6 slices carrot
\item Stalk of celery
\item 2 slices onion
\item Sprig of parsley
\item Bit of bay leaf
\item 1/4 teaspoon peppercorns
\item Flour
\item Salt and pepper
\item Sherry wine
\end{itemize}
\end{multicols}}
\end{minipage}

\vspace{0.3em}
\noindent%
Remove breasts and legs from birds, season with salt and pepper, dredge
with flour, and sauté in butter. To butter in pan add vegetables and
peppercorns, and cook five minutes. Separate backs of birds in pieces,
cover with cold water, add vegetables, and cook slowly one hour. Drain
stock from vegetables, and thicken with flour diluted with enough cold
water to pour easily. Season with salt, pepper, and wine. If not rich
enough, add more butter. Allow one bird to each individual dish, sauce
to make sufficiently moist, and cover with plain or puff paste, in which
make two incisions, through which the legs of the bird should extend.



\needspace{15\baselineskip}
\section*{Aspic Jelly}


\begin{minipage}{1.0\textwidth}
{\setlength{\multicolsep}{0pt}\setlength{\columnsep}{2em}\raggedcolumns%
\begin{multicols}{2}
\begin{itemize}
\setlength{\itemsep}{0pt}
\setlength{\parsep}{0pt}
\item 2 tablespoons carrot
\item 2 tablespoons onion
\item 2 tablespoons celery
\item 2 sprigs parsley
\item 2 sprigs thyme
\item 1 sprig savory
\item 2 cloves
\item 1/2 teaspoon peppercorns
\item 1 bay leaf
\item 7/8 cup white wine
\item 1 box gelatine
\item 1 quart White Stock for vegetables and
\item white meat, or
\item 1 quart Brown Stock for dark meat
\item Juice 1 lemon
\item 3 egg whites
\end{itemize}
\end{multicols}}
\end{minipage}

\vspace{0.3em}
\noindent%
Aspic jelly is always made with meat stock, and is principally used in
elaborate entrées where fish, chicken, game, or vegetables are to be
served moulded in jelly. In making Aspic Jelly, use as much liquid as
the pan which is to contain moulded dish will hold.

Put vegetables, seasonings, and wine (except two tablespoons) in a
saucepan; cook eight minutes, and strain, reserving liquid. Add gelatine
to stock, then add lemon juice. Heat to boiling-point and add strained
liquid. Season with salt and cayenne. Beat whites of eggs slightly, add
two tablespoons wine, and dilute with one cup hot mixture. Add slowly to
remaining mixture, stirring constantly until boiling-point is reached.
Place on back of range and let stand thirty minutes. Strain through a
double cheese-cloth placed over a fine wire strainer, or through a jelly
bag.



\needspace{15\baselineskip}
\section*{Tomatoes In Aspic}

Peel six small firm tomatoes, and remove pulp, having opening in tops as
small as possible. Sprinkle insides with salt, invert, and let stand
thirty minutes. Fill with vegetable or chicken salad. Cover tops with
Mayonnaise to which has been added a small quantity of dissolved
gelatine, and garnish with capers and sliced pickles. Place a pan in
ice-water, cover bottom with aspic jelly mixture, and let stand until
jelly is firm. Arrange tomatoes on jelly garnished side down. Add more
aspic jelly mixture, let stand until firm, and so continue until all is
used. Chill thoroughly, turn on a serving dish, and garnish around base
with parsley.



\needspace{15\baselineskip}
\section*{Stuffed Olives In Aspic}

Stone olives, using an olive stoner, and fill cavities thus made with
green butter. Place small Dario moulds in pan of ice-water, and pour in
aspic jelly mixture (see p. 382) one-fourth inch deep. When firm put an
olive in each mould (keeping olives in place by means of small wooden
skewers) and add aspic by spoonfuls until moulds are filled. Chill
thoroughly, remove to circular slices of liver sausage, garnish with
green butter forced through a pastry bag and tube, yolks of
“hard-boiled” eggs forced through a strainer, and red peppers cut in
fancy shapes.

\textbf{Green Butter.} Mix yolk one “hard-boiled” egg, two tablespoons butter,
one sprig parsley, one sprig tarragon, one small shallot, one-half
teaspoon anchovy paste, one teaspoon capers, and one teaspoon chopped
gherkins, and pound in a mortar; then rub through a very fine sieve.
Season with salt and pepper, and add a few drops vinegar.



\needspace{15\baselineskip}
\section*{Tongue In Aspic}

Cook a tongue according to directions on page 210. After removing skin
and roots, run a skewer through tip of tongue and fleshy part, thus
keeping tongue in shape. When cool, remove skewer. Put a round pan in
ice-water, cover bottom with brown aspic, and when firm decorate with
cooked carrot, turnip, beet cut in fancy shapes, and parsley. Cover with
aspic jelly mixture, adding it by spoonfuls so as not to disarrange
vegetables. When this layer of mixture is firm, put in tongue, adding
gradually remaining mixture as in Tomatoes in Aspic.



\needspace{15\baselineskip}
\section*{Birds In Aspic}

Clean, bone, stuff, and truss a bird, then steam over body bones or
roast. If roasted, do not dredge with flour. Put a pan in ice-water,
cover bottom with aspic jelly mixture, and when firm garnish with
truffles and egg custard thinly sliced and cut in fancy shapes. The
smaller the shapes the more elaborate may be the designs. When
garnishing with small shapes, pieces are so difficult to handle that
they should be taken on the pointed end of a larding-needle, and placed
as desired on jelly. Add aspic mixture by spoonfuls, that designs may
not be disturbed. When mixture is added, and firm to the depth of
three-fourths inch, place in the bird, breast down. If sides of mould
are to be decorated, dip pieces in jelly and they will cling to pan. Add
remaining mixture gradually as in Tomatoes in Aspic. Small birds,
chicken, capon, or turkey, may be put in aspic.



\needspace{15\baselineskip}
\section*{Egg Custard For Decorating}

Separate yolks from whites of two eggs. Beat yolks slightly, add two
tablespoons milk and few grains salt. Strain into a buttered cup, put in
a saucepan, surround with boiling water to one-half depth of cup, cover,
put on back of range, and steam until custard is firm. Beat whites
slightly, add few grains salt, and cook as yolks. Cool, turn from cups,
cut in thin slices, then in desired shapes.



\needspace{15\baselineskip}
\section*{Stuffing For Chicken In Aspic}

Chop finely breast and meat from second joints of an uncooked chicken,
or one pound of uncooked lean veal. Add one-half cup cracker crumbs, hot
stock to moisten, salt, pepper, celery salt, cayenne, lemon juice, and
one egg slightly beaten. In stuffing boned chicken, stuff body, legs,
and wings, being careful that too much stuffing is not used, as an
allowance must be made for the swelling of cracker crumbs.



\needspace{15\baselineskip}
\section*{Spring Mousse}

Chop three-fourths cup cold cooked chicken or veal, and pound in a
mortar. Add gradually one-half cup heavy cream, and force mixture
through purée strainer. Add one-half tablespoon granulated gelatine
dissolved in three tablespoons White Stock. Add another one-half cup
heavy cream and season with salt, cayenne, and horseradish powder. Pour
jelly into small moulds one-third inch deep, using lemon Sauterne, or
aspic. When firm, fill moulds with veal mixture and set aside to chill.
Remove from moulds and serve on lettuce leaves.



\needspace{15\baselineskip}
\section*{Chaud-Froid Of Eggs}

Cut six “hard-boiled” eggs in halves lengthwise and remove yolks. Mix
one-third cup cold cooked chicken finely chopped, two tablespoons cold
cooked ham finely chopped, two tablespoons chopped raw mushroom caps,
one-half tablespoon chopped truffles, and yolks of four of the eggs
rubbed through a sieve. Moisten with Spanish Sauce and refill whites
with mixture. Mask eggs with Spanish Sauce, garnish with truffles, cut
in fancy shapes, and brush over with aspic. Arrange on serving dish and
garnish with cress.

\textbf{Spanish Sauce.} Cook one and one-half cups canned tomatoes fifteen
minutes with one-fourth onion, sprig of parsley, bit of bay leaf, six
cloves, one-third teaspoon salt, one-fourth teaspoon paprika, and a few
grains cayenne; then rub through a sieve. Beat yolks three eggs
slightly, and add, gradually, three tablespoons olive oil. Combine
mixtures and cook over hot water, stirring constantly. Add one
tablespoon granulated gelatine soaked in three-fourths tablespoon each
tarragon vinegar and cold water. Strain, and cool.



\needspace{15\baselineskip}
\section*{Jellied Vegetables}

Soak one tablespoon granulated gelatine in one-fourth cup cold water,
and dissolve in one cup boiling water; then add one-fourth cup, each,
sugar and vinegar, two tablespoons lemon juice, and one teaspoon salt.
Strain, cool, and when beginning to stiffen, add one cup celery cut in
small pieces, one-half cup finely shredded cabbage, and one and one-half
canned pimentoes cut in small pieces. Turn into a mould and chill.
Remove from mould and arrange around jelly thin slices of cold cooked
meat overlapping one another. Garnish with celery tips.



\needspace{15\baselineskip}
\section*{Mayonnaise Of Mackerel}

Clean two medium-sized mackerel, put in baking-dish with one-third cup
each water, cider vinegar, and tarragon vinegar, twelve cloves, one
teaspoon each peppercorns and salt, and a bit of bay leaf. Cover with
buttered paper and cook in a moderate oven. Arrange on serving dish,
remove skin, cool, and mask with Mayonnaise thickened with gelatine. Let
stand until thoroughly chilled, and garnish with sliced cucumbers, lemon
baskets filled with Mayonnaise sprinkled with finely chopped parsley,
and sprigs of parsley.



\needspace{15\baselineskip}
\section*{Chaud-Froid Of Chicken}

  2 tablespoons butter
  3 tablespoons flour
  1 cup White Stock
  Yolk one egg
  2 tablespoons cream
  1 tablespoon lemon juice
  Salt and pepper
  3/4 teaspoon granulated gelatine dissolved in one tablespoon hot water
  Aspic jelly
  Truffles
  6 pieces cooked chicken, shaped in form of cutlets

Make a sauce of butter, flour, and stock; add egg yolk diluted with
cream, lemon juice, salt and pepper; then add dissolved gelatine. Dip
chicken in sauce which has been allowed to cool. When chicken has
cooled, garnish upper side with truffles cut in shapes. Brush over with
aspic jelly mixture, and chill. Arrange a bed of lettuce; in centre pile
cold cooked asparagus tips or celery cut in small pieces, marinated with
French Dressing, and place chicken at base of salad.



\needspace{15\baselineskip}
\section*{Moulded Salmon, Cucumber Sauce}


\begin{minipage}{1.0\textwidth}
{\setlength{\multicolsep}{0pt}\setlength{\columnsep}{2em}\raggedcolumns%
\begin{multicols}{2}
\begin{itemize}
\setlength{\itemsep}{0pt}
\setlength{\parsep}{0pt}
\item 1 can salmon
\item 1/2 tablespoon salt
\item 1 1/2 tablespoons sugar
\item 1/2 tablespoon flour
\item 1 teaspoon mustard
\item Few grains cayenne
\item 4 egg yolks
\item 1 1/2 tablespoons melted butter
\item 3/4 cup milk
\item 1/4 cup vinegar
\item 3/4 tablespoon granulated gelatine
\item 2 tablespoons cold water
\end{itemize}
\end{multicols}}
\end{minipage}

\vspace{0.3em}
\noindent%
Remove salmon from can, rinse thoroughly with hot water, and separate in
flakes. Mix dry ingredients, add egg yolks, butter, milk, and vinegar.
Cook over boiling water, stirring constantly until mixture thickens. Add
gelatine soaked in cold water. Strain, and add to salmon. Fill
individual mould, chill, and serve with





\textbf{Cucumber Sauce II.} Beat one-half cup heavy cream until stiff, add
one-fourth teaspoon salt, a few grains pepper, and gradually two
tablespoons vinegar; then add one cucumber, pared, chopped, and drained.



\needspace{15\baselineskip}
\section*{Moulded Chicken, Sauterne Jelly}

Cover a four-pound fowl with two quarts cold water, and add four slices
carrot, one onion stuck with eight cloves, two stalks celery, bit of bay
leaf, one-half teaspoon peppercorns, and one tablespoon salt. Bring
quickly to boiling-point, and let simmer until meat is tender. Remove
meat from bones, and finely chop. Reduce stock to three-fourths cup,
cool, and remove fat. Soak one teaspoon granulated gelatine in one
teaspoon cold water, and dissolve in stock which has been reheated. Add
to meat, and season with salt, pepper, celery salt, lemon juice, and
onion juice. Pack solidly into a slightly buttered one-pound baking
powder tin, and chill. Remove from tin, cut in thin slices, and arrange
around Sauterne Jelly, beaten with a fork until light.

When making Sauterne Jelly (see p. 420) to serve with meat, use but
three tablespoons sugar.



\needspace{15\baselineskip}
\section*{Lenox Chicken}


\begin{minipage}{1.0\textwidth}
{\setlength{\multicolsep}{0pt}\setlength{\columnsep}{2em}\raggedcolumns%
\begin{multicols}{2}
\begin{itemize}
\setlength{\itemsep}{0pt}
\setlength{\parsep}{0pt}
\item 1 tablespoon granulated gelatine
\item 3/4 cup hot chicken stock
\item 3/4 cup heavy cream
\item 1 1/2 cups cold cooked chicken, cut in dice
\item 1/2 tablespoon granulated gelatine
\item 2 tablespoons cold water
\item 4 egg yolks
\item 1 teaspoon salt
\item 1 1/2 teaspoons sugar
\item 1 teaspoon mustard
\item 1/4 teaspoon pepper
\item 2 tablespoons lemon juice
\item 1 tablespoon vinegar
\item 1/2 cup hot cream
\item 1 1/2 tablespoons butter
\item 2 egg whites
\item 1/2 cup heavy cream
\item 2 cups finely chopped celery
\end{itemize}
\end{multicols}}
\end{minipage}

\vspace{0.3em}
\noindent%
Dissolve one tablespoon gelatine in chicken stock and strain. When
mixture begins to thicken beat until frothy, and add three-fourths cup
heavy cream, beaten until stiff, and chicken dice. Season with salt and
pepper, turn into individual moulds, and chill. Soak remaining gelatine
in cold water, dissolve by standing over hot water, then strain. Beat
yolks of eggs slightly and add salt, sugar, mustard, lemon juice,
vinegar, and hot cream. Cook over hot water until mixture thickens, add
butter and strained gelatine. Add mixture, gradually, to whites of eggs
beaten stiff, and when cold, fold in heavy cream beaten until stiff, and
celery. Remove chicken from mould, surround with sauce, and garnish with
celery tips.



\needspace{15\baselineskip}
\section*{Rum Cakes}

Shape Brioche dough in the form of large biscuits and put into buttered
individual tin moulds, having moulds two-thirds full; cover, and let
rise to fill moulds. Bake twenty-five minutes in a moderate oven. Remove
from moulds and dip in Rum Sauce. Arrange on a dish and pour remaining
sauce around cakes.



\needspace{15\baselineskip}
\section*{Rum Sauce}


\begin{itemize}
\setlength{\itemsep}{0pt}
\setlength{\parsep}{0pt}
\item 1/2 cup sugar
\item 1 cup boiling water
\item 1/4 cup rum or wine
\end{itemize}

\vspace{-0.5em}
\noindent%
Make a syrup by boiling sugar and water five minutes; then add rum or
wine.



\needspace{15\baselineskip}
\section*{Flûtes}

Shape Brioche dough in sticks similar to Bread Sticks. Place on a
buttered sheet, cover, and let rise fifteen minutes. Brush over with
white of one egg slightly beaten and diluted with one-half tablespoon
cold water. Sprinkle with powdered sugar and bake ten minutes. These are
delicious served with coffee or chocolate.



\needspace{15\baselineskip}
\section*{Baba Cakes}

To one and one-half cups Brioche dough add one-third cup each raisins
seeded and cut in pieces, currants, and citron thinly sliced, previously
soaked in Maraschino for one hour. Shape, let rise, and bake same as Rum
Cakes. Dip in sauce made same as Rum Sauce, substituting Maraschino in
place of rum.



\needspace{15\baselineskip}
\section*{Baba Cakes With Apricots}


\begin{minipage}{1.0\textwidth}
{\setlength{\multicolsep}{0pt}\setlength{\columnsep}{2em}\raggedcolumns%
\begin{multicols}{2}
\begin{itemize}
\setlength{\itemsep}{0pt}
\setlength{\parsep}{0pt}
\item 1 1/2 cups flour
\item 1 yeast cake dissolved in
\item 1/2 cup lukewarm water
\item 2/3 cup butter
\item 4 eggs
\item 1/2 cup sugar
\item 1/4 teaspoon salt
\end{itemize}
\end{multicols}}
\end{minipage}

\vspace{0.3em}
\noindent%
Make sponge of one-half cup flour and dissolved yeast cake; cover and
let rise. Mix remaining flour with butter, two eggs, sugar, and salt.
Beat thoroughly, and add, while beating, remaining eggs, one at a time,
then beat until mixture is perfectly smooth. As soon as sponge has
doubled its bulk, combine mixtures, beat thoroughly, and half fill
buttered individual tins. Let rise, and bake in a moderate oven. Remove
from tins, cut a circular piece from top of each, and scoop out a small
quantity of the inside. Fill centres thus made with Apricot Marmalade,
replace circular pieces, and serve with Wine Sauce (see p. 409).





\chapter{Hot Puddings}




\needspace{15\baselineskip}
\section*{Rice Pudding}


\begin{itemize}
\setlength{\itemsep}{0pt}
\setlength{\parsep}{0pt}
\item 4 cups milk
\item 1/3 cup rice
\item 1/2 teaspoon salt
\item 1/3 cup sugar
\item Grated rind 1/2 lemon
\end{itemize}

\vspace{-0.5em}
\noindent%
Wash rice, mix ingredients, and pour into buttered pudding-dish; bake
three hours in very slow oven, stirring three times during first hour of
baking to prevent rice from settling.



\needspace{15\baselineskip}
\section*{Poor Man's Pudding}


\begin{minipage}{1.0\textwidth}
{\setlength{\multicolsep}{0pt}\setlength{\columnsep}{2em}\raggedcolumns%
\begin{multicols}{2}
\begin{itemize}
\setlength{\itemsep}{0pt}
\setlength{\parsep}{0pt}
\item 4 cups milk
\item 1/2 cup rice
\item 1/3 cup molasses
\item 1/2 teaspoon salt
\item 1/2 teaspoon cinnamon
\item 1 tablespoon butter
\end{itemize}
\end{multicols}}
\end{minipage}

\vspace{0.3em}
\noindent%
Wash rice, mix and bake same as Rice Pudding. At last stirring, add
butter.



\needspace{15\baselineskip}
\section*{Indian Pudding}


\begin{itemize}
\setlength{\itemsep}{0pt}
\setlength{\parsep}{0pt}
\item 5 cups scalded milk
\item 1/3 cup Indian meal
\item 1/2 cup molasses
\item 1 teaspoon salt
\item 1 teaspoon ginger
\end{itemize}

\vspace{-0.5em}
\noindent%
Pour milk slowly on meal, cook in double boiler twenty minutes, add
molasses, salt, and ginger; pour into buttered pudding-dish and bake two
hours in slow oven; serve with cream. If baked too rapidly it will not
whey. Ginger may be omitted.



\needspace{15\baselineskip}
\section*{Cerealine Pudding}


\begin{itemize}
\setlength{\itemsep}{0pt}
\setlength{\parsep}{0pt}
\item 4 cups scalded milk
\item 2 cups cerealine
\item 1/2 cup molasses
\item 1 1/2 teaspoons salt
\item 1 1/2 tablespoons butter
\end{itemize}

\vspace{-0.5em}
\noindent%
Pour milk on cerealine, add remaining ingredients, pour into buttered
pudding-dish, and bake one hour in slow oven. Serve with cream.



\needspace{15\baselineskip}
\section*{Newton Tapioca}


\begin{minipage}{1.0\textwidth}
{\setlength{\multicolsep}{0pt}\setlength{\columnsep}{2em}\raggedcolumns%
\begin{multicols}{2}
\begin{itemize}
\setlength{\itemsep}{0pt}
\setlength{\parsep}{0pt}
\item 5 tablespoons pearl tapioca
\item 4 cups scalded milk
\item 4 tablespoons Indian meal
\item 3/4 cup molasses
\item 3 tablespoons butter
\item 1 1/2 teaspoons salt
\item 1 cup milk
\end{itemize}
\end{multicols}}
\end{minipage}

\vspace{0.3em}
\noindent%
Soak tapioca two hours in cold water to cover. Pour scalded milk over
Indian meal, molasses, butter, and salt. Cook in double boiler until
mixture thickens. Add tapioca drained from water, turn into buttered
pudding-dish, and pour over remaining milk, but do not stir. Bake one
and one-fourth hours in a slow oven.



\needspace{15\baselineskip}
\section*{Apple Tapioca}


\begin{minipage}{1.0\textwidth}
{\setlength{\multicolsep}{0pt}\setlength{\columnsep}{2em}\raggedcolumns%
\begin{multicols}{2}
\begin{itemize}
\setlength{\itemsep}{0pt}
\setlength{\parsep}{0pt}
\item 3/4 cup pearl or Minute Tapioca
\item Cold water
\item 2 1/2 cups boiling water
\item 1/2 teaspoon salt
\item 7 sour apples
\item 1/2 cup sugar
\end{itemize}
\end{multicols}}
\end{minipage}

\vspace{0.3em}
\noindent%
Soak tapioca one hour in cold water to cover, drain, add boiling water
and salt; cook in double boiler until transparent. Core and pare apples,
arrange in buttered pudding-dish, fill cavities with sugar, pour over
tapioca, and bake in moderate oven until apples are soft. Serve with
sugar and cream or Cream Sauce I. Minute Tapioca requires no soaking.



\needspace{15\baselineskip}
\section*{Tapioca Custard Pudding}


\begin{minipage}{1.0\textwidth}
{\setlength{\multicolsep}{0pt}\setlength{\columnsep}{2em}\raggedcolumns%
\begin{multicols}{2}
\begin{itemize}
\setlength{\itemsep}{0pt}
\setlength{\parsep}{0pt}
\item 4 cups scalded milk
\item 2/3 cup pearl tapioca
\item 3 eggs
\item 1/2 cup sugar
\item 1 teaspoon salt
\item 1 tablespoon butter
\end{itemize}
\end{multicols}}
\end{minipage}

\vspace{0.3em}
\noindent%
Soak tapioca one hour in cold water to cover, drain, add to milk, and
cook in double boiler thirty minutes; beat eggs slightly, add sugar and
salt, pour on gradually hot mixture, turn into buttered pudding-dish,
add butter, bake thirty minutes in slow oven.



\needspace{15\baselineskip}
\section*{Peach Tapioca}


\begin{minipage}{1.0\textwidth}
{\setlength{\multicolsep}{0pt}\setlength{\columnsep}{2em}\raggedcolumns%
\begin{multicols}{2}
\begin{itemize}
\setlength{\itemsep}{0pt}
\setlength{\parsep}{0pt}
\item 1 can peaches
\item 1/4 cup powdered sugar
\item 1 cup tapioca
\item Boiling water
\item 1/2 cup sugar
\item 1/2 teaspoon salt
\end{itemize}
\end{multicols}}
\end{minipage}

\vspace{0.3em}
\noindent%
Drain peaches, sprinkle with powdered sugar, and let stand one hour;
soak tapioca one hour in cold water to cover; to peach syrup add enough
boiling water to make three cups; heat to boiling-point, add tapioca
drained from cold water, sugar, and salt; then cook in a double boiler
until transparent. Line a mould or pudding-dish with peaches cut in
quarters, fill with tapioca, and bake in moderate oven thirty minutes;
cool slightly, turn on a dish, and serve with Cream Sauce I.



\needspace{15\baselineskip}
\section*{Corn Pudding}


\begin{minipage}{1.0\textwidth}
{\setlength{\multicolsep}{0pt}\setlength{\columnsep}{2em}\raggedcolumns%
\begin{multicols}{2}
\begin{itemize}
\setlength{\itemsep}{0pt}
\setlength{\parsep}{0pt}
\item 2 cups popped corn, finely pounded
\item 3 cups milk
\item 3 eggs, slightly beaten
\item 1/2 cup brown sugar
\item 1 tablespoon butter
\item 3/4 teaspoon salt
\end{itemize}
\end{multicols}}
\end{minipage}

\vspace{0.3em}
\noindent%
Scald milk, pour over corn, and let stand one hour. Add remaining
ingredients, turn into a buttered dish, and bake in a slow oven until
firm. Serve with cream, or maple syrup.



\needspace{15\baselineskip}
\section*{Scalloped Apples}


\begin{minipage}{1.0\textwidth}
{\setlength{\multicolsep}{0pt}\setlength{\columnsep}{2em}\raggedcolumns%
\begin{multicols}{2}
\begin{itemize}
\setlength{\itemsep}{0pt}
\setlength{\parsep}{0pt}
\item 1 small baker's stale loaf
\item 1/4 cup butter
\item 1 quart sliced apples
\item 1/4 cup sugar
\item 1/4 teaspoon grated nutmeg
\item 1/2 lemon, grated rind and juice
\end{itemize}
\end{multicols}}
\end{minipage}

\vspace{0.3em}
\noindent%
Cut loaf in halves, remove soft part, and crumb by rubbing through a
colander; melt butter and stir in lightly with fork; cover bottom of
buttered pudding-dish with crumbs and spread over one-half the apples,
sprinkle with one-half sugar, nutmeg, lemon juice, and rind mixed
together; repeat cover with remaining crumbs, and bake forty minutes in
moderate oven. Cover at first to prevent crumbs browning too rapidly.
Serve with sugar and cream.



\needspace{15\baselineskip}
\section*{Bread Pudding}


\begin{minipage}{1.0\textwidth}
{\setlength{\multicolsep}{0pt}\setlength{\columnsep}{2em}\raggedcolumns%
\begin{multicols}{2}
\begin{itemize}
\setlength{\itemsep}{0pt}
\setlength{\parsep}{0pt}
\item 2 cups stale bread crumbs
\item 1 quart scalded milk
\item 1/3 cup sugar
\item 1/4 cup melted butter
\item 2 eggs
\item 1/2 teaspoon salt
\item 1 teaspoon vanilla or
\item 1/4 teaspoon spice
\end{itemize}
\end{multicols}}
\end{minipage}

\vspace{0.3em}
\noindent%
Soak bread crumbs in milk, set aside until cool; add sugar, butter, eggs
slightly beaten, salt, and flavoring; bake one hour in buttered
pudding-dish in slow oven; serve with Vanilla Sauce. In preparing bread
crumbs for puddings avoid using outside crusts. With a coarse grater
there need be but little waste.



\needspace{15\baselineskip}
\section*{Cracker Custard Pudding}

Make same as Bread Pudding, using two-thirds cup cracker crumbs in place
of bread crumbs; after baking, cover with meringue made of whites two
eggs, one-fourth cup powdered sugar, and one tablespoon lemon juice;
return to oven to cook meringue.



\needspace{15\baselineskip}
\section*{Bread And Butter Pudding}


\begin{minipage}{1.0\textwidth}
{\setlength{\multicolsep}{0pt}\setlength{\columnsep}{2em}\raggedcolumns%
\begin{multicols}{2}
\begin{itemize}
\setlength{\itemsep}{0pt}
\setlength{\parsep}{0pt}
\item 1 small baker's stale loaf
\item Butter
\item 3 eggs
\item 1/2 cup sugar
\item 1/4 teaspoon salt
\item 1 quart milk
\end{itemize}
\end{multicols}}
\end{minipage}

\vspace{0.3em}
\noindent%
Remove end crusts from bread, cut loaf in one-half inch slices, spread
each slice generously with butter; arrange in buttered pudding-dish,
buttered side down. Beat eggs slightly, add sugar, salt, and milk;
strain, and pour over bread; let stand thirty minutes. Bake one hour in
slow oven, covering the first half-hour of baking. The top of pudding
should be well browned. Serve with Hard or Creamy Sauce. Three-fourths
cup raisins, parboiled in boiling water to cover and seeded, may be
sprinkled between layers of bread.



\needspace{15\baselineskip}
\section*{Bread And Butter Apple Pudding}

Cover bottom of a shallow baking-dish with apple sauce. Cut stale bread
in one-third inch slices, spread with softened butter, remove crusts,
and cut in triangular-shaped pieces; then arrange closely together over
apple. Sprinkle generously with sugar, to which is added a few drops
vanilla. Bake in a moderate oven and serve with cream.



\needspace{15\baselineskip}
\section*{Chocolate Bread Pudding}


\begin{minipage}{1.0\textwidth}
{\setlength{\multicolsep}{0pt}\setlength{\columnsep}{2em}\raggedcolumns%
\begin{multicols}{2}
\begin{itemize}
\setlength{\itemsep}{0pt}
\setlength{\parsep}{0pt}
\item 2 cups stale bread crumbs
\item 4 cups scalded milk
\item 2 squares Baker's chocolate
\item 2/3 cup sugar
\item 2 eggs
\item 1/4 teaspoon salt
\item 1 teaspoon vanilla
\end{itemize}
\end{multicols}}
\end{minipage}

\vspace{0.3em}
\noindent%
Soak bread in milk thirty minutes; melt chocolate in saucepan placed
over hot water, add one-half sugar and enough milk taken from bread and
milk to make of consistency to pour; add to mixture with remaining
sugar, salt, vanilla, and eggs slightly beaten; turn into buttered
pudding-dish and bake one hour in a moderate oven. Serve with Hard or
Cream Sauce I.



\needspace{15\baselineskip}
\section*{Mock Indian Pudding}


\begin{itemize}
\setlength{\itemsep}{0pt}
\setlength{\parsep}{0pt}
\item 1/2 small loaf baker's entire wheat bread
\item 3 1/2 cups milk
\item 1/2 cup molasses
\item Butter
\end{itemize}

\vspace{-0.5em}
\noindent%
Remove crusts from bread and cut into five slices of uniform thickness.
Spread generously with butter, arrange in baking-dish, pour over three
cups of milk and molasses. Bake from two to three hours in a very slow
oven, stirring three times during the first hour of baking, then add
remaining milk. Serve with cream or vanilla ice cream.



\needspace{15\baselineskip}
\section*{Bangor Pudding}


\begin{minipage}{1.0\textwidth}
{\setlength{\multicolsep}{0pt}\setlength{\columnsep}{2em}\raggedcolumns%
\begin{multicols}{2}
\begin{itemize}
\setlength{\itemsep}{0pt}
\setlength{\parsep}{0pt}
\item 1 1/3 cups cracker crumbs
\item Boiling water
\item 2 cups milk
\item 1/3 cup molasses
\item 1 egg
\item 1 cup raisins
\end{itemize}
\end{multicols}}
\end{minipage}

\vspace{0.3em}
\noindent%
Moisten cracker crumbs with boiling water, and let stand until cool. Add
milk, molasses, egg slightly beaten, and raisins seeded and cut in
pieces. Turn into a buttered pudding mould, and steam eight hours. Let
stand in mould to cool. Serve cold with Cream Sauce II.



\needspace{15\baselineskip}
\section*{Steamed Lemon Pudding}


\begin{minipage}{1.0\textwidth}
{\setlength{\multicolsep}{0pt}\setlength{\columnsep}{2em}\raggedcolumns%
\begin{multicols}{2}
\begin{itemize}
\setlength{\itemsep}{0pt}
\setlength{\parsep}{0pt}
\item 8 small slices stale bread
\item Lemon mixture
\item 1 cup milk
\item 3 tablespoons sugar
\item 2 eggs
\item Grated rind 1 lemon
\item 1/8 teaspoon salt
\end{itemize}
\end{multicols}}
\end{minipage}

\vspace{0.3em}
\noindent%
Spread bread with lemon mixture, and arrange in buttered pudding mould.
Beat eggs slightly, add sugar, salt, and milk; strain, add lemon rind,
and pour mixture over bread. Cover, set in pan of hot water, and bake
one hour.

\textbf{Lemon Mixture.} Cook three tablespoons lemon juice, grated rind one
lemon, and one-fourth cup butter two minutes. Add one cup sugar and
three eggs slightly beaten; cook until mixture thickens, cool, and add
one tablespoon brandy.



\needspace{15\baselineskip}
\section*{Cottage Pudding}


\begin{minipage}{1.0\textwidth}
{\setlength{\multicolsep}{0pt}\setlength{\columnsep}{2em}\raggedcolumns%
\begin{multicols}{2}
\begin{itemize}
\setlength{\itemsep}{0pt}
\setlength{\parsep}{0pt}
\item 1/4 cup butter
\item 2/3 cup sugar
\item 1 egg
\item 1 cup milk
\item 2 1/4 cups flour
\item 4 teaspoons baking powder
\item 1/2 teaspoon salt
\end{itemize}
\end{multicols}}
\end{minipage}

\vspace{0.3em}
\noindent%
Cream the butter, add sugar gradually, and egg well beaten; mix and sift
flour, baking powder, and salt; add alternately with milk to first
mixture; turn into buttered cake pan; bake thirty-five minutes. Serve
with Vanilla or Hard Sauce.



\needspace{15\baselineskip}
\section*{Strawberry Cottage Pudding}


\begin{minipage}{1.0\textwidth}
{\setlength{\multicolsep}{0pt}\setlength{\columnsep}{2em}\raggedcolumns%
\begin{multicols}{2}
\begin{itemize}
\setlength{\itemsep}{0pt}
\setlength{\parsep}{0pt}
\item 1/3 cup butter
\item 1 cup sugar
\item 1 egg
\item 1/2 cup milk
\item 1 3/4 cups flour
\item 3 teaspoons baking powder
\end{itemize}
\end{multicols}}
\end{minipage}

\vspace{0.3em}
\noindent%
Mix same as Cottage Pudding, and bake twenty-five minutes in shallow
pan; cut in squares and serve with strawberries (sprinkled with sugar
and slightly mashed) and Cream Sauce I. \textit{Sliced peaches} may be used in
place of strawberries.



\needspace{15\baselineskip}
\section*{Orange Puffs}


\begin{minipage}{1.0\textwidth}
{\setlength{\multicolsep}{0pt}\setlength{\columnsep}{2em}\raggedcolumns%
\begin{multicols}{2}
\begin{itemize}
\setlength{\itemsep}{0pt}
\setlength{\parsep}{0pt}
\item 1/3 cup butter
\item 1 cup sugar
\item 2 eggs
\item 1/2 cup milk
\item 1 3/4 cups flour
\item 3 teaspoons baking powder
\end{itemize}
\end{multicols}}
\end{minipage}

\vspace{0.3em}
\noindent%
Mix same as Cottage Pudding, and bake in buttered individual tins. Serve
with Orange Sauce.



\needspace{15\baselineskip}
\section*{Chocolate Pudding}


\begin{minipage}{1.0\textwidth}
{\setlength{\multicolsep}{0pt}\setlength{\columnsep}{2em}\raggedcolumns%
\begin{multicols}{2}
\begin{itemize}
\setlength{\itemsep}{0pt}
\setlength{\parsep}{0pt}
\item 1/4 cup butter
\item 1 cup sugar
\item 4 egg yolks
\item 1/2 cup milk
\item 1 3/8 cups flour
\item 3 teaspoons baking powder
\item 2 egg whites
\item 1 1/3 squares Baker's chocolate
\item 1/8 teaspoon salt
\item 1/4 teaspoon vanilla
\end{itemize}
\end{multicols}}
\end{minipage}

\vspace{0.3em}
\noindent%
Cream the butter, and add one-half the sugar gradually. Beat yolks of
eggs until thick and lemon-colored, and add, gradually, remaining sugar.
Combine mixtures, and add milk alternately with flour mixed and sifted
with baking powder and salt; then add whites of eggs beaten until stiff,
melted chocolate, and vanilla. Bake in an angel-cake pan, remove from
pan, cool, fill the centre with whipped cream, sweetened and flavored,
and pour around

\textbf{Chocolate Sauce}. Boil one cup sugar, one-half cup water, and a few
grains cream of tartar until of the consistency of a thin syrup. Melt
one and one-half squares Baker's chocolate and pour on gradually the hot
syrup. Cool slightly, and flavor with one-fourth teaspoon vanilla.



\needspace{15\baselineskip}
\section*{Custard Soufflé}


\begin{itemize}
\setlength{\itemsep}{0pt}
\setlength{\parsep}{0pt}
\item 3 tablespoons butter
\item 1/4 cup flour
\item 1 cup scalded milk
\item 4 eggs
\item 1/4 cup sugar
\end{itemize}

\vspace{-0.5em}
\noindent%
Melt butter, add flour, and gradually hot milk; when well thickened,
pour on to yolks of eggs beaten until thick and lemon-colored, and mixed
with sugar; cool, and cut and fold in whites of eggs beaten stiff and
dry. Turn into buttered pudding-dish, and bake from thirty to
thirty-five minutes in slow oven; take from oven and serve at once,--if
not served immediately it is sure to fall; serve with Creamy or Foamy
Sauce.



\needspace{15\baselineskip}
\section*{Apricot Soufflé}

Drain and reserve syrup from one can apricots and cut fruit into
quarters, then put closely together on bottom of a buttered baking-dish.
Pour over Custard Soufflé mixture. Bake from thirty-five to forty
minutes in a slow oven. Serve with apricot syrup and whipped cream
sweetened and flavored with vanilla or vanilla ice cream. Canned peaches
may be used in place of apricots.



\needspace{15\baselineskip}
\section*{Lemon Soufflé}


\begin{itemize}
\setlength{\itemsep}{0pt}
\setlength{\parsep}{0pt}
\item 4 egg yolks
\item Grated rind and juice 1 lemon
\item 1 cup sugar
\item 4 egg whitess
\end{itemize}

\vspace{-0.5em}
\noindent%
Beat yolks until thick and lemon-colored, add sugar gradually and
continue beating, then add lemon rind and juice. Cut and fold in whites
of eggs beaten until dry; turn into buttered pudding-dish, set in pan of
hot water, and bake thirty-five to forty minutes. Serve with or without
sauce.



\needspace{15\baselineskip}
\section*{Chocolate Soufflé}


\begin{minipage}{1.0\textwidth}
{\setlength{\multicolsep}{0pt}\setlength{\columnsep}{2em}\raggedcolumns%
\begin{multicols}{2}
\begin{itemize}
\setlength{\itemsep}{0pt}
\setlength{\parsep}{0pt}
\item 2 tablespoons butter
\item 2 tablespoons flour
\item 3/4 cup milk
\item 1 1/2 squares Baker's chocolate
\item 1/3 cup sugar
\item 2 tablespoons hot water
\item 3 eggs
\item 1/2 teaspoon vanilla
\end{itemize}
\end{multicols}}
\end{minipage}

\vspace{0.3em}
\noindent%
Melt the butter, add flour, and pour on gradually, while stirring
constantly, milk; cook until boiling-point is reached. Melt chocolate in
a small saucepan placed over hot water, add sugar and water, and stir
until smooth. Combine mixtures, and add yolks of eggs well beaten; cool.
Fold in whites of eggs beaten stiff, and add vanilla. Turn into a
buttered baking-dish, and bake in a moderate oven twenty-five minutes.
Serve with Cream Sauce I.



\needspace{15\baselineskip}
\section*{Mocha Soufflé}


\begin{minipage}{1.0\textwidth}
{\setlength{\multicolsep}{0pt}\setlength{\columnsep}{2em}\raggedcolumns%
\begin{multicols}{2}
\begin{itemize}
\setlength{\itemsep}{0pt}
\setlength{\parsep}{0pt}
\item 3 tablespoons butter
\item 3 tablespoons bread flour
\item 3/4 cup boiled coffee (Mocha)
\item 1/4 cup cream
\item 1/2 cup sugar
\item 1/4 teaspoon salt
\item 4 eggs
\item 1/2 teaspoon vanilla
\end{itemize}
\end{multicols}}
\end{minipage}

\vspace{0.3em}
\noindent%
Make and bake same as Chocolate Soufflé. Serve with

\textbf{Mocha Sauce.} Mix yolks two eggs, one-fourth cup sugar, and a few
grains salt; then add gradually one-half cup Mocha coffee infusion. Cook
in double boiler until mixture thickens, stirring constantly. Strain,
cool, and fold in one cup whipped cream.



\needspace{15\baselineskip}
\section*{Fruit Soufflé}


\begin{itemize}
\setlength{\itemsep}{0pt}
\setlength{\parsep}{0pt}
\item 3/4 cup fruit pulp, peach, apricot, or quince
\item 3 egg whites
\item Sugar
\item Few grains salt
\end{itemize}

\vspace{-0.5em}
\noindent%
Rub fruit through sieve; if canned fruit is used, first drain from
syrup. Heat, and sweeten if needed; beat whites of eggs until stiff, add
gradually hot fruit pulp, and salt, and continue beating; turn into
buttered and sugared individual moulds, having them three-fourths full;
set moulds in pan of hot water and bake in slow oven until firm, which
may be determined by pressing with finger; serve with Sabyon Sauce.



\needspace{15\baselineskip}
\section*{Spanish Soufflé}


\begin{minipage}{1.0\textwidth}
{\setlength{\multicolsep}{0pt}\setlength{\columnsep}{2em}\raggedcolumns%
\begin{multicols}{2}
\begin{itemize}
\setlength{\itemsep}{0pt}
\setlength{\parsep}{0pt}
\item 1/4 cup butter
\item 1/2 cup stale bread crumbs
\item 1 cup milk
\item 2 tablespoons sugar
\item 3 eggs
\item 1/2 teaspoon vanilla
\end{itemize}
\end{multicols}}
\end{minipage}

\vspace{0.3em}
\noindent%
Melt butter, add crumbs, cook until slightly browned, stirring often;
add milk and sugar, cook twenty minutes in double boiler; remove from
fire, add unbeaten yolks of eggs, then cut and fold in whites of eggs
beaten until stiff, and flavor. Bake same as Fruit Soufflé.



\needspace{15\baselineskip}
\section*{Chestnut Soufflé}


\begin{itemize}
\setlength{\itemsep}{0pt}
\setlength{\parsep}{0pt}
\item 1/4 cup sugar
\item 2 tablespoons flour
\item 1 cup chestnut purée
\item 1/2 cup milk
\item 3 egg whites
\end{itemize}

\vspace{-0.5em}
\noindent%
Mix sugar and flour, add chestnuts and milk gradually; cook five
minutes, stirring constantly; beat whites of eggs until stiff, and cut
and fold into mixture. Bake same as Fruit Soufflé; serve with Cream
Sauce.



\needspace{15\baselineskip}
\section*{Chocolate Rice Meringue}


\begin{minipage}{1.0\textwidth}
{\setlength{\multicolsep}{0pt}\setlength{\columnsep}{2em}\raggedcolumns%
\begin{multicols}{2}
\begin{itemize}
\setlength{\itemsep}{0pt}
\setlength{\parsep}{0pt}
\item 2 cups milk
\item 1/4 cup rice
\item 1/3 teaspoon salt
\item 1 tablespoon butter
\item 1/3 cup sugar
\item 1 square melted chocolate
\item 1/2 teaspoon vanilla
\item 1/2 cup seeded raisins
\item Whites two eggs
\item 1/2 cup heavy cream
\end{itemize}
\end{multicols}}
\end{minipage}

\vspace{0.3em}
\noindent%
Scald milk, add rice and salt, and cook until rice is soft. Add butter,
sugar, chocolate, vanilla, and raisins. Cut and fold in the whites of
eggs, beaten until stiff, and cream, beaten until stiff. Pour into a
buttered baking-dish, and bake fifteen minutes. Cover with a meringue
made of the whites of three eggs, six tablespoons powdered sugar, and
one-half teaspoon vanilla; then brown in a moderate oven.



\needspace{15\baselineskip}
\section*{Steamed Apple Pudding}


\begin{minipage}{1.0\textwidth}
{\setlength{\multicolsep}{0pt}\setlength{\columnsep}{2em}\raggedcolumns%
\begin{multicols}{2}
\begin{itemize}
\setlength{\itemsep}{0pt}
\setlength{\parsep}{0pt}
\item 2 cups flour
\item 4 teaspoons baking powder
\item 1/2 teaspoon salt
\item 2 tablespoons butter
\item 3/4 cup milk
\item 4 apples cut in eighths
\end{itemize}
\end{multicols}}
\end{minipage}

\vspace{0.3em}
\noindent%
Mix and sift dry ingredients; work in butter with tips of fingers, add
milk gradually, mixing with a knife; toss on floured board, pat and roll
out, place apples on middle of dough, and sprinkle with one tablespoon
sugar mixed with one-fourth teaspoon each of salt and nutmeg; bring
dough around apples and carefully lift into buttered mould or five-pound
lard pail; or apples may be sprinkled over dough, and dough rolled like
a jelly roll; cover closely, and steam one hour and twenty minutes;
serve with Vanilla or Cold Sauce. Twice the number of apples may be
sprinkled with sugar and cooked until soft in granite kettle placed on
top of range, covered with dough, rolled size to fit in kettle, then
kettle covered tightly, and dough steamed fifteen minutes. When turned
on dish for serving, apples will be on top.



\needspace{15\baselineskip}
\section*{Steamed Blueberry Pudding}

Mix and sift dry ingredients and work in butter same as for Steamed
Apple Pudding. Add one cup each of milk, and blueberries rolled in
flour; turn into buttered mould and steam one and one-half hours. Serve
with Creamy Sauce.



\needspace{15\baselineskip}
\section*{Steamed Cranberry Pudding}


\begin{minipage}{1.0\textwidth}
{\setlength{\multicolsep}{0pt}\setlength{\columnsep}{2em}\raggedcolumns%
\begin{multicols}{2}
\begin{itemize}
\setlength{\itemsep}{0pt}
\setlength{\parsep}{0pt}
\item 1/2 cup butter
\item 1 cup sugar
\item 3 eggs
\item 3 1/2 cups flour
\item 1 1/4 tablespoons baking powder
\item 1/2 cup milk
\item 1 1/2 cups cranberries
\end{itemize}
\end{multicols}}
\end{minipage}

\vspace{0.3em}
\noindent%
Cream the butter, add sugar gradually, and eggs well beaten. Mix and
sift flour and baking powder and add alternately with milk to first
mixture, stir in berries, turn into buttered mould, cover, and steam
three hours. Serve with thin cream, sweetened and flavored with nutmeg.



\needspace{15\baselineskip}
\section*{Ginger Pudding}


\begin{minipage}{1.0\textwidth}
{\setlength{\multicolsep}{0pt}\setlength{\columnsep}{2em}\raggedcolumns%
\begin{multicols}{2}
\begin{itemize}
\setlength{\itemsep}{0pt}
\setlength{\parsep}{0pt}
\item 1/3 cup butter
\item 1/2 cup sugar
\item 1 egg
\item 2 1/4 cups flour
\item 3 1/2 teaspoons baking powder
\item 1/4 teaspoon salt
\item 2 teaspoons ginger
\item 1 cup milk
\end{itemize}
\end{multicols}}
\end{minipage}

\vspace{0.3em}
\noindent%
Cream the butter, add sugar gradually, and egg well beaten; mix and sift
dry ingredients; add alternately with milk to first mixture. Turn into
buttered mould, cover, and steam two hours; serve with Vanilla Sauce.



\needspace{15\baselineskip}
\section*{Harvard Pudding}


\begin{minipage}{1.0\textwidth}
{\setlength{\multicolsep}{0pt}\setlength{\columnsep}{2em}\raggedcolumns%
\begin{multicols}{2}
\begin{itemize}
\setlength{\itemsep}{0pt}
\setlength{\parsep}{0pt}
\item 1/3 cup butter
\item 1/2 cup sugar
\item 2 1/2 cups flour
\item 3 1/2 teaspoons baking powder
\item 1/4 teaspoon salt
\item 1 egg
\item 1 cup milk
\end{itemize}
\end{multicols}}
\end{minipage}

\vspace{0.3em}
\noindent%
Mix and sift dry ingredients and work in butter with tips of fingers;
beat egg, add milk, and combine mixtures; turn into buttered mould,
cover, and steam two hours; serve with warm Apple Sauce and Hard Sauce.

\textbf{Apple Sauce.} Pick over and wash dried apples, soak over night in cold
water to cover; cook until soft; sweeten, and flavor with lemon juice.



\needspace{15\baselineskip}
\section*{Steamed Chocolate Pudding}


\begin{minipage}{1.0\textwidth}
{\setlength{\multicolsep}{0pt}\setlength{\columnsep}{2em}\raggedcolumns%
\begin{multicols}{2}
\begin{itemize}
\setlength{\itemsep}{0pt}
\setlength{\parsep}{0pt}
\item 3 tablespoons butter
\item 2/3 cup sugar
\item 1 egg
\item 1 cup milk
\item 2 1/4 cups flour
\item 4 1/2 teaspoons baking powder
\item 2 1/2 squares Baker's chocolate
\item 1/4 teaspoon salt
\end{itemize}
\end{multicols}}
\end{minipage}

\vspace{0.3em}
\noindent%
Cream the butter, add sugar gradually, and egg well beaten. Mix and sift
flour with baking powder and salt, and add alternately with milk to
first mixture, then add chocolate, melted. Turn into a buttered mould.
Cover, and steam two hours. Serve with



\needspace{15\baselineskip}
\section*{Cream Sauce}


\begin{itemize}
\setlength{\itemsep}{0pt}
\setlength{\parsep}{0pt}
\item 1/4 cup butter
\item 1 cup powdered sugar
\item 1/2 teaspoon vanilla
\item 1/4 cup heavy cream
\end{itemize}

\vspace{-0.5em}
\noindent%
Cream the butter, add sugar gradually, vanilla, and cream beaten until
stiff.



\needspace{15\baselineskip}
\section*{Swiss Pudding}


\begin{minipage}{1.0\textwidth}
{\setlength{\multicolsep}{0pt}\setlength{\columnsep}{2em}\raggedcolumns%
\begin{multicols}{2}
\begin{itemize}
\setlength{\itemsep}{0pt}
\setlength{\parsep}{0pt}
\item 1/2 cup butter
\item 7/8 cup flour
\item 2 cups milk
\item Grated rind one lemon
\item 5 eggs
\item 1/3 cup powdered sugar
\end{itemize}
\end{multicols}}
\end{minipage}

\vspace{0.3em}
\noindent%
Cream the butter, add flour gradually; scald milk with lemon rind, add
to first mixture, and cook five minutes in double boiler. Beat yolks of
eggs until thick and lemon-colored, add sugar gradually, then add to
cooked mixture; cool, and cut and fold in whites of eggs beaten stiff.
Turn into buttered mould, cover, and steam one and one-fourth hours;
while steaming, be sure water surrounds mould to half its depth, and
never reaches a lower temperature than the boiling-point.



\needspace{15\baselineskip}
\section*{Snowballs}


\begin{minipage}{1.0\textwidth}
{\setlength{\multicolsep}{0pt}\setlength{\columnsep}{2em}\raggedcolumns%
\begin{multicols}{2}
\begin{itemize}
\setlength{\itemsep}{0pt}
\setlength{\parsep}{0pt}
\item 1/2 cup butter
\item 1 cup sugar
\item 1/2 cup milk
\item 2 1/4 cups flour
\item 3 1/2 teaspoons baking powder
\item 4 egg whitess
\end{itemize}
\end{multicols}}
\end{minipage}

\vspace{0.3em}
\noindent%
Cream the butter, add sugar gradually, milk, and flour mixed and sifted
with baking powder; then add the whites of eggs beaten stiff. Steam
thirty-five minutes in buttered cups; serve with preserved fruit, quince
marmalade, or strawberry sauce.



\needspace{15\baselineskip}
\section*{Graham Pudding}


\begin{minipage}{1.0\textwidth}
{\setlength{\multicolsep}{0pt}\setlength{\columnsep}{2em}\raggedcolumns%
\begin{multicols}{2}
\begin{itemize}
\setlength{\itemsep}{0pt}
\setlength{\parsep}{0pt}
\item 1/4 cup butter
\item 1/2 cup molasses
\item 1/2 cup milk
\item 1 egg
\item 1 1/2 cups Graham flour
\item 1/2 teaspoon soda
\item 1 teaspoon salt
\item 1 cup raisins, seeded and cut in pieces
\end{itemize}
\end{multicols}}
\end{minipage}

\vspace{0.3em}
\noindent%
Melt butter, add molasses, milk, egg well beaten, dry ingredients mixed
and sifted, and raisins; turn into buttered mould, cover, and steam two
and one-half hours. Serve with Wine Sauce. Dates or figs cut in small
pieces may be used in place of raisins.



\needspace{15\baselineskip}
\section*{St. James Pudding}


\begin{minipage}{1.0\textwidth}
{\setlength{\multicolsep}{0pt}\setlength{\columnsep}{2em}\raggedcolumns%
\begin{multicols}{2}
\begin{itemize}
\setlength{\itemsep}{0pt}
\setlength{\parsep}{0pt}
\item 3 tablespoons butter
\item 1/2 cup molasses
\item 1/2 cup milk
\item 1 2/3 cups flour
\item 1/2 teaspoon soda
\item 1/4 teaspoon salt
\item 1/4 teaspoon clove
\item 1/4 teaspoon allspice
\item 1/4 teaspoon nutmeg
\item 1/2 lb. dates, stoned and cut in pieces
\end{itemize}
\end{multicols}}
\end{minipage}

\vspace{0.3em}
\noindent%
Mix and steam same as Graham Pudding. Serve with Wine Sauce. A simple,
delicious pudding without egg. Puddings may be steamed in buttered
one-pound baking-powder boxes, providing they do not leak, and are
attractive in shape and easy to serve.



\needspace{15\baselineskip}
\section*{Suet Pudding}


\begin{minipage}{1.0\textwidth}
{\setlength{\multicolsep}{0pt}\setlength{\columnsep}{2em}\raggedcolumns%
\begin{multicols}{2}
\begin{itemize}
\setlength{\itemsep}{0pt}
\setlength{\parsep}{0pt}
\item 1 cup finely chopped suet
\item 1 cup molasses
\item 1 cup milk
\item 3 cups flour
\item 1 teaspoon soda
\item 1 1/2 teaspoons salt
\item 1/2 teaspoon ginger
\item 1/2 teaspoon clove
\item 1/2 teaspoon nutmeg
\item 1 teaspoon cinnamon
\end{itemize}
\end{multicols}}
\end{minipage}

\vspace{0.3em}
\noindent%
Mix and sift dry ingredients. Add molasses and milk to suet; combine
mixtures. Turn into buttered mould, cover, and steam three hours; serve
with Sterling Sauce. Raisins and currants may be added.



\needspace{15\baselineskip}
\section*{Thanksgiving Pudding I}


\begin{minipage}{1.0\textwidth}
{\setlength{\multicolsep}{0pt}\setlength{\columnsep}{2em}\raggedcolumns%
\begin{multicols}{2}
\begin{itemize}
\setlength{\itemsep}{0pt}
\setlength{\parsep}{0pt}
\item 4 cups scalded milk
\item 1 1/4 cups rolled crackers
\item 1 cup sugar
\item 4 eggs
\item 1/3 cup melted butter
\item 1/2 grated nutmeg
\item 1 teaspoon salt
\item 1 1/2 cups raisins
\end{itemize}
\end{multicols}}
\end{minipage}

\vspace{0.3em}
\noindent%
Pour milk over crackers and let stand until cool; add sugar, eggs
slightly beaten, nutmeg, salt, and butter; parboil raisins until soft,
by cooking in boiling water to cover; seed, and add to mixture; turn
into buttered pudding-dish and bake slowly two and one-half hours,
stirring after first half-hour to prevent raisins from settling; serve
with Brandy Sauce.



\needspace{15\baselineskip}
\section*{Thanksgiving Pudding II}


\begin{minipage}{1.0\textwidth}
{\setlength{\multicolsep}{0pt}\setlength{\columnsep}{2em}\raggedcolumns%
\begin{multicols}{2}
\begin{itemize}
\setlength{\itemsep}{0pt}
\setlength{\parsep}{0pt}
\item 1/3 cup suet
\item 1/2 lb. figs, finely chopped
\item 2 1/2 cups stale bread crumbs
\item 3/4 cup milk
\item 1 cup brown sugar
\item 1 teaspoon salt
\item 3/4 teaspoon cinnamon
\item 1/2 teaspoon grated nutmeg
\item 1/2 cup English walnut meats
\item 1/2 cup raisins, seeded and cut in pieces
\item 2 tablespoons flour
\item 4 eggs
\item 2 teaspoons baking powder
\end{itemize}
\end{multicols}}
\end{minipage}

\vspace{0.3em}
\noindent%
Chop suet and work with the hand until creamy, then add figs. Soak bread
crumbs in milk, add eggs well beaten, sugar, salt, and spices. Combine
mixtures, add nut meats and raisins dredged with flour. Sprinkle over
baking powder and beat thoroughly. Turn into a buttered mould, steam
three hours, and serve with Yellow Sauce II (see p. 407), flavored with
brandy.



\needspace{15\baselineskip}
\section*{Hunters' Pudding}


\begin{minipage}{1.0\textwidth}
{\setlength{\multicolsep}{0pt}\setlength{\columnsep}{2em}\raggedcolumns%
\begin{multicols}{2}
\begin{itemize}
\setlength{\itemsep}{0pt}
\setlength{\parsep}{0pt}
\item 1 cup finely chopped suet
\item 1 cup molasses
\item 1 cup milk
\item 3 cups flour
\item 1 teaspoon soda
\item 1 1/2 teaspoons salt
\item 1/2 teaspoon clove
\item 1/2 teaspoon mace
\item 1/2 teaspoon allspice
\item 1 teaspoon cinnamon
\item 1 1/2 cups raisins
\item 2 tablespoons flour
\end{itemize}
\end{multicols}}
\end{minipage}

\vspace{0.3em}
\noindent%
Mix same as Suet Pudding. Stone, cut, and flour raisins, and add to
mixture. Then steam.



\needspace{15\baselineskip}
\section*{French Fruit Pudding}


\begin{minipage}{1.0\textwidth}
{\setlength{\multicolsep}{0pt}\setlength{\columnsep}{2em}\raggedcolumns%
\begin{multicols}{2}
\begin{itemize}
\setlength{\itemsep}{0pt}
\setlength{\parsep}{0pt}
\item 1 cup finely chopped suet
\item 1 cup molasses
\item 1 cup sour milk
\item 1 1/2 teaspoons soda
\item 1 teaspoon cinnamon
\item 1/2 teaspoon clove
\item 1/2 teaspoon salt
\item 1 1/4 cups raisins, seeded
\item and chopped
\item 3/4 cup currants
\item 2 3/4 cups flour
\end{itemize}
\end{multicols}}
\end{minipage}

\vspace{0.3em}
\noindent%
                           \textit{Mrs. Carrie M. Dearborn}

Add molasses and sour milk to suet; add two cups flour mixed and sifted
with soda, salt, and spices; add fruit mixed with remaining flour. Turn
into buttered mould, cover, and steam four hours. Serve with Sterling
Sauce.



\needspace{15\baselineskip}
\section*{Fig Pudding I}


\begin{minipage}{1.0\textwidth}
{\setlength{\multicolsep}{0pt}\setlength{\columnsep}{2em}\raggedcolumns%
\begin{multicols}{2}
\begin{itemize}
\setlength{\itemsep}{0pt}
\setlength{\parsep}{0pt}
\item 1/3 lb. beef suet
\item 1/2 lb. figs, finely chopped
\item 2 1/3 cups stale bread crumbs
\item 1/2 cup milk
\item 2 eggs
\item 1 cup sugar
\item 3/4 teaspoon salt
\end{itemize}
\end{multicols}}
\end{minipage}

\vspace{0.3em}
\noindent%
Chop suet, and work with the hands until creamy, then add figs. Soak
bread crumbs in milk, add eggs well beaten, sugar, and salt. Combine
mixtures, turn into a buttered mould, steam three hours. Serve with
Yellow Sauce I or II.



\needspace{15\baselineskip}
\section*{Fig Pudding II}


\begin{minipage}{1.0\textwidth}
{\setlength{\multicolsep}{0pt}\setlength{\columnsep}{2em}\raggedcolumns%
\begin{multicols}{2}
\begin{itemize}
\setlength{\itemsep}{0pt}
\setlength{\parsep}{0pt}
\item 1/4 lb. suet
\item 1/2 lb. figs (finely chopped)
\item 1 large sour apple (cored, pared, and chopped)
\item 1/4 lb. brown sugar
\item 1/4 lb. bread crumbs
\item 1/4 cup milk
\item 2 eggs
\item 3 oz. flour
\end{itemize}
\end{multicols}}
\end{minipage}

\vspace{0.3em}
\noindent%
Cream the suet, and add figs, apple, and sugar. Pour milk over bread
crumbs, and add yolks of eggs, well beaten; combine mixtures, add flour
and whites of eggs beaten until stiff. Turn into buttered pudding mould,
and steam four hours. Serve with Lemon Sauce III.



\needspace{15\baselineskip}
\section*{English Plum Pudding I}


\begin{minipage}{1.0\textwidth}
{\setlength{\multicolsep}{0pt}\setlength{\columnsep}{2em}\raggedcolumns%
\begin{multicols}{2}
\begin{itemize}
\setlength{\itemsep}{0pt}
\setlength{\parsep}{0pt}
\item 1/2 lb. stale bread crumbs
\item 1 cup scalded milk
\item 1/4 lb. sugar
\item 4 eggs
\item 1/2 lb. raisins, seeded, cut
\item in pieces, and floured
\item 1/4 lb. currants
\item 1/4 lb. finely chopped figs
\item 2 oz. finely cut citron
\item 1/2 lb. suet
\item 1/4 cup wine and brandy mixed
\item 1/2 teaspoon grated nutmeg
\item 3/4 teaspoon cinnamon
\item 1/3 teaspoon clove
\item 1/3 teaspoon mace
\item 1 1/2 teaspoons salt
\end{itemize}
\end{multicols}}
\end{minipage}

\vspace{0.3em}
\noindent%
Soak bread crumbs in milk, let stand until cool, add sugar, beaten yolks
of eggs, raisins, currants, figs, and citron; chop suet, and cream by
using the hand; combine mixtures, then add wine, brandy, nutmeg,
cinnamon, clove, mace, and whites of eggs beaten stiff. Turn into
buttered mould, cover, and steam six hours.



\needspace{15\baselineskip}
\section*{English Plum Pudding II}

                6 ozs. flour

\begin{minipage}{1.0\textwidth}
{\setlength{\multicolsep}{0pt}\setlength{\columnsep}{2em}\raggedcolumns%
\begin{multicols}{2}
\begin{itemize}
\setlength{\itemsep}{0pt}
\setlength{\parsep}{0pt}
\item 6 ozs. stale bread crumbs
\item 3/4 lb. raisins, seeded and cut in pieces
\item 3/4 lb. currants
\item 3/4 lb. suet, finely chopped
\item 10 ozs. sugar
\item 1 cup molasses
\item 3 ozs. candied orange peel, finely cut
\item 1 teaspoon grated nutmeg
\item 1 teaspoon mace
\item 6 eggs, well beaten
\item 2 teaspoons salt
\end{itemize}
\end{multicols}}
\end{minipage}

\vspace{0.3em}
\noindent%
Mix ingredients in order given, turn into a thickly floured square of
unbleached cotton cloth. Tie securely, leaving some space to allow the
pudding to swell, and plunge into a kettle of boiling water. Cook five
hours, allowing pudding to be immersed in water during the entire
cooking. Serve with Hard and Liquid Sauce.

\textbf{Hard Sauce.} Cream one-third cup butter; add gradually one cup brown
sugar and two tablespoons brandy, drop by drop. Force through a pastry
bag with rose tube, and garnish with green leaves and candied cherries.

\textbf{Liquid Sauce.} Mix one-half cup sugar, one-half tablespoon corn-starch,
and a few grains salt. Add gradually, while stirring constantly, one cup
boiling water, and boil five minutes. Remove from fire, add one
tablespoon lemon juice and two tablespoons brandy; then color with fruit
red.





\chapter{Pudding Sauces}




\needspace{15\baselineskip}
\section*{Lemon Sauce I}


\begin{itemize}
\setlength{\itemsep}{0pt}
\setlength{\parsep}{0pt}
\item 3/4 cups sugar
\item 1/4 cup water
\item 2 teaspoons butter
\item 1 tablespoon lemon juice
\end{itemize}

\vspace{-0.5em}
\noindent%
Make a syrup by boiling sugar and water eight minutes; remove from fire;
add butter and lemon juice.



\needspace{15\baselineskip}
\section*{Lemon Sauce II}


\begin{minipage}{1.0\textwidth}
{\setlength{\multicolsep}{0pt}\setlength{\columnsep}{2em}\raggedcolumns%
\begin{multicols}{2}
\begin{itemize}
\setlength{\itemsep}{0pt}
\setlength{\parsep}{0pt}
\item 1/2 cup sugar
\item 1 cup boiling water
\item 1 tablespoon corn-starch or
\item 1 1/2 tablespoons flour
\item 2 tablespoons butter
\item 1 1/2 tablespoons lemon juice
\item Few gratings nutmeg
\item Few grains salt
\end{itemize}
\end{multicols}}
\end{minipage}

\vspace{0.3em}
\noindent%
Mix sugar and corn-starch, add water gradually, stirring constantly;
boil five minutes, remove from fire, add butter, lemon juice, and
nutmeg.



\needspace{15\baselineskip}
\section*{Lemon Sauce III}


\begin{minipage}{1.0\textwidth}
{\setlength{\multicolsep}{0pt}\setlength{\columnsep}{2em}\raggedcolumns%
\begin{multicols}{2}
\begin{itemize}
\setlength{\itemsep}{0pt}
\setlength{\parsep}{0pt}
\item 1/3 cup butter
\item 1 cup sugar
\item 3 egg yolks
\item 1/3 cup boiling water
\item 3 tablespoons lemon juice
\item Few gratings lemon rind
\end{itemize}
\end{multicols}}
\end{minipage}

\vspace{0.3em}
\noindent%
Cream butter, add sugar gradually, and yolks of eggs, slightly beaten;
then add water, and cook over boiling water until mixture thickens.
Remove from range, add lemon juice and rind. Serve with Apple Pudding or
Pop-overs.



\needspace{15\baselineskip}
\section*{Vanilla Sauce}

Make same as Lemon Sauce II, using one teaspoon vanilla in place of
lemon juice and nutmeg.



\needspace{15\baselineskip}
\section*{Molasses Sauce}


\begin{itemize}
\setlength{\itemsep}{0pt}
\setlength{\parsep}{0pt}
\item 1 cup molasses
\item 1 1/2 tablespoons butter
\item 2 tablespoons lemon juice or
\item 1 tablespoon vinegar
\end{itemize}

\vspace{-0.5em}
\noindent%
Boil molasses and butter five minutes; remove from fire and add lemon
juice.



\needspace{15\baselineskip}
\section*{Cream Sauce I}


\begin{itemize}
\setlength{\itemsep}{0pt}
\setlength{\parsep}{0pt}
\item 3/4 cup thick cream
\item 1/4 cup milk
\item 1/3 cup powdered sugar
\item 1/2 teaspoon vanilla
\end{itemize}

\vspace{-0.5em}
\noindent%
Mix cream and milk, beat until stiff, using egg-beater; add sugar and
vanilla.



\needspace{15\baselineskip}
\section*{Cream Sauce II}


\begin{itemize}
\setlength{\itemsep}{0pt}
\setlength{\parsep}{0pt}
\item 1 egg
\item 1 cup powdered sugar
\item 1/2 cup thick cream
\item 1/4 cup milk
\item 1/2 teaspoon vanilla
\end{itemize}

\vspace{-0.5em}
\noindent%
Beat white of egg until stiff; add yolk of egg well beaten, and sugar
gradually; dilute cream with milk, beat until stiff, combine mixtures,
and flavor.



\needspace{15\baselineskip}
\section*{Yellow Sauce I}


\begin{itemize}
\setlength{\itemsep}{0pt}
\setlength{\parsep}{0pt}
\item 2 eggs
\item 1 cup sugar
\item 1 teaspoon vanilla or
\item 1/2 teaspoon vanilla and
\item 1 teaspoon brandy
\end{itemize}

\vspace{-0.5em}
\noindent%
Beat eggs until very light, add sugar gradually and continue beating;
then flavor.



\needspace{15\baselineskip}
\section*{Yellow Sauce II}


\begin{itemize}
\setlength{\itemsep}{0pt}
\setlength{\parsep}{0pt}
\item 2 eggs
\item 1 cup powdered sugar
\item 3 tablespoons wine
\end{itemize}

\vspace{-0.5em}
\noindent%
Beat yolks of eggs until thick, add one-half the sugar gradually; beat
whites of eggs until stiff, add gradually remaining sugar; combine
mixtures, and add wine.



\needspace{15\baselineskip}
\section*{Orange Sauce}


\begin{itemize}
\setlength{\itemsep}{0pt}
\setlength{\parsep}{0pt}
\item 3 egg whites
\item 1 cup powdered sugar
\item Juice and rind 2 oranges
\item Juice 1 lemon
\end{itemize}

\vspace{-0.5em}
\noindent%
Beat whites until stiff, add sugar gradually, and continue beating; add
rind and fruit juices.



\needspace{15\baselineskip}
\section*{Strawberry Sauce}


\begin{itemize}
\setlength{\itemsep}{0pt}
\setlength{\parsep}{0pt}
\item 1/3 cup butter
\item 2/3 cup strawberries
\item 1 cup powdered sugar
\item 1 egg white
\end{itemize}

\vspace{-0.5em}
\noindent%
Cream the butter, add sugar gradually, egg beaten until stiff, and
strawberries. Beat until fruit is mashed.



\needspace{15\baselineskip}
\section*{Creamy Sauce I}


\begin{itemize}
\setlength{\itemsep}{0pt}
\setlength{\parsep}{0pt}
\item 1/4 cup butter
\item 1/2 cup powdered sugar
\item 2 tablespoons milk
\item 2 tablespoons wine
\end{itemize}

\vspace{-0.5em}
\noindent%
Cream the butter, add sugar gradually, and milk and wine drop by drop.
If liquids are added too fast the sauce will have a curdled appearance.



\needspace{15\baselineskip}
\section*{Creamy Sauce II}

Use same proportions as given in recipe I. If not careful in adding
liquids, it will curdle; but this will make no difference, as the sauce
is to be warmed over hot water. By careful watching and constant
stirring, the ingredients will be perfectly blended; it should be creamy
in consistency.



\needspace{15\baselineskip}
\section*{Foamy Sauce I}


\begin{itemize}
\setlength{\itemsep}{0pt}
\setlength{\parsep}{0pt}
\item 1/2 cup butter
\item 1 cup powdered sugar
\item 1 egg
\item 2 tablespoons wine
\end{itemize}

\vspace{-0.5em}
\noindent%
Cream the butter, add gradually sugar, egg well beaten, and wine; beat
while heating over hot water.



\needspace{15\baselineskip}
\section*{Foamy Sauce II}


\begin{itemize}
\setlength{\itemsep}{0pt}
\setlength{\parsep}{0pt}
\item 2 egg whites
\item 1 cup powdered sugar
\item 1/4 cup hot milk
\item 1 teaspoon vanilla
\end{itemize}

\vspace{-0.5em}
\noindent%
Beat eggs until stiff, add sugar gradually, and continue beating; add
milk and vanilla.



\needspace{15\baselineskip}
\section*{Chocolate Sauce}


\begin{minipage}{1.0\textwidth}
{\setlength{\multicolsep}{0pt}\setlength{\columnsep}{2em}\raggedcolumns%
\begin{multicols}{2}
\begin{itemize}
\setlength{\itemsep}{0pt}
\setlength{\parsep}{0pt}
\item 2 cups milk
\item 1 1/2 tablespoons corn-starch
\item 2 squares Baker's chocolate
\item 4 tablespoons powdered sugar
\item 2 tablespoons hot water
\item 2 eggs
\item 2/3 cup powdered sugar
\item 1 teaspoon vanilla
\end{itemize}
\end{multicols}}
\end{minipage}

\vspace{0.3em}
\noindent%
Scald one and three-fourths cups milk, add corn-starch diluted with
remaining milk, and cook eight minutes in double boiler; melt chocolate
over hot water, add four tablespoons sugar and hot water, stir until
smooth, then add to cooked mixture; beat whites of eggs until stiff, add
gradually powdered sugar and continue beating, then add unbeaten yolks,
and stir into cooked mixture; cook one minute, add vanilla, and cool
before serving.



\needspace{15\baselineskip}
\section*{Sabyon Sauce}

                     Grated rind and juice 1/2 lemon

\begin{itemize}
\setlength{\itemsep}{0pt}
\setlength{\parsep}{0pt}
\item 1/2 cup white wine or
\item 1/4 cup Sherry
\item 1/3 cup sugar
\item 2 eggs
\end{itemize}

\vspace{-0.5em}
\noindent%
Mix lemon, wine, sugar, and yolks of eggs; stir vigorously over fire
until it thickens, using a wire whisk; pour on to whites of eggs beaten
stiff.



\needspace{15\baselineskip}
\section*{Hard Sauce}


\begin{itemize}
\setlength{\itemsep}{0pt}
\setlength{\parsep}{0pt}
\item 1/3 cup butter
\item 1 cup powdered sugar
\item 1/3 teaspoon lemon extract
\item 2/3 teaspoon vanilla
\end{itemize}

\vspace{-0.5em}
\noindent%
Cream the butter, add sugar gradually, and flavoring.



\needspace{15\baselineskip}
\section*{Sterling Sauce}


\begin{itemize}
\setlength{\itemsep}{0pt}
\setlength{\parsep}{0pt}
\item 1/2 cup butter
\item 1 cup brown sugar
\item 1 teaspoon vanilla or
\item 2 tablespoons wine
\item 4 tablespoons cream or milk
\end{itemize}

\vspace{-0.5em}
\noindent%
Cream the butter, add sugar gradually, and milk and flavoring drop by
drop to prevent separation.



\needspace{15\baselineskip}
\section*{Wine Sauce}


\begin{itemize}
\setlength{\itemsep}{0pt}
\setlength{\parsep}{0pt}
\item 1/2 cup butter
\item 1 cup powdered sugar
\item 3 tablespoons Sherry or Madeira wine
\item Slight grating nutmeg
\end{itemize}

\vspace{-0.5em}
\noindent%
Cream the butter, add sugar gradually, and wine slowly; pile on glass
dish, and sprinkle with grated nutmeg.



\needspace{15\baselineskip}
\section*{Brandy Sauce}


\begin{minipage}{1.0\textwidth}
{\setlength{\multicolsep}{0pt}\setlength{\columnsep}{2em}\raggedcolumns%
\begin{multicols}{2}
\begin{itemize}
\setlength{\itemsep}{0pt}
\setlength{\parsep}{0pt}
\item 1/4 cup butter
\item 1 cup powdered sugar
\item 2 tablespoon brandy
\item 4 egg yolks
\item 2 egg whites
\item 1/2 cup milk or cream
\end{itemize}
\end{multicols}}
\end{minipage}

\vspace{0.3em}
\noindent%
Cream the butter, add sugar gradually, then brandy very slowly, well
beaten yolks, and milk or cream. Cook over hot water until it thickens
as a custard, pour on to beaten whites.



\needspace{15\baselineskip}
\section*{Caramel Brandy Sauce}

Make same as Brandy Sauce, substituting brown sugar in place of powdered
sugar.



\needspace{15\baselineskip}
\section*{Apricot Sauce}


\begin{itemize}
\setlength{\itemsep}{0pt}
\setlength{\parsep}{0pt}
\item 3/4 cup apricot pulp
\item 3/4 cup heavy cream
\item Sugar
\end{itemize}

\vspace{-0.5em}
\noindent%
Drain canned apricots from their syrup, and rub through a sieve. Beat
cream until stiff, add to apricot pulp, and sweeten to taste. Serve with
German toast.





\chapter{Cold Desserts}




\needspace{15\baselineskip}
\section*{Irish Moss Blanc-Mange}


\begin{itemize}
\setlength{\itemsep}{0pt}
\setlength{\parsep}{0pt}
\item 1/3 cup Irish moss
\item 4 cups milk
\item 1/4 teaspoon salt
\item 1 1/2 teaspoons vanilla
\end{itemize}

\vspace{-0.5em}
\noindent%
Soak moss fifteen minutes in cold water to cover, drain, pick over, and
add to milk; cook in double boiler thirty minutes; the milk will seem
but little thicker than when put on to cook, but if cooked longer
blanc-mange will be too stiff. Add salt, strain, flavor, re-strain, and
fill individual moulds previously dipped in cold water; chill, turn on
glass dish, surround with thin slices of banana, and place a slice on
each mould. Serve with sugar and cream.



\needspace{15\baselineskip}
\section*{Chocolate Blanc-Mange}

Irish Moss Blanc-Mange flavored with chocolate. Melt one and one-half
squares Baker's chocolate, add one-fourth cup sugar and one-third cup
boiling water, stir until perfectly smooth, adding to milk just before
taking from fire. Serve with sugar and cream.



\needspace{15\baselineskip}
\section*{Rebecca Pudding}


\begin{minipage}{1.0\textwidth}
{\setlength{\multicolsep}{0pt}\setlength{\columnsep}{2em}\raggedcolumns%
\begin{multicols}{2}
\begin{itemize}
\setlength{\itemsep}{0pt}
\setlength{\parsep}{0pt}
\item 4 cups scalded milk
\item 1/2 cup corn-starch
\item 1/4 cup sugar
\item 1/4 teaspoon salt
\item 1/2 cup cold milk
\item 1 teaspoon vanilla
\item 3 egg whites
\end{itemize}
\end{multicols}}
\end{minipage}

\vspace{0.3em}
\noindent%
Mix corn-starch, sugar, and salt, dilute with cold milk, add to scalded
milk, stirring constantly until mixture thickens, afterwards
occasionally; cook fifteen minutes. Add flavoring and whites of eggs
beaten stiff, mix thoroughly, mould, chill, and serve with Yellow Sauce
I or II.



\needspace{15\baselineskip}
\section*{Moulded Snow}

Make same as Rebecca Pudding, and serve with Chocolate Ice.



\needspace{15\baselineskip}
\section*{Chocolate Cream}


\begin{minipage}{1.0\textwidth}
{\setlength{\multicolsep}{0pt}\setlength{\columnsep}{2em}\raggedcolumns%
\begin{multicols}{2}
\begin{itemize}
\setlength{\itemsep}{0pt}
\setlength{\parsep}{0pt}
\item 2 cups scalded milk
\item 5 tablespoons corn-starch
\item 1/2 cup sugar
\item 1/4 teaspoon salt
\item 1/3 cup cold milk
\item 1 1/2 squares Baker's chocolate
\item 3 tablespoons hot water
\item 3 egg whites
\item 1 teaspoon vanilla
\end{itemize}
\end{multicols}}
\end{minipage}

\vspace{0.3em}
\noindent%
Mix corn-starch, sugar, and salt, dilute with cold milk, add to scalded
milk, and cook over hot water ten minutes, stirring constantly until
thickened; melt chocolate, add hot water, stir until smooth, and add to
cooked mixture; add whites of eggs beaten stiff, and vanilla. Mould,
chill, and serve with cream.



\needspace{15\baselineskip}
\section*{Pineapple Pudding}


\begin{minipage}{1.0\textwidth}
{\setlength{\multicolsep}{0pt}\setlength{\columnsep}{2em}\raggedcolumns%
\begin{multicols}{2}
\begin{itemize}
\setlength{\itemsep}{0pt}
\setlength{\parsep}{0pt}
\item 2 3/4 cups scalded milk
\item 1/4 cup cold milk
\item 1/3 cup corn-starch
\item 1/4 cup sugar
\item 1/4 teaspoon salt
\item 1/2 can grated pineapple
\item 3 egg whites
\end{itemize}
\end{multicols}}
\end{minipage}

\vspace{0.3em}
\noindent%
Follow directions for Rebecca Pudding, and add pineapple just before
moulding. Fill individual moulds, previously dipped in cold water. Serve
with cream.



\needspace{15\baselineskip}
\section*{Caramel Junket}


\begin{minipage}{1.0\textwidth}
{\setlength{\multicolsep}{0pt}\setlength{\columnsep}{2em}\raggedcolumns%
\begin{multicols}{2}
\begin{itemize}
\setlength{\itemsep}{0pt}
\setlength{\parsep}{0pt}
\item 2 cups milk
\item 1/3 cup sugar
\item 1/3 cup boiling water
\item 1 junket tablet
\item Few grains salt
\item 1 teaspoon vanilla
\item Whipped cream, sweetened and flavored
\item Chopped nut meats
\end{itemize}
\end{multicols}}
\end{minipage}

\vspace{0.3em}
\noindent%
Heat milk until lukewarm. Caramelize sugar, add boiling water, and cook
until syrup is reduced to one-third cup. Cool, and add milk slowly to
syrup. Reduce junket tablet to powder, using a small mallet, add to
mixture, with salt and vanilla. Turn into a glass dish, let stand in
warm place until set, then chill. Cover with whipped cream and sprinkle
with chopped nuts.



\needspace{15\baselineskip}
\section*{Boiled Custard}


\begin{itemize}
\setlength{\itemsep}{0pt}
\setlength{\parsep}{0pt}
\item 2 cups scalded milk
\item 3 egg yolks
\item 1/4 cup sugar
\item 1/8 teaspoon salt
\item 1/2 teaspoon vanilla
\end{itemize}

\vspace{-0.5em}
\noindent%
Beat eggs slightly, add sugar and salt; stir constantly while adding
gradually hot milk. Cook in double boiler, continue stirring until
mixture thickens and a coating is formed on the spoon, strain
immediately; chill and flavor. If cooked too long the custard will
curdle; should this happen, by using a Dover egg-beater it may be
restored to a smooth consistency, but custard will not be as thick. Eggs
should be beaten slightly for custard, that it may be of smooth, thick
consistency. To prevent scum from forming, cover with a perforated tin.
When eggs are scarce, use yolks two eggs and one-half tablespoon
corn-starch.



\needspace{15\baselineskip}
\section*{Tipsy Pudding}

Flavor Boiled Custard with Sherry wine, and pour over slices of stale
sponge cake; cover with Cream Sauce I or II.



\needspace{15\baselineskip}
\section*{Peach Custard}

Arrange alternate layers of stale cake and sections of canned peaches in
glass dish and pour over Boiled Custard. Bananas may be used instead of
peaches; it is then called \textit{Banana Custard}.



\needspace{15\baselineskip}
\section*{Orange Custard}

Arrange slices of sweet oranges in glass dish, pour over them Boiled
Custard; chill, and cover with Meringue I.



\needspace{15\baselineskip}
\section*{Apple Meringue}

Use Meringue I and pile lightly on baked apples, brown in oven, cool,
and serve with Boiled Custard. Canned peaches, drained from their
liquor, may be prepared in the same way.



\needspace{15\baselineskip}
\section*{Apple Snow}


\begin{itemize}
\setlength{\itemsep}{0pt}
\setlength{\parsep}{0pt}
\item 3 egg whites
\item 3/4 cup apple pulp
\item Powdered sugar
\end{itemize}

\vspace{-0.5em}
\noindent%
Pare, quarter, and core four sour apples, steam until soft, and rub
through sieve; there should be three-fourths cup apple pulp. Beat on a
platter whites of eggs until stiff (using wire whisk), add gradually
apple sweetened to taste, and continue beating. Pile lightly on glass
dish, chill, and serve with Boiled Custard.



\needspace{15\baselineskip}
\section*{Prune Whip}


\begin{itemize}
\setlength{\itemsep}{0pt}
\setlength{\parsep}{0pt}
\item 1/3 lb. prunes
\item 5 egg whitess
\item 1/2 cup sugar
\item 1/2 tablespoon lemon juice
\end{itemize}

\vspace{-0.5em}
\noindent%
Pick over and wash prunes, then soak several hours in cold water to
cover; cook in same water until soft; remove stones and rub prunes
through a strainer, add sugar, and cook five minutes; the mixture should
be of the consistency of marmalade. Beat whites of eggs until stiff, add
prune mixture gradually when cold, and lemon juice. Pile lightly on
buttered pudding-dish, bake twenty minutes in slow oven. Serve cold with
Boiled Custard.



\needspace{15\baselineskip}
\section*{Raspberry Whip}


\begin{itemize}
\setlength{\itemsep}{0pt}
\setlength{\parsep}{0pt}
\item 1 1/4 cups raspberries
\item 1 cup powdered sugar
\item 1 egg white
\end{itemize}

\vspace{-0.5em}
\noindent%
Put ingredients in bowl and beat with wire whisk until stiff enough to
hold in shape; about thirty minutes will be required for beating. Pile
lightly on dish, chill, surround with lady fingers, and serve with
Boiled Custard.

\textbf{Strawberry Whip} may be prepared in same way.



\needspace{15\baselineskip}
\section*{Baked Custard}


\begin{itemize}
\setlength{\itemsep}{0pt}
\setlength{\parsep}{0pt}
\item 4 cups scalded milk
\item 4 to 6 eggs
\item 1/2 cup sugar
\item 1/4 teaspoon salt
\item Few gratings nutmeg
\end{itemize}

\vspace{-0.5em}
\noindent%
Beat eggs slightly, add sugar and salt, pour on slowly scalded milk;
strain in buttered mould, set in pan of hot water. Sprinkle with nutmeg,
and bake in slow oven until firm, which may be readily determined by
running a silver knife through custard; if knife comes out clean,
custard is done. During baking, care must be taken that water
surrounding mould does not reach boiling-point, or custard will whey.
Always bear in mind that eggs and milk in combination must be cooked at
a low temperature. For \textit{cup custards} allow four eggs to four cups milk;
for large moulded custard, six eggs; if less eggs are used custard is
liable to crack when turned on a serving dish.



\needspace{15\baselineskip}
\section*{Caramel Custard}


\begin{itemize}
\setlength{\itemsep}{0pt}
\setlength{\parsep}{0pt}
\item 4 cups scalded milk
\item 5 eggs
\item 1/2 teaspoon salt
\item 1 teaspoon vanilla
\item 1/2 cup sugar
\end{itemize}

\vspace{-0.5em}
\noindent%
Put sugar in omelet pan, stir constantly over hot part of range until
melted to a syrup of light brown color. Add gradually to milk, being
careful that milk does not bubble up and go over, as is liable on
account of high temperature of sugar. As soon as sugar is melted in
milk, add mixture gradually to eggs slightly beaten; add salt and
flavoring, then strain in buttered mould. Bake as custard. Chill, and
serve with Caramel Sauce.



\needspace{15\baselineskip}
\section*{Caramel Sauce}


\begin{itemize}
\setlength{\itemsep}{0pt}
\setlength{\parsep}{0pt}
\item 1/2 cup sugar
\item 1/2 cup boiling water
\end{itemize}

\vspace{-0.5em}
\noindent%
Melt sugar as for Caramel Custard, add water, simmer ten minutes; cool
before serving.



\needspace{15\baselineskip}
\section*{Coffee Custard}


\begin{minipage}{1.0\textwidth}
{\setlength{\multicolsep}{0pt}\setlength{\columnsep}{2em}\raggedcolumns%
\begin{multicols}{2}
\begin{itemize}
\setlength{\itemsep}{0pt}
\setlength{\parsep}{0pt}
\item 2 cups milk
\item 2 tablespoons ground coffee
\item 3 eggs
\item 1/4 cup sugar
\item 1/8 teaspoon salt
\item 1/4 teaspoon vanilla
\end{itemize}
\end{multicols}}
\end{minipage}

\vspace{0.3em}
\noindent%
Scald milk with coffee, and strain. Beat eggs slightly; add sugar, salt,
vanilla, and milk. Strain into buttered individual moulds, set in pan of
hot water, and bake until firm.



\needspace{15\baselineskip}
\section*{Tapioca Cream}


\begin{minipage}{1.0\textwidth}
{\setlength{\multicolsep}{0pt}\setlength{\columnsep}{2em}\raggedcolumns%
\begin{multicols}{2}
\begin{itemize}
\setlength{\itemsep}{0pt}
\setlength{\parsep}{0pt}
\item 1/4 cup pearl tapioca or 1 1/2 tablespoons Minute Tapioca
\item 2 cups scalded milk
\item 2 eggs
\item 1/3 cup sugar
\item 1/4 teaspoon salt
\item 1 teaspoon vanilla
\end{itemize}
\end{multicols}}
\end{minipage}

\vspace{0.3em}
\noindent%
Pick over tapioca and soak one hour in cold water to cover, drain, add
to milk, and cook in double boiler until tapioca is transparent. Add
half the sugar to milk and remainder to egg yolks slightly beaten, and
salt. Combine by pouring hot mixture slowly on egg mixture, return to
double boiler, and cook until it thickens. Remove from range and add
whites of eggs beaten stiff. Chill and flavor.



\needspace{15\baselineskip}
\section*{Norwegian Prune Pudding}


\begin{minipage}{1.0\textwidth}
{\setlength{\multicolsep}{0pt}\setlength{\columnsep}{2em}\raggedcolumns%
\begin{multicols}{2}
\begin{itemize}
\setlength{\itemsep}{0pt}
\setlength{\parsep}{0pt}
\item 1/2 lb. prunes
\item 2 cups cold water
\item 1 cup sugar
\item 1 inch piece stick cinnamon
\item 1 1/3 cups boiling water
\item 1/3 cup corn-starch
\item 1 tablespoon lemon juice
\end{itemize}
\end{multicols}}
\end{minipage}

\vspace{0.3em}
\noindent%
Pick over and wash prunes, then soak one hour in cold water, and boil
until soft; remove stones, obtain meat from stones and add to prunes;
then add sugar, cinnamon, boiling water, and simmer ten minutes. Dilute
corn-starch with enough cold water to pour easily, add to prune mixture,
and cook five minutes. Remove cinnamon, mould, then chill, and serve
with cream.



\needspace{15\baselineskip}
\section*{Nut Prune Soufflé}

Follow recipe for Norwegian Prune Pudding, then add whites two eggs
beaten stiff and one-half cup walnut meats broken in pieces.



\needspace{15\baselineskip}
\section*{Apples In Bloom}

Select eight red apples, cook in boiling water until soft, turning them
often. Have water half surround apples. Remove skins carefully, that the
red color may remain, and arrange on serving dish. To the water add one
cup sugar, grated rind one-half lemon, and juice one orange; simmer
until reduced to one cup. Cool, and pour over apples. Serve with Cream
Sauce I or II.



\needspace{15\baselineskip}
\section*{Neapolitan Baskets}

Bake sponge cake in gem pans, cool, and remove centres. Fill with Cream
Sauce I, flavoring half the sauce with chocolate. Melt chocolate, dilute
with hot water, cool, and add Cream Sauce slowly to chocolate. Garnish
with candied cherries and angelica and insert strips of angelica to
represent handles.



\needspace{15\baselineskip}
\section*{Wine Cream}

Arrange lady fingers or slices of sponge cake in a dish, pour over cream
made as follows: Mix one-third cup sugar, grated rind and juice one-half
lemon, one-fourth cup Sherry wine, and yolks of two eggs; place over
fire and stir vigorously with wire whisk until it thickens and is
frothy, then pour over beaten whites of two eggs and continue beating.



\needspace{15\baselineskip}
\subsection*{Orange Salad}

Arrange layers of sliced oranges, sprinkling each layer with powdered
sugar and shredded cocoanut. Sliced oranges when served alone should not
stand long after slicing, as they are apt to become bitter.



\needspace{15\baselineskip}
\section*{Fruit Salad I}

Arrange alternate layers of shredded pineapple, sliced bananas, and
sliced oranges, sprinkling each layer with powdered sugar. Chill before
serving.

\textit{To Shred Pineapple.} Pare and cut out eyes, pick off small pieces with
a silver fork, continuing until all soft part is removed. \_To Slice
Oranges.\_ Remove skin and white covering, slice lengthwise that the
tough centre may not be served; seeds should be removed.



\needspace{15\baselineskip}
\section*{Fruit Salad II}

Pare a pineapple and cut in one-quarter inch slices, remove hard
centres, sprinkle with powdered sugar, set aside one hour in a cool
place; drain, spread on serving dish, arrange a circle of thin slices of
banana on each piece, nearly to the edge, pile strawberries in centre,
pour over syrup drained from pineapple, sprinkle with powdered sugar,
and serve with or without Cream Sauce.



\needspace{15\baselineskip}
\section*{Fruit Salad With Wine Dressing}

Arrange alternate layers of sliced fruit, using pineapples, bananas,
oranges, and grapes; pour over all Wine Dressing, and let stand one hour
in a cold place.



\needspace{15\baselineskip}
\section*{Wine Dressing}

Mix one-half cup sugar, one-third cup Sherry wine, and two tablespoons
Madeira.



\needspace{15\baselineskip}
\section*{Cream Whips}

Sweeten thin cream, flavor with vanilla, brandy, or wine, then whip;
half fill frappé glasses with any preserve, pile on lightly the whip.



\needspace{15\baselineskip}
\section*{Sautéd Pears With Chocolate Sauce}

Pare four Bartlett pears, cut in fourths lengthwise, and sauté in butter
until browned. Canned pears drained from their syrup may be used in
place of fresh fruit. Arrange in serving dish and pour over

\textbf{Chocolate Sauce.} Cook two ounces sweet chocolate, one tablespoon
sugar, and one and one-fourth cups milk in double boiler five minutes;
then add one teaspoon arrowroot mixed with one-fourth cup cream and a
few grains salt, and cook ten minutes. Melt one and one-half tablespoons
butter, add one-fourth cup powdered sugar, and cook until well
caramelized, stirring constantly. Add to first mixture, and flavor with
one-half teaspoon vanilla. Chill thoroughly.



\needspace{15\baselineskip}
\section*{Lemon Jelly}


\begin{minipage}{1.0\textwidth}
{\setlength{\multicolsep}{0pt}\setlength{\columnsep}{2em}\raggedcolumns%
\begin{multicols}{2}
\begin{itemize}
\setlength{\itemsep}{0pt}
\setlength{\parsep}{0pt}
\item 1/2 box gelatine or
\item 2 tablespoons granulated gelatine
\item 1/2 cup cold water
\item 2 1/2 cups boiling water
\item 1 cup sugar
\item 1/2 cup lemon juice
\end{itemize}
\end{multicols}}
\end{minipage}

\vspace{0.3em}
\noindent%
Soak gelatine twenty minutes in cold water, dissolve in boiling water,
strain, and add to sugar and lemon juice. Turn into mould, and chill.



\needspace{15\baselineskip}
\section*{Orange Jelly}


\begin{minipage}{1.0\textwidth}
{\setlength{\multicolsep}{0pt}\setlength{\columnsep}{2em}\raggedcolumns%
\begin{multicols}{2}
\begin{itemize}
\setlength{\itemsep}{0pt}
\setlength{\parsep}{0pt}
\item 1/2 box gelatine or
\item 2 tablespoons granulated gelatine
\item 1/2 cup cold water
\item 1 1/2 cups boiling water
\item 1 cup sugar
\item 1 1/2 cups orange juice
\item 3 tablespoons lemon juice
\end{itemize}
\end{multicols}}
\end{minipage}

\vspace{0.3em}
\noindent%
Make same as Lemon Jelly.

\textbf{To Remove Juice from Oranges.} Cut fruit in halves crosswise, remove
with spoon pulp and juice from sections, and strain through double
cheese-cloth; or use a glass lemon squeezer.



\needspace{15\baselineskip}
\section*{Kumquat Jelly}


\begin{minipage}{1.0\textwidth}
{\setlength{\multicolsep}{0pt}\setlength{\columnsep}{2em}\raggedcolumns%
\begin{multicols}{2}
\begin{itemize}
\setlength{\itemsep}{0pt}
\setlength{\parsep}{0pt}
\item 1 1/2 cups kumquat juice
\item 1/2 cup sugar
\item 1/4 cup Sauterne
\item 1 1/2 tablespoons Orange Curaçoa
\item 1 tablespoon granulated gelatine
\item 2 tablespoons cold water
\item Few grains salt
\end{itemize}
\end{multicols}}
\end{minipage}

\vspace{0.3em}
\noindent%
Wipe three-fourths box kumquats, cut in slices, add cold water to cover,
bring slowly to boiling-point, and cook slowly one-half hour; then
strain; there should be one and one-half cups juice. Add sugar, wine,
and curaçoa. Soak gelatine in cold water, and add to first mixture
heated to boiling-point; then add salt. Strain, turn into individual
mould, and chill. Remove to serving dish, and garnish with halves of
kumquats, cooked in syrup until soft, drained, and rolled in sugar.



\needspace{15\baselineskip}
\section*{Coffee Jelly}


\begin{minipage}{1.0\textwidth}
{\setlength{\multicolsep}{0pt}\setlength{\columnsep}{2em}\raggedcolumns%
\begin{multicols}{2}
\begin{itemize}
\setlength{\itemsep}{0pt}
\setlength{\parsep}{0pt}
\item 1/2 box gelatine or
\item 2 tablespoons granulated gelatine
\item 1/2 cup cold water
\item 1 cup boiling water
\item 1/3 cup sugar
\item 2 cups boiled coffee
\end{itemize}
\end{multicols}}
\end{minipage}

\vspace{0.3em}
\noindent%
Make same as Lemon Jelly. Serve with sugar and cream.



\needspace{15\baselineskip}
\section*{Cider Jelly}


\begin{minipage}{1.0\textwidth}
{\setlength{\multicolsep}{0pt}\setlength{\columnsep}{2em}\raggedcolumns%
\begin{multicols}{2}
\begin{itemize}
\setlength{\itemsep}{0pt}
\setlength{\parsep}{0pt}
\item 1/2 box gelatine or
\item 2 tablespoons granulated gelatine
\item 1/2 cup cold water
\item 1 cup boiling water
\item 2 cups cider
\item Sugar
\end{itemize}
\end{multicols}}
\end{minipage}

\vspace{0.3em}
\noindent%
Make same as Lemon Jelly.



\needspace{15\baselineskip}
\section*{Wine Jelly I}


\begin{minipage}{1.0\textwidth}
{\setlength{\multicolsep}{0pt}\setlength{\columnsep}{2em}\raggedcolumns%
\begin{multicols}{2}
\begin{itemize}
\setlength{\itemsep}{0pt}
\setlength{\parsep}{0pt}
\item 1/2 box gelatine or
\item 2 tablespoons granulated gelatine
\item 1/2 cup cold water
\item 1 2/3 cups boiling water
\item 1 cup sugar
\item 1 cup Sherry or Madeira wine
\item 1/3 cup orange juice
\item 3 tablespoons lemon juice
\end{itemize}
\end{multicols}}
\end{minipage}

\vspace{0.3em}
\noindent%
Soak gelatine twenty minutes in cold water, dissolve in boiling water;
add sugar, wine, orange juice, and lemon juice; strain, mould, and
chill. If a stronger jelly is desired, use additional wine in place of
orange juice.



\needspace{15\baselineskip}
\section*{Wine Jelly II}


\begin{minipage}{1.0\textwidth}
{\setlength{\multicolsep}{0pt}\setlength{\columnsep}{2em}\raggedcolumns%
\begin{multicols}{2}
\begin{itemize}
\setlength{\itemsep}{0pt}
\setlength{\parsep}{0pt}
\item 1/2 box gelatine or
\item 2 1/2 tablespoons granulated gelatine
\item 1/2 cup cold water
\item 1 2/3 cups boiling water
\item 1 cup sugar
\item 1/2 cup Sherry wine
\item 2 tablespoons brandy
\item Kirsch
\item 1/3 cup orange juice
\item 3 tablespoons lemon juice
\item Fruit red
\end{itemize}
\end{multicols}}
\end{minipage}

\vspace{0.3em}
\noindent%
Soak gelatine twenty minutes in cold water, dissolve in hot water, add
sugar, fruit juices, Sherry, brandy, and enough Kirsch to make one cup
of strong liquor, then color with fruit red. Strain, mould, and chill.
Serve with or without Cream Sauce I.



\needspace{15\baselineskip}
\section*{Russian Jelly}


\begin{minipage}{1.0\textwidth}
{\setlength{\multicolsep}{0pt}\setlength{\columnsep}{2em}\raggedcolumns%
\begin{multicols}{2}
\begin{itemize}
\setlength{\itemsep}{0pt}
\setlength{\parsep}{0pt}
\item 1/4 box gelatine or
\item 1 tablespoon granulated gelatine
\item 1/4 cup cold water
\item 1 cup boiling water
\item 2/3 cup sugar
\item 1/2 cup Sauterne
\item 1/4 cup orange juice
\item 1 1/2 tablespoons lemon juice
\end{itemize}
\end{multicols}}
\end{minipage}

\vspace{0.3em}
\noindent%
Make same as other jellies, cool slightly, and beat until frothy and
firm enough to mould. Turn into mould and chill.



\needspace{15\baselineskip}
\section*{Jelly In Glasses}

Use recipe for Wine or Russian Jelly. Fill Apollinaris glasses
three-fourths full, reserving one-fourth of the mixture, which, after
cooling, is to be beaten until frothy (using a Dover egg-beater) and
placed on top of jelly in glasses which represents freshly drawn lager
beer. This is a most attractive way of serving jelly to one who is ill.



\needspace{15\baselineskip}
\section*{Sauterne Jelly}

Soak two tablespoons granulated gelatine in one-half cup cold water, and
dissolve in one and one-half cups boiling water. Add one and one-half
cups Sauterne, three tablespoons lemon juice, and one cup sugar. Color
with leaf green, strain into a shallow pan, chill, and cut in inch
cubes.



\needspace{15\baselineskip}
\section*{Jellied Prunes}


\begin{minipage}{1.0\textwidth}
{\setlength{\multicolsep}{0pt}\setlength{\columnsep}{2em}\raggedcolumns%
\begin{multicols}{2}
\begin{itemize}
\setlength{\itemsep}{0pt}
\setlength{\parsep}{0pt}
\item 1/3 lb. prunes
\item 2 cups cold water
\item Boiling water
\item 1/2 cup cold water
\item 1/2 box gelatine or
\item 2 1/2 tablespoons granulated gelatine
\item 1 cup sugar
\item 1/4 cup lemon juice
\end{itemize}
\end{multicols}}
\end{minipage}

\vspace{0.3em}
\noindent%
Pick over, wash, and soak prunes for several hours in two cups cold
water, and cook in same water until soft; remove prunes; stone, and cut
in quarters. To prune water add enough boiling water to make two cups.
Soak gelatine in half-cup cold water, dissolve in hot liquid, add sugar,
lemon juice, then strain, add prunes, mould, and chill. Stir twice while
cooling to prevent prunes from settling. Serve with sugar and cream.



\needspace{15\baselineskip}
\section*{Jellied Walnuts}


\begin{minipage}{1.0\textwidth}
{\setlength{\multicolsep}{0pt}\setlength{\columnsep}{2em}\raggedcolumns%
\begin{multicols}{2}
\begin{itemize}
\setlength{\itemsep}{0pt}
\setlength{\parsep}{0pt}
\item 1/4 box gelatine or
\item 1 tablespoon granulated gelatine
\item 1/4 cup cold water
\item 1/3 cup boiling water
\item 3/4 cup sugar
\item 1/2 cup Sherry wine
\item 1/2 cup orange juice
\item 3 tablespoons lemon juice
\end{itemize}
\end{multicols}}
\end{minipage}

\vspace{0.3em}
\noindent%
Make same as other jellies and cover bottom of shallow pan with one-half
the mixture. When nearly firm, place over it, one inch apart, halves of
English walnuts. Cover with remaining mixture. Chill, and cut in
squares. Serve with whipped cream sweetened and flavored.



\needspace{15\baselineskip}
\section*{Apricot And Wine Jelly}


\begin{minipage}{1.0\textwidth}
{\setlength{\multicolsep}{0pt}\setlength{\columnsep}{2em}\raggedcolumns%
\begin{multicols}{2}
\begin{itemize}
\setlength{\itemsep}{0pt}
\setlength{\parsep}{0pt}
\item 1/2 box gelatine or
\item 2 tablespoons granulated gelatine
\item 1/2 cup cold water
\item 1 cup boiling water
\item 1 cup apricot juice
\item 1 cup wine
\item 1 cup sugar
\item 1 tablespoon lemon juice
\end{itemize}
\end{multicols}}
\end{minipage}

\vspace{0.3em}
\noindent%
Garnish individual moulds with halves of canned apricots, fill with
mixture made same as for other jellies, and chill. Arrange on serving
dish and garnish with whipped cream forced through a pastry bag and
tube.



\needspace{15\baselineskip}
\section*{Snow Pudding I}


\begin{minipage}{1.0\textwidth}
{\setlength{\multicolsep}{0pt}\setlength{\columnsep}{2em}\raggedcolumns%
\begin{multicols}{2}
\begin{itemize}
\setlength{\itemsep}{0pt}
\setlength{\parsep}{0pt}
\item 1/4 box gelatine or
\item 1 tablespoon granulated gelatine
\item 1/4 cup cold water
\item 1 cup boiling water
\item 1 cup sugar
\item 1/4 cup lemon juice
\item 3 egg whites
\end{itemize}
\end{multicols}}
\end{minipage}

\vspace{0.3em}
\noindent%
Soak gelatine in cold water, dissolve in boiling water, add sugar and
lemon juice, strain, and set aside in cool place; occasionally stir
mixture, and when quite thick, beat with wire spoon or whisk until
frothy; add whites of eggs beaten stiff, and continue beating until
stiff enough to hold its shape. Mould, or pile by spoonfuls on glass
dish; serve cold with Boiled Custard. A very attractive dish may be
prepared by coloring half the mixture with fruit red.



\needspace{15\baselineskip}
\section*{Snow Pudding II}

Beat whites of four eggs until stiff, add one-half tablespoon granulated
gelatine dissolved in three tablespoons boiling water, beat until
thoroughly mixed, add one-fourth cup powdered sugar, and flavor with
one-half teaspoon lemon extract. Pile lightly on dish, serve with Boiled
Custard.



\needspace{15\baselineskip}
\section*{Amber Pudding}

Make as Snow Pudding I, using cider instead of boiling water, and
one-fourth cup boiling water to dissolve gelatine, omitting lemon juice,
and sweeten to taste.



\needspace{15\baselineskip}
\section*{Toasted Marshmallows}


\begin{minipage}{1.0\textwidth}
{\setlength{\multicolsep}{0pt}\setlength{\columnsep}{2em}\raggedcolumns%
\begin{multicols}{2}
\begin{itemize}
\setlength{\itemsep}{0pt}
\setlength{\parsep}{0pt}
\item 1 tablespoon granulated gelatine
\item 1 cup boiling water
\item 1 cup sugar
\item 3 egg whites
\item 1 1/2 teaspoons vanilla
\item Macaroons
\end{itemize}
\end{multicols}}
\end{minipage}

\vspace{0.3em}
\noindent%
Dissolve gelatine in boiling water, add sugar, and as soon as dissolved
set bowl containing mixture in pan of ice-water; then add whites of eggs
and vanilla and beat until mixture thickens. Turn into a shallow pan,
first dipped in cold water, and let stand until thoroughly chilled.
Remove from pan and cut in pieces the size and shape of marshmallows;
then roll in macaroons which have been dried and rolled. Serve with
sugar and cream.







\needspace{15\baselineskip}
\section*{Pudding À La Macédoine}

Make fruit or wine jelly mixture. Place a mould in pan of ice-water,
pour in mixture one-half inch deep; when firm, decorate with slices of
banana from which radiate thin strips of figs (seed side down), cover
fruit, adding mixture by spoonfuls lest the fruit be disarranged. When
firm, add more fruit and mixture; repeat until all is used, each time
allowing mixture to stiffen before fruit is added. In preparing this
dish various fruits may be used: oranges, bananas, dates, figs, and
English walnuts. Serve with Cream Sauce I.



\needspace{15\baselineskip}
\section*{Fruit Chartreuse}

Make fruit or wine jelly mixture. Place a mould in pan of ice-water,
pour in mixture one-half inch deep; when firm, decorate with candied
cherries and angelica; add by spoonfuls more mixture to cover fruit;
when this is firm, place a smaller mould in centre on jelly, and fill
with ice-water. Pour gradually remaining jelly mixture between moulds;
when firm, invert to empty smaller mould of ice-water; then pour in some
tepid water; let stand a few seconds, when small mould may easily be
removed. Fill space thus made with fresh sweetened fruit, using shredded
pineapple, sliced bananas, and strawberries.



\needspace{15\baselineskip}
\section*{Spanish Cream}


\begin{minipage}{1.0\textwidth}
{\setlength{\multicolsep}{0pt}\setlength{\columnsep}{2em}\raggedcolumns%
\begin{multicols}{2}
\begin{itemize}
\setlength{\itemsep}{0pt}
\setlength{\parsep}{0pt}
\item 1/4 box gelatine or
\item 1 tablespoon granulated gelatin
\item 3 cups milk
\item 3 egg whites
\item Yolk 3 eggs
\item 1/2 cup sugar (scant)
\item 1/4 teaspoon salt
\item 1 teaspoon vanilla or
\item 3 tablespoons wine
\end{itemize}
\end{multicols}}
\end{minipage}

\vspace{0.3em}
\noindent%
Scald milk with gelatine, add sugar, pour slowly on yolks of eggs
slightly beaten. Return to double boiler and cook until thickened,
stirring constantly; remove from range, add salt, flavoring, and whites
of eggs beaten stiff. Turn into individual moulds, first dipped in cold
water, and chill; serve with cream. More gelatine will be required if
large moulds are used.



\needspace{15\baselineskip}
\section*{Coffee Soufflé}


\begin{minipage}{1.0\textwidth}
{\setlength{\multicolsep}{0pt}\setlength{\columnsep}{2em}\raggedcolumns%
\begin{multicols}{2}
\begin{itemize}
\setlength{\itemsep}{0pt}
\setlength{\parsep}{0pt}
\item 1 1/2 cups coffee infusion
\item 1/2 cup milk
\item 2/3 cup sugar
\item 1/4 teaspoon salt
\item 3 eggs
\item 1/2 teaspoon vanilla
\item 1 tablespoon granulated gelatine
\end{itemize}
\end{multicols}}
\end{minipage}

\vspace{0.3em}
\noindent%
Mix coffee infusion, milk, one-half of the sugar and gelatine, and heat
in double boiler. Add remaining sugar, salt, and yolks of eggs slightly
beaten; cook until mixture thickens, remove from range, add whites of
eggs beaten until stiff and vanilla. Mould, chill, and serve with cream.



\needspace{15\baselineskip}
\section*{Columbian Pudding}

Cover the bottom of a fancy mould with Wine Jelly. Line the upper part
of mould with figs, cut in halves crosswise, which have been soaked in
jelly, having seed side next to mould. Fill centre with Spanish Cream;
chill, and turn on a serving dish. Garnish with cubes of Wine Jelly.



\needspace{15\baselineskip}
\section*{Macaroon Cream}


\begin{minipage}{1.0\textwidth}
{\setlength{\multicolsep}{0pt}\setlength{\columnsep}{2em}\raggedcolumns%
\begin{multicols}{2}
\begin{itemize}
\setlength{\itemsep}{0pt}
\setlength{\parsep}{0pt}
\item 1/4 box gelatine or
\item 1 tablespoon granulated gelatine
\item 1/4 cup cold water
\item 2 cups scalded milk
\item 3 egg yolks
\item 1/3 cup sugar
\item 1/8 teaspoon salt
\item 2/3 cup pounded macaroons
\item 1 teaspoon vanilla
\item 3 egg whites
\end{itemize}
\end{multicols}}
\end{minipage}

\vspace{0.3em}
\noindent%
Soak gelatine in cold water. Make custard of milk, yolks of eggs, sugar,
and salt; add gelatine, and strain into pan set in ice-water. Add
macaroons and flavoring, stirring until it begins to thicken; then add
whites of eggs beaten stiff, mould, chill, and serve garnished with
macaroons.



\needspace{15\baselineskip}
\section*{Cold Cabinet Pudding}


\begin{minipage}{1.0\textwidth}
{\setlength{\multicolsep}{0pt}\setlength{\columnsep}{2em}\raggedcolumns%
\begin{multicols}{2}
\begin{itemize}
\setlength{\itemsep}{0pt}
\setlength{\parsep}{0pt}
\item 1/4 box gelatine or
\item 1 tablespoon granulated gelatine
\item 1/4 cup cold water
\item 2 cups scalded milk
\item 3 egg yolks
\item 1/3 cup sugar
\item 1/8 teaspoon salt
\item 1 teaspoon vanilla
\item 1 tablespoon brandy
\item 5 lady fingers
\item 6 macaroons
\end{itemize}
\end{multicols}}
\end{minipage}

\vspace{0.3em}
\noindent%
Soak gelatine in cold water and add to custard made of milk, eggs,
sugar, salt; strain, cool slightly, and flavor. Place a mould in pan of
ice-water, decorate with candied cherries and angelica, cover with
mixture, added carefully by spoonfuls; when firm, add layer of lady
fingers (first soaked in custard), then layer of macaroons (also soaked
in custard); repeat, care being taken that each layer is firm before
another is added. Garnish, and serve with Cream Sauce I and candied
cherries.



\needspace{15\baselineskip}
\section*{Mont Blanc}

Remove shells from three cups French chestnuts, cook in small quantity
of boiling water until soft, when there will be no water remaining.
Mash, sweeten to taste with powdered sugar, and moisten with hot milk;
cook two minutes. Rub through strainer, cool, flavor with vanilla,
Kirsch or Maraschino. Pile in form of pyramid, cover with Cream Sauce I,
garnish base with Cream Sauce I forced through pastry bag and tube.




\needspace{15\baselineskip}
\section*{Crême Aux Fruits}


\begin{minipage}{1.0\textwidth}
{\setlength{\multicolsep}{0pt}\setlength{\columnsep}{2em}\raggedcolumns%
\begin{multicols}{2}
\begin{itemize}
\setlength{\itemsep}{0pt}
\setlength{\parsep}{0pt}
\item 1/4 box gelatine or
\item 1 tablespoon granulated gelatine
\item 1/4 cup cold water
\item 1/4 cup scalded milk
\item 1/2 cup sugar
\item 2 egg whites
\item 1/2 pint thick cream
\item 1/3 cup milk
\item 1/3 cup cooked prunes,
\item cut in pieces
\item 1/3 cup chopped figs
\end{itemize}
\end{multicols}}
\end{minipage}

\vspace{0.3em}
\noindent%
Soak gelatine in cold water, dissolve in scalded milk, and add sugar.
Strain in pan set in ice-water, stir constantly, and when it begins to
thicken add whites of eggs beaten stiff, cream (diluted with milk and
beaten), prunes, and figs. Mould and chill.



\needspace{15\baselineskip}
\section*{To Whip Cream}

Thin and heavy cream are both used in making and garnishing desserts.

\textit{Heavy cream} is bought in half-pint, pint, and quart glass jars, and
usually retails at sixty cents per quart; \textit{thin} or \textit{strawberry cream}
comes in glass jars or may be bought in bulk, and usually retails for
thirty cents per quart. Heavy cream is very rich; for which reason, when
whipped without being diluted, it is employed as a garnish; even when so
used, it is generally diluted with one-fourth to one-third its bulk in
milk; when used in combination with other ingredients for making
desserts, it is diluted from one-half to two-thirds its bulk in milk.
Thin cream is whipped without being diluted. Cream should be thoroughly
chilled for whipping. Turn cream to be whipped into a bowl (care being
taken not to select too large a bowl), and set in pan of crushed ice, to
which water is added that cream may be quickly chilled; without addition
of water, cream will not be so thoroughly chilled.

For whipping heavy cream undiluted, or diluted with one-third or less
its bulk in milk, use Dover egg-beater; undiluted heavy cream if beaten
a moment too long will come to butter. Heavy cream diluted, whipped,
sweetened, and flavored, is often served with puddings, and called Cream
Sauce.

Thin cream is whipped by using a whip churn, as is heavy cream when
diluted with one-half to two-thirds its bulk in milk. Place churn in
bowl containing cream, hold down cover with left hand, with right hand
work dasher with quick downward and slow upward motions; avoid raising
dasher too high in cylinder, thus escaping spattering of cream. The
first whip which appears should be stirred into cream, as air bubbles
are too large and will break; second whip should be removed by spoonfuls
to a strainer, strainer to be placed in a pan, as some cream will drain
through. The first cream which drains through may be turned into bowl to
be rewhipped, and continue whipping as long as possible.

There will be some cream left in bowl which does not come above
perforations in whip churn, and cannot be whipped. Cream which remains
may be scalded and used to dissolve gelatine when making desserts which
require gelatine. Cream should treble its bulk in whipping. By following
these directions one need have no difficulty, if cream is of right
consistency; always bearing in mind heavy cream must be whipped with a
Dover egg-beater; thin cream must be whipped with a churn.



\needspace{15\baselineskip}
\section*{Charlotte Russe}


\begin{minipage}{1.0\textwidth}
{\setlength{\multicolsep}{0pt}\setlength{\columnsep}{2em}\raggedcolumns%
\begin{multicols}{2}
\begin{itemize}
\setlength{\itemsep}{0pt}
\setlength{\parsep}{0pt}
\item 1/4 box gelatine or
\item 1 tablespoon granulated gelatine
\item 1/4 cup cold water
\item 1/3 cup scalded cream
\item 1/3 cup powdered sugar
\item Whip from 3 1/2 cups thin cream
\item 1 1/2 teaspoons vanilla
\item 6 lady fingers
\end{itemize}
\end{multicols}}
\end{minipage}

\vspace{0.3em}
\noindent%
Soak gelatine in cold water, dissolve in scalded cream, strain into a
bowl, and add sugar and vanilla. Set bowl in pan of ice-water and stir
constantly until it begins to thicken, then fold in whip from cream,
adding one-third at a time. Should gelatine mixture become too thick,
melt over hot water, and again cool before adding whip. Trim ends and
sides of lady fingers, place around inside of a mould, crust side out,
one-half inch apart. Turn in mixture, and chill. Serve garnished with
cubes of Wine Jelly. Charlotte Russe is sometimes made in individual
moulds; these are often garnished on top with some of mixture forced
through a pastry bag and tube. Individual moulds are frequently lined
with thin slices of sponge cake cut to fit moulds.



\needspace{15\baselineskip}
\section*{Orange Trifle}


\begin{minipage}{1.0\textwidth}
{\setlength{\multicolsep}{0pt}\setlength{\columnsep}{2em}\raggedcolumns%
\begin{multicols}{2}
\begin{itemize}
\setlength{\itemsep}{0pt}
\setlength{\parsep}{0pt}
\item 1/2 box gelatine or
\item 2 tablespoons granulated gelatine
\item 1/2 cup cold water
\item 1/2 cup boiling water
\item 1 cup sugar
\item 1 cup orange juice
\item Grated rind 1 orange
\item 1 tablespoon lemon juice
\item Whip from 3 1/2 cups cream
\end{itemize}
\end{multicols}}
\end{minipage}

\vspace{0.3em}
\noindent%
Make same as Charlotte Russe, and mould; or make orange jelly, color
with fruit red, and cover bottom of mould one-half inch deep; chill, and
when firm fill with Orange Trifle mixture. Cool remaining jelly in
shallow pan, cut in cubes, and garnish base of mould.



\needspace{15\baselineskip}
\section*{Banana Cantaloupe}


\begin{minipage}{1.0\textwidth}
{\setlength{\multicolsep}{0pt}\setlength{\columnsep}{2em}\raggedcolumns%
\begin{multicols}{2}
\begin{itemize}
\setlength{\itemsep}{0pt}
\setlength{\parsep}{0pt}
\item 1/2 box gelatine or
\item 2 tablespoons granulated gelatine
\item 1/2 cup cold water
\item 2 egg whites
\item 1/4 cup powdered sugar
\item 3/4 cup scalded cream
\item 2/3 cup sugar
\item 4 bananas, mashed pulp
\item 1 tablespoon lemon juice
\item Whip from 3 1/2 cups cream
\item 12 lady fingers
\end{itemize}
\end{multicols}}
\end{minipage}

\vspace{0.3em}
\noindent%
Soak gelatine in cold water, beat whites of eggs slightly, add powdered
sugar, and gradually hot cream, cook over hot water until it thickens;
add soaked gelatine and remaining sugar, strain into a pan set in
ice-water, add bananas and lemon juice, stir until it begins to thicken,
then fold in whip from cream. Line a melon mould with lady fingers
trimmed to just fit sections of mould, turn in the mixture, spread
evenly, and chill.



\needspace{15\baselineskip}
\section*{Chocolate Charlotte}


\begin{minipage}{1.0\textwidth}
{\setlength{\multicolsep}{0pt}\setlength{\columnsep}{2em}\raggedcolumns%
\begin{multicols}{2}
\begin{itemize}
\setlength{\itemsep}{0pt}
\setlength{\parsep}{0pt}
\item 1/4 box gelatine or
\item 1 tablespoon granulated gelatine
\item 1/4 cup cold water
\item 1/3 cup scalded cream
\item 1 1/2 squares Baker's chocolate
\item 3 tablespoons hot water
\item 2/3 cup powdered sugar
\item Whip from 3 cups cream
\item 1 teaspoon vanilla
\item 6 lady fingers
\end{itemize}
\end{multicols}}
\end{minipage}

\vspace{0.3em}
\noindent%
Melt chocolate by placing in a small saucepan set in a larger saucepan
of boiling water, add half the sugar, dilute with boiling water, and add
to gelatine mixture while hot. Proceed same as in recipe for Charlotte
Russe.



\needspace{15\baselineskip}
\section*{Caramel Charlotte Russe}


\begin{minipage}{1.0\textwidth}
{\setlength{\multicolsep}{0pt}\setlength{\columnsep}{2em}\raggedcolumns%
\begin{multicols}{2}
\begin{itemize}
\setlength{\itemsep}{0pt}
\setlength{\parsep}{0pt}
\item 1/4 box gelatine or
\item 1 tablespoon granulated gelatine
\item 1/4 cup cold water
\item 1/2 cup scalded cream
\item 1/3 cup sugar, caramelized
\item 1/4 cup powdered sugar
\item 1 1/2 teaspoons vanilla
\item Whip from 3 1/2 cups cream
\item 6 lady fingers
\end{itemize}
\end{multicols}}
\end{minipage}

\vspace{0.3em}
\noindent%
Make same as Charlotte Russe, adding caramelized sugar to scalded cream
before putting into gelatine mixture.



\needspace{15\baselineskip}
\section*{Burnt Almond Charlotte}


\begin{minipage}{1.0\textwidth}
{\setlength{\multicolsep}{0pt}\setlength{\columnsep}{2em}\raggedcolumns%
\begin{multicols}{2}
\begin{itemize}
\setlength{\itemsep}{0pt}
\setlength{\parsep}{0pt}
\item 1/2 box gelatine or
\item 2 tablespoons granulated gelatine
\item 1/2 cup cold water
\item 3/4 cup scalded milk
\item 1/2 cup sugar
\item 1/2 cup sugar, caramelized
\item 3/4 cup blanched and finely chopped almonds
\item 1 teaspoon vanilla
\item Whip from 3 1/2 cups cream
\item 6 lady fingers
\end{itemize}
\end{multicols}}
\end{minipage}

\vspace{0.3em}
\noindent%
Make same as Caramel Charlotte Russe, adding nuts before folding in
cream.



\needspace{15\baselineskip}
\section*{Ginger Cream}


\begin{minipage}{1.0\textwidth}
{\setlength{\multicolsep}{0pt}\setlength{\columnsep}{2em}\raggedcolumns%
\begin{multicols}{2}
\begin{itemize}
\setlength{\itemsep}{0pt}
\setlength{\parsep}{0pt}
\item 1/4 box gelatine or
\item 1 tablespoon granulated gelatine
\item 1/4 cup cold water
\item 1 cup milk
\item 4 egg yolks
\item 1/4 cup sugar
\item Few grains salt
\item 1 tablespoon wine
\item 1/2 tablespoon brandy
\item 2 tablespoons ginger syrup
\item 1/4 cup Canton ginger, cut in pieces
\item Whip from 2 1/2 cups cream
\end{itemize}
\end{multicols}}
\end{minipage}

\vspace{0.3em}
\noindent%
Soak gelatine, and add to custard made of milk, eggs, sugar, and salt.
Strain, chill in pan of ice-water, add flavorings, and when it begins to
thicken fold in whip from cream.



\needspace{15\baselineskip}
\section*{Orange Charlotte}


\begin{minipage}{1.0\textwidth}
{\setlength{\multicolsep}{0pt}\setlength{\columnsep}{2em}\raggedcolumns%
\begin{multicols}{2}
\begin{itemize}
\setlength{\itemsep}{0pt}
\setlength{\parsep}{0pt}
\item 1/3 box gelatine or
\item 1 1/3 tablespoons granulated gelatine
\item 1/3 cup cold water
\item 1/3 cup boiling water
\item 1 cup sugar
\item 3 tablespoons lemon juice
\item 1 cup orange juice and pulp
\item 3 egg whites
\item Whip from 2 cups cream
\end{itemize}
\end{multicols}}
\end{minipage}

\vspace{0.3em}
\noindent%
Soak gelatine in cold water, dissolve in boiling water, strain, and add
sugar, lemon juice, orange juice, and pulp. Chill in pan of ice-water;
when quite thick, beat with wire spoon or whisk until frothy, then add
whites of eggs beaten stiff, and fold in cream. Line a mould with
sections of oranges, turn in mixture, smooth evenly, and chill.



\needspace{15\baselineskip}
\section*{Strawberry Sponge}


\begin{minipage}{1.0\textwidth}
{\setlength{\multicolsep}{0pt}\setlength{\columnsep}{2em}\raggedcolumns%
\begin{multicols}{2}
\begin{itemize}
\setlength{\itemsep}{0pt}
\setlength{\parsep}{0pt}
\item 1/3 box gelatine or
\item 1 1/3 tablespoons granulated gelatine
\item 1/3 cup cold water
\item 1/3 cup boiling water
\item 1 cup sugar
\item 1 tablespoon lemon juice
\item 1 cup strawberry juice
\item 3 egg whites
\item Whip from 3 cups cream
\end{itemize}
\end{multicols}}
\end{minipage}

\vspace{0.3em}
\noindent%
Make same as Orange Charlotte.



\needspace{15\baselineskip}
\section*{Orange Baskets}

Cut two pieces from each orange, leaving what remains in shape of basket
with handle, remove pulp from baskets and pieces, and keep baskets in
ice-water until ready to fill. From orange juice make orange jelly with
which to fill baskets. Serve garnished with Cream Sauce.



\needspace{15\baselineskip}
\section*{Orange Jelly In Ambush}

Cut oranges in halves lengthwise, remove pulp and juice. With juice make
Orange Jelly to fill half the pieces. Fill remaining pieces with
Charlotte Russe mixture. When both are firm, put together in pairs and
tie together with narrow white ribbon.



\needspace{15\baselineskip}
\section*{Bavarian Cream (Quick)}

                     1/2 lemon, grated rind and juice

\begin{itemize}
\setlength{\itemsep}{0pt}
\setlength{\parsep}{0pt}
\item 1/2 cup white wine
\item 1/3 cup sugar
\item 2 eggs
\item 1 teaspoon granulated gelatine
\item 1 tablespoon cold water
\end{itemize}

\vspace{-0.5em}
\noindent%
Mix lemon, wine, sugar, and yolks of eggs; stir vigorously over fire
until mixture thickens, add gelatine soaked in water, then pour over
whites of eggs beaten stiff. Set in pan of ice-water and beat until
thick enough to hold its shape. Turn into a mould lined with lady
fingers, and chill. Orange juice may be used in place of wine, and the
cream served in orange baskets.



\needspace{15\baselineskip}
\section*{Strawberry Bavarian Cream}

Line a mould with large, fresh strawberries cut in halves, fill with
Charlotte Russe mixture.



\needspace{15\baselineskip}
\section*{Pineapple Bavarian Cream}


\begin{minipage}{1.0\textwidth}
{\setlength{\multicolsep}{0pt}\setlength{\columnsep}{2em}\raggedcolumns%
\begin{multicols}{2}
\begin{itemize}
\setlength{\itemsep}{0pt}
\setlength{\parsep}{0pt}
\item 1/2 box gelatine or
\item 2 tablespoons granulated gelatine
\item 1/2 cup cold water
\item 1 can grated pineapple
\item 1/2 cup sugar
\item 1 tablespoon lemon juice
\item Whip from 3 cups cream
\end{itemize}
\end{multicols}}
\end{minipage}

\vspace{0.3em}
\noindent%
Soak gelatine in cold water. Heat pineapple, add sugar, lemon juice, and
soaked gelatine; chill in pan of ice-water, stirring constantly; when it
begins to thicken, fold in whip from cream, mould, and chill.



\needspace{15\baselineskip}
\section*{Royal Diplomatic Pudding}

Place mould in pan of ice-water and pour in Wine Jelly II one-half inch
deep. When firm, decorate with candied cherries and angelica, proceed as
for Fruit Chartreuse, filling the centre with Charlotte Russe mixture or
Fruit Cream.



\needspace{15\baselineskip}
\section*{Fruit Cream}

Peel four bananas, mash, and rub through a sieve; add pulp and juice of
two oranges, one tablespoon lemon juice, one tablespoon Sherry wine,
two-thirds cup powdered sugar, and one and one-fourth tablespoons
granulated gelatine dissolved in one-fourth cup boiling water. Cool in
ice-water, stirring constantly, and fold in whip from two cups cream.



\needspace{15\baselineskip}
\section*{Ivory Cream}


\begin{minipage}{1.0\textwidth}
{\setlength{\multicolsep}{0pt}\setlength{\columnsep}{2em}\raggedcolumns%
\begin{multicols}{2}
\begin{itemize}
\setlength{\itemsep}{0pt}
\setlength{\parsep}{0pt}
\item 3/4 tablespoon granulated gelatine
\item 1 tablespoon cold water
\item 2 tablespoons boiling water
\item 3 cups cream
\item 4 tablespoons powdered sugar
\item 3 tablespoons Madeira wine
\end{itemize}
\end{multicols}}
\end{minipage}

\vspace{0.3em}
\noindent%
Soak gelatine in cold water, dissolve in boiling water, and add sugar
and wine. Strain into a bowl, set in pan of ice-water, and beat until
mixture thickens slightly. Add to mixture whip from cream, and beat
until mixture is thick enough to hold its shape. Mould and chill.
Garnish with Sauterne Jelly.



\needspace{15\baselineskip}
\section*{Pudding À L'Adrea}


\begin{minipage}{1.0\textwidth}
{\setlength{\multicolsep}{0pt}\setlength{\columnsep}{2em}\raggedcolumns%
\begin{multicols}{2}
\begin{itemize}
\setlength{\itemsep}{0pt}
\setlength{\parsep}{0pt}
\item 2 cups thin cream
\item 1 1/2 tablespoons granulated gelatine
\item 2 tablespoons cold water
\item 3/4 cup sugar
\item 4 egg whitess
\item 3 tablespoons Sherry
\item 1 1/2 tablespoons Sauterne
\item Sauterne jelly mixture
\end{itemize}
\end{multicols}}
\end{minipage}

\vspace{0.3em}
\noindent%
Make one-half recipe for Sauterne Jelly (see p. 420), allowing one and
one-third tablespoons granulated gelatine. Color one-half green and
one-half red. Fill sections of a fancy mould alternately with green and
red jelly. In the green jelly mould pistachio nuts cut in quarters; in
red jelly glacéd cherries cut in quarters.

Scald cream, add gelatine soaked in cold water, then add whites of eggs
beaten until stiff; add sugar. Remove from range, set in pan of
ice-water, and stir occasionally until mixture thickens; then add
flavoring and turn into mould. Chill thoroughly and remove from mould.



\needspace{15\baselineskip}
\section*{French Easter Cream}


\begin{minipage}{1.0\textwidth}
{\setlength{\multicolsep}{0pt}\setlength{\columnsep}{2em}\raggedcolumns%
\begin{multicols}{2}
\begin{itemize}
\setlength{\itemsep}{0pt}
\setlength{\parsep}{0pt}
\item 1/3 cup raisins
\item 1/4 cup brandy
\item 2 cups cream
\item 1/2 cup sugar
\item 3 egg yolks
\item 1/8 teaspoon salt
\item 1 tablespoon granulated gelatine
\item 2 tablespoons cold water
\item 1/4 cup maraschino
\item 1/4 cup slow gin
\item 1/4 cup brandy
\item 1 teaspoon vanilla
\end{itemize}
\end{multicols}}
\end{minipage}

\vspace{0.3em}
\noindent%
Seed raisins, add brandy, and cook in double boiler until raisins are
soft. Make a custard of cream, sugar, egg yolks and salt. Remove from
range, add gelatine soaked in cold water. Strain, cool slightly, add
flavorings, stir until mixture thickens, then add raisins. Mould and
chill. Remove from mould, and garnish with Sauterne Jelly (colored
violet), cut in cubes, and fresh violets.



\needspace{15\baselineskip}
\section*{Marshmallow Pudding À La Stanley}


\begin{minipage}{1.0\textwidth}
{\setlength{\multicolsep}{0pt}\setlength{\columnsep}{2em}\raggedcolumns%
\begin{multicols}{2}
\begin{itemize}
\setlength{\itemsep}{0pt}
\setlength{\parsep}{0pt}
\item 1/2 pound marshmallows
\item 1 cup heavy cream
\item 1/2 teaspoon vanilla
\item 1/4 cup candied cherries
\item 1/2 cup English walnut meats
\item 2 tablespoons powdered sugar
\end{itemize}
\end{multicols}}
\end{minipage}

\vspace{0.3em}
\noindent%
Soak cherries in rum to cover one hour, then cut in pieces. Cut walnut
meats and marshmallows in small pieces. Whip cream, add sugar and
vanilla, fold in remaining ingredients. Mould and chill.





\chapter{Ices, Ice Creams, And Other Frozen Desserts}



Ices and other frozen dishes comprise the most popular desserts.
Hygienically speaking, they cannot be recommended for the final course
of a dinner, as cold mixtures reduce the temperature of the stomach,
thus retarding digestion until the normal temperature is again reached.
But how cooling, refreshing, and nourishing, when properly taken, and of
what inestimable value in the sick room!

Frozen dishes include:

\textit{Water ice},--fruit juice sweetened, diluted with water, and frozen.

\textit{Sherbet},--water ice to which is added a small quantity of dissolved
gelatine or beaten whites of eggs.

\textit{Frappé},--water ice frozen to consistency of mush; in freezing, equal
parts of salt and ice being used to make it granular.

\textit{Punch},--water ice to which is added spirit and spice.

\textit{Sorbet},--strictly speaking, frozen punch; the name is often given to a
water ice where several kinds of fruit are used.

\textit{Philadelphia Ice Cream},--thin cream, sweetened, flavored, and frozen.

\textit{Plain Ice Cream},--custard foundation, thin cream, and flavoring.

\textit{Mousse},--heavy cream, beaten until stiff, sweetened, flavored, placed
in a mould, packed in salt and ice (using two parts crushed ice to one
part salt), and allowed to stand three hours; or whip from thin cream
may be used folded into mixture containing small quantity of gelatine.



\needspace{15\baselineskip}
\section*{How To Freeze Desserts}

The prejudice of thinking a frozen dessert difficult to prepare has long
since been overcome. With ice cream freezer, burlap bag, wooden mallet
or axe, small saucepan, sufficient ice and coarse rock salt, the process
neither takes much time nor patience. Snow may be used instead of ice;
if not readily acted on by salt, pour in one cup cold water. Crush ice
finely by placing in bag and giving a few blows with mallet or broad
side of axe; if there are any coarse pieces, remove them. Place can
containing mixture to be frozen in wooden tub, cover, and adjust top.
Turn crank to make sure can fits in socket. Allow three level measures
ice to one of salt, and repeat until ice and salt come to top of can,
packing solidly, using handle of mallet to force it down. If only small
quantity is to be frozen, the ice and salt need come only a little
higher in the tub than mixture to be frozen. These are found the best
proportions of ice and salt to insure smooth, fine-grained cream,
sherbet, or water ice, while equal parts of salt and ice are used for
freezing frappé. If a larger proportion of salt is used, mixture will
freeze in shorter time and be of granular consistency, which is
desirable only for frappé.

The mixture increases in bulk during freezing, so the can should never
be more than three-fourths filled; by overcrowding can, cream will be
made coarse-grained. Turn the crank slowly and steadily to expose as
large surface of mixture as possible to ice and salt. After frozen to a
mush, the crank may be turned more rapidly, adding more ice and salt if
needed; never draw off salt water until mixture is frozen, unless there
is possibility of its getting into the can, for salt water is what
effects freezing; until ice melts, no change will take place. After
freezing is accomplished, draw off water, remove dasher, and with spoon
pack solidly. Put cork in opening of cover, then put on cover. Re-pack
freezer, using four measures ice to one of salt. Place over top
newspapers or piece of carpet; when serving time comes, remove can, wipe
carefully, and place in vessel of cool water; let stand one minute,
remove cover, and run a knife around edge of cream, invert can on
serving dish, and frozen mixture will slip out. Should there be any
difficulty, a cloth wrung out of hot water, passed over can, will aid in
removing mixture.



\needspace{15\baselineskip}
\section*{To Line A Mould}

Allow mould to stand in salt and ice until well chilled. Remove cover,
put in mixture by spoonfuls, and spread with back of spoon or a case
knife evenly three-quarters inch thick.



\needspace{15\baselineskip}
\section*{To Mould Frozen Mixtures}

When frozen mixtures are to be bricked or moulded, avoid freezing too
hard. Pack mixture solidly in moulds and cover with buttered paper,
buttered side up. Have moulds so well filled that mixture is forced down
sides of mould when cover is pressed down. Re-pack in salt and ice,
using four parts ice to one part salt. If these directions are carefully
followed, one may feel no fear that salt water will enter cream, even
though moulds be immersed in salt water.



\needspace{15\baselineskip}
\section*{Lemon Ice}


\begin{itemize}
\setlength{\itemsep}{0pt}
\setlength{\parsep}{0pt}
\item 4 cups water
\item 2 cups sugar
\item 3/4 cup lemon juice
\end{itemize}

\vspace{-0.5em}
\noindent%
Make a syrup by boiling water and sugar twenty minutes; add lemon juice;
cool, strain, and freeze. See directions for freezing, page 434.



\needspace{15\baselineskip}
\section*{Cup St. Jacques}

Serve Lemon Ice in champagne glasses. Put three-fourths teaspoon
Maraschino in each glass, and garnish with bananas cut in one-fourth
inch slices, and slices cut in quarters, candied cherries cut in halves,
Malaga grapes from which skins and seeds have been removed, and angelica
cut in strips.



\needspace{15\baselineskip}
\section*{Orange Ice}


\begin{itemize}
\setlength{\itemsep}{0pt}
\setlength{\parsep}{0pt}
\item 4 cups water
\item 2 cups sugar
\item 2 cups orange juice
\item 1/4 cup lemon juice
\item Grated rind of two oranges
\end{itemize}

\vspace{-0.5em}
\noindent%
Make syrup as for Lemon Ice; add fruit juice and grated rind; cool,
strain, and freeze.



\needspace{15\baselineskip}
\section*{Maraschino Ice}

Prepare Orange Ice mixture, freeze to a mush, flavor with Maraschino,
and finish freezing. Serve in frappé glasses.



\needspace{15\baselineskip}
\section*{Pomegranate Ice}

Same as Orange Ice, made from blood oranges.



\needspace{15\baselineskip}
\section*{Raspberry Ice I}


\begin{itemize}
\setlength{\itemsep}{0pt}
\setlength{\parsep}{0pt}
\item 4 cups water
\item 1 2/3 cups sugar
\item 2 cups raspberry juice
\item 2 tablespoons lemon juice
\end{itemize}

\vspace{-0.5em}
\noindent%
Make a syrup as for Lemon Ice, cool, add raspberries mashed, and
squeezed through double cheese-cloth, and lemon juice; strain and
freeze.



\needspace{15\baselineskip}
\section*{Raspberry Ice II}


\begin{itemize}
\setlength{\itemsep}{0pt}
\setlength{\parsep}{0pt}
\item 1 quart raspberries
\item 1 cup sugar
\item 1 cup water
\item Lemon juice
\end{itemize}

\vspace{-0.5em}
\noindent%
Sprinkle raspberries with sugar, cover, and let stand two hours. Mash,
squeeze through cheese-cloth, add water and lemon juice to taste, then
freeze. Raspberry ice prepared in this way retains the natural color of
the fruit.



\needspace{15\baselineskip}
\section*{Strawberry Ice I}


\begin{itemize}
\setlength{\itemsep}{0pt}
\setlength{\parsep}{0pt}
\item 4 cups water
\item 1 1/2 cups sugar
\item 2 cups strawberry juice
\item 1 tablespoon lemon juice
\end{itemize}

\vspace{-0.5em}
\noindent%
Prepare and freeze same as Raspberry Ice I.



\needspace{15\baselineskip}
\section*{Strawberry Ice II}


\begin{itemize}
\setlength{\itemsep}{0pt}
\setlength{\parsep}{0pt}
\item 1 quart box strawberries
\item 1 cup sugar
\item 1 cup water
\item Lemon juice
\end{itemize}

\vspace{-0.5em}
\noindent%
Make same as Raspberry Ice II.



\needspace{15\baselineskip}
\section*{Currant Ice}


\begin{itemize}
\setlength{\itemsep}{0pt}
\setlength{\parsep}{0pt}
\item 4 cups water
\item 1 1/2 cups sugar
\item 2 cups currant juice
\end{itemize}

\vspace{-0.5em}
\noindent%
Prepare and freeze same as Raspberry Ice I.



\needspace{15\baselineskip}
\section*{Raspberry And Currant Ice}


\begin{itemize}
\setlength{\itemsep}{0pt}
\setlength{\parsep}{0pt}
\item 4 cups water
\item 1 1/3 cups sugar
\item 2/3 cup raspberry juice
\item 1 1/3 cups currant juice
\end{itemize}

\vspace{-0.5em}
\noindent%
Prepare and freeze same as Raspberry Ice I.



\needspace{15\baselineskip}
\section*{Crême De Menthe Ice}


\begin{itemize}
\setlength{\itemsep}{0pt}
\setlength{\parsep}{0pt}
\item 4 cups water
\item 1 cup sugar
\item 1/3 cup Crême de Menthe cordial
\item Burnett's Leaf Green
\end{itemize}

\vspace{-0.5em}
\noindent%
Make a syrup as for Lemon Ice, add cordial and coloring; strain and
freeze.



\needspace{15\baselineskip}
\section*{Icebergs}

Dissolve two cups sugar in three cups boiling water; cool, add
three-fourths cup lemon juice, color with leaf green, and freeze. Serve
in champagne glasses. Put one teaspoon Crême de Menthe in each glass,
and sprinkle with finely chopped nut meats, using almonds, filberts,
pecans, and walnuts in equal proportions. These may be used after the
roast and before the game.



\needspace{15\baselineskip}
\section*{Canton Sherbet}


\begin{itemize}
\setlength{\itemsep}{0pt}
\setlength{\parsep}{0pt}
\item 4 cups water
\item 1 cup sugar
\item 1/4 lb. Canton ginger
\item 1/2 cup orange juice
\item 1/3 cup lemon juice
\end{itemize}

\vspace{-0.5em}
\noindent%
Cut ginger in small pieces, add water and sugar, boil fifteen minutes;
add fruit juice, cool, strain, and freeze. To be used in place of punch
at a course dinner. This quantity is enough to serve twelve persons.



\needspace{15\baselineskip}
\section*{Milk Sherbet}


\begin{itemize}
\setlength{\itemsep}{0pt}
\setlength{\parsep}{0pt}
\item 4 cups milk
\item 1 1/2 cups sugar
\item Juice 3 lemons
\end{itemize}

\vspace{-0.5em}
\noindent%
Mix juice and sugar, stirring constantly while slowly adding milk; if
added too rapidly mixture will have a curdled appearance, which is
unsightly, but will not affect the quality of sherbet; freeze and serve.



\needspace{15\baselineskip}
\section*{Frozen Chocolate With Whipped Cream}


\begin{itemize}
\setlength{\itemsep}{0pt}
\setlength{\parsep}{0pt}
\item 2 squares Baker's chocolate
\item 1 cup sugar
\item Few grains salt
\item 1 cup boiling water
\item 3 cups rich milk
\end{itemize}

\vspace{-0.5em}
\noindent%
Scald milk. Melt chocolate in small saucepan placed over hot water, add
one-half the sugar, salt, and gradually boiling water. Boil one minute,
add to scalded milk with remaining sugar. Cool, freeze, and serve in
glasses. Garnish with whipped cream sweetened and flavored with vanilla.



\needspace{15\baselineskip}
\section*{Pineapple Frappé}


\begin{minipage}{1.0\textwidth}
{\setlength{\multicolsep}{0pt}\setlength{\columnsep}{2em}\raggedcolumns%
\begin{multicols}{2}
\begin{itemize}
\setlength{\itemsep}{0pt}
\setlength{\parsep}{0pt}
\item 2 cups water
\item 1 cup sugar
\item Juice 3 lemons
\item 2 cups ice-water
\item 1 can grated pineapple or
\item 1 pineapple shredded
\end{itemize}
\end{multicols}}
\end{minipage}

\vspace{0.3em}
\noindent%
Make a syrup by boiling water and sugar fifteen minutes; add pineapple
and lemon juice; cool, strain, add ice-water, and freeze to a mush,
using equal parts ice and salt. If fresh fruit is used, more sugar will
be required.



\needspace{15\baselineskip}
\section*{Pineapple Sorbet}


\begin{minipage}{1.0\textwidth}
{\setlength{\multicolsep}{0pt}\setlength{\columnsep}{2em}\raggedcolumns%
\begin{multicols}{2}
\begin{itemize}
\setlength{\itemsep}{0pt}
\setlength{\parsep}{0pt}
\item 2 cups water
\item 2 cups sugar
\item 1 can grated pineapple or
\item 1 pineapple shredded
\item 1 1/3 cups orange juice
\item 1/2 cup lemon juice
\item 1 quart Apollinaris
\end{itemize}
\end{multicols}}
\end{minipage}

\vspace{0.3em}
\noindent%
Prepare and freeze same as Pineapple Frappé.



\needspace{15\baselineskip}
\section*{Sicilian Sorbet}


\begin{itemize}
\setlength{\itemsep}{0pt}
\setlength{\parsep}{0pt}
\item 1 can peaches
\item 1 cup sugar
\item 2 cups orange juice
\item 2 tablespoons lemon juice
\end{itemize}

\vspace{-0.5em}
\noindent%
Press peaches through a sieve, add sugar and fruit juices. Freeze and
serve.



\needspace{15\baselineskip}
\section*{Italian Sorbet}


\begin{minipage}{1.0\textwidth}
{\setlength{\multicolsep}{0pt}\setlength{\columnsep}{2em}\raggedcolumns%
\begin{multicols}{2}
\begin{itemize}
\setlength{\itemsep}{0pt}
\setlength{\parsep}{0pt}
\item 4 cups water
\item 2 cups sugar
\item 1 1/2 cups orange juice
\item 1 1/2 cups grape fruit juice
\item 1/2 cup lemon juice
\item 1/4 cup wine
\end{itemize}
\end{multicols}}
\end{minipage}

\vspace{0.3em}
\noindent%
Prepare and freeze same as Pineapple Frappé.



\needspace{15\baselineskip}
\section*{Apricot Sorbet}


\begin{itemize}
\setlength{\itemsep}{0pt}
\setlength{\parsep}{0pt}
\item 1 can apricots
\item 1 cup sugar
\item 1/2 cup wine
\item 1/4 cup lemon juice
\item 1 pint cream
\end{itemize}

\vspace{-0.5em}
\noindent%
Drain apricots, and add to syrup the pulp rubbed through a sieve. Add
sugar, wine, and lemon juice. Freeze to a mush, then fold in the whip
obtained from cream. Let stand one and one-half hours, and serve in
glasses.



\needspace{15\baselineskip}
\section*{Café Frappé}


\begin{itemize}
\setlength{\itemsep}{0pt}
\setlength{\parsep}{0pt}
\item 1 egg white
\item 1/2 cup cold water
\item 1/2 cup ground coffee
\item 4 cups boiling water
\item 1 cup sugar
\end{itemize}

\vspace{-0.5em}
\noindent%
Beat white of egg slightly, add cold water, and mix with coffee; turn
into scalded coffee-pot, add boiling water, and let boil one minute;
place on back of range ten minutes; strain, add sugar, cool, and freeze
same as Pineapple Frappé. Serve in frappé glasses, with whipped cream,
sweetened and flavored.



\needspace{15\baselineskip}
\section*{Cranberry Frappé}


\begin{itemize}
\setlength{\itemsep}{0pt}
\setlength{\parsep}{0pt}
\item 1 quart cranberries
\item 2 cups water
\item 2 cups sugar
\item Juice 2 lemons
\end{itemize}

\vspace{-0.5em}
\noindent%
Cook cranberries and water eight minutes; then force through a sieve.
Add sugar and lemon juice, and freeze to a mush, using equal parts of
ice and salt.



\needspace{15\baselineskip}
\section*{Grape Frappé}


\begin{itemize}
\setlength{\itemsep}{0pt}
\setlength{\parsep}{0pt}
\item 4 cups water
\item 2 cups sugar
\item 2 cups grape juice
\item 2/3 cup orange juice
\item 1/4 cup lemon juice
\end{itemize}

\vspace{-0.5em}
\noindent%
Prepare and freeze same as Pineapple Frappé.



\needspace{15\baselineskip}
\section*{Pomona Frappé}


\begin{itemize}
\setlength{\itemsep}{0pt}
\setlength{\parsep}{0pt}
\item 1 1/2 cups sugar
\item 4 cups water
\item 1 quart sweet cider
\item 2 cups orange juice
\item 1/2 cup lemon juice
\end{itemize}

\vspace{-0.5em}
\noindent%
Make a syrup by boiling sugar and water twenty minutes. Add cider,
orange juice, and lemon juice. Cool, strain, and freeze to a mush.



\needspace{15\baselineskip}
\section*{Clam Frappé}


\begin{itemize}
\setlength{\itemsep}{0pt}
\setlength{\parsep}{0pt}
\item 20 clams
\item 1/2 cup cold water
\end{itemize}

\vspace{-0.5em}
\noindent%
Wash clams thoroughly, changing water several times; put in stewpan with
cold water, cover closely, and steam until shells open. Strain the
liquor, cool, and freeze to a mush.



\needspace{15\baselineskip}
\section*{Frozen Cranberries}


\begin{itemize}
\setlength{\itemsep}{0pt}
\setlength{\parsep}{0pt}
\item 4 cups cranberries
\item 2 1/4 cups sugar
\item 1 1/2 cups boiling water
\end{itemize}

\vspace{-0.5em}
\noindent%
Pick over and wash cranberries, add water and sugar, and cook ten
minutes, skimming during the cooking. Rub through a sieve, cool, and
pour into one-pound baking-powder boxes. Pack in salt and ice, using
equal parts, and let stand four hours. If there is not sufficient
mixture to fill two boxes, add water to make up the desired quantity.
Serve as a substitute for cranberry sauce or jelly.



\needspace{15\baselineskip}
\section*{Frozen Apricots}


\begin{itemize}
\setlength{\itemsep}{0pt}
\setlength{\parsep}{0pt}
\item 1 can apricots
\item 1 1/2 cups sugar
\item Water
\end{itemize}

\vspace{-0.5em}
\noindent%
Drain apricots, and cut in small pieces. To the syrup add enough water
to make four cups, and cook with sugar five minutes; strain, add
apricots, cool, and freeze. Peaches may be used instead of apricots. To
make a richer dessert, add the whip from two cups cream when frozen to a
mush, and continue freezing.



\needspace{15\baselineskip}
\section*{Pineapple Cream}


\begin{itemize}
\setlength{\itemsep}{0pt}
\setlength{\parsep}{0pt}
\item 2 cups water
\item 1 cup sugar
\item 1 can grated pineapple
\item 2 cups cream
\end{itemize}

\vspace{-0.5em}
\noindent%
Make syrup by boiling sugar and water fifteen minutes; strain, cool, add
pineapple, and freeze to a mush. Fold in whip from cream; let stand
thirty minutes before serving. Serve in frappé glasses and garnish with
candied pineapple.



\needspace{15\baselineskip}
\section*{Cardinal Punch}


\begin{minipage}{1.0\textwidth}
{\setlength{\multicolsep}{0pt}\setlength{\columnsep}{2em}\raggedcolumns%
\begin{multicols}{2}
\begin{itemize}
\setlength{\itemsep}{0pt}
\setlength{\parsep}{0pt}
\item 4 cups water
\item 2 cups sugar
\item 2/3 cup orange juice
\item 1/3 cup lemon juice
\item 1/4 cup brandy
\item 1/4 cup Curaçoa
\item 1/4 cup tea infusion
\end{itemize}
\end{multicols}}
\end{minipage}

\vspace{0.3em}
\noindent%
Make syrup as for Lemon Ice, add fruit juice and tea, freeze to a mush;
add strong liquors and continue freezing. Serve in frappé glasses.



\needspace{15\baselineskip}
\section*{Punch Hollandaise}


\begin{minipage}{1.0\textwidth}
{\setlength{\multicolsep}{0pt}\setlength{\columnsep}{2em}\raggedcolumns%
\begin{multicols}{2}
\begin{itemize}
\setlength{\itemsep}{0pt}
\setlength{\parsep}{0pt}
\item 4 cups water
\item 1 1/3 cups sugar
\item 1/3 cup lemon juice
\item Rind one lemon
\item 1 can grated pineapple
\item 1/4 cup brandy
\item 2 tablespoons gin
\end{itemize}
\end{multicols}}
\end{minipage}

\vspace{0.3em}
\noindent%
Cook sugar, water, and lemon rind fifteen minutes, add lemon juice and
pineapple, cool, strain, freeze to a mush, add strong liquors, and
continue freezing. Serve in frappé glasses on a plate covered with a
doily.



\needspace{15\baselineskip}
\section*{Victoria Punch}


\begin{minipage}{1.0\textwidth}
{\setlength{\multicolsep}{0pt}\setlength{\columnsep}{2em}\raggedcolumns%
\begin{multicols}{2}
\begin{itemize}
\setlength{\itemsep}{0pt}
\setlength{\parsep}{0pt}
\item 3 1/2 cups water
\item 2 cups sugar
\item 1/2 cup lemon juice
\item 1/2 cup orange juice
\item Grated rind two oranges
\item 1 cup angelica wine
\item 1 cup cider
\item 1 1/2 tablespoons gin
\end{itemize}
\end{multicols}}
\end{minipage}

\vspace{0.3em}
\noindent%
Prepare same as Cardinal Punch; strain before freezing, to remove orange
rind.



\needspace{15\baselineskip}
\section*{Lenox Punch}


\begin{minipage}{1.0\textwidth}
{\setlength{\multicolsep}{0pt}\setlength{\columnsep}{2em}\raggedcolumns%
\begin{multicols}{2}
\begin{itemize}
\setlength{\itemsep}{0pt}
\setlength{\parsep}{0pt}
\item 2 cups water
\item 3/4 cup sugar
\item 2/3 tumbler currant jelly
\item Ice
\item 1 cup orange juice
\item 1/2 cup lemon juice
\item 2 bottles ginger ale
\item 1/3 cup brandy
\end{itemize}
\end{multicols}}
\end{minipage}

\vspace{0.3em}
\noindent%
Make a syrup by boiling sugar and water fifteen minutes. Add jelly, and,
as soon as dissolved, add a piece of ice to cool mixture; then add fruit
juices, ale, and brandy. Color red, freeze to a mush, serve in glasses,
and insert in each glass a small sprig of holly with berries.



\needspace{15\baselineskip}
\section*{German Punch}


\begin{minipage}{1.0\textwidth}
{\setlength{\multicolsep}{0pt}\setlength{\columnsep}{2em}\raggedcolumns%
\begin{multicols}{2}
\begin{itemize}
\setlength{\itemsep}{0pt}
\setlength{\parsep}{0pt}
\item 2 cups water
\item 1 3/4 cups tomatoes
\item 3 apples, cored, pared, and chopped
\item 1 cup sugar
\item 3 tablespoons lemon juice
\item Piece ginger root
\item 3 tablespoons Maraschino
\end{itemize}
\end{multicols}}
\end{minipage}

\vspace{0.3em}
\noindent%
Mix ingredients, except cordial, and cook thirty-five minutes. Rub
through a sieve, add Maraschino, and freeze to a mush.



\needspace{15\baselineskip}
\section*{London Sherbet}


\begin{minipage}{1.0\textwidth}
{\setlength{\multicolsep}{0pt}\setlength{\columnsep}{2em}\raggedcolumns%
\begin{multicols}{2}
\begin{itemize}
\setlength{\itemsep}{0pt}
\setlength{\parsep}{0pt}
\item 2 cups sugar
\item 2 cups water
\item 1/3 cup seeded and finely cut raisins
\item 3/4 cup orange juice
\item 3 tablespoons lemon juice
\item 1 cup fruit syrup
\item 1/4 grated nutmeg
\item 1/4 cup port wine
\item 3 egg whites
\end{itemize}
\end{multicols}}
\end{minipage}

\vspace{0.3em}
\noindent%
Make syrup by boiling water and sugar ten minutes; pour over raisins,
cool, and add fruit syrup and nutmeg; freeze to a mush, then add wine
and whites of eggs beaten stiff, and continue freezing. Serve in
glasses. Fruit syrup may be used which has been left from canned
peaches, pears, or strawberries.



\needspace{15\baselineskip}
\section*{Roman Punch}


\begin{minipage}{1.0\textwidth}
{\setlength{\multicolsep}{0pt}\setlength{\columnsep}{2em}\raggedcolumns%
\begin{multicols}{2}
\begin{itemize}
\setlength{\itemsep}{0pt}
\setlength{\parsep}{0pt}
\item 4 cups water
\item 2 cups sugar
\item 1/2 cup lemon juice
\item 1/2 cup orange juice
\item 1/2 cup tea infusion
\item 1/2 cup rum
\end{itemize}
\end{multicols}}
\end{minipage}

\vspace{0.3em}
\noindent%
Prepare and freeze same as Cardinal Punch.



\needspace{15\baselineskip}
\section*{Coup Sicilienne}


\begin{minipage}{1.0\textwidth}
{\setlength{\multicolsep}{0pt}\setlength{\columnsep}{2em}\raggedcolumns%
\begin{multicols}{2}
\begin{itemize}
\setlength{\itemsep}{0pt}
\setlength{\parsep}{0pt}
\item 1 shredded pineapple
\item 3 oranges (pulp)
\item 3 bananas sliced
\item 2 tablespoons Maraschino
\item 1 tablespoon lemon juice
\item Few grains salt
\item Powdered sugar
\end{itemize}
\end{multicols}}
\end{minipage}

\vspace{0.3em}
\noindent%
Mix ingredients, sweeten to taste, and chill. Serve in champagne glasses
having glasses two-thirds full. Cover fruit to fill glasses with
Strawberry Ice II and garnish with strawberries and angelica.



\needspace{15\baselineskip}
\section*{Coup A L'Ananas}

Cut canned sliced pineapple in pieces, pour over pineapple syrup to
which is added Orange Curaçoa, allowing one-half as much syrup as fruit,
cover and let stand one hour. Fill champagne glasses one-third full, add
vanilla ice cream to fill glasses, and garnish with candied cherries and
candied pineapple cut in pieces.



\needspace{15\baselineskip}
\section*{Vanilla Ice Cream I (Philadelphia)}


\begin{itemize}
\setlength{\itemsep}{0pt}
\setlength{\parsep}{0pt}
\item 1 quart thin cream
\item 3/4 cup sugar
\item 1 1/2 tablespoons vanilla
\end{itemize}

\vspace{-0.5em}
\noindent%
Mix ingredients, and freeze.







\needspace{15\baselineskip}
\section*{Vanilla Ice Cream II}


\begin{minipage}{1.0\textwidth}
{\setlength{\multicolsep}{0pt}\setlength{\columnsep}{2em}\raggedcolumns%
\begin{multicols}{2}
\begin{itemize}
\setlength{\itemsep}{0pt}
\setlength{\parsep}{0pt}
\item 2 cups scalded milk
\item 1 tablespoon flour
\item 1 cup sugar
\item 1 egg
\item 1/8 teaspoon salt
\item 1 quart thin cream
\item 2 tablespoons vanilla
\end{itemize}
\end{multicols}}
\end{minipage}

\vspace{0.3em}
\noindent%
Mix flour, sugar, and salt, add egg slightly beaten, and milk gradually;
cook over hot water twenty minutes, stirring constantly at first; should
custard have curdled appearance, it will disappear in freezing. When
cool, add cream and flavoring; strain and freeze.



\needspace{15\baselineskip}
\section*{Chocolate Sauce I}

(To be served with Vanilla Ice Cream)


\begin{minipage}{1.0\textwidth}
{\setlength{\multicolsep}{0pt}\setlength{\columnsep}{2em}\raggedcolumns%
\begin{multicols}{2}
\begin{itemize}
\setlength{\itemsep}{0pt}
\setlength{\parsep}{0pt}
\item 1 1/2 cups water
\item 1/2 cup sugar
\item 6 tablespoons grated chocolate
\item 1 tablespoon arrowroot
\item 1/2 cup cold water
\item Few grains salt
\item 1/2 teaspoon vanilla
\end{itemize}
\end{multicols}}
\end{minipage}

\vspace{0.3em}
\noindent%
Boil water and sugar five minutes. Mix chocolate with arrowroot to which
water has been added. Combine mixtures, add salt, and boil three
minutes. Flavor with vanilla, and serve hot.



\needspace{15\baselineskip}
\section*{Chocolate Sauce II}


\begin{itemize}
\setlength{\itemsep}{0pt}
\setlength{\parsep}{0pt}
\item 1 square Baker's chocolate
\item 1 cup sugar
\item 1 tablespoon butter
\item 1/3 cup water
\item 1/2 teaspoon vanilla
\end{itemize}

\vspace{-0.5em}
\noindent%
Melt chocolate; add butter, sugar, and water. Let boil fifteen minutes,
cool slightly, and add vanilla.



\needspace{15\baselineskip}
\section*{Coffee Sauce}

(To be served with Vanilla Ice Cream)


\begin{itemize}
\setlength{\itemsep}{0pt}
\setlength{\parsep}{0pt}
\item 1 1/2 cups milk
\item 1/2 cup ground coffee
\item 1/3 cup sugar
\item 3/4 tablespoon arrowroot
\item Few grains salt
\end{itemize}

\vspace{-0.5em}
\noindent%
Scald milk with coffee, and let stand twenty minutes. Mix remaining
ingredients, and pour on gradually the hot infusion which has been
strained. Cook five minutes, and serve hot.



\needspace{15\baselineskip}
\section*{Vanilla Ice Cream Croquettes}

Shape Vanilla Ice Cream in individual moulds, roll in macaroon dust made
by pounding and sifting dry macaroons.



\needspace{15\baselineskip}
\section*{Chocolate Ice Cream I}


\begin{minipage}{1.0\textwidth}
{\setlength{\multicolsep}{0pt}\setlength{\columnsep}{2em}\raggedcolumns%
\begin{multicols}{2}
\begin{itemize}
\setlength{\itemsep}{0pt}
\setlength{\parsep}{0pt}
\item 1 quart thin cream
\item 1 cup sugar
\item Few grains salt
\item 1 1/2 squares Baker's chocolate or
\item 1/4 cup prepared cocoa
\item 1 tablespoon vanilla
\end{itemize}
\end{multicols}}
\end{minipage}

\vspace{0.3em}
\noindent%
Melt chocolate, and dilute with hot water to pour easily, add to cream;
then add sugar, salt, and flavoring, and freeze.



\needspace{15\baselineskip}
\section*{Chocolate Ice Cream II}

Use recipe for Vanilla Ice Cream II. Melt two squares Baker's chocolate,
by placing in a small saucepan set in a larger saucepan of boiling
water, and pour hot custard slowly on chocolate; then cool before adding
cream.



\needspace{15\baselineskip}
\section*{Strawberry Ice Cream I}


\begin{itemize}
\setlength{\itemsep}{0pt}
\setlength{\parsep}{0pt}
\item 3 pints thin cream
\item 2 boxes berries
\item 2 cups sugar
\item Few grains salt
\end{itemize}

\vspace{-0.5em}
\noindent%
Wash and hull berries, sprinkle with sugar, cover, and let stand two
hours. Mash, and squeeze through cheese-cloth; then add salt. Freeze
cream to the consistency of a mush, add gradually fruit juice, and
finish freezing. Rich Jersey milk may be substituted for cream.



\needspace{15\baselineskip}
\section*{Strawberry Ice Cream II}


\begin{itemize}
\setlength{\itemsep}{0pt}
\setlength{\parsep}{0pt}
\item 3 pints thin cream
\item 2 boxes strawberries
\item 1 3/4 cups sugar
\item 2 cups milk
\item 1 1/2 tablespoons arrowroot
\end{itemize}

\vspace{-0.5em}
\noindent%
Wash and hull berries, sprinkle with sugar, let stand one hour, mash,
and rub through strainer. Scald one and one-half cups milk; dilute
arrowroot with remaining milk, add to hot milk, and cook ten minutes in
double boiler; cool, add cream, freeze to a mush, add fruit, and finish
freezing.



\needspace{15\baselineskip}
\section*{Orange Ice Cream}


\begin{itemize}
\setlength{\itemsep}{0pt}
\setlength{\parsep}{0pt}
\item 1 cup heavy cream
\item 1 cup thin cream
\item 2 cups orange juice
\item Sugar
\end{itemize}

\vspace{-0.5em}
\noindent%
Add cream slowly to orange juice, sweeten to taste, and freeze. Serve
with canned strawberries or fresh fruit mashed and sweetened.



\needspace{15\baselineskip}
\section*{Pineapple Ice Cream}


\begin{itemize}
\setlength{\itemsep}{0pt}
\setlength{\parsep}{0pt}
\item 3 pints cream
\item 1/2 cup sugar
\item 1 can grated pineapple
\end{itemize}

\vspace{-0.5em}
\noindent%
Add pineapple to cream, let stand thirty minutes; strain, add sugar, and
freeze.



\needspace{15\baselineskip}
\section*{Coffee Ice Cream}


\begin{minipage}{1.0\textwidth}
{\setlength{\multicolsep}{0pt}\setlength{\columnsep}{2em}\raggedcolumns%
\begin{multicols}{2}
\begin{itemize}
\setlength{\itemsep}{0pt}
\setlength{\parsep}{0pt}
\item 1 quart cream
\item 1 1/2 cups milk
\item 1/3 cup Mocha coffee
\item 1 1/4 cups sugar
\item 1/4 teaspoon salt
\item 4 egg yolks
\end{itemize}
\end{multicols}}
\end{minipage}

\vspace{0.3em}
\noindent%
Scald milk with coffee, add one cup sugar; mix egg yolks slightly beaten
with one-fourth cup sugar, and salt; combine mixtures, cook over hot
water until thickened, add one cup cream, and let stand on back of range
twenty-five minutes; cool, add remaining cream, and strain through
double cheese-cloth; freeze. Coffee Ice Cream may be served with
Maraschino cherries or in halves of cantaloupes.



\needspace{15\baselineskip}
\section*{Caramel Ice Cream}


\begin{minipage}{1.0\textwidth}
{\setlength{\multicolsep}{0pt}\setlength{\columnsep}{2em}\raggedcolumns%
\begin{multicols}{2}
\begin{itemize}
\setlength{\itemsep}{0pt}
\setlength{\parsep}{0pt}
\item 1 quart cream
\item 2 cups milk
\item 1 1/3 cups sugar
\item 1 egg
\item 1 tablespoon flour
\item 1/8 teaspoon salt
\item 1 1/2 tablespoons vanilla
\end{itemize}
\end{multicols}}
\end{minipage}

\vspace{0.3em}
\noindent%
Prepare same as Vanilla Ice Cream II, using one-half sugar in custard;
remaining half caramelize, and add slowly to hot custard. See
Caramelization of Sugar, page 586.



\needspace{15\baselineskip}
\section*{Burnt Almond Ice Cream}

It is made same as Caramel Ice Cream, with the addition of one cup
finely chopped blanched almonds.



\needspace{15\baselineskip}
\section*{Brown Bread Ice Cream}


\begin{itemize}
\setlength{\itemsep}{0pt}
\setlength{\parsep}{0pt}
\item 3 pints cream
\item 1 1/4 cups dried brown bread crumbs
\item 7/8 cup sugar
\item 1/4 teaspoon salt
\end{itemize}

\vspace{-0.5em}
\noindent%
Soak crumbs in one quart cream, let stand fifteen minutes, rub through
sieve, add sugar, salt, and remaining cream; then freeze.



\needspace{15\baselineskip}
\section*{Bisque Ice Cream}

Make custard as for Vanilla Ice Cream II, add one quart cream, one
tablespoon vanilla, and one cup hickory nut or English walnut meats
finely chopped.



\needspace{15\baselineskip}
\section*{Burnt Walnut Bisque}


\begin{minipage}{1.0\textwidth}
{\setlength{\multicolsep}{0pt}\setlength{\columnsep}{2em}\raggedcolumns%
\begin{multicols}{2}
\begin{itemize}
\setlength{\itemsep}{0pt}
\setlength{\parsep}{0pt}
\item 2 cups scalded milk
\item 3 egg yolks
\item 1 cup sugar
\item 2/3 cup chopped walnut meats
\item 1 cup heavy cream
\item 3/4 tablespoon vanilla
\item Few grains salt
\end{itemize}
\end{multicols}}
\end{minipage}

\vspace{0.3em}
\noindent%
Make custard of milk, eggs, one-third of the sugar, and salt. Caramelize
remaining sugar, add nut meats, and turn into a slightly buttered pan.
Cool, pound, and pass through a purée strainer. Add to custard, cool,
then add one cup heavy cream, beaten until stiff, and vanilla. Freeze
and mould.



\needspace{15\baselineskip}
\section*{Praline Ice Cream}


\begin{itemize}
\setlength{\itemsep}{0pt}
\setlength{\parsep}{0pt}
\item 3 pints cream
\item 1 1/3 cups sugar
\item 1 cup Jordan almonds
\item 1/4 teaspoon salt
\item 1 tablespoon vanilla
\end{itemize}

\vspace{-0.5em}
\noindent%
Blanch almonds cut in pieces crosswise, and bake in a shallow pan until
well browned, shaking pan frequently; then finely chop. Caramelize
one-half of the sugar, and add slowly to two cups of the cream scalded.
As soon as sugar is melted, add nuts, remaining sugar, and salt. Cool,
add remaining cream, and freeze. A few grains salt is always an
improvement to any ice cream mixture.



\needspace{15\baselineskip}
\section*{Macaroon Ice Cream}


\begin{itemize}
\setlength{\itemsep}{0pt}
\setlength{\parsep}{0pt}
\item 1 quart cream
\item 1 cup macaroons
\item 3/4 cup sugar
\item 1 tablespoon vanilla
\end{itemize}

\vspace{-0.5em}
\noindent%
Dry, pound, and measure macaroons; add to cream, sugar, and vanilla,
then freeze.



\needspace{15\baselineskip}
\section*{Banana Ice Cream}


\begin{itemize}
\setlength{\itemsep}{0pt}
\setlength{\parsep}{0pt}
\item 1 quart cream
\item 4 bananas
\item 1 1/3 tablespoons lemon juice
\item 1 cup sugar
\item A few grains salt
\end{itemize}

\vspace{-0.5em}
\noindent%
Remove skins and scrape bananas, then force through a sieve; add
remaining ingredients; then freeze.



\needspace{15\baselineskip}
\section*{Ginger Ice Cream}

To recipe for Vanilla Ice Cream II, using one-half quantity vanilla, add
one-half cup Canton ginger cut in small pieces, three tablespoons ginger
syrup, and two tablespoons Sherry wine; then freeze.



\needspace{15\baselineskip}
\section*{Pistachio Ice Cream}

Prepare same as Vanilla Ice Cream II, using for flavoring one tablespoon
vanilla and one teaspoon almond extract; color with Burnett's Leaf
Green.



\needspace{15\baselineskip}
\section*{Pistachio Bisque}

To Pistachio Ice Cream add one-half cup each of pounded macaroons,
chopped almonds, and peanuts. Mould, and serve with or without Claret
Sauce.



\needspace{15\baselineskip}
\section*{Fig Ice Cream}


\begin{minipage}{1.0\textwidth}
{\setlength{\multicolsep}{0pt}\setlength{\columnsep}{2em}\raggedcolumns%
\begin{multicols}{2}
\begin{itemize}
\setlength{\itemsep}{0pt}
\setlength{\parsep}{0pt}
\item 3 cups milk
\item 1 cup sugar
\item 5 egg yolks
\item 1 teaspoon salt
\item 1 lb. figs, finely chopped
\item 1 1/2 cups heavy cream
\item 5 egg whitess
\item 1 tablespoon vanilla
\item 2 tablespoons brandy
\end{itemize}
\end{multicols}}
\end{minipage}

\vspace{0.3em}
\noindent%
Make custard of yolks of eggs, sugar, and milk; strain, add figs, cool,
and flavor. Add whites of eggs beaten until stiff and heavy cream beaten
until stiff; freeze and mould.



\needspace{15\baselineskip}
\section*{Junket Ice Cream With Peaches}


\begin{minipage}{1.0\textwidth}
{\setlength{\multicolsep}{0pt}\setlength{\columnsep}{2em}\raggedcolumns%
\begin{multicols}{2}
\begin{itemize}
\setlength{\itemsep}{0pt}
\setlength{\parsep}{0pt}
\item 4 cups lukewarm milk
\item 1 cup heavy cream
\item 1 1/4 cups sugar
\item 1/8 teaspoon salt
\item 1 1/2 Junket Tablets
\item 1 tablespoon cold water
\item 1 tablespoon vanilla
\item 1 teaspoon almond extract
\item Green Coloring
\item 1 can peaches
\end{itemize}
\end{multicols}}
\end{minipage}

\vspace{0.3em}
\noindent%
Mix first four ingredients, and add junket tablets dissolved in cold
water. Turn into a pudding-dish and let stand until set. Add flavoring
and coloring. Freeze, mould, and serve garnished with halves of peaches,
filling cavities with halves of blanched almonds. Turn peaches into a
saucepan, add one-third cup sugar, and cook slowly until syrup is thick.
Cool before garnishing ice cream.



\needspace{15\baselineskip}
\section*{Violet Ice Cream}


\begin{minipage}{1.0\textwidth}
{\setlength{\multicolsep}{0pt}\setlength{\columnsep}{2em}\raggedcolumns%
\begin{multicols}{2}
\begin{itemize}
\setlength{\itemsep}{0pt}
\setlength{\parsep}{0pt}
\item 1 quart cream
\item 3/4 cup sugar
\item Few grains salt
\item 1/3 cup Yvette Cordial
\item 1 small bunch violets
\item Violet coloring
\end{itemize}
\end{multicols}}
\end{minipage}

\vspace{0.3em}
\noindent%
Mix first four ingredients. Remove stems from violets, and pound violets
in a mortar until well macerated, then strain through cheese-cloth. Add
extract to first mixture; color, freeze, and mould. Serve garnished with
fresh or candied violets; the light purple cultivated violets should be
used and the result will be most gratifying.



\needspace{15\baselineskip}
\section*{Neapolitan Or Harlequin Ice Cream}

Two kinds of ice cream and an ice moulded in a brick.



\needspace{15\baselineskip}
\section*{Baked Alaska}


\begin{itemize}
\setlength{\itemsep}{0pt}
\setlength{\parsep}{0pt}
\item 6 egg whitess
\item 6 tablespoons powdered sugar
\item 2 quart brick of ice cream
\item Thin sheet sponge cake
\end{itemize}

\vspace{-0.5em}
\noindent%
Make meringue of eggs and sugar as in Meringue I, cover a board with
white paper, lay on sponge cake, turn ice cream on cake (which should
extend one-half inch beyond cream), cover with meringue, and spread
smoothly. Place on oven grate and brown quickly in hot oven. The board,
paper, cake, and meringue are poor conductors of heat, and prevent the
cream from melting. Slip from paper on ice cream platter.



\needspace{15\baselineskip}
\section*{Pudding Glacé}


\begin{minipage}{1.0\textwidth}
{\setlength{\multicolsep}{0pt}\setlength{\columnsep}{2em}\raggedcolumns%
\begin{multicols}{2}
\begin{itemize}
\setlength{\itemsep}{0pt}
\setlength{\parsep}{0pt}
\item 2 cups milk
\item 2/3 cup raisins
\item 1 cup sugar
\item 1 egg
\item 1 tablespoon flour
\item 1/4 teaspoon salt
\item 1 quart thin cream
\item 1/2 cup almonds
\item 1/2 cup candied pineapple
\item 1/3 cup Canton ginger
\item 3 tablespoons wine
\end{itemize}
\end{multicols}}
\end{minipage}

\vspace{0.3em}
\noindent%
Scald raisins in milk fifteen minutes, strain, make custard of milk,
egg, sugar, flour, and salt; strain, cool, add pineapple, ginger cut in
small pieces, nuts finely chopped, wine, and cream; then freeze. The
raisins should be rinsed and saved for a pudding.



\needspace{15\baselineskip}
\section*{Frozen Pudding I}


\begin{minipage}{1.0\textwidth}
{\setlength{\multicolsep}{0pt}\setlength{\columnsep}{2em}\raggedcolumns%
\begin{multicols}{2}
\begin{itemize}
\setlength{\itemsep}{0pt}
\setlength{\parsep}{0pt}
\item 2 1/2 cups milk
\item 1 cup sugar
\item 1/8 teaspoonful salt
\item 2 eggs
\item 1 cup heavy cream
\item 1/4 cup rum
\item 1 cup candied fruit, cherries, pineapples, pears, and apricots
\end{itemize}
\end{multicols}}
\end{minipage}

\vspace{0.3em}
\noindent%
Cut fruit in small pieces, and soak two or three hours in brandy to
cover, which prevents fruit from freezing; make a custard of milk,
sugar, salt, and eggs; strain, cool, add cream and rum, then freeze.
Fill a brick mould with alternate layers of the cream and fruit; pack in
salt and ice and let stand two hours.



\needspace{15\baselineskip}
\section*{Frozen Pudding II}


\begin{itemize}
\setlength{\itemsep}{0pt}
\setlength{\parsep}{0pt}
\item 1 quart cream
\item 3/4 cup sugar
\item 1/4 cup rum
\item 1 cup candied fruit
\item 8 lady fingers
\end{itemize}

\vspace{-0.5em}
\noindent%
Cut fruit in pieces, and soak several hours in brandy to cover. Mix
cream, sugar, and rum, then freeze. Line a two-quart melon mould with
lady fingers, crust side down; fill with alternate layers of the cream
and fruit, cover, pack in salt and ice, and let stand two hours.
Brandied peaches cut in pieces, with some of their syrup added, greatly
improve the pudding.



\needspace{15\baselineskip}
\section*{Frozen Tom And Jerry}


\begin{minipage}{1.0\textwidth}
{\setlength{\multicolsep}{0pt}\setlength{\columnsep}{2em}\raggedcolumns%
\begin{multicols}{2}
\begin{itemize}
\setlength{\itemsep}{0pt}
\setlength{\parsep}{0pt}
\item 2 cups milk
\item 3/4 cup sugar
\item 6 egg yolks
\item 1/8 teaspoon salt
\item 2 1/2 cups cream
\item 2 tablespoons rum
\item 1 tablespoon brandy
\end{itemize}
\end{multicols}}
\end{minipage}

\vspace{0.3em}
\noindent%
Make a custard of first four ingredients; strain, cool, add cream, and
freeze to a mush. Add rum and brandy, and finish the freezing.



\needspace{15\baselineskip}
\section*{University Pudding}

Prepare same as Frozen Tom and Jerry. Freeze to a mush, add one cup
mixed fruit which has been soaked in brandy to cover for twelve hours,
using glacé cherries, Sultana raisins, sliced citron, and candied
pineapple; then finish freezing. Serve in small beer jugs, and garnish
with cream, whipped, sweetened, and flavored.



\needspace{15\baselineskip}
\section*{Covington Cream}


\begin{itemize}
\setlength{\itemsep}{0pt}
\setlength{\parsep}{0pt}
\item 3/4 cup sugar
\item 1/2 cup Formosa tea infusion
\item 1/3 cup rum
\item 1 quart cream
\end{itemize}

\vspace{-0.5em}
\noindent%
Mix ingredients, and freeze to a mush. Serve in frappé glasses.



\needspace{15\baselineskip}
\section*{Delmonico Ice Cream With Angel Food}


\begin{minipage}{1.0\textwidth}
{\setlength{\multicolsep}{0pt}\setlength{\columnsep}{2em}\raggedcolumns%
\begin{multicols}{2}
\begin{itemize}
\setlength{\itemsep}{0pt}
\setlength{\parsep}{0pt}
\item 2 cups milk
\item 3/4 cup sugar
\item Yolks 7 eggs
\item 1/8 teaspoon salt
\item 2 1/2 cups thin cream
\item 1 tablespoon vanilla
\item 1 teaspoon lemon
\end{itemize}
\end{multicols}}
\end{minipage}

\vspace{0.3em}
\noindent%
Make custard of milk, sugar, eggs, and salt; cool, strain, and flavor;
whip cream, remove whip; there should be two quarts; add to custard, and
freeze. Serve plain or with Angel Food.



\needspace{15\baselineskip}
\section*{Angel Food}


\begin{itemize}
\setlength{\itemsep}{0pt}
\setlength{\parsep}{0pt}
\item 3 egg whites
\item 1/2 cup powdered sugar
\item 1 quart cream whip
\item 1 1/2 teaspoons vanilla
\end{itemize}

\vspace{-0.5em}
\noindent%
Beat eggs until stiff, fold in sugar, cream whip, and flavoring; line a
mould with Delmonico Ice Cream, fill with the mixture, cover, pack in
salt and ice, and let stand two hours.



\needspace{15\baselineskip}
\section*{Manhattan Pudding}


\begin{minipage}{1.0\textwidth}
{\setlength{\multicolsep}{0pt}\setlength{\columnsep}{2em}\raggedcolumns%
\begin{multicols}{2}
\begin{itemize}
\setlength{\itemsep}{0pt}
\setlength{\parsep}{0pt}
\item 1 1/2 cups orange juice
\item 1/4 cup lemon juice
\item Sugar
\item 1 pint heavy cream
\item 1/2 cup powdered sugar
\item 1/2 tablespoon vanilla
\item 2/3 cup chopped walnut meats
\end{itemize}
\end{multicols}}
\end{minipage}

\vspace{0.3em}
\noindent%
Mix fruit juices and sweeten to taste. Turn mixture in brick mould. Whip
cream, and add sugar, vanilla, and nut meats; pour over the first
mixture to overflow mould; cover with buttered paper, fit on cover, pack
in salt and ice, and let stand three hours.



\needspace{15\baselineskip}
\section*{Sultana Roll With Claret Sauce}

Line one-pound baking-powder boxes with Pistachio Ice Cream; sprinkle
with Sultana raisins which have been soaked one hour in brandy; fill
centres with Vanilla Ice Cream or whipped cream, sweetened, and flavored
with vanilla; cover with Pistachio Ice Cream; pack in salt and ice, and
let stand one and one-half hours.



\needspace{15\baselineskip}
\section*{Claret Sauce}


\begin{itemize}
\setlength{\itemsep}{0pt}
\setlength{\parsep}{0pt}
\item 1 cup sugar
\item 1/4 cup water
\item 1/3 cup claret
\end{itemize}

\vspace{-0.5em}
\noindent%
Boil sugar and water eight minutes; cool slightly, and add claret.



\needspace{15\baselineskip}
\section*{Angel Parfait}


\begin{itemize}
\setlength{\itemsep}{0pt}
\setlength{\parsep}{0pt}
\item 1 cup sugar
\item 3/4 cup water
\item 3 egg whites
\item 1 pint heavy cream
\item 1 tablespoon vanilla
\end{itemize}

\vspace{-0.5em}
\noindent%
Boil sugar and water until syrup will thread when dropped from tip of
spoon. Pour slowly on the beaten whites of eggs, and continue the
beating until mixture is cool. Add cream beaten until stiff, and
vanilla; then freeze.



\needspace{15\baselineskip}
\section*{Café Parfait}


\begin{minipage}{1.0\textwidth}
{\setlength{\multicolsep}{0pt}\setlength{\columnsep}{2em}\raggedcolumns%
\begin{multicols}{2}
\begin{itemize}
\setlength{\itemsep}{0pt}
\setlength{\parsep}{0pt}
\item 1 cup milk
\item 1/4 cup Mocha coffee
\item 3 egg yolks
\item 1/8 teaspoon salt
\item 1 cup sugar
\item 3 cups thin cream
\end{itemize}
\end{multicols}}
\end{minipage}

\vspace{0.3em}
\noindent%
Scald milk with coffee, and add one-half the sugar; without straining,
use this mixture for making custard, with eggs, salt, and remaining
sugar; add one cup cream and let stand thirty minutes; cool, strain
through double cheese-cloth, add remaining cream, and freeze. Line a
mould, fill with Italian Meringue, cover, pack in salt and ice, using
two parts crushed ice to one part rock salt, and let stand three hours.



\needspace{15\baselineskip}
\section*{Italian Meringue}


\begin{minipage}{1.0\textwidth}
{\setlength{\multicolsep}{0pt}\setlength{\columnsep}{2em}\raggedcolumns%
\begin{multicols}{2}
\begin{itemize}
\setlength{\itemsep}{0pt}
\setlength{\parsep}{0pt}
\item 1/2 cup sugar
\item 1/4 cup water
\item 1 tablespoon gelatine or
\item 1/4 teaspoon granulated gelatine
\item 3 egg whites
\item 1 cup thin cream
\item 1/2 tablespoon vanilla
\end{itemize}
\end{multicols}}
\end{minipage}

\vspace{0.3em}
\noindent%
Make syrup by boiling sugar and water; pour slowly on beaten whites of
eggs, and continue beating. Place in pan of ice-water, and beat until
cold; dissolve gelatine in small quantity boiling water; strain into
mixture; whip cream, fold in whip, and flavor.



\needspace{15\baselineskip}
\section*{Bombe Glacée}

Line a mould with sherbet or water ice; fill with ice cream or thin
Charlotte Russe mixture; cover, pack in salt and ice, and let stand two
hours. The mould may be lined with ice cream. Pomegranate or Raspberry
Ice and Vanilla or Macaroon Ice Cream make a good combination.



\needspace{15\baselineskip}
\section*{Noisette Bomb}

                    Strawberry Ice I

\begin{minipage}{1.0\textwidth}
{\setlength{\multicolsep}{0pt}\setlength{\columnsep}{2em}\raggedcolumns%
\begin{multicols}{2}
\begin{itemize}
\setlength{\itemsep}{0pt}
\setlength{\parsep}{0pt}
\item 1/2 cup sugar
\item 1/2 cup chopped blanched filberts
\item 3/4 cup hot caramel syrup
\item 4 egg yolks
\item 1 1/3 cups heavy cream
\item 1/2 tablespoon vanilla
\item Few grains salt
\end{itemize}
\end{multicols}}
\end{minipage}

\vspace{0.3em}
\noindent%
Caramelize sugar, add nut meats, turn into a buttered pan, cool, then
pound in mortar and put through a purée strainer. Beat egg yolks until
thick, add gradually caramel syrup, and cook in double boiler until
mixture thickens, then beat until cold. Fold in cream beaten until
stiff. Then add prepared nut meats, vanilla, and salt. Line melon mould
with ice, turn in mixture, pack in salt and ice, and let stand three
hours.



\needspace{15\baselineskip}
\section*{Nesselrode Pudding}


\begin{minipage}{1.0\textwidth}
{\setlength{\multicolsep}{0pt}\setlength{\columnsep}{2em}\raggedcolumns%
\begin{multicols}{2}
\begin{itemize}
\setlength{\itemsep}{0pt}
\setlength{\parsep}{0pt}
\item 3 cups milk
\item 1 1/2 cups sugar
\item 5 egg yolks
\item 1/2 teaspoon salt
\item 1 pint thin cream
\item 1/4 cup pineapple syrup
\item 1 1/2 cups prepared French chestnuts
\end{itemize}
\end{multicols}}
\end{minipage}

\vspace{0.3em}
\noindent%
Make custard of first four ingredients, strain, cool, add cream,
pineapple syrup, and chestnuts; then freeze. To prepare chestnuts,
shell, cook in boiling water until soft, and force through a strainer.
Line a two-quart melon mould with part of mixture; to remainder add
one-half cup candied fruit cut in small pieces, one-quarter cup Sultana
raisins, and eight chestnuts broken in pieces, first soaked several
hours in Maraschino syrup. Fill mould, cover, pack in salt and ice, and
let stand two hours. Serve with whipped cream, sweetened and flavored
with Maraschino syrup.



\needspace{15\baselineskip}
\section*{Pistachio Fruit Ice Cream}


\begin{minipage}{1.0\textwidth}
{\setlength{\multicolsep}{0pt}\setlength{\columnsep}{2em}\raggedcolumns%
\begin{multicols}{2}
\begin{itemize}
\setlength{\itemsep}{0pt}
\setlength{\parsep}{0pt}
\item 3 cups milk
\item 1 1/2 cups sugar
\item 5 egg yolks
\item 1/2 teaspoon salt
\item 1 pint heavy cream
\item 1 1/2 cups chestnut purée
\item 1 teaspoon almond extract
\item 1 tablespoon vanilla
\item 3/4 cup glacé fruits
\item Maraschino
\item Leaf Green
\end{itemize}
\end{multicols}}
\end{minipage}

\vspace{0.3em}
\noindent%
Make a custard of first four ingredients, strain, cool; add cream,
chestnut purée, flavoring, and glacé fruit cut in pieces and previously
soaked in Maraschino three hours. Color with leaf green; freeze, mould,
pack in salt and ice, and let stand two hours. Serve with

\textbf{Fruit Sauce.} Drain syrup from a pint jar of canned strawberry,
raspberry, or pineapple, heat to boiling-point, thicken slightly with
arrowroot, and color with fruit red.



\needspace{15\baselineskip}
\section*{Nougat Ice Cream}

   3 cups milk
   1 cup sugar
   5 egg yolks
   1 teaspoon salt
   1 1/2 cups heavy cream
   5 egg whitess
   1/3 cup, each, pistachio, filbert, English walnut, and almond meats
   1 teaspoon almond extract
   1 tablespoon vanilla

Make a custard of first four ingredients, strain, and cool. Add heavy
cream beaten until stiff, whites of eggs beaten until stiff, nut meats
finely chopped, and flavoring; then freeze.



\needspace{15\baselineskip}
\section*{Orange Pekoe Ice Cream}


\begin{minipage}{1.0\textwidth}
{\setlength{\multicolsep}{0pt}\setlength{\columnsep}{2em}\raggedcolumns%
\begin{multicols}{2}
\begin{itemize}
\setlength{\itemsep}{0pt}
\setlength{\parsep}{0pt}
\item 2 cups milk
\item 3 tablespoons Orange Pekoe tea
\item 1 1/2 cups sugar
\item 4 egg yolks
\item 1/4 teaspoon salt
\item Grated rind 1 orange
\item 1 pint heavy cream
\end{itemize}
\end{multicols}}
\end{minipage}

\vspace{0.3em}
\noindent%
Scald milk to which tea had been added, and let stand five minutes. Add
sugar, and egg yolks slightly beaten, and cook until mixture thickens.
Strain, add remaining ingredients, freeze, and mould. Serve garnished
with Candied Orange Peel (p. 547).



\needspace{15\baselineskip}
\section*{Orange Delicious}


\begin{minipage}{1.0\textwidth}
{\setlength{\multicolsep}{0pt}\setlength{\columnsep}{2em}\raggedcolumns%
\begin{multicols}{2}
\begin{itemize}
\setlength{\itemsep}{0pt}
\setlength{\parsep}{0pt}
\item 2 cups sugar
\item 1 cup water
\item 2 cups orange juice
\item 1 cup cream
\item Yolks two eggs
\item 1 cup heavy cream
\item 1/4 cup shredded candied orange peel
\end{itemize}
\end{multicols}}
\end{minipage}

\vspace{0.3em}
\noindent%
Boil sugar and water eight minutes, then add orange juice. Scald cream,
add yolks of eggs, and cook over hot water until mixture thickens. Cool,
add to first mixture with heavy cream beaten stiff. Freeze; when nearly
frozen, add orange peel. Line a melon mould with Orange Ice, fill with
Orange Delicious, pack in salt and ice, and let stand one and one-half
hours.



\needspace{15\baselineskip}
\section*{Strawberry Mousse}


\begin{minipage}{1.0\textwidth}
{\setlength{\multicolsep}{0pt}\setlength{\columnsep}{2em}\raggedcolumns%
\begin{multicols}{2}
\begin{itemize}
\setlength{\itemsep}{0pt}
\setlength{\parsep}{0pt}
\item 1 quart thin cream
\item 1 box strawberries
\item 1 cup sugar
\item 1/4 box gelatine (scant) or
\item 1 1/4 tablespoons granulated gelatine
\item 2 tablespoons cold water
\item 3 tablespoons hot water
\end{itemize}
\end{multicols}}
\end{minipage}

\vspace{0.3em}
\noindent%
Wash and hull berries, sprinkle with sugar, and let stand one hour;
mash, and rub through a fine sieve; add gelatine soaked in cold and
dissolved in boiling water. Set in pan of ice-water and stir until it
begins to thicken; then fold in whip from cream, put in mould, cover,
pack in salt and ice, and let stand four hours. Raspberries may be used
in place of strawberries.



\needspace{15\baselineskip}
\section*{Coffee Mousse}

Make same as Strawberry Mousse, using one cup boiled coffee in place of
fruit juice.



\needspace{15\baselineskip}
\section*{Pineapple Mousse}


\begin{minipage}{1.0\textwidth}
{\setlength{\multicolsep}{0pt}\setlength{\columnsep}{2em}\raggedcolumns%
\begin{multicols}{2}
\begin{itemize}
\setlength{\itemsep}{0pt}
\setlength{\parsep}{0pt}
\item 1 tablespoon granulated gelatine
\item 1/4 cup cold water
\item 1 cup pineapple syrup
\item 2 tablespoons lemon juice
\item 1 cup sugar
\item 1 quart cream
\end{itemize}
\end{multicols}}
\end{minipage}

\vspace{0.3em}
\noindent%
Heat one can pineapple, and drain. To one cup of the syrup, add gelatine
soaked in cold water, lemon juice, and sugar. Strain and cool. As
mixture thickens, fold in the whip from cream. Mould, pack in salt and
ice, and let stand four hours.



\needspace{15\baselineskip}
\section*{Chocolate Mousse}


\begin{minipage}{1.0\textwidth}
{\setlength{\multicolsep}{0pt}\setlength{\columnsep}{2em}\raggedcolumns%
\begin{multicols}{2}
\begin{itemize}
\setlength{\itemsep}{0pt}
\setlength{\parsep}{0pt}
\item 2 squares Baker's chocolate
\item 1/2 cup powdered sugar
\item 1 cup cream
\item 3/4 tablespoon granulated gelatine
\item 3 tablespoons boiling water
\item 3/4 cup sugar
\item 1 teaspoon vanilla
\item 1 quart cream
\end{itemize}
\end{multicols}}
\end{minipage}

\vspace{0.3em}
\noindent%
Melt chocolate, add powdered sugar, and gradually one cup cream. Stir
over fire until boiling-point is reached, then add gelatine dissolved in
boiling water, sugar, and vanilla. Strain mixture into a bowl, set in a
pan of ice-water, stir constantly until mixture thickens, then fold in
the whip from remaining cream. Mould, pack in salt and ice, and let
stand four hours.



\needspace{15\baselineskip}
\section*{Maple Parfait}


\begin{itemize}
\setlength{\itemsep}{0pt}
\setlength{\parsep}{0pt}
\item 4 eggs
\item 1 cup hot maple syrup
\item 1 pint thick cream
\end{itemize}

\vspace{-0.5em}
\noindent%
Beat eggs slightly, and pour on slowly maple syrup. Cook until mixture
thickens, cool, and add cream beaten until stiff. Mould, pack in salt
and ice, and let stand three hours.



\needspace{15\baselineskip}
\section*{Mousse Marron}


\begin{minipage}{1.0\textwidth}
{\setlength{\multicolsep}{0pt}\setlength{\columnsep}{2em}\raggedcolumns%
\begin{multicols}{2}
\begin{itemize}
\setlength{\itemsep}{0pt}
\setlength{\parsep}{0pt}
\item 1 quart vanilla ice cream
\item 1/2 cup sugar
\item 1/4 cup water
\item Whites two eggs
\item 1 teaspoon granulated gelatine
\item 1 1/2 cups prepared French chestnuts
\item 1 pint cream
\item 1/2 tablespoon vanilla
\end{itemize}
\end{multicols}}
\end{minipage}

\vspace{0.3em}
\noindent%
Cook sugar and water five minutes, pour on to beaten whites of eggs,
dissolve gelatine in one and one-half tablespoons boiling water, and add
to first mixture. Set in a pan of ice-water, and stir until cold; add
chestnuts, and fold in whip from cream and vanilla. Line a mould with
ice cream, and fill with mixture; cover, pack in salt and ice, and let
stand three hours.



\needspace{15\baselineskip}
\section*{Cardinal Mousse, With Iced Madeira Sauce}

Line a mould with Pomegranate Ice; fill with Italian Meringue made of
three-fourths cup sugar, one-third cup hot water, whites two eggs, and
one and one-half teaspoons granulated gelatine dissolved in two
tablespoons boiling water. Beat until cold, and fold in whip from two
cups cream; flavor with one teaspoon vanilla, cover, pack in salt and
ice, and let stand three hours.



\needspace{15\baselineskip}
\section*{Iced Madeira Sauce}


\begin{minipage}{1.0\textwidth}
{\setlength{\multicolsep}{0pt}\setlength{\columnsep}{2em}\raggedcolumns%
\begin{multicols}{2}
\begin{itemize}
\setlength{\itemsep}{0pt}
\setlength{\parsep}{0pt}
\item 1/4 cup orange juice
\item 2 tablespoons lemon juice
\item 1/2 cup Madeira wine
\item 1/2 cup sugar
\item 1 cup boiling water
\item 2 egg whites
\end{itemize}
\end{multicols}}
\end{minipage}

\vspace{0.3em}
\noindent%
Freeze fruit juice and wine; boil sugar and water, pour on slowly to
beaten whites of eggs, set in pan of salted ice-water, and stir until
cold. Add to frozen mixture.



\needspace{15\baselineskip}
\section*{Cocoanut Naples, Sauterne Sauce}

Shape vanilla ice cream in individual moulds, and roll in shredded
cocoanut; serve with



\needspace{15\baselineskip}
\section*{Sauterne Sauce}


\begin{itemize}
\setlength{\itemsep}{0pt}
\setlength{\parsep}{0pt}
\item 1 cup sugar
\item 1/2 cup water
\item 4 tablespoons Sauterne
\item Burnett's Leaf Green
\end{itemize}

\vspace{-0.5em}
\noindent%
Make same as Claret Sauce, and color with leaf green.



\needspace{15\baselineskip}
\section*{Ice À La Margot}

Serve vanilla ice cream in champagne glasses. Cover ice cream with
whipped cream, sweetened, flavored with pistachio, and tinted very light
green. Garnish with pistachio nuts or Malaga grapes cut in halves.



\needspace{15\baselineskip}
\section*{Coup Aux Marrons}

Break marron glacé in pieces, flavor with rum, cover, and let stand one
hour. Put in champagne glasses, allowing one and one-half marrons to
each glass, cover with vanilla ice cream, and garnish with whipped
cream, sweetened and flavored with vanilla, and candied rose leaves.



\needspace{15\baselineskip}
\section*{Plombière Glacé}

Cover the bottom of small paper cases with vanilla ice cream, sprinkle
ice cream with marron glacé broken in pieces, arrange lady fingers at
equal distances, and allow them to extend one inch above cases. Pile
whipped cream, sweetened and flavored, in the centre and garnish with
marron glacé and candied violets or glacé cherries.



\needspace{15\baselineskip}
\section*{Demi-Glacé Aux Fraises}

Line a brick mould with Vanilla Ice Cream, put in layer of lady fingers,
and fill the centre with preserved strawberries or large fresh fruit cut
in halves; cover with ice cream, pack in salt and ice, and let stand one
hour. For ice cream, make custard of two and one-half cups milk, yolks
four eggs, one cup sugar, and one-fourth teaspoon salt; strain, cool,
add one cup heavy cream and one tablespoon vanilla; then freeze.



\needspace{15\baselineskip}
\section*{Mazarine}

Bake Brioche in a Charlotte Russe mould or individual tins, cool, cut a
slice from top of cake or cakes, and remove centre or centres, leaving a
wall or walls one-half inch thick. Fill with rich Vanilla Ice Cream,
invert on serving dish, and pour over

\textbf{Apricot Marmalade.} Drain one can apricots and force the fruit through
a strainer. Cook syrup until sufficiently reduced to add to fruit, and
make of consistency of marmalade. Add a few drops lemon juice and sugar
if necessary. Decorate top with halves of apricots, glacé cherries, and
whipped cream.



\needspace{15\baselineskip}
\section*{Flowering Ice Cream}

Line two and one-half inch flower-pots with paraffine paper. Fill with
ice cream, cover cream with grated vanilla chocolate to represent earth,
and insert a flower in each.



\needspace{15\baselineskip}
\section*{Concord Cream}


\begin{minipage}{1.0\textwidth}
{\setlength{\multicolsep}{0pt}\setlength{\columnsep}{2em}\raggedcolumns%
\begin{multicols}{2}
\begin{itemize}
\setlength{\itemsep}{0pt}
\setlength{\parsep}{0pt}
\item 1 pint cream
\item 1 1/4 cups grape juice
\item 1/3 cup sugar
\item Lemon or fresh lime juice
\item 1/2 cup heavy cream
\item Pistachio nuts, finely chopped
\end{itemize}
\end{multicols}}
\end{minipage}

\vspace{0.3em}
\noindent%
Mix cream, grape juice, and sugar. Add lemon or lime juice to taste.
Freeze, and serve in glasses. Garnish with heavy cream beaten until
stiff, sweetened, and flavored. Sprinkle cream with nuts.



\needspace{15\baselineskip}
\section*{German Ice Cream}

Mix one and one-fourth cups sugar, one tablespoon flour, and one-fourth
teaspoon salt. Add two eggs slightly beaten and two cups scalded milk.
Cook over hot water until mixture thickens, then add two squares melted
chocolate, and cool. Add three cups cream and one tablespoon vanilla.
Strain and freeze. Just before serving add three cups zweiback dried and
broken in small pieces.



\needspace{15\baselineskip}
\section*{Frozen Orange Soufflé}


\begin{minipage}{1.0\textwidth}
{\setlength{\multicolsep}{0pt}\setlength{\columnsep}{2em}\raggedcolumns%
\begin{multicols}{2}
\begin{itemize}
\setlength{\itemsep}{0pt}
\setlength{\parsep}{0pt}
\item 1 1/2 cups orange juice
\item 1 1/2 cups sugar
\item 2 tablespoons lemon juice
\item 5 egg yolks
\item 1 1/2 teaspoons granulated gelatine
\item 3 tablespoons boiling water
\item 2 1/2 cups cream
\item Candied orange peel
\item Pistachio nuts
\end{itemize}
\end{multicols}}
\end{minipage}

\vspace{0.3em}
\noindent%
Mix fruit juice, sugar, and yolks of eggs. Cook over boiling water until
mixture thickens, then add gelatine dissolved in boiling water. Cool,
freeze to a mush, add whip from cream, and continue freezing. Mould, and
serve garnished with candied orange peel and pistachio nuts.



\needspace{15\baselineskip}
\section*{Biscuit Tortoni In Boxes}


\begin{itemize}
\setlength{\itemsep}{0pt}
\setlength{\parsep}{0pt}
\item 1 cup dried macaroons, finely crushed
\item 2 cups thin cream
\item 1/2 cup sugar
\item 1/3 cup sherry
\item 1 pint heavy cream
\end{itemize}

\vspace{-0.5em}
\noindent%
Soak macaroons in thin cream one hour, add sugar, wine, and freeze to a
mush; then add heavy cream beaten stiff. Mould, pack in salt and ice,
and let stand two hours.

Trim lady fingers, arrange on plate in form of box. Keep in place with
ribbon one-half inch wide, and fasten at one corner by tying ribbon in a
bow. Garnish opposite corner with flowers of same color as ribbon.
Remove ice cream from brick, cut a slice three-fourths inch thick, and
place it in box.



\needspace{15\baselineskip}
\section*{Frozen Soufflé Glacé}


\begin{minipage}{1.0\textwidth}
{\setlength{\multicolsep}{0pt}\setlength{\columnsep}{2em}\raggedcolumns%
\begin{multicols}{2}
\begin{itemize}
\setlength{\itemsep}{0pt}
\setlength{\parsep}{0pt}
\item 4 eggs
\item Grated rind 1 lemon
\item 2/3 cup sugar
\item 1 tablespoon lemon juice
\item 1/2 cup Madeira wine
\item Few grains salt
\item 2/3 cup heavy cream
\end{itemize}
\end{multicols}}
\end{minipage}

\vspace{0.3em}
\noindent%
Beat yolks of eggs slightly; add lemon juice, grated rind, wine, sugar,
and salt; cook until mixture thickens, stirring constantly. Add whites
of eggs beaten stiff, and when well mixed, set in a pan of ice-water to
cool, stirring occasionally. Beat cream until stiff, and add. Fill small
paper cases with mixture, cover with macaroon dust, and set in a tin
mould with tight-fitting cover. Pack mould in salt and ice, and let
stand two hours.



\needspace{15\baselineskip}
\section*{Frozen Plum Pudding}


\begin{minipage}{1.0\textwidth}
{\setlength{\multicolsep}{0pt}\setlength{\columnsep}{2em}\raggedcolumns%
\begin{multicols}{2}
\begin{itemize}
\setlength{\itemsep}{0pt}
\setlength{\parsep}{0pt}
\item 2 cups milk
\item 1 cup sugar
\item 6 egg yolks
\item 1/4 teaspoon salt
\item 1/4 cup sherry
\item 2 1/2 cups cream
\item 3/4 cup candied fruit
\item 1/2 cup almonds, blanched and chopped
\item 1/3 cup Sultana raisins
\item 1/2 cup pounded macaroons
\end{itemize}
\end{multicols}}
\end{minipage}

\vspace{0.3em}
\noindent%
Make custard of milk, one-half the sugar, egg yolks, and salt.
Caramelize the remaining sugar and add. Strain, cool, add remaining
ingredients, freeze, and mould. If a baked ice cream is desired, use
whites of eggs for meringue, Baked Alaska (see p. 448).



\needspace{15\baselineskip}
\section*{Frozen Charlotte Glacé}

Mould ice cream in brick form or one-half pound baking-powder boxes.
Remove from mould or moulds, and surround with lady fingers, trimmed to
come to top of cream. Cover top with whipped cream, sweetened and
flavored, and pipe cream between lady fingers. Baking-powder boxes are
used when individual service is desired, the cream being cut in halves
crosswise.





\chapter{Pastry}



Pastry cannot be easily excluded from the menu of the New Englander. Who
can dream of a Thanksgiving dinner without a pie! The last decade has
done much to remove pies from the \textit{daily} bill of fare, and in their
place are found delicate puddings and seasonable fruits.

If pastry is to be served, have it of the best,--light, flaky, and
tender.

To pastry belongs, 1st, Puff Paste; 2d, Plain Paste.

Puff paste, which to many seems so difficult of preparation, is rarely
attempted by any except professionals. As a matter of fact, one who has
never handled a rolling-pin is less liable to fail, under the guidance
of a good teacher, than an old cook, who finds it difficult to overcome
the bad habit of using too much force in rolling. It is necessary to
work rapidly and with a light touch. A cold room is of great advantage.

For making pastry, pastry flour and the best shortenings, thoroughly
chilled, are essential. Its lightness depends on the amount of air
enclosed and expansion of that air in baking. The flakiness depends upon
kind and amount of shortening used. Lard makes more tender crust than
butter, but lacks flavor which butter gives. Puff paste is usually
shortened with butter, though some chefs prefer beef suet. Eggs and ice
were formerly used, but are not essentials.

Butter should be washed if pastry is to be of the best, so as to remove
salt and buttermilk, thus making it of a waxy consistency, easy to
handle.





\textbf{Rules for Washing Butter.} Scald and chill an earthen bowl. Heat palms
of hands in hot water, and chill in cold water. By following these
directions, butter will not adhere to bowl nor hands. Wash butter in
bowl by squeezing with hands until soft and waxy, placing bowl under a
cold-water faucet and allowing water to run. A small amount of butter
may be washed by using a wooden spoon in place of the hands.

For rolling paste, use a smooth wooden board, and wooden rolling-pin
with handles.

Puff paste should be used for vol-au-vents, patties, rissoles, bouchées,
cheese straws, tarts, etc. It may be used for rims and upper crusts of
pies, but never for lower crusts. Plain paste may be used where pastry
is needed, except for vol-au-vents and patties.



\needspace{15\baselineskip}
\section*{Puff Paste}


\begin{itemize}
\setlength{\itemsep}{0pt}
\setlength{\parsep}{0pt}
\item 1 pound butter
\item 1 pound pastry flour
\item Cold water
\end{itemize}

\vspace{-0.5em}
\noindent%
Wash the butter, pat and fold until no water flies. Reserve two
tablespoons of butter, and shape remainder into a circular piece
one-half inch thick, and put on floured board. Work two tablespoons of
butter into flour with the tips of fingers of the right hand. Moisten to
a dough with cold water, turn on slightly floured board, and knead one
minute. Cover with towel, and let stand five minutes.

Pat and roll one-fourth inch thick, keeping paste a little wider than
long, and corners square. If this cannot be accomplished with
rolling-pin, draw into shape with fingers. Place butter on centre of
lower half of paste. Cover butter by folding upper half of paste over
it. Press edges firmly, to enclose as much air as possible.

Fold right side of paste over enclosed butter, the left side under
enclosed butter. Turn paste half-way round, cover, and let stand five
minutes. Pat, and roll one-fourth inch thick, having paste longer than
wide, lifting often to prevent paste from sticking, and dredging board
slightly with flour when necessary. Fold from ends towards centre,
making three layers. Cover, and let stand five minutes. Repeat twice,
turning paste half-way round each time before rolling. After fourth
rolling, fold from ends to centre, and double, making four layers. Put
in cold place to chill; if outside temperature is not sufficiently cold,
fold paste in a towel, put in a dripping-pan, and place between dripping
pans of crushed ice. If paste is to be kept for several days, wrap in a
napkin, put in tin pail and cover tightly, then put in cold place; if in
ice box, do not allow pail to come in direct contact with ice.



\needspace{15\baselineskip}
\section*{To Bake Puff Paste}

Baking of puff paste requires as much care and judgment as making. After
shaping, chill thoroughly before baking. Puff paste requires hot oven,
greatest heat coming from the bottom, that the paste may properly rise.
While rising it is often necessary to decrease the heat by lifting
covers or opening the check to stove. Turn frequently, that it may rise
evenly. When it has risen its full height, slip a pan under the sheet on
which paste is baking to prevent burning on the bottom. Puff paste
should be baked on a tin sheet covered with a double thickness of brown
paper, or dripping-pan may be used, lined with brown paper. The
temperature for baking of patties should be about the same as for raised
biscuit; vol-au-vents require less heat, and are covered for first
half-hour to prevent scorching on top.



\needspace{15\baselineskip}
\section*{Patty Shells}

Roll puff paste one-quarter inch thick, shape with a patty cutter, first
dipped in flour; remove centres from one-half the rounds with smaller
cutter. Brush over with cold water the larger pieces near the edge, and
fit on rings, pressing lightly. Place in towel between pans of crushed
ice, and chill until paste is stiff; if cold weather, chill out of
doors. Place on iron or tin sheet covered with brown paper, and bake
twenty-five minutes in hot oven. The shells should rise their full
height and begin to brown in twelve to fifteen minutes; continue
browning, and finish baking in twenty-five minutes. Pieces cut from
centre of rings of patties may be baked and used for patty covers, or
put together, rolled, and cut for unders. Trimmings from puff paste
should be carefully laid on top of each other, patted, and rolled out.



\needspace{15\baselineskip}
\section*{Vol-Au-Vents}

Roll puff paste one-third inch thick, mark an oval on paste with cutter
or mould, and cut out with sharp knife, first dipped in flour. Brush
over near the edge with cold water, put on a rim three-fourths inch
wide, press lightly, chill, and bake. Vol-au-vents require for baking
forty-five minutes to one hour. During the first half-hour they should
be covered, watched carefully, and frequently turned. The paste cut from
centre of rim should be rolled one-quarter inch thick, shaped same size
as before rolling, chilled, baked, and used for cover to the
Vol-au-vent.



\needspace{15\baselineskip}
\section*{Quick Puff Paste}


\begin{itemize}
\setlength{\itemsep}{0pt}
\setlength{\parsep}{0pt}
\item 1 cup bread flour
\item 1 tablespoon lard
\item Cold water
\item 7/8 cup butter
\end{itemize}

\vspace{-0.5em}
\noindent%
Work lard into flour, first using knife then tips of fingers. Moisten to
a dough with cold water, pat, and roll out same as Puff Paste. Dot paste
with small pieces of butter, using one-third the quantity. Dredge with
flour, fold from ends toward centre, then double, making four layers.
Pat, and roll out. Repeat until butter is used. Roll, shape, chill, and
bake in a hot oven.



\needspace{15\baselineskip}
\section*{Plain Paste}


\begin{itemize}
\setlength{\itemsep}{0pt}
\setlength{\parsep}{0pt}
\item 1 1/2 cups flour
\item 1/4 cup lard
\item 1/4 cup butter
\item 1/2 teaspoon salt.
\item Cold water
\end{itemize}

\vspace{-0.5em}
\noindent%
Wash butter, pat, and form in circular piece. Add salt to flour, and
work in lard with tips of fingers or case knife. Moisten to dough with
cold water; ice-water is not an essential, but is desirable in summer.
Toss on board dredged sparingly with flour, pat, and roll out; fold in
butter as for puff paste, pat, and roll out. Fold so as to make three
layers, turn half-way round, pat, and roll out; repeat. The pastry may
be used at once; if not, fold in cheese-cloth, put in covered tin, and
keep in cold place, but never in direct contact with ice. Plain paste
requires a moderate oven. This is superior paste and quickly made.



\needspace{15\baselineskip}
\section*{Chopped Paste}


\begin{itemize}
\setlength{\itemsep}{0pt}
\setlength{\parsep}{0pt}
\item 2 cups flour
\item 2 tablespoons lard
\item 2/3 cup butter
\item 1/2 teaspoon salt
\item Cold water
\end{itemize}

\vspace{-0.5em}
\noindent%
Wash butter. Mix salt with flour, put in chopping tray, add lard and
butter, and chop until well mixed. Moisten to a dough with cold water.
Toss on floured cloth (Magic Cover), pat, and roll out. Fold so as to
make three layers, turn half-way round, pat, and roll out; repeat.
Should the butter be too hard, it will not mix readily with the flour,
in which case the result will be a tough crust. Omit lard, and use all
butter, if preferred.



\needspace{15\baselineskip}
\section*{Quick Paste}


\begin{itemize}
\setlength{\itemsep}{0pt}
\setlength{\parsep}{0pt}
\item 1 1/2 cups flour
\item 3/4 teaspoon salt
\item 1/4 cup cottolene or cocoanut butter
\item Cold water
\end{itemize}

\vspace{-0.5em}
\noindent%
Mix salt with flour, cut in shortening with knife. Moisten to dough with
cold water. Toss on floured board, pat, roll out, and roll up like a
jelly roll. Use one-third cup of shortening if a richer paste is
desired.



\needspace{15\baselineskip}
\section*{Paste With Lard}


\begin{itemize}
\setlength{\itemsep}{0pt}
\setlength{\parsep}{0pt}
\item 1 1/2 cups flour
\item 1/2 teaspoon salt
\item 1/3 cup lard
\item Cold water
\end{itemize}

\vspace{-0.5em}
\noindent%
Mix salt with flour. Reserve one and one-fourth tablespoons lard, work
in remainder to flour, using tips of fingers or a case knife. Moisten to
a dough with water. Toss on a floured board, pat, and roll out. Spread
with one tablespoon reserved lard, dredge with flour, roll up like a
jelly roll, pat, and roll out; again roll up. Cut from the end of roll a
piece large enough to line a pie plate. Pat and roll out, keeping the
paste as circular in form as possible. With care and experience there
need be no trimmings. Worked-over pastry is never as satisfactory. The
remaining one-fourth tablespoon lard is used to dot over upper crust of
pie just before sending to oven; this gives the pie a flaky appearance.
Ice-water has a similar effect. If milk is brushed over the pie it has a
glazed appearance. This quantity of paste will make one pie with two
crusts and a few puffs, or two pies with one crust where the rim is
built up and fluted.



\needspace{15\baselineskip}
\section*{Entire Wheat Paste}


\begin{minipage}{1.0\textwidth}
{\setlength{\multicolsep}{0pt}\setlength{\columnsep}{2em}\raggedcolumns%
\begin{multicols}{2}
\begin{itemize}
\setlength{\itemsep}{0pt}
\setlength{\parsep}{0pt}
\item 1 cup fine Entire Wheat Flour
\item 1/2 cup pastry flour
\item 1 teaspoon salt
\item 3 tablespoons lard
\item 1/2 cup butter
\item Cold water
\end{itemize}
\end{multicols}}
\end{minipage}

\vspace{0.3em}
\noindent%
Make same as Plain Paste. Roll to one-fourth inch in thickness, cut in
finger-shaped pieces, bake, cool, brush over with slightly beaten white
one egg diluted with one teaspoon cold water, and sprinkle with chopped
nut meat seasoned with salt. Return to oven to slightly brown nut meats.
Serve with salad course.





\chapter{Pies}



Paste for pies should be one-fourth inch thick and rolled a little
larger than the plate to allow for shrinking. In dividing paste for
pies, allow more for upper than under crusts. Always perforate upper
crusts that steam may escape. Some make a design, others pierce with a
large fork.

Flat rims for pies should be cut in strips three-fourths inch wide.
Under crusts should be brushed with cold water before putting on rims,
and rims slightly fulled, otherwise they will shrink from edge of plate.
The pastry jagger, a simple device for cutting paste, makes rims with
fluted edges.

Pies requiring two crusts sometimes have a rim between the crusts. This
is mostly confined to mince pieces, where there is little danger of
juice escaping. Sometimes a rim is placed over upper crust. Where two
pieces of paste are put together, the under piece should always be
brushed with cold water, the upper piece placed over, and the two
pressed lightly together; otherwise they will separate during baking.

When juicy fruit is used for filling pies, some of the juices are apt to
escape during baking. As a precaution, bind with a strip of cotton cloth
wrung out of cold water and cut one inch wide and long enough to
encircle the plate. Squash, pumpkin, and custard pies are much less care
during baking when bound. Where cooked fruits are used for filling, it
is desirable to bake crusts separately. This is best accomplished by
covering an inverted deep pie plate with paste and baking for under
crust. Prick with a fork before baking. Slip from plate, and fill. For
upper crusts, roll a piece of paste a little larger than the pie plate,
prick, and bake on a tin sheet.

For baking pies, perforated tin plates are used. They may be bought
shallow or deep. By the use of such plates the under crust is well
cooked. Pastry should be thoroughly baked and well browned. Pies require
from thirty-five to forty-five minutes for baking. Never grease a pie
plate; good pastry greases its own tin. Slip pies, when slightly cooled,
to earthen plates.



\needspace{15\baselineskip}
\section*{Apple Pie I}


\begin{minipage}{1.0\textwidth}
{\setlength{\multicolsep}{0pt}\setlength{\columnsep}{2em}\raggedcolumns%
\begin{multicols}{2}
\begin{itemize}
\setlength{\itemsep}{0pt}
\setlength{\parsep}{0pt}
\item 4 or 5 sour apples
\item 1/3 cup sugar
\item 1/4 teaspoon grated nutmeg
\item 1/8 teaspoon salt
\item 1 teaspoon butter
\item 1 teaspoon lemon juice
\item Few gratings lemon rind
\end{itemize}
\end{multicols}}
\end{minipage}

\vspace{0.3em}
\noindent%
Line pie plate with paste. Pare, core, and cut the apples into eighths,
put row around plate one-half inch from edge, and work towards centre
until plate is covered; then pile on remainder. Mix sugar, nutmeg, salt,
lemon juice, and grated rind, and sprinkle over apples. Dot over with
butter. Wet edges of under crust, cover with upper crust, and press
edges together.

Bake forty to forty-five minutes in moderate oven. A very good pie may
be made without butter, lemon juice, and grated rind. Cinnamon may be
substituted for nutmeg. Evaporated apples may be used in place of fresh
fruit. If used, they should be soaked over night in cold water.



\needspace{15\baselineskip}
\section*{Apple Pie II}

Use same ingredients as for Apple Pie I. Place in small earthen
baking-dish and add hot water to prevent apples from burning. Cover
closely, and bake three hours in very slow oven, when apples will be a
dark red color. Brown sugar may be used instead of white sugar, a little
more being required. Cool, and bake between two crusts.



\needspace{15\baselineskip}
\section*{Blackberry Pie}

Pick over and wash one and one-half cups berries. Stew until soft with
enough water to prevent burning. Add sugar to taste, and one-eighth
teaspoon salt. Line plate with paste, put on a rim, fill with berries
(which have been cooled); arrange six strips pastry across the top, cut
same width as rim; put on an upper rim. Bake thirty minutes in moderate
oven.



\needspace{15\baselineskip}
\section*{Blueberry Pie}


\begin{itemize}
\setlength{\itemsep}{0pt}
\setlength{\parsep}{0pt}
\item 2 1/2 cups berries
\item Flour
\item 1/2 cup sugar
\item 1/8 teaspoon salt
\end{itemize}

\vspace{-0.5em}
\noindent%
Line a deep plate with Plain Paste, fill with berries slightly dredged
with flour; sprinkle with sugar and salt, cover, and bake forty-five to
fifty minutes in a moderate oven. For sweetening, some prefer to use
one-third molasses, the remaining two-thirds to be sugar. Six green
grapes (from which seeds have been removed) cut in small pieces much
improve the flavor, particularly where huckleberries are used in place
of blueberries.



\needspace{15\baselineskip}
\section*{Cranberry Pie}


\begin{itemize}
\setlength{\itemsep}{0pt}
\setlength{\parsep}{0pt}
\item 1 1/2 cups cranberries
\item 1/2 cup water
\item 3/4 cup sugar
\end{itemize}

\vspace{-0.5em}
\noindent%
Put ingredients in saucepan in order given, and cook ten minutes; cool,
and bake in one crust, with a rim, and strips across the top.



\needspace{15\baselineskip}
\section*{Currant Pie}


\begin{itemize}
\setlength{\itemsep}{0pt}
\setlength{\parsep}{0pt}
\item 1 cup currants
\item 1 cup sugar
\item 1/4 cup flour
\item 2 egg yolks
\item 2 tablespoons water
\end{itemize}

\vspace{-0.5em}
\noindent%
Mix flour and sugar, add yolks of eggs slightly beaten and diluted with
water. Wash currants, drain, remove stems, then measure; add to first
mixture and bake in one crust; cool, and cover with Meringue I. Cook in
slow oven until delicately browned.



\needspace{15\baselineskip}
\section*{Cream Pie}

Bake three crusts on separate pie plates. Put together with Cream
Filling and dust over with powdered sugar. If allowed to stand after
filling for any length of time, the pastry will soften.



\needspace{15\baselineskip}
\section*{Custard Pie}


\begin{itemize}
\setlength{\itemsep}{0pt}
\setlength{\parsep}{0pt}
\item 2 eggs
\item 3 tablespoons sugar
\item 1/8 teaspoon salt
\item 1 1/2 cups milk
\item Few gratings nutmeg
\end{itemize}

\vspace{-0.5em}
\noindent%
Beat eggs slightly, add sugar, salt, and milk. Line plate with paste,
and build up a fluted rim. Strain in the mixture and sprinkle with few
gratings nutmeg. Bake in quick oven at first to set rim, decrease the
heat afterwards, as egg and milk in combination need to be cooked at low
temperature.



\needspace{15\baselineskip}
\section*{Date Pie}


\begin{itemize}
\setlength{\itemsep}{0pt}
\setlength{\parsep}{0pt}
\item 2 cups milk
\item 1/3 pound sugar dates
\item 2 eggs
\item 1/4 teaspoon salt
\item Few gratings nutmeg
\end{itemize}

\vspace{-0.5em}
\noindent%
Cook dates with milk twenty minutes in top of double boiler. Strain, and
rub through sieve, then add eggs and salt. Bake same as Custard Pie.



\needspace{15\baselineskip}
\section*{Lemon Pie I}


\begin{minipage}{1.0\textwidth}
{\setlength{\multicolsep}{0pt}\setlength{\columnsep}{2em}\raggedcolumns%
\begin{multicols}{2}
\begin{itemize}
\setlength{\itemsep}{0pt}
\setlength{\parsep}{0pt}
\item 1/2 cup chopped apple
\item 1 cup sugar
\item 1 beaten egg
\item 1/4 cup rolled common crackers
\item 2 tablespoons lemon juice
\item Grated rind 1 lemon
\item 1 teaspoon melted butter
\end{itemize}
\end{multicols}}
\end{minipage}

\vspace{0.3em}
\noindent%
Mix ingredients in order given and bake with two crusts.



\needspace{15\baselineskip}
\section*{Lemon Pie II}


\begin{minipage}{1.0\textwidth}
{\setlength{\multicolsep}{0pt}\setlength{\columnsep}{2em}\raggedcolumns%
\begin{multicols}{2}
\begin{itemize}
\setlength{\itemsep}{0pt}
\setlength{\parsep}{0pt}
\item 3/4 cup sugar
\item 3/4 cup boiling water
\item 2 tablespoons corn-starch
\item 2 tablespoons flour
\item 2 egg yolks
\item 3 tablespoons lemon juice
\item Grated rind 1 lemon
\item 1 teaspoon butter
\end{itemize}
\end{multicols}}
\end{minipage}

\vspace{0.3em}
\noindent%
Mix corn-starch, flour, and sugar, add boiling water, stirring
constantly. Cook two minutes, add butter, egg yolks, and rind and juice
of lemon. Line plate with paste same as for Custard Pie. Turn in mixture
which has been cooled, and bake until pastry is well browned. Cool
slightly, and cover with Meringue I; then return to oven and bake
meringue.



\needspace{15\baselineskip}
\section*{Lemon Pie III}


\begin{minipage}{1.0\textwidth}
{\setlength{\multicolsep}{0pt}\setlength{\columnsep}{2em}\raggedcolumns%
\begin{multicols}{2}
\begin{itemize}
\setlength{\itemsep}{0pt}
\setlength{\parsep}{0pt}
\item 4 egg yolks
\item 6 tablespoons sugar
\item Few grains salt
\item 1 1/4 cups milk
\item 4 egg whitess
\item 7/8 cup powdered sugar
\item 1 lemon
\end{itemize}
\end{multicols}}
\end{minipage}

\vspace{0.3em}
\noindent%
Beat yolks of eggs slightly, add sugar, salt, grated rind of lemon, and
milk. Line plate with paste as for Custard Pie. Pour in mixture. Bake in
moderate oven until set. Remove from oven, cool slightly, and cover with
Meringue III (see p. 480) made of whites of eggs, powdered sugar, and
lemon juice.



\needspace{15\baselineskip}
\section*{Lemon Pie IV}


\begin{itemize}
\setlength{\itemsep}{0pt}
\setlength{\parsep}{0pt}
\item 3 eggs
\item 2/3 cup sugar
\item 1/4 cup lemon juice
\item Grated rind 1/2 lemon
\item 2 tablespoons water
\end{itemize}

\vspace{-0.5em}
\noindent%
Beat eggs slightly, add sugar, lemon juice, grated rind, and water. Bake
in one crust in a moderate oven. Cool slightly, cover with Meringue II,
then return to oven and bake meringue.



\needspace{15\baselineskip}
\section*{Lemon Pie V}


\begin{minipage}{1.0\textwidth}
{\setlength{\multicolsep}{0pt}\setlength{\columnsep}{2em}\raggedcolumns%
\begin{multicols}{2}
\begin{itemize}
\setlength{\itemsep}{0pt}
\setlength{\parsep}{0pt}
\item 1 cup sugar
\item 3 tablespoons flour
\item 3 tablespoons lemon juice
\item 4 egg yolks
\item 1 cup milk
\item 1 tablespoon melted butter
\item 2 egg whites
\item Few grains salt
\end{itemize}
\end{multicols}}
\end{minipage}

\vspace{0.3em}
\noindent%
Mix sugar and flour, add lemon juice, egg yolks slightly beaten, milk,
butter, whites of eggs beaten stiff, and salt. Bake in one crust, and
cover with meringue or not, as desired.



\needspace{15\baselineskip}
\section*{Mince Pies}

Mince pies should be always baked with two crusts. For Thanksgiving and
Christmas pies, Puff Paste is often used for rims and upper crusts, but
is never satisfactory when used for under crusts.



\needspace{15\baselineskip}
\section*{Mince Pie Meat I}


\begin{minipage}{1.0\textwidth}
{\setlength{\multicolsep}{0pt}\setlength{\columnsep}{2em}\raggedcolumns%
\begin{multicols}{2}
\begin{itemize}
\setlength{\itemsep}{0pt}
\setlength{\parsep}{0pt}
\item 4 lbs. lean beef
\item 2 lbs. beef suet
\item Baldwin apples
\item 3 quinces
\item 3 lbs. sugar
\item 2 cups molasses
\item 2 quarts cider
\item 4 lbs. raisins, seeded and cut in pieces
\item 3 lbs. currants
\item 1/2 lb. finely cut citron
\item 1 quart cooking brandy
\item 1 tablespoon cinnamon and mace
\item 1 tablespoon powdered clove
\item 2 grated nutmegs
\item 1 teaspoon pepper
\item Salt to taste
\end{itemize}
\end{multicols}}
\end{minipage}

\vspace{0.3em}
\noindent%
Cover meat and suet with boiling water and cook until tender, cool in
water in which they are cooked; the suet will rise to top, forming a
cake of fat, which may be easily removed. Finely chop meat, and add it
to twice the amount of finely chopped apples. The apples should be
quartered, cored, and pared, previous to chopping, or skins may be left
on, which is not an objection if apples are finely chopped. Add quinces
finely chopped, sugar, molasses, cider, raisins, currants, and citron;
also suet, and stock in which meat and suet were cooked, reduced to one
and one-half cups. Heat gradually, stir occasionally, and cook slowly
two hours; then add brandy and spices.



\needspace{15\baselineskip}
\section*{Mince Pie Meat II}


\begin{minipage}{1.0\textwidth}
{\setlength{\multicolsep}{0pt}\setlength{\columnsep}{2em}\raggedcolumns%
\begin{multicols}{2}
\begin{itemize}
\setlength{\itemsep}{0pt}
\setlength{\parsep}{0pt}
\item 5 cups chopped cooked beef
\item 2 1/2 cups chopped suet
\item 7 1/2 cups chopped apples
\item 3 cups cider
\item 1/2 cup vinegar
\item 1 cup molasses
\item 5 cups sugar
\item 3/4 lb. citron, finely chopped
\item 2 1/2 cups whole raisins
\item 1 1/2 cups raisins, finely chopped
\item Salt
\item Juice 2 lemons
\item Juice 2 oranges
\item 1 tablespoon mace
\item 2 tablespoons cinnamon
\item 2 tablespoons clove
\item 2 tablespoons allspice
\item 2 nutmegs grated
\item 2 tablespoons lemon extract
\item 1 teaspoon almond extract
\item 1 1/2 cups brandy
\item 3 cups liquor in which beef was cooked
\end{itemize}
\end{multicols}}
\end{minipage}

\vspace{0.3em}
\noindent%
Mix ingredients in the order given, except brandy, and let simmer one
and one-half hours; then add brandy and shavings from the rind of the
lemons and oranges.



\needspace{15\baselineskip}
\section*{English Mince Meat}


\begin{minipage}{1.0\textwidth}
{\setlength{\multicolsep}{0pt}\setlength{\columnsep}{2em}\raggedcolumns%
\begin{multicols}{2}
\begin{itemize}
\setlength{\itemsep}{0pt}
\setlength{\parsep}{0pt}
\item 5 lbs. raisins, seeded
\item 5 lbs. currants
\item 5 lbs. light brown sugar
\item 1/2 teaspoon mace
\item 1/2 teaspoon cinnamon
\item 2 1/2 cups brandy
\end{itemize}
\end{multicols}}
\end{minipage}

\vspace{0.3em}
\noindent%
Cook raisins, suet, apples, citron, currants, and sugar slowly for one
and one-half hours; then add almonds, spices, and brandy.



\needspace{15\baselineskip}
\section*{Mince Meat (Without Alcoholic Liquor)}

Mix together one cup chopped apple, one-half cup raisins seeded and
chopped, one-half cup currants, one-fourth cup butter, one tablespoon
molasses, one tablespoon boiled cider, one cup sugar, one teaspoon
cinnamon, one-half teaspoon cloves, one-half nutmeg grated, one
salt-spoon of mace, and one teaspoon salt. Add enough stock in which
meat was cooked to moisten; heat gradually to boiling-point, and simmer
one hour; then add one cup chopped meat and two tablespoons Barberry
Jelly. Cook fifteen minutes.



\needspace{15\baselineskip}
\section*{Mock Mince Pie}

                   4 common crackers, rolled

\begin{minipage}{1.0\textwidth}
{\setlength{\multicolsep}{0pt}\setlength{\columnsep}{2em}\raggedcolumns%
\begin{multicols}{2}
\begin{itemize}
\setlength{\itemsep}{0pt}
\setlength{\parsep}{0pt}
\item 1 1/2 cups sugar
\item 1 cup molasses
\item 1/3 cup lemon juice or vinegar
\item 1 cup raisins, seeded and chopped
\item 1/2 cup butter
\item 2 eggs well beaten
\item Spices
\end{itemize}
\end{multicols}}
\end{minipage}

\vspace{0.3em}
\noindent%
Mix ingredients in order given, adding spices to taste. Bake between
crusts. This quantity will make two pies.



\needspace{15\baselineskip}
\section*{Mock Cherry Pie}

Mix one cup cranberries cut in halves, one-half cup raisins seeded and
cut in pieces, three-fourths cup sugar, and one tablespoon flour. Dot
over with one teaspoon butter. Bake between crusts.



\needspace{15\baselineskip}
\section*{Peach Pie}

Remove skins from peaches. This may be done easily after allowing
peaches to stand in boiling water one minute. Cut in eighths, cook until
soft with enough water to prevent burning; sweeten to taste. Cool, and
fill crust previously baked. Cover with whipped cream, sweetened and
flavored. Fresh strawberries, cut in halves, slightly mashed and
sweetened, are attractively served in a pastry case.



\needspace{15\baselineskip}
\section*{Prune Pie}


\begin{itemize}
\setlength{\itemsep}{0pt}
\setlength{\parsep}{0pt}
\item 1/2 lb. prunes
\item 1/2 cup sugar (scant)
\item 1 tablespoon lemon juice
\item 1 1/2 teaspoons butter
\item 1 tablespoon flour
\end{itemize}

\vspace{-0.5em}
\noindent%
Wash prunes and soak in enough cold water to cover. Cook in same water
until soft. Remove stones, cut prunes in quarters, and mix with sugar
and lemon juice. Reduce liquor to one and one-half tablespoons. Line
plate with paste, cover with prunes, pour over liquor, dot over with
butter, and dredge with flour. Put on an upper crust and bake in a
moderate oven.



\needspace{15\baselineskip}
\section*{Rhubarb Pie}


\begin{itemize}
\setlength{\itemsep}{0pt}
\setlength{\parsep}{0pt}
\item 1 1/2 cups rhubarb
\item 7/8 cup sugar
\item 1 egg
\item 2 tablespoons flour
\end{itemize}

\vspace{-0.5em}
\noindent%
Skin and cut stalks of rhubarb in half-inch pieces before measuring. Mix
sugar, flour, and egg; add to rhubarb and bake between crusts. Many
prefer to scald rhubarb before using; if so prepared, losing some of its
acidity, less sugar is required.



\needspace{15\baselineskip}
\section*{Squash Pie I}


\begin{minipage}{1.0\textwidth}
{\setlength{\multicolsep}{0pt}\setlength{\columnsep}{2em}\raggedcolumns%
\begin{multicols}{2}
\begin{itemize}
\setlength{\itemsep}{0pt}
\setlength{\parsep}{0pt}
\item 1 1/4 cups steamed and strained squash
\item 1/4 cup sugar
\item 1/2 teaspoon salt
\item 1/4 teaspoon cinnamon, ginger, nutmeg, or
\item 1/2 teaspoon lemon extract
\item 1 egg
\item 7/8 cup milk
\end{itemize}
\end{multicols}}
\end{minipage}

\vspace{0.3em}
\noindent%
Mix sugar, salt, and spice or extract, add squash, egg slightly beaten,
and milk gradually. Bake in one crust, following directions for Custard
Pie. If a richer pie is desired, use one cup squash, one-half cup each
of milk and cream, and an additional egg yolk.



\needspace{15\baselineskip}
\section*{Squash Pie II}


\begin{minipage}{1.0\textwidth}
{\setlength{\multicolsep}{0pt}\setlength{\columnsep}{2em}\raggedcolumns%
\begin{multicols}{2}
\begin{itemize}
\setlength{\itemsep}{0pt}
\setlength{\parsep}{0pt}
\item 1 cup squash, steamed and strained
\item 1 cup heavy cream
\item 1 cup sugar
\item 3 eggs, slightly beaten
\item 4 tablespoons brandy
\item 1 teaspoon cinnamon
\item 1 teaspoon nutmeg 
\item 3/4 teaspoon Ginger
\item Salt
\item 1/4 teaspoon mace
\end{itemize}
\end{multicols}}
\end{minipage}

\vspace{0.3em}
\noindent%
Line a deep pie plate with puff paste. Brush over paste with white of
egg slightly beaten, and sprinkle with stale bread crumbs; fill, and
bake in a moderate oven. Serve warm.



\needspace{15\baselineskip}
\section*{Pumpkin Pie}


\begin{minipage}{1.0\textwidth}
{\setlength{\multicolsep}{0pt}\setlength{\columnsep}{2em}\raggedcolumns%
\begin{multicols}{2}
\begin{itemize}
\setlength{\itemsep}{0pt}
\setlength{\parsep}{0pt}
\item 1 1/2 cups steamed and strained pumpkin
\item 2/3 cup brown sugar
\item 1 teaspoon cinnamon
\item 1/2 teaspoon ginger
\item 1/2 teaspoon salt
\item 2 eggs
\item 1 1/2 cups milk
\item 1/2 cup cream
\end{itemize}
\end{multicols}}
\end{minipage}

\vspace{0.3em}
\noindent%
Mix ingredients in order given and bake in one crust.









\chapter{Pastry Desserts}




\needspace{15\baselineskip}
\section*{Banbury Tarts}


\begin{itemize}
\setlength{\itemsep}{0pt}
\setlength{\parsep}{0pt}
\item 1 cup raisins
\item 1 cup sugar
\item 1 egg
\item 1 cracker
\item Juice and grated rind 1 lemon
\end{itemize}

\vspace{-0.5em}
\noindent%
Stone and chop raisins, add sugar, egg slightly beaten, cracker finely
rolled, and lemon juice and rind. Roll pastry one-eighth inch thick, and
cut pieces three and one-half inches long by three inches wide. Put two
teaspoons of mixture on each piece. Moisten edge with cold water
half-way round, fold over, press edges together with three-tined fork,
first dipped in flour. Bake twenty minutes in slow oven.



\needspace{15\baselineskip}
\section*{Cheese Cakes}


\begin{minipage}{1.0\textwidth}
{\setlength{\multicolsep}{0pt}\setlength{\columnsep}{2em}\raggedcolumns%
\begin{multicols}{2}
\begin{itemize}
\setlength{\itemsep}{0pt}
\setlength{\parsep}{0pt}
\item 1 cup sweet milk
\item 1 cup sour milk
\item 1 cup sugar
\item 4 egg yolks
\item Juice and grated rind one lemon
\item 1/4 cup almonds, blanched and chopped
\item 1/4 teaspoon salt
\end{itemize}
\end{multicols}}
\end{minipage}

\vspace{0.3em}
\noindent%
Scald sweet and sour milk, strain through cheese-cloth. To curd add
sugar, yolks of eggs slightly beaten, lemon, and salt. Line patty pans
with paste, fill with mixture, and sprinkle with chopped almonds. Bake
until mixture is firm to the touch.



\needspace{15\baselineskip}
\section*{Cheese Straws}

Roll puff or plain paste one-fourth inch thick, sprinkle one-half with
grated cheese to which has been added few grains of salt and cayenne.
Fold, press edges firmly together, fold again, pat, and roll out
one-fourth inch thick. Sprinkle with cheese and proceed as before;
repeat twice. Cut in strips five inches long and one-fourth inch wide.
Bake eight minutes in hot oven. Parmesan cheese, or equal parts of
Parmesan and Edam cheese, may be used. Cheese straws are piled log cabin
fashion and served with cheese or salad course.



\needspace{15\baselineskip}
\section*{Condés}


\begin{itemize}
\setlength{\itemsep}{0pt}
\setlength{\parsep}{0pt}
\item 2 egg whites
\item 3/4 cup powdered sugar
\item 2 oz. almonds, blanched and finely chopped
\end{itemize}

\vspace{-0.5em}
\noindent%
Beat whites of eggs until stiff, add sugar gradually, then almonds. Roll
paste, and cut in strips three and one-half inches long by one and
one-half inches wide. Spread with mixture; avoid having it come close to
edge. Dust with powdered sugar and bake fifteen minutes in moderate
oven.



\needspace{15\baselineskip}
\section*{Galattes}

Roll puff or plain paste one-eighth inch thick. Shape with an oblong
cutter three and one-half inches long by one and three-fourths inches
wide. Brush over with white of egg and sprinkle with cinnamon and sugar.
Bake in a hot oven. A lady-finger cutter may be used with satisfaction,
but is more difficult to procure.



\needspace{15\baselineskip}
\section*{Cream Horns}

Roll puff paste in a long rectangular piece, one-eighth inch thick. Cut
in strips three-fourths inch wide. Roll paste over wooden forms bought
for the purpose, having edges overlap. Bake in hot oven until well
puffed and slightly browned. Brush over with white of egg slightly
beaten, diluted with one teaspoon water, then sprinkle with sugar.
Return to oven and finish cooking, and remove from forms. When cold,
fill with Cream Filling or whipped cream sweetened and flavored.



\needspace{15\baselineskip}
\section*{Florentine Meringue}

Roll puff or plain paste one-eighth inch thick; cut a piece ten inches
long by seven inches wide; place on a sheet, wet edges, and put on a
half-inch rim. Prick with fork six times, and bake in hot oven. Cool,
and spread with jam, cover with Meringue II, and almonds blanched and
shredded; sprinkle with powdered sugar and bake.



\needspace{15\baselineskip}
\section*{Cocoanut Tea Cakes}

Roll puff or plain paste to one-fourth inch in thickness. Shape with a
lady-finger cutter and bake on a tin sheet in a hot oven. When nearly
done remove from oven, cool slightly, brush over with beaten white of
egg, sprinkle with shredded cocoanut, and return to oven to finish the
cooking.



\needspace{15\baselineskip}
\section*{Napoleons}

Bake three sheets of pastry, pricking before baking. Put between the
sheets Cream Filling; spread top with Confectioner's Frosting, sprinkle
with pistachio nuts blanched and chopped, crease in pieces about two and
one-half by four inches, and cut with sharp knife.



\needspace{15\baselineskip}
\section*{Orange Sticks}

Cut puff or plain paste rolled one-eighth inch thick in strips five
inches long by one inch wide, and bake in hot oven. Put together in
pairs, with Orange Filling between.



\needspace{15\baselineskip}
\section*{Lemon Sticks}

Lemon Sticks may be made in same manner as Orange Sticks, using Lemon
Filling.



\needspace{15\baselineskip}
\section*{Palm Leaves}

Roll remnants of puff paste one-eighth inch thick; sprinkle one-half
surface with powdered sugar, fold, press edges together, pat and roll
out, using sugar for dredging board; repeat three times. After the last
rolling fold four times. The pastry should be in long strip one and
one-half inches wide. From the end, cut pieces one inch wide; place on
baking-sheet, broad side down, one inch apart, and separate layers of
pastry at one end to suggest a leaf. Bake eight minutes in hot oven;
these will spread while baking.



\needspace{15\baselineskip}
\section*{Raspberry Puffs}

Roll plain paste one-eighth inch thick, and cut in pieces four by three
and one-half inches. Put one-half tablespoon raspberry jam on centre of
lower half of each piece, wet edges half-way around, fold, press edges
firmly together, prick tops, place on sheet, and bake twenty minutes in
hot oven.



\needspace{15\baselineskip}
\section*{Tarts}

Roll puff paste one-eighth inch thick. Shape with a fluted round cutter,
first dipped in flour; with a smaller cutter remove centres from half
the pieces, leaving rings one-half inch wide. Brush with cold water the
larger pieces near the edge; fit on rings, pressing lightly. Chill
thoroughly, and bake fifteen minutes in hot oven. By brushing tops of
rings with beaten yolk of egg diluted with one teaspoonful water, they
will have a glazed appearance. Cool, and fill with jam or jelly.



\needspace{15\baselineskip}
\section*{Polish Tartlets}

Roll puff or plain paste one-eighth inch thick, and cut in two and
one-half inch squares; wet the corners, fold toward the centre, and
press lightly; bake on a sheet; when cool, press down the centres and
fill, using two-thirds quince marmalade and one-third currant jelly.



\needspace{15\baselineskip}
\section*{Almond Tartlets}

Line patty pans with puff or plain paste, fill with the following
mixture, and bake in a moderate oven until firm.

Blanch and finely chop one-third pound Jordan almonds. Add two
tablespoons cracker rolled and sifted, three eggs slightly beaten,
one-third cup sugar, one-third teaspoon salt, two cups milk, and
one-half teaspoon vanilla.



\needspace{15\baselineskip}
\section*{Peach Crusts}

Roll puff or plain paste one-eighth inch thick, cut in two and one-half
inch squares, and bake in hot oven. Cool, press down the centres, and
arrange on each one-half a canned peach drained from syrup and heated in
oven. Sprinkle with powdered sugar and put brandy in each cavity. Light
just before sending to table.



\needspace{15\baselineskip}
\section*{Malaga Boats}

Roll puff or plain paste one-eighth inch thick, line individual
boat-shaped tins, prick, and half fill with rice or barley to keep
pastry in desired shape. Bake in a hot oven. Remove from tins and cover
bottom of boats with marmalade, and on marmalade arrange three or four
malaga grapes cooked in syrup five minutes.



\needspace{15\baselineskip}
\section*{Calvé Tarts}

Roll puff or plain paste one-eighth inch thick, and cut in rounds of
correct size to cover inverted circular tins. Cover tins with paste,
prick several times, and bake until delicately browned. Place one-half a
canned peach in each case and fill each cavity with one-half a blanched
Jordan almond.



\needspace{15\baselineskip}
\section*{Fruit Baskets}

Bake plain paste over inverted patty pans. Roll paste one-eighth inch
thick, and cut in strips one-fourth inch wide. Twist strips in pairs and
bake over a one-fourth pound baking-powder box, thus making handles.
Fill cases with sliced peaches sprinkled generously with sugar, insert
handles, garnish with whipped cream and peach leaves. Strawberries,
raspberries, or other fruit may be used in place of peaches.



\needspace{15\baselineskip}
\section*{Lemon Tartlets}

Bake paste as for Fruit Baskets without handles. Fill with Lemon Pie II
mixture, cover with Meringue II, and bake until meringue is delicately
browned.



\needspace{15\baselineskip}
\section*{Meringues}


\needspace{15\baselineskip}
\subsection*{For Pies, Puddings, and Desserts}

Eggs for meringues should be thoroughly chilled, and beaten with silver
fork, wire spoon, or whisk. Where several eggs are needed, much time is
saved by using a whisk. Meringues on pies, puddings, or desserts may be
spread evenly, spread and piled in the centre, put on lightly by
spoonfuls, or spread evenly with part of the mixture, the remainder
being forced through a pastry bag and tube.

Meringues I and III should be baked fifteen minutes in slow oven.
Meringue II should be cooked eight minutes in moderate oven; if removed
from oven before cooked, the eggs will liquefy and meringue settle; if
cooked too long, meringue is tough.



\needspace{15\baselineskip}
\subsection*{Meringue I}


\begin{itemize}
\setlength{\itemsep}{0pt}
\setlength{\parsep}{0pt}
\item 2 egg whites
\item 2 tablespoons powdered sugar
\item 1/2 tablespoon lemon juice or
\item 1/4 teaspoon vanilla
\end{itemize}

\vspace{-0.5em}
\noindent%
Beat whites until stiff, add sugar gradually and continue beating, then
add flavoring.



\needspace{15\baselineskip}
\subsection*{Meringue II}


\begin{itemize}
\setlength{\itemsep}{0pt}
\setlength{\parsep}{0pt}
\item 3 egg whites
\item 7 1/2 tablespoons powdered sugar
\item 1/2 teaspoon lemon extract or
\item 1/3 teaspoon vanilla
\end{itemize}

\vspace{-0.5em}
\noindent%
Beat whites until stiff, add four tablespoons sugar gradually, and beat
vigorously; fold in remaining sugar, and add flavoring.



\needspace{15\baselineskip}
\subsection*{Meringue III}


\begin{itemize}
\setlength{\itemsep}{0pt}
\setlength{\parsep}{0pt}
\item 4 egg whitess
\item 7/8 cup powdered sugar
\item 2 tablespoons lemon juice
\end{itemize}

\vspace{-0.5em}
\noindent%
Put whites of eggs and sugar in bowl, beat mixture until stiff enough to
hold its shape, add lemon juice drop by drop, continuing the beating. It
will take thirty minutes to beat mixture sufficiently stiff to hold its
shape, but when baked it makes a most delicious meringue.



\needspace{15\baselineskip}
\subsection*{Meringues Glacées, or Kisses}


\begin{itemize}
\setlength{\itemsep}{0pt}
\setlength{\parsep}{0pt}
\item 4 egg whitess
\item 1/2 teaspoon vanilla
\item 1 1/4 cups powdered sugar or
\item 1 cup fine granulated
\end{itemize}

\vspace{-0.5em}
\noindent%
Beat whites until stiff, add gradually two-thirds of sugar, and continue
beating until mixture will hold its shape; fold in remaining sugar, and
add flavoring. Shape with a spoon or pastry bag and tube on wet board
covered with letter paper. Bake thirty minutes in very slow oven, remove
from paper, and put together in pairs, or if intending to fill with
whipped cream or ice cream remove soft part with spoon and place
meringues in oven to dry.



\needspace{15\baselineskip}
\subsection*{Nut Meringues}

To Meringue Glacée mixture add chopped nut meat; almonds, English
walnuts, or hickory nuts are preferred. Shape by dropping mixture from
tip of spoon in small piles one-half inch apart, or by using pastry bag
and tube. Sprinkle with nut meat, and bake.



\needspace{15\baselineskip}
\subsection*{Meringues (Mushrooms)}

Shape Meringue Glacée mixture in rounds the size of mushroom caps, using
pastry bag and tube; sprinkle with grated chocolate. Shape stems like
mushroom stems. Bake, remove from paper, and place caps on stems.



\needspace{15\baselineskip}
\subsection*{Meringues Panachées}

Fill Meringues Glacées with ice cream, or ice cream and water ice.
Garnish with whipped cream forced through pastry bag and tube, and
candied cherries.



\needspace{15\baselineskip}
\subsection*{Creole Kisses}


\begin{minipage}{1.0\textwidth}
{\setlength{\multicolsep}{0pt}\setlength{\columnsep}{2em}\raggedcolumns%
\begin{multicols}{2}
\begin{itemize}
\setlength{\itemsep}{0pt}
\setlength{\parsep}{0pt}
\item 1/2 lb. Jordan almonds
\item 1/4 cup boiling water
\item 1/2 cup sugar
\item 4 egg whitess
\item 1 1/4 cups powdered sugar
\item 1/2 teaspoon vanilla
\item 1/4 teaspoon salt
\end{itemize}
\end{multicols}}
\end{minipage}

\vspace{0.3em}
\noindent%
Blanch almonds, finely shred one-half of them, and dry slowly in oven.
Put water and sugar in a saucepan, and as soon as boiling-point is
reached, add remaining almonds, and cook until the syrup is of a golden
brown color. Turn into a pan, cool, and finely pound in mortar. Beat
whites of eggs until stiff, add gradually sugar, then vanilla, almonds,
and salt. Shape, sprinkle with shredded almonds, sift sugar over them,
and bake in a slow oven twenty-five minutes.





\chapter{Gingerbreads, Cookies, And Wafers}




\needspace{15\baselineskip}
\section*{Hot Water Gingerbread}


\begin{minipage}{1.0\textwidth}
{\setlength{\multicolsep}{0pt}\setlength{\columnsep}{2em}\raggedcolumns%
\begin{multicols}{2}
\begin{itemize}
\setlength{\itemsep}{0pt}
\setlength{\parsep}{0pt}
\item 1 cup molasses
\item 1/2 cup boiling water
\item 2 1/4 cups flour
\item 1 teaspoon soda
\item 1 1/2 teaspoons ginger
\item 1/2 teaspoon salt
\item 4 tablespoons melted butter
\end{itemize}
\end{multicols}}
\end{minipage}

\vspace{0.3em}
\noindent%
Add water to molasses. Mix and sift dry ingredients, combine mixtures,
add butter, and beat vigorously. Pour into a buttered shallow pan, and
bake twenty-five minutes in a moderate oven. Chicken fat dried out and
clarified furnishes an excellent shortening, and may be used in place of
butter.



\needspace{15\baselineskip}
\section*{Sour Milk Gingerbread}


\begin{minipage}{1.0\textwidth}
{\setlength{\multicolsep}{0pt}\setlength{\columnsep}{2em}\raggedcolumns%
\begin{multicols}{2}
\begin{itemize}
\setlength{\itemsep}{0pt}
\setlength{\parsep}{0pt}
\item 1 cup molasses
\item 1 cup sour milk
\item 2 1/3 cups flour
\item 1 3/4 teaspoons soda
\item 2 teaspoons ginger
\item 1/2 teaspoon salt
\item 1/4 cup melted butter
\end{itemize}
\end{multicols}}
\end{minipage}

\vspace{0.3em}
\noindent%
Mix soda with sour milk and add to molasses. Sift together remaining dry
ingredients, combine mixtures, add butter, and beat vigorously. Pour
into a buttered shallow pan, and bake twenty-five minutes in a moderate
oven.



\needspace{15\baselineskip}
\section*{Soft Molasses Gingerbread}


\begin{minipage}{1.0\textwidth}
{\setlength{\multicolsep}{0pt}\setlength{\columnsep}{2em}\raggedcolumns%
\begin{multicols}{2}
\begin{itemize}
\setlength{\itemsep}{0pt}
\setlength{\parsep}{0pt}
\item 1 cup molasses
\item 1/3 cup butter
\item 1 3/4 teaspoons soda
\item 1/2 cup sour milk
\item 1 egg
\item 2 cups flour
\item 2 teaspoons ginger
\item 1/2 teaspoon salt
\end{itemize}
\end{multicols}}
\end{minipage}

\vspace{0.3em}
\noindent%
Put butter and molasses in saucepan and cook until boiling-point is
reached. Remove from fire, add soda, and beat vigorously. Then add milk,
egg well beaten, and remaining ingredients mixed and sifted. Bake
fifteen minutes in buttered small tin pans, having pans two-thirds
filled with mixture.



\needspace{15\baselineskip}
\section*{Cambridge Gingerbread}


\begin{minipage}{1.0\textwidth}
{\setlength{\multicolsep}{0pt}\setlength{\columnsep}{2em}\raggedcolumns%
\begin{multicols}{2}
\begin{itemize}
\setlength{\itemsep}{0pt}
\setlength{\parsep}{0pt}
\item 1/3 cup butter
\item 2/3 cup boiling water
\item 1 cup molasses
\item 1 egg
\item 2 3/4 cups flour
\item 1 1/2 teaspoons soda
\item 1/2 teaspoon salt
\item 1 teaspoon cinnamon
\item 1 teaspoon ginger
\item 1/4 teaspoon clove
\end{itemize}
\end{multicols}}
\end{minipage}

\vspace{0.3em}
\noindent%
Melt butter in water, add molasses, egg well beaten, and dry ingredients
mixed and sifted. Bake in a buttered shallow pan.



\needspace{15\baselineskip}
\section*{Soft Sugar Gingerbread}


\begin{minipage}{1.0\textwidth}
{\setlength{\multicolsep}{0pt}\setlength{\columnsep}{2em}\raggedcolumns%
\begin{multicols}{2}
\begin{itemize}
\setlength{\itemsep}{0pt}
\setlength{\parsep}{0pt}
\item 2 eggs
\item 1 cup sugar
\item 1 3/4 cups flour
\item 3 teaspoons baking powder
\item 1/2 teaspoon salt
\item 1 1/2 teaspoons ginger
\item 2/3 cup thin cream
\end{itemize}
\end{multicols}}
\end{minipage}

\vspace{0.3em}
\noindent%
Beat eggs until light, and add sugar gradually. Mix and sift dry
ingredients, and add alternately with cream to first mixture. Turn into
a buttered cake pan, and bake thirty minutes in a moderate oven.



\needspace{15\baselineskip}
\section*{Gossamer Gingerbread}


\begin{minipage}{1.0\textwidth}
{\setlength{\multicolsep}{0pt}\setlength{\columnsep}{2em}\raggedcolumns%
\begin{multicols}{2}
\begin{itemize}
\setlength{\itemsep}{0pt}
\setlength{\parsep}{0pt}
\item 1/3 cup butter
\item 1 cup sugar
\item 1 egg
\item 1/2 cup milk
\item 1 7/8 cups flour
\item 3 teaspoons baking powder
\item 1 teaspoon yellow ginger
\end{itemize}
\end{multicols}}
\end{minipage}

\vspace{0.3em}
\noindent%
Cream the butter, add sugar gradually, then egg well beaten. Add milk,
and dry ingredients mixed and sifted. Spread in a buttered dripping-pan
as thinly as possible, using the back of mixing-spoon. Bake fifteen
minutes. Sprinkle with sugar, and cut in small squares or diamonds
before removing from pan.



\needspace{15\baselineskip}
\section*{Fairy Gingerbread}


\begin{itemize}
\setlength{\itemsep}{0pt}
\setlength{\parsep}{0pt}
\item 1/2 cup butter
\item 1 cup light brown sugar
\item 1/2 cup milk
\item 1 7/8 cups bread flour
\item 2 teaspoons ginger
\end{itemize}

\vspace{-0.5em}
\noindent%
Cream the butter, add sugar gradually, and milk very slowly. Mix and
sift flour and ginger, and combine mixtures. Spread very thinly with a
broad, long-bladed knife on a buttered, inverted dripping-pan. Bake in a
moderate oven. Cut in squares before removing from pan. Watch carefully
and turn pan frequently during baking, that all may be evenly cooked. If
mixture around edge of pan is cooked before that in the centre, pan
should be removed from oven, cooked part cut off, and remainder returned
to oven to finish cooking.



\needspace{15\baselineskip}
\section*{Hard Sugar Gingerbread}


\begin{minipage}{1.0\textwidth}
{\setlength{\multicolsep}{0pt}\setlength{\columnsep}{2em}\raggedcolumns%
\begin{multicols}{2}
\begin{itemize}
\setlength{\itemsep}{0pt}
\setlength{\parsep}{0pt}
\item 3/4 cup butter
\item 1 1/2 cups sugar
\item 3/4 cup milk
\item 5 cups flour
\item 3/4 tablespoon baking powder
\item 1 1/2 teaspoons salt
\item 3/4 tablespoon ginger
\end{itemize}
\end{multicols}}
\end{minipage}

\vspace{0.3em}
\noindent%
Cream the butter, add sugar gradually, milk, and dry ingredients mixed
and sifted. Put some of mixture on an inverted dripping-pan and roll as
thinly as possible to cover pan. Mark dough with a coarse grater.
Sprinkle with sugar and bake in a moderate oven. Before removing from
pan, cut in strips four and one-half inches long by one and one-half
inches wide.



\needspace{15\baselineskip}
\section*{Christmas English Gingerbread}


\begin{minipage}{1.0\textwidth}
{\setlength{\multicolsep}{0pt}\setlength{\columnsep}{2em}\raggedcolumns%
\begin{multicols}{2}
\begin{itemize}
\setlength{\itemsep}{0pt}
\setlength{\parsep}{0pt}
\item 1 lb. flour
\item 1/2 lb. butter
\item 1 cup sugar
\item 1 tablespoon ginger
\item 1 teaspoon salt
\item Molasses
\end{itemize}
\end{multicols}}
\end{minipage}

\vspace{0.3em}
\noindent%
Mix flour, sugar, ginger, and salt. Work in butter, using tips of
fingers, and add just enough molasses to hold ingredients together. Let
stand over night to get thoroughly chilled. Roll very thin, shape, and
bake in a moderate oven.



\needspace{15\baselineskip}
\section*{Card Gingerbread}


\begin{minipage}{1.0\textwidth}
{\setlength{\multicolsep}{0pt}\setlength{\columnsep}{2em}\raggedcolumns%
\begin{multicols}{2}
\begin{itemize}
\setlength{\itemsep}{0pt}
\setlength{\parsep}{0pt}
\item 1/3 cup butter
\item 1/3 cup brown sugar
\item 1 egg
\item 1/2 cup molasses
\item 1 3/4 cups flour
\item 1/2 tablespoon ginger
\item 3/4 teaspoon salt
\item 1/2 teaspoon soda
\item 1/4 teaspoon cinnamon
\end{itemize}
\end{multicols}}
\end{minipage}

\vspace{0.3em}
\noindent%
Cream the butter, add sugar gradually, egg well beaten, molasses, and
flour mixed and sifted with ginger, salt, soda, and cinnamon. Chill,
roll in sheets to one-fourth inch in thickness, bake on a buttered
sheet, and cut in squares.



\needspace{15\baselineskip}
\section*{Walnut Molasses Bars}


\begin{minipage}{1.0\textwidth}
{\setlength{\multicolsep}{0pt}\setlength{\columnsep}{2em}\raggedcolumns%
\begin{multicols}{2}
\begin{itemize}
\setlength{\itemsep}{0pt}
\setlength{\parsep}{0pt}
\item 1/4 cup butter
\item 1/4 cup lard
\item 1/4 cup boiling water
\item 1/2 cup brown sugar
\item 1/2 cup molasses
\item 1 teaspoon soda
\item 3 cups flour
\item 1/2 tablespoon ginger
\item 1/3 teaspoon grated nutmeg
\item 1/8 teaspoon clove
\item 1 teaspoon salt
\item Chopped walnut meat
\end{itemize}
\end{multicols}}
\end{minipage}

\vspace{0.3em}
\noindent%
Pour water over butter and lard, then add sugar, molasses mixed with
soda, flour, salt, and spices. Chill thoroughly, roll one-fourth inch
thick, cut in strips three and one-half inches long by one and one-half
inches wide. Sprinkle with nut meat and bake ten minutes.



\needspace{15\baselineskip}
\section*{Ginger Snaps}


\begin{minipage}{1.0\textwidth}
{\setlength{\multicolsep}{0pt}\setlength{\columnsep}{2em}\raggedcolumns%
\begin{multicols}{2}
\begin{itemize}
\setlength{\itemsep}{0pt}
\setlength{\parsep}{0pt}
\item 1 cup molasses
\item 1/2 cup shortening
\item 3 1/4 cups flour
\item 1/2 teaspoon soda
\item 1 tablespoon ginger
\item 1 1/2 teaspoons salt
\end{itemize}
\end{multicols}}
\end{minipage}

\vspace{0.3em}
\noindent%
Heat molasses to boiling-point and pour over shortening. Add dry
ingredients mixed and sifted. Chill thoroughly. Toss one-fourth of
mixture on a floured board and roll as thinly as possible; shape with a
small round cutter, first dipped in flour. Place near together on a
buttered sheet and bake in a moderate oven. Gather up the trimmings and
roll with another portion of dough. During rolling, the bowl containing
mixture should be kept in a cool place, or it will be necessary to add
more flour to dough, which makes cookies hard rather than crisp and
short.



\needspace{15\baselineskip}
\section*{Molasses Cookies}


\begin{minipage}{1.0\textwidth}
{\setlength{\multicolsep}{0pt}\setlength{\columnsep}{2em}\raggedcolumns%
\begin{multicols}{2}
\begin{itemize}
\setlength{\itemsep}{0pt}
\setlength{\parsep}{0pt}
\item 1 cup molasses
\item 1/2 cup shortening, butter and lard mixed
\item 2 1/2 cups bread flour
\item 1 tablespoon ginger
\item 1 tablespoon soda
\item 2 tablespoons warm milk
\item 1 teaspoon salt
\end{itemize}
\end{multicols}}
\end{minipage}

\vspace{0.3em}
\noindent%
Heat molasses to boiling-point, add shortening, ginger, soda dissolved
in warm milk, salt, and flour. Proceed as for Ginger Snaps.



\needspace{15\baselineskip}
\section*{Soft Molasses Cookies}


\begin{minipage}{1.0\textwidth}
{\setlength{\multicolsep}{0pt}\setlength{\columnsep}{2em}\raggedcolumns%
\begin{multicols}{2}
\begin{itemize}
\setlength{\itemsep}{0pt}
\setlength{\parsep}{0pt}
\item 1 cup molasses
\item 1 3/4 teaspoons soda
\item 1 cup sour milk
\item 1/2 cup shortening, melted
\item 2 teaspoons ginger
\item 1 teaspoon salt
\item Flour
\end{itemize}
\end{multicols}}
\end{minipage}

\vspace{0.3em}
\noindent%
Add soda to molasses and beat thoroughly; add milk, shortening, ginger,
salt, and flour. Enough flour must be used to make mixture of right
consistency to drop easily from spoon. Let stand several hours in a cold
place to thoroughly chill. Toss one-half mixture at a time on slightly
floured board and roll lightly to one-fourth inch thickness. Shape with
a round cutter, first dipped in flour. Bake on a buttered sheet.



\needspace{15\baselineskip}
\section*{Spice Cookies}


\begin{minipage}{1.0\textwidth}
{\setlength{\multicolsep}{0pt}\setlength{\columnsep}{2em}\raggedcolumns%
\begin{multicols}{2}
\begin{itemize}
\setlength{\itemsep}{0pt}
\setlength{\parsep}{0pt}
\item 1/2 cup molasses
\item 1/4 cup sugar
\item 1 1/2 tablespoons butter
\item 1 1/2 tablespoons lard
\item 1 tablespoon milk
\item 2 cups flour
\item 1/2 teaspoon soda
\item 1/2 teaspoon salt
\item 1/2 teaspoon clove
\item 1/2 teaspoon cinnamon
\item 1/2 teaspoon nutmeg
\end{itemize}
\end{multicols}}
\end{minipage}

\vspace{0.3em}
\noindent%
Heat molasses to boiling-point. Add sugar, shortening, and milk. Mix and
sift dry ingredients, and add to first mixture. Chill thoroughly, and
proceed as with Ginger Snaps.



\needspace{15\baselineskip}
\section*{Scotch Wafers}


\begin{minipage}{1.0\textwidth}
{\setlength{\multicolsep}{0pt}\setlength{\columnsep}{2em}\raggedcolumns%
\begin{multicols}{2}
\begin{itemize}
\setlength{\itemsep}{0pt}
\setlength{\parsep}{0pt}
\item 1 cup fine oatmeal
\item 1 cup Rolled Oats
\item 2 cups flour
\item 1/4 cup sugar
\item 1 teaspoon salt
\item 1/8 teaspoon soda
\item 1/4 cup butter or lard
\item 1/2 cup hot water
\end{itemize}
\end{multicols}}
\end{minipage}

\vspace{0.3em}
\noindent%
Mix first six ingredients. Melt shortening in water and add to first
mixture. Toss on a floured board, pat, and roll as thinly as possible.
Shape with a cutter, or with a sharp knife cut in strips. Bake on a
buttered sheet in a slow oven. These are well adapted for children's
luncheons, and are much enjoyed by the convalescent, taken with a glass
of milk.



\needspace{15\baselineskip}
\section*{Oatmeal Cookies}


\begin{minipage}{1.0\textwidth}
{\setlength{\multicolsep}{0pt}\setlength{\columnsep}{2em}\raggedcolumns%
\begin{multicols}{2}
\begin{itemize}
\setlength{\itemsep}{0pt}
\setlength{\parsep}{0pt}
\item 1 egg
\item 1/4 cup sugar
\item 1/4 cup thin cream
\item 1/4 cup milk
\item 1/2 cup fine oatmeal
\item 2 cups flour
\item 2 teaspoons baking powder
\item 1 teaspoon salt
\end{itemize}
\end{multicols}}
\end{minipage}

\vspace{0.3em}
\noindent%
Beat egg until light, add sugar, cream, and milk; then add oatmeal,
flour, baking powder, and salt, mixed and sifted. Toss on a floured
board, roll, cut in shape, and bake in a moderate oven.



\needspace{15\baselineskip}
\section*{Scottish Fancies}


\begin{minipage}{1.0\textwidth}
{\setlength{\multicolsep}{0pt}\setlength{\columnsep}{2em}\raggedcolumns%
\begin{multicols}{2}
\begin{itemize}
\setlength{\itemsep}{0pt}
\setlength{\parsep}{0pt}
\item 1 egg
\item 1/2 cup sugar
\item 2/3 tablespoon melted butter
\item 1 cup rolled oats
\item 1/3 teaspoon salt
\item 1/4 teaspoon vanilla
\end{itemize}
\end{multicols}}
\end{minipage}

\vspace{0.3em}
\noindent%
Beat egg until light, add gradually sugar, and then stir in remaining
ingredients. Drop mixture by teaspoonfuls on a thoroughly greased
inverted dripping-pan one inch apart. Spread into circular shape with a
case knife first dipped in cold water. Bake in a moderate oven until
delicately browned. To give variety use two-thirds cup rolled oats and
fill cup with shredded cocoanut.



\needspace{15\baselineskip}
\section*{Vanilla Wafers}


\begin{minipage}{1.0\textwidth}
{\setlength{\multicolsep}{0pt}\setlength{\columnsep}{2em}\raggedcolumns%
\begin{multicols}{2}
\begin{itemize}
\setlength{\itemsep}{0pt}
\setlength{\parsep}{0pt}
\item 1/3 cup butter and lard in equal proportions
\item 1 cup sugar
\item 1 egg
\item 1/4 cup milk
\item 2 cups flour
\item 2 teaspoons baking powder
\item 1/2 teaspoon salt
\item 2 teaspoons vanilla
\end{itemize}
\end{multicols}}
\end{minipage}

\vspace{0.3em}
\noindent%
Cream the butter, add sugar, egg well beaten, milk, and vanilla. Mix and
sift dry ingredients and add to first mixture. Proceed as with Ginger
Snaps.



\needspace{15\baselineskip}
\section*{Cream Cookies}


\begin{minipage}{1.0\textwidth}
{\setlength{\multicolsep}{0pt}\setlength{\columnsep}{2em}\raggedcolumns%
\begin{multicols}{2}
\begin{itemize}
\setlength{\itemsep}{0pt}
\setlength{\parsep}{0pt}
\item 1/3 cup butter
\item 1 cup sugar
\item 2 eggs
\item 1/2 cup thin cream
\item 2 teaspoons baking powder
\item 1 teaspoon salt
\item 2 teaspoons yellow ginger
\item Flour to roll
\end{itemize}
\end{multicols}}
\end{minipage}

\vspace{0.3em}
\noindent%
Mix and bake same as Vanilla Wafers.



\needspace{15\baselineskip}
\section*{Imperial Cookies}


\begin{minipage}{1.0\textwidth}
{\setlength{\multicolsep}{0pt}\setlength{\columnsep}{2em}\raggedcolumns%
\begin{multicols}{2}
\begin{itemize}
\setlength{\itemsep}{0pt}
\setlength{\parsep}{0pt}
\item 1/2 cup butter
\item 1 cup sugar
\item 2 eggs
\item 1 tablespoon milk
\item 2 1/2 cups flour
\item 2 teaspoons baking powder
\item 1/2 teaspoon lemon extract
\item 1/2 teaspoon grated nutmeg
\end{itemize}
\end{multicols}}
\end{minipage}

\vspace{0.3em}
\noindent%
Mix and bake same as Vanilla Wafers.



\needspace{15\baselineskip}
\section*{Hermits}


\begin{minipage}{1.0\textwidth}
{\setlength{\multicolsep}{0pt}\setlength{\columnsep}{2em}\raggedcolumns%
\begin{multicols}{2}
\begin{itemize}
\setlength{\itemsep}{0pt}
\setlength{\parsep}{0pt}
\item 1/3 cup butter
\item 2/3 cup sugar
\item 1 egg
\item 2 tablespoons milk
\item 1 3/4 cups flour
\item 2 teaspoons baking powder
\item 1/3 cup raisins, stoned and cut in small pieces
\item 1/2 teaspoon cinnamon
\item 1/4 teaspoon clove
\item 1/4 teaspoon mace
\item 1/4 teaspoon nutmeg
\end{itemize}
\end{multicols}}
\end{minipage}

\vspace{0.3em}
\noindent%
Cream the butter, add sugar gradually, then raisins, egg well beaten,
and milk. Mix and sift dry ingredients and add to first mixture. Roll
mixture a little thicker than for Vanilla Wafers.



\needspace{15\baselineskip}
\section*{Rich Cookies}


\begin{minipage}{1.0\textwidth}
{\setlength{\multicolsep}{0pt}\setlength{\columnsep}{2em}\raggedcolumns%
\begin{multicols}{2}
\begin{itemize}
\setlength{\itemsep}{0pt}
\setlength{\parsep}{0pt}
\item 1/2 cup butter
\item 1/3 cup sugar
\item 1 egg well beaten
\item 3/4 cup flour
\item 1/2 teaspoon vanilla
\item Raisins, nuts, or citron
\end{itemize}
\end{multicols}}
\end{minipage}

\vspace{0.3em}
\noindent%
Cream the butter, add sugar gradually, egg, flour, and vanilla. Drop
from tip of spoon in small portions on buttered sheet two inches apart.
Spread thinly with a knife first dipped in cold water. Put four Sultana
raisins on each cookie, almonds blanched and cut in strips, or citron
cut in small pieces.



\needspace{15\baselineskip}
\section*{Jelly Jumbles}


\begin{minipage}{1.0\textwidth}
{\setlength{\multicolsep}{0pt}\setlength{\columnsep}{2em}\raggedcolumns%
\begin{multicols}{2}
\begin{itemize}
\setlength{\itemsep}{0pt}
\setlength{\parsep}{0pt}
\item 1/2 cup butter
\item 1 cup sugar
\item 1 egg
\item 1/2 teaspoon soda
\item 1/2 cup sour milk
\item 1/4 teaspoon salt
\item Flour
\item Currant jelly
\end{itemize}
\end{multicols}}
\end{minipage}

\vspace{0.3em}
\noindent%
Cream the butter, add sugar gradually, egg well beaten, soda mixed with
milk, salt and flour to make a soft dough. Chill and shape, using a
round cutter. On the centres of one-half the pieces put currant jelly.
Make three small openings in remaining halves, using a thimble, and put
pieces together. Press edges slightly, and bake in a rather hot oven,
that jumbles may keep in good shape.







\needspace{15\baselineskip}
\section*{Royal Fans}

Mix and sift two cups flour and one-half cup brown sugar. Wash
three-fourths cup butter and work into first mixture, using tips of
fingers. Roll to one-third inch in thickness, shape with a fluted round
cutter five inches in diameter. Cut each piece in quarters and crease
with the dull edge of a case knife to represent folds of a fan. Brush
over with yolk of egg diluted with three-fourths teaspoon water. Bake in
a slow oven.



\needspace{15\baselineskip}
\section*{Boston Cookies}


\begin{minipage}{1.0\textwidth}
{\setlength{\multicolsep}{0pt}\setlength{\columnsep}{2em}\raggedcolumns%
\begin{multicols}{2}
\begin{itemize}
\setlength{\itemsep}{0pt}
\setlength{\parsep}{0pt}
\item 1 cup butter
\item 1 1/2 cups sugar
\item 3 eggs
\item 1 teaspoon soda
\item 1 1/2 tablespoons hot water
\item 3 1/4 cups flour
\item 1/2 teaspoon salt
\item 1 teaspoon cinnamon
\item 1 cup chopped nut meat, hickory or English walnut
\item 1/2 cup currants
\item 1/2 cup raisins, seeded and chopped
\end{itemize}
\end{multicols}}
\end{minipage}

\vspace{0.3em}
\noindent%
Cream the butter, add sugar gradually, and eggs well beaten. Add soda
dissolved in hot water, and one-half the flour mixed and sifted with
salt and cinnamon; then add nut meat, fruit, and remaining flour. Drop
by spoonfuls one inch apart on a buttered sheet, and bake in a moderate
oven.



\needspace{15\baselineskip}
\section*{Cocoanut Cream Cookies}


\begin{minipage}{1.0\textwidth}
{\setlength{\multicolsep}{0pt}\setlength{\columnsep}{2em}\raggedcolumns%
\begin{multicols}{2}
\begin{itemize}
\setlength{\itemsep}{0pt}
\setlength{\parsep}{0pt}
\item 2 eggs
\item 1 cup sugar
\item 1 cup thick cream
\item 1/2 cup shredded cocoanut
\item 3 cups flour
\item 3 teaspoons baking powder
\item 1 teaspoon salt
\end{itemize}
\end{multicols}}
\end{minipage}

\vspace{0.3em}
\noindent%
Beat eggs until light, add sugar gradually, cocoanut, cream, and flour
mixed and sifted with baking powder and salt. Chill thoroughly, toss on
a floured board, pat, and roll one-half inch thick. Sprinkle with
cocoanut, roll one-fourth inch thick, and shape with a small round
cutter, first dipped in flour. Bake on a buttered sheet in a moderate
oven.



\needspace{15\baselineskip}
\section*{Peanut Cookies}


\begin{minipage}{1.0\textwidth}
{\setlength{\multicolsep}{0pt}\setlength{\columnsep}{2em}\raggedcolumns%
\begin{multicols}{2}
\begin{itemize}
\setlength{\itemsep}{0pt}
\setlength{\parsep}{0pt}
\item 2 tablespoons butter
\item 1/4 cup sugar
\item 1 egg
\item 1 teaspoon baking powder
\item 1/4 teaspoon salt
\item 1/2 cup flour
\item 2 tablespoons milk
\item 1/2 cup finely chopped peanuts
\item 1/2 teaspoon lemon juice
\end{itemize}
\end{multicols}}
\end{minipage}

\vspace{0.3em}
\noindent%
Cream the butter, add sugar, and egg well beaten. Mix and sift baking
powder, salt, and flour; add to first mixture; then add milk, peanuts,
and lemon juice. Drop from a teaspoon on an unbuttered sheet one inch
apart, and place one-half peanut on top of each. Bake twelve to fifteen
minutes in a slow oven. This recipe will make twenty-four cookies. One
pint peanuts when shelled should yield one-half cup.



\needspace{15\baselineskip}
\section*{Almond Cookies}


\begin{minipage}{1.0\textwidth}
{\setlength{\multicolsep}{0pt}\setlength{\columnsep}{2em}\raggedcolumns%
\begin{multicols}{2}
\begin{itemize}
\setlength{\itemsep}{0pt}
\setlength{\parsep}{0pt}
\item 1/2 cup butter
\item 1 egg
\item 1/3 cup almonds, blanched and finely chopped
\item 1/2 cup sugar
\item 1/2 tablespoon cinnamon
\item 1/2 tablespoon clove
\item 1/2 tablespoon nutmeg
\item Grated rind 1/2 lemon
\item 2 tablespoons brandy
\item 2 cups flour
\end{itemize}
\end{multicols}}
\end{minipage}

\vspace{0.3em}
\noindent%
Cream the butter, add egg well beaten, almonds, sugar, brandy, and
spices mixed and sifted with flour. Roll mixture to one-fourth inch in
thickness, shape with a round cutter first dipped in flour, and bake in
a slow oven.



\needspace{15\baselineskip}
\section*{Nut Cookies}


\begin{minipage}{1.0\textwidth}
{\setlength{\multicolsep}{0pt}\setlength{\columnsep}{2em}\raggedcolumns%
\begin{multicols}{2}
\begin{itemize}
\setlength{\itemsep}{0pt}
\setlength{\parsep}{0pt}
\item 4 egg yolks
\item 1 cup brown sugar
\item 1 cup chopped nut meats
\item 2 egg whites
\item 6 tablespoons flour
\item Few grains salt
\end{itemize}
\end{multicols}}
\end{minipage}

\vspace{0.3em}
\noindent%
Beat yolks of eggs until thick and lemon-colored, add sugar gradually,
nut meats, whites of egg beaten until stiff, and flour mixed with salt.
Drop from tip of spoon on buttered sheet, spread, and bake in a moderate
oven.



\needspace{15\baselineskip}
\section*{Seed Cakes}

Follow recipe for Cocoanut Cream Cookies (see p. 489), using one and
one-half tablespoons caraway seeds in place of cocoanut.



\needspace{15\baselineskip}
\section*{Chocolate Cookies}


\begin{minipage}{1.0\textwidth}
{\setlength{\multicolsep}{0pt}\setlength{\columnsep}{2em}\raggedcolumns%
\begin{multicols}{2}
\begin{itemize}
\setlength{\itemsep}{0pt}
\setlength{\parsep}{0pt}
\item 1/2 cup butter
\item 1 cup sugar
\item 1 egg
\item 1/4 teaspoon salt
\item 2 ozs. Baker's chocolate
\item 2 1/2 cups flour (scant)
\item 2 teaspoons baking powder
\item 1/4 cup milk
\end{itemize}
\end{multicols}}
\end{minipage}

\vspace{0.3em}
\noindent%
Cream the butter, add sugar gradually, egg well beaten, salt, and
chocolate melted. Beat well, and add flour mixed and sifted with baking
powder alternately with milk. Chill, roll very thin, then shape with a
small cutter, first dipped in flour, and bake in a moderate oven.



\needspace{15\baselineskip}
\section*{German Chocolate Cookies}


\begin{minipage}{1.0\textwidth}
{\setlength{\multicolsep}{0pt}\setlength{\columnsep}{2em}\raggedcolumns%
\begin{multicols}{2}
\begin{itemize}
\setlength{\itemsep}{0pt}
\setlength{\parsep}{0pt}
\item 2 eggs
\item 1 cup brown sugar
\item 2 bars German chocolate
\item 1/4 teaspoon cinnamon
\item 1/2 teaspoon salt
\item Grated rind 1/2 lemon
\item 1 1/3 cups almonds, blanched and chopped
\item 1 cup flour
\item 1 teaspoon baking powder
\end{itemize}
\end{multicols}}
\end{minipage}

\vspace{0.3em}
\noindent%
Beat eggs until light, add sugar, gradually, and continue the beating;
then add chocolate, grated, and remaining ingredients. Drop from tip of
spoon on a buttered sheet, and bake in a moderate oven.



\needspace{15\baselineskip}
\section*{Chocolate Fruit Cookies}


\begin{minipage}{1.0\textwidth}
{\setlength{\multicolsep}{0pt}\setlength{\columnsep}{2em}\raggedcolumns%
\begin{multicols}{2}
\begin{itemize}
\setlength{\itemsep}{0pt}
\setlength{\parsep}{0pt}
\item 1/4 cup butter
\item 1/2 cup sugar
\item 2 tablespoons grated chocolate
\item 1 tablespoon sugar
\item 1 tablespoon boiling water
\item 1 egg
\item 1/2 cup nut meats, finely chopped
\item 1/2 cup seeded raisins, finely chopped
\item 1 cup flour
\item 1 teaspoon baking powder
\end{itemize}
\end{multicols}}
\end{minipage}

\vspace{0.3em}
\noindent%
Cream the butter, and add sugar, gradually. Melt chocolate, add
remaining sugar and water, and cook one minute. Combine mixtures, and
add remaining ingredients. Chill, shape, and bake same as Chocolate
Cookies.



\needspace{15\baselineskip}
\section*{Chocolate Cakes}


\begin{minipage}{1.0\textwidth}
{\setlength{\multicolsep}{0pt}\setlength{\columnsep}{2em}\raggedcolumns%
\begin{multicols}{2}
\begin{itemize}
\setlength{\itemsep}{0pt}
\setlength{\parsep}{0pt}
\item 3 eggs
\item 1/4 cup butter
\item 1/2 cup sugar
\item 3 squares Baker's chocolate
\item 1 cup stale bread crumbs
\item 3 tablespoons flour
\end{itemize}
\end{multicols}}
\end{minipage}

\vspace{0.3em}
\noindent%
Beat eggs until light. Cream the butter, add sugar, combine mixtures,
then add chocolate melted, bread crumbs, and flour. Spread mixture in a
shallow buttered pan and bake in a slow oven. Shape with a tiny
biscuit-cutter and put together in pairs with White Mountain Cream (see
p. 528) between and on top.



\needspace{15\baselineskip}
\section*{Neuremburghs}


\begin{minipage}{1.0\textwidth}
{\setlength{\multicolsep}{0pt}\setlength{\columnsep}{2em}\raggedcolumns%
\begin{multicols}{2}
\begin{itemize}
\setlength{\itemsep}{0pt}
\setlength{\parsep}{0pt}
\item 2 eggs
\item 1/2 cup powdered sugar
\item 3/4 cup flour
\item 1/3 teaspoon salt
\item 1/3 teaspoon cinnamon
\item 1/8 teaspoon clove
\item 1 tablespoon orange peel, finely cut
\item Grated rind 1/2 lemon
\item 3/4 cup Jordan almonds
\end{itemize}
\end{multicols}}
\end{minipage}

\vspace{0.3em}
\noindent%
Beat the whites of the eggs until stiff, and add sugar gradually,
continuing the beating. Then add yolks of eggs well beaten, flour mixed
and sifted with salt and spices, orange peel, and lemon rind. Blanch
almonds, cut in small pieces crosswise, and bake in a slow oven until
well browned. Fold into the mixture, and drop by spoonfuls on a sheet
dredged with corn-starch and powdered sugar in equal proportions. Bake
in a moderate oven.



\needspace{15\baselineskip}
\section*{Sand Tarts}


\begin{minipage}{1.0\textwidth}
{\setlength{\multicolsep}{0pt}\setlength{\columnsep}{2em}\raggedcolumns%
\begin{multicols}{2}
\begin{itemize}
\setlength{\itemsep}{0pt}
\setlength{\parsep}{0pt}
\item 1/2 cup butter
\item 1 cup sugar
\item 1 egg
\item 1 3/4 cups flour
\item 2 teaspoons baking powder
\item 1 egg white
\item Blanched almonds
\item 1 tablespoon sugar
\item 1/4 teaspoon cinnamon
\end{itemize}
\end{multicols}}
\end{minipage}

\vspace{0.3em}
\noindent%
Cream the butter, add sugar gradually, and egg well beaten; then add
flour mixed and sifted with baking powder. Chill, toss one-half mixture
on a floured board, and roll one-eighth inch thick. Shape with a
doughnut cutter. Brush over with white of egg, and sprinkle with sugar
mixed with cinnamon. Split almonds, and arrange three halves on each at
equal distances. Place on a buttered sheet, and bake eight minutes in a
slow oven.



\needspace{15\baselineskip}
\section*{Swedish Wafers}


\begin{minipage}{1.0\textwidth}
{\setlength{\multicolsep}{0pt}\setlength{\columnsep}{2em}\raggedcolumns%
\begin{multicols}{2}
\begin{itemize}
\setlength{\itemsep}{0pt}
\setlength{\parsep}{0pt}
\item 1/2 cup butter
\item 1/2 cup sugar
\item 2 eggs
\item 5 ozs. flour
\item 1/4 teaspoon vanilla
\item Shredded almonds
\end{itemize}
\end{multicols}}
\end{minipage}

\vspace{0.3em}
\noindent%
Cream the butter, add sugar gradually, eggs slightly beaten, flour, and
flavoring. Drop by spoonfuls on an inverted buttered dripping-pan.
Spread very thinly, using a knife, in circular shapes about three inches
in diameter. Sprinkle with almonds, and bake in a slow oven. Remove from
pan, and shape at once over the handle of a wooden spoon.



\needspace{15\baselineskip}
\section*{Marguerites I}


\begin{minipage}{1.0\textwidth}
{\setlength{\multicolsep}{0pt}\setlength{\columnsep}{2em}\raggedcolumns%
\begin{multicols}{2}
\begin{itemize}
\setlength{\itemsep}{0pt}
\setlength{\parsep}{0pt}
\item 2 eggs
\item 1 cup brown sugar
\item 1/2 cup flour
\item 1/4 teaspoon baking powder
\item 1/3 teaspoon salt
\item 1 cup pecan nut meats, cut in small pieces
\end{itemize}
\end{multicols}}
\end{minipage}

\vspace{0.3em}
\noindent%
Beat eggs slightly, and add remaining ingredients in the order given.
Fill small buttered tins two-thirds full of mixture, and place pecan nut
meat on each. Bake in a moderate oven fifteen minutes.



\needspace{15\baselineskip}
\section*{Marguerites II}


\begin{minipage}{1.0\textwidth}
{\setlength{\multicolsep}{0pt}\setlength{\columnsep}{2em}\raggedcolumns%
\begin{multicols}{2}
\begin{itemize}
\setlength{\itemsep}{0pt}
\setlength{\parsep}{0pt}
\item 1 cup sugar
\item 1/2 cup water
\item 5 marshmallows
\item 2 egg whites
\item 2 tablespoons shredded cocoanut
\item 1/4 teaspoon vanilla
\item 1 cup English walnut meats
\item Saltines
\end{itemize}
\end{multicols}}
\end{minipage}

\vspace{0.3em}
\noindent%
Boil sugar and water until syrup will thread. Remove to back of range
and add marshmallows cut in pieces. Pour onto the whites of eggs beaten
until stiff; then add cocoanut, vanilla, and nut meats. Spread saltines
with mixture and bake until delicately browned.



\needspace{15\baselineskip}
\section*{Kornettes}


\begin{minipage}{1.0\textwidth}
{\setlength{\multicolsep}{0pt}\setlength{\columnsep}{2em}\raggedcolumns%
\begin{multicols}{2}
\begin{itemize}
\setlength{\itemsep}{0pt}
\setlength{\parsep}{0pt}
\item 3/4 cup finely chopped popped corn
\item 3/4 tablespoon soft butter
\item 1 egg white
\item 1/3 cup sugar
\item 1/4 teaspoon salt
\item 1/2 teaspoon vanilla
\item Blanched and chopped almonds
\item Candied cherries
\end{itemize}
\end{multicols}}
\end{minipage}

\vspace{0.3em}
\noindent%
Add butter to corn. Beat egg white until stiff, and add sugar gradually,
continuing the beating. Combine mixtures; then add salt and vanilla.
Drop mixture from tip of spoon on a well buttered sheet, one inch apart.
Shape in circular form with case knife first dipped in cold water.
Sprinkle with almonds and place a piece of candied cherry on the centre
of each. Bake in a slow oven until delicately browned.



\needspace{15\baselineskip}
\section*{Rolled Wafers}


\begin{itemize}
\setlength{\itemsep}{0pt}
\setlength{\parsep}{0pt}
\item 1/4 cup butter
\item 1/2 cup powdered sugar
\item 1/4 cup milk
\item 7/8 cup bread flour
\item 1/2 teaspoon vanilla
\end{itemize}

\vspace{-0.5em}
\noindent%
Cream the butter, add sugar gradually, and milk drop by drop; then add
flour and flavoring. Spread very thinly with a broad, long-bladed knife
on a buttered inverted dripping-pan. Crease in three-inch squares, and
bake in a slow oven until delicately browned. Place pan on back of
range, cut squares apart with a sharp knife, and roll while warm in
tubular or cornucopia shape. If squares become too brittle to roll,
place in oven to soften. If rolled tubular shape, tie in bunches with
narrow ribbon. These are very attractive, and may be served with
sherbet, ice cream, or chocolate. If rolled cornucopia shape, they may
be filled with whipped cream just before sending to table. Colored
wafers may be made from this mixture by adding leaf green or fruit red.
If colored green, flavor with one-fourth teaspoon almond and
three-fourths teaspoon vanilla. If colored pink, flavor with rose.
Colored wafers must be baked in a very slow oven and turned frequently,
otherwise they will not be of the uniform color that is desired.



\needspace{15\baselineskip}
\section*{Almond Wafers}

Before baking Rolled Wafers, sprinkle with almonds blanched and chopped.
Other nut meats or shredded cocoanut may be used in place of almonds.



\needspace{15\baselineskip}
\section*{English Rolled Wafers I}


\begin{itemize}
\setlength{\itemsep}{0pt}
\setlength{\parsep}{0pt}
\item 1/2 cup molasses
\item 1/2 cup butter
\item 1 cup flour (scant)
\item 2/3 cup sugar
\item 1 tablespoon ginger
\end{itemize}

\vspace{-0.5em}
\noindent%
Heat molasses to boiling-point, add butter, then slowly, stirring
constantly, flour mixed and sifted with ginger and sugar. Drop small
portions from tip of spoon on a buttered inverted dripping-pan two
inches apart. Bake in a slow oven, cool slightly, remove from pan, and
roll over handle of wooden spoon.








\needspace{15\baselineskip}
\section*{English Rolled Wafers II}

To English Rolled Wafers I, add one and one-half cups rolled oats.



\needspace{15\baselineskip}
\section*{Nut Bars}


\begin{minipage}{1.0\textwidth}
{\setlength{\multicolsep}{0pt}\setlength{\columnsep}{2em}\raggedcolumns%
\begin{multicols}{2}
\begin{itemize}
\setlength{\itemsep}{0pt}
\setlength{\parsep}{0pt}
\item 2 tablespoons brown sugar
\item 1/4 cup butter
\item 1/4 cup boiling water
\item 1/2 cup brown sugar
\item 1/2 cup flour
\item 1/8 teaspoon salt
\item 2 tablespoons English walnut meat, finely chopped
\item Halves of walnuts or almonds
\end{itemize}
\end{multicols}}
\end{minipage}

\vspace{0.3em}
\noindent%
Caramelize two tablespoons sugar, add butter and water, and boil two
minutes. Remove from fire, add remaining sugar, flour mixed with salt,
and walnut meat. Spread as Rolled Wafers, crease in two-inch squares,
and decorate with nut meats. Bake in a slow oven, and remove from pan at
once.



\needspace{15\baselineskip}
\section*{Nut Macaroons}


\begin{itemize}
\setlength{\itemsep}{0pt}
\setlength{\parsep}{0pt}
\item 1 egg white
\item 1 cup brown sugar
\item 1 cup pecan nut meats
\item 1/4 teaspoon salt
\end{itemize}

\vspace{-0.5em}
\noindent%
Beat white of egg until light and add gradually, while beating
constantly, sugar. Fold in nut meats, finely chopped and sprinkled with
salt. Drop from tip of spoon, one inch apart, on an unbuttered sheet,
and bake in a moderate oven until delicately browned.



\needspace{15\baselineskip}
\section*{Brownies}


\begin{minipage}{1.0\textwidth}
{\setlength{\multicolsep}{0pt}\setlength{\columnsep}{2em}\raggedcolumns%
\begin{multicols}{2}
\begin{itemize}
\setlength{\itemsep}{0pt}
\setlength{\parsep}{0pt}
\item 1 cup sugar
\item 1/4 cup melted butter
\item 1 egg, unbeaten
\item 2 squares Baker's chocolate, melted
\item 3/4 teaspoon vanilla
\item 1/2 cup flour
\item 1/2 cup walnut meats, cut in pieces
\end{itemize}
\end{multicols}}
\end{minipage}

\vspace{0.3em}
\noindent%
Mix ingredients in order given. Line a seven-inch square pan with
paraffine paper. Spread mixture evenly in pan and bake in a slow oven.
As soon as taken from oven turn from pan, remove paper, and cut cake in
strips, using a sharp knife. If these directions are not followed paper
will cling to cake, and it will be impossible to cut it in shapely
pieces.



\needspace{15\baselineskip}
\section*{Card Cakes}


\begin{minipage}{1.0\textwidth}
{\setlength{\multicolsep}{0pt}\setlength{\columnsep}{2em}\raggedcolumns%
\begin{multicols}{2}
\begin{itemize}
\setlength{\itemsep}{0pt}
\setlength{\parsep}{0pt}
\item 1/3 cup butter
\item 1 cup powdered sugar
\item 2 eggs
\item 1 cup flour
\item 1/3 teaspoon salt
\item Jordan almonds
\item 1 tablespoon Breakfast Cocoa
\item 2 tablespoons sugar
\item 1/4 teaspoon powdered cinnamon
\item 1/4 teaspoon vanilla
\item Shredded cocoanut
\end{itemize}
\end{multicols}}
\end{minipage}

\vspace{0.3em}
\noindent%
Cream the butter, add sugar, eggs well beaten, flour, and salt. Spread
mixture on bottom of a buttered inverted dripping-pan, decorate with
almonds blanched and cut in strips, and bake in slow oven. Cut in
desired shape, using heart, spade, and diamond-shaped cutters before
removing from pan. To give variety, divide mixture in halves. To
one-half add sugar, cocoa, cinnamon, and vanilla, then spread on pan and
sprinkle with shredded cocoanut.





\chapter{Cake}



The mixing and baking of cake requires more care and judgment than any
other branch of cookery; notwithstanding, it seems the one most
frequently attempted by the inexperienced.

Two kinds of cake mixtures are considered:

I. Without butter. Example: Sponge Cakes.

II. With butter. Examples: Cup and Pound Cakes.

In cake making (1) the best ingredients are essential; (2) great care
must be taken in measuring and combining ingredients; (3) pans must be
properly prepared; (4) oven heat must be regulated, and cake watched
during baking.

Best tub butter, fine granulated sugar, fresh eggs, and pastry flour are
essentials for good cake. Coarse granulated sugar, bought by so many, if
used in cake making, gives a coarse texture and hard crust. Pastry flour
contains more starch and less gluten than bread flour, therefore makes a
lighter, more tender cake. If bread flour must be used, allow two
tablespoons less for each cup than the recipe calls for. Flour differs
greatly in thickening properties; for this reason it is always well when
using from a new bag to try a small cake, as the amount of flour given
may not make the perfect loaf. In winter, cake may be made of less flour
than in summer.

Before attempting to mix cake, study How to Measure (p. 25) and How to
Combine Ingredients (p. 26).

Look at the fire, and replenish by sprinkling on a small quantity of
coal if there is not sufficient heat to effect the baking.

\textbf{To Mix Sponge Cake.} Separate yolks from whites of eggs. Beat yolks
until thick and lemon-colored, using an egg-beater; add sugar gradually,
and continue beating; then add flavoring. Beat whites until stiff and
dry,--when they will fly from the beater,--and add to the first mixture.
Mix and sift flour with salt, and cut and fold in at the last. If
mixture is beaten after the addition of flour, much of the work already
done of enclosing a large amount of air will be undone by breaking air
bubbles. These rules apply to a mixture where baking powder is not
employed.

\textbf{To Mix Butter Cakes.} An earthen bowl should always be used for mixing
cake, and a wooden cake-spoon with slits lightens the labor. Measure dry
ingredients, and mix and sift baking powder and spices, if used, with
flour. Count out number of eggs required, breaking each separately that
there may be no loss should a stale egg chance to be found in the
number, separating yolks from whites if rule so specifies. Measure
butter, then liquid. Having everything in readiness, the mixing may be
quickly accomplished. If butter is very hard, by allowing it to stand a
short time in a warm room it is measured and creamed much easier. If
time cannot be allowed for this to be done, warm bowl by pouring in some
hot water, letting stand one minute, then emptying and wiping dry. Avoid
overheating bowl, as butter will become oily rather than creamy. Put
butter in bowl, and cream by working with a wooden spoon until soft and
of a creamy consistency; then add sugar gradually, and continue beating.
Add yolks of eggs or whole eggs beaten until light, liquid, and flour
mixed and sifted with baking powder; or liquid and flour may be added
alternately. When yolks and whites of eggs are beaten separately, whites
are usually added at the last, as is the case when whites of eggs alone
are used. A cake can be made fine-grained only by long beating, although
light and delicate with a small amount of beating. Never stir cake after
the final beating, remembering that beating motion should always be the
last used. Fruit, when added to cake, is usually floured to prevent its
settling to the bottom. This is not necessary if it is added directly
after the sugar, which is desirable in all dark cakes. If a light fruit
cake is made, fruit added in this way discolors the loaf. Citron is
first cut in thin slices, then in strips, floured, and put in between
layers of cake mixtures. Raisins are seeded and cut, rather than
chopped. \textit{To seed raisins}, wet tips of fingers in a cup of warm water.
Then break skins with fingers or cut with a vegetable knife; remove
seeds, and put in cup of water. This is better than covering raisins
with warm water; if this be done, water clings to fruit, and when
dredged with flour a pasty mass is formed on the outside. Washed
currants, put up in packages, are quite free from stems and foreign
substances, and need only picking over and rolling in flour. Currants
bought in bulk need thorough cleaning. First roll in flour, which helps
to start dirt; wash in cold water, drain, and spread to dry; then roll
again in flour before using.

\textbf{To Butter and Fill Pans.} Grease pans with melted fat, applying the
same with a butter brush. If butter is used, put in a small saucepan and
place on back of range; when melted, salt will settle to the bottom;
butter is then called \textit{clarified}. Just before putting in mixture,
dredge pans thoroughly with flour, invert, and shake pan to remove all
superfluous flour, leaving only a thin coating which adheres to butter.
This gives to cake a smooth under surface, which is especially desirable
if cake is to be frosted. Pans may be lined with paper. If this is done,
paper should just cover bottom of pan and project over sides. Then ends
of pan and paper are buttered.

In filling pans, have the mixture come well to the corners and sides of
pans, leaving a slight depression in the centre, and when baked the cake
will be perfectly flat on top. Cake pans should be filled nearly
two-thirds full if cake is expected to rise to top of pan.

\textbf{To Bake Cake.} The baking of cake is more critical than the mixing.
Many a well-mixed cake has been spoiled in the baking. No oven
thermometer has yet proved practical, and although many teachers of
cookery have given oven tests, experience alone has proved the most
reliable teacher. In baking cake, divide the time required into
quarters. During the first quarter the mixture should begin to rise;
second quarter, continue rising and begin to brown; third quarter,
continue browning; fourth quarter, finish baking and shrink from pan. If
oven is too hot, open check and raise back covers, or leave oven door
ajar. It is sometimes necessary to cover cake with brown paper; there
is, however, danger of cake adhering to paper. Cake should be often
looked at during baking, and providing oven door is opened and closed
carefully, there is no danger of this causing cake to fall. Cake should
not be moved in oven until it has risen its full height; after this it
is usually desirable to move it that it may be evenly browned. Cake when
done shrinks from the pan, and in most cases this is a sufficient test;
however, in pound cakes this rule does not apply. Pound and rich fruit
cakes are tested by pressing surface with tip of finger. If cake feels
firm to touch and follows finger back into place, it is safe to remove
it from the oven. When baking cake arrange to have nothing else in the
oven, and place loaf or loaves as near the centre of oven as possible.
If placed close to fire-box, one side of loaf is apt to become burned
before sufficiently risen to turn. If cake is put in too slow an oven,
it often rises over sides of pan and is of very coarse texture; if put
in too hot an oven, it browns on top before sufficiently risen, and in
its attempt to rise breaks through the crust, thus making an unsightly
loaf. Cake will also crack on top if too much flour has been used. The
oven should be kept at as nearly uniform temperature as possible. Small
and layer cakes require a hotter oven than loaf cakes.

\textbf{To Remove Cake From Pans.} Remove cake from pans as soon as it comes
from the oven, by inverting pan on a wire cake cooler, or on a board
covered with a piece of old linen. If cake is inclined to stick, do not
hurry it from pan, but loosen with knife around edges, and rest pan on
its four sides successively; thus by its own weight cake may be helped
out.

\textbf{To Frost Cake.} Where cooked frostings are used, it makes but little
difference whether they are spread on hot or cold cake. Where uncooked
frostings are used, it is best to have the cake slightly warm, with the
exception of Confectioners' Frosting, where boiling water is employed.



\needspace{15\baselineskip}
\section*{Hot Water Sponge Cake}


\begin{minipage}{1.0\textwidth}
{\setlength{\multicolsep}{0pt}\setlength{\columnsep}{2em}\raggedcolumns%
\begin{multicols}{2}
\begin{itemize}
\setlength{\itemsep}{0pt}
\setlength{\parsep}{0pt}
\item 4 egg yolks
\item 1 cup sugar
\item 3/8 cup hot water or milk
\item 1/4 teaspoon lemon extract
\item Whites two eggs
\item 1 cup flour
\item 1 1/2 teaspoons baking powder
\item 1/4 teaspoon salt
\end{itemize}
\end{multicols}}
\end{minipage}

\vspace{0.3em}
\noindent%
Beat yolks of eggs until thick and lemon-colored, add one-half the sugar
gradually, and continue beating; then add water, remaining sugar, lemon
extract, whites of eggs beaten until stiff, and flour mixed and sifted
with baking powder and salt. Bake twenty-five minutes in a moderate oven
in a buttered and floured shallow pan.



\needspace{15\baselineskip}
\section*{Cheap Sponge Cake}


\begin{minipage}{1.0\textwidth}
{\setlength{\multicolsep}{0pt}\setlength{\columnsep}{2em}\raggedcolumns%
\begin{multicols}{2}
\begin{itemize}
\setlength{\itemsep}{0pt}
\setlength{\parsep}{0pt}
\item 3 egg yolks
\item 1 cup sugar
\item 1 tablespoon hot water
\item 1 cup flour
\item 1 1/2 teaspoons baking powder
\item 1/4 teaspoon salt
\item 3 egg whites
\item 2 teaspoons vinegar
\end{itemize}
\end{multicols}}
\end{minipage}

\vspace{0.3em}
\noindent%
Beat yolks of eggs until thick and lemon-colored, add sugar gradually,
and continue beating; then add water, flour mixed and sifted with baking
powder and salt, whites of eggs beaten until stiff, and vinegar. Bake
thirty-five minutes in a moderate oven, in a buttered and floured cake
pan.



\needspace{15\baselineskip}
\section*{Cream Sponge Cake}


\begin{minipage}{1.0\textwidth}
{\setlength{\multicolsep}{0pt}\setlength{\columnsep}{2em}\raggedcolumns%
\begin{multicols}{2}
\begin{itemize}
\setlength{\itemsep}{0pt}
\setlength{\parsep}{0pt}
\item 4 egg yolks
\item 1 cup sugar
\item 3 tablespoons cold water
\item 1 1/2 tablespoons corn-starch
\item Flour
\item 1 1/4 teaspoons baking powder
\item 1/4 teaspoon salt
\item 4 egg whitess
\item 1 teaspoon lemon extract
\end{itemize}
\end{multicols}}
\end{minipage}

\vspace{0.3em}
\noindent%
Beat yolks of eggs until thick and lemon-colored, add sugar gradually,
and beat two minutes; then add water. Put corn-starch in a cup and fill
cup with flour. Mix and sift corn-starch and flour with baking powder
and salt, and add to first mixture. When thoroughly mixed add whites of
eggs beaten until stiff, and flavoring. Bake thirty minutes in a
moderate oven. This is an excellent mixture to use for whipped cream
pies.



\needspace{15\baselineskip}
\section*{Petit Four}

Follow recipe for Cream Sponge Cake. Bake in a shallow pan, cool, and
shape, using a small round cutter. Split, and remove a small portion of
cake from the centre of each piece. Fill cavities of one-half the pieces
with whipped cream sweetened and flavored, cover with remaining pieces,
and press firmly together. Nuts or glacé fruits cut in pieces may be
added to cream. Melt fondant, color, and flavor to taste. Dip cakes in
fondant, decorate tops with pistachio nuts, violets, or glacé cherries,
and place each in a paper case.



\needspace{15\baselineskip}
\section*{Sponge Cake}


\begin{minipage}{1.0\textwidth}
{\setlength{\multicolsep}{0pt}\setlength{\columnsep}{2em}\raggedcolumns%
\begin{multicols}{2}
\begin{itemize}
\setlength{\itemsep}{0pt}
\setlength{\parsep}{0pt}
\item 6 egg yolks
\item 1 cup sugar
\item 1 tablespoon lemon juice
\item Grated rind one-half lemon
\item 6 egg whitess
\item 1 cup flour
\item 1/4 teaspoon salt
\end{itemize}
\end{multicols}}
\end{minipage}

\vspace{0.3em}
\noindent%
Beat yolks until thick and lemon-colored, add sugar gradually, and
continue beating, using Dover egg-beater. Add lemon juice, rind, and
whites of eggs beaten until stiff and dry. When whites are partially
mixed with yolks, remove beater, and carefully cut and fold in flour
mixed and sifted with salt. Bake one hour in a slow oven, in an angel
cake pan or deep narrow pan.

Genuine sponge cake contains no rising properties, but is made light by
the quantity of air beaten into both yolks and whites of eggs, and the
expansion of that air in baking. It requires a slow oven. All so-called
sponge cakes which have the addition of soda and cream of tartar or
baking powder require same oven temperature as butter cakes. When
failures are made in Sunshine and Angel Cake, they are usually traced to
baking in too slow an oven, and removing from oven before thoroughly
cooked.



\needspace{15\baselineskip}
\section*{Sunshine Cake}

                       Whites 10 eggs

\begin{itemize}
\setlength{\itemsep}{0pt}
\setlength{\parsep}{0pt}
\item 1 1/2 cups powdered sugar
\item 6 egg yolks
\item 1 teaspoon lemon extract
\item 1 cup flour
\item 1 teaspoon cream of tartar
\end{itemize}

\vspace{-0.5em}
\noindent%
Beat whites of eggs until stiff and dry, add sugar gradually, and
continue beating; then add yolks of eggs beaten until thick and
lemon-colored, and extract. Cut and fold in flour mixed and sifted with
cream of tartar. Bake fifty minutes in a moderate oven in an angel-cake
pan.



\needspace{15\baselineskip}
\section*{Mocha Cake}

To one-half recipe for Sunshine Cake add one-half cup English walnut
meats broken in pieces. Bake in a medium-sized angel-cake pan; cool,
split, and fill with whipped cream sweetened and flavored with coffee
essence. Cover top with Confectioners' Frosting, flavored with coffee
essence.



\needspace{15\baselineskip}
\section*{Angel Cake}

                       Whites 8 eggs

\begin{itemize}
\setlength{\itemsep}{0pt}
\setlength{\parsep}{0pt}
\item 1 teaspoon cream of tartar
\item 1 cup sugar
\item 3/4 cup flour
\item 1/4 teaspoon salt
\item 3/4 teaspoon vanilla
\end{itemize}

\vspace{-0.5em}
\noindent%
Beat whites of eggs until frothy; add cream of tartar, and continue
beating until eggs are stiff; then add sugar gradually. Fold in flour
mixed with salt and sifted four times, and add vanilla. Bake forty-five
to fifty minutes in an unbuttered angel-cake pan. After cake has risen
and begins to brown, cover with a buttered paper.



\needspace{15\baselineskip}
\section*{Moonshine Cake}

                       Whites 10 eggs

\begin{minipage}{1.0\textwidth}
{\setlength{\multicolsep}{0pt}\setlength{\columnsep}{2em}\raggedcolumns%
\begin{multicols}{2}
\begin{itemize}
\setlength{\itemsep}{0pt}
\setlength{\parsep}{0pt}
\item 1/4 teaspoon salt
\item 7/8 teaspoon cream of tartar
\item Yolks 7 eggs
\item 1 1/2 cups sugar
\item 1 teaspoon almond extract
\item 1 cup pastry flour
\end{itemize}
\end{multicols}}
\end{minipage}

\vspace{0.3em}
\noindent%
Add salt to whites of eggs and beat until light. Sift in cream of tartar
and beat until stiff. Beat yolks of eggs until thick and lemon-colored
and add two heaping tablespoons beaten whites. To remaining whites add
gradually sugar measured after five siftings. Add almond extract and
combine mixtures. Cut and fold in flour, measured after five siftings.
Bake in angel-cake pan, first dipped in cold water, in a slow oven one
hour. Have a pan of hot water in oven during the baking. Cover with

\textbf{Maraschino Frosting.} Follow recipe for Ice Cream Frosting (see p.
528), adding to sugar one-half teaspoon cream of tartar, and flavor with
maraschino. Sprinkle with almonds blanched, shredded, and baked until
delicately browned.



\needspace{15\baselineskip}
\section*{Lady Fingers}


\begin{minipage}{1.0\textwidth}
{\setlength{\multicolsep}{0pt}\setlength{\columnsep}{2em}\raggedcolumns%
\begin{multicols}{2}
\begin{itemize}
\setlength{\itemsep}{0pt}
\setlength{\parsep}{0pt}
\item 3 egg whites
\item 1/3 cup powdered sugar
\item 4 egg yolks
\item 1/3 cup flour
\item 1/8 teaspoon salt
\item 1/4 teaspoon vanilla
\end{itemize}
\end{multicols}}
\end{minipage}

\vspace{0.3em}
\noindent%
Beat whites of eggs until stiff and dry, add sugar gradually, and
continue beating. Then add yolks of eggs beaten until thick and
lemon-colored, and flavoring. Cut and fold in flour mixed and sifted
with salt. Shape four and one-half inches long and one inch wide on a
tin sheet covered with unbuttered paper, using a pastry bag and tube.
Sprinkle with powdered sugar, and bake eight minutes in a moderate oven.
Remove from paper with a knife. Lady Fingers are much used for lining
moulds that are to be filled with whipped cream mixtures. They are often
served with frozen desserts, and sometimes put together in pairs with a
thin coating of whipped cream between, when they are attractive for
children's parties.



\needspace{15\baselineskip}
\section*{Sponge Drops}

Drop Lady Finger mixture from tip of spoon on unbuttered paper. Sprinkle
with powdered sugar, and bake eight minutes in a moderate oven.



\needspace{15\baselineskip}
\section*{Almond Tart}


\begin{minipage}{1.0\textwidth}
{\setlength{\multicolsep}{0pt}\setlength{\columnsep}{2em}\raggedcolumns%
\begin{multicols}{2}
\begin{itemize}
\setlength{\itemsep}{0pt}
\setlength{\parsep}{0pt}
\item 4 eggs
\item 1 cup powdered sugar
\item 1/3 cup grated chocolate
\item 1/2 cup Jordan almonds, blanched and finely chopped
\item 1 teaspoon baking powder
\item 3/4 cup cracker dust
\end{itemize}
\end{multicols}}
\end{minipage}

\vspace{0.3em}
\noindent%
Beat yolks of eggs until thick and lemon-colored; add sugar gradually,
then fold in white of eggs beaten until stiff and dry. Add chocolate,
almonds, baking powder, and cracker dust. Bake in a round pan. Cool,
split, and put whipped cream, sweetened and flavored, between and on
top. Garnish with angelica and candied cherries. This makes a most
attractive dessert when baked in individual tins. As soon as cool,
remove centres, and fill with whipped cream, forced through a pastry
bag.



\needspace{15\baselineskip}
\section*{Jelly Roll}


\begin{minipage}{1.0\textwidth}
{\setlength{\multicolsep}{0pt}\setlength{\columnsep}{2em}\raggedcolumns%
\begin{multicols}{2}
\begin{itemize}
\setlength{\itemsep}{0pt}
\setlength{\parsep}{0pt}
\item 3 eggs
\item 1 cup sugar
\item 1/2 tablespoon milk
\item 1 teaspoon baking powder
\item 1/4 teaspoon salt
\item 1 cup flour
\item 1 tablespoon melted butter
\end{itemize}
\end{multicols}}
\end{minipage}

\vspace{0.3em}
\noindent%
Beat egg until light, add sugar gradually, milk, flour mixed and sifted
with baking powder and salt, then butter. Line the bottom of a
dripping-pan with paper; butter paper and sides of pan. Cover bottom of
pan with mixture, and spread evenly. Bake twelve minutes in a moderate
oven. Take from oven and turn on a paper sprinkled with powdered sugar.
Quickly remove paper, and cut off a thin strip from sides and ends of
cake. Spread with jelly or jam which has been beaten to consistency to
spread easily, and roll. After cake has been rolled, roll paper around
cake that it may better keep in shape. The work must be done quickly, or
cake will crack in rolling.



\needspace{15\baselineskip}
\section*{Election Cake}


\begin{minipage}{1.0\textwidth}
{\setlength{\multicolsep}{0pt}\setlength{\columnsep}{2em}\raggedcolumns%
\begin{multicols}{2}
\begin{itemize}
\setlength{\itemsep}{0pt}
\setlength{\parsep}{0pt}
\item 1/2 cup butter
\item 1 cup bread dough
\item 1 egg
\item 1 cup brown sugar
\item 1/2 cup sour milk
\item 2/3 cup raisins seeded, and cut in pieces
\item 8 finely chopped figs
\item 1 1/4 cups flour
\item 1/2 teaspoon soda
\item 1 teaspoon cinnamon
\item 1/4 teaspoon clove
\item 1/4 teaspoon mace
\item 1/4 teaspoon nutmeg
\item 1 teaspoon salt
\end{itemize}
\end{multicols}}
\end{minipage}

\vspace{0.3em}
\noindent%
Work butter into dough, using the hand. Add egg well beaten, sugar,
milk, fruit dredged with two tablespoons flour, and flour mixed and
sifted with remaining ingredients. Put into a well-buttered bread pan,
cover, and let rise one and one-fourth hours. Bake one hour in a slow
oven. Cover with Boiled Milk Frosting.



\needspace{15\baselineskip}
\section*{One Egg Cake}


\begin{minipage}{1.0\textwidth}
{\setlength{\multicolsep}{0pt}\setlength{\columnsep}{2em}\raggedcolumns%
\begin{multicols}{2}
\begin{itemize}
\setlength{\itemsep}{0pt}
\setlength{\parsep}{0pt}
\item 1/4 cup of butter
\item 1/2 cup sugar
\item 1 egg
\item 1/2 cup milk
\item 1 1/2 cups flour
\item 2 1/2 teaspoons baking powder
\end{itemize}
\end{multicols}}
\end{minipage}

\vspace{0.3em}
\noindent%
Cream the butter, add sugar gradually, and egg well beaten. Mix and sift
flour and baking powder, add alternately with milk to first mixture.
Bake thirty minutes in a shallow pan. Spread with Chocolate Frosting.



\needspace{15\baselineskip}
\section*{Chocolate Cake I}


\begin{minipage}{1.0\textwidth}
{\setlength{\multicolsep}{0pt}\setlength{\columnsep}{2em}\raggedcolumns%
\begin{multicols}{2}
\begin{itemize}
\setlength{\itemsep}{0pt}
\setlength{\parsep}{0pt}
\item 1/2 cup butter
\item 1 cup sugar
\item 2 small eggs
\item 1/2 cup milk
\item 1 1/2 cups flour
\item 2 1/2 teaspoons baking powder
\item 2 ozs. chocolate, melted
\item 1/2 teaspoon vanilla
\end{itemize}
\end{multicols}}
\end{minipage}

\vspace{0.3em}
\noindent%
Cream the butter, add sugar gradually, and yolks of eggs well beaten,
then whites of eggs beaten until stiff. Add milk, flour mixed and sifted
with baking powder, and beat thoroughly. Then add chocolate and vanilla.
Bake forty minutes in a shallow cake pan.



\needspace{15\baselineskip}
\section*{Chocolate Cake II}


\begin{minipage}{1.0\textwidth}
{\setlength{\multicolsep}{0pt}\setlength{\columnsep}{2em}\raggedcolumns%
\begin{multicols}{2}
\begin{itemize}
\setlength{\itemsep}{0pt}
\setlength{\parsep}{0pt}
\item 1/2 cup butter
\item 1 1/2 cups sugar
\item 1/2 cup milk
\item 2 1/4 cups flour
\item 1/4 teaspoon soda
\item 3/4 teaspoon cream of tartar
\item 5 egg whitess
\item 2 squares Baker's chocolate, grated
\end{itemize}
\end{multicols}}
\end{minipage}

\vspace{0.3em}
\noindent%
Cream the butter; add sugar gradually, milk, and flour mixed and sifted
with soda and cream of tartar. Beat whites of eggs, and add to first
mixture; then add chocolate, and beat thoroughly. Bake forty-five
minutes in a moderate oven.



\needspace{15\baselineskip}
\section*{Chocolate Marshmallow Cake}

Follow recipe for Chocolate Cake II. As soon as cake is removed from
pan, cover bottom with marshmallows pulled apart with tips of fingers,
but not quite separated into halves. The exposed soft surface will
quickly adhere to hot cake. Pour over Chocolate Fudge Frosting.



\needspace{15\baselineskip}
\section*{Chocolate Nougat Cake}


\begin{minipage}{1.0\textwidth}
{\setlength{\multicolsep}{0pt}\setlength{\columnsep}{2em}\raggedcolumns%
\begin{multicols}{2}
\begin{itemize}
\setlength{\itemsep}{0pt}
\setlength{\parsep}{0pt}
\item 1/4 cup butter
\item 1 1/2 cups powdered sugar
\item 1 egg
\item 1 cup milk
\item 2 cups bread flour
\item 3 teaspoons baking powder
\item 1/2 teaspoon vanilla
\item 2 squares chocolate, melted
\item 1/3 cup powdered sugar
\item 2/3 cup almonds, blanched and shredded
\end{itemize}
\end{multicols}}
\end{minipage}

\vspace{0.3em}
\noindent%
Cream the butter, add gradually one and one-half cups sugar, and egg
unbeaten; when well mixed, add two-thirds milk, flour mixed and sifted
with baking powder, and vanilla. To melted chocolate add one-third cup
powdered sugar, place on range, add gradually remaining milk, and cook
until smooth. Cool slightly, and add to cake mixture. Bake fifteen to
twenty minutes in round layer cake pans. Put between layers and on top
of cake White Mountain Cream sprinkled with almonds.



\needspace{15\baselineskip}
\section*{Chocolate Dominoes}


\begin{minipage}{1.0\textwidth}
{\setlength{\multicolsep}{0pt}\setlength{\columnsep}{2em}\raggedcolumns%
\begin{multicols}{2}
\begin{itemize}
\setlength{\itemsep}{0pt}
\setlength{\parsep}{0pt}
\item 1/2 cup pecan nut meat
\item 1/2 cup English walnut meat
\item 1/2 cup figs
\item 1/2 cup dates
\item Grated rind 1 orange
\item 1 tablespoon orange juice
\item 1 square chocolate, melted
\end{itemize}
\end{multicols}}
\end{minipage}

\vspace{0.3em}
\noindent%
Mix nut meats, figs, and dates, and force through a meat chopper, or
chop finely. Add remaining ingredients, toss on a board sprinkled with
powdered sugar, and roll to one-third inch in thickness. Cut in domino
shapes, spread thinly with melted unsweetened chocolate, and decorate
with small pieces blanched almonds to imitate dominoes.



\needspace{15\baselineskip}
\section*{Cream Pie I}


\begin{minipage}{1.0\textwidth}
{\setlength{\multicolsep}{0pt}\setlength{\columnsep}{2em}\raggedcolumns%
\begin{multicols}{2}
\begin{itemize}
\setlength{\itemsep}{0pt}
\setlength{\parsep}{0pt}
\item 1/3 cup butter
\item 1 cup sugar
\item 2 eggs
\item 1/2 cup milk
\item 1 3/4 cups flour
\item 2 1/2 teaspoons baking powder
\end{itemize}
\end{multicols}}
\end{minipage}

\vspace{0.3em}
\noindent%
Mix as One Egg Cake. Bake in round layer cake pans. Put Cream Filling
between layers and sprinkle top with powdered sugar.



\needspace{15\baselineskip}
\section*{Cream Pie II}

Make as Cream Pie I, using French Cream Filling in place of Cream
Filling.



\needspace{15\baselineskip}
\section*{Cocoanut Pie}

Mix and bake same as Cream Pie. Put Cocoanut Filling between layers and
on top.



\needspace{15\baselineskip}
\section*{Washington Pie}

Mix and bake same as Cream Pie. Put raspberry jam or jelly between
layers and sprinkle top with powdered sugar.



\needspace{15\baselineskip}
\section*{Chocolate Pie}


\begin{minipage}{1.0\textwidth}
{\setlength{\multicolsep}{0pt}\setlength{\columnsep}{2em}\raggedcolumns%
\begin{multicols}{2}
\begin{itemize}
\setlength{\itemsep}{0pt}
\setlength{\parsep}{0pt}
\item 2 tablespoons butter
\item 3/4 cup sugar
\item 1 egg
\item 1/2 cup milk
\item 1 1/3 cups flour
\item 2 teaspoons baking powder
\end{itemize}
\end{multicols}}
\end{minipage}

\vspace{0.3em}
\noindent%
Mix and bake same as Cream Pie. Split layers, and spread between and on
top of each a thin layer of Chocolate Frosting.



\needspace{15\baselineskip}
\section*{Orange Cake}


\begin{minipage}{1.0\textwidth}
{\setlength{\multicolsep}{0pt}\setlength{\columnsep}{2em}\raggedcolumns%
\begin{multicols}{2}
\begin{itemize}
\setlength{\itemsep}{0pt}
\setlength{\parsep}{0pt}
\item 1/4 cup butter
\item 1 cup sugar
\item 2 eggs
\item 1/2 cup milk
\item 1 2/3 cups flour
\item 2 1/2 teaspoons baking powder
\end{itemize}
\end{multicols}}
\end{minipage}

\vspace{0.3em}
\noindent%
Cream the butter, add sugar gradually, eggs well beaten, and milk. Then
add flour mixed and sifted with baking powder. Bake in a thin sheet in a
dripping-pan. Cut in halves, spread one-half with Orange Filling. Put
over other half, and cover with Orange Frosting.



\needspace{15\baselineskip}
\section*{Quick Cake}


\begin{minipage}{1.0\textwidth}
{\setlength{\multicolsep}{0pt}\setlength{\columnsep}{2em}\raggedcolumns%
\begin{multicols}{2}
\begin{itemize}
\setlength{\itemsep}{0pt}
\setlength{\parsep}{0pt}
\item 1/3 cup soft butter
\item 1 1/3 cups brown sugar
\item 2 eggs
\item 1/2 cup milk
\item 1 3/4 cups flour
\item 3 teaspoons baking powder
\item 1/2 teaspoon cinnamon
\item 1/2 teaspoon grated nutmeg
\item 1/2 lb. dates, stoned and cut in pieces
\end{itemize}
\end{multicols}}
\end{minipage}

\vspace{0.3em}
\noindent%
Put ingredients in a bowl and beat all together for three minutes, using
a wooden cake spoon. Bake in a buttered and floured cake pan thirty-five
to forty minutes. If directions are followed this makes a most
satisfactory cake; but if ingredients are added separately it will not
prove a success.



\needspace{15\baselineskip}
\section*{Boston Favorite Cake}


\begin{minipage}{1.0\textwidth}
{\setlength{\multicolsep}{0pt}\setlength{\columnsep}{2em}\raggedcolumns%
\begin{multicols}{2}
\begin{itemize}
\setlength{\itemsep}{0pt}
\setlength{\parsep}{0pt}
\item 2/3 cup butter
\item 2 cups sugar
\item 4 eggs
\item 1 cup milk
\item 3 1/2 cups flour
\item 5 teaspoons baking powder
\end{itemize}
\end{multicols}}
\end{minipage}

\vspace{0.3em}
\noindent%
Cream the butter, add sugar gradually, eggs beaten until light, then
milk and flour mixed and sifted with baking powder. This recipe makes
two loaves, or one-half the mixture may be baked in individual tins.



\needspace{15\baselineskip}
\section*{Cream Cake}


\begin{minipage}{1.0\textwidth}
{\setlength{\multicolsep}{0pt}\setlength{\columnsep}{2em}\raggedcolumns%
\begin{multicols}{2}
\begin{itemize}
\setlength{\itemsep}{0pt}
\setlength{\parsep}{0pt}
\item 2 eggs
\item 1 cup sugar
\item 2/3 cup thin cream
\item 1 2/3 cups flour
\item 2 1/2 teaspoons baking powder
\item 1/2 teaspoon salt
\item 1/2 teaspoon cinnamon
\item 1/4 teaspoon mace
\item 1/4 teaspoon ginger
\end{itemize}
\end{multicols}}
\end{minipage}

\vspace{0.3em}
\noindent%
Put unbeaten eggs in a bowl, add sugar and cream, and beat vigorously.
Mix and sift remaining ingredients, then add to first mixture. Bake
thirty minutes in a shallow cake pan.



\needspace{15\baselineskip}
\section*{Currant Cake}


\begin{minipage}{1.0\textwidth}
{\setlength{\multicolsep}{0pt}\setlength{\columnsep}{2em}\raggedcolumns%
\begin{multicols}{2}
\begin{itemize}
\setlength{\itemsep}{0pt}
\setlength{\parsep}{0pt}
\item 1/2 cup butter
\item 1 cup sugar
\item 2 eggs
\item Yolk 1 egg
\item 1/2 cup milk
\item 2 cups flour
\item 3 teaspoons baking powder
\item 1 cup currants mixed with
\item 1 tablespoon flour
\end{itemize}
\end{multicols}}
\end{minipage}

\vspace{0.3em}
\noindent%
Cream the butter, add sugar gradually, and eggs and egg yolk well
beaten. Then add milk, flour mixed and sifted with baking powder, and
currants. Bake forty minutes in buttered and floured cake pan.



\needspace{15\baselineskip}
\section*{Citron Cake}


\begin{minipage}{1.0\textwidth}
{\setlength{\multicolsep}{0pt}\setlength{\columnsep}{2em}\raggedcolumns%
\begin{multicols}{2}
\begin{itemize}
\setlength{\itemsep}{0pt}
\setlength{\parsep}{0pt}
\item 1/4 lb. butter
\item 1/2 lb. sugar
\item 3 eggs
\item 1/2 cup milk
\item 1/2 lb. flour
\item 1 tablespoon brandy
\item 1 cup citron, thinly sliced, then cut in strips
\item 1 1/2 teaspoons baking powder
\end{itemize}
\end{multicols}}
\end{minipage}

\vspace{0.3em}
\noindent%
Cream the butter, add sugar gradually, yolks of eggs well beaten, milk,
and flour mixed and sifted with baking powder. Beat whites of eggs until
stiff, and add to first mixture, then add brandy and citron. Bake in a
moderate oven one hour.



\needspace{15\baselineskip}
\section*{Velvet Cake}


\begin{minipage}{1.0\textwidth}
{\setlength{\multicolsep}{0pt}\setlength{\columnsep}{2em}\raggedcolumns%
\begin{multicols}{2}
\begin{itemize}
\setlength{\itemsep}{0pt}
\setlength{\parsep}{0pt}
\item 1/2 cup butter
\item 1 1/2 cups sugar
\item 4 egg yolks
\item 1/2 cup cold water
\item 1 1/2 cups flour
\item 1/2 cup corn-starch
\item 4 teaspoons baking powder
\item 4 egg whitess
\item 1/3 cup almonds, blanched, and shredded
\end{itemize}
\end{multicols}}
\end{minipage}

\vspace{0.3em}
\noindent%
Cream the butter, add sugar gradually, yolks of eggs well beaten, and
water. Mix and sift flour, corn-starch, and baking powder, and add to
first mixture; then add whites of eggs beaten until stiff. After putting
in pan, cover with almonds and sprinkle with powdered sugar. Bake forty
minutes in a moderate oven.



\needspace{15\baselineskip}
\section*{Walnut Cake}


\begin{minipage}{1.0\textwidth}
{\setlength{\multicolsep}{0pt}\setlength{\columnsep}{2em}\raggedcolumns%
\begin{multicols}{2}
\begin{itemize}
\setlength{\itemsep}{0pt}
\setlength{\parsep}{0pt}
\item 1/2 cup butter
\item 1 cup sugar
\item 3 egg yolks
\item 1/2 cup milk
\item 1 3/4 cups flour
\item 2 1/2 teaspoons baking powder
\item 2 egg whites
\item 3/4 cup walnut meat, broken in pieces
\end{itemize}
\end{multicols}}
\end{minipage}

\vspace{0.3em}
\noindent%
Mix ingredients in order given. Bake forty-five minutes in a moderate
oven. Cover with White Mountain Cream, crease in squares, and put
one-half walnut on each square.



\needspace{15\baselineskip}
\section*{Spanish Cake}


\begin{minipage}{1.0\textwidth}
{\setlength{\multicolsep}{0pt}\setlength{\columnsep}{2em}\raggedcolumns%
\begin{multicols}{2}
\begin{itemize}
\setlength{\itemsep}{0pt}
\setlength{\parsep}{0pt}
\item 1/2 cup butter
\item 1 cup sugar
\item 4 egg yolks
\item 1/2 cup milk
\item 1 3/4 cups flour
\item 3 teaspoons baking powder
\item 1 teaspoon cinnamon
\item 2 egg whites
\end{itemize}
\end{multicols}}
\end{minipage}

\vspace{0.3em}
\noindent%
Mix ingredients in order given. Bake in shallow tins and spread between
and on top Caramel Frosting.



\needspace{15\baselineskip}
\section*{Cup Cakes}


\begin{minipage}{1.0\textwidth}
{\setlength{\multicolsep}{0pt}\setlength{\columnsep}{2em}\raggedcolumns%
\begin{multicols}{2}
\begin{itemize}
\setlength{\itemsep}{0pt}
\setlength{\parsep}{0pt}
\item 2/3 cup butter
\item 2 cups sugar
\item 4 eggs
\item 1 cup milk
\item 3 1/4 cups flour
\item 4 teaspoons baking powder
\item 1/4 teaspoon mace
\end{itemize}
\end{multicols}}
\end{minipage}

\vspace{0.3em}
\noindent%
Put butter and sugar in a bowl, and stir until well mixed; add eggs well
beaten, then milk, and flour mixed and sifted with baking powder and
mace. Bake in individual tins. Cover with Chocolate Frosting.



\needspace{15\baselineskip}
\section*{Cinnamon Cakes}


\begin{minipage}{1.0\textwidth}
{\setlength{\multicolsep}{0pt}\setlength{\columnsep}{2em}\raggedcolumns%
\begin{multicols}{2}
\begin{itemize}
\setlength{\itemsep}{0pt}
\setlength{\parsep}{0pt}
\item 1/2 cup butter
\item 1 cup sugar
\item 2 eggs
\item 1/2 cup milk
\item 1 1/4 cups flour
\item 2 1/2 teaspoons baking powder
\item 1 tablespoon cinnamon
\end{itemize}
\end{multicols}}
\end{minipage}

\vspace{0.3em}
\noindent%
Mix ingredients in the order given, and bake in individual buttered cake
tins.



\needspace{15\baselineskip}
\section*{Almond Cakes}


\begin{minipage}{1.0\textwidth}
{\setlength{\multicolsep}{0pt}\setlength{\columnsep}{2em}\raggedcolumns%
\begin{multicols}{2}
\begin{itemize}
\setlength{\itemsep}{0pt}
\setlength{\parsep}{0pt}
\item 1/2 cup butter
\item 3/4 cup sugar
\item 1/3 cup milk
\item 2 eggs
\item 1 1/3 cups flour
\item 2 teaspoons baking powder
\item 1 cup Jordan almonds, blanched and cut in pieces
\end{itemize}
\end{multicols}}
\end{minipage}

\vspace{0.3em}
\noindent%
Mix ingredients in order given, and bake in individual cake pans.



\needspace{15\baselineskip}
\section*{Brownies}


\begin{minipage}{1.0\textwidth}
{\setlength{\multicolsep}{0pt}\setlength{\columnsep}{2em}\raggedcolumns%
\begin{multicols}{2}
\begin{itemize}
\setlength{\itemsep}{0pt}
\setlength{\parsep}{0pt}
\item 1/3 cup butter
\item 1/3 cup powdered sugar
\item 1/3 cup Porto Rico molasses
\item 1 egg, well beaten
\item 7/8 cup bread flour
\item 1 cup pecan meat, cut in pieces
\end{itemize}
\end{multicols}}
\end{minipage}

\vspace{0.3em}
\noindent%
Mix ingredients in order given. Bake in small shallow fancy cake tins,
garnishing top of each cake with one-half pecan.



\needspace{15\baselineskip}
\section*{Chocolate Sponge}


\begin{minipage}{1.0\textwidth}
{\setlength{\multicolsep}{0pt}\setlength{\columnsep}{2em}\raggedcolumns%
\begin{multicols}{2}
\begin{itemize}
\setlength{\itemsep}{0pt}
\setlength{\parsep}{0pt}
\item 1/2 cup butter
\item 1/4 cup prepared powdered cocoa
\item 3 eggs
\item 1 cup sugar
\item 1 teaspoon cinnamon
\item 1/4 teaspoon clove
\item 1/2 cup cold water
\item 1 cup flour
\item 3 teaspoons baking powder
\end{itemize}
\end{multicols}}
\end{minipage}

\vspace{0.3em}
\noindent%
Cream the butter; add cocoa, yolks of eggs well beaten, sugar mixed with
cinnamon and clove, and water. Beat the whites of eggs, and add to first
mixture alternately with flour mixed and sifted with baking powder. Bake
in small tins from fifteen to twenty minutes.



\needspace{15\baselineskip}
\section*{Devil's Food Cake I}


\begin{minipage}{1.0\textwidth}
{\setlength{\multicolsep}{0pt}\setlength{\columnsep}{2em}\raggedcolumns%
\begin{multicols}{2}
\begin{itemize}
\setlength{\itemsep}{0pt}
\setlength{\parsep}{0pt}
\item 1/2 cup butter
\item 2 cups sugar
\item 4 egg yolks
\item 1 cup milk
\item 2 1/3 cups flour
\item 4 teaspoons baking powder
\item 4 egg whitess
\item 2 squares chocolate
\item 1/2 teaspoon vanilla
\end{itemize}
\end{multicols}}
\end{minipage}

\vspace{0.3em}
\noindent%
Cream the butter, and add gradually one-half the sugar. Beat yolks of
eggs until thick and lemon-colored, and add gradually remaining sugar.
Combine mixtures, and add alternately milk and flour mixed and sifted
with baking powder; then add whites of eggs beaten stiff, chocolate
melted, and vanilla. Bake forty-five to fifty minutes in an angel cake
pan. Cover with White Mountain Cream (see p. 528).



\needspace{15\baselineskip}
\section*{Devil's Food Cake II}


\begin{minipage}{1.0\textwidth}
{\setlength{\multicolsep}{0pt}\setlength{\columnsep}{2em}\raggedcolumns%
\begin{multicols}{2}
\begin{itemize}
\setlength{\itemsep}{0pt}
\setlength{\parsep}{0pt}
\item 4 squares Baker's chocolate
\item 1/2 cup sugar
\item 1/2 cup sweet milk
\item Yolk 1 egg
\item 1/4 cup butter
\item 1/2 cup sugar
\item 1/4 cup sour milk
\item 1 egg
\item 1 1/8 cups flour
\item 1/2 teaspoon soda
\item 1/2 teaspoon vanilla
\end{itemize}
\end{multicols}}
\end{minipage}

\vspace{0.3em}
\noindent%
Melt chocolate over hot water, add one-half cup sugar, and gradually
sweet milk; then add yolk of egg, and cook until mixture thickens. Set
aside to cool. Cream the butter, add gradually one-half cup sugar, egg
well beaten, sour milk, and flour mixed and sifted with soda. Combine
mixtures and add vanilla. Bake in shallow cake pans, and put between and
on top boiled frosting. Add to filling one-fourth cup raisins seeded and
cut in pieces, if desired.



\needspace{15\baselineskip}
\section*{Chocolate Vienna Cake}


\begin{minipage}{1.0\textwidth}
{\setlength{\multicolsep}{0pt}\setlength{\columnsep}{2em}\raggedcolumns%
\begin{multicols}{2}
\begin{itemize}
\setlength{\itemsep}{0pt}
\setlength{\parsep}{0pt}
\item 3/4 cup butter
\item 7/8 cup sugar
\item 5 egg yolks
\item 4 squares Baker's chocolate
\item 1 1/2 cups flour
\item 3 teaspoons baking powder
\item 5 egg whitess
\item Apricot or Orange Marmalade
\end{itemize}
\end{multicols}}
\end{minipage}

\vspace{0.3em}
\noindent%
Mix ingredients in order given, and bake in small tins. Remove from
tins, cool, take out a small portion of cake from the centre of each,
and fill cavity with marmalade. Cover tops of cake with Marshmallow
Frosting or Chocolate Frosting IV.



\needspace{15\baselineskip}
\section*{Chocolate Fruit Cake}


\begin{minipage}{1.0\textwidth}
{\setlength{\multicolsep}{0pt}\setlength{\columnsep}{2em}\raggedcolumns%
\begin{multicols}{2}
\begin{itemize}
\setlength{\itemsep}{0pt}
\setlength{\parsep}{0pt}
\item 1/3 cup butter
\item 1 cup sugar
\item 1/4 cup Breakfast Cocoa
\item 3 egg yolks
\item 1/2 cup cold water
\item 1 1/4 cups bread flour
\item 3 teaspoons baking powder
\item 1 teaspoon cinnamon
\item 1/4 teaspoon salt
\item 1/3 cup candied cherries
\item 1/3 cup raisins, seeded and cut in pieces
\item 1 1/2 tablespoons brandy
\item 1/3 cup walnut meats, cut in pieces
\item 3 egg whites
\item 1 teaspoon vanilla
\end{itemize}
\end{multicols}}
\end{minipage}

\vspace{0.3em}
\noindent%
Cover fruit with brandy and let stand several hours. Mix ingredients in
order given, and bake in deep cake pan fifty minutes. Cover with White
Mountain Cream, and as soon as frosting is set, spread as thinly as
possible with melted chocolate.



\needspace{15\baselineskip}
\section*{Ribbon Cake}


\begin{minipage}{1.0\textwidth}
{\setlength{\multicolsep}{0pt}\setlength{\columnsep}{2em}\raggedcolumns%
\begin{multicols}{2}
\begin{itemize}
\setlength{\itemsep}{0pt}
\setlength{\parsep}{0pt}
\item 1/2 cup butter
\item 2 cups sugar
\item 4 egg yolks
\item 1 cup milk
\item 3 1/2 cups flour
\item 5 teaspoons baking powder
\item 4 egg whitess
\item 1/2 teaspoon cinnamon
\item 1/4 teaspoon mace
\item 1/4 teaspoon nutmeg
\item 1/3 cup raisins, seeded and cut in pieces
\item 1/3 cup figs, finely chopped
\item 1 tablespoon molasses
\end{itemize}
\end{multicols}}
\end{minipage}

\vspace{0.3em}
\noindent%
Mix first seven ingredients in order given. Bake two-thirds of the
mixture in two layer cake pans. To the remainder add spices, fruit, and
molasses, and bake in a layer cake pan. Put layers together with jelly
(apple usually being preferred, as it has less flavor), having the dark
layer in the centre.



\needspace{15\baselineskip}
\section*{Golden Spice Cake}


\begin{minipage}{1.0\textwidth}
{\setlength{\multicolsep}{0pt}\setlength{\columnsep}{2em}\raggedcolumns%
\begin{multicols}{2}
\begin{itemize}
\setlength{\itemsep}{0pt}
\setlength{\parsep}{0pt}
\item 1/2 cup butter
\item 1/2 cup brown sugar
\item 1 egg
\item 4 egg yolks
\item 1/2 cup molasses
\item 1/2 cup milk
\item 2 1/4 cups flour
\item 1 teaspoon cinnamon
\item 1/2 teaspoon soda
\item 1/2 teaspoon clove
\item 1/4 teaspoon grated nutmeg
\item Few grains cayenne
\item Few gratings lemon rind
\end{itemize}
\end{multicols}}
\end{minipage}

\vspace{0.3em}
\noindent%
Cream the butter, add sugar gradually, egg and yolks of eggs well
beaten, molasses, milk, flour, mixed and sifted with spices, cayenne,
and lemon rind. Bake in a moderate oven one hour, and cover with White
Mountain Cream (see p. 528).



\needspace{15\baselineskip}
\section*{Walnut Mocha Cake}


\begin{minipage}{1.0\textwidth}
{\setlength{\multicolsep}{0pt}\setlength{\columnsep}{2em}\raggedcolumns%
\begin{multicols}{2}
\begin{itemize}
\setlength{\itemsep}{0pt}
\setlength{\parsep}{0pt}
\item 1/2 cup butter
\item 1 cup sugar
\item 1/2 cup coffee infusion
\item 1 3/4 cups flour
\item 2 1/2 teaspoons baking powder
\item 3 egg whites
\item 3/4 cup walnut meats, broken in pieces
\end{itemize}
\end{multicols}}
\end{minipage}

\vspace{0.3em}
\noindent%
Follow directions for mixing butter cake mixtures. Cover with
Confectioners' Frosting, using cream, and flavoring with vanilla.



\needspace{15\baselineskip}
\section*{Birthday Cake}


\begin{minipage}{1.0\textwidth}
{\setlength{\multicolsep}{0pt}\setlength{\columnsep}{2em}\raggedcolumns%
\begin{multicols}{2}
\begin{itemize}
\setlength{\itemsep}{0pt}
\setlength{\parsep}{0pt}
\item 1/2 cup butter
\item 1 1/4 cups brown sugar
\item 4 egg yolks
\item 2/3 cup milk
\item 2 1/4 cups flour
\item 3 1/2 teaspoons baking powder
\item 1 teaspoon orange extract
\item 1 teaspoon vanilla
\item 2 tablespoons Sherry
\item 1/2 cup raisins, seeded and cut in pieces
\item 1/2 cup walnut meats, cut in pieces
\item 1/3 cup currants
\item 2 tablespoons candied orange peel, finely cut
\item 2 egg whites
\end{itemize}
\end{multicols}}
\end{minipage}

\vspace{0.3em}
\noindent%
Follow directions for making butter cake mixtures. Bake in a buttered
and floured angel-cake pan in a slow oven one and one-quarter hours.
Cover with Ornamental Frosting (see p. 532).



\needspace{15\baselineskip}
\section*{Rich Coffee Cake}


\begin{minipage}{1.0\textwidth}
{\setlength{\multicolsep}{0pt}\setlength{\columnsep}{2em}\raggedcolumns%
\begin{multicols}{2}
\begin{itemize}
\setlength{\itemsep}{0pt}
\setlength{\parsep}{0pt}
\item 1 cup butter
\item 2 cups sugar
\item 4 eggs
\item 2 tablespoons molasses
\item 1 cup cold boiled coffee
\item 3 3/4 cups flour
\item 5 teaspoons baking powder
\item 1 teaspoon cinnamon
\item 1/2 teaspoon clove
\item 1/2 teaspoon mace
\item 1/2 teaspoon allspice
\item 3/4 cup raisins, seeded and cut in pieces
\item 3/4 cup currants
\item 1/4 cup citron, thinly sliced and cut in strips
\item 2 tablespoons brandy
\end{itemize}
\end{multicols}}
\end{minipage}

\vspace{0.3em}
\noindent%
Follow directions for making butter cake mixtures. Bake in deep cake
pans.



\needspace{15\baselineskip}
\section*{Nut Spice Cake}


\begin{minipage}{1.0\textwidth}
{\setlength{\multicolsep}{0pt}\setlength{\columnsep}{2em}\raggedcolumns%
\begin{multicols}{2}
\begin{itemize}
\setlength{\itemsep}{0pt}
\setlength{\parsep}{0pt}
\item 1/2 cup butter
\item 1 cup brown sugar
\item 1/2 cup molasses
\item 4 egg yolks
\item 1 cup sour milk
\item 2 1/2 cups flour
\item 1 teaspoon soda
\item 1 teaspoon cinnamon
\item 1/2 teaspoon clove
\item 1/4 nutmeg, grated
\item 1 cup raisins, seeded and cut in pieces
\item 1/2 cup currants
\item 1/2 cup English walnut meats, cut in pieces
\item 1 1/2 teaspoons baking powder
\end{itemize}
\end{multicols}}
\end{minipage}

\vspace{0.3em}
\noindent%
Mix ingredients in the order given. This recipe makes two loaves.



\needspace{15\baselineskip}
\section*{Dark Fruit Cake}


\begin{minipage}{1.0\textwidth}
{\setlength{\multicolsep}{0pt}\setlength{\columnsep}{2em}\raggedcolumns%
\begin{multicols}{2}
\begin{itemize}
\setlength{\itemsep}{0pt}
\setlength{\parsep}{0pt}
\item 1/2 cup butter
\item 3/4 cup brown sugar
\item 3/4 cup raisins, seeded and cut in pieces
\item 3/4 cup currants
\item 1/2 cup citron, thinly sliced and cut in strips
\item 1/2 cup molasses
\item 2 eggs
\item 1/2 cup milk
\item 2 cups flour
\item 1/2 teaspoon soda
\item 1 teaspoon cinnamon
\item 1/2 teaspoon allspice
\item 1/2 teaspoon mace
\item 1/4 teaspoon clove
\item 1/2 teaspoon lemon extract
\end{itemize}
\end{multicols}}
\end{minipage}

\vspace{0.3em}
\noindent%
Follow directions for mixing butter cake mixtures. Bake in deep cake
pans one and one-quarter hours.



\needspace{15\baselineskip}
\section*{Nut Cakes}

                         Meat from 1 lb. pecans

\begin{itemize}
\setlength{\itemsep}{0pt}
\setlength{\parsep}{0pt}
\item 1 lb. powdered sugar
\item 1/4 cup flour
\item 6 egg whitess
\item 1 teaspoon vanilla
\end{itemize}

\vspace{-0.5em}
\noindent%
Pound nut meat and mix with sugar and flour. Beat whites of eggs until
stiff, add first mixture and vanilla. Drop from tip of tablespoon
(allowing one spoonful for each cake) on a tin sheet covered with
buttered paper. Bake twenty minutes in a moderate oven.



\needspace{15\baselineskip}
\section*{Snow Cake}


\begin{minipage}{1.0\textwidth}
{\setlength{\multicolsep}{0pt}\setlength{\columnsep}{2em}\raggedcolumns%
\begin{multicols}{2}
\begin{itemize}
\setlength{\itemsep}{0pt}
\setlength{\parsep}{0pt}
\item 1/4 cup butter
\item 1 cup sugar
\item 1/2 cup milk
\item 1 2/3 cups flour
\item 2 1/2 teaspoons baking powder
\item 2 egg whites
\item 1/2 teaspoon vanilla or
\item 1/4 teaspoon almond extract
\end{itemize}
\end{multicols}}
\end{minipage}

\vspace{0.3em}
\noindent%
Follow recipe for mixing butter cakes. Bake forty-five minutes in a deep
narrow pan.



\needspace{15\baselineskip}
\section*{Lily Cake}


\begin{minipage}{1.0\textwidth}
{\setlength{\multicolsep}{0pt}\setlength{\columnsep}{2em}\raggedcolumns%
\begin{multicols}{2}
\begin{itemize}
\setlength{\itemsep}{0pt}
\setlength{\parsep}{0pt}
\item 1/3 cup butter
\item 1 cup sugar
\item 1/2 cup milk
\item 1 3/4 cups flour
\item 2 1/2 teaspoons baking powder
\item 3 egg whites
\item 1/3 teaspoon lemon extract
\item 2/3 teaspoon vanilla
\end{itemize}
\end{multicols}}
\end{minipage}

\vspace{0.3em}
\noindent%
Follow recipe for mixing butter cakes.



\needspace{15\baselineskip}
\section*{Corn-Starch Cake}


\begin{minipage}{1.0\textwidth}
{\setlength{\multicolsep}{0pt}\setlength{\columnsep}{2em}\raggedcolumns%
\begin{multicols}{2}
\begin{itemize}
\setlength{\itemsep}{0pt}
\setlength{\parsep}{0pt}
\item 1 cup butter
\item 2 cups sugar
\item 1 cup milk
\item 1 cup corn-starch
\item 2 cups flour
\item 4 1/2 teaspoons baking powder
\item 5 egg whitess
\item 3/4 teaspoon vanilla or
\item 1/2 teaspoon almond extract
\end{itemize}
\end{multicols}}
\end{minipage}

\vspace{0.3em}
\noindent%
Follow recipe for mixing butter cakes. This mixture makes two loaves.



\needspace{15\baselineskip}
\section*{Prune Almond Cake}

Bake one-half Corn-starch Cake mixture in a dripping-pan. Cut in two
crosswise, spread between layers Prune Almond Filling, and cover top
with White Mountain Cream.

\textbf{Prune Almond Filling.} To one-half the recipe for White Mountain Cream
add eight soft prunes stoned and cut in pieces, and one-fourth cup
almonds blanched and cut in pieces.



\needspace{15\baselineskip}
\section*{Marshmallow Cake}


\begin{minipage}{1.0\textwidth}
{\setlength{\multicolsep}{0pt}\setlength{\columnsep}{2em}\raggedcolumns%
\begin{multicols}{2}
\begin{itemize}
\setlength{\itemsep}{0pt}
\setlength{\parsep}{0pt}
\item 1/2 cup butter
\item 1 1/2 cups sugar
\item 1/2 cup milk
\item 2 cups flour
\item 3 teaspoons baking powder
\item 1/4 teaspoon cream of tartar
\item 5 egg whitess
\item 1 teaspoon vanilla
\end{itemize}
\end{multicols}}
\end{minipage}

\vspace{0.3em}
\noindent%
Follow recipe for mixing butter cakes. Bake in shallow pans, and put
Marshmallow Cream between the layers and on the top.



\needspace{15\baselineskip}
\section*{Fig Éclair}


\begin{minipage}{1.0\textwidth}
{\setlength{\multicolsep}{0pt}\setlength{\columnsep}{2em}\raggedcolumns%
\begin{multicols}{2}
\begin{itemize}
\setlength{\itemsep}{0pt}
\setlength{\parsep}{0pt}
\item 1/2 cup butter (scant)
\item 1 cup sugar
\item 1/2 cup milk
\item 1 7/8 cups flour
\item 3 teaspoons baking powder
\item 4 egg whitess
\item 1/2 teaspoon vanilla
\end{itemize}
\end{multicols}}
\end{minipage}

\vspace{0.3em}
\noindent%
Follow recipe for mixing butter cakes. Bake in shallow pans, put between
layers Fig Filling, and sprinkle top with powdered sugar.



\needspace{15\baselineskip}
\section*{Banana Cake}

Mix and bake Fig Éclair mixture; put between layers White Mountain Cream
covered with thin slices of banana, and frost the top. This should be
eaten the day it is made.



\needspace{15\baselineskip}
\section*{Bride's Cake}


\begin{minipage}{1.0\textwidth}
{\setlength{\multicolsep}{0pt}\setlength{\columnsep}{2em}\raggedcolumns%
\begin{multicols}{2}
\begin{itemize}
\setlength{\itemsep}{0pt}
\setlength{\parsep}{0pt}
\item 1/2 cup butter
\item 1 1/2 cups sugar
\item Whites six eggs
\item 1/2 cup milk
\item 2 1/2 cups flour
\item 3 teaspoons baking powder
\item 1/4 teaspoon cream of tartar
\item 1/2 teaspoon almond extract
\end{itemize}
\end{multicols}}
\end{minipage}

\vspace{0.3em}
\noindent%
Follow recipe for mixing butter cakes. Bake forty-five to fifty minutes
in deep, narrow pans. Cover with white frosting.



\needspace{15\baselineskip}
\section*{Ice Cream Cake}


\begin{minipage}{1.0\textwidth}
{\setlength{\multicolsep}{0pt}\setlength{\columnsep}{2em}\raggedcolumns%
\begin{multicols}{2}
\begin{itemize}
\setlength{\itemsep}{0pt}
\setlength{\parsep}{0pt}
\item 1/2 cup butter
\item 2 cups sugar
\item 1 cup milk
\item 3 cups flour
\item 4 teaspoons baking powder
\item 4 egg whitess
\item Vanilla
\end{itemize}
\end{multicols}}
\end{minipage}

\vspace{0.3em}
\noindent%
Follow recipe for mixing butter cakes. Bake in layers, and put between
layers and on top Ice Cream Frosting.



\needspace{15\baselineskip}
\section*{Light Fruit Cake}

To Fig Éclair mixture add one-half cup raisins seeded and cut in pieces,
two ounces citron thinly sliced and cut in strips, and one-third cup
walnut meat cut in pieces. In making mixture, reserve one tablespoon
flour to use for dredging fruit.



\needspace{15\baselineskip}
\section*{White Nut Cake}


\begin{minipage}{1.0\textwidth}
{\setlength{\multicolsep}{0pt}\setlength{\columnsep}{2em}\raggedcolumns%
\begin{multicols}{2}
\begin{itemize}
\setlength{\itemsep}{0pt}
\setlength{\parsep}{0pt}
\item 3/4 cup butter
\item 1 1/2 cups sugar
\item Whites 8 eggs
\item 1/2 cup milk
\item 2 1/2 cups flour
\item 1/2 teaspoon cream of tartar
\item 3 teaspoons baking powder
\item 1 cup walnut meat cut in pieces
\end{itemize}
\end{multicols}}
\end{minipage}

\vspace{0.3em}
\noindent%
Follow recipe for mixing butter cakes. This mixture makes two loaves.



\needspace{15\baselineskip}
\section*{Golden Cake}


\begin{minipage}{1.0\textwidth}
{\setlength{\multicolsep}{0pt}\setlength{\columnsep}{2em}\raggedcolumns%
\begin{multicols}{2}
\begin{itemize}
\setlength{\itemsep}{0pt}
\setlength{\parsep}{0pt}
\item 1/4 cup butter
\item 1/2 cup sugar
\item 1 teaspoon orange extract
\item 5 egg yolks
\item 1/4 cup milk
\item 7/8 cup flour
\item 1 1/2 teaspoons baking powder
\end{itemize}
\end{multicols}}
\end{minipage}

\vspace{0.3em}
\noindent%
Cream the butter, add sugar gradually, and yolks of eggs beaten until
thick and lemon-colored, and extract. Mix and sift flour and baking
powder, and add alternately with milk to first mixture. Omit orange
extract, add one-half cup nut meat cut in small pieces, and bake in
individual tins.



\needspace{15\baselineskip}
\section*{Mocha Cakes}

Bake a sponge cake mixture in sheets. Shape in small rounds, and cut in
three layers. Put layers together with a thin coating of frosting.
Spread frosting around sides and roll in shredded cocoanut. Ornament top
with frosting forced through a pastry bag and tube, using the rose tube.
Begin at centre of top and coil frosting around until surface is
covered. Garnish centre of top with a candied cherry.

\textbf{Frosting.} Wash one-third cup butter, add one cup powdered sugar
gradually, and beat until creamy. Then add one cup Cream Filling which
has been cooled. Flavor with one-half teaspoon vanilla and one and
one-half squares melted chocolate.

This frosting is sometimes colored pink, yellow, green, or lavender, and
flavored with rose, vanilla, or a combination of almond and vanilla.
Large Mocha Cakes are baked in two round layer cake tins, each cake
being cut in two layers. Layers are put together as small cakes. The top
is spread smoothly with frosting, then ornamented with large pieces of
candied fruits arranged in a design, and frosting forced through pastry
bag and tube.



\needspace{15\baselineskip}
\section*{Cream Cakes}


\begin{itemize}
\setlength{\itemsep}{0pt}
\setlength{\parsep}{0pt}
\item 1/2 cup butter
\item 1 cup boiling water
\item 4 eggs
\item 1 cup flour
\end{itemize}

\vspace{-0.5em}
\noindent%
Pour butter and water in saucepan and place on front of range. As soon
as boiling-point is reached, add flour all at once, and stir vigorously.
Remove from fire as soon as mixed, and add unbeaten eggs one at a time,
beating, until thoroughly mixed, between the addition of eggs. Drop by
spoonfuls on a buttered sheet, one and one-half inches apart, shaping
with handle of spoon as nearly circular as possible, having mixture
slightly piled in centre. Bake thirty minutes in a moderate oven. With a
sharp knife make a cut in each large enough to admit of Cream Filling.
This recipe makes eighteen small cream cakes. For flavoring cream
filling use lemon extract. If cream cakes are removed from oven before
being thoroughly cooked, they will fall. If in doubt, take one from
oven, and if it does not fall, this is sufficient proof that others are
cooked.



\needspace{15\baselineskip}
\section*{French Cream Cakes}

Fill Cream Cakes with Cream Sauce I.



\needspace{15\baselineskip}
\section*{French Strawberry Cream Cakes}

Shape cream cake mixture oblong, making twelves cakes. Split, and fill
with Strawberry Cream Filling.



\needspace{15\baselineskip}
\section*{Éclairs}

Shape cream cake mixture four and one-half inches long by one inch wide,
by forcing through a pastry bag and tube. Bake twenty-five minutes in a
moderate oven. Split, and fill with vanilla, coffee, or chocolate cream
filling. Frost with Confectioners' Frosting to which is added one-third
cup melted Fondant, dipping top of éclairs in frosting while it is hot.



\needspace{15\baselineskip}
\section*{Lemon Queens}


\begin{minipage}{1.0\textwidth}
{\setlength{\multicolsep}{0pt}\setlength{\columnsep}{2em}\raggedcolumns%
\begin{multicols}{2}
\begin{itemize}
\setlength{\itemsep}{0pt}
\setlength{\parsep}{0pt}
\item 1/4 lb. butter
\item 1/2 lb. sugar
\item Grated rind 1 lemon
\item 3/4 tablespoon lemon juice
\item 4 egg yolks
\item 5 ozs. flour
\item 1/4 teaspoon salt
\item 1/4 teaspoon soda (scant)
\item 4 egg whitess
\end{itemize}
\end{multicols}}
\end{minipage}

\vspace{0.3em}
\noindent%
Cream the butter, add sugar gradually, and continue beating. Then add
grated rind, lemon juice, and yolks of eggs beaten until thick and
lemon-colored. Mix and sift soda, salt, and flour; add to first mixture
and beat thoroughly. Add whites of eggs beaten stiff. Bake from twenty
to twenty-five minutes in small tins.



\needspace{15\baselineskip}
\section*{Queen Cake}


\begin{minipage}{1.0\textwidth}
{\setlength{\multicolsep}{0pt}\setlength{\columnsep}{2em}\raggedcolumns%
\begin{multicols}{2}
\begin{itemize}
\setlength{\itemsep}{0pt}
\setlength{\parsep}{0pt}
\item 2/3 cup butter
\item 2 cups flour (scant)
\item 1/4 teaspoon soda
\item 6 egg whitess
\item 1 1/4 cups powdered sugar
\item 1 1/2 teaspoons lemon juice
\end{itemize}
\end{multicols}}
\end{minipage}

\vspace{0.3em}
\noindent%
Cream the butter, add flour gradually, mixed and sifted with soda, then
add lemon juice. Beat whites of eggs until stiff; add sugar gradually,
and combine the mixtures. Bake fifty minutes in a long shallow pan.
Cover with Opera Caramel Frosting.



\needspace{15\baselineskip}
\section*{Pound Cake}


\begin{minipage}{1.0\textwidth}
{\setlength{\multicolsep}{0pt}\setlength{\columnsep}{2em}\raggedcolumns%
\begin{multicols}{2}
\begin{itemize}
\setlength{\itemsep}{0pt}
\setlength{\parsep}{0pt}
\item 1 lb. butter
\item 1 lb. sugar
\item Yolks 10 eggs
\item Whites 10 eggs
\item 1 lb. flour
\item 1/2 teaspoon mace
\item 2 tablespoons brandy
\end{itemize}
\end{multicols}}
\end{minipage}

\vspace{0.3em}
\noindent%
Cream the butter, add sugar gradually, and continue beating; then add
yolks of eggs beaten until thick and lemon-colored, whites of eggs
beaten until stiff and dry, flour, mace, and brandy. Beat vigorously
five minutes. Bake in a deep pan one and one-fourth hours in a slow
oven; or if to be used for fancy ornamented cakes, bake thirty to
thirty-five minutes in a dripping-pan.



\needspace{15\baselineskip}
\section*{New York Gingerbread}


\begin{minipage}{1.0\textwidth}
{\setlength{\multicolsep}{0pt}\setlength{\columnsep}{2em}\raggedcolumns%
\begin{multicols}{2}
\begin{itemize}
\setlength{\itemsep}{0pt}
\setlength{\parsep}{0pt}
\item 1 cup butter (scant)
\item 1 1/2 cups flour
\item 2 tablespoons yellow ginger
\item 5 eggs
\item 1 1/2 cups powdered sugar
\item 1 teaspoon baking powder
\end{itemize}
\end{multicols}}
\end{minipage}

\vspace{0.3em}
\noindent%
Cream the butter, and add flour gradually, mixed and sifted with ginger.
Beat the yolks of the eggs until thick and lemon-colored, and add sugar
gradually. Combine mixtures, add whites of eggs, beaten until stiff, and
sift over baking powder. Beat thoroughly, turn into a buttered deep cake
pan, and bake one hour in a moderate oven.



\needspace{15\baselineskip}
\section*{Newport Pound Cake}

Make same as New York Gingerbread, omitting ginger, and substituting one
teaspoon vanilla extract.



\needspace{15\baselineskip}
\section*{Christmas Cakes}

Bake Newport Pound Cake in golden-rod pans, cut in fourths crosswise,
spread with Ice Cream Frosting, and garnish with green leaves, made from
ornamental frosting, and round red candies to imitate berries.



\needspace{15\baselineskip}
\section*{Ginger Pound Cakes}

Cream one half pound butter and add gradually one-half pound sugar,
continuing the beating. Add three-fourths pound flour, mixed and sifted
with two teaspoons baking powder alternately with four eggs beaten until
thick and lemon-colored; then add one-half pound Canton ginger cut in
small pieces. Bake in small buttered and floured individual cake pans in
a slow oven. Cover with White Mountain Cream (see p. 528).



\needspace{15\baselineskip}
\section*{Molasses Pound Cake}


\begin{minipage}{1.0\textwidth}
{\setlength{\multicolsep}{0pt}\setlength{\columnsep}{2em}\raggedcolumns%
\begin{multicols}{2}
\begin{itemize}
\setlength{\itemsep}{0pt}
\setlength{\parsep}{0pt}
\item 2/3 cup butter
\item 3/4 cup sugar
\item 2 eggs
\item 2/3 cup milk
\item 2/3 cup molasses
\item 2 1/8 cups flour
\item 3/4 teaspoon soda
\item 1 teaspoon cinnamon
\item 1/2 teaspoon allspice
\item 1/4 teaspoon clove
\item 1/4 teaspoon mace
\item 1/2 cup raisins, seeded and cut in pieces
\item 1/3 cup citron, thinly sliced and cut in strips
\end{itemize}
\end{multicols}}
\end{minipage}

\vspace{0.3em}
\noindent%
Cream the butter, add sugar gradually, eggs well beaten, and milk and
molasses. Mix and sift flour with soda and spices, and add to first
mixture, then add fruit. Bake in small buttered tins from twenty-five to
thirty minutes in a moderate oven. This recipe makes twenty-four little
cakes.



\needspace{15\baselineskip}
\section*{English Fruit Cake}


\begin{minipage}{1.0\textwidth}
{\setlength{\multicolsep}{0pt}\setlength{\columnsep}{2em}\raggedcolumns%
\begin{multicols}{2}
\begin{itemize}
\setlength{\itemsep}{0pt}
\setlength{\parsep}{0pt}
\item 1 lb. butter
\item 1 lb. light brown sugar
\item 9 eggs
\item 1 lb. flour
\item 2 teaspoons mace
\item 2 teaspoons cinnamon
\item 1 teaspoon soda
\item 2 tablespoons milk
\item 3 lbs. currants
\item 2 lbs. raisins, seeded and finely chopped
\item 1/2 lb. almonds, blanched and shredded
\item 1 lb. citron, thinly sliced and cut in strips
\end{itemize}
\end{multicols}}
\end{minipage}

\vspace{0.3em}
\noindent%
Cream the butter, add sugar gradually, and beat thoroughly. Separate
yolks from whites of eggs; beat yolks until thick and lemon-colored,
whites until stiff and dry, and add to first mixture. Then add milk,
fruit, nuts, and flour mixed and sifted with mace, cinnamon, and soda.
Put in buttered deep pans, cover with buttered paper, steam three hours,
and bake one and one half hours in a slow oven, or bake four hours in a
very slow oven. Rich fruit cake is always more satisfactory when done if
the cooking is accomplished by steaming.



\needspace{15\baselineskip}
\section*{Wedding Cake I}


\begin{minipage}{1.0\textwidth}
{\setlength{\multicolsep}{0pt}\setlength{\columnsep}{2em}\raggedcolumns%
\begin{multicols}{2}
\begin{itemize}
\setlength{\itemsep}{0pt}
\setlength{\parsep}{0pt}
\item 1 lb. butter
\item 1 lb. sugar
\item 12 eggs
\item 1 lb. flour
\item 2 teaspoons cinnamon
\item 3/4 teaspoon nutmeg
\item 3/4 teaspoon allspice
\item 3/4 teaspoon mace
\item 1/2 teaspoon clove
\item 3 lbs. raisins, seeded and cut in pieces
\item 1 lb. currants
\item 1 lb. citron, thinly sliced and cut in strips
\item 1 lb. figs, finely chopped
\item 1/4 cup brandy
\item 2 tablespoons lemon juice
\end{itemize}
\end{multicols}}
\end{minipage}

\vspace{0.3em}
\noindent%
Cream the butter, add sugar gradually, and beat thoroughly. Separate
yolks from whites of eggs, beat yolks until thick and lemon-colored,
whites until stiff and dry, and add to first mixture. Add flour
(excepting one-third cup, which should be reserved to dredge fruit)
mixed and sifted with spices, brandy, and lemon juice. Then add fruit,
except citron, dredged with reserved flour. Dredge citron with flour and
put in layers between cake mixture when putting in the pan. Bake same as
English Fruit Cake.



\needspace{15\baselineskip}
\section*{Wedding Cake II}


\begin{minipage}{1.0\textwidth}
{\setlength{\multicolsep}{0pt}\setlength{\columnsep}{2em}\raggedcolumns%
\begin{multicols}{2}
\begin{itemize}
\setlength{\itemsep}{0pt}
\setlength{\parsep}{0pt}
\item 1 lb. butter
\item 1 lb. brown sugar
\item 12 eggs
\item 1 cup molasses
\item 1 lb. flour
\item 4 teaspoons cinnamon
\item 4 teaspoons allspice
\item 1 1/2 teaspoons mace
\item 1 nutmeg, grated
\item 1/4 teaspoon soda
\item 3 lbs. raisins, seeded and cut in pieces
\item 2 lbs. Sultana raisins
\item 1 1/2 lbs. citron, thinly sliced and cut in strips
\item 1 lb. currants
\item 1/2 preserved lemon rind
\item 1/2 preserved orange rind
\item 1 cup brandy
\item 4 squares chocolate, melted
\item 1 tablespoon hot water
\end{itemize}
\end{multicols}}
\end{minipage}

\vspace{0.3em}
\noindent%
Cream the butter, add sugar gradually, and beat thoroughly. Separate
yolks from whites of eggs, and beat yolks until thick and lemon-colored.
Add to first mixture, then add flour (excepting one-third cup, which
should be reserved to dredge fruit), mixed and sifted with spices, fruit
dredged with flour, lemon rind and orange rind finely chopped, brandy,
chocolate, and whites of eggs beaten until stiff and dry. Just before
putting into pans, add soda dissolved in hot water. Cover pans with
buttered paper, and steam four hours. Finish cooking by leaving in a
warm oven over night.



\needspace{15\baselineskip}
\section*{Imperial Cake}


\begin{minipage}{1.0\textwidth}
{\setlength{\multicolsep}{0pt}\setlength{\columnsep}{2em}\raggedcolumns%
\begin{multicols}{2}
\begin{itemize}
\setlength{\itemsep}{0pt}
\setlength{\parsep}{0pt}
\item 1/2 lb. butter
\item 1/2 lb. sugar
\item 5 egg yolks
\item 5 egg whitess
\item Grated rind 1/2 lemon
\item 2 teaspoons lemon juice
\item 1/2 lb. raisins, seeded and cut in pieces
\item 1/2 cup walnut meat, broken in pieces
\item 1/2 lb. flour
\item 1/4 teaspoon soda
\end{itemize}
\end{multicols}}
\end{minipage}

\vspace{0.3em}
\noindent%
Mix same as Pound Cake, adding raisins dredged with flour, and nuts at
the last.





\chapter{Cake Fillings And Frostings}




\needspace{15\baselineskip}
\section*{Cream Filling}


\begin{minipage}{1.0\textwidth}
{\setlength{\multicolsep}{0pt}\setlength{\columnsep}{2em}\raggedcolumns%
\begin{multicols}{2}
\begin{itemize}
\setlength{\itemsep}{0pt}
\setlength{\parsep}{0pt}
\item 7/8 cup sugar
\item 1/3 cup flour
\item 1/8 teaspoon salt
\item 2 eggs
\item 2 cups scalded milk
\item 1 teaspoon vanilla or
\item 1/2 teaspoon lemon extract
\end{itemize}
\end{multicols}}
\end{minipage}

\vspace{0.3em}
\noindent%
Mix dry ingredients, add eggs slightly beaten, and pour on gradually
scalded milk. Cook fifteen minutes in double boiler, stirring constantly
until thickened, afterwards occasionally. Cool and flavor.



\needspace{15\baselineskip}
\section*{Chocolate Cream Filling}

Put one and one-fourth squares Baker's chocolate in a saucepan and melt
over hot water. Add to Cream Filling, using in making one cup sugar in
place of seven-eighths cup.



\needspace{15\baselineskip}
\section*{Coffee Cream Filling}

Flavor Cream Filling with one and one-half tablespoons coffee extract.



\needspace{15\baselineskip}
\section*{French Cream Filling}


\begin{itemize}
\setlength{\itemsep}{0pt}
\setlength{\parsep}{0pt}
\item 3/4 cup thick cream
\item 1/4 cup milk
\item 1/4 cup powdered sugar
\item White one egg
\item 1/2 teaspoon vanilla
\end{itemize}

\vspace{-0.5em}
\noindent%
Dilute cream with milk and beat until stiff, using Dover egg-beater. Add
sugar, white of egg beaten until stiff, and vanilla.



\needspace{15\baselineskip}
\section*{Strawberry Filling}


\begin{itemize}
\setlength{\itemsep}{0pt}
\setlength{\parsep}{0pt}
\item 1 cup thick cream
\item 1/3 cup sugar
\item 1 egg white
\item 1/2 cup strawberries
\item 1/2 teaspoon vanilla
\end{itemize}

\vspace{-0.5em}
\noindent%
Beat cream until stiff, using Dover egg-beater, add sugar, white of egg
beaten until stiff, strawberries mashed, and vanilla.



\needspace{15\baselineskip}
\section*{Lemon Filling}


\begin{minipage}{1.0\textwidth}
{\setlength{\multicolsep}{0pt}\setlength{\columnsep}{2em}\raggedcolumns%
\begin{multicols}{2}
\begin{itemize}
\setlength{\itemsep}{0pt}
\setlength{\parsep}{0pt}
\item 1 cup sugar
\item 2 1/2 tablespoons flour
\item Grated rind 2 lemons
\item 1/4 cup lemon juice
\item 1 egg
\item 1 teaspoon butter
\end{itemize}
\end{multicols}}
\end{minipage}

\vspace{0.3em}
\noindent%
Mix sugar and flour, add grated rind, lemon juice, and egg slightly
beaten. Put butter in saucepan; when melted, add mixture, and stir
constantly until boiling-point is reached. Care must be taken that
mixture does not adhere to bottom of saucepan. Cool before spreading.



\needspace{15\baselineskip}
\section*{Orange Filling}


\begin{minipage}{1.0\textwidth}
{\setlength{\multicolsep}{0pt}\setlength{\columnsep}{2em}\raggedcolumns%
\begin{multicols}{2}
\begin{itemize}
\setlength{\itemsep}{0pt}
\setlength{\parsep}{0pt}
\item 1/2 cup sugar
\item 2 1/2 tablespoons flour
\item Grated rind 1/2 orange
\item 1/4 cup orange juice
\item 1/2 tablespoon lemon juice
\item 1 egg slightly beaten
\item 1 teaspoon butter
\end{itemize}
\end{multicols}}
\end{minipage}

\vspace{0.3em}
\noindent%
Mix ingredients in order given. Cook ten minutes in double boiler,
stirring constantly. Cool before spreading.



\needspace{15\baselineskip}
\section*{Chocolate Filling}


\begin{itemize}
\setlength{\itemsep}{0pt}
\setlength{\parsep}{0pt}
\item 2 1/2 squares chocolate
\item 1 cup powdered sugar
\item 3 tablespoons milk
\item Yolk 1 egg
\item 1/2 teaspoon vanilla
\end{itemize}

\vspace{-0.5em}
\noindent%
Melt chocolate over hot water, add one-half the sugar, and milk; add
remaining sugar, and yolk of egg; then cook in double boiler until it
thickens, stirring constantly at first, that mixture may be perfectly
smooth. Cool slightly, flavor, and spread.



\needspace{15\baselineskip}
\section*{Nut Or Fruit Filling}

To White Mountain Cream add chopped walnuts, almonds, figs, dates, or
raisins, separately or in combination.



\needspace{15\baselineskip}
\section*{Cocoanut Filling}


\begin{itemize}
\setlength{\itemsep}{0pt}
\setlength{\parsep}{0pt}
\item 2 egg whites
\item Fresh grated cocoanut
\item Powdered sugar
\end{itemize}

\vspace{-0.5em}
\noindent%
Beat whites of eggs on a platter with a fork until stiff. Add enough
powdered sugar to spread. Spread over cake, sprinkle thickly with
cocoanut. Use for layer cake, having filling between and on top.



\needspace{15\baselineskip}
\section*{Lemon Cocoanut Cream}

                     Juice and grated rind 1 lemon

\begin{itemize}
\setlength{\itemsep}{0pt}
\setlength{\parsep}{0pt}
\item 1 cup powdered sugar
\item 4 egg yolks
\item 1 cup shredded cocoanut
\end{itemize}

\vspace{-0.5em}
\noindent%
Mix lemon juice and rind with sugar and yolks of eggs slightly beaten;
cook ten minutes in double boiler, stirring constantly; then add
cocoanut. Cool, and use as a filling for Corn-starch Cake, or any cake
made from the whites of eggs.



\needspace{15\baselineskip}
\section*{Fig Filling}


\begin{itemize}
\setlength{\itemsep}{0pt}
\setlength{\parsep}{0pt}
\item 1/2 lb. figs, finely chopped
\item 1/3 cup sugar
\item 1/3 cup boiling water
\item 1 tablespoon lemon juice
\end{itemize}

\vspace{-0.5em}
\noindent%
Mix ingredients in the order given and cook in double boiler until thick
enough to spread. Spread while hot. Figs may be chopped quickly by
forcing through a meat chopper, stirring occasionally.



\needspace{15\baselineskip}
\section*{Marshmallow Paste}


\begin{itemize}
\setlength{\itemsep}{0pt}
\setlength{\parsep}{0pt}
\item 3/4 cup sugar
\item 1/4 cup milk
\item 1/4 lb. marshmallows
\item 2 tablespoons hot water
\item 1/2 teaspoon vanilla
\end{itemize}

\vspace{-0.5em}
\noindent%
Put sugar and milk in a saucepan, heat slowly to boiling-point without
stirring, and boil six minutes. Break marshmallows in pieces and melt in
double boiler, add hot water, and cook until mixture is smooth, then add
hot syrup gradually, stirring constantly. Beat until cool enough to
spread, then add vanilla. This may be used for both filling and
frosting.



\needspace{15\baselineskip}
\section*{Pistachio Paste}

To Marshmallow Paste add a few drops extract of almond, one-third cup
pistachio nuts blanched and chopped, and leaf green to color. Use same
as Marshmallow Paste.



\needspace{15\baselineskip}
\section*{Prune Almond Filling}

To White Mountain Cream (see p. 528) add one-half cup selected prunes,
stoned and cut in pieces, and one-third cup almonds blanched and
chopped.



\needspace{15\baselineskip}
\section*{Confectioners' Frosting}


\begin{itemize}
\setlength{\itemsep}{0pt}
\setlength{\parsep}{0pt}
\item 2 tablespoons boiling water or cream
\item Confectioners' sugar
\item Flavoring
\end{itemize}

\vspace{-0.5em}
\noindent%
To liquid add enough sifted sugar to make of right consistency to
spread; then add flavoring. Fresh fruit juice may be used in place of
boiling water. This is a most satisfactory frosting, and is both easily
and quickly made.



\needspace{15\baselineskip}
\section*{Orange Frosting}

                       Grated rind 1 orange

\begin{itemize}
\setlength{\itemsep}{0pt}
\setlength{\parsep}{0pt}
\item 1 teaspoon brandy
\item 1/2 teaspoon lemon juice
\item 1 tablespoon orange juice
\item Yolk 1 egg
\item Confectioners' sugar
\end{itemize}

\vspace{-0.5em}
\noindent%
Add rind to brandy and fruit juices; let stand fifteen minutes. Strain,
and add gradually to yolk of egg slightly beaten. Stir in confectioners'
sugar until of right consistency to spread.



\needspace{15\baselineskip}
\section*{Gelatine Frosting}


\begin{itemize}
\setlength{\itemsep}{0pt}
\setlength{\parsep}{0pt}
\item 2 1/2 tablespoons boiling water
\item 1/2 teaspoon granulated gelatine
\item 3/4 cup confectioners' sugar
\item 1/2 teaspoon vanilla
\end{itemize}

\vspace{-0.5em}
\noindent%
Dissolve gelatine in boiling water. Add sugar and flavoring and beat
until of right consistency to spread. Crease in squares when slightly
hardened.



\needspace{15\baselineskip}
\section*{Plain Frosting}


\begin{itemize}
\setlength{\itemsep}{0pt}
\setlength{\parsep}{0pt}
\item 1 egg white
\item 2 teaspoons cold water
\item 3/4 cup confectioners' sugar
\item 1/2 teaspoon vanilla or
\item 1/2 tablespoon lemon juice
\end{itemize}

\vspace{-0.5em}
\noindent%
Beat white of egg until stiff; add water and sugar. Beat thoroughly,
then add flavoring. Use more sugar if needed. Spread with a broad-bladed
knife.



\needspace{15\baselineskip}
\section*{Chocolate Frosting I}


\begin{minipage}{1.0\textwidth}
{\setlength{\multicolsep}{0pt}\setlength{\columnsep}{2em}\raggedcolumns%
\begin{multicols}{2}
\begin{itemize}
\setlength{\itemsep}{0pt}
\setlength{\parsep}{0pt}
\item 1 1/2 squares chocolate
\item 1/3 cup scalded cream
\item Few grains salt
\item Yolk 1 egg
\item 1/2 teaspoon melted butter
\item Confectioners' sugar
\item 1/2 teaspoon vanilla
\end{itemize}
\end{multicols}}
\end{minipage}

\vspace{0.3em}
\noindent%
Melt chocolate over hot water, add cream gradually, salt, yolk of egg,
and butter. Stir in confectioners' sugar until of right consistency to
spread; then add flavoring.



\needspace{15\baselineskip}
\section*{Chocolate Frosting II}


\begin{itemize}
\setlength{\itemsep}{0pt}
\setlength{\parsep}{0pt}
\item 1 3/4 cups sugar
\item 3/4 cup hot water
\item 4 squares chocolate, melted
\item 1/2 teaspoon vanilla
\end{itemize}

\vspace{-0.5em}
\noindent%
Boil sugar and water, without stirring, until syrup will thread when
dropped from tip of spoon. Pour syrup gradually on melted chocolate, and
continue beating until of right consistency to spread; then add
flavoring.



\needspace{15\baselineskip}
\section*{Chocolate Frosting III}


\begin{itemize}
\setlength{\itemsep}{0pt}
\setlength{\parsep}{0pt}
\item 2 squares chocolate
\item 1 teaspoon butter
\item 3 tablespoons hot water
\item Confectioners' sugar
\item 1/4 teaspoon vanilla
\end{itemize}

\vspace{-0.5em}
\noindent%
Melt chocolate over boiling water, add butter and hot water. Cool, and
add sugar to make of right consistency to spread. Flavor with vanilla.



\needspace{15\baselineskip}
\section*{White Mountain Cream}


\begin{itemize}
\setlength{\itemsep}{0pt}
\setlength{\parsep}{0pt}
\item 1 cup sugar
\item 1/3 cup boiling water
\item 1 egg white
\item 1 teaspoon vanilla or
\item 1/2 tablespoon lemon juice
\end{itemize}

\vspace{-0.5em}
\noindent%
Put sugar and water in saucepan, and stir to prevent sugar from adhering
to saucepan; heat gradually to boiling-point, and boil without stirring
until syrup will thread when dropped from tip of spoon or tines of
silver fork. Pour syrup gradually on beaten white of egg, beating
mixture constantly, and continue beating until of right consistency to
spread; then add flavoring and pour over cake, spreading evenly with
back of spoon. Crease as soon as firm. If not beaten long enough,
frosting will run; if beaten too long, it will not be smooth. Frosting
beaten too long may be improved by adding a few drops of lemon juice or
boiling water. This frosting is soft inside, and has a glossy surface.
If frosting is to be ornamented with nuts or candied cherries, place
them on frosting as soon as spread.



\needspace{15\baselineskip}
\section*{Ice Cream Frosting}


\begin{itemize}
\setlength{\itemsep}{0pt}
\setlength{\parsep}{0pt}
\item 2 cups sugar
\item 6 tablespoons water
\item 2 egg whites
\item 1/2 teaspoon vanilla
\end{itemize}

\vspace{-0.5em}
\noindent%
Follow directions for White Mountain Cream.



\needspace{15\baselineskip}
\section*{Boiled Frosting}


\begin{itemize}
\setlength{\itemsep}{0pt}
\setlength{\parsep}{0pt}
\item 1 cup sugar
\item 1/2 cup water
\item 2 egg whites
\item 1 teaspoon vanilla, or
\item 1/2 tablespoon lemon juice
\end{itemize}

\vspace{-0.5em}
\noindent%
Make same as White Mountain Cream. This frosting, on account of the
larger quantity of egg, does not stiffen so quickly as White Mountain
Cream, therefore is more successfully made by the inexperienced.



\needspace{15\baselineskip}
\section*{Boiled Chocolate Frosting}

To White Mountain Cream or Boiled Frosting add one and one half squares
melted chocolate as soon as syrup is added to whites of eggs.



\needspace{15\baselineskip}
\section*{Brown Frosting}

Make same as Boiled Frosting, using brown sugar in place of white sugar.



\needspace{15\baselineskip}
\section*{Maple Sugar Frosting}


\begin{itemize}
\setlength{\itemsep}{0pt}
\setlength{\parsep}{0pt}
\item 1 lb. soft maple sugar
\item 1/2 cup boiling water
\item 2 egg whites
\end{itemize}

\vspace{-0.5em}
\noindent%
Break sugar in small pieces, put in saucepan with boiling water, and
stir occasionally until sugar is dissolved. Boil without stirring until
syrup will thread when dropped from tip of spoon. Pour syrup gradually
on beaten whites, beating mixture constantly, and continue beating until
of right consistency to spread.



\needspace{15\baselineskip}
\section*{Cream Maple Sugar Frosting}


\begin{itemize}
\setlength{\itemsep}{0pt}
\setlength{\parsep}{0pt}
\item 1 lb. soft maple sugar
\item 1 cup cream
\end{itemize}

\vspace{-0.5em}
\noindent%
Break sugar in small pieces, put in saucepan with cream, and stir
occasionally until sugar is dissolved. Boil without stirring until a
ball can be formed when mixture is tried in cold water. Beat until of
right consistency to spread.



\needspace{15\baselineskip}
\section*{Milk Frosting}


\begin{itemize}
\setlength{\itemsep}{0pt}
\setlength{\parsep}{0pt}
\item 1 1/2 cups sugar
\item 1/2 cup milk
\item 1 teaspoon butter
\item 1/2 teaspoon vanilla
\end{itemize}

\vspace{-0.5em}
\noindent%
Put butter in saucepan; when melted, add sugar and milk. Stir, to be
sure that sugar does not adhere to saucepan, heat to boiling-point, and
boil without stirring thirteen minutes. Remove from fire, and beat until
of right consistency to spread; then add flavoring and pour over cake,
spreading evenly with back of spoon. Crease as soon as firm.



\needspace{15\baselineskip}
\section*{Caramel Frosting I}

Make same as Milk Frosting, adding one and one-half squares melted
chocolate as soon as boiling-point is reached, and flavoring with
one-eighth teaspoon cinnamon.



\needspace{15\baselineskip}
\section*{Caramel Frosting II}


\begin{itemize}
\setlength{\itemsep}{0pt}
\setlength{\parsep}{0pt}
\item 1 1/3 cups sugar
\item 2/3 cup grated maple sugar
\item 1/2 cup butter
\item 2/3 cup cream
\end{itemize}

\vspace{-0.5em}
\noindent%
Mix ingredients and boil thirteen minutes. Beat until of right
consistency to spread.



\needspace{15\baselineskip}
\section*{Nut Caramel Frosting}


\begin{minipage}{1.0\textwidth}
{\setlength{\multicolsep}{0pt}\setlength{\columnsep}{2em}\raggedcolumns%
\begin{multicols}{2}
\begin{itemize}
\setlength{\itemsep}{0pt}
\setlength{\parsep}{0pt}
\item 1 1/4 cups brown sugar
\item 1/3 cup water
\item 1/4 cup white sugar
\item 2 egg whites
\item 1 teaspoon vanilla
\item 1/4 cup English walnut meats, broken in pieces
\end{itemize}
\end{multicols}}
\end{minipage}

\vspace{0.3em}
\noindent%
Boil sugar and water as for White Mountain Cream. Pour gradually, while
beating constantly, on beaten whites of eggs, and continue the beating
until mixture is nearly cool. Set pan containing mixture in pan of
boiling water, and cook over range, stirring constantly, until mixture
becomes granular around edge of pan. Remove from pan of hot water and
beat, using a spoon, until mixture will hold its shape. Add nuts and
vanilla, pour on cake, and spread with back of spoon, leaving a rough
surface.



\needspace{15\baselineskip}
\section*{Opera Caramel Frosting}


\begin{itemize}
\setlength{\itemsep}{0pt}
\setlength{\parsep}{0pt}
\item 1 1/2 cups brown sugar
\item 3/4 cup thin cream
\item 1/2 tablespoon butter
\end{itemize}

\vspace{-0.5em}
\noindent%
Boil ingredients together in a smooth granite saucepan until a ball can
be formed when mixture is tried in cold water. It takes about forty
minutes for boiling. Beat until of right consistency to spread.



\needspace{15\baselineskip}
\section*{Chocolate Fudge Frosting}


\begin{minipage}{1.0\textwidth}
{\setlength{\multicolsep}{0pt}\setlength{\columnsep}{2em}\raggedcolumns%
\begin{multicols}{2}
\begin{itemize}
\setlength{\itemsep}{0pt}
\setlength{\parsep}{0pt}
\item 1 1/2 tablespoons butter
\item 1/3 cup unsweetened powdered cocoa
\item 1 1/4 cups confectioners' sugar
\item Few grains salt
\item 1/4 cup milk
\item 1/2 teaspoon vanilla
\end{itemize}
\end{multicols}}
\end{minipage}

\vspace{0.3em}
\noindent%
Melt butter, add cocoa, sugar, salt, and milk. Heat to boiling-point,
and boil about eight minutes. Remove from fire and beat until creamy.
Add vanilla and pour over cake.



\needspace{15\baselineskip}
\section*{Mocha Frosting}


\begin{itemize}
\setlength{\itemsep}{0pt}
\setlength{\parsep}{0pt}
\item 1/3 cup butter
\item 1 1/2 cups confectioners' sugar
\item 1 tablespoon breakfast cocoa
\item Coffee infusion
\end{itemize}

\vspace{-0.5em}
\noindent%
Cream butter, and add sugar gradually, continuing the beating; then add
cocoa and coffee infusion, drop by drop, until of right consistency to
spread or force through a pastry bag and tube.



\needspace{15\baselineskip}
\section*{Fondant Icing}

The mixture in which small cakes are dipped for icing is fondant, the
recipe for which may be found in chapter on Confections. Cakes for
dipping must first be glazed.

\textbf{To Glaze Cakes.} Beat white of one egg slightly, and add one tablespoon
powdered sugar. Apply with a brush to top and sides of cakes. After
glazing, cakes should stand over night before dipping.

\textbf{To Dip Cakes.} Melt fondant over hot water, and color and flavor as
desired. Stir, to prevent crust from forming on top. Take cake to be
dipped on a three-tined fork and lower in fondant three-fourths the
depth of cake. Remove from fondant, invert, and slip from fork to a
board. Decorate with ornamental frosting and nut meat, candied cherries,
angelica, or candied violets. For small ornamented cakes, pound cake
mixture is baked a little more than one inch thick in shallow pans, and
when cool cut in squares, diamonds, triangles, circles, crescents, etc.



\needspace{15\baselineskip}
\section*{Marshmallow Frosting}

Melt one cup white fondant; add the white of one egg beaten until stiff,
and stir over the fire two minutes. Remove from range, and beat until of
right consistency to spread. Flavor with one-fourth teaspoon water white
vanilla. This is a most delicious frosting for chocolate cake, but will
never spread perfectly smooth.



\needspace{15\baselineskip}
\section*{Ornamental Frosting I}


\begin{itemize}
\setlength{\itemsep}{0pt}
\setlength{\parsep}{0pt}
\item 2 cups sugar
\item 1 cup water
\item 3 egg whites
\item 1/4 teaspoon tartaric acid
\end{itemize}

\vspace{-0.5em}
\noindent%
Boil sugar and water until syrup when dropped from tip of spoon forms a
long thread. Pour syrup gradually on beaten whites of eggs, beating
constantly; then add acid and continue beating. When stiff enough to
spread, put a thin coating over cake. Beat remaining frosting until cold
and stiff enough to keep in shape after being forced through a pastry
tube. After first coating on cake has hardened, cover with a thicker
layer, and crease for cutting. If frosting is too stiff to spread
smoothly, thin with a few drops of water. With a pastry bag and variety
of tubes, cake may be ornamented as desired.



\needspace{15\baselineskip}
\section*{Ornamental Frosting II}


\begin{itemize}
\setlength{\itemsep}{0pt}
\setlength{\parsep}{0pt}
\item 3 egg whites
\item 1 tablespoon lemon juice
\item Confectioners' sugar, sifted
\end{itemize}

\vspace{-0.5em}
\noindent%
Put eggs in a large bowl, add two tablespoons sugar, and beat three
minutes, using a perforated wooden spoon. Repeat until one and one-half
cups sugar are used. Add lemon juice gradually, as mixture thickens.
Continue adding sugar by spoonfuls, and beating until frosting is stiff
enough to spread. This may be determined by taking up some of mixture on
back of spoon, and with a case knife making a cut through mixture; if
knife makes a clean cut and frosting remains parted, it is of right
consistency. Spread cake thinly with frosting; when this has hardened,
put on a thicker layer, having mixture somewhat stiffer than first
coating, and then crease for cutting. To remaining frosting add enough
more sugar, that frosting may keep in shape after being forced through a
pastry bag and tube.

With a pastry bag and variety of tubes, cake may be ornamented as
desired.









\chapter{Fancy Cakes And Confections}



Almond paste for making macaroons and small fancy cakes may be bought of
dealers who keep confectioners' supplies, although sometimes a resident
baker or confectioner will sell a small quantity. Almond paste is put up
in five-pound tin pails, and retails for one and one-half dollars per
pail. During the cold weather it will keep after being opened for a long
time.



\needspace{15\baselineskip}
\section*{Macaroons}


\begin{itemize}
\setlength{\itemsep}{0pt}
\setlength{\parsep}{0pt}
\item 1/2 lb. almond paste
\item 3 egg whites
\item 3/8 lb. powdered sugar
\end{itemize}

\vspace{-0.5em}
\noindent%
Work together almond paste and sugar on a smooth board or marble slab.
Then add whites of eggs gradually, and work until mixture is perfectly
smooth. Confectioners at first use the hand, afterwards a palette knife,
which is not only of use for mixing but for keeping board clean. Shape,
using a pastry bag and tube, on a tin sheet covered with buttered paper,
one-half inch apart; or drop mixture from tip of spoon in small piles.
Macaroon mixture is stiff enough to hold its shape, but in baking
spreads. Bake fifteen to twenty minutes in a slow oven. If liked soft,
they should be slightly baked. After removing from oven, invert paper,
and wet with a cloth wrung out of cold water, when macaroons will easily
slip off.



\needspace{15\baselineskip}
\section*{Almond Macaroons}

Sprinkle Macaroons, before baking, with almonds blanched and shredded,
or chopped.



\needspace{15\baselineskip}
\section*{Crescents}


\begin{itemize}
\setlength{\itemsep}{0pt}
\setlength{\parsep}{0pt}
\item 1/2 lb. almond paste
\item 2 ozs. confectioners' sugar
\item White 1 small egg
\item Almonds, blanched and finely chopped
\end{itemize}

\vspace{-0.5em}
\noindent%
Mix same as Macaroons. Shape mixture, which is quite soft, in a long
roll. Cut pieces from roll three-fourths inch long. Roll each separately
in chopped nuts, at the same time shaping to form a crescent. Bake
twenty minutes on a buttered tin sheet in a slow oven. Cool, and frost
with Confectioners' Frosting, made thin enough to apply with a brush,
and flavored with lemon juice until quite acid. Other nuts may be used
in place of almonds.



\needspace{15\baselineskip}
\section*{Cinnamon Bars}


\begin{itemize}
\setlength{\itemsep}{0pt}
\setlength{\parsep}{0pt}
\item 10 ozs. almond paste
\item 5 ozs. confectioners' sugar
\item 1 egg white
\item 1/2 teaspoon cinnamon
\end{itemize}

\vspace{-0.5em}
\noindent%
Mix same as Macaroons. Dredge a board with sugar, knead mixture
slightly, and shape in a long roll. Pat, and roll one-fourth inch thick,
using a rolling-pin. After rolling the piece should be four inches wide.
Spread with frosting made of white of one egg and two-thirds cup
confectioners' sugar beaten together until stiff enough to spread. Cut
in strips four inches long by three-fourths inch wide. This must be
quickly done, as a crust soon forms over frosting. To accomplish this,
use two knives, one placed through mixture where dividing line is to be
made, and the other used to make a clean sharp cut on both sides of
first knife. Knives should be kept clean by wiping on a damp cloth.
Remove strips as soon as cut, to a tin sheet, greased with lard and then
floured. Bake twenty minutes on centre grate in a slow oven.



\needspace{15\baselineskip}
\section*{Horseshoes}

Use Cinnamon Bar mixture. Cover with frosting colored with fruit red.
Cut in strips six inches long by one-half inch wide. As soon as cut,
shape quickly, at the same time carefully, in form of horseshoes. Bake
same as Cinnamon Bars. When cool, make eight dots with chocolate
frosting to represent nails.



\needspace{15\baselineskip}
\section*{Cocoanut Cakes I}


\begin{itemize}
\setlength{\itemsep}{0pt}
\setlength{\parsep}{0pt}
\item 1/2 lb. fresh grated cocoanut
\item Whites 1 1/2 eggs
\item 6 ozs. sugar and glucose, using one mixing-spoon glucose
\end{itemize}

\vspace{-0.5em}
\noindent%
Cook cocoanut, sugar, and glucose in double boiler until mixture clings
to spoon, add whites of eggs, stir vigorously, and cook until mixture
feels sticky when tried between the fingers. Spread in a wet pan, cover
with wet paper, and chill on ice. Shape in small balls, first dipping
hands in cold water. Bake twenty minutes in a slow oven on a tin sheet
greased with white wax.



\needspace{15\baselineskip}
\section*{Cocoanut Cakes II}


\begin{itemize}
\setlength{\itemsep}{0pt}
\setlength{\parsep}{0pt}
\item 1 lb. fresh grated cocoanut
\item 3/4 lb. sugar
\item 2 egg whites
\end{itemize}

\vspace{-0.5em}
\noindent%
Cook, shape, and bake same as Cocoanut Cakes I.



\needspace{15\baselineskip}
\section*{Stuffed Dates I}

Make a cut the entire length of dates and remove stones. Fill cavities
with castanea nuts, English walnuts, or blanched almonds, and shape in
original form. Roll in granulated sugar. Pile in rows on a small plate
covered with a doily. If castanea nuts are used, with a sharp knife cut
off the brown skin which lies next to shell.



\needspace{15\baselineskip}
\section*{Stuffed Dates II}

Remove stones from dates and fill cavities with Neufchâtel cheese.



\needspace{15\baselineskip}
\section*{Salted Almonds I}

Blanch one-fourth pound Jordan almonds and dry on a towel. Put one-third
cup olive oil in a very small saucepan. When hot, put in one-fourth of
the almonds and fry until delicately browned, stirring to keep almonds
constantly in motion. Remove with a spoon or small skimmer, taking up as
little oil as possible. Drain on brown paper and sprinkle with salt;
repeat until all are fried. It may be necessary to remove some of the
salt by wiping nuts with a napkin.



\needspace{15\baselineskip}
\section*{Salted Almonds II}

Prepare almonds as for Salted Almonds I. Fry in one-third cup fat, using
half lard and half clarified butter or all cocoanut butter. Drain, and
sprinkle with salt.



\needspace{15\baselineskip}
\section*{Salted Peanuts}

In buying peanuts for salting, get those which have not been roasted.
Remove skins and fry same as Salted Almonds I or II.



\needspace{15\baselineskip}
\section*{Salted Pecans}

Shelled pecans may be bought by the pound, which is much the best way
when used for salting, as it is difficult to remove the nut meat without
breaking. Fry same as salted Almonds I or II. Care must be taken that
they do not remain in fat too long; having a dark skin, color does not
determine when they are sufficiently cooked.



\needspace{15\baselineskip}
\section*{Parisian Sweets}


\begin{itemize}
\setlength{\itemsep}{0pt}
\setlength{\parsep}{0pt}
\item 1 lb. figs
\item 1 lb. dates
\item 1 lb. English walnut meat
\item Confectioners' sugar
\end{itemize}

\vspace{-0.5em}
\noindent%
Pick over and remove stems from figs and stones from dates. Mix fruit
with walnut meat, and force through a meat chopper. Work, using the
hands, on a board dredged with confectioners' sugar, until well blended.
Roll to one-fourth inch thickness, using confectioners' sugar for
dredging board and pin. Shape with a small round cutter, first dipped in
sugar, or cut with a sharp knife in three-fourth inch squares. Roll each
piece in confectioners' sugar, and shake to remove superfluous sugar.
Pack in layers in a tin box, putting paper between each layer. These
confections may be used at dinner in place of bonbons or ginger chips. A
combination of nut meat (walnut, almond, and filbert) may be used in
equal proportions.



\needspace{15\baselineskip}
\section*{Sugared Popped Corn}


\begin{itemize}
\setlength{\itemsep}{0pt}
\setlength{\parsep}{0pt}
\item 2 quarts popped corn
\item 2 tablespoons butter
\item 2 cups brown sugar
\item 1/2 cup water
\end{itemize}

\vspace{-0.5em}
\noindent%
Put butter in saucepan, and when melted add sugar and water. Bring to
boiling-point, and let boil sixteen minutes. Pour over corn, and stir
until every kernel is well coated with sugar.



\needspace{15\baselineskip}
\section*{Molasses Candy}


\begin{itemize}
\setlength{\itemsep}{0pt}
\setlength{\parsep}{0pt}
\item 2 cups Porto Rico molasses
\item 2/3 cup sugar
\item 3 tablespoons butter
\item 1 tablespoon vinegar
\end{itemize}

\vspace{-0.5em}
\noindent%
An iron kettle with a rounding bottom (Scotch kettle) or copper kettle
is best for candy making. If one has no copper kettle, a granite kettle
is best for sugar candies.

Put butter in kettle, place over fire, and when melted, add molasses and
sugar. Stir until sugar is dissolved. During the first of the boiling
stirring is unnecessary, but when nearly cooked, it should be constantly
stirred. Boil until, when tried in cold water, mixture will become
brittle. Add vinegar just before taking from fire. Pour into a well
buttered pan. When cool enough to handle, pull until porous and
light-colored, allowing candy to come in contact with tips of fingers
and thumbs, not to be squeezed in the hand. Cut in small pieces, using
large shears or a sharp knife, and then arrange on slightly buttered
plates to cool.



\needspace{15\baselineskip}
\section*{Velvet Molasses Candy}


\begin{minipage}{1.0\textwidth}
{\setlength{\multicolsep}{0pt}\setlength{\columnsep}{2em}\raggedcolumns%
\begin{multicols}{2}
\begin{itemize}
\setlength{\itemsep}{0pt}
\setlength{\parsep}{0pt}
\item 1 cup molasses
\item 3 cups sugar
\item 1 cup boiling water
\item 3 tablespoons vinegar
\item 1/2 teaspoon cream of tartar
\item 1/2 cup melted butter
\item 1/4 teaspoon soda
\end{itemize}
\end{multicols}}
\end{minipage}

\vspace{0.3em}
\noindent%
Put first four ingredients in kettle placed over front of range. As soon
as boiling-point is reached, add cream of tartar. Boil until, when tried
in cold water, mixture will become brittle. Stir constantly during last
part of cooking. When nearly done, add butter and soda. Pour into a
buttered pan and pull same as Molasses Candy. While pulling, add one
teaspoon vanilla, one-half teaspoon lemon extract, few drops oil of
peppermint, or few drops oil of wintergreen.



\needspace{15\baselineskip}
\section*{Buttercups}


\begin{minipage}{1.0\textwidth}
{\setlength{\multicolsep}{0pt}\setlength{\columnsep}{2em}\raggedcolumns%
\begin{multicols}{2}
\begin{itemize}
\setlength{\itemsep}{0pt}
\setlength{\parsep}{0pt}
\item 2 cups molasses
\item 1 cup sugar
\item 1/2 cup boiling water
\item 2 tablespoons butter
\item 1/3 teaspoon cream of tartar
\item Fondant flavored with vanilla
\end{itemize}
\end{multicols}}
\end{minipage}

\vspace{0.3em}
\noindent%
Boil ingredients (except fondant) until, when tried in cold water, a
firm ball may be formed in the fingers, not stirring until the last few
minutes of cooking. Pour on a buttered platter, and when cool enough to
handle, pull until light-colored. Shape on a floured board, having strip
wide enough to enclose a roll of fondant one inch in diameter. Place
fondant on candy, bring edges of candy together, and press firmly over
fondant. With both hands pull candy into a long strip. Cut in small
pieces; each piece will consist of fondant encircled with molasses
candy. Care must be taken that candy is not cooked too long, as it
should be soft rather than brittle.



\needspace{15\baselineskip}
\section*{Vinegar Candy}


\begin{itemize}
\setlength{\itemsep}{0pt}
\setlength{\parsep}{0pt}
\item 2 cups sugar
\item 1/2 cup vinegar
\item 2 tablespoons butter
\end{itemize}

\vspace{-0.5em}
\noindent%
Put butter into kettle; when melted, add sugar and vinegar. Stir until
sugar is dissolved, afterwards occasionally. Boil until, when tried in
cold water, mixture will become brittle. Turn on a buttered platter to
cool. Pull, and cut same as Molasses Candy.



\needspace{15\baselineskip}
\section*{Ice Cream Candy}


\begin{itemize}
\setlength{\itemsep}{0pt}
\setlength{\parsep}{0pt}
\item 3 cups sugar
\item 1/4 teaspoon cream of tartar
\item 1/2 cup boiling water
\item 1/2 tablespoon vinegar
\end{itemize}

\vspace{-0.5em}
\noindent%
Boil ingredients together without stirring, until, when tried in cold
water, mixture will become brittle. Turn on a well buttered platter to
cool. As edges cool, fold towards centre. As soon as it can be handled,
pull until white and glossy. While pulling, flavor as desired, using
vanilla, orange extract, coffee extract, oil of sassafras, or melted
chocolate. Cut in sticks or small pieces.



\needspace{15\baselineskip}
\section*{Butter Scotch}


\begin{itemize}
\setlength{\itemsep}{0pt}
\setlength{\parsep}{0pt}
\item 1 cup sugar
\item 1/4 cup molasses
\item 1 tablespoon vinegar
\item 2 tablespoons boiling water
\item 1/2 cup butter
\end{itemize}

\vspace{-0.5em}
\noindent%
Boil ingredients together until, when tried in cold water, mixture will
become brittle. Turn into a well buttered pan; when slightly cool, mark
with a sharp-pointed knife in squares. This candy is much improved by
cooking a small piece of vanilla bean with other ingredients.



\needspace{15\baselineskip}
\section*{Butter Taffy}


\begin{minipage}{1.0\textwidth}
{\setlength{\multicolsep}{0pt}\setlength{\columnsep}{2em}\raggedcolumns%
\begin{multicols}{2}
\begin{itemize}
\setlength{\itemsep}{0pt}
\setlength{\parsep}{0pt}
\item 2 cups light brown sugar
\item 1/4 cup molasses
\item 2 tablespoons vinegar
\item 2 tablespoons water
\item 7/8 teaspoon salt
\item 1/4 cup butter
\item 2 teaspoons vanilla
\end{itemize}
\end{multicols}}
\end{minipage}

\vspace{0.3em}
\noindent%
Boil first five ingredients until, when tried in cold water mixture will
become brittle. When nearly done, add butter, and just before turning
into pan, vanilla. Cool, and mark in squares.



\needspace{15\baselineskip}
\section*{Horehound Candy}


\begin{itemize}
\setlength{\itemsep}{0pt}
\setlength{\parsep}{0pt}
\item 3/4 square inch pressed horehound
\item 2 cups boiling water
\item 3 cups sugar
\item 1/2 teaspoon cream of tartar
\end{itemize}

\vspace{-0.5em}
\noindent%
Pour boiling water over horehound which has been separated in pieces;
let stand one minute, then strain through double cheese-cloth. Put into
a granite kettle with remaining ingredients, and boil until, when tried
in cold water, mixture will become brittle. Turn into a buttered pan,
cool slightly, then mark in small squares. Small square packages of
horehound may be bought for five cents.



\needspace{15\baselineskip}
\section*{Chocolate Caramels}


\begin{minipage}{1.0\textwidth}
{\setlength{\multicolsep}{0pt}\setlength{\columnsep}{2em}\raggedcolumns%
\begin{multicols}{2}
\begin{itemize}
\setlength{\itemsep}{0pt}
\setlength{\parsep}{0pt}
\item 2 1/2 tablespoons butter
\item 2 cups molasses
\item 1 cup brown sugar
\item 1/2 cup milk
\item 3 squares chocolate
\item 1 teaspoon vanilla
\end{itemize}
\end{multicols}}
\end{minipage}

\vspace{0.3em}
\noindent%
Put butter into kettle; when melted, add molasses, sugar, and milk. Stir
until sugar is dissolved, and when boiling-point is reached, add
chocolate, stirring constantly until chocolate is melted. Boil until,
when tried in cold water, a firm ball may be formed in the fingers. Add
vanilla just after taking from fire. Turn into a buttered pan, cool, and
mark in small squares.



\needspace{15\baselineskip}
\section*{Nut Chocolate Caramels}

To Chocolate Caramels add the meat from one pound English walnuts broken
in pieces, or one-half pound almonds blanched and chopped.



\needspace{15\baselineskip}
\section*{Rich Chocolate Caramels}


\begin{minipage}{1.0\textwidth}
{\setlength{\multicolsep}{0pt}\setlength{\columnsep}{2em}\raggedcolumns%
\begin{multicols}{2}
\begin{itemize}
\setlength{\itemsep}{0pt}
\setlength{\parsep}{0pt}
\item 2 tablespoons butter
\item 1/2 cup milk
\item 1/2 cup sugar
\item 1 cup molasses
\item 4 squares chocolate
\item 1 cup walnut meats, broken in pieces
\item 2 teaspoons vanilla
\end{itemize}
\end{multicols}}
\end{minipage}

\vspace{0.3em}
\noindent%
Put butter in saucepan and when melted add milk, sugar and molasses.
When boiling-point is reached add chocolate, and cook until brittle when
tried in cold water, stirring occasionally to prevent mixture from
adhering to pan. Remove from fire, beat three minutes, add nut meats and
vanilla, and turn into a buttered pan. When cold cut in squares and wrap
in paraffine paper.



\needspace{15\baselineskip}
\section*{Peanut Nougat}


\begin{itemize}
\setlength{\itemsep}{0pt}
\setlength{\parsep}{0pt}
\item 1 lb. sugar
\item 1 quart peanuts
\end{itemize}

\vspace{-0.5em}
\noindent%
Shell, remove skins, and finely chop peanuts. Sprinkle with one-fourth
teaspoon salt. Put sugar in a perfectly smooth granite saucepan, place
on range, and stir constantly until melted to a syrup, taking care to
keep sugar from sides of pan. Add nut meat, pour at once into a warm
buttered tin, and mark in small squares. If sugar is not removed from
range as soon as melted, it will quickly caramelize.



\needspace{15\baselineskip}
\section*{Nut Bar}

Cover the bottom of a buttered shallow pan with one and one-third cups
nut meat (castaneas, English walnuts, or almonds) cut in quarters. Pour
over one pound sugar, melted as for Peanut Nougat. Mark in bars.



\needspace{15\baselineskip}
\section*{French Nougat}


\begin{itemize}
\setlength{\itemsep}{0pt}
\setlength{\parsep}{0pt}
\item 1/2 lb. confectioners' sugar
\item 1/4 lb. almonds, blanched and finely chopped
\item Confectioners' chocolate
\end{itemize}

\vspace{-0.5em}
\noindent%
Put sugar in a saucepan, place on range, and stir constantly until
melted; add almonds, and pour on an oiled marble. Fold mixture as it
spreads with a broad-bladed knife, keeping it constantly in motion.
Divide in four parts, and as soon as cool enough to handle shape in long
rolls about one-third inch in diameter, keeping rolls in motion until
almost cold. When cold, snap in pieces one and one-half inches long.
This is done by holding roll at point to be snapped over the sharp edge
of a broad-bladed knife and snapping. Melt confectioners' chocolate over
hot water, beat with a fork until light and smooth, and when slightly
cooled dip pieces in chocolate and with a two-tined fork or bonbon
dipper remove from chocolate to oiled paper, drawing dipper through top
of each the entire length, thus leaving a ridge. Chocolate best adapted
for dipping bonbons and confections must be bought where confectioners'
supplies are kept.



\needspace{15\baselineskip}
\section*{Nougatine Drops}

Drop French Nougat mixture from the tip of a spoon on an oiled marble
very soon after taking from fire. These drops have a rough surface. When
cold, dip in melted confectioners' chocolate.



\needspace{15\baselineskip}
\section*{Wintergreen Wafers}

                          1 oz. gum tragacanth

\begin{itemize}
\setlength{\itemsep}{0pt}
\setlength{\parsep}{0pt}
\item 1 cup cold water
\item Confectioners' sugar
\item Oil of wintergreen
\end{itemize}

\vspace{-0.5em}
\noindent%
Soak gum tragacanth in water twenty-four hours and rub through a fine
wire sieve; add enough confectioners' sugar to knead. Flavor with a few
drops of oil of wintergreen. If liked pink, color with fruit red. Roll
until very thin on a board or marble dredged with sugar. Shape with a
small round cutter or cut in three-fourths inch squares. Spread wafers,
cover, and let stand until dry and brittle. This mixture may be flavored
with oil of lemon, clove, sassafras, etc., and colored as desired.



\needspace{15\baselineskip}
\section*{Cocoanut Cream Candy}


\begin{itemize}
\setlength{\itemsep}{0pt}
\setlength{\parsep}{0pt}
\item 1 1/2 cups sugar
\item 1/2 cup milk
\item 2 teaspoons butter
\item 1/3 cup shredded cocoanut
\item 1/2 teaspoon vanilla
\end{itemize}

\vspace{-0.5em}
\noindent%
Put butter into granite saucepan; when melted, add sugar and milk, and
stir until sugar is dissolved. Heat to boiling-point, and boil twelve
minutes; remove from fire, add cocoanut and vanilla, and beat until
creamy and mixture begins to sugar slightly around edge of saucepan.
Pour at once into a buttered pan, cool slightly, and mark in squares.
One-half cup nut meat, broken in pieces, may be used in place of
cocoanut.



\needspace{15\baselineskip}
\section*{Chocolate Cream Candy}


\begin{itemize}
\setlength{\itemsep}{0pt}
\setlength{\parsep}{0pt}
\item 2 cups sugar
\item 2/3 cup milk
\item 1 tablespoon butter
\item 2 squares chocolate
\item 1 teaspoon vanilla
\end{itemize}

\vspace{-0.5em}
\noindent%
Put butter into granite saucepan; when melted, add sugar and milk. Heat
to boiling-point; then add chocolate, and stir constantly until
chocolate is melted. Boil thirteen minutes, remove from fire, add
vanilla, and beat until creamy and mixture begins to sugar slightly
around edge of saucepan. Pour at once into a buttered pan, cool
slightly, and mark in squares. Omit vanilla, if desired, and add, while
cooking, one-fourth teaspoon cinnamon.



\needspace{15\baselineskip}
\section*{Maple Sugar Candy}


\begin{itemize}
\setlength{\itemsep}{0pt}
\setlength{\parsep}{0pt}
\item 1 lb. soft maple sugar
\item 3/4 cup thin cream
\item 1/4 cup boiling water
\item 2/3 cup English walnut or pecan meat, cut in pieces
\end{itemize}

\vspace{-0.5em}
\noindent%
Break sugar in pieces; put into a saucepan with cream and water. Bring
to boiling-point, and boil until a soft ball is formed when tried in
cold water. Remove from fire, beat until creamy, add nut meat, and pour
into a buttered tin. Cool slightly, and mark in squares.



\needspace{15\baselineskip}
\section*{Sultana Caramels}


\begin{minipage}{1.0\textwidth}
{\setlength{\multicolsep}{0pt}\setlength{\columnsep}{2em}\raggedcolumns%
\begin{multicols}{2}
\begin{itemize}
\setlength{\itemsep}{0pt}
\setlength{\parsep}{0pt}
\item 2 cups sugar
\item 1/2 cup milk
\item 1/4 cup molasses
\item 1/4 cup butter
\item 2 squares chocolate
\item 1 teaspoon vanilla
\item 1/2 cup English walnut or hickory nut meat, cut in pieces
\item 2 tablespoons Sultana raisins
\end{itemize}
\end{multicols}}
\end{minipage}

\vspace{0.3em}
\noindent%
Put butter into a saucepan; when melted, add sugar, milk, and molasses.
Heat to boiling-point, and boil seven minutes. Add chocolate, and stir
until chocolate is melted; then boil seven minutes longer. Remove from
fire, beat until creamy, add nuts, raisins, and vanilla, and pour at
once into a buttered tin. Cool slightly, and mark in squares. The nut
meats and raisins may be omitted.



\needspace{15\baselineskip}
\section*{Pralines}


\begin{itemize}
\setlength{\itemsep}{0pt}
\setlength{\parsep}{0pt}
\item 1 7/8 cups powdered sugar
\item 1 cup maple syrup
\item 1/2 cup cream
\item 2 cups hickory nut or pecan meat, cut in pieces
\end{itemize}

\vspace{-0.5em}
\noindent%
Boil first three ingredients until, when tried in cold water, a soft
ball may be formed. Remove from fire, and beat until of a creamy
consistency; add nuts, and drop from tip of spoon in small piles on
buttered paper, or mixture may be poured into a buttered tin and cut in
squares, using a sharp knife.



\needspace{15\baselineskip}
\section*{Creamed Walnuts}


\begin{itemize}
\setlength{\itemsep}{0pt}
\setlength{\parsep}{0pt}
\item 1 egg white
\item 1/2 tablespoon cold water
\item 3/4 teaspoon vanilla
\item 1 lb. confectioners' sugar
\item English walnuts
\end{itemize}

\vspace{-0.5em}
\noindent%
Put egg, water, and vanilla in a bowl, and beat until well blended. Add
sugar gradually until stiff enough to knead. Shape in balls, flatten,
and place halves of walnuts opposite each other on each piece. Sometimes
all the sugar will not be required.



\needspace{15\baselineskip}
\section*{Peppermints}


\begin{itemize}
\setlength{\itemsep}{0pt}
\setlength{\parsep}{0pt}
\item 1 1/2 cups sugar
\item 1/2 cup boiling water
\item 6 drops oil peppermint
\end{itemize}

\vspace{-0.5em}
\noindent%
Put sugar and water into a granite saucepan and stir until sugar is
dissolved. Boil ten minutes; remove from fire, add peppermint, and beat
until of right consistency. Drop from tip of spoon on slightly buttered
paper.



\needspace{15\baselineskip}
\section*{Boiled Sugar For Confections}

Eleven tests are considered for boiling sugar:


\begin{itemize}
\setlength{\itemsep}{0pt}
\item Small thread: 215deg F.
\item Large thread: 217deg
\item Pearl:        220deg
\item Large pearl:  222deg
\item The blow:     230deg
\item The feather:  232deg
\item Soft ball:    238deg
\item Hard ball:    248deg
\item Small crack:  290deg
\item Crack:        310deg
\item Caramel:      350deg
\end{itemize}


Fondant, the basis of all French candy, is made of sugar and water
boiled together (with a small quantity of cream of tartar to prevent
sugar from granulating) to soft ball, 238deg F. The professional
confectioner is able to decide when syrup has boiled to the right
temperature by sound while boiling, and by testing in cold water; these
tests at first seem somewhat difficult to the amateur, but only a little
experience is necessary to make fondant successfully. A sugar
thermometer is often employed, and proves valuable, as by its use one
need not exercise his judgment.



\needspace{15\baselineskip}
\subsection*{White Fondant}


\begin{itemize}
\setlength{\itemsep}{0pt}
\setlength{\parsep}{0pt}
\item 2 1/2 lbs. sugar
\item 1 1/2 cups hot water
\item 1/4 teaspoon cream of tartar
\end{itemize}

\vspace{-0.5em}
\noindent%
Put ingredients into a smooth granite stewpan. Stir, place on range, and
heat gradually to boiling-point. Boil without stirring until, when tried
in cold water, a soft ball may be formed that will just keep in shape,
which is 238deg F. After a few minutes' boiling, sugar will adhere to
sides of kettle; this should be washed off with the hand first dipped in
cold water. Have a pan of cold water near at hand, dip hand in cold
water, then quickly wash off a small part of the sugar with tips of
fingers, and repeat until all sugar adhering to side of saucepan is
removed. If this is quickly done, there is no danger of burning the
fingers. Pour slowly on a slightly oiled marble slab. Let stand a few
minutes to cool, but not long enough to become hard around the edge.
Scrape fondant with chopping knife to one end of marble, and work with a
wooden spatula until white and creamy. It will quickly change from this
consistency, and begin to lump, when it should be kneaded with the hands
until perfectly smooth.

Put into a bowl, cover with oiled paper to exclude air, that a crust may
not form on top, and let stand twenty-four hours. A large oiled platter
and wooden spoon may be used in place of marble slab and spatula. Always
make fondant on a clear day, as a damp, heavy atmosphere has an
unfavorable effect on the boiling of sugar.



\needspace{15\baselineskip}
\subsection*{Coffee Fondant}


\begin{itemize}
\setlength{\itemsep}{0pt}
\setlength{\parsep}{0pt}
\item 2 1/2 lbs. sugar
\item 1 1/2 cups cold water
\item 1/4 cup ground coffee
\item 1/4 teaspoon cream of tartar
\end{itemize}

\vspace{-0.5em}
\noindent%
Put water and coffee in saucepan, and heat to boiling-point. Strain
through double cheese-cloth; then add sugar and cream of tartar. Boil,
and work same as White Fondant.



\needspace{15\baselineskip}
\subsection*{Maple Fondant}


\begin{itemize}
\setlength{\itemsep}{0pt}
\setlength{\parsep}{0pt}
\item 1 1/4 lbs. maple sugar
\item 1 1/4 lbs. sugar
\item 1 cup hot water
\item 1/4 teaspoon cream of tartar
\end{itemize}

\vspace{-0.5em}
\noindent%
Break maple sugar in pieces and add to remaining ingredients. Boil, and
work same as White Fondant.



\needspace{15\baselineskip}
\subsection*{Bonbons}

The centres of bonbons are made of fondant shaped in small balls. If
White Fondant is used, flavor as desired,--vanilla being usually
preferred. For cocoanut centres, work as much shredded cocoanut as
possible into a small quantity of fondant; for nut centres, surround
pieces of nut meat with fondant, using just enough to cover. French
candied cherries are often used in this way. Allow balls to stand over
night, and dip the following day.

\textbf{To Dip Bonbons.} Put fondant in saucepan, and melt over hot water;
color and flavor as desired. In coloring fondant, dip a small wooden
skewer in coloring paste, take up a small quantity, and dip skewer in
fondant. If care is not taken, the color is apt to be too intense.
During dipping, keep fondant over hot water that it may be kept of right
consistency. For dipping, use a two-tined fork or confectioners' bonbon
dipper. Drop centres in fondant one at a time, stir until covered,
remove from fondant, put on oiled paper, and bring end of dipper over
the top of bonbon, thus leaving a tail-piece which shows that bonbons
have been hand dipped. Stir fondant between dippings to prevent a crust
from forming.



\needspace{15\baselineskip}
\subsection*{Cream Mints}

Melt fondant over hot water, flavor with a few drops of oil of
peppermint, wintergreen, clove, cinnamon, or orange, and color if
desired. Drop from tip of spoon on oiled paper. Confectioners use rubber
moulds for shaping cream mints; but these are expensive for home use,
unless one is to make mints in large quantities.



\needspace{15\baselineskip}
\subsection*{Cream Nut Bars}

Melt fondant and flavor, stir in any kind of nut meat, cut in pieces.
Turn in an oiled pan, cool, and cut in bars with a sharp knife. Maple
Fondant is delicious with nuts.



\needspace{15\baselineskip}
\subsection*{Dipped Walnuts}

Melt fondant and flavor. Dip halves of walnuts as bonbon centres are
dipped. Halves of pecan or whole blanched almonds may be similarly
dipped.



\needspace{15\baselineskip}
\subsection*{Tutti-Frutti Candy}

Fill an oiled border-mould with three layers of melted fondant. Have
bottom layer maple, well mixed with English walnut meat; the second
layer colored pink, flavored with rose, and mixed with candied cherries
cut in quarters and figs finely chopped; the third layer white, flavored
with vanilla, mixed with nuts, candied cherries cut in quarters, and
candied pineapple cut in small pieces. Cover mould with oiled paper, and
let stand over night. Remove from mould, and place on a plate covered
with a lace paper napkin. Fill centre with Bonbons and Glacé Nuts.



\needspace{15\baselineskip}
\subsection*{Glacé Nuts}


\begin{itemize}
\setlength{\itemsep}{0pt}
\setlength{\parsep}{0pt}
\item 2 cups sugar
\item 1 cup boiling water
\item 1/8 teaspoon cream of tartar
\end{itemize}

\vspace{-0.5em}
\noindent%
Put ingredients in a smooth saucepan, stir, place on range, and heat to
boiling-point. Boil without stirring until syrup begins to discolor,
which is 310deg F. Wash off sugar which adheres to sides of saucepan, as
in making fondant. Remove saucepan from fire, and place in larger pan of
cold water to instantly stop boiling. Remove from cold water and place
in a saucepan of hot water during dipping. Take nuts separately on a
long pin, dip in syrup to cover, remove from syrup, and place on oiled
paper.



\needspace{15\baselineskip}
\subsection*{Glacé Fruits}

For Glacé Fruits, grapes, strawberries, sections of mandarins and
oranges, and candied cherries are most commonly used. Take grapes
separately from clusters, leaving a short stem on each grape. Dip in
syrup made as for Glacé Nuts, holding by stem with pincers. Remove to
oiled paper. Glacé fruits keep but a day, and should only be attempted
in cold and clear weather.



\needspace{15\baselineskip}
\subsection*{Candied Orange Peel}

Remove peel from four thin-skinned oranges in quarters. Cover with cold
water, bring to boiling-point, and cook slowly until soft. Drain, remove
white portion, using a spoon, and cut yellow portion in thin strips,
using scissors. Boil one-half cup water and one cup sugar until syrup
will thread when dropped from tip of spoon. Cook strips in syrup five
minutes, drain, and coat with fine granulated sugar.



\needspace{15\baselineskip}
\subsection*{Spun Sugar}


\begin{itemize}
\setlength{\itemsep}{0pt}
\setlength{\parsep}{0pt}
\item 2 lbs. sugar
\item 2 cups boiling water
\item 1/4 teaspoon cream of tartar
\end{itemize}

\vspace{-0.5em}
\noindent%
Put ingredients in a smooth saucepan. Boil without stirring until syrup
begins to discolor, which is 300deg F. Wash off sugar which adheres to
sides of saucepan, as in making fondant. Remove saucepan from fire, and
place in a larger pan of cold water to instantly stop boiling. Remove
from cold water, and place in saucepan of hot water. Place two
broomstick-handles over backs of chairs, and spread paper on the floor
under them. When syrup is slightly cooled, put dipper in syrup, remove
from syrup, and shake quickly back and forth over broom-handles.
Carefully take off spun sugar as soon as formed, and shape in nests, or
pile lightly on a cold dish. Syrup may be colored if desired. Spun Sugar
is served around bricks or moulds of frozen creams and ices.

Dippers for spinning sugar are made of coarse wires; about twenty wires,
ten inches long, are put in a bundle, and fastened with wire coiled
round and round to form a handle.









\chapter{Sandwiches And Canapés}



In preparing bread for sandwiches, cut slices as thinly as possible, and
remove crusts. If butter is used, cream the butter, and spread bread
before cutting from loaf. Spread half the slices with mixture to be used
for filling, cover with remaining pieces, and cut in squares, oblongs,
or triangles. If sandwiches are shaped with round or fancy cutters,
bread should be shaped before spreading, that there may be no waste of
butter. Sandwiches which are prepared several hours before serving time
may be kept fresh and moist by wrapping in a napkin wrung as dry as
possible out of hot water, and keeping in a cool place. Paraffine paper
is often used for the same purpose. Bread for sandwiches cuts better
when a day old. Serve sandwiches piled on a plate covered with a doily.



\needspace{15\baselineskip}
\section*{Rolled Bread}

Cut fresh bread, while still warm, in as thin slices as possible, using
a very sharp knife. Spread evenly with butter which has been creamed.
Roll slices separately, and tie each with baby ribbon.



\needspace{15\baselineskip}
\section*{Bread And Butter Folds}

Remove end slice from bread. Spread end of loaf sparingly and evenly
with butter which has been creamed. Cut off as thin a slice as possible.
Repeat until the number of slices required are prepared. Remove crusts,
put together in pairs, and cut in squares, oblongs, or triangles. Use
white, entire wheat, Graham, or brown bread. Three layer sandwiches are
attractive when made of entire wheat bread between white slices.



\needspace{15\baselineskip}
\section*{Lettuce Sandwiches}

Put fresh, crisp lettuce leaves, washed and thoroughly dried, between
thin slices of buttered bread prepared as for Bread and Butter Folds,
having a teaspoon of Mayonnaise on each leaf.



\needspace{15\baselineskip}
\section*{Egg Sandwiches}

Chop finely the whites of “hard-boiled” eggs; force the yolks through a
strainer or potato ricer. Mix yolks and whites, season with salt and
pepper, and moisten with Mayonnaise or Cream Salad Dressing. Spread
mixture between thin slices of buttered bread prepared as for Bread and
Butter Folds.



\needspace{15\baselineskip}
\section*{Sardine Sandwiches}

Remove skin and bones from sardines, and mash to a paste. Add to an
equal quantity of yolks of “hard-boiled” eggs rubbed through a sieve.
Season with salt, cayenne, and a few drops of lemon juice; moisten with
olive oil or melted butter. Spread mixture between thin slices of
buttered bread prepared as for Bread and Butter Folds.



\needspace{15\baselineskip}
\section*{Sliced Ham Sandwiches}

Slice cold boiled ham as thinly as possible. Put between thin slices of
buttered bread prepared as for Bread and Butter Folds.



\needspace{15\baselineskip}
\section*{Chopped Ham Sandwiches}

Finely chop cold boiled ham, and moisten with Sauce Tartare. Spread
between thin slices of buttered bread prepared as for Bread and Butter
Folds.



\needspace{15\baselineskip}
\section*{Anchovy Sandwiches}

Rub the yolks of “hard-boiled eggs” to a paste. Moisten with soft butter
and season with Anchovy essence. Spread mixture between thin slices of
buttered bread prepared as for Bread and Butter Folds.



\needspace{15\baselineskip}
\section*{Chicken Sandwiches}

Chop cold boiled chicken, and moisten with Mayonnaise or Cream Salad
Dressing; or season with salt and pepper, and moisten with rich chicken
stock. Prepare as other sandwiches.



\needspace{15\baselineskip}
\section*{Lobster Sandwiches}

Remove lobster meat from shell, and chop. Season with salt, cayenne,
made mustard, and lemon juice; or moisten with any salad dressing.
Spread mixture on a crisp lettuce leaf, and prepare as other sandwiches.



\needspace{15\baselineskip}
\section*{Lobster Sandwiches À La Boulevard}

Mix an equal quantity of finely chopped lobster meat and the yolks of
“hard-boiled” eggs forced through a sieve. Moisten with melted butter,
and season with German mustard, beef extract diluted with a very small
quantity of boiling water, and salt. Spread mixture between thin slices
of buttered bread, remove crusts, and cut into fancy shapes. A small
quantity of lobster meat is most successfully utilized in this way.



\needspace{15\baselineskip}
\section*{Oyster Sandwiches}

Arrange fried oysters on crisp lettuce leaves, allowing two oysters for
each leaf, and one leaf for each sandwich. Prepare as other sandwiches.



\needspace{15\baselineskip}
\section*{Nut And Cheese Sandwiches}

Mix equal parts of grated Gruyère cheese and chopped English walnut
meat; then season with salt and cayenne. Prepare as other sandwiches.



\needspace{15\baselineskip}
\section*{Cheese And Anchovy Sandwiches}

Cream two tablespoons butter, and add one-fourth cup grated Young
America Cheese and one teaspoon vinegar. Season with salt, paprika,
mustard, and anchovy essence. Spread mixture between thin slices of
bread.



\needspace{15\baselineskip}
\section*{Windsor Sandwiches}

Cream one-third cup butter, and add one-half cup each of finely chopped
cold boiled ham and cold boiled chicken. Season with salt and paprika.
Spread mixture between thin slices of bread.



\needspace{15\baselineskip}
\section*{Club Sandwiches}

Arrange on slices of bread thin slices of cooked bacon; cover with
slices of cold roast chicken, and cover chicken with Mayonnaise
Dressing. Cover with slices of bread.



\needspace{15\baselineskip}
\section*{Ginger Sandwiches}

Cut preserved Canton ginger in very thin slices. Prepare as other
sandwiches.



\needspace{15\baselineskip}
\section*{Fruit Sandwiches}

Remove stems and finely chop figs; add a small quantity of water, cook
in double boiler until a paste is formed, then add a few drops of lemon
juice. Cool mixture, and spread on thin slices of buttered bread;
sprinkle with finely chopped peanuts and cover with pieces of buttered
bread.



\needspace{15\baselineskip}
\section*{Brown Bread Sandwiches}

Brown Bread to be used for sandwiches is best steamed in one-pound
baking-powder boxes. Spread and cut bread as for other sandwiches. Put
between layers finely chopped peanuts seasoned with salt; or grated
cheese mixed with chopped English walnut meat seasoned with salt.



\needspace{15\baselineskip}
\section*{Noisette Sandwiches}

Use one-half recipe for Milk and Water Bread made with entire wheat
flour (see p. 54), and add two tablespoons molasses and one cup English
walnut meats or pecan nut broken in small pieces. Let stand twenty-four
hours, slice as thinly as possible, spread sparingly and evenly with
butter, and put between slices orange marmalade. Remove crusts, cut in
fancy shapes, and garnish with nut meats.



\needspace{15\baselineskip}
\section*{Colonial Sandwiches}

Make one-half the recipe for Milk and Water Bread (see p. 54), using
entire wheat flour, and adding one and one-half tablespoons molasses,
and after the first rising adding, while kneading, one-half cup, each,
candied orange peel finely cut and pecan nut meats broken in pieces. Put
into buttered one-pound baking-powder tins until one-third full; let
rise and bake. Cool, and make into sandwiches.



\needspace{15\baselineskip}
\section*{German Sandwiches}

Use Zweiback (see p. 61). Spread slices, thinly cut, with jelly or
marmalade, and sprinkle with finely cut English walnut meats. Cover with
thinly cut slices and remove crusts.



\needspace{15\baselineskip}
\section*{Russian Sandwiches}

Spread zephyrettes with thin slices of Neufchâtel cheese, cover with
finely chopped olives moistened with Mayonnaise Dressing. Place a
zephyrette over each and press together.



\needspace{15\baselineskip}
\section*{Jelly Sandwiches}

Spread zephyrettes with quince jelly and sprinkle with chopped English
walnut meat. Place a zephyrette over each and press together.



\needspace{15\baselineskip}
\section*{Cheese Wafers}

Sprinkle zephyrettes with grated cheese mixed with a few grains of
cayenne. Put on a sheet and bake until the cheese melts.



\needspace{15\baselineskip}
\section*{Canapés}

Canapés are made by cutting bread in slices one-fourth inch thick, and
cutting slices in strips four inches long by one and one-half inches
wide, or in circular pieces. Then bread is toasted, fried in deep fat,
or buttered and browned in the oven, and covered with a seasoned mixture
of eggs, cheese, fish, or meat, separately or in combination. Canapés
are served hot or cold, and used in place of oysters at a dinner or
luncheon. At a gentleman's dinner they are served with a glass of Sherry
before entering the dining-room.



\needspace{15\baselineskip}
\section*{Cheese Canapés I}

Toast circular pieces of bread, sprinkle with a thick layer of grated
cheese seasoned with salt and cayenne. Place on a tin sheet and bake
until cheese is melted. Serve at once.



\needspace{15\baselineskip}
\section*{Cheese Canapés II}

Spread circular pieces of toasted bread with French Mustard, then
proceed as for Cheese Canapés I.



\needspace{15\baselineskip}
\section*{Sardine Canapés}

Spread circular pieces of toasted bread with sardines (from which bones
have been removed) rubbed to a paste, with a small quantity of creamed
butter and seasoned with Worcestershire Sauce and a few grains cayenne.
Place in the centre of each a stuffed olive, made by removing stone and
filling cavity with sardine mixture. Around each arrange a border of the
finely chopped whites of “hard-boiled” eggs.



\needspace{15\baselineskip}
\section*{Lobster Canapés}

Finely chop lobster meat and add an equal quantity of yolks of
“hard-boiled” eggs forced through a sieve. Moisten with melted butter
and heavy cream, using equal parts, and season highly with salt,
cayenne, German mustard and beef extract. Spread on sautéd circular
slices of bread and garnish with rings cut from whites of “hard-boiled”
eggs, yolks of “hard-boiled” eggs, and lobster coral forced through a
sieve.



\needspace{15\baselineskip}
\section*{Canapés Martha}

Beat yolk one egg, add one and one-half tablespoons cream, one-fourth
teaspoon salt, one-eighth teaspoon paprika, one-fourth teaspoon
Worcestershire Sauce, and a few grains cayenne; then add one-fourth
pound cheese cut in small pieces, and cook until smooth, stirring
constantly. Spread on sautéd slices of bread, cut in fancy shapes, and
cover with finely chopped lobster meat held together with a thick sauce
made of Chicken Stock or cream, garnish with rings of whites of
“hard-boiled” eggs, yolks of “hard-boiled” eggs, and lobster coral
forced through a strainer, and rings of olives.



\needspace{15\baselineskip}
\section*{Anchovy Canapés}

Spread circular pieces of toasted bread with Anchovy Butter. Chop
separately yolks and whites of “hard-boiled” eggs. Cover canapés by
quarters with egg, alternating yolks and whites. Divide yolks from
whites with anchovies split in two lengthwise, and pipe around a border
of Anchovy Butter, using a pastry bag and tube.







\needspace{15\baselineskip}
\section*{Cheese And Olive Canapés}

Cut stale bread in one-fourth inch slices. Shape with a small oblong
cutter with rounded corners. Cream butter, add an equal quantity of soft
cheese, and work until smooth; then season with salt. Spread on bread
and garnish with a one-fourth inch border of finely chopped olives and a
piece of red or green pepper cut in fancy shape, in centre of each. To
be served in place of sandwiches on a plate covered with a doily.



\needspace{15\baselineskip}
\section*{Canapés Lorenzo}

Toast slices of bread cut in shape of horseshoes. Cream two tablespoons
butter, and add one teaspoon white of egg. Spread slices of bread,
rounding with Crab Mixture, cover with creamed butter, sprinkle with
cheese, and brown in the oven. Serve on a napkin, ends towards centre of
dish, and garnish with parsley.

\textbf{Crab Mixture.} Finely chop crab meat, season with salt, cayenne, and a
few drops of lemon juice, then moisten with Thick White Sauce. Lobster
meat may be used in place of crab meat.



\needspace{15\baselineskip}
\section*{Algonquin Canapés}

Fry one-half tablespoon finely chopped onion, three tablespoons butter,
and one-third cup chopped mushroom caps five minutes. Add two
tablespoons flour, and two-thirds cup cream. Cook until mixture
thickens, then add one cup finnan haddie (soaked in lukewarm water to
cover forty-five minutes, then separated into flakes), two tablespoons
grated cheese, and yolks two eggs slightly beaten. Season with salt and
cayenne and pile on circular pieces of toasted bread. Sprinkle with
grated cheese, then with buttered, soft bread crumbs, and bake until
crumbs are browned. Serve at once.





\chapter{Recipes For The Chafing-Dish}



The chafing-dish, which, within the last few years, has gained so much
favor, is by no means a utensil of modern invention, as its history may
be traced to the time of Louis XIV. It finds its place on the breakfast
table, when the eggs may be cooked to suit the most fastidious; on the
luncheon table, when a dainty hot dish may be prepared to serve in place
of the so-oft-seen cold meat; but it is made of greatest use for the
cooking of late suppers, and always seems to accompany hospitality and
good cheer.

It is appreciated and enjoyed by the housekeeper who does her own work,
or has but one maid, as well as by the society girl who, by its use,
first gains a taste for the art of cooking. The simple tin
chafing-dishes may be bought for as small a sum as ninety cents, while
the elaborate silver ones command as high a price as one hundred
dollars. Very attractive dishes are made of granite ware, nickel, or
copper. The latest patterns have the lamp with a screw adjustment to
regulate the flame, and a metal tray on which to set dish, that it may
be moved if necessary while hot, without danger of burnt fingers, and
that it may not injure the polished table.

A chafing-dish has two pans, the under one for holding hot water, the
upper one with long handle for holding food to be cooked. A blazer
differs from a chafing-dish, inasmuch as it has no hot-water pan.

Wood alcohol, which is much lower in price than high-proof spirits, is
generally used in chafing-dishes.

The Davy Toaster may be used over the chafing-dish for toasting bread
and broiling.

List of dishes previously given that may be prepared on the
Chafing-Dish:


\begin{itemize}
\setlength{\itemsep}{0pt}
\item German Toast
\item Dropped Eggs
\item Eggs à la Finnoise
\item Eggs à la Suisse
\item Scrambled Eggs
\item Scrambled Eggs with Tomato Sauce
\item Scrambled Eggs with Anchovy Toast
\item Buttered Eggs
\item Buttered Eggs with Tomatoes
\item Curried Eggs
\item French Omelet
\item Spanish Omelet
\item Creamed Fish
\item Halibut à la Rarebit
\item Creamed Oysters
\item Buttered Lobster
\item Creamed Lobster
\item Broiled Meat Cakes
\item Salmi of Lamb
\item Creamed Sweetbreads
\item Sautéd Sweetbreads
\item Chickens' Livers with Madeira Sauce
\item Chickens' Livers with Curry
\item Sautéd Chickens' Livers Creamed Chicken
\item Chicken and Oysters à la Métropole
\item Stewed Mushrooms
\item Sautéd Mushrooms
\item Mushrooms à la Sabine
\item Soufflé au Rhum
\end{itemize}




\needspace{15\baselineskip}
\section*{Scrambled Eggs With Sweetbreads}


\begin{minipage}{1.0\textwidth}
{\setlength{\multicolsep}{0pt}\setlength{\columnsep}{2em}\raggedcolumns%
\begin{multicols}{2}
\begin{itemize}
\setlength{\itemsep}{0pt}
\setlength{\parsep}{0pt}
\item 4 eggs
\item 1/2 teaspoon salt
\item 1/8 teaspoon pepper
\item 1/2 cup milk
\item 1 sweetbread, parboiled and cut in dice
\item 2 tablespoons butter
\end{itemize}
\end{multicols}}
\end{minipage}

\vspace{0.3em}
\noindent%
Beat eggs slightly, using a silver folk, add salt, pepper, milk, and
sweetbread. Put butter in hot chafing-dish; when melted, pour in the
mixture. Cook until of creamy consistency, constantly stirring and
scraping from bottom of the pan.



\needspace{15\baselineskip}
\section*{Scrambled Eggs With Calf's Brains}

Follow recipe for Scrambled Eggs with Sweetbreads, using calf's brains
in place of sweetbreads.

\textbf{To Prepare Calf's Brains.} Soak one hour in cold water to cover. Remove
membrane, and parboil twenty minutes in boiling, salted, acidulated
water. Drain, put in cold water; as soon as cold, drain again, and
separate in small pieces.



\needspace{15\baselineskip}
\section*{Cheese Omelet}


\begin{itemize}
\setlength{\itemsep}{0pt}
\setlength{\parsep}{0pt}
\item 2 eggs
\item 1 tablespoon melted butter
\item 1/8 tablespoon salt
\item Few grains cayenne
\item 1 tablespoon grated cheese
\end{itemize}

\vspace{-0.5em}
\noindent%
Beat eggs slightly, add one-half teaspoon melted butter, salt, cayenne,
and cheese. Melt remaining butter, add mixture, and cook until firm,
without stirring. Roll, and sprinkle with grated cheese. Serve with
Graham bread sandwiches.



\needspace{15\baselineskip}
\section*{Eggs Au Beurre Noir}


\begin{itemize}
\setlength{\itemsep}{0pt}
\setlength{\parsep}{0pt}
\item Butter
\item Salt
\item Pepper
\item 4 eggs
\item 1 teaspoon vinegar
\end{itemize}

\vspace{-0.5em}
\noindent%
Put one tablespoon butter in a hot chafing-dish; when melted, slip in
carefully four eggs, one at a time. Sprinkle with salt and pepper, and
cook until whites are firm. Remove to a hot platter, care being taken
not to break yolks. In same dish brown two tablespoons butter, add
vinegar, and pour over eggs.



\needspace{15\baselineskip}
\section*{Eggs À La Caracas}


\begin{minipage}{1.0\textwidth}
{\setlength{\multicolsep}{0pt}\setlength{\columnsep}{2em}\raggedcolumns%
\begin{multicols}{2}
\begin{itemize}
\setlength{\itemsep}{0pt}
\setlength{\parsep}{0pt}
\item 2 ozs. smoked dried beef
\item 1 cup tomatoes
\item 1/4 cup grated cheese
\item Few drops onion juice
\item Few grains cinnamon
\item Few grains cayenne
\item 2 tablespoons butter
\item 3 eggs
\end{itemize}
\end{multicols}}
\end{minipage}

\vspace{0.3em}
\noindent%
Pick over beef and chop finely, add tomatoes, cheese, onion juice,
cinnamon, and cayenne. Melt butter, add mixture, and when heated, add
eggs well beaten. Cook until eggs are of creamy consistency, stirring
and scraping from bottom of pan.



\needspace{15\baselineskip}
\section*{Union Grill}

Clean one pint of oysters and drain off all the liquor possible. Put
oysters in chafing-dish, and as liquor flows from oysters, remove with a
spoon, and so continue until oysters are plump. Sprinkle with salt and
pepper, and add two tablespoons butter. Serve on zephyrettes.



\needspace{15\baselineskip}
\section*{Oysters À La D'Uxelles}


\begin{minipage}{1.0\textwidth}
{\setlength{\multicolsep}{0pt}\setlength{\columnsep}{2em}\raggedcolumns%
\begin{multicols}{2}
\begin{itemize}
\setlength{\itemsep}{0pt}
\setlength{\parsep}{0pt}
\item 1 pint oysters
\item 2 tablespoons chopped mushrooms
\item 2 tablespoons butter
\item 2 tablespoons flour
\item 1/2 teaspoon salt
\item 1/2 teaspoon lemon juice
\item Few grains cayenne
\item 1 egg yolk
\item 1 tablespoon Sherry wine
\end{itemize}
\end{multicols}}
\end{minipage}

\vspace{0.3em}
\noindent%
Clean oysters, heat to boiling-point, and drain. Reserve liquor and
strain through double thickness of cheese-cloth; there should be
three-fourths cup. Cook butter and mushrooms five minutes, add flour,
and oyster liquor gradually; then cook three minutes. Add seasonings,
oysters, egg, and Sherry wine. Serve on zephyrettes or pieces of toasted
bread.



\needspace{15\baselineskip}
\section*{Oysters À La Thorndike}


\begin{minipage}{1.0\textwidth}
{\setlength{\multicolsep}{0pt}\setlength{\columnsep}{2em}\raggedcolumns%
\begin{multicols}{2}
\begin{itemize}
\setlength{\itemsep}{0pt}
\setlength{\parsep}{0pt}
\item 1 pint oysters
\item 2 tablespoons butter
\item 1/2 teaspoon salt
\item Few grains cayenne
\item Slight grating nutmeg
\item 1/4 cup thin cream
\item 4 egg yolks
\end{itemize}
\end{multicols}}
\end{minipage}

\vspace{0.3em}
\noindent%
Clean and drain oysters. Melt butter, add oysters, and cook until
oysters are plump. Then add seasonings, cream, and egg yolks slightly
beaten. Cook until sauce is slightly thickened, stirring constantly.
Serve on zephyrettes or pieces of toasted bread.



\needspace{15\baselineskip}
\section*{Jack's Oyster Ragout}

Parboil fresh honeycomb tripe, and cut in three-fourths inch pieces;
there should be one cup. Add an equal quantity of small boiled onions,
and twice the quantity of raw oysters which have been previously
cleaned. Melt three tablespoons butter, add four tablespoons flour, and
pour on gradually while stirring constantly one and one-half cups thin
cream. Add tripe, onion, and oysters. When thoroughly heated add yolks
two eggs slightly beaten, and season highly with salt, pepper, and
paprika. Serve on pieces toasted bread.



\needspace{15\baselineskip}
\section*{Lobster À La Delmonico}


\begin{minipage}{1.0\textwidth}
{\setlength{\multicolsep}{0pt}\setlength{\columnsep}{2em}\raggedcolumns%
\begin{multicols}{2}
\begin{itemize}
\setlength{\itemsep}{0pt}
\setlength{\parsep}{0pt}
\item 2 lb. lobster
\item 1/4 cup butter
\item 3/4 tablespoons flour
\item 1/2 teaspoon salt
\item Few grains cayenne
\item Slight grating nutmeg
\item 1 cup thin cream
\item 4 egg yolks
\item 2 tablespoons Sherry wine
\end{itemize}
\end{multicols}}
\end{minipage}

\vspace{0.3em}
\noindent%
Remove lobster meat from shell and cut in small cubes. Melt butter, add
flour, seasonings, and cream gradually. Add lobster, and when heated,
add egg yolks and wine.



\needspace{15\baselineskip}
\section*{Lobster À La Newburg}


\begin{minipage}{1.0\textwidth}
{\setlength{\multicolsep}{0pt}\setlength{\columnsep}{2em}\raggedcolumns%
\begin{multicols}{2}
\begin{itemize}
\setlength{\itemsep}{0pt}
\setlength{\parsep}{0pt}
\item 2 lb. lobster
\item 1/4 cup butter
\item 1/2 teaspoon salt
\item Few grains cayenne
\item Slight grating nutmeg
\item 1 tablespoon Sherry
\item 1 tablespoon brandy
\item 1/3 cup thin cream
\item 4 egg yolks
\end{itemize}
\end{multicols}}
\end{minipage}

\vspace{0.3em}
\noindent%
Remove lobster meat from shell and cut in slices. Melt butter, add
lobster, and cook three minutes. Add seasonings and wine, cook one
minute, then add cream and yolks of eggs slightly beaten. Stir until
thickened. Serve with toast or Puff Paste Points.



\needspace{15\baselineskip}
\section*{Clams À La Newburg}


\begin{minipage}{1.0\textwidth}
{\setlength{\multicolsep}{0pt}\setlength{\columnsep}{2em}\raggedcolumns%
\begin{multicols}{2}
\begin{itemize}
\setlength{\itemsep}{0pt}
\setlength{\parsep}{0pt}
\item 1 pint clams
\item 3 tablespoons butter
\item 1/2 teaspoon salt
\item Few grains cayenne
\item 3 tablespoons Sherry or Madeira wine
\item 1/2 cup thin cream
\item 3 egg yolks
\end{itemize}
\end{multicols}}
\end{minipage}

\vspace{0.3em}
\noindent%
Clean clams, remove soft parts, and finely chop hard parts. Melt butter,
add chopped clams, seasonings, and wine. Cook eight minutes, add soft
part of clams, and cream. Cook two minutes, then add egg yolks slightly
beaten, diluted with some of the hot sauce.



\needspace{15\baselineskip}
\section*{Shrimps À La Newburg}


\begin{minipage}{1.0\textwidth}
{\setlength{\multicolsep}{0pt}\setlength{\columnsep}{2em}\raggedcolumns%
\begin{multicols}{2}
\begin{itemize}
\setlength{\itemsep}{0pt}
\setlength{\parsep}{0pt}
\item 1 pint shrimps
\item 3 tablespoons butter
\item 1/2 teaspoon salt
\item Few grains cayenne
\item 1 teaspoon lemon juice
\item 1 teaspoon flour
\item 1/2 cup cream
\item 4 egg yolks
\item 2 tablespoons Sherry wine
\end{itemize}
\end{multicols}}
\end{minipage}

\vspace{0.3em}
\noindent%
Clean shrimps and cook three minutes in two tablespoons butter. Add
salt, cayenne, and lemon juice, and cook one minute. Remove shrimps, and
put remaining butter in chafing-dish, add flour and cream; when
thickened, add yolks of eggs slightly beaten, shrimps, and wine. Serve
with toast or Puff Paste Points.



\needspace{15\baselineskip}
\section*{Fish À La Provençale}


\begin{minipage}{1.0\textwidth}
{\setlength{\multicolsep}{0pt}\setlength{\columnsep}{2em}\raggedcolumns%
\begin{multicols}{2}
\begin{itemize}
\setlength{\itemsep}{0pt}
\setlength{\parsep}{0pt}
\item 1/4 cup butter
\item 2 1/2 tablespoons flour
\item 2 cups milk
\item Yolks 4 “hard-boiled” eggs
\item 1 teaspoon Anchovy essence
\item 2 cups cold boiled flaked fish
\end{itemize}
\end{multicols}}
\end{minipage}

\vspace{0.3em}
\noindent%
Make a sauce of butter, flour, and milk. Mash yolks of eggs and mix with
Anchovy essence, add to sauce, then add fish. Serve as soon as heated.
Serve on pieces of toasted Graham bread.



\needspace{15\baselineskip}
\section*{Grilled Sardines}

Drain twelve sardines and cook in a chafing-dish until heated, turning
frequently. Place on small oblong pieces of dry toast, and serve with
Maître d'Hôtel or Lemon Butter.



\needspace{15\baselineskip}
\section*{Sardines With Anchovy Sauce}

Drain twelve sardines and cook in a chafing-dish until heated, turning
frequently. Remove from chafing-dish. Make one cup Brown Sauce with one
and one-half tablespoons sardine oil, two tablespoons flour, and one cup
Brown Stock. Season with Anchovy essence. Reheat sardines in sauce.
Serve with Brown Bread Sandwiches, having a slice of cucumber marinated
with French Dressing between slices of bread.



\needspace{15\baselineskip}
\section*{Creamed Sardines}

Drain from oil one small box sardines, remove backbones from fish, then
mash. Melt one-fourth cup butter, add one-fourth cup soft stale bread
crumbs, and one cup cream. When thoroughly heated add two “hard-boiled”
eggs finely chopped, the sardines, salt, pepper, and paprika to taste.
Serve on pieces of toasted bread.



\needspace{15\baselineskip}
\section*{Welsh Rarebit I}


\begin{minipage}{1.0\textwidth}
{\setlength{\multicolsep}{0pt}\setlength{\columnsep}{2em}\raggedcolumns%
\begin{multicols}{2}
\begin{itemize}
\setlength{\itemsep}{0pt}
\setlength{\parsep}{0pt}
\item 1 tablespoon butter
\item 1 teaspoon corn-starch
\item 1/2 cup thin cream
\item 1/2 lb. soft mild cheese cut in small pieces
\item 1/4 teaspoon salt
\item 1/4 teaspoon mustard
\item Few grains cayenne
\item Toast or zephyrettes
\end{itemize}
\end{multicols}}
\end{minipage}

\vspace{0.3em}
\noindent%
Melt butter, add corn-starch, and stir until well mixed, then add cream
gradually, while stirring constantly, and cook two minutes. Add cheese,
and stir until cheese is melted. Season, and serve on zephyrettes or
bread toasted on one side, rarebit being poured over untoasted side.
Much of the success of a rarebit depends upon the quality of the cheese.
A rarebit should be smooth and of a creamy consistency, never stringy.



\needspace{15\baselineskip}
\section*{Welsh Rarebit II}


\begin{minipage}{1.0\textwidth}
{\setlength{\multicolsep}{0pt}\setlength{\columnsep}{2em}\raggedcolumns%
\begin{multicols}{2}
\begin{itemize}
\setlength{\itemsep}{0pt}
\setlength{\parsep}{0pt}
\item 1 tablespoon butter
\item 1/2 lb. soft mild cheese, cut in small pieces
\item 1/4 teaspoon salt
\item 1/4 teaspoon mustard
\item Few grains cayenne
\item 1/3 to 1/4 cup ale or lager beer
\item 1 egg
\end{itemize}
\end{multicols}}
\end{minipage}

\vspace{0.3em}
\noindent%
Put butter in chafing-dish, and when melted, add cheese and seasonings;
as cheese melts, add ale gradually, while stirring constantly; then egg
slightly beaten. Serve same as Welsh Rarebit I.



\needspace{15\baselineskip}
\section*{Oyster Rarebit}


\begin{minipage}{1.0\textwidth}
{\setlength{\multicolsep}{0pt}\setlength{\columnsep}{2em}\raggedcolumns%
\begin{multicols}{2}
\begin{itemize}
\setlength{\itemsep}{0pt}
\setlength{\parsep}{0pt}
\item 1 cup oysters
\item 2 tablespoons butter
\item 1/2 lb. soft mild cheese, cut in small pieces
\item 1/4 teaspoon salt
\item Few grains cayenne
\item 2 eggs
\end{itemize}
\end{multicols}}
\end{minipage}

\vspace{0.3em}
\noindent%
Clean, parboil, and drain oysters, reserving liquor; then remove and
discard tough muscle. Melt butter, add cheese and seasonings; as cheese
melts, add gradually oyster liquor, and eggs slightly beaten. As soon as
mixture is smooth, add soft part of oysters. Serve on zephyrettes or
bread toasted on one side, rarebit being poured over untoasted side.



\needspace{15\baselineskip}
\section*{Tomato Rarebit}


\begin{minipage}{1.0\textwidth}
{\setlength{\multicolsep}{0pt}\setlength{\columnsep}{2em}\raggedcolumns%
\begin{multicols}{2}
\begin{itemize}
\setlength{\itemsep}{0pt}
\setlength{\parsep}{0pt}
\item 2 tablespoons butter
\item 2 tablespoons flour
\item 3/4 cup thin cream
\item 3/4 cup stewed and strained tomatoes
\item 1/8 teaspoon soda
\item 2 cups finely cut cheese
\item 2 eggs, slightly beaten
\item Salt
\item Mustard
\item Cayenne
\end{itemize}
\end{multicols}}
\end{minipage}

\vspace{0.3em}
\noindent%
Put butter in chafing-dish; when melted, add flour. Pour on, gradually,
cream, and as soon as mixture thickens add tomatoes mixed with soda;
then add cheese, eggs, and seasonings to taste. Serve, as soon as cheese
has melted, on Graham Toast.



\needspace{15\baselineskip}
\section*{English Monkey}


\begin{minipage}{1.0\textwidth}
{\setlength{\multicolsep}{0pt}\setlength{\columnsep}{2em}\raggedcolumns%
\begin{multicols}{2}
\begin{itemize}
\setlength{\itemsep}{0pt}
\setlength{\parsep}{0pt}
\item 1 cup stale bread crumbs
\item 1 cup milk 
\item 1 tablespoon butter 
\item 1/2 cup soft mild cheese, cut in small pieces 
\item 1 egg 
\item 1/2 teaspoon salt 
\item Few grains cayenne
\end{itemize}
\end{multicols}}
\end{minipage}

\vspace{0.3em}
\noindent%
Soak bread crumbs fifteen minutes in milk. Melt butter, add cheese, and
when cheese has melted, add soaked crumbs, egg slightly beaten, and
seasonings. Cook three minutes, and pour over toasted crackers which
have been spread sparingly with butter.



\needspace{15\baselineskip}
\section*{Breaded Tongue With Tomato Sauce}

Cut cold boiled corned tongue in slices one-third inch thick. Sprinkle
with salt and pepper, dip in egg and crumbs, and sauté in butter. Serve
with Tomato Sauce I.



\needspace{15\baselineskip}
\section*{Scotch Woodcock}


\begin{minipage}{1.0\textwidth}
{\setlength{\multicolsep}{0pt}\setlength{\columnsep}{2em}\raggedcolumns%
\begin{multicols}{2}
\begin{itemize}
\setlength{\itemsep}{0pt}
\setlength{\parsep}{0pt}
\item 4 “hard-boiled” eggs
\item 3 tablespoons butter
\item 1 1/2 tablespoons flour
\item 1 cup milk
\item 1/4 teaspoon salt
\item Few grains cayenne
\item Anchovy essence
\end{itemize}
\end{multicols}}
\end{minipage}

\vspace{0.3em}
\noindent%
Make a thin white sauce of butter, flour, milk, and seasonings; add eggs
finely chopped, and season with Anchovy essence. Serve same as Welsh
Rarebit I.



\needspace{15\baselineskip}
\section*{Shredded Ham With Currant Jelly Sauce}


\begin{itemize}
\setlength{\itemsep}{0pt}
\setlength{\parsep}{0pt}
\item 1/2 tablespoon butter
\item 1/3 cup currant jelly
\item Few grains cayenne
\item 1/4 cup Sherry wine
\item 1 cup cold cooked ham, cut in small strips
\end{itemize}

\vspace{-0.5em}
\noindent%
Put butter and currant jelly into the chafing-dish. As soon as melted,
add cayenne, wine, and ham; simmer five minutes.



\needspace{15\baselineskip}
\section*{Venison Cutlets With Apples}

Wipe, core, and cut four apples in one-fourth inch slices. Sprinkle with
powdered sugar, and add one-third cup Port wine; cover, and let stand
thirty minutes. Drain, and sauté in butter. Cut a slice of venison
one-half inch thick in cutlets. Sprinkle with salt and pepper, and cook
three or four minutes in a hot chafing-dish, using just enough butter to
prevent sticking. Remove from dish; then melt three tablespoons butter,
add wine drained from apples, and twelve candied cherries cut in halves.
Reheat cutlets in sauce, and serve with apples.



\needspace{15\baselineskip}
\section*{Mutton With Currant Jelly Sauce}


\begin{minipage}{1.0\textwidth}
{\setlength{\multicolsep}{0pt}\setlength{\columnsep}{2em}\raggedcolumns%
\begin{multicols}{2}
\begin{itemize}
\setlength{\itemsep}{0pt}
\setlength{\parsep}{0pt}
\item 2 tablespoons butter
\item 2 tablespoons flour
\item 1/4 teaspoon salt
\item Few grains pepper
\item 1 cup Brown Stock
\item 1/3 cup currant jelly
\item 1 1/2 tablespoons Sherry wine
\item 6 slices cold cooked mutton
\end{itemize}
\end{multicols}}
\end{minipage}

\vspace{0.3em}
\noindent%
Brown the butter, add flour, seasonings, and stock, gradually; then add
jelly, and when melted, add mutton. When meat is heated, add wine. If
mutton gravy is at hand, use instead of making a Brown Sauce.



\needspace{15\baselineskip}
\section*{Minced Mutton}


\begin{minipage}{1.0\textwidth}
{\setlength{\multicolsep}{0pt}\setlength{\columnsep}{2em}\raggedcolumns%
\begin{multicols}{2}
\begin{itemize}
\setlength{\itemsep}{0pt}
\setlength{\parsep}{0pt}
\item 2 cups chopped cooked mutton
\item Yolks 6 “hard-boiled” eggs
\item 3/4 teaspoon mixed mustard
\item Salt
\item Cayenne
\item 1 cup of cream
\item 1/4 cup wine
\end{itemize}
\end{multicols}}
\end{minipage}

\vspace{0.3em}
\noindent%
Mash the yolks, and season with mustard, salt, and cayenne. Add cream
and mutton. When thoroughly heated add wine. Serve on toast.



\needspace{15\baselineskip}
\section*{Devilled Bones}


\begin{minipage}{1.0\textwidth}
{\setlength{\multicolsep}{0pt}\setlength{\columnsep}{2em}\raggedcolumns%
\begin{multicols}{2}
\begin{itemize}
\setlength{\itemsep}{0pt}
\setlength{\parsep}{0pt}
\item 2 tablespoons butter
\item 1 tablespoon Chili Sauce
\item 1 tablespoon Worcestershire Sauce
\item 1 tablespoon Walnut Catsup
\item 1 teaspoon made mustard
\item Few grains cayenne
\item Drumsticks, second joints, and wings of a cooked chicken
\item Salt
\item Pepper
\item Flour
\item Cup hot stock
\item Finely chopped parsley
\end{itemize}
\end{multicols}}
\end{minipage}

\vspace{0.3em}
\noindent%
Melt butter, and add Chili Sauce, Worcestershire Sauce, Walnut Catsup,
mustard, and cayenne. Cut four small gashes in each piece of chicken.
Sprinkle with salt and pepper, dredge with flour, and cook in the
seasoned butter until well browned. Pour on stock, simmer five minutes,
and sprinkle with chopped parsley.



\needspace{15\baselineskip}
\section*{Devilled Almonds}


\begin{minipage}{1.0\textwidth}
{\setlength{\multicolsep}{0pt}\setlength{\columnsep}{2em}\raggedcolumns%
\begin{multicols}{2}
\begin{itemize}
\setlength{\itemsep}{0pt}
\setlength{\parsep}{0pt}
\item 2 ozs. blanched and shredded almonds
\item Butter
\item 1 tablespoon Chutney
\item 2 tablespoons chopped pickles
\item 1 tablespoon Worcestershire Sauce
\item 1/4 teaspoon salt
\item Few grains cayenne
\end{itemize}
\end{multicols}}
\end{minipage}

\vspace{0.3em}
\noindent%
Fry almonds until well browned, using enough butter to prevent almonds
from burning. Mix remaining ingredients, pour over nuts, and serve as
soon as thoroughly heated. Serve with oysters.



\needspace{15\baselineskip}
\section*{Devilled Chestnuts}

Shell one cup chestnuts, cut in thin slices, and fry until well browned,
using enough butter to prevent chestnuts from burning. Season with
Tabasco Sauce or few grains paprika.



\needspace{15\baselineskip}
\section*{Fruit Canapés}

Make German Toast in circular pieces, cover with stewed prunes, figs, or
jam. Serve with Cream Sauce I.



\needspace{15\baselineskip}
\section*{Peach Canapés}

Sauté circular pieces of sponge cake in butter until delicately browned.
Drain canned peaches, sprinkle with powdered sugar, few drops lemon
juice, and slight grating nutmeg. Melt one tablespoon butter, add
peaches, and when heated, serve on cake.



\needspace{15\baselineskip}
\section*{Fig Cups}


\begin{itemize}
\setlength{\itemsep}{0pt}
\setlength{\parsep}{0pt}
\item 1/2 lb. washed figs
\item Chopped salted almonds
\item 2 tablespoons sugar
\item 1 teaspoon lemon juice
\item 1/2 cup wine
\end{itemize}

\vspace{-0.5em}
\noindent%
Stuff figs with almonds. Put sugar, lemon juice, and wine in
chafing-dish; when heated, add figs, cover, and cook until figs are
tender, turning and basting often. Serve with Lady Fingers.





\chapter{Fruits: Fresh, Preserved, And Canned}



Fruits are usually at their best when served ripe and in season;
however, a few cannot be taken in their raw state, and still others are
rendered more easy of digestion by cooking. The methods employed are
stewing and baking. Fruit should be cooked in earthen or granite ware
utensils, and silver or wooden spoons should be employed for stirring.
It must be remembered that all fruits contain one or more acids, and
when exposed to air and brought in contact with an iron or tin surface,
a poisonous compound may be formed.



\needspace{15\baselineskip}
\section*{How To Prepare Strawberries For Serving}

1. Pick over strawberries, place in colander, pour over cold water,
drain thoroughly, hull, and turn into dish. Serve with powdered sugar
and cream.

2. Pick over selected strawberries, place in colander, pour over cold
water, and drain thoroughly. Press powdered sugar into cordial glasses.
Remove from glasses on centres of fruit plates. Arrange twelve berries
around each mound of sugar. Berries served in this way should not be
hulled.



\needspace{15\baselineskip}
\section*{How To Prepare Cantaloupes And Muskmelons For Serving}

Cantaloupes and muskmelons should be very ripe and thoroughly chilled in
ice box before being prepared for serving. Wipe melons,--if small, cut in
halves lengthwise; if larger, cut in sections, and remove seeds and
stringy portion. If one-half is served as a portion, put in cavity one
tablespoon crushed ice. Serve with salt or powdered sugar.



\needspace{15\baselineskip}
\section*{How To Prepare Grapes For Serving}

Put bunches in colander and pour over cold water, drain, chill, and
arrange on serving dish. Imperfect grapes, as well as those under-ripe
or over-ripe, should be removed. Garnish with grape leaves, if at hand.



\needspace{15\baselineskip}
\section*{Ways Of Preparing Oranges For Serving}

1. Wipe orange and cut in halves crosswise. Place one-half on a fruit
plate, having an orange spoon or teaspoon on plate at right of fruit.

2. Peel an orange and remove as much of the white portion as possible.
Remove pulp by sections, which may be accomplished by using a sharp
knife and cutting pulp from tough portion first on one side of section,
then on the other. Should there be any white portion of skin remaining
on pulp it should be cut off. Arrange sections on glass dish or fruit
plate. If the orange is a seeded one, remove seeds.

3. Remove peel from an orange in such a way that there remains a
one-half inch band of peel equal distance from stem and blossom end. Cut
band, separate sections, and arrange around a mould of sugar.



\needspace{15\baselineskip}
\section*{How To Prepare Grape Fruit For Serving}

Wipe grape fruit and cut in halves crosswise. With a small,
sharp-pointed knife make a cut separating pulp from skin around entire
circumference; then make cuts separating pulp from tough portion which
divides fruit into sections. Remove tough portion in one piece, which
may be accomplished by one cutting with scissors at stem or blossom end
close to skin. Sprinkle fruit pulp left in grape fruit skin generously
with sugar. Let stand ten minutes, and serve very cold. Place on fruit
plate and garnish with a candied cherry.



\needspace{15\baselineskip}
\section*{Grape Fruit With Sherry}

Prepare grape fruit for serving, add to each portion one tablespoon
Sherry wine, and let stand one hour in ice box or cold place.



\needspace{15\baselineskip}
\section*{Grape Fruit With Apricot Brandy}

Prepare grape fruit for serving and add to each portion one-half
tablespoon apricot brandy.



\needspace{15\baselineskip}
\section*{Grape Fruit With Sloe Gin}

Prepare grape fruit for serving and add to each portion one-half
tablespoon sloe gin.



\needspace{15\baselineskip}
\section*{Fruit Cocktail}

Remove pulp from grape fruit, and mix with shredded pineapple, bananas
cut in slices and slices cut in quarters, and strawberries cut in
halves, using half as much pineapple and banana as grape fruit, and
allowing four strawberries to each serve. There should be two cups
fruit. Pour over a dressing made of one-third cup Sherry wine, three
tablespoons apricot brandy, one-half cup sugar, and a few grains salt.
Chill thoroughly, serve in double cocktail glasses, and garnish with
candied cherries and leaves.



\needspace{15\baselineskip}
\section*{Baked Apples}

Wipe and core sour apples. Put in a baking-dish, and fill cavities with
sugar and spice. Allow one-half cup sugar and one-fourth teaspoon
cinnamon or nutmeg to eight apples. If nutmeg is used, a few drops lemon
juice and few gratings from rind of lemon to each apple is an
improvement. Cover bottom of dish with boiling water, and bake in a hot
oven until soft, basting often with syrup in dish. Serve hot or cold
with cream. Many prefer to pare apples before baking. When this is done,
core before paring, that fruit may keep in shape. In the fall, when
apples are at their best, do not add spices to apples, as their flavor
cannot be improved; but towards spring they become somewhat tasteless,
and spice is an improvement.



\needspace{15\baselineskip}
\section*{Baked Sweet Apples}

Wipe and core eight sweet apples. Put in a baking-dish, and fill
cavities with sugar, allowing one-third cup, or sweeten with molasses.
Add two-thirds cup boiling water. Cover, and bake three hours in a slow
oven, adding more water if necessary.



\needspace{15\baselineskip}
\section*{Apple Sauce}

Wipe, quarter, core, and pare eight sour apples. Make a syrup by boiling
seven minutes one cup sugar and one cup water with thin shaving from
rind of a lemon. Remove lemon, add enough apples to cover bottom of
saucepan, watch carefully during cooking, and remove as soon as soft.
Continue until all are cooked. Strain remaining syrup over apples.



\needspace{15\baselineskip}
\section*{Spiced Apple Sauce}

Wipe, quarter, core, and pare eight sour apples. Put in a saucepan,
sprinkle with one cup sugar, add eight cloves, and enough water to
prevent apples from burning. Cook to a mush, stirring occasionally.



\needspace{15\baselineskip}
\section*{Apple Ginger}

Wipe, quarter, core, pare, and chop sour apples; there should be two and
one-half pounds. Put in a stewpan and add one and one-half pounds light
brown sugar, juice and rind of one and one-half lemons, one-half ounce
ginger root, a few grains salt, and enough water to prevent apples from
burning. Cover, and cook slowly four hours, adding water as necessary.
Apple Ginger may be kept for several weeks.



\needspace{15\baselineskip}
\section*{Apple Porcupine}

Make a syrup by boiling eight minutes one and one-half cups sugar and
one and one-half cups water. Wipe, core, and pare eight apples. Put
apples in syrup as soon as pared, that they may not discolor. Cook until
soft, occasionally skimming syrup during cooking. Apples cook better
covered with the syrup; therefore it is better to use a deep saucepan
and have two cookings. Drain apples from syrup, cool, fill cavities with
jelly, marmalade, or preserved fruit, and stick apples with almonds
blanched and split in halves lengthwise. Serve with Cream Sauce I.



\needspace{15\baselineskip}
\section*{Baked Bananas I}

Remove skins from six bananas and cut in halves lengthwise. Put in a
shallow granite pan or on an old platter. Mix two tablespoons melted
butter, one-third cup sugar, and two tablespoons lemon juice. Baste
bananas with one-half the mixture. Bake twenty minutes in a slow oven,
basting during baking with remaining mixture.



\needspace{15\baselineskip}
\section*{Baked Bananas II}

Arrange bananas in a shallow pan, cover, and bake until skins become
very dark in color. Remove from skins, and serve hot sprinkled with
sugar.



\needspace{15\baselineskip}
\section*{Sautéd Bananas}

Remove skins from bananas, cut in halves lengthwise, and again cut in
halves crosswise. Dredge with flour, and sauté in clarified butter.
Drain, and sprinkle with powdered sugar.



\needspace{15\baselineskip}
\section*{Baked Peaches}

Peel, cut in halves, and remove stones from six peaches. Place in a
shallow granite pan. Fill each cavity with one teaspoon sugar, one-half
teaspoon butter, few drops lemon juice, and a slight grating nutmeg.
Cook twenty minutes, and serve on circular pieces of buttered dry toast.



\needspace{15\baselineskip}
\section*{Baked Pears}

Wipe, quarter, and core pears. Put in a deep pudding-dish, sprinkle with
sugar or add a small quantity of molasses, then add water to prevent
pears from burning. Cover, and cook two or three hours in a very slow
oven. Small pears may be baked whole. Seckel pears are delicious when
baked.



\needspace{15\baselineskip}
\section*{Baked Quinces}

Wipe, quarter, core, and pare eight quinces. Put in a baking dish,
sprinkle with three-fourths cup sugar, add one and one-half cups water,
cover, and cook until soft in a slow oven. Quinces require a long time
for cooking.



\needspace{15\baselineskip}
\section*{Cranberry Sauce}

Pick over and wash three cups cranberries. Put in a stewpan, add one and
one-fourth cups sugar and one cup boiling water. Cover, and boil ten
minutes. Care must be taken that they do not boil over. Skim and cool.



\needspace{15\baselineskip}
\section*{Cranberry Jelly}

Pick over and wash four cups cranberries. Put in a stewpan with one cup
boiling water, and boil twenty minutes. Rub through a sieve, add two
cups sugar, and cook five minutes. Turn into a mould or glasses.



\needspace{15\baselineskip}
\section*{Stewed Prunes}

Wash and pick over prunes. Put in a saucepan, cover with cold water, and
soak two hours; then cook until soft in same water. When nearly cooked,
add sugar or molasses to sweeten. Many prefer the addition of a small
quantity of lemon juice.



\needspace{15\baselineskip}
\section*{Rhubarb Sauce}

Peel and cut rhubarb in one-inch pieces. Put in a saucepan, sprinkle
generously with sugar, and add enough water to prevent rhubarb from
burning. Rhubarb contains such a large percentage of water that but
little additional water is needed. Cook until soft. If rhubarb is
covered with boiling water, allowed to stand five minutes, then drained
and cooked, less sugar will be required. Rhubarb is sometimes baked in
an earthen pudding-dish. If baked slowly for a long time it has a rich
red color.



\needspace{15\baselineskip}
\section*{Jellies}

Jellies are made of cooked fruit juice and sugar, in nearly all cases
the proportions being equal. Where failures occur, they may usually be
traced to the use of too ripe fruit.

\textbf{To Prepare Glasses for Jelly.} Wash glasses and put in a kettle of cold
water; place on range, and heat water gradually to boiling-point. Remove
glasses, and drain. Place glasses while filling on a cloth wrung out of
hot water.

\textbf{To Cover Jelly Glasses.} Cut letter paper in circular pieces just to
fit in top of glasses. Dip in brandy, and cover jelly. Put on tin covers
or circular pieces of paper cut larger than the glasses, and fastened
securely over the edge with mucilage. Some prefer to cover jelly with
melted paraffine then to adjust covers.

\textbf{To Make a Jelly Bag.} Fold two opposite corners of a piece of cotton
and wool flannel three-fourths yard long. Sew up in the form of a
cornucopia, rounding at the end. Fell the seam to make more secure. Bind
the top with tape, and furnish with two or three heavy loops by which it
may be hung.



\needspace{15\baselineskip}
\subsection*{Apple Jelly}

Wipe apples, remove stem and blossom ends, and cut in quarters. Put in a
granite or porcelain-lined preserving kettle, and add cold water to come
nearly to top of apples. Cover, and cook slowly until apples are soft;
mash, and drain through a coarse sieve. Avoid squeezing apples, which
makes jelly cloudy. Then allow juice to drip through a double thickness
of cheese-cloth or a jelly bag. Boil twenty minutes, and add an equal
quantity of heated sugar; boil five minutes, skim, and turn in glasses.
Put in a sunny window, and let stand twenty-four hours. Cover, and keep
in a cool, dry place. Porter apples make a delicious flavored jelly. If
apples are pared, a much lighter jelly may be made. Gravenstein apples
make a very spicy jelly.

\textbf{To Heat Sugar.} Put in a granite dish, place in oven, leaving oven door
ajar, and stir occasionally.



\needspace{15\baselineskip}
\subsection*{Quince Jelly}

Follow recipe for Apple Jelly, using quinces in place of apples, and
removing seeds from fruit. Quince parings are often used for jelly, the
better part of the fruit being used for canning.



\needspace{15\baselineskip}
\subsection*{Crab Apple Jelly}

Follow recipe for Apple Jelly, leaving apples whole instead of cutting
in quarters.



\needspace{15\baselineskip}
\subsection*{Currant Jelly}

Currants are in the best condition for making jelly between June
twenty-eighth and July third, and should not be picked directly after a
rain. Cherry currants make the best jelly. Equal proportions of red and
white currants are considered desirable, and make a lighter colored
jelly.

Pick over currants, but do not remove stems; wash and drain. Mash a few
in the bottom of a preserving kettle, using a wooden potato masher; so
continue until berries are used. Cook slowly until currants look white.
Strain through a course strainer, then allow juice to drop through a
double thickness of cheese-cloth or a jelly bag. Measure, bring to
boiling-point, and boil five minutes; add unequal measure of heated
sugar, boil three minutes, skim, and pour into glasses. Place in a sunny
window, and let stand twenty-four hours. Cover, and keep in a cool, dry
place.



\needspace{15\baselineskip}
\subsection*{Currant and Raspberry Jelly}

Follow recipe for Currant Jelly, using equal parts of currants and
raspberries.



\needspace{15\baselineskip}
\subsection*{Blackberry Jelly}

Follow recipe for Currant Jelly, using blackberries in place of
currants.



\needspace{15\baselineskip}
\subsection*{Raspberry Jelly}

Follow recipe for Currant Jelly, using raspberries in place of currants.
Raspberry Jelly is the most critical to make, and should not be
attempted if fruit is thoroughly ripe, or if it has been long picked.



\needspace{15\baselineskip}
\subsection*{Barberry Jelly}

Barberry Jelly is firmer and of better color if made from fruit picked
before the frost comes, while some of the berries are still green. Make
same as Currant Jelly, allowing one cup water to one peck barberries.



\needspace{15\baselineskip}
\subsection*{Grape Jelly}

Grapes should be picked over, washed, and stems removed before putting
into a preserving kettle. Heat to boiling-point, mash, and boil thirty
minutes; then proceed as for Currant Jelly. Wild grapes make the best
jelly.



\needspace{15\baselineskip}
\subsection*{Green Grape Jelly}

Grapes should be picked when just beginning to turn. Make same as Grape
Jelly.



\needspace{15\baselineskip}
\subsection*{Venison Jelly}


\begin{itemize}
\setlength{\itemsep}{0pt}
\setlength{\parsep}{0pt}
\item 1 peck wild grapes
\item 1 quart vinegar
\item 1/4 cup whole cloves
\item 1/4 cup stick cinnamon
\item 6 pounds sugar
\end{itemize}

\vspace{-0.5em}
\noindent%
Put first four ingredients into a preserving kettle, heat slowly to the
boiling-point, and cook until grapes are soft. Strain through a double
thickness of cheese-cloth or a jelly bag, and boil liquid twenty
minutes; then add sugar heated, and boil five minutes. Turn into
glasses.



\needspace{15\baselineskip}
\subsection*{Damson Jelly}

Wipe and pick over damsons; then prick several times with a large pin.
Make same as Currant Jelly, using three-fourths as much sugar as fruit
juice.



\needspace{15\baselineskip}
\section*{Jams}

Raspberries and blackberries are the fruits most often employed for
making jams, and require equal weight of sugar and fruit.



\needspace{15\baselineskip}
\subsection*{Raspberry Jam}

Pick over raspberries. Mash a few in the bottom of a preserving kettle,
using a wooden potato masher, and so continue until the fruit is used.
Heat slowly to boiling-point, and add gradually an equal quantity of
heated sugar. Cook slowly forty-five minutes. Put in a stone jar or
tumblers.



\needspace{15\baselineskip}
\subsection*{Blackberry Jam}

Follow recipe for Raspberry Jam, using blackberries in place of
raspberries.



\needspace{15\baselineskip}
\section*{Marmalades}

Marmalades are made of the pulp and juice of fruits with sugar.



\needspace{15\baselineskip}
\subsection*{Grape Marmalade}

Pick over, wash, drain, and remove stems from grapes. Separate pulp from
skins. Put pulp in preserving kettle. Heat to boiling-point, and cook
slowly until seeds separate from pulp; then rub through a hair sieve.
Return to kettle with skins, add an equal measure of sugar, and cook
slowly thirty minutes, occasionally stirring to prevent burning. Put in
a stone jar or tumblers.



\needspace{15\baselineskip}
\subsection*{Quince Marmalade}

Wipe quinces, remove blossom ends, cut in quarters, remove seeds; then
cut in small pieces. Put into a preserving kettle, and add enough water
to nearly cover. Cook slowly until soft. Rub through a hair sieve, and
add three-fourths its measure of heated sugar. Cook slowly twenty
minutes, stirring occasionally to prevent burning. Put in tumblers.



\needspace{15\baselineskip}
\subsection*{Orange Marmalade I}

Select sour, smooth-skinned oranges. Weigh oranges, and allow
three-fourths their weight in cut sugar. Remove peel from oranges in
quarters. Cook peel until soft in enough boiling water to cover; drain,
remove white part from peel by scraping it with a spoon. Cut thin yellow
rind in strips, using a pair of scissors. This is more quickly
accomplished by cutting through two or three pieces at a time. Divide
oranges in sections, remove seeds and tough part of the skin. Put into a
preserving kettle and heat to boiling-point, add sugar gradually, and
cook slowly one hour; add rind, and cook one hour longer. Turn into
glasses.



\needspace{15\baselineskip}
\subsection*{Orange Marmalade II}

Slice nine oranges and six lemons crosswise with a sharp knife as thinly
as possible, remove seeds, and put in a preserving kettle with four
quarts water. Cover, and let stand thirty-six hours; then boil for two
hours, add eight pounds sugar, and boil one hour longer.







\needspace{15\baselineskip}
\subsection*{Orange and Rhubarb Marmalade}

Remove peel in quarters from eight oranges and prepare as for Orange
Marmalade. Divide oranges in sections, remove seeds and tough part of
skin. Put into a preserving kettle, add five pounds rhubarb, skinned and
cut in one-half inch pieces. Heat to boiling-point, and boil one-half
hour; then add four pounds cut sugar and cut rind. Cook slowly two
hours. Turn into glasses.



\needspace{15\baselineskip}
\subsection*{Quince Honey}

Pare and grate five large quinces. To one pint boiling water add five
pounds sugar. Stir over fire until sugar is dissolved, add quince, and
cook fifteen or twenty minutes. Turn into glasses. When cold it should
be about the color and consistency of honey.



\needspace{15\baselineskip}
\section*{Canning And Preserving}

Preserving fruit is cooking it with from three-fourths to its whole
weight of sugar. By so doing, much of the natural flavor of the fruit is
destroyed; therefore canning is usually preferred to preserving.

Canning fruit is preserving sterilized fruit in sterilized air-tight
jars, the sugar being added to give sweetness. Fruits may be canned
without sugar if perfectly sterilized, that is, freed from all germ
life.



\needspace{15\baselineskip}
\subsection*{Directions for Canning}

Fruit for canning should be fresh, firm, of good quality, and not
over-ripe; if over-ripe, some of the spores may survive the boiling,
then fermentation will take place in a short time.

For canning fruit, allow one-third its weight in sugar, and two and
one-half to three cups water to each pound of sugar. Boil sugar and
water ten minutes to make a thin syrup; then cook a small quantity of
the fruit at a time in the syrup; by so doing, fruit may be kept in
perfect shape. Hard fruits, like pineapple and quince, are cooked in
boiling water until nearly soft, then put in syrup to finish cooking.
Sterilized jars are then filled with fruit, and enough syrup added to
overflow jars. If there is not sufficient syrup, add boiling water, as
jars must be filled to overflow. Introduce a spoon between fruit and
jar, that air bubbles may rise to the top and break; then quickly put on
rubbers and screw on sterilized covers. Let stand until cold, again
screw covers, being sure this time that jars are air-tight. While
filling jars, place them on a cloth wrung out of hot water.



\needspace{15\baselineskip}
\subsection*{To Sterilize Jars}

Wash jars and fill with cold water. Set in a kettle on a trivet, and
surround with cold water. Heat gradually to boiling-point, remove from
water, empty, and fill while hot. Put covers in hot water and let stand
five minutes. Dip rubber bands in hot water, but do not allow them to
stand. New rubbers should be used each season, and care must be taken
that rims of covers are not bent, as jars cannot then be hermetically
sealed.



\needspace{15\baselineskip}
\subsection*{Canned Porter Apples}

Wipe, quarter, core, and pare Porter apples, then weigh. Make a syrup by
boiling for ten minutes one-third their weight in sugar with water,
allowing two and one-half cups to each pound of sugar. Cook apples in
syrup until soft, doing a few at a time. Fill jars, following Directions
for Canning.



\needspace{15\baselineskip}
\subsection*{Canned Peaches}

Wipe peaches and put in boiling water, allowing them to stand just long
enough to easily loosen skins. Remove skins and cook fruit at once, that
it may not discolor, following Directions for Canning. Some prefer to
pare peaches, sprinkle with sugar, and let stand over night. In morning
drain, add water to fruit syrup, bring to boiling-point, and then cook
fruit. Peaches may be cut in halves, or smaller pieces if desired.



\needspace{15\baselineskip}
\subsection*{Canned Pears}

Wipe and pare fruit. Cook whole with stems left on, or remove stems, cut
in quarters, and core. Follow Directions for Canning. A small piece of
ginger root or a few slicings of lemon rind may be cooked with syrup.
Bartlett pears are the best for canning.



\needspace{15\baselineskip}
\subsection*{Canned Pineapples}

Remove skin and eyes from pineapples; then cut in half-inch slices, and
slices in cubes, at the same time discarding the core. Follow Directions
for Canning. Pineapples may be shredded and cooked in one-half their
weight of sugar without water, and then put in jars. When put up in this
way they are useful for the making of sherbets and fancy desserts.



\needspace{15\baselineskip}
\subsection*{Canned Quinces}

Wipe, quarter, core, and pare quinces. Follow Directions for Canning.
Quinces may be cooked with an equal weight of sweet apples wiped,
quartered, cored, and pared; in this case use no extra sugar for apples.



\needspace{15\baselineskip}
\subsection*{Canned Cherries}

Use large white or red cherries. Wash, remove stems, then follow
Directions for Canning.



\needspace{15\baselineskip}
\subsection*{Canned Huckleberries}

Pick over and wash berries, then put in a preserving kettle with a small
quantity of water to prevent berries from burning. Cook until soft,
stirring occasionally, and put in jars. No sugar is required, but a
sprinkling of salt is an agreeable addition.



\needspace{15\baselineskip}
\subsection*{Canned Rhubarb}

Pare rhubarb and cut in one-inch pieces. Pack in a jar, put under cold
water faucet, and let water run twenty minutes, then screw on cover.
Rhubarb canned in this way has often been known to keep a year.



\needspace{15\baselineskip}
\subsection*{Canned Tomatoes}

Wipe tomatoes, cover with boiling water, and let stand until skins may
be easily removed. Cut in pieces and cook until thoroughly scalded; skim
often during cooking. Fill jars, following directions given.



\needspace{15\baselineskip}
\subsection*{Damson Preserves}

Wipe damsons with a piece of cheese-cloth wrung out of cold water, and
prick each fruit five or six times, using a large needle; then weigh.
Make a syrup by boiling three-fourths their weight in sugar with water,
allowing one cup to each pound of sugar. As soon as syrup reaches
boiling-point, skim, and add plums, a few at a time, that fruit may
better keep in shape during cooking. Cook until soft. It is well to use
two kettles, that work may be more quickly done, and syrup need not cook
too long a time. Put into glass or stone jars.



\needspace{15\baselineskip}
\subsection*{Strawberry Preserves}

Pick over, wash, drain, and hull strawberries; then weigh. Fill glass
jars with berries. Make a syrup same as for Damson Preserve, cooking the
syrup fifteen minutes. Add syrup to overflow jars; let stand fifteen
minutes, when fruit will have shrunk, and more fruit must be added to
fill jars. Screw on covers, put on a trivet in a kettle of cold water,
heat water to boiling-point, and keep just below boiling-point one hour.

Raspberries may be preserved in the same way.



\needspace{15\baselineskip}
\subsection*{Pear Chips}


\begin{itemize}
\setlength{\itemsep}{0pt}
\setlength{\parsep}{0pt}
\item 8 lbs. pears
\item 4 lbs. sugar
\item 1/4 lb. Canton ginger
\item 4 lemons
\end{itemize}

\vspace{-0.5em}
\noindent%
Wipe pears, remove stems, quarter, and core; then cut in small pieces.
Add sugar and ginger, and let stand over night. In the morning add
lemons cut in small pieces, rejecting seeds, and cook slowly three
hours. Put into a stone jar.



\needspace{15\baselineskip}
\subsection*{Raspberry and Currant Preserve}


\begin{itemize}
\setlength{\itemsep}{0pt}
\setlength{\parsep}{0pt}
\item 6 lbs. currants
\item 6 lbs. sugar
\item 8 quarts raspberries
\end{itemize}

\vspace{-0.5em}
\noindent%
Pick over, wash, and drain currants. Put into a preserving kettle,
adding a few at a time, and mash. Cook one hour, strain through double
thickness of cheese-cloth. Return to kettle, add sugar, heat to
boiling-point, and cook slowly twenty minutes. Add one quart raspberries
when syrup again reaches boiling-point, skim out raspberries, put in
jar, and repeat until raspberries are used. Fill jars to overflowing
with syrup, and screw on tops.



\needspace{15\baselineskip}
\subsection*{Brandied Peaches}


\begin{itemize}
\setlength{\itemsep}{0pt}
\setlength{\parsep}{0pt}
\item 1 peck peaches
\item Half their weight in sugar
\item 1 quart high-proof alcohol or brandy
\end{itemize}

\vspace{-0.5em}
\noindent%
Remove skins from peaches, and put alternate layers of peaches and sugar
in a stone jar; then add alcohol. Cover closely, having a heavy piece of
cloth under cover of jar.



\needspace{15\baselineskip}
\subsection*{Tutti-Frutti}

Put one pint brandy into a stone jar, add the various fruits as they
come into market; to each quart of fruit add the same quantity of sugar,
and stir the mixture each morning until all the fruit has been added.
Raspberries, strawberries, apricots, peaches, cherries, and pineapples
are the best to use.



\needspace{15\baselineskip}
\subsection*{Canned Red Peppers}

Wash one peck red peppers, cut a slice from stem end of each, and remove
seeds; then cut in thin strips by working around and around the peppers,
using scissors or a sharp vegetable knife. Cover with boiling water, let
stand two minutes, drain, and plunge into ice-water. Let stand ten
minutes, again drain, and pack solidly into pint glass jars. Boil one
quart vinegar and two cups sugar fifteen minutes. Pour over peppers to
overflow jars, cover, and keep in a cold place.



\needspace{15\baselineskip}
\subsection*{Preserved Melon Rind}

Pare and cut in strips the rind of ripe melons. Soak in alum water to
cover, allowing two teaspoons powdered alum to each quart of water. Heat
gradually to boiling-point and cook slowly ten minutes. Drain, cover
with ice-water, and let stand two hours; again drain, and dry between
towels. Weigh, allow one pound sugar to each pound of fruit, and one cup
water to each pound of sugar. Boil sugar and water ten minutes. Add
melon rind, and cook until tender. Remove rind to a stone jar, and cover
with syrup. Two lemons cut in slices may be cooked ten minutes in the
syrup.



\needspace{15\baselineskip}
\subsection*{Tomato Preserve}


\begin{itemize}
\setlength{\itemsep}{0pt}
\setlength{\parsep}{0pt}
\item 1 lb. yellow pear tomatoes
\item 1 lb. sugar
\item 2 ozs. preserved Canton ginger
\item 2 lemons
\end{itemize}

\vspace{-0.5em}
\noindent%
Wipe tomatoes, cover with boiling water, and let stand until skins may
be easily removed. Add sugar, cover, and let stand over night. In the
morning pour off syrup and boil until quite thick; skim, then add
tomatoes, ginger, and lemons which have been sliced and the seeds
removed. Cook until tomatoes have a clarified appearance.



\needspace{15\baselineskip}
\section*{Pickling}

Pickling is preserving in any salt or acid liquor.



\needspace{15\baselineskip}
\subsection*{Spiced Currants}


\begin{itemize}
\setlength{\itemsep}{0pt}
\setlength{\parsep}{0pt}
\item 7 lbs. currants
\item 5 lbs. brown sugar
\item 3 tablespoons cinnamon
\item 3 tablespoons clove
\item 1 pint vinegar
\end{itemize}

\vspace{-0.5em}
\noindent%
Pick over currants, wash, drain, and remove stems. Put in a preserving
kettle, add sugar, vinegar, and spices tied in a piece of muslin. Heat
to boiling-point, and cook slowly one and one-half hours. Store in a
stone jar and keep in a cool place. Spiced currants are a delicious
accompaniment to cold meat.



\needspace{15\baselineskip}
\subsection*{Sweet Pickled Peaches}


\begin{itemize}
\setlength{\itemsep}{0pt}
\setlength{\parsep}{0pt}
\item 1/2 peck peaches
\item 2 lbs. brown sugar
\item 1 pint vinegar
\item 1 oz. stick cinnamon
\item Cloves
\end{itemize}

\vspace{-0.5em}
\noindent%
Boil sugar, vinegar, and cinnamon twenty minutes. Dip peaches quickly in
hot water, then rub off the fur with a towel. Stick each peach with four
cloves. Put into syrup, and cook until soft, using one-half peaches at a
time.



\needspace{15\baselineskip}
\subsection*{Sweet Pickled Pears}

Follow recipe for Sweet Pickled Peaches, using pears in place of
peaches.



\needspace{15\baselineskip}
\subsection*{Chili Sauce}


\begin{minipage}{1.0\textwidth}
{\setlength{\multicolsep}{0pt}\setlength{\columnsep}{2em}\raggedcolumns%
\begin{multicols}{2}
\begin{itemize}
\setlength{\itemsep}{0pt}
\setlength{\parsep}{0pt}
\item 12 medium-sized ripe tomatoes
\item 1 pepper, finely chopped
\item 1 onion, finely chopped
\item 2 cups vinegar
\item 3 tablespoons sugar
\item 1 tablespoon salt
\item 2 teaspoons clove
\item 2 teaspoons cinnamon
\item 2 teaspoons allspice
\item 2 teaspoons grated nutmeg
\end{itemize}
\end{multicols}}
\end{minipage}

\vspace{0.3em}
\noindent%
Peel tomatoes and slice. Put in a preserving kettle with remaining
ingredients. Heat gradually to boiling-point, and cook slowly two and
one-half hours.



\needspace{15\baselineskip}
\subsection*{Ripe Tomato Pickle}


\begin{minipage}{1.0\textwidth}
{\setlength{\multicolsep}{0pt}\setlength{\columnsep}{2em}\raggedcolumns%
\begin{multicols}{2}
\begin{itemize}
\setlength{\itemsep}{0pt}
\setlength{\parsep}{0pt}
\item 3 pints tomatoes, peeled and chopped
\item 1 cup chopped celery
\item 4 tablespoons chopped red pepper
\item 4 tablespoons chopped onion
\item 4 tablespoons salt
\item 6 tablespoons sugar
\item 6 tablespoons mustard seed
\item 1/2 teaspoon clove
\item 1/2 teaspoon cinnamon
\item 1 teaspoon grated nutmeg
\item 2 cups vinegar
\end{itemize}
\end{multicols}}
\end{minipage}

\vspace{0.3em}
\noindent%
Mix ingredients in order given. Put in a stone jar and cover. This
uncooked mixture must stand a week before using, but may be kept a year.



\needspace{15\baselineskip}
\subsection*{Ripe Cucumber Pickle}

Cut cucumbers in halves lengthwise. Cover with alum water, allowing two
teaspoons powdered alum to each quart of water. Heat gradually to
boiling-point, then let stand on back of range two hours. Remove from
alum water and chill in ice-water. Make a syrup by boiling five minutes
two pounds sugar, one pint vinegar, with two tablespoons each of whole
cloves and stick cinnamon tied in a piece of muslin. Add cucumbers and
cook ten minutes. Remove cucumbers to a stone jar, and pour over the
syrup. Scald syrup three successive mornings, and return to cucumbers.



\needspace{15\baselineskip}
\subsection*{Unripe Cucumber Pickles (Gherkins)}

Wipe four quarts small unripe cucumbers. Put in a stone jar and add one
cup salt dissolved in two quarts boiling water and let stand three days.
Drain cucumbers from brine, bring brine to boiling-point, pour over
cucumbers, and again let stand three days; repeat. Drain, wipe
cucumbers, and pour over one gallon boiling water in which one
tablespoon alum has been dissolved. Let stand six hours, then drain from
alum water. Cook cucumbers ten minutes, a few at a time, in one-fourth
the following mixture heated to the boiling-point and boiled ten
minutes:


\begin{itemize}
\setlength{\itemsep}{0pt}
\setlength{\parsep}{0pt}
\item 1 gallon vinegar
\item 4 red peppers
\item 2 sticks cinnamon
\item 2 tablespoons allspice berries
\item 2 tablespoons cloves
\end{itemize}

\vspace{-0.5em}
\noindent%
Strain remaining liquor over pickles which have been put in a stone jar.



\needspace{15\baselineskip}
\subsection*{Chopped Pickles}


\begin{minipage}{1.0\textwidth}
{\setlength{\multicolsep}{0pt}\setlength{\columnsep}{2em}\raggedcolumns%
\begin{multicols}{2}
\begin{itemize}
\setlength{\itemsep}{0pt}
\setlength{\parsep}{0pt}
\item 4 quarts chopped green tomatoes
\item 3/4 cup salt
\item 2 teaspoons pepper
\item 3 teaspoons mustard
\item 3 teaspoons cinnamon
\item 3 teaspoons allspice
\item 3 teaspoons cloves
\item 1/2 cup white mustard seed
\item 4 green peppers, sliced
\item 2 chopped onions
\item 2 quarts vinegar
\end{itemize}
\end{multicols}}
\end{minipage}

\vspace{0.3em}
\noindent%
Add salt to tomatoes, cover, let stand twenty-four hours, and drain. Add
spices to vinegar, and heat to boiling-point; then add tomatoes,
peppers, and onions, bring to boiling-point, and cook fifteen minutes
after boiling-point is reached. Store in a stone jar and keep in a cool
place.



\needspace{15\baselineskip}
\subsection*{Spanish Pickles}


\begin{minipage}{1.0\textwidth}
{\setlength{\multicolsep}{0pt}\setlength{\columnsep}{2em}\raggedcolumns%
\begin{multicols}{2}
\begin{itemize}
\setlength{\itemsep}{0pt}
\setlength{\parsep}{0pt}
\item 1 peck green tomatoes, thinly sliced
\item 4 onions, thinly sliced
\item 1 cup salt
\item 1/2 oz. cloves
\item 1/2 oz. allspice berries
\item 1/2 oz. peppercorns
\item 1/2 cup brown mustard seed
\item 1 lb. brown sugar
\item 4 green peppers, finely chopped
\item Cider vinegar
\end{itemize}
\end{multicols}}
\end{minipage}

\vspace{0.3em}
\noindent%
Sprinkle alternate layers of tomatoes and onions with salt, and let
stand over night. In the morning drain, and put in a preserving kettle,
adding remaining ingredients, using enough vinegar to cover all. Heat
gradually to boiling-point and boil one-half hour.



\needspace{15\baselineskip}
\subsection*{Chow-Chow}


\begin{minipage}{1.0\textwidth}
{\setlength{\multicolsep}{0pt}\setlength{\columnsep}{2em}\raggedcolumns%
\begin{multicols}{2}
\begin{itemize}
\setlength{\itemsep}{0pt}
\setlength{\parsep}{0pt}
\item 2 quarts small green tomatoes
\item 12 small cucumbers
\item 3 red peppers
\item 1 cauliflower
\item 2 bunches celery
\item 1 pint small onions
\item 2 quarts string beans
\item 1/4 lb. mustard seed
\item 2 oz. turmeric
\item 1/2 oz. allspice
\item 1/2 oz. pepper
\item 1/2 oz. clove
\item Salt
\item 1 gallon vinegar
\end{itemize}
\end{multicols}}
\end{minipage}

\vspace{0.3em}
\noindent%
Prepare vegetables and cut in small pieces, cover with salt, let stand
twenty-four hours, and drain. Heat vinegar and spices to boiling-point,
add vegetables, and cook until soft.



\needspace{15\baselineskip}
\subsection*{Pickled Onions}

Peel small white onions, cover with brine, allowing one and one-half
cups salt to two quarts boiling water, and let stand two days; drain,
and cover with more brine; let stand two days, and again drain. Make
more brine and heat to boiling-point; put in onions and boil three
minutes. Put in jars, interspersing with bits of mace, white
peppercorns, cloves, bits of bay leaf, and slices of red pepper. Fill
jars to overflow with vinegar scalded with sugar, allowing one cup sugar
to one gallon vinegar. Cork while hot.





\chapter{Helpful Hints}



\textbf{To Scald Milk.} Put in top of double boiler, having water boiling in
under part. Cover, and let stand on top of range until milk around edge
of double boiler has a bead-like appearance.


\textbf{For Buttered Cracker Crumbs}, allow from one-fourth to one-third cup
melted butter to each cup of crumbs. Stir lightly with a fork in mixing,
that crumbs may be evenly coated and light rather than compact.


\textbf{To Cream Butter.} Put in a bowl and work with a wooden spoon until soft
and of creamy consistency. Should buttermilk exude from butter it should
be poured off.


\textbf{To Extract Juice from Onion.} Cut a slice from root end of onion, draw
back the skin, and press onion on a coarse grater, working with a rotary
motion.


\textbf{To Chop Parsley.} Remove leaves from parsley. If parsley is wet, first
dry in a towel. Gather parsley between thumb and fingers and press
compactly. With a sharp vegetable knife cut through and through. Again
gather in fingers and recut, so continuing until parsley is finely cut.


\textbf{To Caramelize Sugar.} Put in a smooth granite saucepan or omelet pan,
place over hot part of range, and stir constantly until melted and of
the color of maple syrup. Care must be taken to prevent sugar from
adhering to sides of pan or spoon.


\textbf{To Make Caramel.} Continue the caramelization of sugar until syrup is
quite brown and a whitish smoke arises from it. Add an equal quantity of
boiling water, and simmer until of the consistency of a thick syrup. Of
use in coloring soups, sauces, etc.


\textbf{Acidulated Water} is water to which vinegar or lemon juice is added.
One tablespoon of the acid is allowed to one quart water.


\textbf{To Blanch Almonds.} Cover Jordan almonds with boiling water and let
stand two minutes; drain, put into cold water, and rub off the skins.
Dry between towels.


\textbf{To Shred Almonds.} Cut blanched almonds in thin strips lengthwise of
the nut.


\textbf{Macaroon Dust.} Dry macaroons pounded and sifted.


\textbf{To Shell Chestnuts.} Cut a half-inch gash on flat sides and put in an
omelet pan, allowing one-half teaspoon butter to each cup chestnuts.
Shake over range until butter is melted. Put in oven and let stand five
minutes. Remove from oven, and with a small knife take off shells. By
this method shelling and blanching is accomplished at the same time, as
skins adhere to shells.


\textbf{Flavoring Extracts and Wine} should be added if possible to a mixture
when cold. If added while mixture is hot, much of the goodness passes
off with the steam.


\textbf{Meat Glaze.} Four quarts stock reduced to one cup.


\textbf{Mixed Mustard.} Mix two tablespoons mustard and one teaspoon sugar, add
hot water gradually until of the consistency of a thick paste. Vinegar
may be used in place of water.


\textbf{To Prevent Salt from Lumping.} Mix with corn-starch, allowing one
teaspoon corn-starch to six teaspoons salt.


\textbf{To Wash Carafes.} Half fill with hot soapsuds, to which is added one
teaspoon washing soda. Put in newspaper torn in small pieces. Let stand
one-half hour, occasionally shaking. Empty, rinse with hot water, drain,
wipe outside, and let stand to dry inside.


\textbf{After Broiling or Frying}, if any fat has spattered on range, wipe
surface at once with newspaper.


\textbf{To Remove Fruit Stains.} Pour boiling water over stained surface,
having it fall from a distance of three feet. This is a much better way
than dipping stain in and out of hot water; or wring articles out of
cold water and hang out of doors on a frosty night.


\textbf{To Remove Stains of Claret Wine.} As soon as claret is spilt, cover
spot with salt. Let stand a few minutes, then rinse in cold water.


\textbf{To Clean Graniteware} where mixtures have been cooked or burned on.
Half fill with cold water, add washing soda, heat water gradually to
boiling-point, then empty, when dish may be easily washed. Pearline or
any soap-powder may be used in place of washing soda.


\textbf{To Wash Mirrors and Windows.} Rub over with chamois skin wrung out of
warm water, then wipe with a piece of dry chamois skin. This method
saves much strength.


\textbf{To Remove White Spots from Furniture.} Dip a cloth in hot water nearly
to boiling-point. Place over spot, remove quickly, and rub over spot
with a dry cloth. Repeat if spot is not removed. Alcohol or camphor
quickly applied may be used.


\textbf{Tumblers} which have contained milk should be first rinsed in cold
water before washing in hot water.


To keep a \textbf{Sink Drain} free from grease, pour down once a week at night
one-half can Babbitt's potash dissolved in one quart water.


Should \textbf{Sink Drain} chance to get choked, pour into sink one-fourth
pound copperas dissolved in two quarts boiling water. If this is not
efficacious, repeat before sending for a plumber.


Never put \textbf{Knives} with ivory handles in water. Hot water causes them to
crack and discolor.


To prevent \textbf{Glassware} from being easily broken, put in a kettle of cold
water, heat gradually until water has reached boiling-point. Set aside;
when water is cold take out glass. This is a most desirable way to
toughen lamp chimneys.


\textbf{To Remove Grease Spots.} Cold water and Ivory Soap will remove grease
spots from cotton and woollen fabrics. Castilian Cream is useful for
black woollen goods, but leaves a light ring on delicately colored
goods. Ether is always sure and safe to use.


\textbf{To Remove Iron Rust.} Saturate spot with lemon juice, then cover with
salt. Let stand in the sun for several hours; or a solution of
hydrochloric acid may be used.


\textbf{Iron Rust} may be removed from delicate fabrics by covering spot
thickly with cream of tartar, then twisting cloth to keep cream of
tartar over spot; put in a saucepan of cold water, and heat water
gradually to boiling-point.


\textbf{To Remove Grass Stains} from cotton goods, wash in alcohol.


\textbf{To Remove Ink Stains.} Wash in a solution of hydrochloric acid, and
rinse in ammonia water. Wet the spot with warm water, put on Sapolio,
rub gently between the hands, and generally the spot will disappear.


\textbf{Cut Glass} should be washed and rinsed in water that is not very hot
and of same temperature.


In \textbf{Sweeping Carpets}, keep broom close to floor and work with the grain
of the carpet. Occasionally turn broom that it may wear evenly.


\textbf{Tie Strands of a New Broom} closely together, put into a pail of
boiling water, and soak two hours. Dry thoroughly before using.


Never wash the inside of \textbf{Tea or Coffee Pots} with soapsuds. If granite
or agate ware is used, and becomes badly discolored, nearly fill pot
with Cold water, add one tablespoon borax, and heat gradually until
water reaches the boiling-point. Rinse with hot water, wipe, and keep on
back of range until perfectly dry.


Never put cogs of a \textbf{Dover Egg-beater} in water.


Never wash \textbf{Bread Boards} in a sink. Scrub with grain of wood, using a
small brush.


Before using a new \textbf{Iron Kettle}, grease inside and outside, and let
stand forty-eight hours; then wash in hot water in which a large lump of
cooking soda has been dissolved.


To clean a \textbf{Copper Boiler}, use Putz Pomade Cream. Apply with a woollen
cloth when boiler is warm, not hot; then rub off with second woollen
cloth and polish with flannel or chamois. If badly tarnished, use oxalic
acid. Faucets and brasses are treated in the same way.


A bottle containing \textbf{Oxalic Acid} should be marked poison, and kept on a
high shelf.


To keep an \textbf{Ice Chest} in good condition, wash thoroughly once a week
with cold or lukewarm water in which washing soda has been dissolved. If
by chance anything is spilt in an ice chest, it should be wiped off at
once.

Milk and butter very quickly absorb odors, and if in ice chest with
other foods, should be kept closely covered.


\textbf{Hard Wood Floors and Furniture} may be polished by using a small
quantity of kerosene oil applied with a woollen cloth, then rubbing with
a clean woollen cloth. A very good furniture polish is made by using
equal parts linseed oil and turpentine.


\textbf{Polish for Hard Wood Floors.} Use one part beeswax to two parts
turpentine. Put in saucepan on range, and when wax is dissolved a paste
will be formed.


To clean \textbf{Piano Keys}, rub over with alcohol.


To remove old \textbf{Tea and Coffee Stains}, wet spot with cold water, cover
with glycerine, and let stand two or three hours. Then wash with cold
water and hard soap. Repeat if necessary.


Before \textbf{Sweeping Old Carpets}, sprinkle with pieces of newspaper wrung
out of water. After sweeping, wipe over with a cloth wrung out of a weak
solution of ammonia water, which seems to brighten colors.


Platt's Chloride is one of the best \textbf{Disinfectants}. Chloride of lime is
a valuable disinfectant, and much cheaper than Platt's Chloride.


\textbf{Listerine} is an excellent disinfectant to use for the mouth and
throat.


\textbf{To Make a Pastry Bag.} Fold a twelve-inch square of rubber cloth from
two opposite corners. Sew edges together, forming a triangular bag. Cut
off point to make opening large enough to insert a tin pastry tube. A
set comprising bag and twelve adjustable tubes may be bought for two and
one-half dollars.


\textbf{Smoked Ceilings} may be cleaned by washing with cloths wrung out of
water in which a small piece of washing soda has been dissolved.


\textbf{For a Burn} apply equal parts of white of egg and olive oil mixed
together, then cover with a piece of old linen; if applied at once no
blister will form. Or apply at once cooking soda, then cover with cloth
and keep the same wet with cold water. This takes out the pain and
prevents blistering.


\textbf{Curtain and Portière Poles} allow the hangings to slip easily if rubbed
with hard soap. This is much better than greasing.


\textbf{Creaking Doors and Drawers} should be treated in the same way.


\textbf{To Remove Dust from Rattan Furniture} use a painter's small brush.





\chapter{Suitable Combinations For Serving}


\section*{\centering Breakfast Menus}


\vspace{1em}

\noindent
\textbf{Menu 1}

\vspace{0.5em}

{\renewcommand{\arraystretch}{1.5}
\begin{tabular*}{\textwidth}{@{\extracolsep{\fill}}|>{\hspace{0.5em}}c<{\hspace{0.5em}}|>{\hspace{0.5em}}c<{\hspace{0.5em}}|>{\hspace{0.5em}}c<{\hspace{0.5em}}|}
\hline
\multicolumn{3}{|c|}{Oranges} \\ \hline
\multicolumn{3}{|c|}{Oatmeal with Sugar and Cream} \\ \hline
Boiled Ham & Creamed Potatoes & Pop-overs or Fadges \\ \hline
\multicolumn{3}{|c|}{Coffee} \\ \hline
\end{tabular*}}
\vspace{0.5em}


\vspace{1em}

\noindent
\textbf{Menu 2}

\vspace{0.5em}

{\renewcommand{\arraystretch}{1.5}
\begin{tabular*}{\textwidth}{@{\extracolsep{\fill}}|>{\hspace{0.5em}}c<{\hspace{0.5em}}|>{\hspace{0.5em}}c<{\hspace{0.5em}}|>{\hspace{0.5em}}c<{\hspace{0.5em}}|}
\hline
\multicolumn{3}{|c|}{Quaker Rolled Oats with Baked Apples, Sugar and Cream} \\ \hline
Creamed Fish & Baked Potatoes & Golden Corn Cake \\ \hline
\multicolumn{3}{|c|}{Coffee} \\ \hline
\end{tabular*}}
\vspace{0.5em}


\vspace{1em}

\noindent
\textbf{Menu 3}

\vspace{0.5em}

{\renewcommand{\arraystretch}{1.5}
\begin{tabular*}{\textwidth}{@{\extracolsep{\fill}}|>{\hspace{0.5em}}c<{\hspace{0.5em}}|>{\hspace{0.5em}}c<{\hspace{0.5em}}|>{\hspace{0.5em}}c<{\hspace{0.5em}}|}
\hline
\multicolumn{3}{|c|}{Bananas} \\ \hline
\multicolumn{3}{|c|}{Old Grist Mill Toasted Wheat with Sugar and Cream} \\ \hline
Scrambled Eggs & Sautéd Potatoes & Graham Gems \\ \hline
\multicolumn{3}{|c|}{Griddle Cakes} \\ \hline
\multicolumn{3}{|c|}{Coffee} \\ \hline
\end{tabular*}}
\vspace{0.5em}


\vspace{1em}

\noindent
\textbf{Menu 4}

\vspace{0.5em}

{\renewcommand{\arraystretch}{1.5}
\begin{tabular*}{\textwidth}{@{\extracolsep{\fill}}|>{\hspace{0.5em}}c<{\hspace{0.5em}}|>{\hspace{0.5em}}c<{\hspace{0.5em}}|>{\hspace{0.5em}}c<{\hspace{0.5em}}|}
\hline
\multicolumn{3}{|c|}{Grape Fruit} \\ \hline
\multicolumn{3}{|c|}{Wheatlet with Sugar and Cream} \\ \hline
Beefsteak & Lyonnaise Potatoes & Twin Mountain Muffins \\ \hline
\multicolumn{3}{|c|}{Coffee} \\ \hline
\end{tabular*}}
\vspace{0.5em}


\vspace{1em}

\noindent
\textbf{Menu 5}

\vspace{0.5em}

{\renewcommand{\arraystretch}{1.5}
\begin{tabular*}{\textwidth}{@{\extracolsep{\fill}}|>{\hspace{0.5em}}c<{\hspace{0.5em}}|>{\hspace{0.5em}}c<{\hspace{0.5em}}|>{\hspace{0.5em}}c<{\hspace{0.5em}}|}
\hline
\multicolumn{3}{|c|}{Sliced Oranges} \\ \hline
\multicolumn{3}{|c|}{Wheat Germ with Sugar and Cream} \\ \hline
Warmed over Lamb & French Fried Potatoes & Raised Biscuits \\ \hline
\multicolumn{3}{|c|}{Buckwheat Cakes with Maple Syrup} \\ \hline
\multicolumn{3}{|c|}{Old Grist Mill Coffee} \\ \hline
\end{tabular*}}
\vspace{0.5em}


\vspace{1em}

\noindent
\textbf{Menu 6}

\vspace{0.5em}

{\renewcommand{\arraystretch}{1.5}
\begin{tabular*}{\textwidth}{@{\extracolsep{\fill}}|>{\hspace{0.5em}}c<{\hspace{0.5em}}|>{\hspace{0.5em}}c<{\hspace{0.5em}}|>{\hspace{0.5em}}c<{\hspace{0.5em}}|}
\hline
\multicolumn{3}{|c|}{Strawberries} \\ \hline
\multicolumn{3}{|c|}{Hominy with Sugar and Cream} \\ \hline
Bacon and Fried Eggs & Baked Potatoes & Rye Muffins \\ \hline
\multicolumn{3}{|c|}{Coffee} \\ \hline
\end{tabular*}}
\vspace{0.5em}


\vspace{1em}

\noindent
\textbf{Menu 7}

\vspace{0.5em}

{\renewcommand{\arraystretch}{1.5}
\begin{tabular*}{\textwidth}{@{\extracolsep{\fill}}|>{\hspace{0.5em}}c<{\hspace{0.5em}}|>{\hspace{0.5em}}c<{\hspace{0.5em}}|}
\hline
\multicolumn{2}{|c|}{Raspberries} \\ \hline
\multicolumn{2}{|c|}{Shredded Wheat Biscuit} \\ \hline
Dried Smoked Beef in Cream & Hashed Brown Potatoes \\ \hline
\multicolumn{2}{|c|}{Baking-Powder Biscuit} \\ \hline
\multicolumn{2}{|c|}{Coffee} \\ \hline
\end{tabular*}}
\vspace{0.5em}


\vspace{1em}

\noindent
\textbf{Menu 8}

\vspace{0.5em}

{\renewcommand{\arraystretch}{1.5}
\begin{tabular*}{\textwidth}{@{\extracolsep{\fill}}|>{\hspace{0.5em}}c<{\hspace{0.5em}}|>{\hspace{0.5em}}c<{\hspace{0.5em}}|>{\hspace{0.5em}}c<{\hspace{0.5em}}|}
\hline
\multicolumn{3}{|c|}{Watermelon} \\ \hline
\multicolumn{3}{|c|}{Old Grist Mill Rolled Oats with Sugar and Cream} \\ \hline
Broiled Halibut & Potato Cakes & Sliced Cucumbers \\ \hline
\multicolumn{3}{|c|}{Quaker Biscuit} \\ \hline
\multicolumn{3}{|c|}{Coffee} \\ \hline
\end{tabular*}}
\vspace{0.5em}


\vspace{1em}

\noindent
\textbf{Menu 9}

\vspace{0.5em}

{\renewcommand{\arraystretch}{1.5}
\begin{tabular*}{\textwidth}{@{\extracolsep{\fill}}|>{\hspace{0.5em}}c<{\hspace{0.5em}}|>{\hspace{0.5em}}c<{\hspace{0.5em}}|>{\hspace{0.5em}}c<{\hspace{0.5em}}|}
\hline
\multicolumn{3}{|c|}{Cantaloupe} \\ \hline
\multicolumn{3}{|c|}{Pettijohn's with Sugar and Cream} \\ \hline
Cecils with Tomato Sauce & Potato Balls & Rice Muffins \\ \hline
\multicolumn{3}{|c|}{Coffee} \\ \hline
\end{tabular*}}
\vspace{0.5em}


\vspace{1em}

\noindent
\textbf{Menu 10}

\vspace{0.5em}

{\renewcommand{\arraystretch}{1.5}
\begin{tabular*}{\textwidth}{@{\extracolsep{\fill}}|>{\hspace{0.5em}}c<{\hspace{0.5em}}|>{\hspace{0.5em}}c<{\hspace{0.5em}}|>{\hspace{0.5em}}c<{\hspace{0.5em}}|}
\hline
\multicolumn{3}{|c|}{Peaches} \\ \hline
\multicolumn{3}{|c|}{Farinose with Sugar and Cream} \\ \hline
Omelette & Potatoes à la Maître d'Hôtel & Berry Muffins \\ \hline
\multicolumn{3}{|c|}{Coffee} \\ \hline
\end{tabular*}}
\vspace{0.5em}


\vspace{1em}

\noindent
\textbf{Menu 11}

\vspace{0.5em}

{\renewcommand{\arraystretch}{1.5}
\begin{tabular*}{\textwidth}{@{\extracolsep{\fill}}|>{\hspace{0.5em}}c<{\hspace{0.5em}}|>{\hspace{0.5em}}c<{\hspace{0.5em}}|}
\hline
\multicolumn{2}{|c|}{Blackberries} \\ \hline
H-O with Sugar and Cream & Dropped Eggs on Toast \\ \hline
\multicolumn{2}{|c|}{Waffles with Maple Syrup} \\ \hline
\multicolumn{2}{|c|}{Coffee} \\ \hline
\end{tabular*}}
\vspace{0.5em}


\vspace{1em}

\noindent
\textbf{Menu 12}

\vspace{0.5em}

{\renewcommand{\arraystretch}{1.5}
\begin{tabular*}{\textwidth}{@{\extracolsep{\fill}}|>{\hspace{0.5em}}c<{\hspace{0.5em}}|>{\hspace{0.5em}}c<{\hspace{0.5em}}|}
\hline
\multicolumn{2}{|c|}{Pears} \\ \hline
\multicolumn{2}{|c|}{Old Grist Mill Rolled Wheat with Sugar and Cream} \\ \hline
Corned Beef Hash & Milk Toast \\ \hline
\multicolumn{2}{|c|}{Coffee} \\ \hline
\end{tabular*}}
\vspace{0.5em}


\vspace{1em}

\noindent
\textbf{Menu 13}

\vspace{0.5em}

{\renewcommand{\arraystretch}{1.5}
\begin{tabular*}{\textwidth}{@{\extracolsep{\fill}}|>{\hspace{0.5em}}c<{\hspace{0.5em}}|>{\hspace{0.5em}}c<{\hspace{0.5em}}|>{\hspace{0.5em}}c<{\hspace{0.5em}}|}
\hline
\multicolumn{3}{|c|}{Grapes} \\ \hline
\multicolumn{3}{|c|}{Cereal with Fruit} \\ \hline
Fried Smelts & Baked Sweet Potatoes & Sliced Tomatoes \\ \hline
\multicolumn{3}{|c|}{Oatmeal Muffins} \\ \hline
\multicolumn{3}{|c|}{Coffee} \\ \hline
\end{tabular*}}
\vspace{0.5em}


\vspace{1em}

\noindent
\textbf{Menu 14}

\vspace{0.5em}

{\renewcommand{\arraystretch}{1.5}
\begin{tabular*}{\textwidth}{@{\extracolsep{\fill}}|>{\hspace{0.5em}}c<{\hspace{0.5em}}|>{\hspace{0.5em}}c<{\hspace{0.5em}}|>{\hspace{0.5em}}c<{\hspace{0.5em}}|}
\hline
\multicolumn{3}{|c|}{Oatmeal Mush with Apples} \\ \hline
Hamburg Steaks & Creamed Potatoes & White Corn Cake \\ \hline
\multicolumn{3}{|c|}{Coffee} \\ \hline
\end{tabular*}}
\vspace{0.5em}


\vspace{1em}

\noindent
\textbf{Menu 15}

\vspace{0.5em}

{\renewcommand{\arraystretch}{1.5}
\begin{tabular*}{\textwidth}{@{\extracolsep{\fill}}|>{\hspace{0.5em}}c<{\hspace{0.5em}}|>{\hspace{0.5em}}c<{\hspace{0.5em}}|>{\hspace{0.5em}}c<{\hspace{0.5em}}|}
\hline
\multicolumn{3}{|c|}{Plums and Pears} \\ \hline
\multicolumn{3}{|c|}{Cracked Wheat with Sugar and Cream} \\ \hline
Baked Beans & Fish Balls & Brown Bread \\ \hline
\multicolumn{3}{|c|}{Old Grist Mill Coffee} \\ \hline
\end{tabular*}}
\vspace{0.5em}


\vspace{1em}

\noindent
\textbf{Menu 16}

\vspace{0.5em}

{\renewcommand{\arraystretch}{1.5}
\begin{tabular*}{\textwidth}{@{\extracolsep{\fill}}|>{\hspace{0.5em}}c<{\hspace{0.5em}}|>{\hspace{0.5em}}c<{\hspace{0.5em}}|}
\hline
\multicolumn{2}{|c|}{Sliced Peaches} \\ \hline
Germea with Sugar and Cream & Brown Bread Toast \\ \hline
Cold Sliced Meat & Sautéd Sweet Potatoes \\ \hline
\multicolumn{2}{|c|}{Coffee} \\ \hline
\end{tabular*}}
\vspace{0.5em}


\vspace{1em}

\noindent
\textbf{Menu 17}

\vspace{0.5em}

{\renewcommand{\arraystretch}{1.5}
\begin{tabular*}{\textwidth}{@{\extracolsep{\fill}}|>{\hspace{0.5em}}c<{\hspace{0.5em}}|>{\hspace{0.5em}}c<{\hspace{0.5em}}|}
\hline
\multicolumn{2}{|c|}{Wheatena with Sugar and Cream} \\ \hline
Fish Hash & Buttered Graham Toast \\ \hline
\multicolumn{2}{|c|}{Strawberry Short Cake} \\ \hline
\multicolumn{2}{|c|}{Coffee} \\ \hline
\end{tabular*}}
\vspace{0.5em}


\section*{\centering Luncheon Menus}


\vspace{1em}

\noindent
\textbf{Menu 18}

\vspace{0.5em}

{\renewcommand{\arraystretch}{1.5}
\begin{tabular*}{\textwidth}{@{\extracolsep{\fill}}|>{\hspace{0.5em}}c<{\hspace{0.5em}}|>{\hspace{0.5em}}c<{\hspace{0.5em}}|>{\hspace{0.5em}}c<{\hspace{0.5em}}|}
\hline
\multicolumn{3}{|c|}{Grapes} \\ \hline
\multicolumn{3}{|c|}{Old Grist Mill Rye Flakes with Sugar and Cream} \\ \hline
Lamb Chops & Baked Potatoes & Raised Muffins \\ \hline
\multicolumn{3}{|c|}{Doughnuts and Coffee} \\ \hline
\multicolumn{3}{|c|}{Grilled Sardines} \\ \hline
Baked Apples with Cream & Rolls & Sponge Cake \\ \hline
\multicolumn{3}{|c|}{Cocoa} \\ \hline
\end{tabular*}}
\vspace{0.5em}


\vspace{1em}

\noindent
\textbf{Menu 19}

\vspace{0.5em}

{\renewcommand{\arraystretch}{1.5}
\begin{tabular*}{\textwidth}{@{\extracolsep{\fill}}|>{\hspace{0.5em}}c<{\hspace{0.5em}}|>{\hspace{0.5em}}c<{\hspace{0.5em}}|}
\hline
\multicolumn{2}{|c|}{Creamed Chicken} \\ \hline
Celery & Rolls \\ \hline
\multicolumn{2}{|c|}{Grapes and Apples} \\ \hline
\multicolumn{2}{|c|}{Tea} \\ \hline
\end{tabular*}}
\vspace{0.5em}


\vspace{1em}

\noindent
\textbf{Menu 20}

\vspace{0.5em}

{\renewcommand{\arraystretch}{1.5}
\begin{tabular*}{\textwidth}{@{\extracolsep{\fill}}|>{\hspace{0.5em}}c<{\hspace{0.5em}}|>{\hspace{0.5em}}c<{\hspace{0.5em}}|}
\hline
\multicolumn{2}{|c|}{Lamb Croquettes} \\ \hline
Dressed Lettuce & Baking-Powder Biscuit \\ \hline
Gingerbread & Cheese \\ \hline
\multicolumn{2}{|c|}{Tea} \\ \hline
\end{tabular*}}
\vspace{0.5em}


\vspace{1em}

\noindent
\textbf{Menu 21}

\vspace{0.5em}

{\renewcommand{\arraystretch}{1.5}
\begin{tabular*}{\textwidth}{@{\extracolsep{\fill}}|>{\hspace{0.5em}}c<{\hspace{0.5em}}|>{\hspace{0.5em}}c<{\hspace{0.5em}}|}
\hline
Split Pea Soup & Crisp Crackers \\ \hline
Egg Salad & Entire Wheat Bread \\ \hline
\multicolumn{2}{|c|}{Oranges} \\ \hline
\multicolumn{2}{|c|}{Cocoa} \\ \hline
\end{tabular*}}
\vspace{0.5em}


\vspace{1em}

\noindent
\textbf{Menu 22}

\vspace{0.5em}

{\renewcommand{\arraystretch}{1.5}
\begin{tabular*}{\textwidth}{@{\extracolsep{\fill}}|>{\hspace{0.5em}}c<{\hspace{0.5em}}|>{\hspace{0.5em}}c<{\hspace{0.5em}}|}
\hline
Cold Sliced Meat & Cheese Fondue \\ \hline
\multicolumn{2}{|c|}{Bread and Butter} \\ \hline
Sliced Peaches & Cookies \\ \hline
\multicolumn{2}{|c|}{Old Grist Mill Coffee} \\ \hline
\end{tabular*}}
\vspace{0.5em}


\vspace{1em}

\noindent
\textbf{Menu 23}

\vspace{0.5em}

{\renewcommand{\arraystretch}{1.5}
\begin{tabular*}{\textwidth}{@{\extracolsep{\fill}}|>{\hspace{0.5em}}c<{\hspace{0.5em}}|>{\hspace{0.5em}}c<{\hspace{0.5em}}|}
\hline
Broiled Ham & Scalloped Potatoes \\ \hline
\multicolumn{2}{|c|}{Brown Bread and Butter} \\ \hline
Sliced Oranges & Wafers \\ \hline
\end{tabular*}}
\vspace{0.5em}


\vspace{1em}

\noindent
\textbf{Menu 24}

\vspace{0.5em}

{\renewcommand{\arraystretch}{1.5}
\begin{tabular*}{\textwidth}{@{\extracolsep{\fill}}|>{\hspace{0.5em}}c<{\hspace{0.5em}}|>{\hspace{0.5em}}c<{\hspace{0.5em}}|}
\hline
Scalloped Oysters & Rolls \\ \hline
\multicolumn{2}{|c|}{Dressed Celery} \\ \hline
Polish Tartlets & Tea \\ \hline
\end{tabular*}}
\vspace{0.5em}


\vspace{1em}

\noindent
\textbf{Menu 25}

\vspace{0.5em}

{\renewcommand{\arraystretch}{1.5}
\begin{tabular*}{\textwidth}{@{\extracolsep{\fill}}|>{\hspace{0.5em}}c<{\hspace{0.5em}}|>{\hspace{0.5em}}c<{\hspace{0.5em}}|}
\hline
Salmi of Lamb & Olives \\ \hline
\multicolumn{2}{|c|}{Bread and Butter} \\ \hline
Cake & Chocolate \\ \hline
\end{tabular*}}
\vspace{0.5em}


\vspace{1em}

\noindent
\textbf{Menu 26}

\vspace{0.5em}

{\renewcommand{\arraystretch}{1.5}
\begin{tabular*}{\textwidth}{@{\extracolsep{\fill}}|>{\hspace{0.5em}}c<{\hspace{0.5em}}|>{\hspace{0.5em}}c<{\hspace{0.5em}}|}
\hline
\multicolumn{2}{|c|}{Oyster Stew} \\ \hline
\multicolumn{2}{|c|}{Oyster Crackers or Dry Toast} \\ \hline
\multicolumn{2}{|c|}{Pickles} \\ \hline
Cream Whips & Lady Fingers \\ \hline
\end{tabular*}}
\vspace{0.5em}


\vspace{1em}

\noindent
\textbf{Menu 27}

\vspace{0.5em}

{\renewcommand{\arraystretch}{1.5}
\begin{tabular*}{\textwidth}{@{\extracolsep{\fill}}|>{\hspace{0.5em}}c<{\hspace{0.5em}}|>{\hspace{0.5em}}c<{\hspace{0.5em}}|}
\hline
\multicolumn{2}{|c|}{Scalloped Turkey} \\ \hline
\multicolumn{2}{|c|}{Brown Bread Sandwiches} \\ \hline
Lettuce Salad & Cheese Straws \\ \hline
\multicolumn{2}{|c|}{Tea} \\ \hline
\end{tabular*}}
\vspace{0.5em}


\vspace{1em}

\noindent
\textbf{Menu 28}

\vspace{0.5em}

{\renewcommand{\arraystretch}{1.5}
\begin{tabular*}{\textwidth}{@{\extracolsep{\fill}}|>{\hspace{0.5em}}c<{\hspace{0.5em}}|>{\hspace{0.5em}}c<{\hspace{0.5em}}|>{\hspace{0.5em}}c<{\hspace{0.5em}}|}
\hline
Turban of Fish & Saratoga Potatoes &  \\ \hline
\multicolumn{3}{|c|}{Warmed over Muffins} \\ \hline
Nuts & Crackers & Cheese \\ \hline
\multicolumn{3}{|c|}{Tea} \\ \hline
\end{tabular*}}
\vspace{0.5em}


\vspace{1em}

\noindent
\textbf{Menu 29}

\vspace{0.5em}

{\renewcommand{\arraystretch}{1.5}
\begin{tabular*}{\textwidth}{@{\extracolsep{\fill}}|>{\hspace{0.5em}}c<{\hspace{0.5em}}|>{\hspace{0.5em}}c<{\hspace{0.5em}}|}
\hline
Cream of Tomato Soup & Croûtons \\ \hline
\multicolumn{2}{|c|}{Omelet with Vegetables} \\ \hline
\multicolumn{2}{|c|}{Bread and Butter} \\ \hline
Bananas & Tea \\ \hline
\end{tabular*}}
\vspace{0.5em}


\vspace{1em}

\noindent
\textbf{Menu 30}

\vspace{0.5em}

{\renewcommand{\arraystretch}{1.5}
\begin{tabular*}{\textwidth}{@{\extracolsep{\fill}}|>{\hspace{0.5em}}c<{\hspace{0.5em}}|>{\hspace{0.5em}}c<{\hspace{0.5em}}|}
\hline
\multicolumn{2}{|c|}{Salad à la Russe} \\ \hline
\multicolumn{2}{|c|}{Graham Bread and Butter} \\ \hline
Peach Sauce & Scotch Wafers \\ \hline
\multicolumn{2}{|c|}{Tea} \\ \hline
\end{tabular*}}
\vspace{0.5em}


\vspace{1em}

\noindent
\textbf{Menu 31}

\vspace{0.5em}

{\renewcommand{\arraystretch}{1.5}
\begin{tabular*}{\textwidth}{@{\extracolsep{\fill}}|>{\hspace{0.5em}}c<{\hspace{0.5em}}|>{\hspace{0.5em}}c<{\hspace{0.5em}}|}
\hline
\multicolumn{2}{|c|}{Cold Sliced Tongue} \\ \hline
\multicolumn{2}{|c|}{Macaroni and Cheese} \\ \hline
Lettuce Salad & Crackers \\ \hline
Wafers & Coffee \\ \hline
\end{tabular*}}
\vspace{0.5em}


\vspace{1em}

\noindent
\textbf{Menu 32}

\vspace{0.5em}

{\renewcommand{\arraystretch}{1.5}
\begin{tabular*}{\textwidth}{@{\extracolsep{\fill}}|>{\hspace{0.5em}}c<{\hspace{0.5em}}|>{\hspace{0.5em}}c<{\hspace{0.5em}}|}
\hline
Salmon Croquettes & Rolls \\ \hline
\multicolumn{2}{|c|}{Dressed Lettuce} \\ \hline
\multicolumn{2}{|c|}{Strawberries and Cream} \\ \hline
\multicolumn{2}{|c|}{Tea} \\ \hline
\end{tabular*}}
\vspace{0.5em}


\vspace{1em}

\noindent
\textbf{Menu 33}

\vspace{0.5em}

{\renewcommand{\arraystretch}{1.5}
\begin{tabular*}{\textwidth}{@{\extracolsep{\fill}}|>{\hspace{0.5em}}c<{\hspace{0.5em}}|>{\hspace{0.5em}}c<{\hspace{0.5em}}|}
\hline
\multicolumn{2}{|c|}{Beef Stew with Dumplings} \\ \hline
Sliced Oranges & Cake \\ \hline
\multicolumn{2}{|c|}{Tea} \\ \hline
\end{tabular*}}
\vspace{0.5em}


\vspace{1em}

\noindent
\textbf{Menu 34}

\vspace{0.5em}

{\renewcommand{\arraystretch}{1.5}
\begin{tabular*}{\textwidth}{@{\extracolsep{\fill}}|>{\hspace{0.5em}}c<{\hspace{0.5em}}|>{\hspace{0.5em}}c<{\hspace{0.5em}}|}
\hline
Lobster Salad & Rolls \\ \hline
Raspberries and Cream & Wafers \\ \hline
\multicolumn{2}{|c|}{Russian Tea} \\ \hline
\end{tabular*}}
\vspace{0.5em}


\end{document}